\documentclass[aspectratio=169,11pt]{beamer}
\usetheme{Madrid}
\usecolortheme{whale}

% 中文支持
\usepackage{ctex}
\usepackage{fontspec}

% 其他包
\usepackage{graphicx}
\usepackage{booktabs}
\usepackage{array}
\usepackage{multirow}
\usepackage{hyperref}
\usepackage{tikz}
\usepackage{pgfplots}
\pgfplotsset{compat=1.18}
\usetikzlibrary{shapes,arrows,positioning,calc}
\usepackage{xcolor}
\usepackage{listings}
\usepackage{fontawesome5}

% 自定义颜色
\definecolor{openaigreen}{RGB}{16,163,127}
\definecolor{anthropicred}{RGB}{204,78,78}
\definecolor{googleblue}{RGB}{66,133,244}
\definecolor{githubblack}{RGB}{36,41,46}
\definecolor{cursorpurple}{RGB}{138,43,226}
\definecolor{deepseekblue}{RGB}{0,102,204}

% 设置代码样式
\lstset{
    basicstyle=\ttfamily\footnotesize,
    breaklines=true,
    frame=single,
    backgroundcolor=\color{gray!10}
}

% 标题信息
\title[AI大模型与编程工具]{大模型及其驱动的编程工具\\——科研与开发的实用指南}
\subtitle{2025年12月10日更新版}
\author{技术分享}
\institute{团队技术培训}
\date{\today}

\begin{document}

% 标题页
\begin{frame}
    \titlepage
\end{frame}

% 目录
\begin{frame}{内容大纲}
    \tableofcontents
\end{frame}

%========================================
\section{大模型发展概览}
%========================================

\begin{frame}{2025年末AI大模型格局}
    \begin{columns}
        \begin{column}{0.5\textwidth}
            \textbf{三大主流厂商:}
            \begin{itemize}
                \item \textcolor{openaigreen}{\faRobot\ OpenAI} - GPT-5.2系列
                \item \textcolor{anthropicred}{\faComment\ Anthropic} - Claude Opus 4.5
                \item \textcolor{googleblue}{\faGoogle\ Google} - Gemini 3 Pro
            \end{itemize}
            \vspace{0.5cm}
            \textbf{中国优秀模型:}
            \begin{itemize}
                \item \textcolor{deepseekblue}{\faBrain\ DeepSeek} - V3.2 (中文能力最强)
                \item 智谱AI - GLM-5
                \item 阿里云 - Qwen 3.0
            \end{itemize}
        \end{column}
        \begin{column}{0.5\textwidth}
            \begin{tikzpicture}[scale=0.8]
                \begin{axis}[
                    ybar,
                    symbolic x coords={GPT-5.2,Claude 4.5,Gemini 3P,DeepSeek},
                    xtick=data,
                    ylabel={基准测试得分},
                    ymin=80,ymax=100,
                    bar width=20pt,
                    nodes near coords,
                    title={2025.12 主流模型性能对比}
                ]
                \addplot coordinates {(GPT-5.2,98) (Claude 4.5,97) (Gemini 3P,96) (DeepSeek,95)};
                \end{axis}
            \end{tikzpicture}
        \end{column}
    \end{columns}
\end{frame}

\begin{frame}{最新动态(2025年12月)}
    \begin{alertblock}{重要更新(本周!)}
        \begin{itemize}
            \item \textbf{12月10日}:Anthropic将MCP协议捐赠给Agentic AI Foundation
            \item \textbf{12月9日}:OpenAI成立Agentic AI Foundation,发布AGENTS.md标准
            \item \textbf{12月9日}:OpenAI发布首批官方认证课程
            \item \textbf{12月8日}:Copilot Coding Agent支持模型选择(Pro/Pro+用户)
            \item \textbf{12月4日}:GPT-5.1-Codex-Max在GitHub Copilot公测
            \item \textbf{12月3日}:Claude Opus 4.5登陆VS/JetBrains/Xcode/Eclipse
            \item \textbf{12月3日}:Anthropic收购Bun,Claude Code达\$10亿收入
        \end{itemize}
    \end{alertblock}
    
    \vspace{0.3cm}
    \begin{block}{趋势观察}
        \begin{itemize}
            \item MCP协议成为行业标准,AI Agent互操作性大幅提升
            \item 多模型选择成为标配,用户可根据任务选择最优模型
            \item Agentic AI(智能体AI)成为2025年末最热门方向
        \end{itemize}
    \end{block}
\end{frame}

%========================================
\section{主流AI大模型详解}
%========================================

\begin{frame}{OpenAI GPT系列}
    \begin{columns}
        \begin{column}{0.6\textwidth}
            \textbf{模型家族(2025年12月最新):}
            \begin{itemize}
                \item \textbf{GPT-5.1} - 当前主力模型(11/12发布)
                \item \textbf{GPT-5.1-Codex-Max} - 编程专用(12/4公测)
                \item \textbf{o3/o4-mini} - 推理模型系列
            \end{itemize}
            
            \vspace{0.3cm}
            \textbf{最新动态:}
            \begin{itemize}
                \item 12/9:成立Agentic AI Foundation
                \item 12/9:发布首批OpenAI认证课程
                \item 12/8:企业AI报告发布
                \item 12/3:收购Neptune公司
                \item 10/21:发布ChatGPT Atlas浏览器
            \end{itemize}
        \end{column}
        \begin{column}{0.4\textwidth}
            \begin{block}{ChatGPT订阅}
                \begin{itemize}
                    \item \textbf{Free}:有限使用
                    \item \textbf{Plus}:\$20/月
                    \item \textbf{Pro}:\$200/月
                    \item \textbf{Team}:\$25-30/用户/月
                \end{itemize}
            \end{block}
            \vspace{0.2cm}
            \begin{exampleblock}{新功能}
                \small
                \begin{itemize}
                    \item Shopping Research(11/24)
                    \item Company Knowledge(10/23)
                    \item Parental Controls(9/29)
                \end{itemize}
            \end{exampleblock}
        \end{column}
    \end{columns}
\end{frame}

\begin{frame}{Anthropic Claude系列}
    \begin{columns}
        \begin{column}{0.55\textwidth}
            \textbf{模型家族(2025年12月最新):}
            \begin{itemize}
                \item \textbf{Claude Opus 4.5}(11/25发布)\\
                    {\small 世界最强编程、Agent和Computer Use模型}
                \item \textbf{Claude Sonnet 4.5}(9/30发布)\\
                    {\small 平衡性能与成本,含Agent SDK}
                \item \textbf{Claude Haiku 4.5}(10/16发布)\\
                    {\small 超快速度,极具性价比}
            \end{itemize}
            
            \vspace{0.3cm}
            \textbf{本周重大新闻:}
            \begin{itemize}
                \item \textcolor{red}{12/10}:MCP协议捐赠给Agentic AI Foundation
                \item \textcolor{red}{12/3}:收购Bun(JS运行时)
                \item \textcolor{red}{12/3}:Claude Code达\$10亿收入!
                \item 12/4:与Snowflake达成\$2亿合作
                \item 11/18:登陆Microsoft 365 Copilot
            \end{itemize}
        \end{column}
        \begin{column}{0.45\textwidth}
            \begin{block}{Claude订阅价格}
                \begin{itemize}
                    \item \textbf{Pro}:\$20/月
                    \item \textbf{Max 5x}:\$100/月
                    \item \textbf{Max 20x}:\$200/月
                    \item \textbf{Team}:\$30/用户/月
                    \item \textbf{Enterprise}:定制价格
                \end{itemize}
            \end{block}
            \vspace{0.2cm}
            \begin{alertblock}{估值飙升}
                \small
                9月:Series F融资\$130亿\\
                估值:\$1830亿\\
                收入:8个月从\$10亿→\$50亿+
            \end{alertblock}
        \end{column}
    \end{columns}
\end{frame}

\begin{frame}{Google Gemini系列}
    \begin{columns}
        \begin{column}{0.6\textwidth}
            \textbf{模型家族(2025年12月最新):}
            \begin{itemize}
                \item \textbf{Gemini 3}(最新旗舰,全新智能时代)
                \item \textbf{Gemini 3 Pro}(视觉AI前沿)
                \item \textbf{Gemini 3 Pro Image}(图像生成)
                \item \textbf{Nano Banana Pro}(轻量版)
            \end{itemize}
            
            \vspace{0.3cm}
            \textbf{最新动态:}
            \begin{itemize}
                \item 12/9:Gemini AI职场研究报告发布
                \item Gemini 3 Pro已集成GitHub Copilot
                \item Google AI Pro/Ultra提高API限额
                \item Project Suncatcher:太空ML计算
            \end{itemize}
            
            \vspace{0.3cm}
            \textbf{适用场景:}
            \begin{itemize}
                \item 科研文献分析和总结
                \item 多模态理解(图像/视频/音频)
                \item 与Google Workspace集成办公
            \end{itemize}
        \end{column}
        \begin{column}{0.4\textwidth}
            \begin{tikzpicture}[scale=0.7]
                \begin{axis}[
                    xbar,
                    xmin=0, xmax=50,
                    symbolic y coords={知识问答,多模态理解,代码生成},
                    ytick=data,
                    nodes near coords,
                    nodes near coords align={horizontal},
                    xlabel={使用占比 (\%)},
                    title={Gemini应用场景分布}
                ]
                \addplot coordinates {(25,知识问答) (35,多模态理解) (40,代码生成)};
                \end{axis}
            \end{tikzpicture}
        \end{column}
    \end{columns}
\end{frame}

\begin{frame}{DeepSeek(中国)}
    \begin{columns}
        \begin{column}{0.6\textwidth}
            \textbf{最新版本:DeepSeek-V3.2(2025年12月)}
            
            \vspace{0.3cm}
            \textbf{核心特点:}
            \begin{itemize}
                \item \textcolor{red}{中文能力全球第一}:在中文语境理解、古文、成语及中国文化相关任务上碾压所有国外模型
                \item \textcolor{red}{完全免费}网页端和APP使用
                \item 强化Agent能力与数学推理
                \item API价格极具竞争力
            \end{itemize}
            
            \vspace{0.3cm}
            \textbf{适用场景:}
            \begin{itemize}
                \item \textbf{所有中文相关任务的首选}
                \item 预算有限的个人开发者
                \item 数学和推理任务
                \item API集成开发
            \end{itemize}
        \end{column}
        \begin{column}{0.4\textwidth}
            \begin{alertblock}{国产之光}
                DeepSeek不仅是性价比之选,更是\textbf{中文领域的最强王者}。对于国内科研人员和开发者,它是处理中文文档和数据的最佳工具。
            \end{alertblock}
            
            \vspace{0.2cm}
            \textbf{访问方式:}\\
            \url{https://chat.deepseek.com}\\
            \url{https://platform.deepseek.com}
        \end{column}
    \end{columns}
\end{frame}

\begin{frame}{大模型对比总结}
    \footnotesize
    \begin{tabular}{p{2cm}p{2.5cm}p{2.5cm}p{2.5cm}p{2.5cm}}
        \toprule
        \textbf{特性} & \textbf{GPT-5.2系列} & \textbf{Claude 4.5} & \textbf{Gemini 3 Pro} & \textbf{DeepSeek V3.2} \\
        \midrule
        代码能力 & \textcolor{green!60!black}{极强} & \textcolor{green!60!black}{最强} & \textcolor{green!60!black}{强} & \textcolor{green!60!black}{强} \\
        推理能力 & \textcolor{green!60!black}{极强} & \textcolor{green!60!black}{极强} & \textcolor{green!60!black}{极强} & \textcolor{green!60!black}{极强} \\
        多模态 & \textcolor{green!60!black}{全面} & \textcolor{orange}{文本为主} & \textcolor{green!60!black}{最强} & \textcolor{orange}{基础} \\
        上下文长度 & 128K & 200K & 2M & 128K \\
        中文支持 & \textcolor{orange}{良好} & \textcolor{orange}{良好} & \textcolor{orange}{良好} & \textcolor{red}{无敌} \\
        价格 & \textcolor{red}{较高} & \textcolor{orange}{中等} & \textcolor{orange}{中等} & \textcolor{green!60!black}{最低} \\
        免费额度 & 有限 & 有限 & 较多 & \textcolor{green!60!black}{充足} \\
        \bottomrule
    \end{tabular}
    
    \vspace{0.5cm}
    \begin{block}{选择建议}
        \begin{itemize}
            \item \textbf{复杂编程任务}:Claude Opus 4.5 或 GPT-5.2
            \item \textbf{中文任务/预算敏感}:DeepSeek V3.2 (中文最强)
            \item \textbf{多模态需求}:Gemini 3 Pro 或 GPT-5.2
            \item \textbf{长文本处理}:Gemini 3 Pro (2M) 或 Claude (200K)
        \end{itemize}
    \end{block}
\end{frame}

%========================================
\section{AI编程工具实战}
%========================================

\begin{frame}{常用的编程智能体 (Programming Agents)}
    \begin{columns}
        \begin{column}{0.5\textwidth}
            \textbf{OpenAI Codex}
            \begin{itemize}
                \item \textbf{简介}:GitHub Copilot背后的核心动力,专为代码生成优化的GPT模型。
                \item \textbf{特点}:
                    \begin{itemize}
                        \item 经过数十亿行公开代码训练
                        \item 支持多种编程语言
                        \item 擅长将自然语言转化为代码
                    \end{itemize}
                \item \textbf{应用}:GitHub Copilot, OpenAI API
            \end{itemize}
        \end{column}
        \begin{column}{0.5\textwidth}
            \textbf{Google Jules}
            \begin{itemize}
                \item \textbf{简介}:Google DeepMind开发的编程智能体,旨在处理复杂的编程任务。
                \item \textbf{特点}:
                    \begin{itemize}
                        \item 具备多步推理能力
                        \item 能理解和修改大型代码库
                        \item 专注于解决GitHub Issue级别的任务
                    \end{itemize}
                \item \textbf{应用}:集成于Google内部工具及Gemini生态
            \end{itemize}
        \end{column}
    \end{columns}
    \vspace{0.5cm}
    \begin{block}{Agent vs. Assistant}
        \begin{itemize}
            \item \textbf{Assistant (助手)}:补全代码,回答问题(如Copilot早期版本)。
            \item \textbf{Agent (智能体)}:自主规划,执行多步操作,调用工具,解决复杂问题(如Jules, Copilot Workspace)。
        \end{itemize}
    \end{block}
\end{frame}

\begin{frame}{VS Code + GitHub Copilot}
    \begin{columns}
        \begin{column}{0.5\textwidth}
            \textbf{为什么选择这个组合?}
            \begin{itemize}
                \item 全球最流行的代码编辑器
                \item 微软官方AI助手
                \item 深度IDE集成
                \item 企业级安全保障
            \end{itemize}
            
            \vspace{0.3cm}
            \textbf{2025年12月最新更新:}
            \begin{itemize}
                \item \textcolor{red}{12/8}:Coding Agent支持模型选择
                \item \textcolor{red}{12/5}:代码生成指标仪表板
                \item \textcolor{red}{12/4}:GPT-5.1-Codex-Max公测
                \item \textcolor{red}{12/3}:Claude Opus 4.5全平台支持
                \item \textcolor{red}{12/3}:API分配Issue给Copilot
                \item \textcolor{red}{12/1}:Copilot Spaces公共空间
            \end{itemize}
        \end{column}
        \begin{column}{0.5\textwidth}
            \begin{block}{订阅价格(2025年12月)}
                \begin{tabular}{ll}
                    \textbf{Free} & 免费(有限功能) \\
                    \textbf{Pro} & \$10/月 \\
                    \textbf{Pro+} & \$39/月 \\
                    \textbf{Business} & \$19/用户/月 \\
                    \textbf{Enterprise} & \$39/用户/月 \\
                \end{tabular}
            \end{block}
            
            \vspace{0.3cm}
            \begin{exampleblock}{实用技巧}
                \begin{itemize}
                    \item 按 \texttt{Tab} 接受建议
                    \item 按 \texttt{Ctrl+Enter} 查看多个建议
                    \item 使用 \texttt{@workspace} 引用项目
                    \item 使用 \texttt{/fix} 修复代码问题
                \end{itemize}
            \end{exampleblock}
        \end{column}
    \end{columns}
\end{frame}

\begin{frame}[fragile]{GitHub Copilot 核心功能演示}
    \begin{columns}
        \begin{column}{0.5\textwidth}
            \textbf{1. 代码补全}
            \begin{lstlisting}[language=Python]
# 只需输入注释,Copilot自动生成
def calculate_fibonacci(n):
    # Copilot会自动补全实现
    if n <= 1:
        return n
    return calculate_fibonacci(n-1) + 
           calculate_fibonacci(n-2)
            \end{lstlisting}
            
            \vspace{0.3cm}
            \textbf{2. Copilot Chat}
            \begin{itemize}
                \item \texttt{@workspace} - 项目上下文
                \item \texttt{@vscode} - 编辑器操作
                \item \texttt{/explain} - 解释代码
                \item \texttt{/tests} - 生成测试
                \item \texttt{/fix} - 修复问题
            \end{itemize}
        \end{column}
        \begin{column}{0.5\textwidth}
            \textbf{3. Agent模式(最新)}
            \begin{itemize}
                \item 自动执行多步骤任务
                \item 读取/修改多个文件
                \item 运行终端命令
                \item 创建Pull Request
            \end{itemize}
            
            \vspace{0.3cm}
            \textbf{4. 模型选择(12月更新)}
            \begin{itemize}
                \item GPT-5.1-Codex-Max
                \item Claude Opus 4.5
                \item Gemini 3 Pro
                \item 自动模型选择
            \end{itemize}
            
            \vspace{0.3cm}
            \begin{alertblock}{Pro技巧}
                在设置中启用Agent模式:\\
                \texttt{github.copilot.chat.agent.enabled}
            \end{alertblock}
        \end{column}
    \end{columns}
\end{frame}

\begin{frame}{Cursor IDE}
    \begin{columns}
        \begin{column}{0.55\textwidth}
            \textbf{什么是Cursor?}
            \begin{itemize}
                \item 基于VS Code的AI-native IDE
                \item 专为AI编程设计
                \item 100万+活跃用户
                \item 被Y Combinator、Stripe等推荐
            \end{itemize}
            
            \vspace{0.3cm}
            \textbf{核心功能:}
            \begin{itemize}
                \item \textbf{Tab}:超智能代码补全
                \item \textbf{Multi-Agent}:8个Agent并行运行
                \item \textbf{Composer}:自研4x快速模型
                \item \textbf{Browser}:内置浏览器调试
                \item \textbf{Plan Mode}:复杂任务规划
            \end{itemize}
            
            \vspace{0.3cm}
            \textbf{支持模型:}\\
            OpenAI、Anthropic、Gemini、xAI全系列
        \end{column}
        \begin{column}{0.45\textwidth}
            \begin{block}{价格(2025年12月)}
                \begin{tabular}{ll}
                    \textbf{Hobby} & 免费 \\
                    \textbf{Pro} & \$20/月 \\
                    \textbf{Pro+} & \$60/月 \\
                    \textbf{Ultra} & \$200/月 \\
                    \textbf{Teams} & \$40/用户/月 \\
                \end{tabular}
            \end{block}
            
            \vspace{0.3cm}
            \begin{alertblock}{最新更新(v2.1 11/21)}
                \begin{itemize}
                    \item AI Code Review编辑器内
                    \item Instant Grep(即时搜索)
                    \item Plan Mode交互式问答
                    \item Cloud Agent 99.9\%可靠性
                \end{itemize}
            \end{alertblock}
        \end{column}
    \end{columns}
\end{frame}

\begin{frame}{Windsurf IDE}
    \begin{columns}
        \begin{column}{0.55\textwidth}
            \textbf{什么是Windsurf?}
            \begin{itemize}
                \item 首个真正的"Agentic IDE"
                \item Codeium公司出品
                \item 100万+活跃用户
                \item 59\%财富500强企业使用
            \end{itemize}
            
            \vspace{0.3cm}
            \textbf{核心功能 - Cascade:}
            \begin{itemize}
                \item 深度代码库理解
                \item 实时感知用户操作
                \item 智能上下文管理
                \item 自动修复Linter错误
                \item MCP协议支持
            \end{itemize}
            
            \vspace{0.3cm}
            \textbf{亮点功能:}
            \begin{itemize}
                \item Windsurf Previews(实时预览)
                \item 一键部署到生产环境
            \end{itemize}
        \end{column}
        \begin{column}{0.45\textwidth}
            \begin{block}{使用数据}
                \begin{itemize}
                    \item 每天7000万+行AI生成代码
                    \item 94\%代码由AI编写
                    \item JPMorgan Chase创新奖
                \end{itemize}
            \end{block}
            
            \vspace{0.3cm}
            \begin{exampleblock}{适用人群}
                \begin{itemize}
                    \item 前端开发者(实时预览)
                    \item 快速原型开发
                    \item 企业级安全需求
                    \item 需要本地部署
                \end{itemize}
            \end{exampleblock}
            
            \vspace{0.2cm}
            \textbf{下载:}\url{https://windsurf.com}
        \end{column}
    \end{columns}
\end{frame}

\begin{frame}[fragile]{Claude Code - 终端编程神器}
    \begin{columns}
        \begin{column}{0.55\textwidth}
            \textbf{什么是Claude Code?}
            \begin{itemize}
                \item Anthropic官方终端工具
                \item 直接在命令行中与Claude交互
                \item 已达到\textbf{\$10亿}收入里程碑
            \end{itemize}
            
            \vspace{0.3cm}
            \textbf{核心能力:}
            \begin{itemize}
                \item 代码库快速上手(几秒内理解整个项目)
                \item Issue到PR一站式处理
                \item 多文件智能编辑
                \item 与CLI工具无缝集成
                \item 支持GitHub/GitLab集成
            \end{itemize}
            
            \vspace{0.3cm}
            \textbf{安装命令(Windows):}
            \begin{lstlisting}[language=bash]
irm https://claude.ai/install.ps1 | iex
            \end{lstlisting}
        \end{column}
        \begin{column}{0.45\textwidth}
            \begin{block}{包含在Claude订阅中}
                \begin{itemize}
                    \item Pro(\$20/月)- 短期任务
                    \item Max 5x(\$100/月)- 日常使用
                    \item Max 20x(\$200/月)- 重度用户
                \end{itemize}
            \end{block}
            
            \vspace{0.3cm}
            \begin{exampleblock}{典型工作流}
                \begin{enumerate}
                    \item 进入项目目录
                    \item 运行 \texttt{claude}
                    \item 描述任务
                    \item Claude自动分析、编辑、测试
                    \item 审核并提交
                \end{enumerate}
            \end{exampleblock}
        \end{column}
    \end{columns}
\end{frame}

\begin{frame}{AI编程工具对比}
    \footnotesize
    \begin{tabular}{p{2.2cm}p{2.5cm}p{2.5cm}p{2.5cm}p{2.5cm}}
        \toprule
        \textbf{特性} & \textbf{Copilot} & \textbf{Cursor} & \textbf{Windsurf} & \textbf{Claude Code} \\
        \midrule
        类型 & VS Code插件 & 独立IDE & 独立IDE & 终端工具 \\
        基础价格 & \$10/月 & \$20/月 & 免费起 & \$20/月起 \\
        Agent能力 & \textcolor{green!60!black}{强} & \textcolor{green!60!black}{最强} & \textcolor{green!60!black}{强} & \textcolor{green!60!black}{强} \\
        模型选择 & 多模型 & 多模型 & 专有模型 & Claude系列 \\
        学习曲线 & \textcolor{green!60!black}{低} & \textcolor{orange}{中} & \textcolor{orange}{中} & \textcolor{orange}{中} \\
        企业支持 & \textcolor{green!60!black}{最完善} & \textcolor{orange}{有} & \textcolor{green!60!black}{强} & \textcolor{orange}{有} \\
        离线使用 & 否 & 否 & 可选 & 否 \\
        \bottomrule
    \end{tabular}
    
    \vspace{0.5cm}
    \begin{block}{选择建议}
        \begin{itemize}
            \item \textbf{VS Code忠实用户}:GitHub Copilot(无缝集成)
            \item \textbf{追求最佳AI体验}:Cursor(AI-native设计)
            \item \textbf{前端开发/快速原型}:Windsurf(实时预览)
            \item \textbf{终端党/命令行爱好者}:Claude Code
            \item \textbf{预算有限}:Windsurf免费版 + DeepSeek
        \end{itemize}
    \end{block}
\end{frame}

%========================================
\section{科研场景应用}
%========================================

\begin{frame}{AI辅助论文写作}
    \begin{columns}
        \begin{column}{0.5\textwidth}
            \textbf{文献综述:}
            \begin{itemize}
                \item 使用Claude/GPT总结多篇论文
                \item 生成文献对比表格
                \item 识别研究空白
            \end{itemize}
            
            \vspace{0.3cm}
            \textbf{论文撰写:}
            \begin{itemize}
                \item 结构化大纲生成
                \item 段落润色和改写
                \item 学术语言优化
                \item LaTeX公式辅助
            \end{itemize}
            
            \vspace{0.3cm}
            \textbf{推荐工具组合:}
            \begin{enumerate}
                \item Claude Max(长文本处理)
                \item ChatGPT + 联网搜索
                \item Gemini(多模态图表分析)
            \end{enumerate}
        \end{column}
        \begin{column}{0.5\textwidth}
            \begin{exampleblock}{实用Prompt示例}
                \small
                \texttt{请帮我总结以下5篇论文的主要贡献,并以表格形式对比它们的方法、数据集和实验结果:}
                
                \vspace{0.2cm}
                \texttt{请将这段中文摘要改写成学术英语,保持专业性但提高可读性:}
                
                \vspace{0.2cm}
                \texttt{请帮我用LaTeX写出这个损失函数的数学公式,并解释每个符号的含义:}
            \end{exampleblock}
            
            \vspace{0.3cm}
            \begin{alertblock}{注意事项}
                \begin{itemize}
                    \item 始终核实AI生成的引用
                    \item 保持学术诚信
                    \item 作为辅助而非替代
                \end{itemize}
            \end{alertblock}
        \end{column}
    \end{columns}
\end{frame}

\begin{frame}{AI辅助数据分析与可视化}
    \begin{columns}
        \begin{column}{0.5\textwidth}
            \textbf{数据处理:}
            \begin{itemize}
                \item 自动生成数据清洗代码
                \item 统计分析脚本
                \item 异常值检测
            \end{itemize}
            
            \vspace{0.3cm}
            \textbf{可视化:}
            \begin{itemize}
                \item Matplotlib/Seaborn代码生成
                \item 交互式图表(Plotly)
                \item 论文级别图表美化
            \end{itemize}
            
            \vspace{0.3cm}
            \textbf{最佳实践:}
            \begin{enumerate}
                \item 用Cursor/Copilot生成基础代码
                \item 用Claude优化和解释
                \item 用ChatGPT Code Interpreter验证
            \end{enumerate}
        \end{column}
        \begin{column}{0.5\textwidth}
            \begin{tikzpicture}[scale=0.75]
                \begin{axis}[
                    xlabel={实验组},
                    ylabel={性能指标},
                    legend pos=north west,
                    ymajorgrids=true,
                    title={AI生成的示例图表}
                ]
                \addplot[color=blue,mark=*] coordinates {
                    (1,75) (2,78) (3,82) (4,85) (5,89)
                };
                \addplot[color=red,mark=square] coordinates {
                    (1,70) (2,74) (3,79) (4,83) (5,86)
                };
                \legend{Method A, Method B}
                \end{axis}
            \end{tikzpicture}
            
            \begin{exampleblock}{Prompt示例}
                \small
                \texttt{用Python matplotlib生成一个对比我们方法和baseline的折线图,要求论文发表级别质量}
            \end{exampleblock}
        \end{column}
    \end{columns}
\end{frame}

\begin{frame}{AI辅助实验代码开发}
    \begin{columns}
        \begin{column}{0.5\textwidth}
            \textbf{典型工作流:}
            \begin{enumerate}
                \item \textbf{架构设计}:用Claude讨论整体方案
                \item \textbf{代码生成}:用Cursor/Copilot实现
                \item \textbf{调试优化}:用Agent模式自动修复
                \item \textbf{文档生成}:自动生成docstring和README
            \end{enumerate}
            
            \vspace{0.3cm}
            \textbf{深度学习场景:}
            \begin{itemize}
                \item PyTorch/TensorFlow模型代码
                \item 训练循环和验证逻辑
                \item 超参数搜索脚本
                \item 实验日志和可视化
            \end{itemize}
        \end{column}
        \begin{column}{0.5\textwidth}
            \begin{block}{高效技巧}
                \begin{itemize}
                    \item 提供清晰的输入输出规范
                    \item 附上相关论文的伪代码
                    \item 指定使用的库版本
                    \item 要求生成单元测试
                \end{itemize}
            \end{block}
            
            \vspace{0.3cm}
            \begin{alertblock}{Agent模式应用}
                \texttt{@workspace 请帮我实现论文X中的算法Y,参考现有的model.py结构,并添加相应的测试用例}
            \end{alertblock}
        \end{column}
    \end{columns}
\end{frame}

%========================================
\section{实用技巧与最佳实践}
%========================================

\begin{frame}{Prompt工程基础}
    \begin{columns}
        \begin{column}{0.5\textwidth}
            \textbf{CRISPE框架:}
            \begin{itemize}
                \item \textbf{C}apacity - 角色定义
                \item \textbf{R}equest - 任务说明
                \item \textbf{I}nput - 输入数据
                \item \textbf{S}teps - 步骤指引
                \item \textbf{P}ersonalization - 个性化
                \item \textbf{E}xpectation - 期望输出
            \end{itemize}
            
            \vspace{0.3cm}
            \textbf{编程场景优化:}
            \begin{itemize}
                \item 指定编程语言和版本
                \item 说明代码风格偏好
                \item 提供示例输入输出
                \item 要求添加注释
            \end{itemize}
        \end{column}
        \begin{column}{0.5\textwidth}
            \begin{exampleblock}{好的Prompt示例}
                \small
                \texttt{你是一位Python专家。请帮我实现一个高效的LRU缓存类,要求:}
                \begin{itemize}
                    \item \texttt{使用Python 3.10+}
                    \item \texttt{时间复杂度O(1)}
                    \item \texttt{包含类型注解}
                    \item \texttt{附带pytest测试}
                    \item \texttt{添加详细docstring}
                \end{itemize}
            \end{exampleblock}
            
            \begin{alertblock}{避免的做法}
                \begin{itemize}
                    \item 模糊的描述
                    \item 没有上下文
                    \item 期望太宽泛
                \end{itemize}
            \end{alertblock}
        \end{column}
    \end{columns}
\end{frame}

\begin{frame}{工具链整合建议}
    \begin{tikzpicture}[
        node distance=1.5cm,
        box/.style={rectangle, draw, rounded corners, minimum width=2.5cm, minimum height=0.8cm, align=center},
        arrow/.style={->, >=stealth, thick}
    ]
        % 日常开发流程
        \node[box, fill=blue!20] (vscode) {VS Code + Copilot};
        \node[box, fill=purple!20, right=of vscode] (cursor) {Cursor IDE};
        \node[box, fill=green!20, right=of cursor] (claude) {Claude Code};
        
        % 模型层
        \node[box, fill=orange!20, below=of cursor] (models) {GPT-5 / Claude / Gemini};
        
        % 任务类型
        \node[box, fill=gray!20, below left=1cm and 0cm of models] (daily) {日常编码};
        \node[box, fill=gray!20, below=of models] (complex) {复杂重构};
        \node[box, fill=gray!20, below right=1cm and 0cm of models] (terminal) {命令行任务};
        
        % 连接
        \draw[arrow] (vscode) -- (models);
        \draw[arrow] (cursor) -- (models);
        \draw[arrow] (claude) -- (models);
        \draw[arrow] (models) -- (daily);
        \draw[arrow] (models) -- (complex);
        \draw[arrow] (models) -- (terminal);
    \end{tikzpicture}
    
    \vspace{0.5cm}
    \begin{block}{推荐组合}
        \begin{itemize}
            \item \textbf{入门用户}:VS Code + GitHub Copilot Free
            \item \textbf{进阶用户}:Cursor Pro + Claude Pro
            \item \textbf{重度用户}:Cursor Ultra + Claude Max 20x + 多模型切换
            \item \textbf{预算敏感}:Windsurf免费版 + DeepSeek API
        \end{itemize}
    \end{block}
\end{frame}

\begin{frame}{常见问题与解决方案}
    \begin{columns}
        \begin{column}{0.5\textwidth}
            \textbf{Q1: 生成的代码有bug怎么办?}
            \begin{itemize}
                \item 使用Agent模式自动修复
                \item 提供错误信息让AI诊断
                \item 要求生成测试用例验证
            \end{itemize}
            
            \vspace{0.3cm}
            \textbf{Q2: 如何处理大型代码库?}
            \begin{itemize}
                \item 使用\texttt{@workspace}引用上下文
                \item 分模块逐步处理
                \item 使用Cursor的索引功能
            \end{itemize}
            
            \vspace{0.3cm}
            \textbf{Q3: 如何保护代码隐私?}
            \begin{itemize}
                \item 选择企业版(数据不训练)
                \item 考虑本地部署选项
                \item 使用Windsurf企业版
            \end{itemize}
        \end{column}
        \begin{column}{0.5\textwidth}
            \textbf{Q4: 模型选择困难?}
            \begin{itemize}
                \item 启用自动模型选择
                \item 简单任务用轻量模型
                \item 复杂任务用旗舰模型
            \end{itemize}
            
            \vspace{0.3cm}
            \textbf{Q5: 如何控制成本?}
            \begin{itemize}
                \item 使用缓存功能
                \item 合理选择模型级别
                \item 批量处理降低调用次数
                \item 利用免费额度
            \end{itemize}
            
            \vspace{0.3cm}
            \textbf{Q6: 中文支持不好?}
            \begin{itemize}
                \item 优先考虑DeepSeek
                \item Claude中文表现也不错
                \item 用英文prompt + 中文输出
            \end{itemize}
        \end{column}
    \end{columns}
\end{frame}

%========================================
\section{总结与展望}
%========================================

\begin{frame}{关键要点回顾}
    \begin{columns}
        \begin{column}{0.5\textwidth}
            \textbf{大模型选择:}
            \begin{enumerate}
                \item GPT-5系列 - 全能选手
                \item Claude 4.5 - 代码之王
                \item Gemini 3 - 多模态强手
                \item DeepSeek V3 - 性价比首选
            \end{enumerate}
            
            \vspace{0.3cm}
            \textbf{工具选择:}
            \begin{enumerate}
                \item GitHub Copilot - 稳定可靠
                \item Cursor - 极致AI体验
                \item Windsurf - 企业首选
                \item Claude Code - 终端神器
            \end{enumerate}
        \end{column}
        \begin{column}{0.5\textwidth}
            \textbf{实用技能清单:}
            \begin{itemize}
                \item[\checkmark] 掌握至少一个AI编程工具
                \item[\checkmark] 学会基础Prompt工程
                \item[\checkmark] 了解不同模型特点
                \item[\checkmark] 能在科研中合理应用AI
                \item[\checkmark] 知道如何控制成本
                \item[\checkmark] 了解数据隐私保护
            \end{itemize}
            
            \vspace{0.3cm}
            \begin{alertblock}{核心原则}
                AI是\textbf{增强}而非\textbf{替代}\\
                保持批判性思维,验证AI输出
            \end{alertblock}
        \end{column}
    \end{columns}
\end{frame}

%========================================
\section{高级技巧与前沿话题}
%========================================

\begin{frame}{MCP协议:让AI连接真实世界}
    \begin{columns}
        \begin{column}{0.55\textwidth}
            \textbf{什么是MCP (Model Context Protocol)?}
            \begin{itemize}
                \item Anthropic提出的开放协议
                \item 让AI直接操作本地文件、数据库、API
                \it
            \end{itemize}
            
            \vspace{0.3cm}
            \textbf{支持的工具:}
            \begin{itemize}
                \item Claude Desktop / Claude Code
                \item Cursor IDE (原生支持)
                \item Windsurf
                \item VS Code Copilot (部分支持)
            \end{itemize}
            
            \vspace{0.3cm}
            \textbf{科研应用场景:}
            \begin{itemize}
                \item 直接读取/分析本地数据文件
                \item 自动执行实验脚本
                \item 连接文献数据库(Zotero, Mendeley)
            \end{itemize}
        \end{column}
        \begin{column}{0.45\textwidth}
            \begin{block}{常用MCP服务器}
                \begin{itemize}
                    \item \texttt{filesystem} - 文件操作
                    \item \texttt{github} - 代码仓库
                    \item \texttt{postgres/sqlite} - 数据库
                    \item \texttt{brave-search} - 网络搜索
                    \item \texttt{arxiv} - 论文检索
                \end{itemize}
            \end{block}
            
            \vspace{0.3cm}
            \begin{alertblock}{配置示例}
                \small
                在 \texttt{claude\_desktop\_config.json} 中添加MCP服务器配置,即可让Claude直接访问本地资源。
            \end{alertblock}
        \end{column}
    \end{columns}
\end{frame}

\begin{frame}{多模型协作工作流 (Multi-Agent Workflow)}
    \begin{tikzpicture}[
        node distance=1.2cm,
        box/.style={rectangle, draw, rounded corners, minimum width=2.2cm, minimum height=0.7cm, align=center, font=\small},
        arrow/.style={->, >=stealth, thick}
    ]
        % 工作流节点
        \node[box, fill=blue!20] (input) {研究问题};
        \node[box, fill=googleblue!30, right=of input] (gemini) {Gemini 3 Pro\\文献阅读};
        \node[box, fill=deepseekblue!30, right=of gemini] (deepseek) {DeepSeek V3.2\\数学推导};
        \node[box, fill=anthropicred!30, right=of deepseek] (claude) {Claude Opus 4.5\\代码实现};
        \node[box, fill=openaigreen!30, right=of claude] (gpt) {GPT-5.2\\论文写作};
        \node[box, fill=green!20, right=of gpt] (output) {完成论文};
        
        \draw[arrow] (input) -- (gemini);
        \draw[arrow] (gemini) -- (deepseek);
        \draw[arrow] (deepseek) -- (claude);
        \draw[arrow] (claude) -- (gpt);
        \draw[arrow] (gpt) -- (output);
    \end{tikzpicture}
    
    \vspace{0.5cm}
    \begin{block}{各模型分工建议}
        \footnotesize
        \begin{tabular}{p{3cm}p{4cm}p{5cm}}
            \toprule
            \textbf{任务阶段} & \textbf{推荐模型} & \textbf{理由} \\
            \midrule
            文献阅读/长文本 & Gemini 3 Pro & 2M超长上下文,多模态理解 \\
            中文任务/数学推理 & DeepSeek V3.2 & 中文最强,数学能力顶尖 \\
            代码实现 & Claude Opus 4.5 & 代码质量最高,Agent能力强 \\
            论文润色/写作 & GPT-5.2 / Claude & 语言流畅,学术风格好 \\
            图表理解/修改 & Gemini 3 Pro & 多模态能力最强 \\
            \bottomrule
        \end{tabular}
    \end{block}
\end{frame}

\begin{frame}{AI辅助论文自查清单 (Pre-submission Checklist)}
    \begin{columns}
        \begin{column}{0.5\textwidth}
            \textbf{让AI扮演审稿人:}
            \begin{exampleblock}{Prompt (English)}
                \small
                \texttt{"Act as a critical reviewer for [Conference/Journal]. Review my paper and identify: 1. Weak claims without sufficient evidence; 2. Missing related work; 3. Unclear methodology; 4. Overclaiming in the conclusion; 5. Grammar and style issues. Be harsh but constructive."}
            \end{exampleblock}
            
            \vspace{0.3cm}
            \textbf{检查清单:}
            \begin{itemize}
                \item[$\square$] 摘要是否准确概括贡献?
                \item[$\square$] Related Work是否全面?
                \item[$\square$] 方法描述是否可复现?
                \item[$\square$] 实验设置是否公平?
                \item[$\square$] 结论是否过度夸大?
            \end{itemize}
        \end{column}
        \begin{column}{0.5\textwidth}
            \textbf{分段检查Prompt:}
            \begin{exampleblock}{Abstract检查}
                \small
                \texttt{"Does this abstract clearly state: 1. The problem; 2. Our approach; 3. Key results; 4. Main contribution? Suggest improvements."}
            \end{exampleblock}
            
            \begin{exampleblock}{实验检查}
                \small
                \texttt{"Review my experimental section. Check: 1. Are baselines fair and up-to-date? 2. Are datasets appropriate? 3. Is statistical significance reported? 4. Are ablation studies sufficient?"}
            \end{exampleblock}
            
            \begin{alertblock}{重要提醒}
                提交前务必人工核实所有AI建议!
            \end{alertblock}
        \end{column}
    \end{columns}
\end{frame}

\begin{frame}{常见"坑"与避坑指南}
    \begin{columns}
        \begin{column}{0.5\textwidth}
            \textbf{1. AI幻觉 (Hallucination)}
            \begin{itemize}
                \item \textcolor{red}{问题}:AI编造不存在的论文、数据
                \item \textcolor{green!60!black}{解决}:要求提供来源,交叉验证
            \end{itemize}
            
            \vspace{0.2cm}
            \textbf{2. 引用编造}
            \begin{itemize}
                \item \textcolor{red}{问题}:生成看起来真实但不存在的引用
                \item \textcolor{green!60!black}{解决}:永远在Google Scholar验证
            \end{itemize}
            
            \vspace{0.2cm}
            \textbf{3. 代码看似正确实则有bug}
            \begin{itemize}
                \item \textcolor{red}{问题}:逻辑错误、边界条件遗漏
                \item \textcolor{green!60!black}{解决}:要求生成单元测试
            \end{itemize}
        \end{column}
        \begin{column}{0.5\textwidth}
            \textbf{4. 过度自信的错误答案}
            \begin{itemize}
                \item \textcolor{red}{问题}:AI用自信的语气说错话
                \item \textcolor{green!60!black}{解决}:追问"Are you sure?"
            \end{itemize}
            
            \vspace{0.2cm}
            \textbf{5. 上下文遗忘}
            \begin{itemize}
                \item \textcolor{red}{问题}:长对话后忘记早期内容
                \item \textcolor{green!60!black}{解决}:定期总结,使用长上下文模型
            \end{itemize}
            
            \vspace{0.2cm}
            \textbf{6. 隐私泄露}
            \begin{itemize}
                \item \textcolor{red}{问题}:敏感数据被发送到云端
                \item \textcolor{green!60!black}{解决}:使用企业版或本地模型
            \end{itemize}
        \end{column}
    \end{columns}
\end{frame}

\begin{frame}{本地部署开源模型指南}
    \begin{columns}
        \begin{column}{0.55\textwidth}
            \textbf{为什么需要本地部署?}
            \begin{itemize}
                \item 处理敏感/保密数据
                \item 无网络环境
                \item 降低长期成本
                \item 完全的数据控制
            \end{itemize}
            
            \vspace{0.3cm}
            \textbf{推荐开源模型 (2025.12):}
            \begin{itemize}
                \item \textbf{Llama 3.2} (Meta) - 通用能力强
                \item \textbf{Qwen 2.5/3.0} (阿里) - 中文最佳开源
                \item \textbf{DeepSeek-Coder} - 代码专精
                \item \textbf{Mistral Large} - 欧洲之光
            \end{itemize}
            
            \vspace{0.3cm}
            \textbf{部署工具:}
            \begin{itemize}
                \item \texttt{ollama} - 最简单,一键部署
                \item \texttt{vLLM} - 高性能推理
                \item \texttt{LM Studio} - 图形界面
            \end{itemize}
        \end{column}
        \begin{column}{0.45\textwidth}
            \begin{block}{硬件要求参考}
                \small
                \begin{tabular}{ll}
                    \textbf{模型大小} & \textbf{显存需求} \\
                    \midrule
                    7B参数 & 8GB VRAM \\
                    13B参数 & 16GB VRAM \\
                    70B参数 & 48GB+ VRAM \\
                \end{tabular}
            \end{block}
            
            \vspace{0.3cm}
            \begin{exampleblock}{Ollama快速开始}
                \small
                \texttt{\# 安装Ollama}\\
                \texttt{winget install Ollama.Ollama}\\
                \vspace{0.1cm}
                \texttt{\# 运行Qwen中文模型}\\
                \texttt{ollama run qwen2.5:7b}
            \end{exampleblock}
        \end{column}
    \end{columns}
\end{frame}

\begin{frame}{下一步行动}
    \begin{block}{立即可以做的事情}
        \begin{enumerate}
            \item \textbf{今天}:安装VS Code + GitHub Copilot(免费版即可开始)
            \item \textbf{本周}:尝试用AI辅助完成一个小任务
            \item \textbf{本月}:比较Cursor和Copilot,选择适合自己的
            \item \textbf{持续}:关注模型更新,保持学习
        \end{enumerate}
    \end{block}
    
    \vspace{0.5cm}
    \begin{columns}
        \begin{column}{0.5\textwidth}
            \textbf{推荐资源:}
            \begin{itemize}
                \item GitHub Copilot官方文档
                \item Cursor官方教程
                \item Claude提示词工程指南
                \item DeepSeek开发者文档
            \end{itemize}
        \end{column}
        \begin{column}{0.5\textwidth}
            \textbf{学习社区:}
            \begin{itemize}
                \item GitHub Discussions
                \item Cursor Discord
                \item Reddit r/ChatGPT
                \item 知乎AI编程话题
            \end{itemize}
        \end{column}
    \end{columns}
\end{frame}

%========================================
\section{API省钱与数据安全}
%========================================

\begin{frame}{API省钱技巧(一):免费额度汇总}
    \begin{columns}
        \begin{column}{0.5\textwidth}
            \textbf{完全免费的服务:}
            \begin{itemize}
                \item \textcolor{green!60!black}{\faStar\ DeepSeek} - 网页端无限免费
                \item \textcolor{green!60!black}{\faStar\ Kimi} - 长文本处理免费
                \item \textcolor{green!60!black}{\faStar\ 通义千问} - 基础功能免费
                \item \textcolor{green!60!black}{\faStar\ Google AI Studio} - Gemini免费额度
                \item \textcolor{green!60!black}{\faStar\ GitHub Copilot Free} - 有限代码补全
            \end{itemize}
            
            \vspace{0.3cm}
            \textbf{API免费额度:}
            \begin{itemize}
                \item \textbf{OpenAI}:新用户\$5额度(3个月有效)
                \item \textbf{Anthropic}:\$5免费额度
                \item \textbf{Google}:Gemini API免费层
                \item \textbf{DeepSeek}:100万Token/月免费
                \item \textbf{智谱AI}:新用户赠送Token
            \end{itemize}
        \end{column}
        \begin{column}{0.5\textwidth}
            \textbf{学生/教育优惠:}
            \begin{itemize}
                \item \textbf{GitHub Education Pack}
                    \begin{itemize}
                        \item 免费Copilot Pro(价值\$10/月)
                        \item 需.edu邮箱认证
                    \end{itemize}
                \item \textbf{Azure for Students}
                    \begin{itemize}
                        \item \$100免费Azure额度
                        \item 可用于Azure OpenAI
                    \end{itemize}
                \item \textbf{Google Cloud}
                    \begin{itemize}
                        \item 学生\$300免费额度
                    \end{itemize}
            \end{itemize}
            
            \vspace{0.3cm}
            \begin{alertblock}{省钱核心原则}
                \small
                日常用免费服务,关键任务用付费API,学会组合使用!
            \end{alertblock}
        \end{column}
    \end{columns}
\end{frame}

\begin{frame}{API省钱技巧(二):Token计费原理}
    \begin{columns}
        \begin{column}{0.5\textwidth}
            \textbf{什么是Token?}
            \begin{itemize}
                \item 模型处理文本的基本单位
                \item \textbf{英文}:约4字符 = 1 Token
                \item \textbf{中文}:约1-2字 = 1 Token
                \item 1000 Token $\approx$ 750英文单词
                \item 1000 Token $\approx$ 500-700中文字
            \end{itemize}
            
            \vspace{0.3cm}
            \textbf{计费方式:}
            \begin{itemize}
                \item \textbf{输入Token}:你发送的内容
                \item \textbf{输出Token}:AI返回的内容
                \item 通常输出比输入贵2-4倍
                \item 按百万Token (M Token) 计价
            \end{itemize}
        \end{column}
        \begin{column}{0.5\textwidth}
            \textbf{主流模型价格对比(输入/输出):}
            \footnotesize
            \begin{tabular}{lll}
                \toprule
                \textbf{模型} & \textbf{输入} & \textbf{输出} \\
                \midrule
                GPT-4o & \$2.5/M & \$10/M \\
                GPT-4o-mini & \$0.15/M & \$0.6/M \\
                Claude Opus & \$15/M & \$75/M \\
                Claude Sonnet & \$3/M & \$15/M \\
                Gemini Pro & \$1.25/M & \$5/M \\
                DeepSeek V3 & \textcolor{green!60!black}{\$0.27/M} & \textcolor{green!60!black}{\$1.1/M} \\
                \bottomrule
            \end{tabular}
            \normalsize
            
            \vspace{0.3cm}
            \begin{exampleblock}{省钱计算}
                \small
                处理10篇论文(约50万Token):\\
                GPT-4o: $\approx$\$6.25\\
                DeepSeek: $\approx$\$0.68 \textcolor{green!60!black}{(省90\%!)}
            \end{exampleblock}
        \end{column}
    \end{columns}
\end{frame}

\begin{frame}{API省钱技巧(三):优化策略}
    \begin{columns}
        \begin{column}{0.5\textwidth}
            \textbf{1. 缓存策略}
            \begin{itemize}
                \item 相同问题不重复调用
                \item 本地存储常用回答
                \item 使用Redis/SQLite缓存
                \item Anthropic原生支持Prompt缓存
            \end{itemize}
            
            \vspace{0.3cm}
            \textbf{2. Prompt优化}
            \begin{itemize}
                \item 精简System Prompt
                \item 避免重复上下文
                \item 使用few-shot而非长说明
                \item 及时清理对话历史
            \end{itemize}
            
            \vspace{0.3cm}
            \textbf{3. 模型选择策略}
            \begin{itemize}
                \item 简单任务用小模型(mini/haiku)
                \item 复杂任务再用大模型
                \item 先用便宜模型验证Prompt
            \end{itemize}
        \end{column}
        \begin{column}{0.5\textwidth}
            \textbf{4. 批量处理 vs 单次调用}
            \begin{block}{单次调用}
                \small
                每次请求都带完整System Prompt\\
                10次调用 = 10次System Prompt费用
            \end{block}
            
            \begin{exampleblock}{批量处理}
                \small
                一次请求处理多个任务\\
                1次调用 = 1次System Prompt费用\\
                \textcolor{green!60!black}{节省70-90\%成本!}
            \end{exampleblock}
            
            \vspace{0.2cm}
            \textbf{5. 使用中转服务}
            \begin{itemize}
                \item 云雾AI等中转价格更优
                \item 支持多模型切换
                \item 适合国内开发者
            \end{itemize}
            
            \vspace{0.2cm}
            \begin{alertblock}{黄金法则}
                \small
                开发调试用DeepSeek,生产环境按需选择
            \end{alertblock}
        \end{column}
    \end{columns}
\end{frame}

\begin{frame}{数据安全与隐私保护}
    \begin{columns}
        \begin{column}{0.5\textwidth}
            \textbf{\textcolor{red}{绝对不要发送的数据:}}
            \begin{itemize}
                \item[$\times$] 未发表的核心研究成果
                \item[$\times$] 专利申请中的技术细节
                \item[$\times$] 公司机密/商业秘密
                \item[$\times$] 个人隐私信息(身份证、银行卡)
                \item[$\times$] 患者医疗数据
                \item[$\times$] 学生成绩等敏感教育数据
                \item[$\times$] 未脱敏的用户数据
            \end{itemize}
            
            \vspace{0.3cm}
            \textbf{数据脱敏技巧:}
            \begin{itemize}
                \item 用占位符替换敏感信息
                \item 只发送代码结构,不发数据
                \item 使用假数据测试
            \end{itemize}
        \end{column}
        \begin{column}{0.5\textwidth}
            \textbf{企业版 vs 个人版数据政策:}
            \footnotesize
            \begin{tabular}{p{2cm}p{1.8cm}p{1.8cm}}
                \toprule
                \textbf{特性} & \textbf{个人版} & \textbf{企业版} \\
                \midrule
                数据用于训练 & 可能 & 否 \\
                数据保留 & 30天+ & 可配置 \\
                SOC2认证 & 否 & 是 \\
                数据加密 & 传输中 & 全程 \\
                审计日志 & 无 & 有 \\
                \bottomrule
            \end{tabular}
            \normalsize
            
            \vspace{0.3cm}
            \textbf{本地部署 vs 云端服务:}
            \begin{block}{本地部署优势}
                \small
                数据完全本地、无网络依赖、长期成本低
            \end{block}
            \begin{alertblock}{云端服务优势}
                \small
                模型更强、无需硬件、即开即用
            \end{alertblock}
        \end{column}
    \end{columns}
\end{frame}

%========================================
\section{科研场景Prompt模板}
%========================================

\begin{frame}[fragile]{科研Prompt模板(一):LaTeX公式生成}
    \begin{columns}
        \begin{column}{0.5\textwidth}
            \textbf{基础公式生成:}
            \begin{exampleblock}{Prompt模板}
                \scriptsize
                \texttt{请将以下数学表达式转换为LaTeX格式:}\\
                \texttt{[你的公式描述]}\\
                \texttt{要求:}\\
                \texttt{1. 使用equation环境}\\
                \texttt{2. 添加公式编号标签}\\
                \texttt{3. 符号要有注释说明}
            \end{exampleblock}
            
            \vspace{0.2cm}
            \textbf{复杂公式推导:}
            \begin{exampleblock}{Prompt模板}
                \scriptsize
                \texttt{请用LaTeX写出从公式A推导到公式B的完整过程:}\\
                \texttt{起始:[公式A]}\\
                \texttt{目标:[公式B]}\\
                \texttt{使用align环境,每步加注释}
            \end{exampleblock}
        \end{column}
        \begin{column}{0.5\textwidth}
            \textbf{示例输出:}
            \begin{lstlisting}[basicstyle=\ttfamily\tiny,language=TeX]
\begin{equation}
\label{eq:loss}
\mathcal{L} = -\sum_{i=1}^{N} 
  y_i \log(\hat{y}_i)
\end{equation}
% where:
% $\mathcal{L}$: cross-entropy loss
% $N$: number of samples
% $y_i$: ground truth label
% $\hat{y}_i$: predicted probability
            \end{lstlisting}
            
            \vspace{0.2cm}
            \begin{alertblock}{技巧}
                \small
                \begin{itemize}
                    \item 指定使用的宏包(amsmath等)
                    \item 要求符号与论文其他部分一致
                    \item 让AI解释每个符号的含义
                \end{itemize}
            \end{alertblock}
        \end{column}
    \end{columns}
\end{frame}

\begin{frame}[fragile]{科研Prompt模板(二):Related Work撰写}
    \begin{columns}
        \begin{column}{0.5\textwidth}
            \textbf{文献总结Prompt:}
            \begin{exampleblock}{模板}
                \scriptsize
                \texttt{请阅读以下论文摘要,为我的Related Work部分撰写一段总结:}\\
                \texttt{[粘贴3-5篇论文摘要]}\\
                \texttt{要求:}\\
                \texttt{1. 按时间/方法类型分组}\\
                \texttt{2. 指出各方法的优缺点}\\
                \texttt{3. 引用用[Author, Year]格式}\\
                \texttt{4. 最后指出研究空白}\\
                \texttt{5. 学术英语,客观语气}
            \end{exampleblock}
            
            \vspace{0.2cm}
            \textbf{对比表格生成:}
            \begin{exampleblock}{模板}
                \scriptsize
                \texttt{根据以下论文,生成一个方法对比表格(LaTeX tabular格式):}\\
                \texttt{列:方法名、数据集、指标、优点、缺点}
            \end{exampleblock}
        \end{column}
        \begin{column}{0.5\textwidth}
            \textbf{注意事项:}
            \begin{alertblock}{重要!}
                \begin{itemize}
                    \item \textcolor{red}{永远不要让AI编造引用}
                    \item 只让AI总结你提供的文献
                    \item 所有引用必须人工核实
                    \item 在Google Scholar验证每篇论文
                \end{itemize}
            \end{alertblock}
            
            \vspace(0.2cm)
            \textbf{推荐工作流:}
            \begin{enumerate}
                \item 在Semantic Scholar/Google Scholar找论文
                \item 复制摘要给AI总结
                \item AI生成初稿
                \item 人工核实和润色
                \item 添加正确的BibTeX引用
            \end{enumerate}
        \end{column}
    \end{columns}
\end{frame}

\begin{frame}[fragile]{科研Prompt模板(三):实验结果分析}
    \begin{columns}
        \begin{column}{0.5\textwidth}
            \textbf{数据分析Prompt:}
            \begin{exampleblock}{模板}
                \scriptsize
                \texttt{请分析以下实验结果:}\\
                \texttt{[粘贴数据表格或CSV]}\\
                \texttt{要求:}\\
                \texttt{1. 指出主要发现}\\
                \texttt{2. 比较不同方法的性能差异}\\
                \texttt{3. 分析可能的原因}\\
                \texttt{4. 指出异常值或有趣的模式}\\
                \texttt{5. 建议后续实验方向}
            \end{exampleblock}
            
            \vspace{0.2cm}
            \textbf{图表代码生成:}
            \begin{exampleblock}{模板}
                \scriptsize
                \texttt{用Python matplotlib生成论文级别图表:}\\
                \texttt{数据:[你的数据]}\\
                \texttt{要求:Nature/Science风格,300dpi,}\\
                \texttt{Times New Roman字体,合适的图例位置}
            \end{exampleblock}
        \end{column}
        \begin{column}{0.5\textwidth}
            \textbf{统计检验Prompt:}
            \begin{exampleblock}{模板}
                \scriptsize
                \texttt{对以下两组数据进行统计显著性检验:}\\
                \texttt{组A:[数据]}\\
                \texttt{组B:[数据]}\\
                \texttt{请:1. 选择合适的检验方法}\\
                \texttt{2. 报告p值和效应量}\\
                \texttt{3. 给出学术写作中的标准表述}
            \end{exampleblock}
            
            \vspace{0.2cm}
            \textbf{Ablation Study分析:}
            \begin{exampleblock}{模板}
                \scriptsize
                \texttt{分析以下消融实验结果:}\\
                \texttt{[消融实验表格]}\\
                \texttt{写一段300字的分析,说明}\\
                \texttt{每个组件对最终性能的贡献}
            \end{exampleblock}
        \end{column}
    \end{columns}
\end{frame}

\begin{frame}[fragile]{科研Prompt模板(四):Rebuttal撰写}
    \begin{columns}
        \begin{column}{0.5\textwidth}
            \textbf{回复审稿意见Prompt:}
            \begin{exampleblock}{模板}
                \scriptsize
                \texttt{审稿人提出以下问题:}\\
                \texttt{"[审稿意见原文]"}\\
                \texttt{我们的回应要点:}\\
                \texttt{[你的回应要点]}\\
                \texttt{请帮我撰写正式的Rebuttal回复:}\\
                \texttt{1. 首先感谢审稿人}\\
                \texttt{2. 逐点回应,礼貌但有力}\\
                \texttt{3. 引用论文具体章节/图表}\\
                \texttt{4. 说明我们做了哪些修改}\\
                \texttt{5. 专业学术语气}
            \end{exampleblock}
        \end{column}
        \begin{column}{0.5\textwidth}
            \textbf{常用回复句式:}
            \begin{block}{感谢+回应模板}
                \scriptsize
                \texttt{"We thank the reviewer for this}\\
                \texttt{insightful comment. We have}\\
                \texttt{addressed this concern by..."}
            \end{block}
            
            \begin{block}{补充实验模板}
                \scriptsize
                \texttt{"Following the reviewer's}\\
                \texttt{suggestion, we conducted}\\
                \texttt{additional experiments..."}\\
                \texttt{"The new results (Table X)}\\
                \texttt{demonstrate that..."}
            \end{block}
            
            \begin{alertblock}{Rebuttal黄金法则}
                \small
                \begin{itemize}
                    \item 永远保持礼貌和专业
                    \item 直接回应,不要绕弯子
                    \item 承认合理的批评
                    \item 用数据和实验说话
                \end{itemize}
            \end{alertblock}
        \end{column}
    \end{columns}
\end{frame}

%========================================
% 附录
%========================================
\appendix

\begin{frame}{附录:国内用户访问方案}
    \begin{columns}
        \begin{column}{0.5\textwidth}
            \textbf{直接可用(无需翻墙):}
            \begin{itemize}
                \item \textcolor{green!60!black}{\faCheck\ DeepSeek} - \url{chat.deepseek.com}
                \item \textcolor{green!60!black}{\faCheck\ 通义千问} - \url{tongyi.aliyun.com}
                \item \textcolor{green!60!black}{\faCheck\ 智谱清言} - \url{chatglm.cn}
                \item \textcolor{green!60!black}{\faCheck\ 豆包} - \url{doubao.com}
                \item \textcolor{green!60!black}{\faCheck\ Kimi} - \url{kimi.moonshot.cn}
                \item \textcolor{green!60!black}{\faCheck\ 文心一言} - \url{yiyan.baidu.com}
            \end{itemize}
            
            \vspace{0.3cm}
            \textbf{需要科学上网:}
            \begin{itemize}
                \item ChatGPT / OpenAI API
                \item Claude / Anthropic API
                \item Gemini(部分地区可用)
                \item GitHub Copilot(正常可用)
            \end{itemize}
        \end{column}
        \begin{column}{0.5\textwidth}
            \begin{block}{API中转服务推荐}
                \small
                \textbf{云雾AI (yunwu.ai)} - 稳定的API中转:
                \begin{itemize}
                    \item 支持GPT/Claude/Gemini等主流模型
                    \item OpenAI兼容格式,接入方便
                    \item 国内直连,无需特殊网络
                \end{itemize}
            \end{block}
            
            \vspace{0.3cm}
            \begin{alertblock}{推荐方案}
                \begin{enumerate}
                    \item \textbf{日常使用}:DeepSeek(免费+中文最强)
                    \item \textbf{编程任务}:GitHub Copilot(直连)
                    \item \textbf{长文本}:Kimi(支持超长文档)
                    \item \textbf{代码IDE}:Cursor(正常可用)
                    \item \textbf{API开发}:云雾AI中转服务
                \end{enumerate}
            \end{alertblock}
        \end{column}
    \end{columns}
\end{frame}

\begin{frame}[fragile]{附录:云雾AI (yunwu.ai) API中转服务详解}
    \begin{columns}
        \begin{column}{0.5\textwidth}
            \textbf{什么是云雾AI?}
            \begin{itemize}
                \item 国内可直连的API中转服务
                \item 支持GPT-4/Claude/Gemini等主流模型
                \item \textbf{OpenAI兼容格式},几乎零成本迁移
                \item 按量付费,价格透明
            \end{itemize}
            
            \vspace{0.3cm}
            \textbf{核心功能入口:}
            \begin{itemize}
                \item \textbf{在线聊天}:\url{https://yunwu.ai/console/playground}
                \item \textbf{获取Token}:\url{https://yunwu.ai/console/token}
                \item \textbf{查看余额}:\url{https://yunwu.ai/console/topup}
                \item \textbf{使用日志}:\url{https://yunwu.ai/console/log}
            \end{itemize}
            
            \vspace{0.3cm}
            \textbf{API Base URL:}\\
            \texttt{https://yunwu.ai/v1}
        \end{column}
        \begin{column}{0.5\textwidth}
            \begin{exampleblock}{Python调用示例}
                \begin{lstlisting}[language=Python,basicstyle=\ttfamily\tiny]
from openai import OpenAI

client = OpenAI(
    api_key="your-yunwu-token",
    base_url="https://yunwu.ai/v1"
)

response = client.chat.completions.create(
    model="gpt-4o",  # 或其他模型
    messages=[
        {"role": "user", 
         "content": "你好!"}
    ]
)
print(response.choices[0].message.content)
                \end{lstlisting}
            \end{exampleblock}
            
            \begin{alertblock}{使用提示}
                \begin{itemize}
                    \item 先在官网注册获取API Token
                    \item 支持几乎所有OpenAI兼容的SDK
                    \item 敏感数据请评估风险后使用
                \end{itemize}
            \end{alertblock}
        \end{column}
    \end{columns}
\end{frame}

\begin{frame}[fragile]{附录:通过NextChat接入云雾AI(推荐方案)}
    \begin{columns}
        \begin{column}{0.55\textwidth}
            \textbf{什么是NextChat?}
            \begin{itemize}
                \item GitHub 8.6万+ Star的开源项目
                \item 支持Web/桌面/移动端全平台
                \item 可私有部署,数据本地存储
                \item 支持自定义API接口地址
            \end{itemize}
            
            \vspace{0.3cm}
            \textbf{配置云雾AI的方法:}
            \begin{enumerate}
                \item 访问NextChat设置页面
                \item 在"自定义接口"中设置:
                    \begin{itemize}
                        \item API Key: 你的云雾Token
                        \item Base URL: \texttt{https://yunwu.ai}
                    \end{itemize}
                \item 选择需要的模型即可使用
            \end{enumerate}
            
            \vspace{0.3cm}
            \textbf{NextChat官方地址:}\\
            \url{https://github.com/ChatGPTNextWeb/NextChat}
        \end{column}
        \begin{column}{0.45\textwidth}
            \begin{block}{自部署环境变量配置}
                \begin{lstlisting}[basicstyle=\ttfamily\tiny]
# .env.local 文件
OPENAI_API_KEY=your-yunwu-token
BASE_URL=https://yunwu.ai

# 可选:设置访问密码
CODE=your-password

# 可选:自定义模型列表
CUSTOM_MODELS=+gpt-4o,+claude-3-opus
                \end{lstlisting}
            \end{block}
            
            \vspace{0.2cm}
            \begin{exampleblock}{快速体验}
                \small
                \begin{enumerate}
                    \item 访问 \url{https://app.nextchat.club}
                    \item 点击设置 → 自定义接口
                    \item 填入云雾AI的Token和地址
                    \item 开始聊天!
                \end{enumerate}
            \end{exampleblock}
        \end{column}
    \end{columns}
\end{frame}

\begin{frame}[fragile]{附录:更多软件接入云雾API的方法}
    \begin{columns}
        \begin{column}{0.5\textwidth}
            \textbf{1. Cursor IDE 配置}
            \begin{itemize}
                \item 设置 → Models → OpenAI API Key
                \item 填入云雾Token
                \item Override OpenAI Base URL:\\
                    \texttt{https://yunwu.ai/v1}
            \end{itemize}
            
            \vspace{0.3cm}
            \textbf{2. VS Code + Continue扩展}
            \begin{lstlisting}[basicstyle=\ttfamily\tiny]
// config.json
{
  "models": [{
    "title": "GPT-4o via Yunwu",
    "provider": "openai",
    "model": "gpt-4o",
    "apiKey": "your-yunwu-token",
    "apiBase": "https://yunwu.ai/v1"
  }]
}
            \end{lstlisting}
            
            \vspace{0.3cm}
            \textbf{3. LangChain / LlamaIndex}
            \begin{lstlisting}[language=Python,basicstyle=\ttfamily\tiny]
from langchain_openai import ChatOpenAI
llm = ChatOpenAI(
    api_key="your-yunwu-token",
    base_url="https://yunwu.ai/v1",
    model="gpt-4o"
)
            \end{lstlisting}
        \end{column}
        \begin{column}{0.5\textwidth}
            \textbf{4. 命令行工具 (cURL)}
            \begin{lstlisting}[basicstyle=\ttfamily\tiny]
curl https://yunwu.ai/v1/chat/completions \
  -H "Authorization: Bearer YOUR_TOKEN" \
  -H "Content-Type: application/json" \
  -d '{
    "model": "gpt-4o",
    "messages": [{"role": "user", 
                  "content": "Hello!"}]
  }'
            \end{lstlisting}
            
            \vspace{0.3cm}
            \begin{block}{通用接入原则}
                任何支持\textbf{OpenAI API格式}的软件都可以接入:
                \begin{enumerate}
                    \item 将API Key替换为云雾Token
                    \item 将Base URL改为 \texttt{https://yunwu.ai/v1}
                    \item 选择云雾支持的模型名称
                \end{enumerate}
            \end{block}
            
            \vspace{0.2cm}
            \begin{alertblock}{支持的软件列表}
                \small
                NextChat, Cursor, Continue, BotGem, ChatBox, Chatkit, LibreChat, LobeChat, OpenCat, Typing Mind 等
            \end{alertblock}
        \end{column}
    \end{columns}
\end{frame}

\begin{frame}{附录:免费资源大汇总}
    \begin{columns}
        \begin{column}{0.5\textwidth}
            \textbf{完全免费的AI服务:}
            \begin{itemize}
                \item \textcolor{green!60!black}{\faStar\ DeepSeek} - 网页版无限使用
                \item \textcolor{green!60!black}{\faStar\ Kimi} - 超长文本处理
                \item \textcolor{green!60!black}{\faStar\ 通义千问} - 阿里全能助手
                \item \textcolor{green!60!black}{\faStar\ Google AI Studio} - Gemini免费额度
                \item \textcolor{green!60!black}{\faStar\ Colab + Gemini} - 免费GPU+AI
            \end{itemize}
            
            \vspace{0.3cm}
            \textbf{学生/教育免费:}
            \begin{itemize}
                \item \textbf{GitHub Education Pack}
                    \begin{itemize}
                        \item 免费Copilot Pro
                        \item 需edu邮箱认证
                    \end{itemize}
                \item \textbf{Azure for Students}
                    \begin{itemize}
                        \item \$100免费额度
                        \item 可用于OpenAI API
                    \end{itemize}
            \end{itemize}
        \end{column}
        \begin{column}{0.5\textwidth}
            \textbf{开源本地部署(永久免费):}
            \begin{itemize}
                \item \textbf{Ollama} - 一键运行本地模型
                \item \textbf{LM Studio} - 图形化界面
                \item \textbf{推荐模型}:
                    \begin{itemize}
                        \item Qwen2.5 (中文最佳)
                        \item Llama 3.2
                        \item DeepSeek-Coder
                        \item CodeLlama
                    \end{itemize}
            \end{itemize}
            
            \vspace{0.3cm}
            \begin{exampleblock}{零成本科研方案}
                \small
                DeepSeek(对话) + Kimi(长文) + \\
                Copilot Free(代码) + Ollama(本地) \\
                = 完整AI辅助科研工具链
            \end{exampleblock}
        \end{column}
    \end{columns}
\end{frame}

\begin{frame}[fragile]{附录:好Prompt vs 坏Prompt 对比}
    \begin{columns}
        \begin{column}{0.5\textwidth}
            \textbf{\textcolor{red}{\faTimes\ 坏的Prompt:}}
            \begin{lstlisting}[basicstyle=\ttfamily\scriptsize]
帮我写个Python程序
            \end{lstlisting}
            {\scriptsize 问题:太模糊,AI不知道写什么}
            
            \vspace{0.3cm}
            \begin{lstlisting}[basicstyle=\ttfamily\scriptsize]
用最好的方法解决这个问题
            \end{lstlisting}
            {\scriptsize 问题:没有具体问题描述}
            
            \vspace{0.3cm}
            \begin{lstlisting}[basicstyle=\ttfamily\scriptsize]
这代码为什么不work?
[没有附上代码和错误信息]
            \end{lstlisting}
            {\scriptsize 问题:缺少必要上下文}
        \end{column}
        \begin{column}{0.5\textwidth}
            \textbf{\textcolor{green!60!black}{\faCheck\ 好的Prompt:}}
            \begin{lstlisting}[basicstyle=\ttfamily\scriptsize]
用Python实现一个LRU缓存类,要求:
1. 容量可配置
2. O(1)时间复杂度
3. 线程安全
4. 包含类型注解和docstring
            \end{lstlisting}
            
            \vspace{0.2cm}
            \begin{lstlisting}[basicstyle=\ttfamily\scriptsize]
我在用PyTorch训练ResNet时遇到:
RuntimeError: CUDA out of memory
环境:RTX 3080 10GB, batch=64
请分析原因并给出3种解决方案
            \end{lstlisting}
        \end{column}
    \end{columns}
    
    \vspace{0.3cm}
    \begin{block}{Prompt黄金法则}
        \textbf{具体} + \textbf{上下文} + \textbf{期望格式} + \textbf{约束条件} = 高质量输出
    \end{block}
\end{frame}

\begin{frame}[fragile]{附录:可直接复制的System Prompt模板}
    \begin{columns}
        \begin{column}{0.5\textwidth}
            \textbf{1. Python专家模板:}
            \begin{lstlisting}[basicstyle=\ttfamily\tiny]
You are a senior Python developer with 
10+ years experience. Follow these rules:
1. Use Python 3.11+ features
2. Follow PEP8 and type hints
3. Write comprehensive docstrings
4. Prefer functional over OOP when simpler
5. Always consider edge cases
6. Suggest tests for critical functions
            \end{lstlisting}
            
            \vspace{0.2cm}
            \textbf{2. 论文写作助手:}
            \begin{lstlisting}[basicstyle=\ttfamily\tiny]
You are an academic writing expert. 
Guidelines:
1. Use formal, precise language
2. Avoid overclaiming (no "breakthrough")
3. Suggest citations as [Ref] placeholders
4. Follow Nature/Science style
5. Be objective and evidence-based
            \end{lstlisting}
        \end{column}
        \begin{column}{0.5\textwidth}
            \textbf{3. 代码审查专家:}
            \begin{lstlisting}[basicstyle=\ttfamily\tiny]
You are a code reviewer. For each review:
1. Check for bugs and edge cases
2. Evaluate performance implications
3. Assess code readability
4. Suggest specific improvements
5. Rate severity: Critical/Major/Minor
            \end{lstlisting}
            
            \vspace{0.2cm}
            \textbf{4. 数据分析师:}
            \begin{lstlisting}[basicstyle=\ttfamily\tiny]
You are a data scientist. When analyzing:
1. First understand the data structure
2. Check for missing values/outliers
3. Use appropriate statistical tests
4. Visualize results with matplotlib
5. Explain findings for non-experts
            \end{lstlisting}
        \end{column}
    \end{columns}
    
    \vspace{0.2cm}
    {\small \faLightbulb\ 提示:在Claude/ChatGPT的"Custom Instructions"中设置,或在Cursor的Rules中配置}
\end{frame}

\begin{frame}{附录:科研工作流实战案例}
    \begin{block}{案例:用AI工具完成一篇论文的实验部分}
        \small
        \textbf{任务}:实现论文中的算法,跑实验,生成图表
    \end{block}
    
    \begin{columns}
        \begin{column}{0.5\textwidth}
            \textbf{Step 1: 理解算法}
            \begin{itemize}
                \item 工具:Claude Opus 4.5 / Gemini
                \item 上传PDF,让AI解释算法细节
                \item 生成伪代码和流程图
            \end{itemize}
            
            \vspace{0.2cm}
            \textbf{Step 2: 代码实现}
            \begin{itemize}
                \item 工具:Cursor + Claude
                \item 用Agent模式生成代码框架
                \item 逐步实现各个模块
            \end{itemize}
            
            \vspace{0.2cm}
            \textbf{Step 3: 调试修复}
            \begin{itemize}
                \item 工具:Copilot /fix 命令
                \item 粘贴错误信息,自动修复
            \end{itemize}
        \end{column}
        \begin{column}{0.5\textwidth}
            \textbf{Step 4: 运行实验}
            \begin{itemize}
                \item 工具:Claude Code (终端)
                \item 用AI管理实验配置
                \item 自动记录结果
            \end{itemize}
            
            \vspace{0.2cm}
            \textbf{Step 5: 分析结果}
            \begin{itemize}
                \item 工具:ChatGPT Code Interpreter
                \item 上传CSV,自动统计分析
                \item 生成对比表格
            \end{itemize}
            
            \vspace{0.2cm}
            \textbf{Step 6: 生成图表}
            \begin{itemize}
                \item 工具:Cursor + Matplotlib
                \item 生成论文级别图表
                \item 用Gemini检查图表问题
            \end{itemize}
        \end{column}
    \end{columns}
    
    \vspace{0.2cm}
    {\small \textbf{预计时间}:传统方式2-3周 → AI辅助3-5天}
\end{frame}

\begin{frame}{附录:不同任务的最佳工具组合}
    \footnotesize
    \begin{tabular}{p{3cm}p{4cm}p{5cm}}
        \toprule
        \textbf{任务类型} & \textbf{推荐工具组合} & \textbf{说明} \\
        \midrule
        日常编程 & Cursor + Claude Opus & 最强代码体验 \\
        预算有限编程 & VS Code + Copilot Free + DeepSeek & 零成本方案 \\
        论文阅读总结 & Gemini 3 Pro / Kimi & 超长上下文 \\
        中文论文写作 & DeepSeek + Claude润色 & 中文理解+英文润色 \\
        数学推导 & DeepSeek V3.2 / GPT-5.1 Pro & 推理能力强 \\
        数据分析 & ChatGPT Code Interpreter & 直接执行代码 \\
        图表生成 & Cursor + Matplotlib & 快速迭代 \\
        代码审查 & Claude Opus 4.5 & 理解代码最深 \\
        Debug调试 & Copilot Chat /fix & 快速定位问题 \\
        学习新技术 & 任意模型 + 官方文档 & 配合官方资料 \\
        \bottomrule
    \end{tabular}
    
    \vspace{0.3cm}
    \begin{alertblock}{核心原则}
        没有万能工具,根据任务特点选择最合适的组合。熟练掌握2-3个工具比浅尝辄止更有效。
    \end{alertblock}
\end{frame}

\begin{frame}{附录:本周AI热点速递(12月8-10日)}
    \begin{columns}
        \begin{column}{0.5\textwidth}
            \textbf{\textcolor{red}{重磅:MCP成为行业标准}}
            \begin{itemize}
                \item 12/10:Anthropic将MCP协议捐赠给新成立的Agentic AI Foundation
                \item 12/9:OpenAI共同成立该基金会,发布AGENTS.md标准
                \item 意义:AI Agent互操作性将大幅提升
            \end{itemize}
            
            \vspace{0.3cm}
            \textbf{OpenAI动态:}
            \begin{itemize}
                \item 12/9:首批官方认证课程发布
                \item 12/9:任命新CRO Denise Dresser
                \item 12/8:与Deutsche Telekom合作
                \item 12/8:与Instacart购物体验合作
                \item 12/3:收购Neptune公司
            \end{itemize}
        \end{column}
        \begin{column}{0.5\textwidth}
            \textbf{Anthropic动态:}
            \begin{itemize}
                \item 12/3:收购Bun(JavaScript运行时)
                \item 12/3:Claude Code达\$10亿收入!
                \item 12/4:与Snowflake \$2亿合作
                \item 12/9:与Accenture多年合作
            \end{itemize}
            
            \vspace{0.3cm}
            \textbf{GitHub Copilot更新:}
            \begin{itemize}
                \item 12/8:Coding Agent模型选择
                \item 12/5:代码生成指标仪表板
                \item 12/4:GPT-5.1-Codex-Max公测
                \item 12/3:Claude Opus 4.5全平台
                \item 12/1:Copilot Spaces公共空间
            \end{itemize}
        \end{column}
    \end{columns}
    
    \vspace{0.2cm}
    {\small \textbf{数据来源}:OpenAI Blog, Anthropic News, GitHub Changelog(2025年12月10日更新)}
\end{frame}

\begin{frame}{附录:价格速查表}
    \footnotesize
    \begin{tabular}{llll}
        \toprule
        \textbf{产品} & \textbf{免费版} & \textbf{个人版} & \textbf{团队/企业版} \\
        \midrule
        ChatGPT Plus & 有限使用 & \$20/月 & \$25-30/用户/月 \\
        Claude Pro/Max & 有限使用 & \$20-200/月 & \$30/用户/月起 \\
        Gemini Advanced & 免费额度 & \$20/月 & 企业定制 \\
        GitHub Copilot & 有限功能 & \$10-39/月 & \$19-39/用户/月 \\
        Cursor & 有限使用 & \$20-200/月 & \$40/用户/月 \\
        Windsurf & 免费版可用 & 定制 & 企业定制 \\
        DeepSeek & \textcolor{green!60!black}{完全免费} & API按量 & API按量 \\
        \bottomrule
    \end{tabular}
    
    \vspace{0.5cm}
    \begin{block}{省钱技巧}
        \begin{itemize}
            \item 学生/教师可申请GitHub Education Pack(免费Copilot)
            \item 开源项目维护者可申请免费额度
            \item 合理使用缓存和批处理API降低成本
            \item DeepSeek提供极具竞争力的免费服务
        \end{itemize}
    \end{block}
\end{frame}

\begin{frame}{附录:快捷键速查}
    \begin{columns}
        \begin{column}{0.5\textwidth}
            \textbf{GitHub Copilot (VS Code):}
            \begin{itemize}
                \item \texttt{Tab} - 接受建议
                \item \texttt{Esc} - 拒绝建议
                \item \texttt{Alt+]} - 下一个建议
                \item \texttt{Alt+[} - 上一个建议
                \item \texttt{Ctrl+Enter} - 打开建议面板
                \item \texttt{Ctrl+I} - 打开Copilot Chat
            \end{itemize}
            
            \vspace{0.3cm}
            \textbf{Cursor:}
            \begin{itemize}
                \item \texttt{Tab} - 接受补全
                \item \texttt{Cmd/Ctrl+K} - 行内编辑
                \item \texttt{Cmd/Ctrl+L} - 打开Chat
                \item \texttt{Cmd/Ctrl+I} - Composer
            \end{itemize}
        \end{column}
        \begin{column}{0.5\textwidth}
            \textbf{Copilot Chat命令:}
            \begin{itemize}
                \item \texttt{/explain} - 解释代码
                \item \texttt{/fix} - 修复问题
                \item \texttt{/tests} - 生成测试
                \item \texttt{/doc} - 生成文档
                \item \texttt{/optimize} - 优化代码
                \item \texttt{@workspace} - 项目上下文
                \item \texttt{@vscode} - 编辑器操作
            \end{itemize}
            
            \vspace{0.3cm}
            \textbf{Claude Code:}
            \begin{itemize}
                \item 直接在终端输入\texttt{claude}
                \item 支持管道输入
                \item 可以执行shell命令
            \end{itemize}
        \end{column}
    \end{columns}
\end{frame}

\begin{frame}{附录:科研全流程 Prompt 指南 (1/6) - 模型选择}
    \begin{block}{不同任务的最佳模型选择 (2025.12版)}
        \begin{itemize}
            \item \textbf{长文本阅读与润色}:\textcolor{googleblue}{Gemini 3 Pro} (2M context), \textcolor{anthropicred}{Claude Opus 4.5} (200k)
                \begin{itemize}
                    \item \textit{理由:超长上下文窗口,能一次性读完整本书或多篇论文,记忆力强。}
                \end{itemize}
            \item \textbf{图形修改与多模态}:\textcolor{googleblue}{Gemini 3 Pro}
                \begin{itemize}
                    \item \textit{理由:原生多模态能力最强,能精准理解图表细节并提出修改建议。}
                \end{itemize}
            \item \textbf{数学推理与理论推导}:\textcolor{deepseekblue}{DeepSeek V3.2 (中文最强)}, \textcolor{openaigreen}{GPT-5.2 Pro}
                \begin{itemize}
                    \item \textit{理由:DeepSeek在数学和逻辑推理上已超越旧版o1模型,且对中文数学术语支持最好。}
                \end{itemize}
            \item \textbf{代码实现与实验}:\textcolor{anthropicred}{Claude Opus 4.5}, \textcolor{openaigreen}{GPT-5.2}
                \begin{itemize}
                    \item \textit{理由:Claude 4.5是目前公认的代码之王,GPT-5.2紧随其后。}
                \end{itemize}
        \end{itemize}
    \end{block}
\end{frame}

\begin{frame}{附录:科研全流程 Prompt 指南 (2/6) - 选题与文献}
    \textbf{1. 选题与头脑风暴 (Brainstorming)}
    \begin{exampleblock}{Prompt 示例 (加入探索/联想关键词)}
        \small
        \texttt{"I am researching [field/topic], and the current pain point is [specific problem]. Please \textbf{explore} potential research directions by \textbf{associating} with the latest trends in [conference/journal]. \textbf{Discover} 3-5 innovative ideas. For each: 1. Core novelty; 2. Feasibility analysis; 3. Potential challenges."}
    \end{exampleblock}
    
    \vspace{0.3cm}
    \textbf{2. 文献调研与总结 (Literature Review)}
    \begin{exampleblock}{Prompt 示例 (推荐模型: Gemini 3 Pro / Claude Opus 4.5)}
        \small
        \texttt{"Please read the 5 uploaded PDF papers. Summarize them for a beginner. Requirements: 1. Extract the core contribution of each paper; 2. Compare their methodological similarities and differences (create a comparison table); 3. \textbf{Identify} and \textbf{discover} the common limitations, and \textbf{explore} possible future improvements."}
    \end{exampleblock}
\end{frame}

\begin{frame}{附录:科研全流程 Prompt 指南 (3/6) - 数学与理论}
    \textbf{3. 数学公式推导 (Mathematical Derivation)}
    \begin{exampleblock}{Prompt 示例 (推荐模型: DeepSeek V3.2, GPT-5.2 Pro)}
        \small
        \texttt{"I have defined a loss function $L = \dots$. Please derive the gradient $\nabla_x L$ with respect to variable $x$. Show the derivation step by step, and check for potential gradient vanishing or explosion risks."}
    \end{exampleblock}
    
    \vspace{0.3cm}
    \textbf{4. 理论证明检查 (Proof Checking)}
    \begin{exampleblock}{Prompt 示例}
        \small
        \texttt{"Here is my proof draft for Theorem X: [paste proof]. Act as a rigorous reviewer. Check for logical gaps, especially whether the assumptions are sufficient and whether the derivation steps are rigorous. If there are errors, point out the exact location and provide correction suggestions."}
    \end{exampleblock}
\end{frame}

\begin{frame}{附录:科研全流程 Prompt 指南 (4/6) - 代码与实验}
    \textbf{5. 算法实现 (Implementation)}
    \begin{exampleblock}{Prompt 示例 (推荐模型: Claude Opus 4.5)}
        \small
        \texttt{"Please implement Algorithm 1 from the appendix using PyTorch. Requirements: 1. PEP8 compliant code style; 2. Include detailed type annotations and docstrings; 3. Optimize for GPU parallel computation; 4. Provide a simple dummy data test case to verify dimension correctness."}
    \end{exampleblock}
    
    \vspace{0.3cm}
    \textbf{6. 实验报错调试 (Debugging)}
    \begin{exampleblock}{Prompt 示例}
        \small
        \texttt{"My code encountered the following error: [paste error message]. Here is the relevant code snippet: [paste code]. Please analyze the root cause and provide the fixed code. Explain why this error occurred and how to avoid it in the future."}
    \end{exampleblock}
\end{frame}

\begin{frame}{附录:科研全流程 Prompt 指南 (5/6) - 数据与绘图}
    \textbf{7. 数据分析 (Data Analysis)}
    \begin{exampleblock}{Prompt 示例 (推荐模型: GPT-5.2 Code Interpreter)}
        \small
        \texttt{"I have uploaded the experimental results CSV file. Please analyze: 1. The impact of different hyperparameters on model performance; 2. Whether there is a statistically significant difference (perform t-test); 3. \textbf{Identify} and \textbf{discover} the Pareto-optimal configurations."}
    \end{exampleblock}
    
    \vspace{0.3cm}
    \textbf{8. 科学绘图与修改 (Visualization)}
    \begin{exampleblock}{Prompt 示例 (推荐模型: Gemini 3 Pro)}
        \small
        \texttt{"Please use Python Matplotlib to create a line chart comparing Method A and Method B. Use academic-style colors (e.g., Science journal style), Times New Roman font, size 12."}
        \vspace{0.1cm}
        \texttt{"(Upload image) The legend in this figure blocks the data curve. Please move the legend to the upper left corner, change the blue curve to dashed line, and make the red curve thicker."}
    \end{exampleblock}
\end{frame}

\begin{frame}{附录:科研全流程 Prompt 指南 (6/6) - 写作与投稿}
    \textbf{9. 论文写作 (Writing)}
    \begin{exampleblock}{Prompt 示例 (推荐模型: Claude Opus 4.5)}
        \small
        \texttt{"Here are my experimental results and method description. Please write the Introduction section. Requirements: 1. Use inverted pyramid structure; 2. Emphasize the novelty of our method; 3. Professional and objective language, avoid overclaiming; 4. Cite relevant classic literature (use [Ref] as placeholder)."}
    \end{exampleblock}
    
    \vspace{0.3cm}
    \textbf{10. 审稿回复 (Rebuttal)}
    \begin{exampleblock}{Prompt 示例}
        \small
        \texttt{"The reviewer raised this question: [paste question]. We have actually discussed this in the experimental section (see Section 4.2). Please help me draft a polite yet strong response. Requirements: 1. Thank the reviewer for the suggestion; 2. Point to the corresponding content in our paper; 3. Add a new explanation to further clarify."}
    \end{exampleblock}
\end{frame}

\begin{frame}{附录:科研全流程 Prompt 指南 - 高级技巧}
    \begin{block}{何时使用"探索/联想/发现"关键词?}
        \begin{itemize}
            \item \textbf{Explore}:当需要模型发散思考、提出多种可能性时
            \item \textbf{Associate / Connect}:当希望模型找到不同概念之间的联系时
            \item \textbf{Discover / Identify}:当希望模型从数据或文本中挖掘隐藏的规律时
            \item \textbf{Brainstorm}:当需要创意和头脑风暴时
        \end{itemize}
    \end{block}
    
    \vspace{0.3cm}
    \begin{exampleblock}{高级 Prompt 示例 (带探索关键词)}
        \small
        \texttt{"Please \textbf{explore} unconventional approaches to solve [problem]. \textbf{Associate} techniques from [Field A] with methods in [Field B]. \textbf{Discover} potential synergies. Think step-by-step and \textbf{brainstorm} at least 5 creative solutions, even if some seem unconventional."}
    \end{exampleblock}
\end{frame}

\begin{frame}
    \begin{center}
        \Huge\textbf{谢谢!}
        
        \vspace{1cm}
        \Large 问题与讨论
        
        \vspace{1cm}
        \normalsize
        \begin{tabular}{rl}
            \faGlobe & \url{https://github.com/copilot} \\
            \faGlobe & \url{https://cursor.com} \\
            \faGlobe & \url{https://claude.ai} \\
            \faGlobe & \url{https://chat.deepseek.com} \\
        \end{tabular}
        
        \vspace{0.5cm}
        {\small 演示文稿更新日期:2025年12月10日}\\
        {\scriptsize 数据来源:OpenAI Blog, Anthropic News, GitHub Changelog, Google AI Blog, Cursor Changelog}
    \end{center}
\end{frame}

\end{document}
