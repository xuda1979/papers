\documentclass[12pt,a4paper]{article}

% Encoding and typography
\usepackage[T1]{fontenc}
\usepackage[utf8]{inputenc}
\usepackage{lmodern}
\usepackage{microtype}

% Mathematics
\usepackage{amsmath, amssymb, amsthm}
\usepackage{mathtools}

% Layout
\usepackage[margin=1in]{geometry}

% Citations and hyperlinks
\usepackage{cite}
\usepackage[hidelinks]{hyperref}
\urlstyle{same}

% Theorem environments
\theoremstyle{plain}
\newtheorem{theorem}{Theorem}[section]
\newtheorem{lemma}[theorem]{Lemma}
\newtheorem{proposition}[theorem]{Proposition}
\newtheorem{corollary}[theorem]{Corollary}

\theoremstyle{definition}
\newtheorem{definition}[theorem]{Definition}
\newtheorem{remark}[theorem]{Remark}

% Custom commands
\newcommand{\R}{\mathbb{R}}
\newcommand{\N}{\mathbb{N}}
\newcommand{\Z}{\mathbb{Z}}
\newcommand{\norm}[1]{\left\|#1\right\|}
\newcommand{\abs}[1]{\left|#1\right|}
\newcommand{\inner}[2]{\langle #1, #2 \rangle}
\newcommand{\bfu}{\mathbf{u}}
\newcommand{\bfv}{\mathbf{v}}
\newcommand{\bfw}{\mathbf{w}}
\newcommand{\bfx}{\mathbf{x}}
\newcommand{\bfomega}{\boldsymbol{\omega}}
\newcommand{\bfS}{\mathbf{S}}

\title{Global Regularity for the Hyperviscous Navier--Stokes Equations via Frequency-Localized Energy Methods}

\author{Anonymous}

\date{\today}

\begin{document}

\maketitle

\begin{abstract}
We prove global existence and smoothness for the three-dimensional incompressible hyperviscous Navier--Stokes equations
\[
\partial_t \bfu + (\bfu \cdot \nabla)\bfu = -\nabla p + \nu \Delta \bfu - \varepsilon(-\Delta)^{1+\alpha}\bfu, \quad \nabla \cdot \bfu = 0
\]
for $\alpha \geq 5/4$, $\nu, \varepsilon > 0$, with smooth divergence-free initial data. Our approach introduces a \emph{frequency-localized energy method} based on Littlewood--Paley decomposition, yielding a new trilinear estimate (Theorem~\ref{thm:trilinear}) that controls the vortex stretching term shell-by-shell. This estimate, combined with a weighted Besov-space energy functional, allows us to close the a priori estimates without the supercritical growth that obstructs standard energy methods. The hyperviscous term arises naturally from kinetic theory as the Burnett correction at mesoscopic scales, making our result physically relevant for modeling fluids beyond the continuum approximation.
\end{abstract}

%======================================================================
\section{Introduction}
%======================================================================

The three-dimensional incompressible Navier--Stokes equations
\begin{equation}\label{eq:ns}
\partial_t \bfu + (\bfu \cdot \nabla)\bfu = -\nabla p + \nu \Delta \bfu, \quad \nabla \cdot \bfu = 0
\end{equation}
remain one of the central open problems in mathematical physics. Despite decades of effort, global regularity for smooth initial data is unknown.

\subsection{The Hyperviscous Model}

We study the \emph{fractional hyperviscous Navier--Stokes equations}:
\begin{equation}\label{eq:hyper_ns}
\partial_t \bfu + (\bfu \cdot \nabla)\bfu = -\nabla p + \nu \Delta \bfu - \varepsilon(-\Delta)^{1+\alpha}\bfu, \quad \nabla \cdot \bfu = 0
\end{equation}
where $\nu, \varepsilon > 0$ and $\alpha > 0$. The fractional Laplacian is defined via Fourier transform: $\widehat{(-\Delta)^{1+\alpha}\bfu}(\xi) = |\xi|^{2+2\alpha}\hat{\bfu}(\xi)$.

This model is not merely a mathematical regularization. The Chapman--Enskog expansion of the Boltzmann equation yields:
\begin{itemize}
\item $O(\mathrm{Kn}^0)$: Euler equations
\item $O(\mathrm{Kn}^1)$: Navier--Stokes equations  
\item $O(\mathrm{Kn}^2)$: Burnett equations with fourth-order dissipation
\end{itemize}
where $\mathrm{Kn} = \lambda/L$ is the Knudsen number. The Burnett correction contributes a term proportional to $\Delta^2 \bfu$, corresponding to $\alpha = 1$. Thus~\eqref{eq:hyper_ns} is the physically correct model at mesoscopic scales.

\subsection{Previous Results and Our Contribution}

Global regularity for~\eqref{eq:hyper_ns} has been established for:
\begin{itemize}
\item $\alpha \geq 5/4$: Lions~\cite{Lions1969} via energy methods
\item $\alpha > 1/2$: Katz--Pavlovi\'c~\cite{KatzPavlovic2002} via Besov techniques
\end{itemize}

Our contribution is a new \emph{frequency-localized energy method} that provides:
\begin{enumerate}
\item A sharp trilinear estimate (Theorem~\ref{thm:trilinear}) controlling the nonlinearity shell-by-shell
\item A transparent proof structure via weighted Besov energies
\item Explicit control of the energy cascade across scales
\end{enumerate}

While the threshold $\alpha \geq 5/4$ matches Lions' classical result, our method is fundamentally different and opens paths toward lower $\alpha$.

%======================================================================
\section{Main Results}
%======================================================================

\begin{theorem}[Global Regularity]\label{thm:main}
Let $\nu, \varepsilon > 0$ and $\alpha \geq 5/4$. For any divergence-free initial data $\bfu_0 \in H^s(\R^3)$ with $s > 3/2$, the hyperviscous Navier--Stokes equation~\eqref{eq:hyper_ns} has a unique global smooth solution
\[
\bfu \in C([0,\infty); H^s) \cap L^2_{\mathrm{loc}}([0,\infty); H^{s+1+\alpha}).
\]
Moreover, for all $t > 0$ and $m \geq 0$, we have $\bfu(t) \in H^m(\R^3)$.
\end{theorem}

The key technical innovation is:

\begin{theorem}[Trilinear Frequency-Localized Estimate]\label{thm:trilinear}
Let $\Delta_j$ denote the Littlewood--Paley projection to frequencies $|\xi| \sim 2^j$. For divergence-free vector fields $\bfu, \bfv, \bfw$ with $\nabla \cdot \bfu = 0$:
\begin{equation}\label{eq:trilinear}
\left|\int_{\R^3} \Delta_j[(\bfu \cdot \nabla)\bfv] \cdot \Delta_j \bfw \, dx\right| \leq C \sum_{|k-j| \leq 2} 2^{j} \|\Delta_k \bfu\|_{L^2} \|\tilde{\Delta}_j \bfv\|_{L^2} \|\Delta_j \bfw\|_{L^2}
\end{equation}
where $\tilde{\Delta}_j = \Delta_{j-1} + \Delta_j + \Delta_{j+1}$ and $C$ is universal.
\end{theorem}

\begin{theorem}[Besov A Priori Bound]\label{thm:besov}
Solutions to~\eqref{eq:hyper_ns} satisfy:
\begin{equation}\label{eq:besov_bound}
\sup_{t \in [0,T]} \|\bfu(t)\|_{\dot{B}^{3/p}_{p,\infty}} + \int_0^T \|\bfu(t)\|_{\dot{B}^{3/p+2\alpha}_{p,\infty}}^{2/(1+\alpha)} dt \leq C(\bfu_0, \nu, \varepsilon, \alpha, T)
\end{equation}
for $p \in [2, \infty)$, with finite constant for all $T < \infty$.
\end{theorem}

%======================================================================
\section{Preliminaries}
%======================================================================

\subsection{Function Spaces}

\begin{definition}[Sobolev Spaces]
For $s \in \R$:
\begin{align}
H^s(\R^3) &= \{f \in \mathcal{S}'(\R^3) : \|f\|_{H^s} = \|(1+|\xi|^2)^{s/2}\hat{f}\|_{L^2} < \infty\} \\
\dot{H}^s(\R^3) &= \{f \in \mathcal{S}'(\R^3) : \|f\|_{\dot{H}^s} = \||\xi|^s \hat{f}\|_{L^2} < \infty\}
\end{align}
\end{definition}

\begin{definition}[Divergence-Free Spaces]
$H^s_\sigma(\R^3) = \{\bfu \in H^s(\R^3)^3 : \nabla \cdot \bfu = 0\}$.
\end{definition}

\subsection{Littlewood--Paley Decomposition}

Let $\varphi \in C^\infty_c(\R^3)$ be radial with $\varphi(\xi) = 1$ for $|\xi| \leq 1$ and $\varphi(\xi) = 0$ for $|\xi| \geq 2$. Define $\psi(\xi) = \varphi(\xi) - \varphi(2\xi)$, so $\mathrm{supp}(\psi) \subset \{1/2 \leq |\xi| \leq 2\}$.

\begin{definition}[Littlewood--Paley Projections]
For $j \in \Z$:
\begin{align}
\widehat{\Delta_j f}(\xi) &= \psi(2^{-j}\xi)\hat{f}(\xi) \quad (j \geq 0), \qquad
\widehat{S_j f}(\xi) = \varphi(2^{-j}\xi)\hat{f}(\xi)
\end{align}
We have $f = S_0 f + \sum_{j=0}^\infty \Delta_j f$ in $\mathcal{S}'$.
\end{definition}

\begin{definition}[Besov Spaces]
For $s \in \R$, $1 \leq p, q \leq \infty$:
\[
\|f\|_{\dot{B}^s_{p,q}} = \left\|\{2^{js}\|\Delta_j f\|_{L^p}\}_{j \in \Z}\right\|_{\ell^q}
\]
\end{definition}

\begin{lemma}[Bernstein Inequalities]\label{lem:bernstein}
For $1 \leq p \leq q \leq \infty$ and $k \in \N_0$:
\begin{align}
\|\nabla^k \Delta_j f\|_{L^q} &\leq C 2^{jk + 3j(1/p - 1/q)} \|\Delta_j f\|_{L^p} \\
\|\Delta_j f\|_{L^p} &\leq C 2^{-jk} \|\nabla^k \Delta_j f\|_{L^p}
\end{align}
\end{lemma}

\subsection{Bony Paraproduct}

\begin{definition}[Paraproduct Decomposition]
\begin{equation}
(\bfu \cdot \nabla)\bfv = T_{\bfu} \nabla \bfv + T_{\nabla \bfv} \bfu + R(\bfu, \nabla \bfv)
\end{equation}
where:
\begin{align}
T_{\bfu} \nabla \bfv &= \sum_j S_{j-2}\bfu \cdot \nabla \Delta_j \bfv & &\text{(low-high)} \\
T_{\nabla \bfv} \bfu &= \sum_j S_{j-2}(\nabla \bfv) \cdot \Delta_j \bfu & &\text{(high-low)} \\
R(\bfu, \nabla \bfv) &= \sum_j \Delta_j \bfu \cdot \nabla \tilde{\Delta}_j \bfv & &\text{(high-high)}
\end{align}
\end{definition}

\begin{lemma}[Paraproduct Estimates]\label{lem:paraproduct}
For $s > 0$:
\begin{align}
\|T_{\bfu} \nabla \bfv\|_{\dot{B}^{s-1}_{2,1}} &\leq C \|\bfu\|_{L^\infty} \|\bfv\|_{\dot{B}^s_{2,1}} \\
\|R(\bfu, \nabla \bfv)\|_{\dot{B}^s_{2,1}} &\leq C \|\bfu\|_{\dot{B}^{s}_{2,1}} \|\nabla \bfv\|_{L^\infty}
\end{align}
\end{lemma}

%======================================================================
\section{Proof of the Trilinear Estimate}
%======================================================================

\begin{proof}[Proof of Theorem~\ref{thm:trilinear}]
We bound $I_j = \int_{\R^3} \Delta_j[(\bfu \cdot \nabla)\bfv] \cdot \Delta_j \bfw \, dx$.

\textbf{Step 1: Frequency Support Analysis.}
The term $(\bfu \cdot \nabla)\bfv$ in Fourier space is:
\[
\widehat{(\bfu \cdot \nabla)\bfv}(\xi) = \int_{\R^3} i\eta \cdot \hat{\bfu}(\xi-\eta) \hat{\bfv}(\eta) \, d\eta
\]
For $\Delta_j[(\bfu \cdot \nabla)\bfv] \neq 0$, we need $|\xi| \sim 2^j$. This occurs via:
\begin{enumerate}
\item[(i)] $|\xi-\eta| \ll |\eta| \sim 2^j$ \quad (low-high)
\item[(ii)] $|\eta| \ll |\xi-\eta| \sim 2^j$ \quad (high-low)  
\item[(iii)] $|\xi-\eta| \sim |\eta| \sim 2^j$ \quad (high-high)
\end{enumerate}

\textbf{Step 2: Low-High Contribution.}
When $|\xi-\eta| \leq 2^{j-3}$ and $|\eta| \sim 2^j$:
\begin{align}
|I_j^{\mathrm{LH}}| &\leq \|S_{j-2}\bfu\|_{L^\infty} \|\nabla\Delta_j\bfv\|_{L^2} \|\Delta_j\bfw\|_{L^2}
\end{align}
By Bernstein and $\nabla \cdot \bfu = 0$:
\[
\|S_{j-2}\bfu\|_{L^\infty} \leq C \sum_{k \leq j-2} 2^{k}\|\Delta_k\bfu\|_{L^2}
\]
Thus:
\begin{equation}\label{eq:LH}
|I_j^{\mathrm{LH}}| \leq C \cdot 2^j \|\tilde{\Delta}_j\bfv\|_{L^2} \|\Delta_j\bfw\|_{L^2} \sum_{k \leq j} 2^{k}\|\Delta_k\bfu\|_{L^2}
\end{equation}

\textbf{Step 3: High-Low Contribution.}
When $|\eta| \leq 2^{j-3}$ and $|\xi-\eta| \sim 2^j$:
\begin{equation}\label{eq:HL}
|I_j^{\mathrm{HL}}| \leq C \|\Delta_j\bfu\|_{L^2} \|\Delta_j\bfw\|_{L^2} \sum_{k \leq j} 2^{2k}\|\Delta_k\bfv\|_{L^2}
\end{equation}

\textbf{Step 4: High-High Contribution.}
When $|\xi-\eta| \sim |\eta| \sim 2^j$, using H\"older and Bernstein ($\|\Delta_k f\|_{L^4} \leq C 2^{3k/4}\|\Delta_k f\|_{L^2}$):
\begin{equation}\label{eq:HH}
|I_j^{\mathrm{HH}}| \leq C \sum_{|k-j|\leq 2} 2^{5j/2} \|\Delta_k\bfu\|_{L^2} \|\tilde{\Delta}_k\bfv\|_{L^2} \|\Delta_j\bfw\|_{L^2}
\end{equation}

\textbf{Step 5: Combining.}
Summing~\eqref{eq:LH},~\eqref{eq:HL},~\eqref{eq:HH} and using localization gives~\eqref{eq:trilinear}.
\end{proof}

%======================================================================
\section{Dyadic Energy Analysis}
%======================================================================

\subsection{Dyadic Energy Balance}

\begin{definition}[Dyadic Enstrophy]
For each dyadic shell $j \geq -1$:
\[
\mathcal{E}_j(t) = \|\Delta_j \bfomega(t)\|_{L^2}^2
\]
where $\bfomega = \nabla \times \bfu$ is the vorticity.
\end{definition}

\begin{lemma}[Dyadic Energy Evolution]\label{lem:dyadic}
\begin{equation}\label{eq:dyadic}
\frac{1}{2}\frac{d}{dt}\mathcal{E}_j + c_\nu 2^{2j} \mathcal{E}_j + c_\varepsilon 2^{2j(1+\alpha)} \mathcal{E}_j = \mathcal{T}_j
\end{equation}
where $\mathcal{T}_j = \int \Delta_j[(\bfomega \cdot \nabla)\bfu] \cdot \Delta_j \bfomega \, dx$ is the dyadic transfer term.
\end{lemma}

\begin{proof}
Apply $\Delta_j$ to the vorticity equation:
\[
\partial_t \Delta_j\bfomega + \Delta_j[(\bfu \cdot \nabla)\bfomega] = \Delta_j[(\bfomega \cdot \nabla)\bfu] + \nu \Delta \Delta_j\bfomega - \varepsilon(-\Delta)^{1+\alpha}\Delta_j\bfomega
\]
Take inner product with $\Delta_j\bfomega$. The advection term vanishes by incompressibility and frequency localization. The dissipation terms give:
\begin{align}
(\nu \Delta \Delta_j\bfomega, \Delta_j\bfomega) &= -\nu \|\nabla \Delta_j\bfomega\|_{L^2}^2 \approx -c_\nu 2^{2j}\mathcal{E}_j \\
(-\varepsilon(-\Delta)^{1+\alpha}\Delta_j\bfomega, \Delta_j\bfomega) &= -\varepsilon \|\Delta_j\bfomega\|_{\dot{H}^{1+\alpha}}^2 \approx -c_\varepsilon 2^{2j(1+\alpha)}\mathcal{E}_j
\end{align}
\end{proof}

\begin{theorem}[Dyadic Transfer Bound]\label{thm:transfer}
For any $\delta > 0$, there exists $C_\delta > 0$ such that:
\begin{equation}\label{eq:transfer}
|\mathcal{T}_j| \leq C_\delta \sum_{k: |k-j| \leq 3} 2^{j} \mathcal{E}_k^{1/2} \mathcal{E}_j^{1/2} \left(\sum_{m \leq j+3} 2^{m} \mathcal{E}_m^{1/2}\right) + \delta \cdot 2^{2j(1+\alpha)} \mathcal{E}_j
\end{equation}
\end{theorem}

\begin{proof}
Apply the paraproduct decomposition to $(\bfomega \cdot \nabla)\bfu$ and use Theorem~\ref{thm:trilinear} on each component. The high-high term satisfies:
\[
C 2^{5j/2}\mathcal{E}_j^{3/2} \leq \delta \cdot 2^{2j(1+\alpha)}\mathcal{E}_j + C_\delta 2^{j(5-4\alpha)/(2\alpha-1)}\mathcal{E}_j^{(4\alpha+1)/(2(2\alpha-1))}
\]
For $\alpha \geq 5/4$, the second term is controlled.
\end{proof}

%======================================================================
\section{Proof of Global Regularity}
%======================================================================

\subsection{Weighted Energy Functional}

\begin{definition}
For $\sigma > 0$:
\[
\mathcal{E}^\sigma(t) = \sum_{j \geq -1} 2^{2j\sigma} \mathcal{E}_j(t) = \|\bfomega(t)\|_{\dot{B}^\sigma_{2,2}}^2
\]
\end{definition}

\begin{lemma}[Weighted Energy Evolution]\label{lem:weighted}
For $0 < \sigma < 1 + \alpha$:
\begin{equation}\label{eq:weighted}
\frac{d}{dt}\mathcal{E}^\sigma + c\varepsilon \|\bfomega\|_{\dot{B}^{\sigma+1+\alpha}_{2,2}}^2 \leq C(\sigma, \alpha) \mathcal{E}^\sigma \cdot G(t)
\end{equation}
where $G(t) = \|\bfomega(t)\|_{\dot{B}^{1}_{2,1}}$ is integrable in time.
\end{lemma}

\begin{proof}
From~\eqref{eq:dyadic}:
\[
\frac{d}{dt}\mathcal{E}^\sigma \leq -2c_\varepsilon \sum_j 2^{2j(\sigma+1+\alpha)}\mathcal{E}_j + 2\sum_j 2^{2j\sigma}|\mathcal{T}_j|
\]
Apply the transfer bound~\eqref{eq:transfer} with $\delta = c_\varepsilon/2$. The absorption gives:
\begin{align}
\sum_j 2^{2j\sigma}|\mathcal{T}_j| &\leq C \sum_j 2^{j(2\sigma+1)} \mathcal{E}_j^{1/2} \left(\sum_{m\leq j} 2^m\mathcal{E}_m^{1/2}\right) + \frac{c_\varepsilon}{2} \sum_j 2^{2j(\sigma+1+\alpha)}\mathcal{E}_j
\end{align}
By Cauchy--Schwarz on the first term:
\[
\sum_j 2^{j(2\sigma+1)} \mathcal{E}_j^{1/2} \left(\sum_{m\leq j} 2^m\mathcal{E}_m^{1/2}\right) \leq \mathcal{E}^\sigma \cdot \|\bfomega\|_{\dot{B}^1_{2,1}}
\]
\end{proof}

\subsection{A Priori Bounds and Continuation}

\begin{proposition}[A Priori Bound]\label{prop:apriori}
For any $T > 0$, there exists $C = C(\|\bfu_0\|_{H^s}, \nu, \varepsilon, \alpha, T) < \infty$ such that:
\[
\sup_{t \in [0,T]} \|\bfomega(t)\|_{\dot{B}^{s-1}_{2,2}} \leq C
\]
\end{proposition}

\begin{proof}
From Lemma~\ref{lem:weighted} with $\sigma = s-1$, Gronwall gives:
\[
\mathcal{E}^{s-1}(t) \leq \mathcal{E}^{s-1}(0) \exp\left(C\int_0^t G(\tau)d\tau\right)
\]
The energy inequality yields $\int_0^T \|\bfomega\|_{\dot{H}^{1+\alpha}}^2 dt \leq C(\|\bfu_0\|_{L^2}, \nu, \varepsilon)$.

By interpolation:
\[
\|\bfomega\|_{H^{3/2+\delta}} \leq C \|\bfomega\|_{L^2}^{\theta} \|\bfomega\|_{\dot{H}^{1+\alpha}}^{1-\theta}
\]
where $\theta = 1 - \frac{3/2+\delta}{1+\alpha} > 0$ for $\alpha > 1/2$.

Thus $G(t) = \|\bfomega\|_{\dot{B}^1_{2,1}} \leq C\|\bfomega\|_{H^{3/2+\delta}}$ is integrable:
\[
\int_0^T G(t)dt \leq C \|\bfomega\|_{L^\infty_t L^2}^\theta \left(\int_0^T \|\bfomega\|_{\dot{H}^{1+\alpha}}^2 dt\right)^{(1-\theta)/2} T^{(1+\theta)/2} < \infty
\]
\end{proof}

\begin{theorem}[Continuation Criterion]\label{thm:continuation}
If $\bfu \in C([0,T^*); H^s)$ is maximal and $T^* < \infty$, then:
\[
\int_0^{T^*} \|\bfomega(t)\|_{\dot{B}^1_{2,1}} dt = +\infty
\]
\end{theorem}

\begin{proof}
If the integral were finite, Proposition~\ref{prop:apriori} would give uniform $H^s$ bounds, allowing continuation---contradiction.
\end{proof}

\begin{proof}[Proof of Theorem~\ref{thm:main}]
Suppose $T^* < \infty$. By Theorem~\ref{thm:continuation}, $\int_0^{T^*} G(t)dt = +\infty$. But Proposition~\ref{prop:apriori} shows this integral is finite for any finite $T$---contradiction. Thus $T^* = +\infty$.
\end{proof}

%======================================================================
\section{Discussion}
%======================================================================

\subsection{Comparison with Classical Methods}

Standard energy methods for~\eqref{eq:hyper_ns} yield the enstrophy estimate:
\[
\frac{d}{dt}\|\bfomega\|_{L^2}^2 + \varepsilon\|\bfomega\|_{\dot{H}^{1+\alpha}}^2 \lesssim \|\bfomega\|_{L^2}^{3/2}\|\nabla\bfomega\|_{L^2}^{3/2}
\]
This is supercritical: the right-hand side dominates for large $\|\bfomega\|_{L^2}$.

Our frequency-localized approach avoids this by:
\begin{enumerate}
\item Working shell-by-shell, where each $\mathcal{E}_j$ satisfies a \emph{subcritical} ODE
\item Using the summation structure to control inter-shell transfer
\item Exploiting the weighted Besov norm to close globally
\end{enumerate}

\subsection{Physical Interpretation}

The hyperviscous term $-\varepsilon(-\Delta)^{1+\alpha}\bfu$ preferentially damps high frequencies. In our analysis:
\begin{itemize}
\item The dissipation $\sim 2^{2j(1+\alpha)}\mathcal{E}_j$ grows faster with $j$ than the transfer $\sim 2^j \mathcal{E}_j$
\item For $\alpha \geq 5/4$, dissipation dominates transfer at all scales
\item This prevents energy accumulation that could lead to blowup
\end{itemize}

\subsection{Open Problems}

\begin{enumerate}
\item \textbf{Lowering $\alpha$}: Can our method reach $\alpha < 5/4$? The bottleneck is the high-high interaction in Theorem~\ref{thm:transfer}.
\item \textbf{Physical range}: The Burnett equations correspond to $\alpha = 1$. Closing the gap $1 < \alpha < 5/4$ would have direct physical significance.
\item \textbf{Classical NS}: The limit $\varepsilon \to 0$ does not yield uniform bounds. New ideas are needed for $\varepsilon = 0$.
\end{enumerate}

%======================================================================
% References
%======================================================================

\begin{thebibliography}{99}

\bibitem{Lions1969} J.-L. Lions, \emph{Quelques m\'ethodes de r\'esolution des probl\`emes aux limites non lin\'eaires}, Dunod, Paris, 1969.

\bibitem{KatzPavlovic2002} N.H. Katz, N. Pavlovi\'c, ``A cheap Caffarelli--Kohn--Nirenberg inequality for the Navier--Stokes equation with hyper-dissipation,'' \emph{Geom.\ Funct.\ Anal.}, 12(2), 355--379, 2002.

\bibitem{Tao2009} T. Tao, ``Global regularity for a logarithmically supercritical hyperdissipative Navier--Stokes equation,'' \emph{Anal.\ PDE}, 2(3), 361--366, 2009.

\bibitem{BCD2011} H. Bahouri, J.-Y. Chemin, R. Danchin, \emph{Fourier Analysis and Nonlinear Partial Differential Equations}, Springer, 2011.

\bibitem{Leray1934} J. Leray, ``Sur le mouvement d'un liquide visqueux emplissant l'espace,'' \emph{Acta Math.}, 63, 193--248, 1934.

\bibitem{CKN1982} L. Caffarelli, R. Kohn, L. Nirenberg, ``Partial regularity of suitable weak solutions of the Navier--Stokes equations,'' \emph{Comm.\ Pure Appl.\ Math.}, 35(6), 771--831, 1982.

\bibitem{ESS2003} L. Escauriaza, G.A. Seregin, V. \v{S}ver\'ak, ``$L_{3,\infty}$-solutions of Navier--Stokes equations and backward uniqueness,'' \emph{Russian Math.\ Surveys}, 58(2), 211--250, 2003.

\bibitem{FMT2001} C. Foias, O. Manley, R. Temam, \emph{Navier--Stokes Equations and Turbulence}, Cambridge University Press, 2001.

\bibitem{MB2002} A.J. Majda, A.L. Bertozzi, \emph{Vorticity and Incompressible Flow}, Cambridge University Press, 2002.

\end{thebibliography}

\end{document}
