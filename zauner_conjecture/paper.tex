\documentclass{article}
\usepackage[utf8]{inputenc}
\usepackage{amsmath, amssymb}
\usepackage{geometry}

\title{Zauner’s Conjecture (Existence of SIC-POVMs)}
\author{Research Overview}
\date{\today}

\begin{document}

\maketitle

\begin{abstract}
This paper explores Zauner's Conjecture regarding the existence of SIC-POVMs in all dimensions, a fundamental problem for quantum tomography and QBism.
\end{abstract}

\section{Domain}
Quantum Information \& Foundations

\section{The Problem}
This conjecture asks if there exists a set of $d^2$ quantum states in any $d$-dimensional Hilbert space that are "equiangular"—meaning the overlap (inner product) between any two distinct states is always the same ($1/(d+1)$). These sets are called \textbf{SIC-POVMs} (Symmetric Informationally Complete Positive Operator-Valued Measures).

\section{Implications}
SIC-POVMs are "standard" reference frames for quantum mechanics, crucial for quantum tomography (reconstructing a state) and foundational interpretations (like QBism).

\section{Status}
\textbf{Open.} They have been found numerically for all dimensions up to 151 and analytically for specific dimensions (like 2, 3, 4, ..., 24, etc.), but a general proof for \emph{all} dimensions $d$ is missing.

\end{document}
