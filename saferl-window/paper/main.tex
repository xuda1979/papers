\documentclass[conference]{IEEEtran}
\IEEEoverridecommandlockouts

\usepackage{amsmath,amssymb,amsfonts}
\usepackage{graphicx}
\usepackage{booktabs}
\usepackage{hyperref}
\usepackage{algorithm}
\usepackage{algpseudocode}
\usepackage{xcolor}
\usepackage{cite}

\title{SafeRL-Window: An Architectural Proposal and Simulation Skeleton for Safe RL Control of Quantum Decoders}

\author{
\IEEEauthorblockN{Agentic Research Group}
\IEEEauthorblockA{
Email: agent@research.com}
}

\begin{document}
\maketitle

\begin{abstract}
The dynamic control of quantum error correction decoders in response to time-varying noise is a significant challenge. We propose \emph{SafeRL-Window}, an architecture for a safe reinforcement learning (Safe RL) controller designed to adaptively manage sliding-window decoding of quantum low-density parity-check (QLDPC) codes. This paper outlines the architecture, which aims to schedule window size, shift, and decoder parameters to optimize performance under hard constraints on latency and compute. We formalize the control problem as a constrained Markov decision process (CMDP) solvable with algorithms like Lagrangian-PPO. To accelerate research in this domain, we are releasing a complete open-source simulation skeleton in Python, implementing the proposed architecture with placeholder components for the channel, decoder, and state-space. This provides a framework for researchers to integrate their own models and test novel control policies.
\end{abstract}

\begin{IEEEkeywords}
QLDPC, CSS codes, sliding-window decoding, safe reinforcement learning, quantum information, simulation framework.
\end{IEEEkeywords}

\section{Introduction}
Quantum communication links often face time-varying, bursty noise. Spatially-coupled QLDPC (SC-QLDPC) codes with sliding-window decoding offer a promising approach to achieve low-latency error correction. However, practical systems often rely on static or heuristic settings for window size ($W$), shift ($F$), iteration count, and other decoder parameters. Such fixed controls can be inefficient, wasting resources in low-noise regimes or compromising reliability when conditions worsen.

We introduce \emph{SafeRL-Window}, a proposed architecture for a lightweight Safe RL controller that adapts these parameters online. The controller is designed to use only syndrome-derived features and runtime statistics, while guaranteeing hard budgets on latency and compute. The safety aspect is critical, ensuring that any learned policy performs no worse than a reliable baseline.

\textbf{Contributions:}
This work makes the following contributions:
\begin{enumerate}
    \item We propose a unified Safe RL \emph{control layer architecture} for sliding-window decoding that can jointly schedule parameters like window size, shift, and decoder iterations.
    \item We formalize the problem as a Constrained Markov Decision Process (CMDP), providing a principled foundation for applying advanced RL techniques.
    \item We outline a safety mechanism based on action masking and a fallback to a pre-defined safe baseline policy.
    \item We release a complete, open-source Python simulation skeleton that implements this architecture, enabling researchers to easily experiment with and validate their own control policies and channel models.
\end{enumerate}

\section{Background}
\textbf{Stabilizer/CSS codes and QLDPC.} Stabilizer codes specify commuting generators; CSS uses two classical codes to correct $X$/$Z$ errors separately with $H_X H_Z^\top \equiv 0$. QLDPC entails sparse $H_X,H_Z$ for scalable iterative decoding.

\textbf{Spatial coupling and sliding windows.} Spatial coupling forms a banded structure that improves thresholds at finite length. Sliding-window decoding processes a local band before shifting, reducing latency and memory.

\textbf{AI for communications/QEC.} While many works focus on modifying the decoding algorithm itself (e.g., Neural-BP/GNN decoders), we target the \emph{control layer}: scheduling the decoder's resources over time with safety guarantees.

\section{Problem Formulation}
We consider an SC-QLDPC code with a sliding window decoder. At each step $t$, the controller observes a state $s_t$ and selects an action $a_t$.

\textbf{State Space $s_t$:} A complete implementation would featurize the state from (i) syndrome statistics and residuals from the decoder; (ii) decoder runtime signals (convergence, iterations); (iii) a learned noise fingerprint $\hat{\theta}_t$ (loss, dephasing, depolarization, burstiness). Our skeleton provides a minimal placeholder for this.

\textbf{Action Space $a_t$:} The action is a tuple of key decoder parameters, e.g., $a_t=(W,F,\text{iters},\text{damp},\text{schedule})$, chosen from discrete sets.

\textbf{Constraints:} The actions are subject to \emph{hard constraints} on average end-to-end latency $\bar{\tau}\le\tau_{\max}$ and compute $\bar{C}\le C_{\max}$.

\textbf{Objective:} The goal is to learn a policy $\pi(a_t|s_t)$ that maximizes a reward function, such as:
\begin{align}
r_t = \text{throughput}(a_t,s_t) - \lambda_1 \mathbf{1}[\text{logical failure}] - \lambda_2 \text{latency}(a_t).
\end{align}
This formulation as a CMDP can be solved by algorithms like Lagrangian-PPO.

\section{The SafeRL-Window Architecture}
\subsection{Core Components}
The architecture consists of a lightweight actor-critic policy that selects $a_t$. This policy treats the decoder backend (e.g., BP or GNN-BP) as a black box, interacting only with its exposed tunable parameters. A safety layer is crucial: it first clips actions to a pre-defined feasible set, and a fallback mechanism can switch to a safe baseline policy if the learned policy is deemed too uncertain or is predicted to violate constraints.

\subsection{Training Methodology}
The proposed training method is entropy-regularized Lagrangian-PPO. To promote robustness, the training environment should employ domain randomization across a wide range of channel parameters and burst noise processes. Uncertainty estimation (e.g., via Monte Carlo dropout in the policy network) can be used to trigger the fallback to the conservative baseline policy.

\begin{algorithm}[t]
\caption{SafeRL-Window (Conceptual Training Loop)}
\begin{algorithmic}[1]
\State Initialize policy $\pi_\theta$, value function $V_\psi$, Lagrange multipliers $\eta$
\State Initialize a safe, fixed-parameter baseline policy $a^{\text{base}}$
\For{each episode}
  \State Sample a code instance and a time-varying channel model
  \For{$t=1$ to $L$}
    \State $s_t \gets$ get\_features(syndrome, decoder\_stats, ...)
    \State $\tilde a_t \gets \pi_\theta(s_t)$
    \State $a_t \gets$ mask\_and\_clip\_to\_feasible\_set$(\tilde a_t)$
    \If{is\_unsafe$(\pi_\theta, s_t)$}
      \State $a_t \gets a^{\text{base}}$ \Comment{Fallback to baseline}
    \EndIf
    \State Run decoder with parameters $a_t$
    \State Collect reward $r_t$ and constraint costs $c_t$
  \EndFor
  \State Update $(\theta,\psi)$ via PPO on the Lagrangian objective
  \State Update $\eta$ based on constraint violations
\EndFor
\end{algorithmic}
\end{algorithm}

\section{Simulation Skeleton and Usage}
We provide an open-source Python skeleton to accompany this proposal. It implements the overall training loop and environment structure but uses placeholders for key components.

\textbf{Code Structure:} The `code/` directory contains the environment (`qldpc_env.py`), the PPO training harness (`safe_rl.py`), and configuration files.

\textbf{Placeholders:} Users should note the following are toy models:
\begin{itemize}
    \item \textbf{Codes:} The environment uses a placeholder for QLDPC matrices. Users should provide their own.
    \item \textbf{Channel:} The channel model generates random noise events but does not simulate the full quantum state evolution.
    \item \textbf{Decoder:} The `css_bp_round` function is a placeholder that does not perform actual belief propagation.
    \item \textbf{State/Reward:} The observation space and reward function are minimalistic and should be replaced with more meaningful, problem-specific versions.
\end{itemize}

\textbf{Running the Skeleton:} The provided code can be run out-of-the-box to demonstrate the training loop:
\begin{verbatim}
cd code
pip install -r requirements.txt
python -m experiments.run_saferl --config configs/saferl.yaml
\end{verbatim}
This will start a training run using the PPO algorithm from `stable-baselines3` with the toy environment. It serves as a starting point for implementing and testing real control policies.

\section{Discussion and Limitations}
The proposed architecture is a control-layer solution, complementary to algorithm-layer improvements like GNN-based decoders. The effectiveness of a fully implemented system would depend on the quality of the feature engineering for the state space and the accuracy of the noise models. This skeleton aims to lower the barrier to entry for researchers exploring these questions.

\section{Reproducibility}
We release the simulation skeleton described here to facilitate research in this area. The code is available in the `code/` directory of this repository.

\bibliographystyle{IEEEtran}
\bibliography{refs}

\end{document}
