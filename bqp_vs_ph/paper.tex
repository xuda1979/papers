\documentclass{article}
\usepackage[utf8]{inputenc}
\usepackage{amsmath, amssymb}
\usepackage{geometry}

\title{BQP vs. PH (Polynomial Hierarchy)}
\author{Research Overview}
\date{\today}

\begin{document}

\maketitle

\begin{abstract}
This paper provides an overview of the relationship between Bounded-Error Quantum Polynomial Time (BQP) and the Polynomial Hierarchy (PH), a central open problem in Quantum Complexity Theory.
\end{abstract}

\section{Domain}
Quantum Complexity Theory

\section{The Problem}
We know BQP (Bounded-Error Quantum Polynomial Time) is the class of problems a quantum computer can solve efficiently. A major open question is whether BQP is contained within the classical Polynomial Hierarchy (PH).

\section{Implications}
If BQP is \emph{not} in PH, it implies quantum computers can solve problems that are structurally outside the "hierarchy" of classical complexity classes generated by P and NP.

\section{Status}
\textbf{Open.} There is an oracle separation (meaning in a hypothetical "black box" world, they are different), but proving this in the real world is as hard as the P vs. NP problem.

\end{document}
