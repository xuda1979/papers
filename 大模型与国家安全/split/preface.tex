% ==================== 序言 ====================
\chapter*{序\quad 言}
\addcontentsline{toc}{chapter}{序言}
\markboth{序言}{序言}

% [此处可邀请院士或行业权威撰写序言]

我们正站在一个历史性的技术转折点上。

大型语言模型(Large Language Models,LLMs)的横空出世,标志着人工智能从"能做什么"进入到"能想什么"的新阶段。这不仅仅是一次技术迭代,而是一场可能重塑人类文明进程的智能革命。从ChatGPT在2022年底引爆全球关注,到2025年GPT-5、Claude Opus 4.5、Gemini 3等超级模型的相继问世,大模型技术正以前所未有的速度演进,其影响已经渗透到经济、科技、教育、医疗、文化、国防等几乎所有领域。

在这样的背景下,许达博士撰写了这部《大模型强国》。作为中国移动研究院的资深研究员,许达博士长期从事AI for Science和人工智能大模型的研究工作,对这一领域有着深刻的洞察。本书不是一部单纯的技术著作,而是一部具有强烈问题意识和政策关怀的战略研究。作者从国家发展的高度,系统论证了大模型技术对国家各个领域的深远影响,深入分析了技术竞争态势和潜在风险,并提出了富有建设性的政策建议。

本书最鲜明的特点是其强烈的战略意识和紧迫感。作者明确提出,大模型是继蒸汽机、电力、互联网之后的第四代"通用目的技术",其重要性堪比20世纪的"两弹一星"工程,应当给予国家战略层面的最高度重视。这一判断或许会引发争议,但正如作者所言——在技术革命的关键节点,战略判断和决策果断至关重要。历史已经多次证明,错失技术革命的窗口期,代价将是数十年的追赶。

我欣慰地看到,新一代科研工作者不仅关注技术本身,更能将技术置于国家发展和国际竞争的宏观视野中加以审视。我相信,本书的出版将为决策者、研究者和广大读者提供有价值的参考,也期待能够引发更多关于我国AI发展战略的深入讨论。

\vspace{2em}
\begin{flushright}
{[序言作者姓名]}\\
{[职务/头衔]}\\
2025年12月于北京
\end{flushright}
