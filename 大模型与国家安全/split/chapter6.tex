% ==================== 第六章 ====================
\chapter{制度护航:人才、治理与国际合作}

\begin{tcolorbox}[colback=purple!5!white,colframe=purple!75!black,title=本章要点]
技术能力建设需要制度保障。本章讨论支撑大模型战略的三大制度支柱:人才培养体系、治理框架和国际合作策略,并提出短中长期行动路线图。

\textbf{核心理念}:人才是根本,制度是保障,合作是补充。
\end{tcolorbox}

\section{人才培养体系}

人才是AI发展的核心要素,尤其是AI安全领域的专业人才。

\subsection{AI安全人才现状与需求}

AI安全领域的人才短缺是全球性问题,国内尤为突出。

\textbf{对齐研究人才}:从事对齐研究的人全球也没多少。这是个新兴领域,很多概念的定义都还在演化中。国际上做这块的顶尖研究者可能就几百人,国内能算得上入门的更是凤毛麟角。

\textbf{红队测试人才}:能够系统性地对大模型进行安全评估和漏洞挖掘的,国内估计不超过百人量级。

\textbf{复合型人才}:既懂AI底层技术,又理解国家安全的实际需求,还能把两者结合起来做研究或做工程——这种人培养周期长,而且现有的学科设置和评价体系都不太支持这种"跨界"发展。

\textbf{治理研究人才}:能够参与国际AI治理讨论、起草政策建议、与国际同行对话的专业人才,国内屈指可数。

\subsection{培养重点方向}

针对国家安全需求,重点培养四类人才:
\begin{itemize}
    \item \textbf{AI安全对齐}:确保AI系统行为符合人类意图
    \item \textbf{对抗机器学习}:AI系统的攻防技术
    \item \textbf{可解释AI}:决策过程的可解释性和可审计性
    \item \textbf{AI伦理与治理}:技术的社会影响和治理框架
\end{itemize}

\subsection{分阶段人才培养目标}

\textbf{短期(2025-2026年)}:培养AI安全方向硕士100-150名、博士50-80名。在"计算机科学与技术"等一级学科下设立"AI安全对齐"二级学科方向;资助5-10所高校设立"AI安全研究中心"。

\textbf{中期(2027-2029年)}:培养AI安全方向硕士500-800名、博士200-300名,建立3-5个国家级AI安全研究基地。在工学学科门类增设"人工智能安全"一级学科;支持高校与企业开展联合培养模式。

\textbf{长期(2030-2035年)}:建成较为完整的AI安全人才体系,积极参与国际AI安全研究社区。推动AI安全成为计算机学科的主流研究方向之一。

\subsection{人才培养模式创新}

\textbf{产教融合}:与头部AI企业共建实训基地,让学生接触真实问题。

\textbf{国际联培}:与海外顶尖实验室建立联合培养机制。

\textbf{跨学科交叉}:鼓励计算机、法学、哲学、安全等领域的交叉培养。

\textbf{实战导向}:将CTF竞赛、红队演练等纳入培养环节。

\section{评价机制改革}

\subsection{改革评价体系}

应建立适应AI安全特点的人才评价机制,改变唯论文、唯性能指标的评价导向:

\begin{itemize}
    \item 将安全对齐算法贡献、红队测试漏洞挖掘成果、安全标准制定等纳入评价指标
    \item 认可"不仅跑得快,还要跑得稳"的科研价值
    \item 支持国内研究者参与国际AI安全学术社区
\end{itemize}

\subsection{高层次人才政策}

设立AI安全领域专项人才计划,吸引海外AI安全研究人员回国或来华工作。

\textbf{平衡视角}:人才政策应避免两个极端——既不能因过度担忧而设置不合理限制导致人才流失,也不能放任关键人才和技术的无序流动。关键是建立基于信任的管理机制,让人才能够安心工作、自由探索。

\subsection{激励机制设计}

\textbf{薪酬竞争力}:确保关键岗位薪酬具有国际竞争力。

\textbf{科研自主权}:给予顶尖人才足够的科研自由度。

\textbf{职业发展}:建立清晰的职业晋升通道。

\textbf{配套支持}:解决住房、子女教育等后顾之忧。

\section{治理框架设计}

\subsection{建立"反马赛克"数据分类分级制度}

在发布政府采购、科研立项、人事任免等公开信息前,建议进行"反AI推演"。使用大模型模拟分析视角,检测是否可以通过聚合多条公开信息推导出敏感信息。

具体措施包括:
\begin{itemize}
    \item \textbf{引入"红队测试"机制}:在关键信息发布前,组织专业团队利用主流商用大模型进行模拟挖掘测试
    \item \textbf{建立AI模拟审查流程}:开发专门的"马赛克效应"检测工具,自动化评估信息聚合风险
    \item \textbf{培训相关人员}:提升信息发布人员对AI情报挖掘能力的认知
    \item \textbf{建立跨部门协调机制}:统筹不同部门的信息发布,防止跨部门信息拼图
\end{itemize}

\subsection{动态密级管理}

传统的静态密级分类难以应对信息聚合带来的风险。应建立动态的密级管理机制:
\begin{itemize}
    \item 根据信息的累积效应动态调整保护级别
    \item 对相关联信息实施关联保护
    \item 定期评估已公开信息的安全影响
    \item 建立信息解密和公开的审查程序
\end{itemize}

\subsection{科技文献与学术发表管理}

对于人工智能、量子信息、集成电路、航空航天等关键领域的学术论文,在投稿国际期刊前,建议经过安全审查,评估发表后的潜在风险。

\textbf{审查边界与平衡机制}:为避免阻碍正常学术交流,审查应严格限定在"特定敏感领域",并设立明确的豁免清单(如纯理论基础研究)。同时,建立申诉与复核机制,并设定严格的审查时限(如15个工作日内反馈)。

\subsection{国际经验借鉴}

\textbf{美国NSDD-189框架}:1985年发布的《国家安全决策指令第189号》确立了"基础研究原则上不受限制"的立场,明确区分基础研究与涉及国家安全的应用研究。

\textbf{英国"可信研究"框架}:采用风险评估方法对国际合作进行分类管理,而非一刀切限制。该框架强调基于具体风险而非合作方国籍进行评估。

\textbf{对我国的启示}:有效的管控机制应当:
\begin{enumerate}
    \item 区分基础研究与敏感应用研究
    \item 采用风险评估方法进行分类管理
    \item 保持科研机构的自主性
    \item 在激励与限制之间寻求平衡
\end{enumerate}

\section{法规体系建设}

\subsection{现有法规基础}

我国已出台一系列AI相关法规:
\begin{itemize}
    \item 《生成式人工智能服务管理暂行办法》(2023年)
    \item 《互联网信息服务深度合成管理规定》(2022年)
    \item 《互联网信息服务算法推荐管理规定》(2022年)
    \item 《新一代人工智能伦理规范》(2021年)
\end{itemize}

\subsection{完善方向}

\textbf{基础法律层面}:推动《人工智能法》立法,建立AI治理的基本法律框架。

\textbf{监管协调层面}:明确各部门监管职责,避免多头管理或监管真空。

\textbf{标准规范层面}:加快AI安全标准制定,推动标准国际化。

\textbf{伦理准则层面}:完善AI伦理指引,建立伦理审查机制。

\section{国际合作策略}

AI是全球性技术,单靠一国的治理不可能奏效。中国应积极参与国际AI治理,在保障自身安全的同时推动建立公平合理的国际规则。

\begin{tcolorbox}[colback=blue!5!white,colframe=blue!75!black,title=2025年国际AI治理格局(截至2025年12月21日)]
\textbf{美国}:
\begin{itemize}
    \item 2024年10月发布《国家安全备忘录》(NSM),确立AI作为国家安全优先事项
    \item 2025年持续收紧对华AI芯片出口管制
    \item 国防部、情报界大规模部署AI系统
\end{itemize}

\textbf{欧盟}:
\begin{itemize}
    \item 《人工智能法案》(AI Act)于2024年生效
    \item 建立分级风险管理框架
    \item 持续推动全球AI治理标准化
\end{itemize}

\textbf{英国}:
\begin{itemize}
    \item 2023年AI安全峰会后持续推动国际对话
    \item 建立AI安全研究所(AISI)
    \item 采取"促创新、轻监管"路线
\end{itemize}

\textbf{中国}:
\begin{itemize}
    \item 《生成式人工智能服务管理暂行办法》实施
    \item 积极参与联合国AI治理框架讨论
    \item 在国内外双轨推进AI发展与安全
\end{itemize}
\end{tcolorbox}

\subsection{参与国际AI治理机制}

\textbf{多边平台参与}:
\begin{itemize}
    \item 在联合国框架下积极参与AI相关讨论
    \item 利用G20数字经济工作组等机制推动对话与合作
    \item 深度参与ISO/IEC JTC 1/SC 42(人工智能分委会)的标准制定
    \item 支持中国学者在顶级学术会议中担任组织职务
\end{itemize}

\textbf{双边合作机制}:

\textit{中美AI对话}:基于利益分析,AI安全对齐是中美可能达成合作共识的领域。防止AI被用于生物恐怖主义、确保核指挥系统与AI隔离等底线问题对双方都至关重要。建议建立"AI安全对话工作组",定期交流对齐技术进展与风险认知。

\begin{tcolorbox}[colback=yellow!5!white,colframe=yellow!75!black,title=中美AI竞争的结构性特征(2025年12月分析)]
\textbf{差距与追赶}:
\begin{itemize}
    \item 前沿模型能力:美国领先(GPT-5.2、Gemini 3、Claude Opus 4.5)
    \item 开源模型:中国快速追赶(DeepSeek-V3.2与GPT-4o相当)
    \item 算力:受出口管制影响,存在差距但通过MoE架构优化弥补
    \item 应用落地:各有优势,中国在部分垂直场景领先
\end{itemize}

\textbf{合作可能空间}:
\begin{itemize}
    \item AI安全对齐研究(双方都有强烈动机)
    \item 防止AI用于大规模杀伤性武器
    \item 核指挥控制系统的AI隔离
    \item 学术交流和人才流动
\end{itemize}

\textbf{竞争焦点}:
\begin{itemize}
    \item 半导体供应链控制
    \item 军事和情报AI应用
    \item 国际标准制定权
    \item AI人才争夺
\end{itemize}
\end{tcolorbox}

\textit{中欧合作}:与欧盟在AI伦理、标准协调、人才交流等领域开展合作。

\textit{发展中国家合作}:在"一带一路"框架下推广AI应用合作,帮助发展中国家提升AI能力。

\subsection{推动公平的国际AI秩序}

\textbf{技术中立原则}:把AI技术政治化、武器化,滥用出口管制来限制正常技术交流,对全球科技进步没有好处。

\textbf{发展权保障}:AI不应该成为富国俱乐部的专利,发展中国家有权分享技术进步的红利。

\textbf{多边主义}:AI治理涉及全人类利益,应该在联合国框架下讨论,不能由少数国家垄断规则制定权。

\textbf{包容性治理}:政府、企业、学术界、民间社会——各方利益相关者的声音都应该得到反映。

\subsection{应对技术脱钩风险}

技术脱钩对中国AI发展的影响是多层面的,需要冷静评估和应急准备。

\textbf{应急准备}:
\begin{itemize}
    \item \textbf{供应链备份}:关键芯片和零部件要有战略储备
    \item \textbf{技术路线多元化}:GPU不是唯一的AI计算路径
    \item \textbf{应急预案}:提前做好不同脱钩情景下的沙盘推演
    \item \textbf{自主生态建设}:越早动手越好
\end{itemize}

\textbf{保持开放}:在防风险的同时不能把自己封起来。欢迎外国企业和人才参与中国AI发展;学术交流和科技合作该怎么做还怎么做,避免"自我脱钩"。

\subsection{国际话语权建设}

\textbf{学术影响力}:支持中国学者在顶级会议和期刊发表高水平成果。

\textbf{标准制定}:积极参与国际标准制定,争取标准主导权。

\textbf{智库交流}:加强与国际知名AI智库的对话交流。

\textbf{媒体传播}:讲好中国AI发展故事,展示负责任的AI大国形象。

\section{行动路线图}

\subsection{短期(1-2年)}

\begin{itemize}
    \item 完成政府与关键行业的AI安全审计试点
    \item 建立红队框架与工具链,覆盖不少于30家重点单位
    \item 建立国家级大模型安全漏洞库(CNVD-LLM)
    \item 制定并发布《生成式人工智能服务安全基线规范》
    \item 启动AI安全人才专项培养计划
\end{itemize}

\subsection{中期(2-3年)}

\begin{itemize}
    \item 形成国家级测评年报与开源模型安全名录
    \item 关键系统部署安全网关覆盖率$\geq$80\%
    \item 建立3-5个国家级AI安全研究基地
    \item 培养AI安全方向硕士500-800名、博士200-300名
    \item 推动《人工智能法》立法进程
\end{itemize}

\subsection{长期(3-5年)}

\begin{itemize}
    \item 完善算力供给多元化与国产生态兼容性
    \item 建立持续对抗机制与人才梯队
    \item 建成较为完整的AI技术体系
    \item 在国际AI治理中发挥积极作用
    \item 形成可复制推广的AI治理"中国方案"
\end{itemize}

\subsection{政策建议优先级}

\begin{table}[H]
\centering
\caption{核心政策建议优先级与可行性分析}
\small
\begin{tabular}{p{2.5cm}p{1cm}p{1cm}p{1.2cm}p{3cm}p{3cm}}
\toprule
\textbf{建议事项} & \textbf{紧迫性} & \textbf{可行性} & \textbf{资源需求} & \textbf{主要障碍} & \textbf{政策着力点} \\
\midrule
国产芯片生态建设 & 高 & 中 & 极高 & 技术积累、人才缺口 & 产业政策、研发投入 \\
算力基础设施 & 高 & 高 & 高 & 资金、能源 & 基础设施规划 \\
人才培养体系 & 高 & 高 & 中 & 培养周期 & 教育改革、产教融合 \\
信息安全管理优化 & 中高 & 高 & 低 & 部门协调 & 制度建设 \\
AI安全对齐研究 & 中 & 中 & 中 & 研究基础薄弱 & 基础研究 \\
国际治理参与 & 中 & 中 & 低 & 国际环境 & 多边外交 \\
\bottomrule
\end{tabular}
\end{table}

\subsection{跨部门协调机制}

鉴于AI发展涉及多个主管部门,建议建立以下协调机制:

\textbf{顶层协调机制}:在国家层面设立AI发展与安全协调机制,统筹科技部、工信部、网信办、发改委等部门的相关职能。

\textbf{联席会议制度}:针对重大事项建立跨部门联席会议制度,定期研判AI发展态势和安全风险。

\textbf{责任边界界定}:明确各部门在AI芯片、算法、数据、应用、安全等不同环节的主管责任。

\textbf{地方对接机制}:建立中央与地方在AI政策执行层面的协调机制,确保政策落地的一致性。

\section{组织保障}

\subsection{领导体制}

建议建立"国家大模型发展领导小组"或同等级别协调机制,统筹各方资源,协调各方利益,推动战略实施。

\subsection{执行机制}

\textbf{专项办公室}:设立日常运作机构,负责政策协调和任务分解。

\textbf{专家委员会}:组建跨领域专家团队,提供技术和政策咨询。

\textbf{评估监督}:建立目标考核和动态评估机制。

\subsection{资源保障}

\textbf{财政投入}:设立专项基金,保障重点任务的资金需求。

\textbf{金融支持}:引导社会资本投向AI关键领域。

\textbf{税收优惠}:对AI研发投入给予税收激励。

\textbf{政府采购}:发挥政府采购对国产AI产品的牵引作用。
