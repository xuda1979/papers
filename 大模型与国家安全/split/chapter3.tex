% ==================== 第三章 ====================
\chapter{全球竞争:主要国家AI战略比较}

\begin{tcolorbox}[colback=blue!5!white,colframe=blue!75!black,title=本章要点]
全球AI竞争已进入白热化阶段。美国凭借技术生态、人才储备和资本优势保持领先,并正在将AI与国家安全、能源战略、科学研究深度绑定;欧盟以监管框架和价值导向寻求差异化定位,但对AGI地缘政治影响的担忧日益加深;中国在应用落地和工程优化方面展现出强劲竞争力,但在芯片、基础软件生态等"硬科技"领域仍需追赶。

\textbf{核心判断}:当前正处于技术范式确立的关键窗口期,先发优势将形成"强者恒强"的马太效应。情报界的AI整合正在静悄悄地改变国家竞争的底层逻辑。
\end{tcolorbox}

\section{美国:技术领先与国家战略深度整合}

美国在AI领域保持全球领先。2025年12月的最新动态显示,美国正在系统性地将AI与国家战略深度绑定。

\subsection{2025年12月的关键进展}

\textbf{模型能力的持续突破}:
\begin{itemize}
    \item OpenAI于12月11日发布GPT-5.2,12月18日发布GPT-5.2-Codex(专为编程优化)
    \item Google DeepMind推出Gemini 3系列,12月发布Gemini 3 Flash(为速度优化的前沿模型)
    \item Anthropic发布Claude Opus 4.5,被称为"世界上最好的编程、代理和企业工作流模型"
\end{itemize}

\textbf{与国家战略的深度绑定}:
\begin{itemize}
    \item \textbf{能源战略}:OpenAI与美国能源部深化合作(12月18日公布),AI与国家能源基础设施整合加深
    \item \textbf{科研加速}:Google DeepMind支持能源部Genesis计划——"加速创新和科学发现的国家级使命"
    \item \textbf{安全研究}:OpenAI发布思维链监控研究(12月18日),AI安全对齐研究持续深化
    \item \textbf{企业AI}:OpenAI发布"2025企业AI状态报告"(12月8日),系统分析AI在企业中的应用
\end{itemize}

\textbf{AI安全研究的领先地位}:
\begin{itemize}
    \item Google DeepMind发布Gemma Scope 2(12月),"帮助AI安全社区深入理解复杂语言模型行为"
    \item Anthropic持续推进可解释性研究,3月发布"Tracing the thoughts of a large language model"
    \item 6月发布"Agentic Misalignment"研究,系统分析代理AI的对齐风险
\end{itemize}

\subsection{美国AI战略的完整图景}

美国的战略优势体现在多个层面的协同:

\textbf{人才虹吸效应}:全球顶尖的AI研究者中,相当比例在美国工作。高薪酬、优越的科研环境、完善的创业生态,使美国成为AI人才的首选目的地。这种人才集聚效应形成正反馈——越多人才聚集,越能吸引更多人才。

\textbf{资本充裕}:硅谷的风险投资体系为AI创业提供了充足的资金支持。从种子轮到IPO,完整的融资链条使创新想法能够快速转化为产品和服务。

\textbf{完整产业链}:从芯片设计(Nvidia、AMD)到云计算平台(AWS、Azure、GCP),从模型研发(OpenAI、Anthropic、Google)到应用落地,美国构建了从上游到下游的完整AI产业链。

\textbf{国家战略整合}:最值得关注的是,美国正在将AI与能源、科研、国防、情报等国家战略领域系统性整合。这不是单纯的产业竞争,而是国家能力的全面升级。

\subsection{美国AI战略的演进}

美国的AI战略经历了四个阶段:

\textbf{第一阶段(2016-2019):学术引领}。以Google、Facebook等企业研究院和高校为主导,开源文化盛行,学术论文大量发表。

\textbf{第二阶段(2019-2022):商业化加速}。GPT系列的成功证明了大模型的商业价值,风险投资大量涌入。

\textbf{第三阶段(2022-2024):国家安全化}。将AI竞争提升至国家安全层面,实施芯片出口管制,限制技术流动。

\textbf{第四阶段(2025-至今):战略深度整合}。AI与能源、科研、国防的深度融合,形成国家竞争力的系统性优势。

\subsection{出口管制与技术封锁}

2022年10月以来,美国商务部连续出台对华芯片出口管制措施:

\begin{itemize}
    \item 限制先进制程芯片(A100、H100等)出口
    \item 限制半导体制造设备出口
    \item 将中国AI企业列入实体清单
    \item 扩大"外国直接产品规则"适用范围
\end{itemize}

这些措施旨在"卡住中国AI发展的脖子",但效果存在争议。一方面确实给中国AI产业带来短期困难;另一方面也刺激了中国自主创新的决心和投入。

\section{欧盟:监管引领与战略焦虑}

2024年欧盟《人工智能法案》(AI Act)正式通过,成为全球首部全面规范AI的法律,采用基于风险的分级监管方法。该法案实施分阶段推进:2025年2月起,禁止社会评分、有害AI操纵等八类"不可接受风险"的AI应用;2025年8月起,通用人工智能(GPAI)模型相关规则生效;2026年8月起,高风险AI系统的完整合规要求生效。

\textbf{2025年12月的战略焦虑}:

RAND Europe于2025年12月16日发布重要报告"Europe and the geopolitics of AGI: The need for a preparedness plan"(欧洲与AGI的地缘政治:对准备计划的需求)。这份报告系统分析了通用人工智能(AGI)对欧洲的地缘政治影响,呼吁欧洲制定应对更高级AI的战略准备计划。

这表明欧洲智库已经认识到:\textbf{欧洲可能在AI竞争中被边缘化}。在模型能力上,欧洲没有能与GPT-5.2、Gemini 3、Claude Opus 4.5竞争的产品;在战略布局上,欧洲缺乏将AI与国家战略整合的系统性规划。

\textbf{人才流失}:欧洲培养的顶尖AI人才,相当比例被美国科技公司吸引。薪酬差距、科研资源差异、创业环境差异,使"欧洲培养、美国使用"成为常态。

\textbf{产业化不足}:欧洲在AI基础研究方面有优势,但将研究成果转化为商业产品的能力相对较弱。缺乏世界级的AI科技公司,是欧洲AI生态的明显短板。

\textbf{监管与创新的张力}:严格的监管可能保护消费者权益和伦理价值,但也可能增加企业合规成本,影响创新活力。

\subsection{欧盟AI治理框架的特点}

欧盟的AI监管框架具有以下特点:

\textbf{风险分级}:将AI系统分为不可接受风险、高风险、有限风险和最低风险四个等级。

\textbf{透明度要求}:对通用AI模型提出信息披露要求。

\textbf{合规认证}:高风险AI系统需要第三方合规评估。

\textbf{域外效力}:适用于所有向欧盟市场提供服务的企业。

这种"布鲁塞尔效应"可能影响全球AI治理规则的走向。

\section{其他重要参与者}

\textbf{英国}:2021年发布《国家人工智能战略》,2023年主办首届全球AI安全峰会,发起《布莱切利宣言》,试图在全球AI治理中发挥引领作用。英国在AI安全研究方面具有优势,DeepMind就是英国AI实力的代表。

\textbf{日本}:将AI视为应对人口老龄化的关键手段,强调"以人为本"和"社会5.0"愿景,在养老护理、防灾减灾、农业智能化等领域积极应用。日本的优势在于制造业基础和机器人技术积累。

\textbf{韩国}:三星、SK海力士在全球DRAM和HBM市场占主导份额,Naver开发的HyperCLOVA系列在韩语处理上具有显著优势。韩国在AI芯片供应链中扮演关键角色。

\textbf{印度}:拥有庞大工程人才储备,2023年宣布"IndiaAI"计划,重点建设国家级AI算力基础设施和支持多语言的大模型。印度的人口红利和英语优势,使其在AI服务外包和人才输出方面具有潜力。

\subsection{新兴力量的崛起}

除传统AI强国外,一些新兴力量值得关注:

\textbf{阿联酋}:凭借主权财富基金的支持,积极布局AI产业。Falcon系列开源模型展现了中东国家的AI雄心。

\textbf{以色列}:虽然体量小,但在AI安全、网络安全领域有深厚积累。

\textbf{新加坡}:以政府主导的方式推进"智慧国家"战略,在AI治理方面积极探索。

\section{主要国家AI战略对比}

\begin{table}[H]
\centering
\caption{主要国家/地区AI战略对比分析}
\small
\begin{tabular}{p{1.5cm}p{2.5cm}p{2.5cm}p{2.5cm}p{3cm}}
\toprule
\textbf{国家/地区} & \textbf{战略重点} & \textbf{核心优势} & \textbf{主要短板} & \textbf{对中国的启示} \\
\midrule
美国 & 技术领先、生态主导 & 顶尖人才、资本充裕、完整产业链 & 监管滞后、社会分化 & 重视生态系统建设 \\
欧盟 & 监管引领、价值导向 & 学术传统、标准制定影响力 & 人才流失、产业化不足 & 平衡创新与监管 \\
英国 & 安全治理、国际协调 & 金融科技、AI安全研究 & 脱欧后资源受限 & 积极参与国际规则制定 \\
日本 & 社会应用、老龄化应对 & 制造业基础、机器人技术 & 语言壁垒、创业文化弱 & 聚焦场景化应用 \\
韩国 & 半导体主导、语言模型 & HBM/DRAM领先、三星生态 & 市场规模有限 & 发挥产业链优势 \\
印度 & 人才输出、多语言AI & 工程师储备、英语优势 & 基础设施薄弱 & 重视人才培养 \\
\bottomrule
\end{tabular}
\end{table}

\section{中国的比较优势与短板}

说差距不等于唱衰。换个视角看,中国在AI竞争中握有几张独特的牌。

\subsection{比较优势}

\textbf{市场规模}。14亿人口的市场规模本身就是稀缺资源。美欧企业训练中文模型要靠爬取数据,我们则坐拥海量的原生中文语料和应用场景。微信、抖音、淘宝每天产生的交互数据,是任何实验室都无法模拟的真实用户行为样本。这种"数据土壤"的差异,会随着模型规模扩大而愈发凸显。

\textbf{工程化落地能力}。DeepSeek团队用远低于OpenAI的成本训练出性能接近的模型,靠的不是什么秘密武器,而是扎实的工程优化——数据清洗、训练调度、推理加速,每个环节都在"抠细节"。这种"卷"的能力,恰恰是国内技术团队的强项。Qwen开源后短短几个月内的迭代速度,同样印证了这一点。

\textbf{政策支持与基础设施}。《新一代人工智能发展规划》确立的顶层设计也在持续发挥作用。政府引导与市场竞争相结合的模式,至少到目前为止,在AI基础设施建设和应用推广上展现出一定效率。

\textbf{架构创新}。在混合专家模型(MoE)等特定技术方向上,国内团队展现出不俗的创新能力。DeepSeek-V2、V3的架构设计得到了国际同行的认可。

\subsection{需要正视的短板}

\textbf{高端芯片}。先进制程芯片(7nm以下)的自主制造能力仍在追赶中。出口管制使高端AI芯片(如H100/A100)获取受限,直接影响大模型训练能力。

\textbf{软件生态}。CUDA生态的护城河短期内难以逾越。Nvidia多年构建的软件生态——编译器、函数库、开发工具、社区支持——形成了强大的用户黏性。国产芯片的软件生态建设还有很长的路要走。

\textbf{基础研究}。在Transformer架构创新、训练效率优化、对齐技术等前沿方向,国内研究的原创性贡献仍需加强。

\textbf{顶尖人才}。虽然AI人才总量不少,但在最前沿研究方向(如AI对齐、可解释性)上的顶尖人才储备不足。

\subsection{SWOT分析}

\begin{table}[H]
\centering
\caption{中国AI发展SWOT分析}
\small
\begin{tabular}{p{6cm}p{6cm}}
\toprule
\textbf{优势(Strengths)} & \textbf{劣势(Weaknesses)} \\
\midrule
• 超大规模市场和数据资源 & • 高端芯片自主能力不足 \\
• 工程化落地能力强 & • 基础研究原创性不足 \\
• 政策支持力度大 & • CUDA生态依赖 \\
• 完整的产业配套 & • 顶尖人才储备不足 \\
\midrule
\textbf{机会(Opportunities)} & \textbf{威胁(Threats)} \\
\midrule
• 开源生态降低追赶门槛 & • 芯片出口管制持续收紧 \\
• 架构创新实现弯道超车 & • 技术脱钩风险加剧 \\
• 垂直场景应用领先 & • 人才外流压力 \\
• 国产替代市场空间大 & • 国际合作受限 \\
\bottomrule
\end{tabular}
\end{table}

\section{前沿科研与军事模型获取受限}

一个容易被忽视但至关重要的事实是:\textbf{国外真正最强大的AI模型——尤其是专门用于前沿科学研究和军事领域的模型——从不对外公开,或者有意推迟、限制公开}。我们日常接触到的ChatGPT、Claude等面向大众的商业模型,与这些机构内部用于突破性科研的专用模型之间,存在着难以逾越的能力鸿沟。

\subsection{科研专用模型:隐藏的能力前沿}

面向大众开放的商业大模型,本质上是经过"消费级优化"的产品——它们需要兼顾成本控制、内容合规、用户体验等多重约束。而真正用于前沿科学研究的专用模型则完全不同。

Google DeepMind的AlphaFold系列在蛋白质结构预测领域取得了革命性突破,但其最新迭代版本和内部研究工具从未完全公开;用于药物发现的专用模型、用于材料科学的AI系统、用于气候模拟的大规模模型——这些真正推动科学边界的工具,外界只能通过发表的论文窥见冰山一角。

即便是面向学术界的AI研究,信息披露也在收紧。2023年以来,OpenAI、Anthropic、Google DeepMind等机构对其最新模型的技术细节披露越来越少——GPT-4的技术报告几乎不包含任何架构和训练细节,Claude的技术路线同样高度保密。这与早期GPT-2、GPT-3时代相对开放的学术发表形成鲜明对比。

\subsection{军事领域模型:绝对的技术黑箱}

更值得警惕的是,\textbf{军事领域的AI大模型完全处于保密状态,外界对其能力边界几乎一无所知}。

美国国防部通过DARPA、国防创新单元(DIU)等机构,长期资助军事AI研发。2024年,美国国防部宣布启动"复制者"(Replicator)计划,目标是在18-24个月内部署数千个AI驱动的自主无人系统。这些系统背后的决策模型、态势感知模型、目标识别模型的真实能力,不可能出现在任何公开论文或API文档中。

2024年1月,OpenAI悄然修改了其使用政策,删除了此前明确禁止军事用途的条款。随后,OpenAI与美国国防部建立合作关系,为军方提供定制化AI服务。这一转变意味着:即便是最知名的"民用"AI公司,其最强能力也可能优先服务于国家安全需求,而非面向普通用户开放。

\subsection{对我国的战略启示}

这种"能力不对称"对中国意味着什么?

\textbf{第一,我们面对的不是"公开模型的差距",而是"未知能力的黑洞"。}当我们用GPT-4或Claude的商业API进行科研时,我们获取的是一个经过多重裁剪的"消费级"版本。而对方用于前沿科研的专用模型、用于军事决策的作战模型,其真实能力我们无从知晓。

\textbf{第二,关键时刻的"断供"风险不容忽视。}2022年以来的芯片出口管制已经证明,技术脱钩是真实的政策选项。如果地缘政治紧张进一步升级,API服务随时可能被切断。

\textbf{第三,核心技术"黑箱化"阻碍深层理解。}仅仅通过API调用,我们无法理解模型内部的工作机制,无法进行针对性的优化和改进,无法发现和修复潜在的安全漏洞。

因此,\textbf{发展自主可控的大模型能力——特别是面向前沿科学研究和国防安全的专用模型——不仅是产业竞争的需要,更是保障科学研究主权和国家安全的战略必须}。

\section{情报界的AI革命:一场静悄悄的颠覆}

\begin{tcolorbox}[colback=red!5!white,colframe=red!75!black,title=本节核心警示]
\textbf{这是本书最重要的章节之一,也是当前公开讨论中最被忽视的领域。}

美国情报界(Intelligence Community, IC)正在系统性地将大模型能力整合进情报工作全流程。这不是未来趋势,而是\textbf{正在发生的现实}。对此缺乏认知,可能导致战略误判。
\end{tcolorbox}

\subsection{美国情报界的AI战略架构}

美国情报界由18个机构组成,由国家情报总监办公室(ODNI)统筹协调。根据ODNI公开信息,其核心使命中心包括:
\begin{itemize}
    \item \textbf{国家反恐中心(NCTC)}:整合国内外反恐信息
    \item \textbf{国家反情报与安全中心(NCSC)}:负责反情报和安全威胁应对
    \item \textbf{网络威胁情报整合中心(CTIIC)}:领导情报界网络威胁情报整合,"为国家利益提供信息支持,支持国家网络政策和规划"
    \item \textbf{外国恶意影响中心(FMIC)}:应对外国势力通过公开或隐蔽手段影响政府、公众舆论和行为的威胁
    \item \textbf{国家反扩散与生物安全中心(NCBC)}:防止大规模杀伤性武器扩散
\end{itemize}

\textbf{关键文件:"AIM倡议"(Augmenting Intelligence using Machines)}

ODNI发布的"AIM倡议:利用机器增强情报的战略"是理解美国情报界AI战略的核心文件。该文件明确阐述了情报界如何将AI/机器学习能力整合进情报工作,并"解决情报界关键的法律、政策、文化、技术和结构性挑战"。

这一战略的核心理念是:\textbf{不是用AI取代人类分析员,而是用AI"增强"人类能力}——处理人类无法处理的数据规模,发现人类难以发现的隐藏模式,释放人类分析员进行更高层次的判断和决策。

\subsection{大模型在情报工作中的具体应用}

根据公开信息和技术可行性分析,大模型正在以下情报工作环节发挥作用:

\subsubsection{开源情报(OSINT)的革命性变革}

开源情报是受大模型影响最直接的领域。传统OSINT依赖人工筛选和分析,效率低下且容易遗漏。大模型带来的变革包括:

\textbf{多语言实时监控}:大模型可以同时监控数十种语言的社交媒体、新闻报道、学术论文、政府公告,实时识别潜在情报价值的信息。

\textbf{实体识别与关系图谱}:自动从海量文本中提取人物、组织、地点、事件等实体,构建复杂的关系网络图谱。

\textbf{情感与立场分析}:大规模分析公众舆论、精英观点、政策信号的微妙变化。

\textbf{案例}:2022年俄乌冲突中,开源情报社区Bellingcat等机构展示了惊人能力——通过社交媒体帖子追踪军事部署,利用商业卫星图像监测设施变化,分析移动数据推断部队调动。\textbf{大模型将使这种分析能力提升数个数量级,同时大幅降低准入门槛}。

\subsubsection{信号情报(SIGINT)的处理能力跃升}

信号情报涉及通信拦截和电子信号分析。大模型的应用包括:

\begin{itemize}
    \item \textbf{语音转文本与翻译}:实时将拦截的通话转换为可分析的文本,支持数十种语言
    \item \textbf{语义理解}:识别隐语、暗号、编码通信的真实含义
    \item \textbf{模式识别}:从海量通信元数据中识别异常模式和目标活动特征
    \item \textbf{预测分析}:基于历史通信模式预测目标下一步行动
\end{itemize}

\subsubsection{图像情报(IMINT)的自动化分析}

多模态大模型对图像情报的影响深远:

\textbf{卫星图像自动解译}:识别军事设施、武器装备、部队部署的变化。传统上需要数周的分析工作可压缩到小时级别。

\textbf{变化检测}:自动对比不同时间点的图像,识别建设活动、设备移动、异常行为。

\textbf{目标识别与分类}:自动识别和分类舰艇、飞机、车辆、导弹系统等军事目标。

\textbf{跨模态融合}:将图像证据与其他情报来源(通信、人力情报)进行交叉验证。

\subsubsection{人力情报(HUMINT)的支持系统}

虽然人力情报本质上依赖人际互动,但大模型可以提供强大的后台支持:

\begin{itemize}
    \item \textbf{目标档案生成}:自动整合多源信息,生成目标人物的综合画像
    \item \textbf{关系网络分析}:识别目标的社会关系、影响圈、潜在弱点
    \item \textbf{面谈准备}:根据目标特征和历史信息,生成个性化的接触策略建议
    \item \textbf{虚假身份构建}:协助生成一致且难以识破的掩护身份背景
\end{itemize}

\subsubsection{反情报的主动防御}

NCSC(国家反情报与安全中心)的职责包括"领导情报界和跨部门反情报活动,应对威胁国家安全的关键信息和资产的威胁"。大模型在反情报领域的应用包括:

\begin{itemize}
    \item \textbf{内部威胁检测}:分析员工行为模式,识别潜在泄密或叛逃迹象
    \item \textbf{网络钓鱼防御}:识别针对敏感人员的社会工程攻击
    \item \textbf{外国情报活动识别}:监测和分析外国情报机构的活动模式
    \item \textbf{信息战对抗}:识别和追踪外国恶意影响行动
\end{itemize}

\subsection{CTIIC与网络情报整合}

网络威胁情报整合中心(CTIIC)的职责是"领导情报界网络威胁情报整合,为国家利益提供信息支持,支持国家网络政策和规划,协调情报界网络收集和投资的统一方法"。

大模型在网络情报领域的应用尤为关键:

\begin{itemize}
    \item \textbf{攻击归因}:分析恶意代码风格、攻击手法、基础设施特征,推断攻击者身份
    \item \textbf{威胁预测}:基于历史攻击模式和当前态势,预测潜在攻击目标和时机
    \item \textbf{漏洞情报}:自动分析新披露漏洞的可利用性和影响范围
    \item \textbf{暗网监控}:监测地下论坛的威胁情报交易和攻击工具流通
\end{itemize}

\subsection{情报分析的范式转换}

大模型正在改变情报分析的基本范式:

\textbf{从人工筛选到机器过滤}:情报分析员不再需要从海量原始信息中大海捞针,而是专注于评估和解读机器预筛选的高价值信息。

\textbf{从事后分析到实时预警}:大模型可以持续监控多源信息流,在威胁信号初现时即触发预警,而不是等到威胁已经显现。

\textbf{从单一视角到多维融合}:大模型可以同时处理文本、图像、音频、结构化数据,提供多维度的综合分析。

\textbf{从确定性结论到概率分布}:大模型可以生成多种可能情景及其概率评估,支持决策者在不确定性中做出判断。

\subsection{对我国的深层启示}

\textbf{第一,情报能力的代差正在形成。}如果我们仍然依赖传统方法进行情报分析,而对手已经用大模型实现了效率和能力的数量级提升,这种能力代差可能在关键时刻转化为战略劣势。

\textbf{第二,"商业模型不等于情报模型"。}我们日常使用的ChatGPT、Claude等商业产品,与情报机构内部使用的专用模型在能力上可能存在显著差距。后者可能在特定任务上经过深度优化,并获得商业模型无法获取的专有数据。

\textbf{第三,反情报工作面临新挑战。}当对手使用大模型进行目标分析和接触策略优化时,我们的反情报体系需要相应升级。传统的人工甄别方法可能难以应对AI增强的渗透手段。

\textbf{第四,需要建立自己的情报AI能力。}这不是可选项,而是必需项。需要培养既懂AI技术又懂情报业务的复合型人才,需要建设专门的情报AI基础设施,需要发展适合我国情报需求的专用模型。

\section{战略窗口期的判断}

当前正处于技术范式确立的关键窗口期。技术范式尚未完全定型,后发者仍有追赶甚至超越的可能。但这个窗口期不会永远敞开——一旦领先者形成数据飞轮、人才虹吸、生态锁定效应,后来者将面临"强者恒强"的马太效应。

\subsection{窗口期的时间判断}

基于技术发展规律和产业周期,我们对窗口期有以下判断:

\textbf{短期窗口(2025-2027)}:基础模型能力仍在快速迭代,架构创新可能改变竞争格局。这是追赶的黄金时期。

\textbf{中期窗口(2027-2030)}:应用生态逐步成熟,平台级产品形成锁定效应。追赶难度显著增加。

\textbf{长期态势(2030年后)}:技术范式趋于稳定,产业格局基本定型。错过窗口期的后果将长期显现。

\subsection{国际比较的启示}

通过国际比较可得出以下启示:

\begin{itemize}
    \item 生态系统的完整性是竞争力核心
    \item 各国根据自身禀赋选择差异化路径
    \item 成功的AI战略需要政府引导与市场力量的有效结合
    \item 开放与自主并非对立,需要在两者之间寻求动态平衡
\end{itemize}

\section{竞争格局的未来演进}

展望未来,全球AI竞争格局可能呈现以下特征:

\textbf{技术层面}:从规模竞赛转向效率竞赛,架构创新和工程优化的重要性上升。

\textbf{产业层面}:垂直整合与专业分工并存,生态系统竞争日趋激烈。

\textbf{地缘层面}:"技术民族主义"抬头,供应链安全成为各国核心关切。

\textbf{治理层面}:国际规则博弈加剧,不同治理模式竞争共存。

中国的战略选择,将深刻影响这一格局的最终形态。
