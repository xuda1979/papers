% ==================== 前言 ====================
\chapter*{前\quad 言}
\addcontentsline{toc}{chapter}{前言}
\markboth{前言}{前言}

\section*{为什么写这本书}

2022年11月30日,OpenAI发布ChatGPT,人工智能的历史从此分为"前ChatGPT时代"和"后ChatGPT时代"。作为一名长期从事人工智能研究的科研工作者,我亲眼见证了这场技术革命的爆发,也深刻感受到它正在如何改变我们的工作方式、思维模式,乃至整个社会的运行逻辑。

然而,在这场全球性的AI竞赛中,我同时也感受到一种深深的忧虑。

忧虑来自于差距。尽管中国在AI应用落地、工程优化方面展现出令人瞩目的竞争力,但在基础研究、高端芯片、核心算法等"硬科技"领域,我们与世界领先水平仍存在不可忽视的差距。更令人担忧的是,随着地缘政治紧张加剧,技术封锁和供应链断裂的风险与日俱增。

忧虑也来自于认识。大模型技术的战略重要性,可能尚未被充分理解。许多人将其视为一项普通的技术进步,未能意识到它正在重构国家竞争力的基础。如果说核技术决定了20世纪的战略格局,那么大模型很可能决定21世纪的竞争态势——它直接作用于人类最核心的能力:认知与创造。

正是出于这种忧虑和责任感,我决定写这本书。我的目标很明确:\textbf{呼吁国家对大模型技术给予最高度的重视},将其提升到与"两弹一星"同等的战略高度。这不是危言耸听,而是基于对技术发展规律和国际竞争态势的冷静判断。

\section*{本书的核心观点}

本书的核心论点可以概括为以下几点:

\textbf{第一,大模型是"通用目的技术"(General-Purpose Technology)}。它不是某一个行业的专用工具,而是一个能够赋能几乎所有行业的"技术引擎"。正如蒸汽机催生了第一次工业革命、电力催生了第二次工业革命,大模型正在催生"智能革命"。

\textbf{第二,大模型将全方位影响国家发展}。本书从经济、科技创新、社会治理、教育、医疗、文化、国家安全七个维度,系统论证了大模型对国家的深远影响。任何一个领域的落后,都可能产生连锁反应。

\textbf{第三,当前正处于关键的战略窗口期}。技术范式尚未完全定型,后发者仍有追赶甚至超越的可能。但这个窗口不会永远敞开——一旦领先者形成数据飞轮、人才虹吸、生态锁定效应,后来者将面临"强者恒强"的马太效应。

\textbf{第四,必须以最高战略优先级推进大模型发展}。这意味着在算力基础设施、顶尖人才、基础研究、应用生态等方面进行超常规投入,建立最高层级的统筹协调机制,以举国体制与市场机制双轮驱动。

\section*{本书的结构}

本书分为七章,外加附录:

\textbf{第一章}阐述大模型的技术本质和发展态势,揭示其作为"通用目的技术"的战略地位,并提出本书的核心论点。

\textbf{第二章}从七个维度——经济、科技创新、社会治理、教育、医疗、文化、国家安全——系统论证大模型对国家发展的全方位影响。

\textbf{第三章}分析全球AI竞争格局,比较主要国家的AI战略,评估中国的比较优势与面临的挑战。

\textbf{第四章}深入分析大模型带来的各类风险,包括技术差距风险、供应链风险、信息安全风险、认知安全风险等,构建系统的风险评估框架。

\textbf{第五章}提出应对策略,包括技术能力建设、安全防护体系、"算法拒止"机制等创新方案。

\textbf{第六章}讨论人才培养、制度建设、国际合作等保障措施,提出短中长期行动路线图。

\textbf{第七章}总结全书观点,提出核心政策建议,发出"给予大模型最高度重视"的呼吁。

\section*{致谢}

本书的写作得到了许多人的帮助和支持。

感谢中国移动研究院的领导和同事们,为本研究提供了良好的工作环境和资源支持。感谢在专家咨询环节提供宝贵意见的各位专家学者(名单见附录)。感谢家人的理解和支持,在我埋头写作的日子里给予了无尽的包容。

本书的观点仅代表作者个人学术见解,不代表任何机构立场。由于作者水平有限,书中难免存在不足之处,恳请读者批评指正。

如果本书能够引起决策者和社会各界对大模型战略重要性的更多关注,能够为我国AI发展战略的制定提供一些参考,将是作者最大的欣慰。

\vspace{2em}
\begin{flushright}
许\quad 达\\
2025年12月于北京
\end{flushright}
