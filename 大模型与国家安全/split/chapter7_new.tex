% ==================== 第七章 ====================
\chapter{行动纲领:可操作的政策建议清单}

\begin{tcolorbox}[colback=red!5!white,colframe=red!75!black,title=本章定位]
\textbf{本章不再分析问题,而是直接给出解决方案清单。}

前六章已经完成了"为什么"、"是什么"、"怎么办"的论证。本章将所有建议汇总为可直接执行的政策清单,明确:
\begin{itemize}
    \item 具体的行动事项
    \item 明确的责任主体
    \item 清晰的时间节点
    \item 可量化的评估标准
\end{itemize}

\textbf{原则}:每条建议都能直接进入政策文件。
\end{tcolorbox}

\section{战略定位确认}

\begin{tcolorbox}[colback=yellow!10!white,colframe=orange!75!black,title=核心判断]
\textbf{大型语言模型是21世纪国家竞争的决定性力量。}

它不仅是一项技术,而是国家的\textbf{认知基础设施}——与电网、互联网同等重要,甚至更加关键,因为它直接作用于人类最核心的能力:认知与创造。

\textbf{历史类比}:
\begin{itemize}
    \item 核技术决定了20世纪的军事平衡
    \item 大模型将决定21世纪的认知平衡
    \item 核技术的差距可以通过威慑维持稳定
    \item 认知能力的差距则意味着全面落后
\end{itemize}

\textbf{结论}:必须以"两弹一星"的战略高度对待大模型发展。
\end{tcolorbox}

\section{顶层设计行动项}

\begin{tcolorbox}[colback=blue!5!white,colframe=blue!75!black,title=行动项1:成立国家认知基础设施建设领导小组]
\textbf{优先级}:P0(最高)

\textbf{责任主体}:中央

\textbf{时间节点}:2025年Q2前完成

\textbf{具体内容}:
\begin{enumerate}
    \item 成立由政治局常委分管的领导小组
    \item 成员包括科技部、工信部、发改委、教育部、财政部、国安委、网信办负责人
    \item 设立专职秘书处,配备50-100名专业人员
    \item 建立季度汇报和年度评估制度
\end{enumerate}

\textbf{评估标准}:
\begin{itemize}
    \item 领导小组是否正式成立并开始运作
    \item 第一次全体会议是否召开
    \item 《国家认知基础设施建设规划》是否启动编制
\end{itemize}
\end{tcolorbox}

\begin{tcolorbox}[colback=blue!5!white,colframe=blue!75!black,title=行动项2:发布《国家认知基础设施建设规划(2025-2035)》]
\textbf{优先级}:P0

\textbf{责任主体}:国家发改委牵头,科技部、工信部配合

\textbf{时间节点}:2025年Q3前发布

\textbf{具体内容}:
\begin{enumerate}
    \item 明确2025-2035年建设目标和里程碑
    \item 确定投资规模(建议2-3万亿元)
    \item 明确资金来源和分配机制
    \item 建立项目管理和考核机制
\end{enumerate}

\textbf{评估标准}:
\begin{itemize}
    \item 规划是否正式发布
    \item 是否获得足额预算支持
    \item 配套实施方案是否制定
\end{itemize}
\end{tcolorbox}

\section{算力基础设施行动项}

\begin{tcolorbox}[colback=green!5!white,colframe=green!75!black,title=行动项3:启动国家智算中心建设工程]
\textbf{优先级}:P0

\textbf{责任主体}:国家发改委牵头,工信部配合,地方政府执行

\textbf{时间节点}:
\begin{itemize}
    \item 2025年Q4:首批3个万卡级中心开工
    \item 2027年Q2:首个万卡级中心投入运营
    \item 2030年底:10个万卡级中心全部运营
\end{itemize}

\textbf{投资规模}:1.2-1.5万亿元(十年累计)

\textbf{布局方案}:
\begin{itemize}
    \item 西部训练集群:内蒙古2-3个、甘肃/青海1-2个、四川/贵州2-3个
    \item 东部推理集群:北京/天津2个、上海/杭州2个、深圳/广州1-2个
    \item 专用安全集群:位置保密,用于敏感任务
\end{itemize}

\textbf{评估标准}:
\begin{itemize}
    \item 开工数量和进度是否达标
    \item 算力总量是否按计划增长
    \item 能耗和运营效率是否达标
\end{itemize}
\end{tcolorbox}

\begin{tcolorbox}[colback=green!5!white,colframe=green!75!black,title=行动项4:建设国家算力网络]
\textbf{优先级}:P1

\textbf{责任主体}:工信部牵头,三大运营商执行

\textbf{时间节点}:2027年底前完成主干网建设

\textbf{投资规模}:1000-1500亿元

\textbf{具体内容}:
\begin{enumerate}
    \item 骨干网:400G-1T带宽,连接各智算中心
    \item 城域网:100G-400G带宽,连接用户
    \item RDMA网络:支持分布式训练
    \item 智能调度系统:跨中心任务调度
\end{enumerate}

\textbf{评估标准}:
\begin{itemize}
    \item 网络覆盖范围和带宽
    \item 跨中心训练的延迟和效率
    \item 算力利用率
\end{itemize}
\end{tcolorbox}

\section{芯片攻关行动项}

\begin{tcolorbox}[colback=yellow!5!white,colframe=orange!75!black,title=行动项5:国产AI芯片生态建设专项]
\textbf{优先级}:P0

\textbf{责任主体}:工信部牵头,科技部配合,华为等企业执行

\textbf{时间节点}:
\begin{itemize}
    \item 2025-2026:存量优化,软件生态初步形成
    \item 2026-2028:国产替代,80\%训练任务使用国产芯片
    \item 2028-2035:自主领先,关键指标达到国际先进
\end{itemize}

\textbf{投资规模}:5000-6000亿元

\textbf{重点任务}:
\begin{enumerate}
    \item 芯片设计能力提升(华为Ascend系列)
    \item 先进封装技术突破(弥补制程差距)
    \item 软件生态建设(类CUDA框架)
    \item 算子库完善和优化
    \item 开发者社区培育
\end{enumerate}

\textbf{评估标准}:
\begin{itemize}
    \item 国产芯片在智算中心的部署比例
    \item 软件迁移的完成度和性能
    \item 开发者数量和活跃度
\end{itemize}
\end{tcolorbox}

\section{模型研发行动项}

\begin{tcolorbox}[colback=blue!5!white,colframe=blue!75!black,title=行动项6:启动"国家基础大模型"工程]
\textbf{优先级}:P0

\textbf{责任主体}:科技部牵头,新成立国家实验室或指定头部企业执行

\textbf{时间节点}:
\begin{itemize}
    \item 2026年Q2:工程正式启动
    \item 2027年Q4:V1.0发布,达到GPT-5同等水平
    \item 2030年:综合能力世界一流
\end{itemize}

\textbf{投资规模}:1000-1500亿元

\textbf{组织模式}:
\begin{enumerate}
    \item 牵头单位:新设国家实验室或指定企业(如深度求索)
    \item 参与单位:阿里、百度、华为、腾讯等头部企业,清华、北大等顶尖高校
    \item 资源保障:国家智算中心算力优先保障
    \item 知识产权:国家持有核心IP,参与单位享有使用权
\end{enumerate}

\textbf{评估标准}:
\begin{itemize}
    \item 模型能力在国际基准测试上的排名
    \item 关键能力(推理、代码、数学)的突破情况
    \item 国产化程度(数据、算力、框架)
\end{itemize}
\end{tcolorbox}

\begin{tcolorbox}[colback=blue!5!white,colframe=blue!75!black,title=行动项7:行业垂直模型专项]
\textbf{优先级}:P1

\textbf{责任主体}:各行业主管部门牵头

\textbf{时间节点}:2027年底前完成重点领域模型开发

\textbf{投资规模}:500-800亿元

\textbf{重点领域}:
\begin{enumerate}
    \item 科研模型(科技部):辅助科学研究
    \item 教育模型(教育部):个性化学习
    \item 医疗模型(卫健委):辅助诊断、药物研发
    \item 政务模型(国办):政务服务、决策支持
    \item 法律模型(司法部):法律咨询、合同审查
    \item 金融模型(金融监管总局):风控、投研
\end{enumerate}

\textbf{评估标准}:
\begin{itemize}
    \item 各领域模型的部署和应用情况
    \item 应用效果评估(效率提升、准确率等)
    \item 用户反馈和满意度
\end{itemize}
\end{tcolorbox}

\section{人才培养行动项}

\begin{tcolorbox}[colback=purple!5!white,colframe=purple!75!black,title=行动项8:AI安全人才专项培养计划]
\textbf{优先级}:P0

\textbf{责任主体}:教育部牵头,科技部、人社部配合

\textbf{时间节点}:2025年启动,2030年形成完整体系

\textbf{培养目标}:
\begin{itemize}
    \item 对齐研究人才:100人
    \item 红队测试人才:1000人
    \item AI安全工程师:10000人
    \item AI治理研究人才:500人
    \item 复合型战略人才:100人
\end{itemize}

\textbf{具体措施}:
\begin{enumerate}
    \item 设立"AI安全"本科专业方向(10所高校)
    \item 设立"AI治理"硕士项目(5所高校)
    \item 设立"对齐研究英才班"(博士层次)
    \item 建立职业培训和认证体系
    \item 举办"国家AI安全挑战赛"
\end{enumerate}

\textbf{评估标准}:
\begin{itemize}
    \item 各类人才培养数量是否达标
    \item 毕业生就业和发展情况
    \item 国际排名和影响力
\end{itemize}
\end{tcolorbox}

\begin{tcolorbox}[colback=purple!5!white,colframe=purple!75!black,title=行动项9:顶尖人才引进专项]
\textbf{优先级}:P0

\textbf{责任主体}:科技部牵头,各省市人才办配合

\textbf{时间节点}:立即启动,持续推进

\textbf{引进目标}:
\begin{itemize}
    \item 引进20-30名在顶级AI实验室有对齐研究经验的华人
    \item 引进50-100名资深AI安全工程师
    \item 引进20-30名AI治理国际专家
\end{itemize}

\textbf{待遇标准}:
\begin{itemize}
    \item 顶尖对齐研究者:年薪300-800万元+安家费+科研经费
    \item 资深安全工程师:年薪150-300万元
    \item 治理专家:年薪100-200万元
\end{itemize}

\textbf{配套措施}:
\begin{enumerate}
    \item 解决住房、子女教育、配偶就业
    \item 简化签证和居留手续
    \item 提供充足科研启动经费
    \item 建立灵活的考核机制
\end{enumerate}

\textbf{评估标准}:
\begin{itemize}
    \item 引进人才数量和质量
    \item 引进人才的留任率
    \item 引进人才的成果产出
\end{itemize}
\end{tcolorbox}

\section{安全防护行动项}

\begin{tcolorbox}[colback=red!5!white,colframe=red!75!black,title=行动项10:部署四层安全网关架构]
\textbf{优先级}:P0

\textbf{责任主体}:国家网信办牵头,公安部、国安部配合

\textbf{时间节点}:
\begin{itemize}
    \item 2025年底:完成架构设计和试点部署
    \item 2026年底:覆盖所有政务和关键基础设施AI应用
    \item 2027年底:覆盖所有高风险商业应用
\end{itemize}

\textbf{四层架构}:
\begin{enumerate}
    \item 第一层:系统提示硬化(防止用户覆盖)
    \item 第二层:输入净化(检测注入和越狱)
    \item 第三层:工具/资源隔离(最小权限)
    \item 第四层:输出审计与溯源(风险检测+水印)
\end{enumerate}

\textbf{评估标准}:
\begin{itemize}
    \item 覆盖率是否达标
    \item 拒止成功率 $\geq$ 95\%
    \item 误杀率 $\leq$ 5\%
\end{itemize}
\end{tcolorbox}

\begin{tcolorbox}[colback=red!5!white,colframe=red!75!black,title=行动项11:建立国家AI安全测评体系]
\textbf{优先级}:P1

\textbf{责任主体}:科技部牵头,设立专门机构

\textbf{时间节点}:2026年底前建成并运营

\textbf{具体内容}:
\begin{enumerate}
    \item 成立"国家AI安全测评中心"
    \item 建立大模型安全测评标准
    \item 组建专业红队测试队伍
    \item 建立漏洞库和情报共享机制
    \item 定期发布安全评估报告
\end{enumerate}

\textbf{测评范围}:
\begin{itemize}
    \item 所有在境内提供服务的大模型
    \item 政务和关键基础设施使用的AI系统
    \item 高风险应用场景
\end{itemize}

\textbf{评估标准}:
\begin{itemize}
    \item 测评能力是否建立
    \item 测评覆盖率
    \item 漏洞发现和修复情况
\end{itemize}
\end{tcolorbox}

\section{治理体系行动项}

\begin{tcolorbox}[colback=green!5!white,colframe=green!75!black,title=行动项12:推进《人工智能法》立法]
\textbf{优先级}:P1

\textbf{责任主体}:全国人大法工委牵头,科技部、网信办配合

\textbf{时间节点}:
\begin{itemize}
    \item 2025年:立项并启动调研
    \item 2026年:形成草案并征求意见
    \item 2027年:提请审议并出台
\end{itemize}

\textbf{核心内容}:
\begin{enumerate}
    \item AI发展与治理的基本原则
    \item 监管体制和职责分工
    \item 高风险AI系统的管理要求
    \item AI安全的基本制度
    \item 数据和算法治理
    \item 法律责任和救济机制
\end{enumerate}

\textbf{评估标准}:
\begin{itemize}
    \item 立法进度是否按计划推进
    \item 法律出台后的实施效果
\end{itemize}
\end{tcolorbox}

\begin{tcolorbox}[colback=green!5!white,colframe=green!75!black,title=行动项13:建立AI治理"一委一办多中心"架构]
\textbf{优先级}:P1

\textbf{责任主体}:中央编办牵头

\textbf{时间节点}:2026年底前完成

\textbf{具体架构}:
\begin{enumerate}
    \item \textbf{一委}:国家AI发展与安全委员会(中央层面)
    \item \textbf{一办}:国家AI治理办公室(正部级或副部级)
    \item \textbf{多中心}:
    \begin{itemize}
        \item 国家AI安全测评中心
        \item 国家AI标准研究中心
        \item 国家AI伦理研究中心
        \item 国家AI应急响应中心
    \end{itemize}
\end{enumerate}

\textbf{评估标准}:
\begin{itemize}
    \item 机构是否按计划设立
    \item 职能划分是否清晰
    \item 协调机制是否有效运作
\end{itemize}
\end{tcolorbox}

\section{国际合作行动项}

\begin{tcolorbox}[colback=blue!5!white,colframe=blue!75!black,title=行动项14:建立中美AI安全对话机制]
\textbf{优先级}:P1

\textbf{责任主体}:外交部牵头,科技部配合

\textbf{时间节点}:争取2025年底前启动

\textbf{对话议题}:
\begin{enumerate}
    \item AI安全对齐研究合作
    \item 防止AI用于大规模杀伤性武器
    \item 核指挥控制系统的AI隔离
    \item AI军备控制框架探讨
    \item 学术交流机制化
\end{enumerate}

\textbf{评估标准}:
\begin{itemize}
    \item 对话机制是否建立
    \item 对话频率和层级
    \item 是否达成任何共识
\end{itemize}
\end{tcolorbox}

\begin{tcolorbox}[colback=blue!5!white,colframe=blue!75!black,title=行动项15:参与国际AI标准制定]
\textbf{优先级}:P2

\textbf{责任主体}:国家标准委牵头,科技部、工信部配合

\textbf{时间节点}:持续推进

\textbf{重点领域}:
\begin{enumerate}
    \item ISO/IEC JTC 1/SC 42(AI国际标准)
    \item IEEE AI标准工作组
    \item 联合国AI治理框架
    \item 区域合作标准(如RCEP框架)
\end{enumerate}

\textbf{评估标准}:
\begin{itemize}
    \item 中国专家在国际标准组织中的职位数
    \item 中国主导或参与制定的标准数量
    \item 国际标准采纳中国方案的比例
\end{itemize}
\end{tcolorbox}

\section{总体时间表}

\begin{table}[H]
\centering
\caption{行动纲领总体时间表}
\small
\begin{tabular}{p{2cm}p{10cm}}
\toprule
\textbf{时间} & \textbf{关键里程碑} \\
\midrule
2025Q2 & 国家认知基础设施建设领导小组成立 \\
2025Q3 & 《国家认知基础设施建设规划》发布 \\
2025Q4 & 首批3个国家智算中心开工;AI安全人才计划启动 \\
2025Q4 & 中美AI安全对话机制启动(争取) \\
2026Q2 & 国家基础大模型工程启动 \\
2026Q4 & AI治理"一委一办多中心"架构建成 \\
2026Q4 & 四层安全网关架构完成政务系统部署 \\
2027Q2 & 首个万卡级国家智算中心投入运营 \\
2027Q4 & 国家基础大模型V1.0发布;《人工智能法》出台 \\
2028Q4 & 5个万卡级智算中心运营;国产芯片占比超80\% \\
2030Q4 & 10+万卡级智算中心;基础模型达世界一流 \\
2035 & 全面建成自主可控的国家认知基础设施 \\
\bottomrule
\end{tabular}
\end{table}

\section{资源需求汇总}

\begin{table}[H]
\centering
\caption{十年投资规模汇总(2025-2035)}
\small
\begin{tabular}{p{4cm}p{3cm}p{5cm}}
\toprule
\textbf{领域} & \textbf{投资规模} & \textbf{主要用途} \\
\midrule
算力基础设施 & 1.2-1.5万亿元 & 智算中心、算力网络、能源配套 \\
芯片攻关 & 5000-6000亿元 & 先进制程、设备材料、软件生态 \\
模型研发 & 2000-3000亿元 & 基础模型、垂直模型、开源生态 \\
人才培养 & 1000-1500亿元 & 高校建设、人才引进、职业培训 \\
应用推广 & 1000-1500亿元 & 政务、教育、医疗等应用 \\
安全体系 & 300-500亿元 & 测评中心、安全网关、应急响应 \\
\midrule
\textbf{总计} & \textbf{2-3万亿元} & \\
\bottomrule
\end{tabular}
\end{table}

\section{风险与应急}

\textbf{主要风险}:
\begin{enumerate}
    \item \textbf{芯片断供升级}:美国可能进一步收紧管制
    \item \textbf{技术路线变化}:AI技术可能出现颠覆性变革
    \item \textbf{人才流失}:顶尖人才持续外流
    \item \textbf{安全事件}:重大AI安全事件发生
\end{enumerate}

\textbf{应急准备}:
\begin{enumerate}
    \item 保持6-12个月关键芯片战略储备
    \item 建立技术路线多元化机制
    \item 完善人才留用和回流政策
    \item 建立AI安全应急响应机制
\end{enumerate}

\section{结语}

\begin{tcolorbox}[colback=yellow!10!white,colframe=orange!75!black,title=致决策者]
历史经验表明,在技术革命的关键节点,战略判断和决策果断至关重要。

\textbf{蒸汽机时代},英国的领先奠定了一个世纪的霸权。\\
\textbf{电气化时代},美国和德国的崛起改变了世界格局。\\
\textbf{信息革命时代},硅谷的创新塑造了数字经济版图。\\
\textbf{核武器时代},"两弹一星"确保了中国的大国地位。

\textbf{大模型革命,我们不能错过。}

这不是一项普通的技术,而是关乎国家认知能力的基础设施。\\
这不是一场普通的竞争,而是决定21世纪国家地位的战略角逐。

\textbf{窗口期有限,行动刻不容缓。}

本书给出了详细的路线图、时间表和资源需求。\\
剩下的,是决策和执行。

\vspace{1em}
\textit{愿这本书能为中国的认知基础设施建设贡献绵薄之力。}\\
\textit{愿我们不负这个时代赋予的机遇与使命。}
\end{tcolorbox}

