% ==================== 后记 ====================
\chapter*{后\quad 记}
\addcontentsline{toc}{chapter}{后记}
\markboth{后记}{后记}

写完这本书的最后一个字,窗外已是深夜。

回顾过去一年的写作历程,最大的感受是"紧迫"二字。每次觉得某个章节已经写完,新的模型发布、新的政策出台、新的行业动态,又迫使我不得不更新内容。GPT-5来了,Claude Opus 4.5来了,Gemini 3来了……技术演进的速度,远超我落笔的速度。

这恰恰印证了本书的核心判断:我们正处于一个技术革命的关键节点,时间窗口稍纵即逝。

作为一名科研工作者,我深知本书的局限性。AI领域发展太快,任何静态的分析都可能很快过时;作为技术背景出身的研究者,我对政策实施层面的理解可能不够深入;本书的许多判断带有一定的主观性,不同视角可能得出不同结论。

但我仍然选择写下这本书,因为我相信:在这样一个关键时刻,提出问题、引发讨论,比追求完美更重要。如果本书能够让更多人认识到大模型的战略重要性,能够为决策者提供一些参考,能够激发更多更深入的研究和讨论,那么写作的目的就达到了。

在写作过程中,我多次问自己:作为一个普通的科研工作者,写这样一本涉及国家战略的书,是否有些"越位"?但我最终说服了自己——在这个时代,每个人都有责任为国家发展贡献自己的思考。正如古人所言,"国家兴亡,匹夫有责"。AI时代的国家竞争,同样需要每一个人的参与和贡献。

技术发展有其客观规律,但国家战略的选择权在我们自己手中。

历史总是青睐那些在关键时刻做出正确选择的民族。蒸汽机革命,英国崛起;电气化革命,美德崛起;信息革命,硅谷崛起。站在智能革命的门槛上,中国面临着前所未有的机遇与挑战。

我坚信,只要我们以战略眼光认识大模型的重要性,以超常决心投入资源,以开放心态参与全球竞争,中国完全有能力在这场智能革命中占据一席之地,甚至引领潮流。

最后,我想对年轻的读者说:你们是AI时代的原住民,也是这场变革的主力军。无论你从事什么职业,AI都将与你的工作和生活深度交织。希望你们能够拥抱这场变革,成为变革的参与者和塑造者,而不是旁观者。

这是一个充满挑战的时代,也是一个充满机遇的时代。

让我们一起,不负这个时代。

\vspace{2em}
\begin{flushright}
许\quad 达\\
2025年12月于北京
\end{flushright}
