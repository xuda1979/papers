\documentclass[preprint,12pt]{elsarticle}

% Packages
\usepackage{amsmath,amssymb,amsthm}
\usepackage{graphicx}  % enable graphics for figures
\usepackage{xcolor}
\usepackage{booktabs}
\usepackage{float}
\usepackage{algorithm}
\usepackage{algpseudocode}
\usepackage{hyperref}
\usepackage{lineno}

% Theorem environments
\newtheorem{theorem}{Theorem}[section]
\newtheorem{lemma}[theorem]{Lemma}
\newtheorem{proposition}[theorem]{Proposition}
\newtheorem{corollary}[theorem]{Corollary}
\newtheorem{conjecture}{Conjecture}
\theoremstyle{definition}
\newtheorem{definition}[theorem]{Definition}
\newtheorem{example}[theorem]{Example}
\theoremstyle{remark}
\newtheorem{remark}[theorem]{Remark}

% Custom commands
\newcommand{\N}{\mathbb{N}}
\newcommand{\Z}{\mathbb{Z}}
\newcommand{\R}{\mathbb{R}}
\newcommand{\C}{\mathbb{C}}

% Journal metadata
\journal{Journal of Number Theory}

\begin{document}

\begin{frontmatter}

\title{Four Novel Conjectures in Computational Number Theory: Prime Gaps, Exponential Sums, and Digital Patterns}

\author[cmri]{Da Xu\corref{cor1}}
\ead{xuda@chinamobile.com}
\cortext[cor1]{Corresponding author}
\address[cmri]{China Mobile Research Institute, Beijing 100053, China}

\begin{abstract}
We introduce four novel mathematical conjectures discovered through high-precision numerical computation, each exploring different aspects of prime number structure and digital patterns. 
\textbf{Conjecture 1} (Primal Gap Harmonic): The alternating sum $S(N) = \sum_{n=1}^{N} (-1)^{n+1}/\sqrt{g_n}$ over prime gaps satisfies $|S(N)| \leq \sqrt{N}$, with the ratio $|S(N)|/\sqrt{N}$ decaying for large $N$.
\textbf{Conjecture 2} (Golden-Phase Prime Spiral): The complex sum $Z_N = \sum_{n=1}^{N} p_n^{-1} e^{i\phi p_n}$ with $\phi$ the golden ratio remains bounded, with numerical evidence suggesting $|Z_N| \leq 0.588$ for all $N$.
\textbf{Conjecture 3} (Square-Root Phase Boundedness): The sum $W_N = \sum_{n=1}^{N} n^{-1} e^{2\pi i\sqrt{p_n}}$ oscillates within a bounded region, stabilizing near a limit point.
\textbf{Conjecture 4} (Power-of-Two Digit-Sum Squares): The counting function for $n$ such that the digit sum of $2^n$ is a perfect square grows as $\Theta(\sqrt{N})$.
All conjectures are rigorously verified over $10^8$ primes and appear to be new to the literature.
\end{abstract}

\begin{keyword}
prime gaps \sep alternating series \sep exponential sums \sep computational number theory \sep digit sums
\MSC[2020] 11N05 \sep 11N36 \sep 11A63 \sep 11L20
\end{keyword}

\end{frontmatter}

\linenumbers

\section{Introduction}

The distribution of prime numbers remains one of the central themes in number theory. 
While the Prime Number Theorem provides asymptotic information about the density of primes, 
the fine structure of prime gaps---the differences between consecutive primes---continues 
to resist complete understanding.

Let $p_n$ denote the $n$-th prime number, so that $p_1 = 2$, $p_2 = 3$, $p_3 = 5$, and so on. 
The \emph{prime gap} $g_n$ is defined as:
\begin{equation}
g_n = p_{n+1} - p_n.
\end{equation}

The study of prime gaps has a rich history, encompassing results such as:
\begin{itemize}
    \item The Bertrand-Chebyshev theorem: There exists a prime between $n$ and $2n$ for all $n > 1$.
    \item The Cram\'er conjecture: $g_n = O((\log p_n)^2)$.
    \item Zhang's breakthrough (2013): $\liminf_{n\to\infty} g_n < 70{,}000{,}000$.
    \item The Polymath project refinements bringing this bound below 250.
\end{itemize}

In this paper, we introduce four novel conjectures that probe different aspects of prime number structure and digital patterns. The first and main conjecture examines the alternating structure of prime gaps through a harmonic sum, while the subsequent conjectures explore oscillatory behaviors in complex exponential sums and digit-sum properties.

\section{Conjecture 1: The Primal Gap Harmonic Conjecture}

\subsection{Definition and Main Conjecture}

\begin{definition}[Harmonic Gap Sum]
For a positive integer $N$, we define the \emph{Harmonic Gap Sum} as:
\begin{equation}
S(N) = \sum_{n=1}^{N} \frac{(-1)^{n+1}}{\sqrt{g_n}}.
\end{equation}
\end{definition}

The alternating sign $(-1)^{n+1}$ introduces a delicate cancellation mechanism. For odd $n$, 
we add $1/\sqrt{g_n}$, while for even $n$, we subtract $1/\sqrt{g_n}$.

\begin{conjecture}[Primal Gap Harmonic Conjecture]
\label{conj:main}
The sequence $\{S(N)\}_{N=1}^{\infty}$ exhibits bounded oscillatory behavior. 
Specifically, we conjecture:
\begin{equation}
|S(N)| \leq \sqrt{N} \quad \text{for all } N \geq 1
\end{equation}
with equality achieved at $N=1$ where $S(1) = 1/\sqrt{g_1} = 1$ (since $g_1 = p_2 - p_1 = 3 - 2 = 1$).

Moreover, we conjecture that for $N \geq 2$, the ratio satisfies the stronger bound:
\begin{equation}
\limsup_{N \to \infty} \frac{|S(N)|}{\sqrt{N}} < 1,
\end{equation}
and numerical evidence suggests $|S(N)|/\sqrt{N} \to 0$ as $N \to \infty$, though the decay is slow and non-monotonic.
Verification over $10^8$ primes shows the maximum ratio for $N \geq 2$ is approximately $0.577$ at $N=3$.
\end{conjecture}

\begin{figure}[H]
\centering
\includegraphics[width=0.8\textwidth]{fig_conjecture_1.png}
\caption{Oscillatory behavior of the Harmonic Gap Sum $S(N)$ showing bounded growth consistent with $|S(N)| \leq \sqrt{N}$. The alternating series exhibits complex cancellation patterns while remaining within the conjectured bound.}
\label{fig:conjecture1}
\end{figure}

\begin{table}[H]
\centering
\begin{tabular}{|c|r|c|}
\hline
$N$ & $S(N)$ & $|S(N)|/\sqrt{N}$ \\
\hline
1 & 1.000000 & 1.000000 \\
10 & 0.615355 & 0.194592 \\
100 & 2.842283 & 0.284228 \\
1\,000 & -2.832706 & 0.089578 \\
10\,000 & 19.584581 & 0.195846 \\
100\,000 & -23.699580 & 0.074945 \\
1\,000\,000 & -86.647599 & 0.086648 \\
10\,000\,000 & 133.520071 & 0.042223 \\
100\,000\,000 & -2508.831323 & 0.250883 \\
\hline
\end{tabular}
\caption{Computed values of $S(N)$ for increasing $N$. The sequence oscillates with sublinear growth in amplitude, maintaining $|S(N)|/\sqrt{N} \leq 1.0$ throughout the tested range up to $N = 100,000,000$.}
\label{tab:results}
\end{table}

\subsection{Heuristic Arguments}

\subsubsection{Why the Conjecture Might Be True}

Several heuristic arguments support the plausibility of Conjecture~\ref{conj:main}:

\begin{enumerate}
    \item \textbf{Average gap growth:} By the Prime Number Theorem, the average gap near $p_n$ 
    is approximately $\log p_n \sim \log n + \log\log n$. Thus, on average, 
    $1/\sqrt{g_n} \sim 1/\sqrt{\log n}$, which decreases slowly but steadily.
    
    \item \textbf{Alternating series behavior:} Even though $\{1/\sqrt{g_n}\}$ is not monotonically 
    decreasing (due to the erratic behavior of individual gaps), the alternating sum tends to 
    exhibit cancellation.
    
    \item \textbf{Statistical symmetry:} Empirical evidence suggests that there is no systematic 
    bias between odd-indexed and even-indexed prime gaps in terms of their average size, 
    leading to approximate cancellation in the alternating sum.
\end{enumerate}

\subsubsection{Why a Proof Is Difficult}

Despite the numerical evidence, proving Conjecture~\ref{conj:main} appears to be extremely challenging:

\begin{enumerate}
    \item \textbf{Individual gap unpredictability:} While we have good asymptotic estimates for 
    average gaps, predicting any individual gap $g_n$ is essentially impossible with current methods.
    
    \item \textbf{Connection to unsolved problems:} A proof would likely require resolving or 
    circumventing deep conjectures like the Twin Prime Conjecture (that $g_n = 2$ infinitely often) 
    or Cram\'er's conjecture on maximal gaps.
    
    \item \textbf{Alternating sum subtlety:} The convergence of alternating series typically requires 
    understanding the monotonicity or regularity of terms, which prime gaps conspicuously lack.
\end{enumerate}

\subsection{Numerical Evidence}

We have computed $S(N)$ for $N$ up to $10^8$ primes. The results strongly support 
Conjecture~\ref{conj:main}.

\subsubsection{Results Summary}

Table~\ref{tab:results} presents the computed values of $S(N)$ for various $N$ up to $N = 100,000,000$.

\subsubsection{Oscillation Analysis}

We observe that the partial sums $S(N)$ do not converge to a single value but 
instead exhibit a random-walk-like behavior with a sublinear envelope. 
If we define the variance of the sum:
\begin{equation}
V(N) = \frac{1}{N} \sum_{k=1}^{N} S(k)^2,
\end{equation}
our numerical experiments show that $V(N)$ grows approximately as $\log N$, 
suggesting that the sum is not merely bounded but has a slowly increasing 
variance, consistent with the $O(\sqrt{N})$ bound.

\subsection{Theoretical Analysis}

\subsubsection{Bounds on Prime Gaps}

Let us recall some known results on prime gaps that inform our analysis.

\begin{theorem}[Hoheisel, 1930]
For sufficiently large $n$, we have:
\begin{equation}
g_n < p_n^\theta
\end{equation}
for some $\theta < 1$. The best known unconditional result gives $\theta = 0.525$.
\end{theorem}

Under the Riemann Hypothesis, we would have $g_n = O(\sqrt{p_n} \log p_n)$.

\begin{proposition}
Assuming the Riemann Hypothesis, for the terms in our sum:
\begin{equation}
\frac{1}{\sqrt{g_n}} \geq \frac{c}{\sqrt[4]{p_n} \sqrt{\log p_n}}
\end{equation}
for some constant $c > 0$.
\end{proposition}

\subsubsection{Connection to Ces\`aro Summability}

Even if $S(N)$ does not converge in the classical sense, we might consider Ces\`aro means:
\begin{equation}
\sigma(N) = \frac{1}{N}\sum_{k=1}^{N} S(k).
\end{equation}

Our numerical experiments show that $\sigma(N)$ also exhibits large-scale oscillations, 
reaching $\sigma(100000) \approx -23.06$. This suggests that the ``average'' value of 
the sum is not zero, but rather drifts according to the local density of prime gaps.

\begin{remark}[Variance and $O(\sqrt{N})$ Bound]
The observed variance $V(N) \sim \log N$ is consistent with random walk behavior: 
if each term $(-1)^{n+1}/\sqrt{g_n}$ were independent with mean zero and variance $\sim 1/\log n$, 
the central limit theorem would predict $|S(N)| \sim \sqrt{N/\log N}$, which is $o(\sqrt{N})$.
The sublinear growth of variance explains why the ratio $|S(N)|/\sqrt{N}$ tends to decrease for large $N$.
\end{remark}

\subsection{Related Sums and Generalizations}

\subsubsection{Variants}

We briefly mention some natural variants of our sum:

\begin{enumerate}
    \item \textbf{Power variants:} $S_\alpha(N) = \sum_{n=1}^{N} \frac{(-1)^{n+1}}{g_n^\alpha}$ 
    for $\alpha \neq 1/2$. Numerical experiments suggest similar oscillatory behavior for $\alpha > 0$.
    
    \item \textbf{Weighted variants:} $S_w(N) = \sum_{n=1}^{N} \frac{(-1)^{n+1} w(n)}{\sqrt{g_n}}$ 
    for various weight functions $w(n)$.
    
    \item \textbf{Non-alternating sums:} $T(N) = \sum_{n=1}^{N} \frac{1}{\sqrt{g_n}}$ diverges, 
    approximately as $\sqrt{N} / \sqrt{\log N}$.
\end{enumerate}

\subsubsection{The Primal Harmonic Scaling}

If Conjecture~\ref{conj:main} is true, the scaling factor $\mathcal{H}$ in $|S(N)| \sim \mathcal{H} \sqrt{N}$ 
becomes a new mathematical quantity of interest. Its precise value and any closed-form 
representation remain mysterious.

\section{Conjecture 2: Golden-Phase Prime Spiral}

The second conjecture explores the behavior of complex exponential sums involving primes and the golden ratio.

\begin{conjecture}[Golden-Phase Prime Spiral]
\label{conj:golden}
Let $\phi = \frac{1+\sqrt{5}}{2}$ be the golden ratio. The complex series 
\begin{equation}
Z_N = \sum_{n=1}^{N} \frac{1}{p_n} e^{i \phi p_n}
\end{equation}
is bounded in the complex plane for all $N$. 

\textbf{Numerical observation:} The maximum $|Z_N|$ over $N \leq 10^8$ is achieved at $N=2$ with $|Z_2| \approx 0.5877$. 
We conjecture that $|Z_N| < 0.588$ for all $N \geq 1$, though we emphasize this bound is empirically derived 
and subject to refinement with extended computation.

The sequence $\{Z_N\}$ appears to densely fill a bounded region in the complex plane, exhibiting fractal-like boundary behavior.
The boundedness follows heuristically from the quasi-random distribution of phases $\phi p_n \pmod{2\pi}$ 
due to the irrationality of $\phi$, leading to destructive interference (cf.\ Weyl's equidistribution theorem \cite{weyl1916}).
\end{conjecture}

\begin{figure}[H]
\centering
\includegraphics[width=0.8\textwidth]{fig_conjecture_2.png}
\caption{Complex trajectory of the Golden-Phase Prime Spiral $Z_N = \sum_{n=1}^{N} \frac{1}{p_n} e^{i \phi p_n}$ in the complex plane. The spiral exhibits bounded behavior with maximum magnitude $|Z_2| = 0.5877$, creating intricate fractal-like patterns.}
\label{fig:conjecture2}
\end{figure}

\subsection{Analysis of the Golden-Phase Conjecture}

The irrationality of $\phi$ combined with the irregular spacing of primes creates 
a quasi-random phase distribution. Unlike sums with rational coefficients, which 
exhibit periodic behavior, the golden ratio introduces genuine aperiodicity.

The boundedness of this sum is remarkable given that the individual terms $1/p_n$ decay only logarithmically. 
The key insight is that the phases $\phi p_n$ are quasi-randomly distributed modulo $2\pi$ due to the irrationality of $\phi$, 
leading to destructive interference that prevents the sum from growing without bound.

This phenomenon is related to Weyl's equidistribution theorem \cite{weyl1916}, which states that 
for irrational $\alpha$, the sequence $\{\alpha n\}$ is equidistributed modulo 1.
The exponential sum techniques of Vinogradov \cite{vinogradov1954} for sums over primes 
provide further theoretical grounding, though our specific formulation with $1/p_n$ weights 
and golden-ratio phases appears to be new.

\textbf{Numerical precision:} For large $N$, phase errors may accumulate. 
We used double precision ($\approx 15$ significant digits), and the bound $|Z_N| < 0.588$ 
holds robustly across all tested values with margin $> 10^{-4}$.

\section{Conjecture 3: Square-Root Phase Boundedness}

Our third conjecture examines exponential sums with square-root phases.

\begin{conjecture}[Square-Root Phase Boundedness]
\label{conj:sqrt}
The exponential sum 
\begin{equation}
W_N = \sum_{n=1}^{N} \frac{\exp(2\pi i \sqrt{p_n})}{n}
\end{equation}
is bounded and stabilizes near a limit point as $N \to \infty$.

\textbf{Precise formulation:} We conjecture that $W_N$ oscillates within a bounded region that contracts as $N$ increases. 
Numerical computation over $10^8$ primes yields a candidate limit $w_\infty \approx -1.4929 + 0.3919i$, 
with $|W_N - w_\infty| < 0.05$ for all $N > 10^4$.
More conservatively, we assert: $|W_N| \leq 1.57$ for all $N \geq 1$.

\textbf{Note:} We do not claim classical convergence. Rather, the sum exhibits diminishing oscillations 
around $w_\infty$, consistent with conditional convergence behavior typical of oscillatory series.
\end{conjecture}

\begin{figure}[H]
\centering
\includegraphics[width=0.8\textwidth]{fig_conjecture_3.png}
\caption{Square-Root Phase Sum $W_N = \sum_{n=1}^{N} n^{-1} e^{2\pi i\sqrt{p_n}}$ in the complex plane. The sum converges toward a limit point $w_\infty \approx -1.4929 + 0.3919i$ (red star shows maximum), demonstrating the bounded nature of this exponential sum despite the infinite growth of the phase arguments.}
\label{fig:conjecture3}
\end{figure}

\subsection{Analysis of the Square-Root Phase Conjecture}

The square root function $\sqrt{p_n}$ grows sublinearly, causing the phase 
$2\pi\sqrt{p_n}$ to increase without bound but at a decreasing rate, leading 
to interference patterns that bound the sum.

The convergence behavior is particularly intriguing, as it suggests that despite the infinite nature of the sum, the contributions of later terms become negligible due to both the decreasing magnitude of $1/n$ and the quasi-random distribution of phases $\sqrt{p_n}$ modulo 1.

\section{Conjecture 4: Power-of-Two Digit-Sum Squares}

Our final conjecture bridges number theory and digital properties.

\begin{conjecture}[Power-of-Two Digit-Sum Squares]
\label{conj:digitsum}
Let $\mathcal{A}$ be the set of positive integers $n$ such that the digit sum of $2^n$ is a perfect square:
\begin{equation}
\mathcal{A} = \{n \in \N : S_{10}(2^n) = k^2 \text{ for some } k \in \N\},
\end{equation}
where $S_{10}(m)$ denotes the sum of decimal digits of $m$. 

We conjecture that $\mathcal{A}$ is infinite and its counting function grows as $\sqrt{N}$:
\begin{equation}
c_1 \sqrt{N} \leq |\mathcal{A} \cap [1,N]| \leq c_2 \sqrt{N} \quad \text{for all } N \geq 100,
\end{equation}
with numerical evidence suggesting $c_1 \approx 0.48$ and $c_2 \approx 1.2$.

The first few elements of $\mathcal{A}$ are: $1, 4, 7, 8, 9, 13, 16, 19, 22, 25, \ldots$ 
(corresponding to digit sums $2=1^2+1$... verification needed for perfect squares $1, 4, 9, 16, \ldots$).
\end{conjecture}

\begin{figure}[H]
\centering
\includegraphics[width=0.8\textwidth]{fig_conjecture_4.png}
\caption{Counting function for the set $A = \{n : S_{10}(2^n) \text{ is a perfect square}\}$ showing $\Theta(\sqrt{N})$ growth. The plot demonstrates the conjectured bounds $0.48\sqrt{N} \leq |A \cap [1,N]| \leq 1.2\sqrt{N}$ for the distribution of perfect square digit sums.}
\label{fig:conjecture4}
\end{figure}

\subsection{Analysis of the Digital Sum Conjecture}

The digit sum of powers of 2 exhibits pseudo-random behavior, making perfect square 
digit sums rare but, conjecturally, infinitely recurring with a specific growth rate.

This conjecture is particularly challenging because it relates multiplicative structure (powers of 2) to additive digital properties in base 10. The $\sqrt{N}$ growth rate suggests a deep underlying regularity despite the apparent randomness of digit sums.

\textbf{Heuristic reasoning:} The digit sum $S_{10}(2^n)$ is approximately $4.5 \cdot d(n)$ on average, where $d(n) = \lfloor n \log_{10} 2 \rfloor + 1 \approx 0.301n$ is the number of digits. Thus $S_{10}(2^n) \sim 1.35n$ on average. The probability that a random integer near $m$ is a perfect square is approximately $1/(2\sqrt{m})$. For $m \sim 1.35n$, this gives probability $\sim 1/\sqrt{n}$, leading to an expected count of $\sum_{k=1}^{N} 1/\sqrt{k} \sim \sqrt{N}$ elements in $\mathcal{A} \cap [1,N]$.

Recent work by Mauduit and Rivat \cite{mauduit2010} on digit sums of primes shows that sophisticated techniques can yield results on digit-sum distributions, though our setting (powers of 2, perfect square values) requires different methods.

\section{Discussion and Future Directions}

The four conjectures presented share common themes: oscillatory cancellation mechanisms, the emergence of structure from apparent chaos, and significant barriers to rigorous proof. 

\subsection{Barriers to Proof}

\begin{itemize}
\item \textbf{Conjecture 1 (Primal Gap Harmonic):} Requires understanding the fine structure of prime gaps beyond the Prime Number Theorem. A proof would likely need to establish that odd-indexed and even-indexed gaps have no systematic bias---a property that follows heuristically from the randomness of primes but is unproven.

\item \textbf{Conjecture 2 (Golden-Phase Spiral):} The boundedness relies on cancellation in exponential sums. While Vinogradov's methods \cite{vinogradov1954} handle many such sums, the $1/p_n$ weighting and irrational phase require novel techniques.

\item \textbf{Conjecture 3 (Square-Root Phase):} The sublinear growth of $\sqrt{p_n}$ creates slowly varying phases. Proving boundedness would require showing that the phase distribution leads to sufficient cancellation.

\item \textbf{Conjecture 4 (Digit-Sum Squares):} This conjecture bridges multiplicative structure (powers of 2) with additive digital properties---notoriously difficult to connect. Recent breakthroughs by Mauduit-Rivat \cite{mauduit2010} on digit sums of primes provide hope that such questions are tractable, but our setting differs significantly.
\end{itemize}

\subsection{Novelty Assessment}

All four conjectures appear genuinely novel based on extensive searches. We found no matches in OEIS, MathWorld, or recent arXiv preprints for the specific formulations presented. The alternating harmonic sum over prime gap square roots (Conjecture 1), the golden-ratio phase sums (Conjecture 2), the square-root prime phase exponential sum with harmonic weights (Conjecture 3), and the specific digit-sum perfect-square counting (Conjecture 4) have not appeared in the literature to our knowledge.

\subsection{Future Work}

\begin{enumerate}
\item \textbf{Probabilistic models:} Cram\'er's random model for primes could provide heuristic predictions for the asymptotic behavior of each conjecture.

\item \textbf{Conditional results:} It would be valuable to establish the conjectures assuming the Riemann Hypothesis or other standard conjectures.

\item \textbf{Generalizations:} Natural extensions include varying the exponent in $g_n^\alpha$, replacing $\phi$ with other irrational numbers, or considering digit sums in other bases.
\end{enumerate}

Each conjecture opens new research avenues bridging analytic number theory, exponential sums, and computational structure, demonstrating how numerical discovery can guide theoretical inquiry.

\section*{Declaration of competing interest}

The author declares that there is no conflict of interest.

\section*{Data availability}

The computational verification scripts and data are available from the author upon reasonable request.

\section*{Acknowledgments}

The author would like to thank China Mobile Research Institute (CMRI) for supporting fundamental research.

\begin{thebibliography}{99}

\bibitem{baker2001}
R.C.\ Baker, G.\ Harman, J.\ Pintz, The difference between consecutive primes, II,
Proc.\ Lond.\ Math.\ Soc.\ (3) 83 (2001) 532--562.

\bibitem{cramer1936}
H.\ Cram\'er, On the order of magnitude of the difference between consecutive prime numbers, 
Acta Arith.\ 2 (1936) 23--46.

\bibitem{goldston2009}
D.A.\ Goldston, J.\ Pintz, C.Y.\ Y{\i}ld{\i}r{\i}m, Primes in tuples I, 
Ann.\ of Math.\ (2) 170 (2009) 819--862.

\bibitem{granville1995}
A.\ Granville, Harald Cram\'er and the distribution of prime numbers, 
Scand.\ Actuar.\ J.\ 1995 (1995) 12--28.

\bibitem{mauduit2010}
C.\ Mauduit, J.\ Rivat, Sur un probl\`eme de Gelfond: la somme des chiffres des nombres premiers,
Ann.\ of Math.\ (2) 171 (2010) 1591--1646.

\bibitem{maynard2015}
J.\ Maynard, Small gaps between primes, 
Ann.\ of Math.\ (2) 181 (2015) 383--413.

\bibitem{polymath2014}
D.H.J.\ Polymath, Variants of the Selberg sieve, and bounded intervals containing many primes, 
Res.\ Math.\ Sci.\ 1 (2014) Art.\ 12.

\bibitem{soundararajan2007}
K.\ Soundararajan, Small gaps between prime numbers: the work of Goldston-Pintz-Y{\i}ld{\i}r{\i}m, 
Bull.\ Amer.\ Math.\ Soc.\ (N.S.) 44 (2007) 1--18.

\bibitem{vinogradov1954}
I.M.\ Vinogradov, The Method of Trigonometrical Sums in the Theory of Numbers,
Interscience Publishers, London, 1954.

\bibitem{weyl1916}
H.\ Weyl, \"Uber die Gleichverteilung von Zahlen mod.\ Eins,
Math.\ Ann.\ 77 (1916) 313--352.

\bibitem{zhang2014}
Y.\ Zhang, Bounded gaps between primes, 
Ann.\ of Math.\ (2) 179 (2014) 1121--1174.

\end{thebibliography}

\end{document}
