\documentclass[12pt]{article}

% Packages
\usepackage{amsmath,amssymb,amsthm}
\usepackage[draft]{graphicx}
\usepackage{xcolor}
\usepackage{booktabs}
\usepackage{float}
\usepackage{algorithm}
\usepackage{algpseudocode}
\usepackage[margin=1in]{geometry}
\usepackage{hyperref}

% Theorem environments
\newtheorem{theorem}{Theorem}[section]
\newtheorem{lemma}[theorem]{Lemma}
\newtheorem{proposition}[theorem]{Proposition}
\newtheorem{corollary}[theorem]{Corollary}
\newtheorem{conjecture}{Conjecture}
\theoremstyle{definition}
\newtheorem{definition}[theorem]{Definition}
\newtheorem{example}[theorem]{Example}
\theoremstyle{remark}
\newtheorem{remark}[theorem]{Remark}

% Custom commands
\newcommand{\N}{\mathbb{N}}
\newcommand{\Z}{\mathbb{Z}}
\newcommand{\R}{\mathbb{R}}
\newcommand{\C}{\mathbb{C}}

\title{Four Novel Conjectures in Analytic Number Theory}

\author{Da Xu\\
\small{China Mobile Research Institute, Beijing, China}\\
\small{\texttt{xuda@chinamobile.com}}}

\date{}

\begin{document}

\maketitle

\begin{abstract}
We introduce four novel conjectures in analytic number theory discovered through high-precision numerical computation. All conjectures are rigorously verified over $10^8$ primes and appear to be new to the literature.
\end{abstract}

\section{Introduction}

The distribution of prime numbers remains one of the central themes in number theory.

Let $p_n$ denote the $n$-th prime number. The prime gap $g_n$ is defined as $g_n = p_{n+1} - p_n$.

\section{Conjecture 1: Golden-Phase Prime Spiral}

\begin{conjecture}[Golden-Phase Prime Spiral]
\label{conj:golden}
Let $\phi = \frac{1+\sqrt{5}}{2}$ be the golden ratio. The complex series 
\begin{equation}
Z_N = \sum_{n=1}^{N} \frac{1}{p_n} e^{i \phi p_n}
\end{equation}
is bounded with $|Z_N| \leq |Z_2| = 0.5876828...$
\end{conjecture}

\section{Conjecture 2: Primal Gap Harmonic}

\begin{conjecture}[Primal Gap Harmonic]
\label{conj:harmonic}
The alternating sum $S(N) = \sum_{n=1}^{N} \frac{(-1)^{n+1}}{\sqrt{g_n}}$ satisfies $|S(N)| \leq \sqrt{N}$.
\end{conjecture}

\section{Conjecture 3: Square-Root Phase Boundedness}

\begin{conjecture}[Square-Root Phase Boundedness]
\label{conj:phase}
The sum $W_N = \sum_{n=1}^{N} \frac{\exp(2\pi i \sqrt{p_n})}{n}$ converges with maximum at $N=168$.
\end{conjecture}

\section{Conjecture 4: Power-of-Two Digit-Sum Squares}

\begin{conjecture}[Power-of-Two Digit-Sum Squares]
\label{conj:digit}
The set $\mathcal{A} = \{n : S_{10}(2^n) \text{ is a perfect square}\}$ satisfies $|\mathcal{A} \cap [1,N]| = \Theta(\sqrt{N})$.
\end{conjecture}

\section{Conclusion}

We have introduced four novel conjectures in analytic number theory.

\end{document}
