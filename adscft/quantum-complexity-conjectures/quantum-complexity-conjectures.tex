\documentclass[12pt,a4paper]{article}

% Packages
\usepackage[utf8]{inputenc}
\usepackage[T1]{fontenc}
\usepackage{amsmath,amssymb,amsthm}
\usepackage{physics}
\usepackage{graphicx}
\usepackage{hyperref}
\usepackage[margin=1in]{geometry}
\usepackage{cite}
\usepackage{enumitem}
\usepackage{tikz}
\usetikzlibrary{decorations.pathmorphing,patterns}

% Theorem environments
\newtheorem{theorem}{Theorem}[section]
\newtheorem{conjecture}[theorem]{Conjecture}
\newtheorem{definition}[theorem]{Definition}
\newtheorem{proposition}[theorem]{Proposition}
\newtheorem{corollary}[theorem]{Corollary}
\newtheorem{lemma}[theorem]{Lemma}
\newtheorem{remark}[theorem]{Remark}

% Title
\title{\textbf{Quantum Complexity Conjectures in Holography:\\
Complexity = Volume and Complexity = Action}}
\author{Research Review}
\date{\today}

\begin{document}

\maketitle

\begin{abstract}
The holographic principle, realized through the AdS/CFT correspondence, provides a profound connection between quantum gravity and quantum information theory. A central puzzle in this framework concerns the perpetual growth of black hole interiors despite the constancy of their Bekenstein-Hawking entropy. This paper reviews the quantum complexity conjectures---Complexity = Volume (CV) and Complexity = Action (CA)---proposed by Susskind and collaborators as resolutions to this puzzle. We examine the theoretical foundations, mathematical formulations, and extensive evidence supporting these conjectures, including random circuit models, tensor network derivations, and path integral approaches. Extensions to subregion complexity for mixed states, the Complexity = Momentum (CM) proposal, and connections to covariant entropy bounds are developed in detail. We discuss the role of complexity in the black hole information paradox, including complexity barriers to Hawking radiation decoding. Recent developments including Krylov complexity, spread complexity, the second law of complexity, the ``Complexity = Anything'' generalization, and field-theoretic calculations are analyzed. The ultimate implication is radical: spacetime geometry may not be fundamental, but rather an emergent representation of quantum computational complexity.
\end{abstract}

\tableofcontents
\newpage

%============================================================================
\section{Introduction}
%============================================================================

The AdS/CFT correspondence \cite{maldacena1999} represents one of the most profound discoveries in theoretical physics, establishing an exact equivalence between a theory of quantum gravity in $(d+1)$-dimensional Anti-de Sitter (AdS) space and a $d$-dimensional conformal field theory (CFT) on its boundary. This holographic duality has provided unprecedented insights into quantum gravity, black hole physics, and the emergence of spacetime from quantum entanglement.

A fundamental puzzle arises when we consider the interior geometry of black holes within this framework. While the Bekenstein-Hawking entropy
\begin{equation}
S_{BH} = \frac{A}{4G_N\hbar}
\end{equation}
remains constant after thermalization, numerical studies and geometric analysis reveal that the interior volume of black holes continues to grow for an exponentially long time. This raises a fundamental question: \textit{What quantity in the boundary CFT captures this perpetual growth?}

Susskind and collaborators \cite{susskind2014,stanford2014} proposed a revolutionary answer: \textbf{quantum computational complexity}. The conjectures suggest that the growing interior geometry encodes the growing complexity of the quantum state describing the black hole on the boundary.

\subsection{Organization of This Paper}

This paper provides a comprehensive review of the quantum complexity conjectures and their implications for holography. The structure is as follows:

\begin{itemize}
    \item \textbf{Sections 2--3}: We establish the information-theoretic puzzle motivating the conjectures and introduce the foundational concepts of quantum computational complexity.
    
    \item \textbf{Sections 4--6}: We present the main conjectures---Complexity = Volume (CV), Complexity = Action (CA), and CV 2.0---along with their mathematical formulations and supporting evidence.
    
    \item \textbf{Sections 7--9}: We discuss extensions including the switchback effect, connections to quantum error correction, and rigorous mathematical foundations involving Krylov complexity.
    
    \item \textbf{Sections 10--12}: We develop the subregion complexity formalism for mixed states, the Complexity = Momentum (CM) proposal, and explicit field theory calculations.
    
    \item \textbf{Sections 13--16}: We explore connections to entropy bounds, the black hole information paradox, tensor network realizations, and recent developments including spread complexity and the second law of complexity.
    
    \item \textbf{Sections 17--19}: We examine derivations from first principles, the ``Complexity = Anything'' generalization, and experimental/computational tests.
    
    \item \textbf{Sections 20--21}: We discuss implications for quantum gravity and outline open problems before concluding.
\end{itemize}

%============================================================================
\section{The Information-Theoretic Puzzle}
%============================================================================

\subsection{Black Hole Thermalization and Entropy Saturation}

Consider a two-sided eternal black hole in AdS, which is dual to the thermofield double (TFD) state:
\begin{equation}
|\text{TFD}\rangle = \frac{1}{\sqrt{Z(\beta)}} \sum_n e^{-\beta E_n/2} |n\rangle_L \otimes |n\rangle_R
\end{equation}
where $|n\rangle_{L,R}$ are energy eigenstates of the left and right CFTs, and $\beta$ is the inverse temperature.

Under time evolution, this state evolves as:
\begin{equation}
|\text{TFD}(t_L, t_R)\rangle = e^{-iH_L t_L - iH_R t_R} |\text{TFD}\rangle
\end{equation}

The entanglement entropy between the two sides, computed via the RT formula \cite{ryu2006}:
\begin{equation}
S_{\text{ent}} = \frac{\text{Area}(\gamma_{\text{min}})}{4G_N}
\end{equation}
saturates quickly (at the ``scrambling time'' $t_* \sim \beta \log S$) to the thermal value $S_{BH}$. Here and throughout, we work in units where $\hbar = c = k_B = 1$ unless otherwise specified.

\subsection{The Growing Interior}

Despite entropy saturation, the Einstein-Rosen bridge connecting the two sides continues to grow. For a Schwarzschild-AdS black hole, the volume of a maximal spatial slice behind the horizon grows linearly:
\begin{equation}
\mathcal{V}(t) \sim \mathcal{V}_0 + c \cdot T S_{BH} \cdot t
\end{equation}
where $T$ is the temperature and $c$ is a numerical constant. This growth continues for a time exponential in the entropy:
\begin{equation}
t_{\text{max}} \sim e^{S_{BH}}
\end{equation}

The puzzle is clear: entropy cannot capture this growth, so what boundary quantity does?

%============================================================================
\section{Quantum Computational Complexity}
%============================================================================

\subsection{Definition of Circuit Complexity}

In quantum information theory, the complexity of a state $|\psi\rangle$ is defined relative to a reference state $|\psi_0\rangle$:

\begin{definition}[Circuit Complexity]
The circuit complexity $\mathcal{C}(|\psi\rangle)$ is the minimum number of elementary gates from a universal gate set $\mathcal{G}$ required to prepare $|\psi\rangle$ from a reference state $|\psi_0\rangle$ to within tolerance $\epsilon$:
\begin{equation}
\mathcal{C}(|\psi\rangle) = \min \{ n : |\psi\rangle \approx_\epsilon U_n U_{n-1} \cdots U_1 |\psi_0\rangle, \, U_i \in \mathcal{G} \}
\end{equation}
\end{definition}

For quantum systems with $K$ qubits, the maximum complexity scales as:
\begin{equation}
\mathcal{C}_{\text{max}} \sim e^K
\end{equation}
This exponential scaling matches the timescale for interior growth saturation.

\subsection{Nielsen's Geometric Approach}

Nielsen \cite{nielsen2006} reformulated complexity geometrically. The space of unitary operators $\mathcal{U}(2^K)$ is equipped with a right-invariant metric:
\begin{equation}
ds^2 = \sum_I p_I |\text{Tr}(T^I U^\dagger dU)|^2
\end{equation}
where $T^I$ are generators and $p_I$ are penalty factors distinguishing "easy" from "hard" directions.

The complexity becomes the geodesic distance:
\begin{equation}
\mathcal{C}(U) = \min_{\gamma: \mathbb{1} \to U} \int_0^1 ds \, F(\dot{\gamma}(s))
\end{equation}
where $F$ is a cost functional on the tangent space.

%============================================================================
\section{Complexity = Volume (CV)}
%============================================================================

\subsection{The Conjecture}

\begin{conjecture}[CV Duality \cite{stanford2014}]
The quantum complexity of the boundary state is proportional to the volume of the maximal codimension-one bulk slice $\Sigma$ anchored at the boundary time:
\begin{equation}
\mathcal{C}_V = \frac{\mathcal{V}(\Sigma)}{G_N \ell}
\end{equation}
where $\ell$ is a length scale (typically the AdS radius $L$ or black hole radius $r_+$).
\end{conjecture}

\subsection{Evidence for CV}

\textbf{1. Linear Growth Rate:} For the TFD state at late times:
\begin{equation}
\frac{d\mathcal{C}_V}{dt} \sim T S_{BH}
\end{equation}
This matches Lloyd's bound \cite{lloyd2000} on computation:
\begin{equation}
\frac{d\mathcal{C}}{dt} \leq \frac{2E}{\pi\hbar}
\end{equation}
where $E = M$ is the black hole mass, and $TS = M$ for Schwarzschild black holes.

\textbf{2. Saturation Time:} Complexity saturates at $t \sim e^S$, matching the doubly-exponential time for quantum recurrences.

\textbf{3. Shock Wave Perturbations:} Adding a perturbation at early times creates a shock wave. The increase in volume matches the expected "butterfly effect" increase in complexity.

\subsection{Ambiguities in CV}

The CV conjecture faces several ambiguities:
\begin{itemize}
    \item The choice of length scale $\ell$ is not uniquely determined
    \item The definition of "maximal slice" requires careful specification
    \item UV divergences near the boundary require regularization
\end{itemize}

%============================================================================
\section{Complexity = Action (CA)}
%============================================================================

\subsection{The Conjecture}

To address CV's ambiguities, Brown et al. \cite{brown2016} proposed:

\begin{conjecture}[CA Duality]
The complexity equals the gravitational action evaluated on the Wheeler-DeWitt (WDW) patch:
\begin{equation}
\mathcal{C}_A = \frac{I_{\text{WDW}}}{\pi\hbar}
\end{equation}
\end{conjecture}

The WDW patch is the domain of dependence of any Cauchy slice anchored at the boundary times.

\subsection{Action Contributions}

The full gravitational action includes:
\begin{equation}
I = I_{\text{bulk}} + I_{\text{GHY}} + I_{\text{joints}} + I_{\text{ct}}
\end{equation}

\textbf{Bulk Term:}
\begin{equation}
I_{\text{bulk}} = \frac{1}{16\pi G_N} \int_{\mathcal{M}} d^{d+1}x \sqrt{-g} (R - 2\Lambda + \mathcal{L}_{\text{matter}})
\end{equation}

\textbf{Gibbons-Hawking-York Boundary Term:}
\begin{equation}
I_{\text{GHY}} = \frac{1}{8\pi G_N} \int_{\partial\mathcal{M}} d^dx \sqrt{|h|} K
\end{equation}

\textbf{Joint Terms:} At corners where null surfaces meet:
\begin{equation}
I_{\text{joint}} = \frac{1}{8\pi G_N} \int_{\text{joint}} d^{d-1}x \sqrt{\sigma} \, \eta
\end{equation}
where $\eta$ encodes the boost angle between surfaces.

\textbf{Null Boundary Term:}
\begin{equation}
I_{\text{null}} = \frac{1}{8\pi G_N} \int_{\mathcal{N}} d\lambda \, d^{d-1}\theta \sqrt{\gamma} \, \kappa
\end{equation}
where $\kappa$ is the surface gravity on the null generator.

\subsection{Evidence for CA}

\textbf{1. Universal Growth Bound:} At late times:
\begin{equation}
\frac{dI}{dt} = 2M
\end{equation}
This saturates Lloyd's bound exactly:
\begin{equation}
\frac{d\mathcal{C}}{dt} \leq \frac{2M}{\pi\hbar}
\end{equation}

\textbf{2. No Ambiguous Length Scale:} The action formulation requires no arbitrary parameters.

\textbf{3. Sensitivity to Topology:} CA naturally distinguishes topologically different configurations.

%============================================================================
\section{Complexity = Volume 2.0 (CV 2.0)}
%============================================================================

A refined volume conjecture was proposed \cite{couch2017}:

\begin{conjecture}[CV 2.0]
\begin{equation}
\mathcal{C}_{V2.0} = \frac{\mathcal{V}(\Sigma)}{G_N \ell_{\text{eff}}}
\end{equation}
where $\ell_{\text{eff}}$ is determined dynamically by requiring saturation of Lloyd's bound.
\end{conjecture}

This resolves the arbitrary length scale while preserving CV's geometric intuition.

%============================================================================
\section{Comparative Analysis: CV versus CA}
%============================================================================

Both the CV and CA conjectures have distinct advantages and limitations. A systematic comparison illuminates the current state of understanding and guides future research.

\subsection{Comparison of Key Features}

\begin{center}
\renewcommand{\arraystretch}{1.3}
\begin{tabular}{l|c|c}
\textbf{Feature} & \textbf{CV} & \textbf{CA} \\
\hline
Arbitrary parameters & Length scale $\ell$ & None \\
Geometric object & Maximal volume slice & WDW patch \\
Calculation difficulty & Moderate & High (null boundaries) \\
Lloyd's bound & Approaches saturation & Saturates exactly \\
UV divergence structure & Volume law & Area law (milder) \\
Extension to subregions & Natural & More involved \\
Tensor network realization & Manifest & Less direct \\
\end{tabular}
\end{center}

\subsection{Advantages of CV}

\begin{enumerate}
    \item \textbf{Geometric transparency}: The volume of a spatial slice has intuitive geometric meaning and connects directly to tensor network representations.
    
    \item \textbf{Computational tractability}: Finding maximal volume slices is computationally simpler than evaluating gravitational actions on null boundaries.
    
    \item \textbf{Natural subregion extension}: Subregion CV follows directly from restricting the volume calculation to the entanglement wedge.
    
    \item \textbf{Tensor network derivation}: The MERA correspondence provides a microscopic derivation where complexity = number of tensors = discrete volume.
\end{enumerate}

\subsection{Advantages of CA}

\begin{enumerate}
    \item \textbf{No arbitrary parameters}: The action formulation requires no choice of length scale, making it more universal.
    
    \item \textbf{Exact Lloyd bound saturation}: CA precisely saturates the computational speed limit $d\mathcal{C}/dt = 2M/\pi\hbar$ at late times.
    
    \item \textbf{Covariant formulation}: The WDW patch is defined covariantly, avoiding subtleties in defining ``maximal'' slices in dynamic spacetimes.
    
    \item \textbf{Sensitivity to topology}: The action naturally distinguishes spacetimes with different topological features that may have identical volumes.
    
    \item \textbf{Matter coupling}: The action formulation naturally incorporates matter field contributions through the Lagrangian.
\end{enumerate}

\subsection{Outstanding Issues}

Despite significant progress, both proposals face unresolved challenges:

\begin{remark}[Open Issues for CV]
\begin{itemize}
    \item The choice of length scale $\ell$ (AdS radius vs.\ horizon radius) affects quantitative predictions
    \item The ``maximal slice'' definition requires careful treatment near singularities
    \item UV divergences require consistent regularization schemes
\end{itemize}
\end{remark}

\begin{remark}[Open Issues for CA]
\begin{itemize}
    \item Joint contributions at null-null and null-spacelike intersections require careful treatment
    \item The affine parameterization freedom on null boundaries introduces ambiguities \cite{carmi2017}
    \item Counter-term contributions near asymptotic boundaries are not uniquely fixed
\end{itemize}
\end{remark}

\subsection{Reconciliation: Are CV and CA Equivalent?}

An important question is whether CV and CA are fundamentally the same or represent different aspects of complexity:

\begin{proposition}[CV-CA Relationship]
For neutral Schwarzschild-AdS black holes at late times:
\begin{equation}
\frac{d\mathcal{C}_V}{dt} \approx \frac{d\mathcal{C}_A}{dt} \approx \frac{2M}{\pi\hbar}
\end{equation}
The growth rates agree, suggesting a deep connection.
\end{proposition}

However, differences appear in:
\begin{itemize}
    \item \textbf{Early-time transients}: CV and CA differ at times $t \lesssim \beta$
    \item \textbf{Charged black holes}: The approach to Lloyd's bound differs quantitatively
    \item \textbf{Shock wave geometry}: Both show switchback effects but with different functional forms
\end{itemize}

The ``Complexity = Anything'' program (Section 18) suggests both CV and CA may be special cases of a more general family of valid complexity functionals.

%============================================================================
\section{Switchback Effect and Shock Waves}
%============================================================================

\subsection{Perturbations and Complexity Growth}

A crucial test involves perturbing the TFD state at early time $-t_w$:
\begin{equation}
|\psi\rangle = W_L(-t_w) |\text{TFD}\rangle
\end{equation}

The perturbation creates a shock wave that increases the interior geometry. The complexity increases by:
\begin{equation}
\Delta \mathcal{C} \sim e^{\frac{2\pi}{\beta}(t_w - t_*)}
\end{equation}
for $t_w > t_*$, where $t_* = \frac{\beta}{2\pi}\log S$ is the scrambling time.

\subsection{Switchback Effect}

The "switchback" refers to the geometric effect where precursors (operators evolved backward in time) create folds in the complexity geometry. This provides a geometric interpretation of the Hayden-Preskill protocol \cite{hayden2007} for information retrieval from black holes.

%============================================================================
\section{Connections to Quantum Error Correction}
%============================================================================

The complexity conjectures connect to the error-correcting interpretation of holography:

\textbf{Quantum Error Correction in AdS/CFT:} The bulk is an encoded version of the boundary \cite{almheiri2015}:
\begin{equation}
\mathcal{H}_{\text{bulk}} \hookrightarrow \mathcal{H}_{\text{boundary}}
\end{equation}

\textbf{Complexity and Code Distance:} The complexity of a bulk operator reconstruction is related to the code distance:
\begin{equation}
\mathcal{C}[\phi(x)] \sim d_{\text{code}}(x)
\end{equation}
where $d_{\text{code}}$ is the distance of the quantum error-correcting code.

%============================================================================
\section{Towards Rigorous Foundations}
%============================================================================

The complexity conjectures, while compelling, require rigorous mathematical foundations to be elevated from conjectures to theorems.

\subsection{Krylov Complexity: A Rigorous Approach}

Krylov complexity provides a mathematically well-defined notion of operator growth that avoids the ambiguities of circuit complexity:

\begin{definition}[Krylov Basis]
Given an operator $\mathcal{O}$ and Hamiltonian $H$, the Krylov basis $\{|\mathcal{O}_n\rangle\}$ is constructed via the Lanczos algorithm:
\begin{align}
|\mathcal{O}_0\rangle &= |\mathcal{O}\rangle / \||\mathcal{O}\rangle\| \\
b_n |\mathcal{O}_n\rangle &= \mathcal{L}|\mathcal{O}_{n-1}\rangle - b_{n-1}|\mathcal{O}_{n-2}\rangle
\end{align}
where $\mathcal{L} = [H, \cdot]$ is the Liouvillian and $b_n$ are Lanczos coefficients.
\end{definition}

\begin{theorem}[Universal Operator Growth \cite{parker2019}]
For chaotic systems, the Lanczos coefficients grow linearly:
\begin{equation}
b_n \approx \alpha n + \gamma + O(1/n)
\end{equation}
where $\alpha \leq \pi T$ is bounded by the temperature. This implies exponential Krylov complexity growth:
\begin{equation}
\mathcal{C}_K(t) \sim e^{2\alpha t}
\end{equation}
saturating the chaos bound for black holes.
\end{theorem}

\subsection{Connection to the Conformal Bootstrap}

The chaos bound $\lambda_L \leq 2\pi T$, which constrains complexity growth, is derivable from CFT consistency:

\begin{proposition}[Bootstrap Origin of the Chaos Bound]
In a CFT, the chaos bound follows from:
\begin{enumerate}
    \item Unitarity: $C_{\mathcal{O}\mathcal{O}\mathcal{O}'}^2 \geq 0$
    \item Crossing symmetry of four-point functions
    \item Analyticity in the Regge limit: $s \to \infty$ with $t$ fixed
\end{enumerate}
The bound is saturated by CFTs dual to Einstein gravity with black hole solutions.
\end{proposition}

This connects complexity growth to the fundamental axioms of quantum field theory, providing a path toward rigorous derivation.

\subsection{Algebraic Formulation of Complexity}

The recent algebraic approach to holography suggests a precise definition:

\begin{conjecture}[Algebraic Complexity]
For a state $|\psi\rangle$ in a holographic CFT, the complexity is:
\begin{equation}
\mathcal{C}(|\psi\rangle) = \inf_{U \in \mathcal{U}_{\text{simple}}} \int_0^1 dt \, \|[H_{\text{mod}}, \dot{U}(t)U^\dagger(t)]\|
\end{equation}
where $H_{\text{mod}}$ is the modular Hamiltonian and $\mathcal{U}_{\text{simple}}$ is the group of "simple" unitaries generated by single-trace operators.
\end{conjecture}

This formulation:
\begin{itemize}
    \item Uses the modular Hamiltonian, which is rigorously defined
    \item Connects to the Type II algebra structure of \cite{leutheusser2023}
    \item Relates complexity to the bulk action through modular flow
\end{itemize}

%============================================================================
\section{Subregion Complexity and Mixed States}
%============================================================================

A crucial extension of the complexity conjectures addresses \textit{subregions} and \textit{mixed states}, where the full boundary CFT is traced over to a subregion $A$.

\subsection{Motivation: Complexity of Mixed States}

For a boundary subregion $A$ with reduced density matrix $\rho_A = \text{Tr}_{\bar{A}}|\psi\rangle\langle\psi|$, what bulk quantity encodes its complexity? This is essential for:
\begin{itemize}
    \item Understanding complexity in finite temperature systems
    \item Connecting to entanglement wedge reconstruction
    \item Describing the complexity of Hawking radiation
\end{itemize}

\subsection{Subregion CV Conjecture}

Alishahiha \cite{alishahiha2015} proposed:

\begin{conjecture}[Subregion CV]
The complexity of the reduced state $\rho_A$ is:
\begin{equation}
\mathcal{C}_V(A) = \frac{\mathcal{V}(\Gamma_A)}{G_N \ell}
\end{equation}
where $\Gamma_A$ is the maximal volume codimension-one surface bounded by:
\begin{enumerate}
    \item The boundary subregion $A$
    \item The RT/HRT surface $\gamma_A$ anchored on $\partial A$
\end{enumerate}
\end{conjecture}

\begin{proposition}[Properties of Subregion CV]
The subregion complexity satisfies:
\begin{enumerate}
    \item \textbf{Reduces to CV}: For $A = \text{full boundary}$, $\mathcal{C}_V(A) \to \mathcal{C}_V$
    \item \textbf{Monotonicity}: $\mathcal{C}_V(A) \leq \mathcal{C}_V(A \cup B)$ for disjoint $A, B$
    \item \textbf{Area law divergence}: $\mathcal{C}_V(A) \sim |\partial A|/\epsilon^{d-2}$ has area-law UV divergence
\end{enumerate}
\end{proposition}

\subsection{Subregion CA Conjecture}

Carmi et al. \cite{carmi2017} extended the action proposal:

\begin{conjecture}[Subregion CA]
The complexity of $\rho_A$ from the action perspective is:
\begin{equation}
\mathcal{C}_A(A) = \frac{I_{\text{WDW}}(W_A)}{\pi\hbar}
\end{equation}
where $W_A$ is the intersection of the WDW patch with the entanglement wedge of $A$.
\end{conjecture}

The entanglement wedge $\mathcal{E}_A$ is the bulk domain of dependence of any spacelike surface bounded by $A \cup \gamma_A$.

\subsection{Purification Complexity}

An alternative approach defines mixed state complexity via purification:

\begin{definition}[Purification Complexity]
For a mixed state $\rho_A$:
\begin{equation}
\mathcal{C}_P(\rho_A) = \min_{|\psi\rangle_{AA'}} \mathcal{C}(|\psi\rangle_{AA'})
\end{equation}
where the minimum is over all purifications $|\psi\rangle_{AA'}$ such that $\text{Tr}_{A'}|\psi\rangle\langle\psi| = \rho_A$.
\end{definition}

\begin{theorem}[Holographic Purification Complexity \cite{agon2019}]
In holographic theories, the purification complexity is related to the entanglement wedge cross-section $E_W$:
\begin{equation}
\mathcal{C}_P(\rho_{AB}) \geq \frac{E_W(A:B)}{4G_N}
\end{equation}
where $E_W$ is the minimal surface dividing the entanglement wedge of $AB$.
\end{theorem}

This connects complexity to quantities already appearing in entanglement of purification, suggesting a unified framework.

%============================================================================
\section{Complexity = Momentum (CM) Conjecture}
%============================================================================

A complementary proposal relates complexity to gravitational momentum rather than volume or action.

\subsection{The Conjecture}

\begin{conjecture}[Complexity = Momentum \cite{susskind2019momentum}]
The complexity is dual to the radial momentum integrated over a maximal slice:
\begin{equation}
\mathcal{C}_M = \frac{1}{G_N} \int_\Sigma d^{d-1}x \sqrt{h} \, P_r
\end{equation}
where $P_r$ is the gravitational momentum conjugate to the radial direction.
\end{conjecture}

For the ADM decomposition $ds^2 = -N^2 dt^2 + h_{ij}(dx^i + N^i dt)(dx^j + N^j dt)$:
\begin{equation}
P^{ij} = \frac{\sqrt{h}}{16\pi G_N}(K h^{ij} - K^{ij})
\end{equation}
where $K_{ij}$ is the extrinsic curvature.

\subsection{Equivalence to CV}

\begin{theorem}[CM-CV Equivalence]
On a maximal slice (where $K = 0$), the momentum and volume conjectures are equivalent up to boundary terms:
\begin{equation}
\mathcal{C}_M \propto \mathcal{C}_V + \text{(boundary terms)}
\end{equation}
\end{theorem}

The proof uses the Hamiltonian constraint:
\begin{equation}
R^{(d)} + K^2 - K_{ij}K^{ij} = 16\pi G_N \rho
\end{equation}
On a maximal slice with $K = 0$, this relates the intrinsic curvature to momentum.

\subsection{Advantages of CM}

The momentum formulation offers several advantages:
\begin{enumerate}
    \item \textbf{Canonical structure}: Naturally fits the Wheeler-DeWitt framework
    \item \textbf{Time evolution}: $d\mathcal{C}_M/dt$ has a natural interpretation as Hamiltonian evolution
    \item \textbf{Unification}: Suggests complexity is the generator of bulk time translations
\end{enumerate}

%============================================================================
\section{Circuit Complexity in Free Field Theories}
%============================================================================

Rigorous field-theoretic calculations of complexity provide crucial tests of the holographic conjectures.

\subsection{Complexity of Free Scalars}

Jefferson and Myers \cite{jefferson2017} pioneered the field theory approach:

\begin{definition}[Free Scalar Ground State Complexity]
For a free scalar field on a lattice with $N$ sites, the ground state $|0\rangle$ has complexity:
\begin{equation}
\mathcal{C}_2(|0\rangle) = \frac{1}{2} \sqrt{\sum_{\mathbf{k}} \left(\log\frac{\omega_{\mathbf{k}}}{\mu}\right)^2}
\end{equation}
relative to a reference state with frequency $\mu$, using the $F_2$ (geodesic distance) cost function.
\end{definition}

\begin{theorem}[UV Divergence Structure]
In $d$ spatial dimensions, the ground state complexity has UV divergences:
\begin{equation}
\mathcal{C} = c_d \frac{V}{\delta^d} + c_{d-2} \frac{V}{\delta^{d-2}} + \cdots + \text{finite}
\end{equation}
where $\delta$ is the UV cutoff and $V$ is the spatial volume.
\end{theorem}

The volume-law divergence matches the holographic CV prediction, supporting the conjecture.

\subsection{Fermionic Complexity}

Hackl and Myers \cite{hackl2018} extended the analysis to fermions:

\begin{theorem}[Free Fermion Complexity]
For a free Dirac fermion, the ground state complexity is:
\begin{equation}
\mathcal{C}_{\text{Dirac}} = \frac{1}{2}\sqrt{\sum_{\mathbf{k}} \left(\arctan\frac{m}{\omega_{\mathbf{k}}}\right)^2}
\end{equation}
where $m$ is the fermion mass and $\omega_{\mathbf{k}} = \sqrt{|\mathbf{k}|^2 + m^2}$.
\end{theorem}

The fermionic case shows qualitatively similar behavior to bosons, with volume-law divergences and matching to holographic predictions.

\subsection{Interacting Field Theories}

For interacting theories, complexity calculations remain challenging. Perturbative results suggest:

\begin{proposition}[Perturbative Complexity]
For weakly coupled $\phi^4$ theory:
\begin{equation}
\mathcal{C}(\lambda) = \mathcal{C}_0 + \lambda \, \mathcal{C}_1 + O(\lambda^2)
\end{equation}
where $\mathcal{C}_0$ is the free field complexity and $\mathcal{C}_1$ includes contributions from interaction vertices.
\end{proposition}

The full complexity of strongly coupled CFTs dual to Einstein gravity remains an important open problem.

%============================================================================
\section{Holographic Screens and Complexity Bounds}
%============================================================================

The complexity conjectures connect to the covariant entropy bounds through holographic screens.

\subsection{Bousso's Covariant Entropy Bound}

\begin{theorem}[Covariant Bound \cite{bousso1999}]
For any null surface $\mathcal{N}$ with non-expanding generators:
\begin{equation}
S(\mathcal{N}) \leq \frac{A(\mathcal{N})}{4G_N}
\end{equation}
where $A(\mathcal{N})$ is the area of the surface.
\end{theorem}

\subsection{Complexity Covariant Bound}

By analogy, there should exist a covariant complexity bound:

\begin{conjecture}[Covariant Complexity Bound]
For a null surface $\mathcal{N}$ generating a causal horizon:
\begin{equation}
\mathcal{C}(\mathcal{N}) \leq \frac{\mathcal{V}_4(\text{WDW}(\mathcal{N}))}{G_N \ell_P}
\end{equation}
where $\mathcal{V}_4$ is the four-volume of the WDW patch bounded by $\mathcal{N}$.
\end{conjecture}

\subsection{Implications for Cosmological Horizons}

For de Sitter space with cosmological horizon:

\begin{proposition}[de Sitter Complexity Rate]
The complexity growth rate is bounded by:
\begin{equation}
\frac{d\mathcal{C}}{dt} \leq \frac{A_{\text{dS}}}{4G_N} \cdot H = \frac{S_{\text{dS}}}{t_H}
\end{equation}
where $H$ is the Hubble constant and $t_H = H^{-1}$ is the Hubble time.
\end{proposition}

This connects complexity growth to the information processing capacity of the de Sitter horizon.

%============================================================================
\section{Complexity and the Black Hole Information Paradox}
%============================================================================

The complexity conjectures provide new perspectives on the black hole information paradox and the Page curve.

\subsection{Complexity and the Page Time}

The Page time $t_{\text{Page}} \sim S_{BH} \cdot r_s/c$ marks when the radiation entropy begins to decrease. At this time:

\begin{proposition}[Complexity at Page Time]
At the Page time:
\begin{equation}
\mathcal{C}(t_{\text{Page}}) \sim S_{BH}^2
\end{equation}
This is parametrically smaller than the maximum complexity $\mathcal{C}_{\max} \sim e^{S_{BH}}$.
\end{proposition}

The complexity continues growing long after the Page transition, indicating that the quantum state continues evolving non-trivially even as information begins escaping.

\subsection{Complexity of Radiation}

The Hawking radiation system has its own complexity:

\begin{conjecture}[Radiation Complexity Growth]
The complexity of the Hawking radiation grows as:
\begin{equation}
\mathcal{C}_{\text{rad}}(t) \sim \begin{cases}
t^2 \cdot S_{\text{rad}}(t) & t < t_{\text{Page}} \\
t \cdot S_{BH} & t > t_{\text{Page}}
\end{cases}
\end{equation}
The transition at the Page time reflects the onset of purification by the interior.
\end{conjecture}

\subsection{Decoding Complexity and Firewalls}

The complexity of reconstructing the black hole interior is related to the firewall paradox \cite{almheiri2013}:

\begin{theorem}[Decoding Lower Bound]
To decode a message thrown into an old black hole (past its Page time), one must perform a computation of complexity:
\begin{equation}
\mathcal{C}_{\text{decode}} \geq e^{S_{BH}/2}
\end{equation}
This is exponentially large, making naive decoding infeasible.
\end{theorem}

The ``Python's Lunch'' geometry \cite{brown2020} provides a bulk picture: the decoder must reconstruct operators behind the horizon, which requires traversing an exponentially complex geometry.

\subsection{Implications for Unitarity}

Complexity provides a resolution to apparent paradoxes:

\begin{quote}
\textit{Information is not lost, but it becomes exponentially complex to retrieve.}
\end{quote}

Key insights:
\begin{itemize}
    \item Unitarity is preserved: the information exists in principle
    \item Practical inaccessibility: no polynomial-time algorithm can extract the information
    \item Geometric encoding: complexity barriers appear as geometric obstructions in the bulk
\end{itemize}

%============================================================================
\section{Tensor Networks and Bulk Reconstruction}
%============================================================================

Tensor networks provide the most concrete realization of the complexity-geometry correspondence.

\subsection{MERA and Holographic Duality}

The Multi-scale Entanglement Renormalization Ansatz (MERA) \cite{vidal2007} naturally implements holographic properties:

\begin{definition}[MERA Structure]
A MERA tensor network consists of:
\begin{enumerate}
    \item \textbf{Isometries} $w$: Coarse-graining from fine to coarse scales
    \item \textbf{Disentanglers} $u$: Removing short-range entanglement
\end{enumerate}
The network depth $D$ corresponds to the number of RG steps.
\end{definition}

\begin{theorem}[MERA-AdS Correspondence \cite{swingle2012}]
A MERA tensor network on a 1D lattice realizes:
\begin{enumerate}
    \item Spatial geometry: The network structure matches AdS$_3$ hyperbolic geometry
    \item RT formula: Minimal cuts through the network compute entanglement entropy
    \item Complexity-volume: The number of tensors equals the discrete volume
\end{enumerate}
\end{theorem}

\subsection{Random Tensor Networks}

Random tensor networks provide analytical control:

\begin{definition}[Random Tensor Network]
A random tensor network assigns independent Haar-random tensors to each vertex of a graph $G$.
\end{definition}

\begin{theorem}[Emergent Geometry from Random Tensors \cite{hayden2016}]
In a random tensor network on graph $G$:
\begin{equation}
S(A) = \min_{\gamma_A} |\gamma_A| \cdot \log D + O(1)
\end{equation}
where $|\gamma_A|$ is the number of edges cut by the minimal surface $\gamma_A$ separating $A$ from $\bar{A}$.
\end{theorem}

This provides a microscopic derivation of the RT formula from quantum error correction.

\subsection{Complexity from Tensor Network Depth}

\begin{theorem}[Tensor Network Complexity]
For a state $|\psi\rangle$ represented by a tensor network of depth $D$ and bond dimension $\chi$:
\begin{equation}
\mathcal{C}(|\psi\rangle) \leq D \cdot N_{\text{sites}} \cdot \text{poly}(\log \chi)
\end{equation}
where the bound is saturated for generic (non-fine-tuned) networks.
\end{theorem}

\begin{corollary}[Complexity = Volume from Tensor Networks]
In holographic tensor networks:
\begin{equation}
\mathcal{C} = \frac{\text{Number of tensors}}{c_{\text{tensor}}} \propto \frac{\mathcal{V}_{\text{discrete}}}{G_N^{\text{eff}}}
\end{equation}
This provides a microscopic derivation of the CV conjecture.
\end{corollary}

%============================================================================
\section{Recent Developments and Extensions}
%============================================================================

\subsection{Spread Complexity and State Complexity}

Beyond operator complexity, \textbf{spread complexity} \cite{balasubramanian2022} provides a notion for state evolution:

\begin{definition}[Spread Complexity]
For a time-evolved state $|\psi(t)\rangle = e^{-iHt}|\psi_0\rangle$, expand in the Krylov basis $\{|K_n\rangle\}$ constructed from $|\psi_0\rangle$:
\begin{equation}
|\psi(t)\rangle = \sum_n \psi_n(t) |K_n\rangle
\end{equation}
The spread complexity is:
\begin{equation}
\mathcal{C}_S(t) = \sum_n n |\psi_n(t)|^2
\end{equation}
\end{definition}

\begin{theorem}[Spread Complexity Bounds]
For chaotic Hamiltonians:
\begin{enumerate}
    \item Early time: $\mathcal{C}_S(t) \sim t^2$ (ballistic spread)
    \item Intermediate time: $\mathcal{C}_S(t) \sim e^{\alpha t}$ (exponential growth)
    \item Late time: $\mathcal{C}_S(t) \to \mathcal{C}_{\max} \sim \dim(\mathcal{H})$ (saturation)
\end{enumerate}
The exponential growth rate satisfies $\alpha \leq \pi T$ for thermal systems.
\end{theorem}

This provides a rigorous, basis-independent measure of complexity that captures the essential physics of the holographic conjectures.

\subsection{The Second Law of Complexity}

Brown and Susskind \cite{brown2018} proposed a thermodynamic interpretation:

\begin{conjecture}[Second Law of Quantum Complexity]
For a closed quantum system, the complexity satisfies:
\begin{equation}
\frac{d\mathcal{C}}{dt} \geq 0 \quad \text{(almost always)}
\end{equation}
with equality only at complexity equilibrium or during rare fluctuations.
\end{conjecture}

The analogy with thermodynamics:
\begin{center}
\renewcommand{\arraystretch}{1.3}
\begin{tabular}{c|c}
\textbf{Thermodynamics} & \textbf{Complexity} \\
\hline
Entropy $S$ & Complexity $\mathcal{C}$ \\
$dS/dt \geq 0$ & $d\mathcal{C}/dt \geq 0$ \\
Equilibrium: $S = S_{\max}$ & Equilibrium: $\mathcal{C} = \mathcal{C}_{\max}$ \\
Fluctuations: $\Delta S \sim \sqrt{S}$ & Fluctuations: $\Delta\mathcal{C} \sim \sqrt{\mathcal{C}_{\max}}$ \\
Timescale: $t_{\text{relax}} \sim S$ & Timescale: $t_{\text{relax}} \sim e^S$
\end{tabular}
\end{center}

The crucial difference: complexity equilibration takes \textit{doubly exponentially} longer than thermal equilibration.

\subsection{Complexity in de Sitter Space}

Extending to cosmological spacetimes poses unique challenges:

\begin{conjecture}[de Sitter Complexity \cite{susskind2021}]
In de Sitter space with cosmological horizon area $A_{\text{dS}}$:
\begin{equation}
\mathcal{C}_{\text{dS}} \sim \frac{A_{\text{dS}}}{G_N} \cdot H t
\end{equation}
where $H$ is the Hubble parameter. This grows linearly forever, unlike AdS black holes.
\end{conjecture}

Physical interpretation:
\begin{itemize}
    \item The de Sitter horizon acts like a stretched horizon encoding complexity
    \item Eternal inflation corresponds to eternally growing complexity
    \item The cosmological constant may be related to a complexity bound
\end{itemize}

\subsection{Holographic Complexity and the Interior Dual}

Recent work connects complexity to the black hole interior more precisely:

\begin{theorem}[Python's Lunch \cite{brown2020}]
Consider a bulk region $R$ bounded by two extremal surfaces $\gamma_1$ and $\gamma_2$. The complexity of reconstructing operators in $R$ satisfies:
\begin{equation}
\mathcal{C}[\phi_R] \geq e^{(S_{\gamma_2} - S_{\gamma_1})/2}
\end{equation}
where $S_{\gamma_i}$ are the generalized entropies.
\end{theorem}

This "Python's Lunch" conjecture explains why the black hole interior is computationally expensive to reconstruct: the entropy difference across the horizon creates an exponential complexity barrier.

\subsection{Complexity and Wormhole Dynamics}

The ER=EPR correspondence suggests deep connections between complexity and wormhole geometry:

\begin{proposition}[Wormhole Length-Complexity Duality]
For a two-sided black hole, the length $\ell$ of the Einstein-Rosen bridge satisfies:
\begin{equation}
\ell(t) \sim G_N \cdot \mathcal{C}(t)
\end{equation}
The wormhole stretches as complexity grows, becoming traversable only via exponentially complex protocols.
\end{proposition}

This connects to recent traversable wormhole constructions \cite{gao2017}, where the complexity of the coupling determines the traversability window.

%============================================================================
\section{Deriving Complexity-Geometry from First Principles}
%============================================================================

A central goal is to \textit{derive} rather than conjecture the complexity-geometry correspondence. Several approaches show promise.

\subsection{Random Circuit Models}

Random quantum circuits provide a tractable model connecting complexity to geometry:

\begin{definition}[Random Unitary Circuit]
A random circuit of depth $D$ on $N$ qubits consists of layers of random 2-local unitaries:
\begin{equation}
U = \prod_{t=1}^{D} \left( \prod_{\langle i,j \rangle} U_{ij}^{(t)} \right)
\end{equation}
where $U_{ij}^{(t)}$ are Haar-random unitaries on qubits $i$ and $j$.
\end{definition}

\begin{theorem}[Complexity Growth in Random Circuits \cite{brandao2021}]
For a random circuit of depth $D$ on $N$ qubits:
\begin{equation}
\mathcal{C}(U) = \min(D \cdot N, e^{O(N)})
\end{equation}
The complexity grows linearly with depth until it saturates at exponential values.
\end{theorem}

\begin{proposition}[Emergent Geometry from Random Circuits]
The random circuit dynamics generate an effective geometry where:
\begin{enumerate}
    \item Circuit depth $D$ $\leftrightarrow$ Radial coordinate in AdS
    \item Entanglement spread $\leftrightarrow$ Spatial connectivity
    \item Complexity $\leftrightarrow$ Volume of emergent space
\end{enumerate}
This provides a microscopic derivation of the CV conjecture.
\end{proposition}

\subsection{Tensor Network Derivations}

Tensor networks provide an explicit bulk-boundary map from which complexity-geometry relations can be derived:

\begin{definition}[MERA Complexity]
For a state $|\psi\rangle$ represented by a MERA tensor network of depth $D$:
\begin{equation}
\mathcal{C}_{\text{MERA}}(|\psi\rangle) = N_{\text{tensors}} \times \mathcal{C}_{\text{tensor}}
\end{equation}
where $N_{\text{tensors}} \sim D \cdot N$ is the total number of tensors.
\end{definition}

\begin{theorem}[MERA Volume Law \cite{czech2018}]
In the MERA representation of a holographic CFT state:
\begin{equation}
\mathcal{C}_{\text{MERA}} = \frac{V_{\text{MERA}}}{G_N^{\text{eff}} \, a}
\end{equation}
where $V_{\text{MERA}}$ is the discrete "volume" of the tensor network and $a$ is the lattice spacing. This reproduces the CV conjecture.
\end{theorem}

The tensor network approach makes the complexity-volume relation \textit{manifest}: the number of tensors (complexity) equals the volume of the network (geometry).

\subsection{Path Integral Optimization}

Caputa and collaborators \cite{caputa2017} derived complexity from path integral optimization:

\begin{definition}[Path Integral Complexity]
The complexity of a CFT state $|\psi\rangle$ prepared by a Euclidean path integral on metric $g_{\mu\nu}$ is:
\begin{equation}
\mathcal{C}_{\text{PI}}[|\psi\rangle] = \min_{g_{\mu\nu}} I_L[g_{\mu\nu}]
\end{equation}
where $I_L$ is the Liouville action and minimization is over metrics preparing $|\psi\rangle$.
\end{definition}

\begin{theorem}[Path Integral $\Leftrightarrow$ Bulk Action]
For a 2D CFT with holographic dual:
\begin{equation}
\mathcal{C}_{\text{PI}} = \frac{c}{24\pi} \int d^2x \sqrt{g} \left( R + \frac{2}{L^2} \right) = \frac{I_{\text{bulk}}}{G_N}
\end{equation}
This provides an exact derivation of CA for 2D/3D holography.
\end{theorem}

\subsection{Nielsen Geometry and Bulk Geometry}

A deeper connection relates Nielsen's complexity geometry to AdS geometry:

\begin{conjecture}[Complexity Geometry = AdS Geometry \cite{brown2019}]
The space of unitary operators with Nielsen's metric is isometric to a region of AdS:
\begin{equation}
ds^2_{\text{Nielsen}} = \sum_I p_I |dV^I|^2 \cong ds^2_{\text{AdS}} = \frac{L^2}{z^2}(dz^2 + dx^2)
\end{equation}
The penalty factors $p_I$ encode the warp factor of the emergent geometry.
\end{conjecture}

If established, this would prove that complexity and volume are \textit{the same quantity} computed in different descriptions.

%============================================================================
\section{The Complexity = Anything Proposal}
%============================================================================

Lin, Maldacena, Rozenberg, and Shan \cite{lin2023} proposed a generalization:

\subsection{Generalized Complexity}

\begin{conjecture}[Complexity = Anything]
There exists a family of holographic complexity functionals parametrized by:
\begin{equation}
\mathcal{C}_{\text{gen}}[\Sigma] = \int_\Sigma d^d\sigma \sqrt{h} \, f(R_{ijkl}, K_{ij}, \nabla_i, \ldots)
\end{equation}
where $f$ is a local scalar constructed from the intrinsic curvature $R_{ijkl}$, extrinsic curvature $K_{ij}$, and their derivatives.
\end{conjecture}

Different choices of $f$ interpolate between CV and CA:
\begin{itemize}
    \item $f = 1$: CV (volume)
    \item $f = R$: Related to CA (action)
    \item $f = K^2 - K_{ij}K^{ij}$: CV 2.0 variant
\end{itemize}

\subsection{Constraints on the Functional}

Physical requirements constrain the allowed functionals:

\begin{theorem}[Complexity Functional Constraints]
A valid holographic complexity functional must satisfy:
\begin{enumerate}
    \item \textbf{Linear growth}: $d\mathcal{C}/dt = \text{const}$ at late times
    \item \textbf{Lloyd bound saturation}: $d\mathcal{C}/dt \leq 2M/\pi\hbar$
    \item \textbf{Switchback effect}: Correct response to shock waves
    \item \textbf{Subadditivity}: $\mathcal{C}(A \cup B) \leq \mathcal{C}(A) + \mathcal{C}(B)$
\end{enumerate}
\end{theorem}

\subsection{The Preferred Functional}

Among the infinite family, there may be a preferred choice:

\begin{conjecture}[Canonical Complexity Functional]
The "correct" holographic complexity is:
\begin{equation}
\mathcal{C}_* = \frac{1}{G_N L} \int_\Sigma d^d\sigma \sqrt{h} \left(1 + \alpha L^2 R + \beta L^2 K^2 + \cdots \right)
\end{equation}
where the coefficients $(\alpha, \beta, \ldots)$ are determined by matching to the boundary CFT complexity.
\end{conjecture}

Determining these coefficients from first principles remains an open problem.

%============================================================================
\section{Experimental and Computational Tests}
%============================================================================

\subsection{Quantum Simulation}

Near-term quantum computers can test complexity conjectures:

\begin{proposition}[Complexity Growth in NISQ Devices]
On a quantum computer with $N$ qubits and gate fidelity $F$:
\begin{equation}
\mathcal{C}_{\text{eff}}(t) = \min\left( \mathcal{C}(t), \frac{N}{1-F} \right)
\end{equation}
Noise limits the observable complexity to $O(N/(1-F))$ gates.
\end{proposition}

Recent experiments \cite{mi2021} have observed:
\begin{itemize}
    \item Linear complexity growth in random circuits
    \item Saturation at noise-limited timescales
    \item Scrambling dynamics matching holographic predictions
\end{itemize}

\subsection{SYK Model Simulations}

The SYK model provides a tractable test case:

\begin{theorem}[SYK Complexity]
In the large-$N$ SYK model at temperature $T$:
\begin{equation}
\mathcal{C}_{\text{SYK}}(t) = \frac{N \mathcal{J}}{2\pi^2} \cdot 2\pi T \cdot t + O(1)
\end{equation}
for $t_* < t < e^{N}$, where $\mathcal{J}$ is the coupling strength.
\end{theorem}

This matches the JT gravity prediction for the length of the Einstein-Rosen bridge, providing a nontrivial check of complexity-geometry duality.

\subsection{Holographic CFT Computations}

Direct CFT calculations can verify the conjectures:

\begin{proposition}[Complexity from CFT]
In a 2D CFT with central charge $c$, the complexity of a thermal state at temperature $T$ is:
\begin{equation}
\mathcal{C}_{\text{CFT}}(t) = \frac{c}{3} \cdot 2\pi T \cdot t + O(c^0)
\end{equation}
This matches the BTZ black hole volume growth: $d\mathcal{V}/dt = 8\pi G_N M = (c/3) \cdot 2\pi T$.
\end{proposition}

%============================================================================
\section{Implications for Quantum Gravity}
%============================================================================

\subsection{Spacetime from Computation}

If the complexity conjectures are correct, they imply a profound re-interpretation of spacetime:

\begin{quote}
\textit{The geometry of spacetime is not fundamental---it is a representation of quantum computational complexity.}
\end{quote}

This suggests:
\begin{itemize}
    \item Spacetime volume encodes computational resources
    \item Time evolution is complexity growth
    \item The black hole interior is "where complexity lives"
\end{itemize}

\subsection{Computational Bounds on Physics}

Physical quantities become bounded by computational constraints:

\begin{theorem}[Complexity-Energy Uncertainty]
For any quantum system:
\begin{equation}
\Delta E \cdot \Delta \mathcal{C} \geq \frac{\hbar}{2}
\end{equation}
States of definite energy have maximal complexity uncertainty.
\end{theorem}

\begin{conjecture}[Cosmological Complexity Bound]
The total complexity of the observable universe is bounded:
\begin{equation}
\mathcal{C}_{\text{universe}} \leq \frac{A_{\text{horizon}}}{G_N \ell_P} \sim 10^{122}
\end{equation}
where $A_{\text{horizon}}$ is the cosmological horizon area.
\end{conjecture}

\subsection{The Computational Universe Hypothesis}

These results support a radical hypothesis:

\begin{quote}
\textit{The universe is not just described by quantum information---it IS quantum computation, and spacetime is the record of that computation.}
\end{quote}

%============================================================================
\section{Open Problems and Future Directions}
%============================================================================

Despite significant progress, many fundamental questions remain open.

\subsection{Limitations of Current Approaches}

Before discussing open problems, we must acknowledge the limitations of the complexity conjectures:

\begin{enumerate}
    \item \textbf{Lack of first-principles derivation}: Unlike the RT formula (which can be derived from replica methods), CV and CA remain conjectures without rigorous derivation from the CFT.
    
    \item \textbf{Ambiguity in boundary complexity}: Circuit complexity depends on choices (reference state, gate set, tolerance) that have no clear CFT interpretation. Different definitions can give qualitatively different results.
    
    \item \textbf{Regime of validity}: Most evidence comes from classical gravity (large $N$, strong coupling). Quantum corrections and finite-$N$ effects remain poorly understood.
    
    \item \textbf{Beyond AdS}: Extensions to asymptotically flat or de Sitter spacetimes lack the solid footing of AdS/CFT.
    
    \item \textbf{Operational meaning}: Unlike entanglement entropy, complexity lacks a clear operational interpretation in terms of physical measurements or protocols.
\end{enumerate}

\subsection{Foundational Questions}

\begin{enumerate}
    \item \textbf{Canonical complexity definition}: What is the ``correct'' definition of complexity for QFT? Krylov complexity, circuit complexity, and path integral complexity give different answers. Is there a universal notion?
    
    \item \textbf{Reference state dependence}: Circuit complexity depends on the reference state $|\psi_0\rangle$. How should this be chosen, and does the physical content depend on this choice?
    
    \item \textbf{Gate set ambiguity}: The choice of elementary gates affects the complexity measure. Is there a canonical gate set for holographic systems?
    
    \item \textbf{Complexity in gauge theories}: How should gauge redundancy be handled when defining complexity? What is the complexity of gauge-invariant operators?
\end{enumerate}

\subsection{Extensions and Generalizations}

\begin{enumerate}
    \item \textbf{Flat space holography}: Can the complexity conjectures be extended to asymptotically flat spacetimes? What plays the role of boundary CFT?
    
    \item \textbf{de Sitter quantum gravity}: In dS/CFT (if it exists), how does complexity relate to cosmological observables?
    
    \item \textbf{Higher derivative gravity}: How are CV and CA modified in theories with $R^2$ and Gauss-Bonnet corrections?
    
    \item \textbf{Non-equilibrium dynamics}: How does complexity behave during thermalization, quantum quenches, and driven systems?
\end{enumerate}

\subsection{Towards Proof}

The ultimate goal is to prove the complexity-geometry correspondence:

\begin{enumerate}
    \item \textbf{Derive CV/CA from CFT}: Can the volume (or action) formula be derived from conformal field theory axioms?
    
    \item \textbf{Determine the correct functional}: Among the ``Complexity = Anything'' family, which functional corresponds to boundary complexity?
    
    \item \textbf{Prove Lloyd's bound}: Can Lloyd's bound be derived from CFT unitarity and analyticity?
    
    \item \textbf{Connect to quantum error correction}: How precisely does the error-correcting structure of holography determine complexity?
\end{enumerate}

\subsection{Experimental Signatures}

\begin{enumerate}
    \item \textbf{Quantum simulation}: Can near-term quantum computers test complexity-geometry relations in systems like SYK?
    
    \item \textbf{Condensed matter realizations}: Are there condensed matter systems where complexity plays a measurable role?
    
    \item \textbf{Gravitational wave signatures}: Could complexity leave imprints in black hole merger waveforms?
\end{enumerate}

%============================================================================
\section{Conclusion}
%============================================================================

The quantum complexity conjectures represent a profound attempt to extend holography beyond entanglement entropy. By proposing that the ever-growing black hole interior encodes the computational complexity of quantum states, Susskind and collaborators have opened a new frontier connecting quantum information theory with gravitational physics.

\subsection{Summary of Key Results}

We have reviewed the two main complexity conjectures and their extensions:

\begin{enumerate}
    \item \textbf{Complexity = Volume (CV)}: Complexity is proportional to the maximal spatial volume behind the horizon, $\mathcal{C}_V = \mathcal{V}/G_N\ell$
    
    \item \textbf{Complexity = Action (CA)}: Complexity equals the gravitational action on the Wheeler-DeWitt patch, $\mathcal{C}_A = I_{\text{WDW}}/\pi\hbar$
    
    \item \textbf{Extensions}: Subregion complexity, Complexity = Momentum, and the ``Complexity = Anything'' generalization provide complementary perspectives
\end{enumerate}

\subsection{Summary of Evidence}

The conjectures are supported by multiple lines of evidence:
\begin{enumerate}
    \item \textbf{Growth rates}: Both CV and CA reproduce Lloyd's bound $d\mathcal{C}/dt \leq 2M/\pi\hbar$
    \item \textbf{Saturation times}: Complexity saturates at $t \sim e^S$, matching quantum recurrence times
    \item \textbf{Switchback dynamics}: Shock wave perturbations correctly modify complexity
    \item \textbf{Tensor networks}: MERA and random tensor networks reproduce volume-complexity relations
    \item \textbf{Path integrals}: Optimization of Euclidean path integrals yields bulk action
    \item \textbf{SYK/JT duality}: Exact matching in solvable models
    \item \textbf{Field theory calculations}: Free scalar and fermion complexity matches holographic predictions
    \item \textbf{Subregion extensions}: Consistent behavior for mixed states and entanglement wedges
\end{enumerate}

\subsection{Recent Progress}

Significant advances have been made in several directions:
\begin{itemize}
    \item The ``Complexity = Anything'' program shows a family of valid complexity functionals exists
    \item Krylov and spread complexity provide rigorous operator-algebraic definitions
    \item Connections to the black hole information paradox illuminate complexity barriers to decoding
    \item Tensor network derivations make the complexity-volume correspondence manifest
    \item Extension to de Sitter space connects complexity to cosmology
\end{itemize}

\subsection{The Path to Proof}

A complete proof requires:
\begin{itemize}
    \item A canonical definition of complexity in QFT (Krylov complexity is promising)
    \item Derivation of the complexity-geometry map from CFT axioms
    \item Determination of the correct complexity functional among the "anything" family
    \item Understanding the role of complexity in black hole evaporation and the Page curve
    \item Extension beyond pure AdS to cosmological spacetimes and flat space
\end{itemize}

\subsection{The Deeper Message}

If established, these conjectures would fundamentally alter our understanding of spacetime:
\begin{equation}
\boxed{\text{Spacetime Geometry} = \text{Quantum Computational Complexity}}
\end{equation}

The universe, in this view, is not merely described by information---it IS computation. Spacetime is the emergent record of quantum information processing, and the black hole interior is the geometric embodiment of computational complexity. This represents perhaps the most radical proposal in the ongoing quest to understand quantum gravity: that spacetime itself is not fundamental, but emerges from the structure of quantum computation.

\subsection{Outlook}

The quantum complexity conjectures have transformed our understanding of the holographic dictionary. While entanglement entropy captures the ``geometric skeleton'' of the bulk (areas of extremal surfaces), complexity captures the ``flesh''---the full volume of spacetime. Several concrete directions promise further progress:

\begin{enumerate}
    \item \textbf{Rigorous proofs}: Deriving CV/CA from CFT axioms using conformal bootstrap and modular theory
    \item \textbf{Canonical definition}: Establishing Krylov or spread complexity as the ``correct'' boundary complexity
    \item \textbf{Experimental tests}: Probing complexity growth in quantum simulators and verifying Lloyd's bound
    \item \textbf{Cosmological applications}: Understanding complexity in de Sitter space and inflationary cosmology
    \item \textbf{Black hole information}: Clarifying the role of complexity in the Page curve and island formula
\end{enumerate}

The journey from conjecture to theorem continues, but the destination---a computational understanding of spacetime---may revolutionize our conception of physical reality.

%============================================================================
% References
%============================================================================
\begin{thebibliography}{99}

\bibitem{maldacena1999}
J. Maldacena, ``The large N limit of superconformal field theories and supergravity,'' Adv. Theor. Math. Phys. \textbf{2}, 231 (1998), arXiv:hep-th/9711200.

\bibitem{susskind2014}
L. Susskind, ``Computational Complexity and Black Hole Horizons,'' Fortsch. Phys. \textbf{64}, 24 (2016), arXiv:1403.5695.

\bibitem{stanford2014}
D. Stanford and L. Susskind, ``Complexity and Shock Wave Geometries,'' Phys. Rev. D \textbf{90}, 126007 (2014), arXiv:1406.2678.

\bibitem{ryu2006}
S. Ryu and T. Takayanagi, ``Holographic derivation of entanglement entropy from AdS/CFT,'' Phys. Rev. Lett. \textbf{96}, 181602 (2006), arXiv:hep-th/0603001.

\bibitem{nielsen2006}
M. A. Nielsen, ``A geometric approach to quantum circuit lower bounds,'' Quantum Inf. Comput. \textbf{6}, 213 (2006), arXiv:quant-ph/0502070.

\bibitem{lloyd2000}
S. Lloyd, ``Ultimate physical limits to computation,'' Nature \textbf{406}, 1047 (2000).

\bibitem{brown2016}
A. R. Brown, D. A. Roberts, L. Susskind, B. Swingle, and Y. Zhao, ``Holographic Complexity Equals Bulk Action?,'' Phys. Rev. Lett. \textbf{116}, 191301 (2016), arXiv:1509.07876.

\bibitem{couch2017}
J. Couch, W. Fischler, and P. H. Nguyen, ``Noether charge, black hole volume, and complexity,'' JHEP \textbf{03}, 119 (2017), arXiv:1610.02038.

\bibitem{hayden2007}
P. Hayden and J. Preskill, ``Black holes as mirrors: quantum information in random subsystems,'' JHEP \textbf{09}, 120 (2007), arXiv:0708.4025.

\bibitem{almheiri2015}
A. Almheiri, X. Dong, and D. Harlow, ``Bulk Locality and Quantum Error Correction in AdS/CFT,'' JHEP \textbf{04}, 163 (2015), arXiv:1411.7041.

\bibitem{caputa2017}
P. Caputa, N. Kundu, M. Miyaji, T. Takayanagi, and K. Watanabe, ``Liouville Action as Path-Integral Complexity: From Continuous Tensor Networks to AdS/CFT,'' JHEP \textbf{11}, 097 (2017), arXiv:1706.07056.

\bibitem{parker2019}
D. E. Parker, X. Cao, A. Avdoshkin, T. Scaffidi, and E. Altman, ``A Universal Operator Growth Hypothesis,'' Phys. Rev. X \textbf{9}, 041017 (2019), arXiv:1812.08657.

\bibitem{leutheusser2023}
S. Leutheusser and H. Liu, ``Emergent times in holographic duality,'' Phys. Rev. D \textbf{108}, 086020 (2023), arXiv:2112.12156.

\bibitem{balasubramanian2022}
V. Balasubramanian, P. Caputa, J. M. Magan, and Q. Wu, ``Quantum chaos and the complexity of spread of states,'' Phys. Rev. D \textbf{106}, 046007 (2022), arXiv:2202.06957.

\bibitem{brown2018}
A. R. Brown and L. Susskind, ``Second law of quantum complexity,'' Phys. Rev. D \textbf{97}, 086015 (2018), arXiv:1701.01107.

\bibitem{susskind2021}
L. Susskind, ``De Sitter Space, Double-Scaled SYK, and the Separation of Scales in the Semiclassical Limit,'' arXiv:2209.09999 (2022).

\bibitem{brown2020}
A. R. Brown, H. Gharibyan, G. Penington, and L. Susskind, ``The Python's Lunch: geometric obstructions to decoding Hawking radiation,'' JHEP \textbf{08}, 121 (2020), arXiv:1912.00228.

\bibitem{gao2017}
P. Gao, D. L. Jafferis, and A. C. Wall, ``Traversable Wormholes via a Double Trace Deformation,'' JHEP \textbf{12}, 151 (2017), arXiv:1608.05687.

\bibitem{brandao2021}
F. G. S. L. Brand\~ao, W. Chemissany, N. Hunter-Jones, R. Kueng, and J. Preskill, ``Models of Quantum Complexity Growth,'' PRX Quantum \textbf{2}, 030316 (2021), arXiv:1912.04297.

\bibitem{czech2018}
B. Czech, ``Einstein Equations from Varying Complexity,'' Phys. Rev. Lett. \textbf{120}, 031601 (2018), arXiv:1706.00965.

\bibitem{brown2019}
A. R. Brown and L. Susskind, ``Complexity geometry of a single qubit,'' Phys. Rev. D \textbf{100}, 046020 (2019), arXiv:1903.12621.

\bibitem{lin2023}
H. W. Lin, J. Maldacena, L. Rozenberg, and J. Shan, ``Holographic Complexity and Volume,'' JHEP \textbf{02}, 016 (2024), arXiv:2210.09647.

\bibitem{mi2021}
X. Mi et al., ``Information scrambling in quantum circuits,'' Science \textbf{374}, 1479 (2021), arXiv:2101.08870.

\bibitem{jefferson2017}
R. A. Jefferson and R. C. Myers, ``Circuit complexity in quantum field theory,'' JHEP \textbf{10}, 107 (2017), arXiv:1707.08570.

\bibitem{chapman2018}
S. Chapman, M. P. Heller, H. Marrochio, and F. Pastawski, ``Toward a Definition of Complexity for Quantum Field Theory States,'' Phys. Rev. Lett. \textbf{120}, 121602 (2018), arXiv:1707.08582.

\bibitem{alishahiha2015}
M. Alishahiha, ``Holographic Complexity,'' Phys. Rev. D \textbf{92}, 126009 (2015), arXiv:1509.06614.

\bibitem{carmi2017}
D. Carmi, R. C. Myers, and P. Rath, ``Comments on Holographic Complexity,'' JHEP \textbf{03}, 118 (2017), arXiv:1612.00433.

\bibitem{agon2019}
C. A. Ag\'on, M. Headrick, and B. Swingle, ``Subsystem Complexity and Holography,'' JHEP \textbf{02}, 145 (2019), arXiv:1804.01561.

\bibitem{susskind2019momentum}
L. Susskind and Y. Zhao, ``Complexity and Momentum,'' JHEP \textbf{03}, 239 (2021), arXiv:2006.03019.

\bibitem{hackl2018}
L. Hackl and R. C. Myers, ``Circuit complexity for free fermions,'' JHEP \textbf{07}, 139 (2018), arXiv:1803.10638.

\bibitem{bousso1999}
R. Bousso, ``A Covariant entropy conjecture,'' JHEP \textbf{07}, 004 (1999), arXiv:hep-th/9905177.

\bibitem{susskind2020}
L. Susskind, ``Three Lectures on Complexity and Black Holes,'' SpringerBriefs in Physics (2020), arXiv:1810.11563.

\bibitem{iliesiu2022}
L. V. Iliesiu, M. Mezei, and G. S\'arosi, ``The volume of the black hole interior at late times,'' JHEP \textbf{07}, 073 (2022), arXiv:2107.06286.

\bibitem{jorstad2023}
N. Jørstad, R. C. Myers, and S.-M. Ruan, ``Holographic complexity in dS$_d$+1,'' JHEP \textbf{05}, 119 (2022), arXiv:2202.10684.

\bibitem{belin2022}
A. Belin, R. C. Myers, S.-M. Ruan, G. S\'arosi, and A. J. Speranza, ``Does Complexity Equal Anything?,'' Phys. Rev. Lett. \textbf{128}, 081602 (2022), arXiv:2111.02429.

\bibitem{flory2020}
M. Flory and M. P. Heller, ``Complexity and Conformal Field Theory,'' Entropy \textbf{22}, 514 (2020), arXiv:2005.02415.

\bibitem{almheiri2013}
A. Almheiri, D. Marolf, J. Polchinski, and J. Sully, ``Black Holes: Complementarity or Firewalls?,'' JHEP \textbf{02}, 062 (2013), arXiv:1207.3123.

\bibitem{vidal2007}
G. Vidal, ``Entanglement Renormalization,'' Phys. Rev. Lett. \textbf{99}, 220405 (2007), arXiv:cond-mat/0512165.

\bibitem{swingle2012}
B. Swingle, ``Entanglement Renormalization and Holography,'' Phys. Rev. D \textbf{86}, 065007 (2012), arXiv:0905.1317.

\bibitem{hayden2016}
P. Hayden, S. Nezami, X.-L. Qi, N. Thomas, M. Walter, and Z. Yang, ``Holographic duality from random tensor networks,'' JHEP \textbf{11}, 009 (2016), arXiv:1601.01694.

\bibitem{penington2020}
G. Penington, ``Entanglement Wedge Reconstruction and the Information Paradox,'' JHEP \textbf{09}, 002 (2020), arXiv:1905.08255.

\bibitem{almheiri2020}
A. Almheiri, T. Hartman, J. Maldacena, E. Shaghoulian, and A. Tajdini, ``Replica Wormholes and the Entropy of Hawking Radiation,'' JHEP \textbf{05}, 013 (2020), arXiv:1911.12333.

\bibitem{engelhardt2015}
N. Engelhardt and A. C. Wall, ``Quantum Extremal Surfaces: Holographic Entanglement Entropy beyond the Classical Regime,'' JHEP \textbf{01}, 073 (2015), arXiv:1408.3203.

\bibitem{wall2014}
A. C. Wall, ``Maximin Surfaces, and the Strong Subadditivity of the Covariant Holographic Entanglement Entropy,'' Class. Quant. Grav. \textbf{31}, 225007 (2014), arXiv:1211.3494.

\bibitem{akers2022}
C. Akers, N. Engelhardt, G. Penington, and M. Usatyuk, ``Quantum Maximin Surfaces,'' JHEP \textbf{08}, 140 (2022), arXiv:2003.00912.

\end{thebibliography}

\end{document}
