%------
% This is a template file for articles to appear
% in one of the following journals:
%
% AIHPC = Annales de l'Institut Henri Poincaré. Analyse Non Linéaire
% AIHPD = Annales de l'Institut Henri Poincaré D
% CMH = Commentarii Mathematici Helvetici
% DM = Documenta Mathematica
% EMSS = EMS Surveys in Mathematical Sciences
% GGD = Groups, Geometry, and Dynamics
% IFB = Interfaces and Free Boundaries
% JCA = Journal of Combinatorial Algebra
% JFG = Journal of Fractal Geometry
% JNCG = Journal of Noncommutative Geometry
% JST = Journal of Spectral Theory
% MSL Mathematical Statistics and Learning
% PM = Portugaliae Mathematica
% QT = Quantum Topology
% ZAA = Zeitschrift für Analysis und ihre Anwendungen
%
%------
% Before you edit this file, please read
% Guidelines-Journals.pdf
%------
\documentclass{article}
\usepackage[
  journal=JST,  %% Journal of Spectral Theory - appropriate for spectral analysis content
  lang=british  %% Change to `american' if you use American English.
]{ems-journal}

%------
% Include here your personal symbol definitions
% and macros as well as any extra LaTeX packages
% you need. Do not include any commands/packages
% that alter the layout of the page, e.g. height/width.
%------
% Do not include packages that are already loaded:
%   amsthm
%   amsmath
%   amssymb
%   enumitem
%   geometry
%   caption
%   graphicx
%   hyperref
%   fontenc
%   inputenc
% as well as:
%   array, babel, booktabs, cite, float, footmisc, kvoptions,
%   multicol, nag, newtxmath, newtxtext, pdf14, pdftexcmds,
%   ragged2e, upref, url, xcolor, xpatch, zref-base
%------

% Additional packages not already loaded
% MODIFICATION: Removed unused algorithm, algorithmic packages
\usepackage{microtype} % Improves typography
\usepackage{cleveref}
\usepackage{mathtools}

% REMOVED (caused conflict / unnecessary with class): \DeclareUnicodeCharacter{00A0}{~}

% (4.5) Allow display breaks for long equations
\allowdisplaybreaks

% To include the section number in the equation numbering:
\numberwithin{equation}{section}

% ------------------- Theorem Environments ----------------------
\newtheorem{theorem}{Theorem}[section]
\newtheorem{lemma}[theorem]{Lemma}
\newtheorem{proposition}[theorem]{Proposition}
\newtheorem{corollary}[theorem]{Corollary}
\theoremstyle{definition}
\newtheorem{definition}[theorem]{Definition}
\newtheorem{assumption}[theorem]{Assumption} % Define Assumption environment
\newtheorem{example}[theorem]{Example}
\theoremstyle{remark}
\newtheorem{remark}[theorem]{Remark}

% ---------------- cleveref configuration ----------------------
\crefname{theorem}{Theorem}{Theorems}
\crefname{lemma}{Lemma}{Lemmas}
\crefname{section}{Section}{Sections}
\crefname{table}{Tables}{Tables}
\crefname{assumption}{Assumption}{Assumptions}
\creflabelformat{equation}{(#2#1#3)}
\crefname{equation}{}{}

% -------------------- Custom Commands --------------------------
\newcommand{\cG}{\mathcal{G}}
\newcommand{\cH}{\mathcal{H}}
\DeclareMathOperator{\diam}{diam}
\DeclareMathOperator{\tr}{tr}
\newcommand{\cE}{\mathcal{E}}
\newcommand{\LL}{\mathcal{L}}
\newcommand{\R}{\mathbb{R}}
\newcommand{\Z}{\mathbb{Z}}
\newcommand{\N}{\mathbb{N}}
\newcommand{\Prob}{\mathbb{P}}
\newcommand{\E}{\mathbb{E}}

\begin{document}

\title{Sharp Spectral Zeta Asymptotics on Graphs of Quadratic Growth}

\emsauthor{1}{
  \givenname{Da}
  \surname{Xu}
  \mrid{}
  \orcid{}}{D.~Xu}

\Emsaffil{1}{
  \department{}
  \organisation{China Mobile Research Institute}
  \rorid{}
  \address{}
  \zip{}
  \city{Beijing}
  \country{P.R.~China}
  \affemail{xudayj@chinamobile.com}}

\classification{05C81}[60J10, 39A12, 31C20, 58J50]
\keywords{Spectral zeta function, graph Laplacian, heat kernel asymptotics, quadratic volume growth, Poincaré inequality, intrinsic ultracontractivity, random walk homogenisation}

\begin{abstract}
We investigate the spectral properties of the Dirichlet Laplacian on large finite metric balls within \emph{irregular} infinite graphs of quadratic volume growth.
We consider an exhaustion $G_n=B_{R_n}(x_0)$ and the spectral zeta value
$Z_n(1)=\tr(\LL_n^{-1})$ of the killed generator $\LL_n$.

We establish a sharp asymptotic law under the assumptions that the graph satisfies uniform quadratic volume growth (VG(2)) and a Poincaré inequality (PI). These analytic–geometric hypotheses imply large-scale regularity. We further assume a standard quantitative homogenisation property (a uniform local central limit theorem with rate), which holds in the principal example classes, and implies the existence of a global heat-kernel constant $\cG>0$ (independent of $x$) such that the lazy simple random walk (LSRW) satisfies
\[
  p_t(x,x)\;\sim\;\frac{\cG}{t}\qquad(t\to\infty).
\]

Our main theorem establishes the sharp asymptotic
\[
  Z_n(1)\;=\;\cG\,N_n\log N_n\;+\; O(N_n),\qquad N_n:=|V(G_n)|\xrightarrow[n\to\infty]{}\infty.
\]
This implies a relative error of $O(1/\log N_n)$, with constants depending only on the structural parameters of $G$ (the VG(2) and PI constants and $\Delta$).
This result is valid far beyond the class of homogeneous lattices.
For $\mathbb Z^2$, this yields the constant identification $\cG = 2/\pi$, providing a new "$\pi$-free'' limit formula.
Our techniques highlight the robustness of spectral asymptotics under homogenisation in this critical, recurrent setting.
\end{abstract}

\maketitle
% ===============================================================
\section{Introduction}

The relationship between the geometry of a space and the spectrum of its associated Laplacian is a fundamental area of study. In Riemannian geometry, Weyl's law and the Minakshisundaram--Pleijel heat trace expansion provide deep connections between volume, curvature, and eigenvalue asymptotics. In the discrete setting, analogous investigations explore how the combinatorial and geometric structure of a graph influences the spectrum of the graph Laplacian (see \cite{Chung97}).

This paper focuses on the spectral zeta function on graphs. For a finite graph $H$, we consider the Laplacian associated with the lazy simple random walk (see \Cref{sec:prelim}). With Laplacian eigenvalues $0 < \lambda_1 \leq \lambda_2 \leq \cdots$, the spectral zeta function is defined as $\zeta_H(s) = \sum_k \lambda_k^{-s}$. We are interested in the value at $s=1$, $Z_H(1) := \zeta_H(1)$, which corresponds to the trace of the inverse Laplacian (the Green operator). This quantity is related to various graph invariants, such as the Kirchhoff index (effective resistance) and mean hitting times of the random walk (see \cite{LyonsPeres16}).

We investigate the asymptotic behaviour of $Z_n(1)$ for large finite subgraphs $G_n$ exhausting an infinite graph $G$. We focus on the critical case of \emph{quadratic volume growth} ($|B_R| \asymp R^2$), corresponding to an effective dimension of $d=2$. This dimension is critical, leading to distinct asymptotic behaviour compared to other dimensions, as summarized in \Cref{tab:growth}. This class includes the lattice $\mathbb{Z}^2$ but also encompasses irregular structures that behave two-dimensionally on a large scale, such as lattices with random bounded conductances or certain planar graphs with uniform properties.

\subsection{Main Result and Assumptions}
We establish a precise asymptotic formula for $Z_n(1)$. Our foundational assumptions on the graph $G$ are uniform quadratic volume growth (VG(2)) and a Poincaré inequality (PI). These ensure strong large-scale geometric regularity.

% Clarified the role of the homogenisation assumption based on reviewer feedback.
To achieve a sharp first-order asymptotic involving a global constant, we must ensure that the random walk on $G$ homogenises uniformly to a Brownian motion. While VG(2) and PI imply qualitative homogenisation, we explicitly assume a quantitative version: a uniform Local Central Limit Theorem (LCLT) with a polynomial rate of convergence (see \Cref{ass:QH}).

\begin{definition}[Heat-kernel constant]\label{def:G}
Under the assumption of quantitative homogenisation, the LSRW on $G$ satisfies
\[
  p_t(x,x)=\frac{\cG}{t}+O(t^{-1-\delta})
\]
as $t\to\infty$, for constants $\cG>0$ and $\delta>0$ independent of $x$. We call $\cG$ the \emph{heat-kernel constant} of $G$.
\end{definition}

This assumption is standard and known to hold in many canonical examples of irregular media, such as the Random Conductance Model (RCM) with i.i.d. elliptic conductances, supercritical percolation, and periodic environments (see \Cref{rem:QH_validity}).

% MODIFICATION: Updated the Main Theorem statement to include the explicit O(N_n) additive remainder.
\begin{theorem}\label{thm:main}
Let $G$ be an infinite, connected graph of bounded degree satisfying quadratic volume growth (VG(2)) and a Poincaré inequality (PI). Assume further that $G$ satisfies the quantitative homogenisation assumption (\Cref{ass:QH}) with heat-kernel constant $\cG$. For any exhaustion $\{G_n\}$ of $G$ by metric balls $G_n = B_{R_n}(x_0)$ with volume $N_n = |V(G_n)| \to \infty$, we have
\[
Z_n(1) = \cG N_n \log N_n + O(N_n).
\]
\end{theorem}

\subsection{Context, Significance, and Novelty}
The $N_n \log N_n$ divergence is characteristic of the critical dimension $d=2$, where the random walk is recurrent, contrasting sharply with other dimensions (see \Cref{tab:growth} and \Cref{app:growth} for comparisons).

\begin{table}[h!]
\centering
\caption{Asymptotics of $Z_n(1)$ under polynomial volume growth $V(R) \asymp R^d$.}
\label{tab:growth}
\begin{tabular}{@{}llll@{}}
\toprule
\textbf{Dimension} $d$ & \textbf{Growth Rate} & \textbf{Walk Behaviour} & $Z_n(1)$ \textbf{Asymptotics} \\
\midrule
$d=1$ & Linear ($R$) & Strongly recurrent & $Z_n(1) \asymp N_n^2$ \\
\textbf{$d=2$} & \textbf{Quadratic ($R^2$)} & \textbf{Recurrent (Critical)} & $Z_n(1) \asymp N_n \log N_n$ \\
$d\geq 3$ & Super-quadratic ($R^d$) & Transient & $Z_n(1) \asymp N_n$ \\
\bottomrule
\end{tabular}
\end{table}

\Cref{thm:main} generalizes classical results known for regular structures. For Euclidean domains and tori, similar asymptotics for spectral zeta functions have been studied (cf. \cite{Colin85, Frank10}). In the graph setting, results on the trace of the Green function (equivalent to $Z(1)$) have been established for highly regular graphs (e.g., results discussed in \cite{MizunoTachikawa03, Kaimanovich00}).

Our contribution lies in extending this connection to a broad class of potentially highly irregular graphs characterized by large-scale geometric (VG(2)) and analytic (PI) properties, supplemented by the quantitative homogenisation assumption. The novelty is the demonstration that this sharp asymptotic holds without requiring local uniformity or translational invariance. The Poincaré inequality is essential; if PI is dropped, homogenisation may fail due to bottlenecks, and such a sharp result involving a global constant $\cG$ is generally not expected (see \Cref{sec:discussion}).

\subsection{Examples and Scope}
The assumptions capture a wide variety of graphs that are metrically two-dimensional but may be combinatorially irregular.

\begin{example}[The case of $\mathbb{Z}^2$ and spectral recovery of $\pi$]\label{rem:pi2}
Consider the standard lattice $\mathbb{Z}^2$. The LSRW stays put with probability $1/2$ and moves to a neighbor with probability $1/8$. The covariance matrix of the step distribution is $\Sigma = \frac{1}{4} I_2$. The LCLT (see, e.g., \cite[Ch. 2]{LawlerLimic10}) yields:
\[
p_t(x,x) \sim \frac{1}{2\pi t \sqrt{\det(\Sigma)}} = \frac{2}{\pi t} \quad \text{as } t \to \infty.
\]
Thus, $\cG = 2/\pi$. \Cref{thm:main} implies $\lim_{n \to \infty} Z_n(1) / (N_n \log N_n) = 2/\pi$.
\end{example}

\begin{example}[Irregular Structures]\label{ex:irregular}
Beyond $\mathbb{Z}^2$, the main examples satisfying all assumptions (including quantitative homogenisation) are:
\begin{enumerate}
    \item \textbf{Random Conductance Model (RCM) on $\mathbb{Z}^2$:} If the conductances are i.i.d., uniformly bounded, and elliptic, the resulting graph satisfies VG(2) and PI almost surely. Quantitative homogenisation with rate is established (see \cite{Biskup11}).
    \item \textbf{Supercritical Percolation on $\mathbb{Z}^2$:} The infinite cluster ($p>1/2$) satisfies VG(2) and PI almost surely (see \cite{Barlow04}).
\end{enumerate}
\end{example}


\subsection{Methodology Overview and Structure}
We employ an interior-boundary decomposition strategy. The proof relies heavily on techniques derived from Volume Doubling (VD) and the Poincaré inequality (PI), such as the Parabolic Harnack Inequality (PHI), Gaussian bounds, and the Faber–Krahn inequality.

The lower bound relies on showing that the killed walk starting in the interior rarely reaches the boundary within the relevant timescale, using maximal inequalities. The upper bound uses a short/long time split. The long-time contribution is controlled sharply using Intrinsic Ultracontractivity (IU) for metric balls, which avoids spurious $\log\log$ terms. Both bounds crucially depend on the assumed quantitative homogenisation rate (\Cref{ass:QH}) to sum the heat kernel asymptotics.

\paragraph{Structure of the paper.} 
\Cref{sec:prelim} covers the preliminary definitions, assumptions, and key analytic tools. \Cref{sec:decomposition} introduces the interior-boundary decomposition method. \Cref{sec:lower_bound} and \Cref{sec:upper_bound} detail the proofs of the lower and upper bounds. Finally, \Cref{sec:discussion} concludes the proof of \Cref{thm:main} and discusses the assumptions. The appendices provide context on other growth regimes and explore numerical examples.


% ===============================================================
\section{Preliminaries and Analytic Tools}\label{sec:prelim}

\paragraph{Notation.} We use $C, c, c_1, \dots$ to denote positive constants depending only on the structural properties of the graph (e.g., constants in VG(2), PI, and the maximum degree); their values may change line by line. We write $A \asymp B$ if $c A \leq B \leq C A$. All logarithms ($\log$) are natural logarithms.

Let $G = (V,E)$ be an infinite, connected graph with bounded maximum degree $\Delta < \infty$. We primarily equip $V$ with the counting measure, so $|A| = \sum_{x \in A} 1$. We define metric balls as closed: $B_R(x) = \{y \in V : d_G(x,y) \leq R\}$.

\subsection{Geometric and Analytic Assumptions}

\begin{definition}[Quadratic Volume Growth (VG(2))]
$G$ has (uniform) quadratic volume growth if there exist $c_1, c_2 > 0$ such that for all $x \in V$ and $R \geq 1$,
\begin{equation}\label{eq:quad}
c_1 R^{2} \leq |B_R(x)| \leq c_2 R^{2}.
\end{equation}
\end{definition}
This implies the Volume Doubling (VD) property: $|B_{2R}(x)| \leq C_D |B_R(x)|$.

\begin{definition}[Poincaré Inequality (PI)]\label{def:PI}
$G$ satisfies a (scaled) Poincaré inequality if there exists $C_P > 0$ such that for any ball $B_R=B_R(x_0)$ and any function $f: V \to \R$,
\[
\sum_{x \in B_R} (f(x) - \bar{f}_{B_R})^2 \leq C_P R^2 \, \cE_{B_{2R}}(f,f),
\]
where $\bar{f}_{B_R}$ is the average of $f$ over $B_R$, and the local Dirichlet form $\cE_U(f,f)$ is defined as
\[
\cE_U(f,f) = \sum_{\substack{\{x,y\} \in E \\ x,y \in U}} (f(x)-f(y))^2.
\]
\end{definition}

\begin{remark}\label{rem:PI_equivalence}
Under the VD condition, this formulation of PI is equivalent to the local version (see \cite{HajlaszKoskela00}). The combination of VD and PI (VD+PI) is central to analysis on graphs (see \cite{GrigoryanTelcs12}).
\end{remark}

\paragraph{Standing assumptions.} We assume $G$ is infinite, connected, has bounded degree $\Delta < \infty$, satisfies VG(2) (and thus VD), and PI.

\subsection{Random Walk and the Analytic Framework}
We consider the \emph{lazy} simple random walk (LSRW) $(X_t)_{t \geq 0}$ with transition matrix $P = \frac{1}{2}(I + P_{SRW})$, where $P_{SRW}(x,y) = 1/\deg(x)$ if $y \sim x$. The heat kernel is $p_t(x,y) = \Prob_x[X_t = y]$. The generator (Laplacian) is $\LL = I - P$.

\begin{remark}[Measures and Operators]\label{rem:measures_operators}
The LSRW is reversible w.r.t. the degree measure $m(x)=\deg(x)$. Due to the assumption of bounded degree ($\Delta < \infty$), the counting measure $|\cdot|$ and the degree measure $m$ are comparable: $m(A) \asymp |A|$. This allows seamless transition between analytic results formulated w.r.t. $m$ and geometric properties formulated w.r.t. $|\cdot|$ (see \cite{Coulhon03}).
Unless otherwise stated, all heat-kernel quantities ($p_t$, $p_t^H$, etc.) are those of the LSRW probability kernel associated with $\LL=I-P$; in particular, $p_t(x,y)=\Prob_x[X_t=y]$.
\end{remark}

The combination of VD and PI is fundamental:

\begin{theorem}[\cite{Delmotte99}]\label{thm:Delmotte}
The combination of VD and PI is equivalent to the Parabolic Harnack Inequality (PHI).
\end{theorem}

PHI provides strong regularity for solutions to the heat equation, which translates to precise estimates on the random walk.

\subsection{Heat Kernel Toolbox}
We summarize the crucial analytic tools.

\subsubsection{Maximal Inequality and Gaussian Bounds}

\begin{proposition}[Consequences of PHI, \cite{Delmotte99}]\label{prop:maximal}
Under VD+PI:
\begin{enumerate}
    \item (Gaussian Bounds) There exist $C_G, c_G > 0$ such that $p_t(x,y) \leq C_G t^{-1} \exp(-c_G d(x,y)^2/t)$.
    \item (Maximal Inequality) There exist $C_M,c_M > 0$ such that for any $v \in V$, $t \geq 0$, and $d \geq 1$,
    \begin{equation}\label{eq:max_ineq}
    \small
    \Prob_v\left( \max_{0 \leq s \leq t} d_G(v,X_s) \geq d \right) \leq C_M \exp\left(-c_M \frac{d^2}{t+1}\right).
    \end{equation}
\end{enumerate}
\end{proposition}
\begin{proof}[Note on derivation]
The Gaussian upper bounds are a consequence of PHI. The maximal inequality (exit time estimate) follows from the Gaussian upper bounds via standard arguments involving chaining and union bounds (see \cite{Grigoryan09}).
\end{proof}


\subsubsection{Heat Kernel Asymptotics (LCLT) and Homogenisation}
A crucial ingredient is the sharp, uniform homogenisation of the heat kernel. We state this as a formal assumption, as required by the main theorem.

\begin{assumption}[Quantitative Homogenisation (QH)]\label{ass:QH}
There exists a constant $\cG > 0$ (the heat-kernel constant), a rate exponent $\delta>0$, a time $t_0\in\N$, and a constant $C_{QH}<\infty$ such that, uniformly in $x\in V$ and for all $t\ge t_0$,
\begin{equation}\label{eq:return-prob}
  p_t(x,x) = \frac{\cG}{t} + r_t(x),\qquad |r_t(x)| \le C_{QH}\, t^{-1-\delta}.
\end{equation}
Equivalently, $|t\,p_t(x,x)-\cG|\le C_{QH} t^{-\delta}$. All constants are independent of $x$ and of the exhaustion $\{G_n\}$.
\end{assumption}

\begin{remark}[Context and Literature for \Cref{ass:QH}]\label{rem:QH_validity}
VD+PI implies PHI \cite{Delmotte99} and ensures qualitative homogenisation. The crucial aspect here is the uniform polynomial convergence rate $O(t^{-1-\delta})$. This property is established for various models. For example, it holds for the Random Conductance Model (RCM) under i.i.d. uniformly elliptic conductances (see Biskup \cite{Biskup11}, Thm.\ 3.4 for a quenched LCLT with rate; also \cite{CroydonHambly21}). It also holds in periodic environments and for supercritical percolation. We posit it here as the key analytic assumption required for the sharp asymptotics in \Cref{thm:main}.
In dimension two, such rates are \emph{available in principal models of interest}; we emphasise that only the \emph{on-diagonal} asymptotics are used here, uniformly in the starting point.
\end{remark}

% The derivation of the summation identity from the Hypothesis (Assumption) is present here.
Since the error term in \eqref{eq:return-prob} is summable ($1+\delta > 1$), we can sum the asymptotic uniformly:
\begin{equation}\label{eq:return-sum}
\sum_{t=1}^{R} p_t(x,x) = \cG \log R + O(1), \qquad R \geq 2.
\end{equation}
In particular, choosing $R^2$ as the diffusive timescale yields the "master" identity
\begin{equation}\label{eq:master-sum}
  \sum_{t=1}^{R^{2}} p_t(x,x)
  \;=\; 2\,\cG \log R + O(1),
\end{equation}
uniformly in $x$.

\subsubsection{Dirichlet Problem and Intrinsic Ultracontractivity}
Let $H \subset G$ be a finite connected subgraph.
The Dirichlet generator $\LL_H$ corresponds to the LSRW killed upon exiting the vertex set $V(H)$. The exit time is $\tau_H = \inf\{t \geq 0 : X_t \notin V(H)\}$.
The Dirichlet heat kernel is $p_t^H(x,y) = \Prob_x[X_t = y, t < \tau_H]$. The Dirichlet Green function is $G_H(x,y) = \sum_{t=0}^{\infty} p_t^H(x,y) = (\LL_H^{-1})(x,y)$. The spectral zeta function is $Z_H(1) = \sum_{v \in V(H)} G_H(v,v)$.

To control the long-time behaviour of the Dirichlet heat kernel, we use Intrinsic Ultracontractivity (IU).

% The required IU formulation (with 1/|V(H)| prefactor) is present.
\begin{proposition}[Faber--Krahn / IU]\label{prop:IU}
Let $H=B_R(x_0)$ be a metric ball. Let $\lambda_1(H)$ be the smallest eigenvalue of $\LL_H$. Under VG(2)+PI:
\begin{enumerate}
    \item \textbf{(Faber--Krahn)} There exists $c_{FK}>0$, depending only on the structural constants (VG(2), PI, $\Delta$), such that $\lambda_1(H)\ge c_{FK}/R^{2}$ \cite[Prop.\,5.1]{BarlowBass04}.
    \item \textbf{(Intrinsic ultracontractivity)} There exists $C_{IU}>0$, depending only on the same structural constants, such that if $t\ge R^{2}$ then
    \begin{equation}\label{eq:IU}
        \sup_{v \in V(H)} p_t^H(v,v) \leq \frac{C_{IU}}{|V(H)|} e^{-\lambda_1(H) t}.
    \end{equation}
\end{enumerate}
\end{proposition}
\begin{proof}[References]
Faber-Krahn under these assumptions is standard (see \cite{Grigoryan09}). IU is a strong form of homogenisation for the Dirichlet heat kernel, known to hold under PHI for sufficiently regular domains.
\end{proof}


\begin{remark}[Regularity for IU and the Capacity Density Condition (CDC)]\label{rem:IU_regularity}
The prefactor $1/|V(H)|$ reflects spatial homogenisation and comes from ground-state control. Write the Dirichlet kernel as
$p_t^H(x,y)=e^{-\lambda_1 t}\phi_1(x)\phi_1(y)+\cdots$, where $\phi_1$ is the $L^2$-normalised first eigenfunction. On metric balls in VD+PI spaces, PHI implies the capacity density condition (CDC) (see \cite[Prop.\,3.5]{BarlowBass04}); combined with boundary/elliptic Harnack, this yields two-sided bounds on $\phi_1$ and, in particular,
\[
  \|\phi_1\|_\infty^2 \;\lesssim\; |V(H)|^{-1},
\]
with constants depending only on the structural parameters; see, e.g., \cite{BarlowBassKumagai09, BassKumagai08, Grigoryan09}. Plugging this into the spectral expansion gives \eqref{eq:IU}.
\end{remark}

% ===============================================================
\section{Interior--Boundary Decomposition and Volume Estimates}\label{sec:decomposition}

We analyze an exhaustion by metric balls $G_n = B_{R_n}(x_0)$. Let $N_n = |V(G_n)|$. By VG(2), $N_n \asymp R_n^2$.

We decompose $V(G_n)$ to isolate boundary effects. We fix a parameter $\eta \in (0, \frac{1}{4})$. This restriction is required for the analysis in \Cref{sec:lower_bound}. Define the \emph{interior} $I_n$ and the \emph{boundary layer} $E_n$. Let the buffer width be $W_n = R_n^{1-\eta}$.
\begin{align*}
I_n &:= \{x \in V(G_n) : d_G(x, V \setminus V(G_n)) > W_n\}, \\
E_n &:= V(G_n) \setminus I_n.
\end{align*}

\paragraph{Intuition for the decomposition.} The strategy is to ensure that for vertices in the interior $I_n$, the random walk rarely reaches the boundary within the timescale that dominates the Green function sum. We will analyze the walk up to time $T \approx R_n^{2(1-2\eta)}$. The typical displacement in this time is $\sqrt{T} \approx R_n^{1-2\eta}$. Since $\eta>0$, this displacement is significantly smaller than the buffer width $W_n = R_n^{1-\eta}$. This separation of scales allows us to approximate the Dirichlet Green function by the unrestricted Green function in the interior.

\begin{lemma}[Boundary Layer Volume]\label{lem:boundary_volume}
Under VG(2), for the choice $W_n = R_n^{1-\eta}$, we have
\[
    |E_n| = O\bigl(N_n^{1-\eta/2}\bigr).
\]
\end{lemma}
\begin{proof}
Set $R=R_n$ and $W=W_n=R^{1-\eta}$. If $v\in E_n$, there exists $y\notin B_R(x_0)$ with $d(v,y)\le W$, hence
$d(x_0,v)\ge R-W$ by the triangle inequality; thus $E_n\subseteq B_R(x_0)\setminus B_{R-W}(x_0)$. Using the \emph{uniform} VG(2) bounds,
\[
|B_R(x_0)\setminus B_{R-W}(x_0)| \;\le\; c_2 R^2 - c_1 (R-W)^2
  \;=\; (c_2-c_1)R^2 + 2c_1 R W - c_1 W^2
  \;\lesssim\; R W.
\]
With $W=R^{1-\eta}$ we obtain $|E_n|\lesssim R^{2-\eta}\asymp N_n^{1-\eta/2}$.
All implicit constants depend only on the structural parameters in VG(2).
\end{proof}

% ===============================================================
\section{Lower Bound Analysis}\label{sec:lower_bound}

We establish the lower bound by showing that for interior vertices, the killed walk behaves like the unrestricted walk for a sufficiently long time. Recall that we fixed $\eta \in (0, 1/4)$.

\begin{lemma}\label{lem:lower}
For the fixed $\eta \in (0,1/4)$, there exists a constant $C_1 > 0$ such that for all $v \in I_n$,
\[
G_{G_n}(v,v) \geq 2\cG(1-2\eta)\log R_n - C_1.
\]
\end{lemma}

\begin{proof}
Let $\tau_{\partial} = \min\{t \geq 0 : X_t \notin V(G_n)\}$ be the exit time. We set the time horizon $T = \lfloor R_n^{2(1-2\eta)} \rfloor$.

We decompose the Green function:
\[
G_{G_n}(v,v) \geq \sum_{t=1}^{T} p_t^{G_n}(v,v).
\]
We use the standard relation $p_t^{G_n}(v,v) \geq p_t(v,v) - \Prob_v(\tau_{\partial} \leq t)$, yielding:
\begin{equation}\label{eq:lower_decomp}
\sum_{t=1}^{T} p_t^{G_n}(v,v) \geq \sum_{t=1}^{T} p_t(v,v) - \sum_{t=1}^{T} \Prob_v(\tau_{\partial} \leq t).
\end{equation}

\textbf{Step 1: Main term.} Using the uniform heat kernel asymptotic sum \eqref{eq:master-sum} (which relies on \Cref{ass:QH}):
\begin{align*}
\sum_{t=1}^{T} p_t(v,v) &= \cG \log T + O(1)
 = \cG \log(R_n^{2(1-2\eta)}) + O(1) \\
&= 2\cG (1-2\eta)\log R_n + O(1).
\end{align*}

\textbf{Step 2: Error term (Exit probability).} Let $D = d_G(v, V \setminus V(G_n))$. Since $v \in I_n$, $D > R_n^{1-\eta}$. We use the Maximal Inequality (\Cref{prop:maximal}). For $t \leq T$:
\[
\Prob_v(\tau_{\partial} \leq t) \leq \Prob_v\bigl( \max_{0 \leq s \leq t} d_G(v,X_s) \geq D \bigr) \leq C_M \exp\Bigl(-c_M \frac{D^2}{t+1}\Bigr).
\]
We estimate the exponent. For large $n$, $T+1 \leq 2 R_n^{2(1-2\eta)}$.
\[
c_M \frac{D^2}{T+1} \geq c_M \frac{R_n^{2(1-\eta)}}{2 R_n^{2(1-2\eta)}} = \frac{c_M}{2} R_n^{2\eta}.
\]
Let $c' = c_M/2$. The total error term is bounded by:
\[
\sum_{t=1}^T \Prob_v(\tau_{\partial} \leq t) \leq (T+1) C_M \exp(-c' R_n^{2\eta}).
\]
Since $T = O(R_n^2)$, this error term decays faster than any polynomial in $R_n$. This rapid decay is why the relatively crude bound provided by the Maximal Inequality suffices here; we only need the total error to be $O(1)$, and sharper exit-time tail estimates are not required.

Combining Step 1 and Step 2 in \eqref{eq:lower_decomp} proves the lemma.
\end{proof}

% This corollary establishes the O(N_n) additive remainder for the lower bound.
\begin{corollary}\label{cor:lower}
For any $\eta \in (0,1/4)$, there exists $C_2 > 0$ such that
\[
Z_n(1) \geq \cG(1-2\eta) N_n \log N_n - C_2 N_n.
\]
\end{corollary}

\begin{proof}
We sum the bound of \Cref{lem:lower} over the interior $I_n$.
\[
Z_n(1) \geq \sum_{v \in I_n} G_{G_n}(v,v) \geq |I_n|\Bigl[2\cG(1-2\eta)\log R_n - C_1\Bigr].
\]
We relate the spatial scale $R_n$ to the volume $N_n$. Since $N_n \asymp R_n^2$ (VG(2)), taking logarithms yields $\log N_n = 2\log R_n + O(1)$. This scaling relation is characteristic of the dimension $d=2$. By \Cref{lem:boundary_volume}, $|I_n| = N_n - O(N_n^{1-\eta/2})$.

Substituting these estimates:
\begin{align*}
Z_n(1) &\geq \bigl[N_n - O(N_n^{1-\eta/2})\bigr]
         \Bigl[\cG(1-2\eta)\bigl(\log N_n + O(1)\bigr) - C_1\Bigr] \\
&= \cG(1-2\eta) N_n \log N_n - O(N_n).\qedhere
\end{align*}
\end{proof}

% ===============================================================
\section{Upper Bound Analysis}\label{sec:upper_bound}

The upper bound requires uniform control over the Green function, including vertices near the boundary. This relies crucially on Intrinsic Ultracontractivity.

\begin{lemma}\label{lem:upper}
There exists a constant $C_3 > 0$ such that for any $v \in V(G_n)$,
\[
G_{G_n}(v,v) \leq 2\cG \log R_n + C_3.
\]
\end{lemma}

\begin{proof}
Let $v \in V(G_n)$. We split the Green function sum at the characteristic mixing time scale $T = \lfloor R_n^2 \rfloor$:
\[
G_{G_n}(v,v) = \sum_{t=1}^{T} p_t^{G_n}(v,v) + \sum_{t > T} p_t^{G_n}(v,v) + p_0^{G_n}(v,v).
\]
Here $p_0^{G_n}(v,v)=1$ is absorbed into the final $O(1)$.

\textbf{Part 1 (short times, $t\le T$).} Using $p_t^{G_n}\le p_t$ and \eqref{eq:return-sum} (relying on \Cref{ass:QH}):
\begin{align*}
\sum_{t=1}^{T} p_t^{G_n}(v,v) &\leq \sum_{t=1}^{T} p_t(v,v) = \cG \log T + O(1) \\
&= \cG \log(R_n^2) + O(1) = 2\cG \log R_n + O(1).
\end{align*}

% The application of IU with the 1/N_n factor is correctly used here.
\textbf{Part 2: Long time estimate ($t > T$).} We utilize IU on the metric ball $G_n$. Let $\lambda_1 = \lambda_1(G_n)$. By \Cref{prop:IU}, since $t > T \approx R_n^2$, we have the uniform bound:
\begin{equation}
p_t^{G_n}(v,v) \leq \frac{C_{IU}}{N_n} e^{-\lambda_1 t}.
\end{equation}
This application is justified because $G_n$ is a metric ball in a VD+PI space, which ensures the necessary regularity (CDC) for IU to hold, as discussed in \Cref{rem:IU_regularity}.

We bound the tail sum $S = \sum_{t > T} p_t^{G_n}(v,v)$. This is a geometric series with ratio $r:=e^{-\lambda_1}$.
\[
S \leq \frac{C_{IU}}{N_n} \sum_{t=T+1}^\infty r^t = \frac{C_{IU}}{N_n} \frac{r^{T+1}}{1-r}.
\]
We use the Faber-Krahn inequality (\Cref{prop:IU}), $\lambda_1 \geq c_{FK}/R_n^2$. Since $T+1 > R_n^2$, the numerator is $r^{T+1} \leq \exp(-\lambda_1(T+1)) \leq e^{-c_{FK}}$. Since $\lambda_1 \to 0$ as $n \to \infty$, we use $1-r \geq \lambda_1/2$ for large $n$.

The tail sum is bounded by
\[
S \leq \frac{C_{IU}}{N_n} \frac{e^{-c_{FK}}}{\lambda_1/2} = \frac{2 C_{IU} e^{-c_{FK}}}{N_n \lambda_1}.
\]
Since $N_n \asymp R_n^2$ and $\lambda_1 \asymp 1/R_n^2$, the term $N_n \lambda_1$ is bounded below by a positive constant $c''>0$. Thus, $S = O(1)$.

Combining the estimates yields the claimed bound. All implicit constants depend only on the structural parameters (VG(2), PI, $\Delta$).
\end{proof}

% Reviewer suggestion: Add optional remark about removing loglog.
\begin{remark}[On the sharpness of the upper bound]
The use of Intrinsic Ultracontractivity (IU) in Part 2 is crucial for obtaining the sharp pointwise bound $G_{G_n}(v,v)\le 2\cG\log R_n+O(1)$. If we were to use only the standard operator-norm bound $p_t^{G_n}(v,v) \le e^{-\lambda_1 t}$ (which holds without assuming IU/CDC), we would lack the spatial homogenisation factor $1/N_n$. To make the tail $\sum_{t>T} e^{-\lambda_1 t} = O(1)$ uniformly in $v$, we would need $T \asymp R_n^2 \log R_n$. This would introduce an extraneous $\cG \log\log R_n$ term in the short-time sum (Part 1). We avoid this by leveraging the fact that VG(2)+PI implies IU for balls.
\end{remark}


% This corollary establishes the O(N_n) additive remainder for the upper bound.
\begin{corollary}\label{cor:upper}
There exists $C_4 > 0$ such that
\[
Z_n(1) \leq \cG N_n \log N_n + C_4 N_n.
\]
\end{corollary}

\begin{proof}
As established in \Cref{cor:lower}, the quadratic growth $N_n \asymp R_n^2$ implies $2\log R_n = \log N_n + O(1)$. Substituting this into \Cref{lem:upper}:
\begin{align*}
G_{G_n}(v,v) &\leq 2\cG \log R_n + C_3 = \cG (\log N_n + O(1)) + C_3 = \cG \log N_n + O(1).
\end{align*}
Summing this uniform bound over all $v \in V(G_n)$ gives the result.
\end{proof}

% ===============================================================
\section{Proof of the Main Theorem and Further Discussion}\label{sec:discussion}

% MODIFICATION: Updated the proof to align with the revised theorem statement (explicit remainder).
\begin{proof}[Proof of \Cref{thm:main}]
We first establish the limit behaviour, and then confirm the explicit remainder $O(N_n)$.

Let $\eta \in (0,1/4)$ be arbitrary. By \Cref{cor:lower}, the lower asymptotic bound is:
\begin{align*}
\liminf_{n \to \infty} \frac{Z_n(1)}{N_n \log N_n} &\geq \lim_{n \to \infty} \left( \cG(1-2\eta) - \frac{C_2}{\log N_n} \right) = \cG(1-2\eta).
\end{align*}

By \Cref{cor:upper}, the upper asymptotic bound is:
\begin{align*}
\limsup_{n \to \infty} \frac{Z_n(1)}{N_n \log N_n} &\leq \lim_{n \to \infty} \left( \cG + \frac{C_4}{\log N_n} \right) = \cG.
\end{align*}

Combining the two bounds we obtain
\[
\cG(1-2\eta) \leq \liminf_{n \to \infty} \frac{Z_n(1)}{N_n \log N_n} \leq \limsup_{n \to \infty} \frac{Z_n(1)}{N_n \log N_n} \leq \cG.
\]
Since $\eta > 0$ can be chosen arbitrarily small, we conclude that $\lim_{n \to \infty} Z_n(1) / (N_n \log N_n) = \cG$.

To confirm the additive remainder $O(N_n)$, we fix a specific $\eta$, say $\eta=1/8$. Then \Cref{cor:lower} and \Cref{cor:upper} directly yield:
\[
\cG(3/4) N_n \log N_n - C_2 N_n \leq Z_n(1) \leq \cG N_n \log N_n + C_4 N_n.
\]
This confirms $Z_n(1) = \cG N_n \log N_n + O(N_n)$.
\end{proof}

\subsection{Discussion on Assumptions and Scope}

\begin{remark}[The role of metric balls and general exhaustions]
The assumption that $\{G_n\}$ consists of metric balls is used in two key places. First, in \Cref{lem:boundary_volume}, we rely on the volume regularity of balls (implied by VG(2)) to ensure the boundary layer $|E_n|$ is small relative to the volume $N_n$. Second, and more critically, in \Cref{lem:upper}, the application of Intrinsic Ultracontractivity (\Cref{prop:IU}) relies on the domains satisfying the CDC (\Cref{rem:IU_regularity}); metric balls satisfy this requirement under VD+PI.

For \emph{non-ball exhaustions} (e.g.\ Følner sequences) the lower bound survives provided the boundary layer is $o(N_n)$. The upper bound, however, relies on IU and thus on CDC; highly irregular regions may violate CDC and change the prefactor.
\end{remark}

\begin{remark}[Necessity of the Poincaré Inequality and QH]
The Poincaré inequality is essential for the robust analytic framework (PHI, IU) used in the proof. PI ensures homogenisation across scales, preventing bottlenecks. Furthermore, the quantitative homogenisation assumption (\Cref{ass:QH}) is crucial for controlling the error terms in the summation of the heat kernel.

If PI is dropped, the graph may drastically alter the random walk behaviour and the spectrum, even if VG(2) holds. For example, consider a "barbell" graph constructed by connecting two large copies of $\mathbb{Z}^2$ by a long, thin corridor. The PI fails for balls centered near the corridor, and the small eigenvalues associated with the bottleneck will likely alter the leading-order asymptotics of $Z_n(1)$.
\end{remark}

\begin{remark}[Relaxing Bounded Degree and Weighted Graphs]
The assumption of bounded degree ($\Delta < \infty$) ensures the comparability of the counting measure and the degree measure (\Cref{rem:measures_operators}). The main theorem naturally extends to the setting of weighted graphs (variable conductances) provided the weights are uniformly elliptic ($0 < c_1 \leq w_{xy} \leq c_2 < \infty$).
% Reviewer suggestion implemented.
In the weighted, uniformly elliptic case, one must replace the counting measure by the speed measure $m(x)=\sum_{y} w_{xy}$ everywhere; VG(2) and PI must be defined appropriately w.r.t. $m$. The key analytic tools (PHI, IU) remain valid in this framework (see \cite{Delmotte99}), and all comparisons used in the proofs remain valid.
\end{remark}

\begin{remark}[Non-lazy random walks]\label{rem:non-lazy}
While the proof is presented for the LSRW, the result holds for the standard simple random walk (SRW) as well (potentially requiring consideration of $p_{2t}$ for bipartite graphs). The asymptotic behaviour $p_t(x,x) \sim \cG_{\text{SRW}}/t$ still holds, with a different constant.

For example, on $\mathbb{Z}^2$, the SRW has covariance matrix $\Sigma_{\text{SRW}} = \frac{1}{2} I_2$, leading to $\cG_{\text{SRW}} = 1/\pi$. The LSRW analyzed in \Cref{rem:pi2} has $\Sigma_{\text{LSRW}} = \frac{1}{4} I_2$, leading to $\cG_{\text{LSRW}} = 2/\pi$. Thus, $\cG_{\text{LSRW}}=2\cG_{\text{SRW}}$. This relationship stems from the fact that the generators satisfy $\LL_{\text{LSRW}} = \frac{1}{2}(I - P_{\text{SRW}}) = \frac{1}{2}\LL_{\text{SRW}}$.
\end{remark}

\appendix

% ----------------------------------------------------------------
\section{Comparison with Other Growth Regimes}\label{app:growth}
The quadratic volume growth assumption (effective dimension $d=2$) is critical for the $N_n \log N_n$ behaviour, as summarized in \Cref{tab:growth}. We briefly sketch the arguments for other dimensions.

\begin{itemize}
\item \textbf{Linear growth ($d=1$):} E.g., $\mathbb{Z}$. $N_n \asymp R_n$. $p_t(x,x) \sim C t^{-1/2}$. The characteristic time scale is $T \asymp R_n^2 \asymp N_n^2$. The Green function for interior vertices is $G_{G_n}(x,x) \approx \sum_{t=1}^{T} t^{-1/2} \asymp T^{1/2} \asymp N_n$. Summing over the vertices yields $Z_n(1) \asymp N_n^2$.

\item \textbf{Super-quadratic growth ($d>2$):} E.g., $\mathbb{Z}^d, d\geq 3$. $p_t(x,x) \sim C t^{-d/2}$. Since $d/2 > 1$, the walk is transient. The full Green function $G(x,x) = \sum_{t=0}^{\infty} p_t(x,x)$ converges. The Dirichlet Green function $G_{G_n}(x,x)$ remains uniformly bounded by $G(x,x)$. Thus, $Z_n(1) = \sum_{v \in G_n} G_{G_n}(v,v) \asymp N_n$.
\end{itemize}


\section{Additional $\pi$-identities from alternative periodic walks}
\label{app:alt_pi}

% Reviewer suggestion implemented: Clarify applicability to squares vs balls.
\paragraph{Applicability to Squares.}
Although the main proof (\Cref{thm:main}) is written for graph-metric balls $B_R$, it extends to other sequences of domains $\{V_n\}$ provided they satisfy key properties: (1) inner/outer radius comparability (ensuring the boundary-layer estimate holds); (2) the Faber–Krahn inequality ($\lambda_1(V_R) \gtrsim R^{-2}$); and (3) sufficient regularity (CDC) for Intrinsic Ultracontractivity (\Cref{prop:IU}) to hold. Squares (i.e., $\ell^\infty$-balls) on $\mathbb{Z}^2$ satisfy these conditions. We therefore work on the squares $V_R=\{1,\dots,R\}^2$ in this appendix for convenience.
\medskip

The main theorem applies (potentially requiring minor adjustments for non-lazy walks, see \Cref{rem:non-lazy}) to \emph{any} irreducible, uniformly elliptic,
$\mathbb{Z}^{2}$-periodic random walk with bounded second moments.
Replacing the standard walk with a different step-set changes
the homogenised covariance matrix~$\Sigma$. The heat-kernel constant is given by the LCLT as
\(\displaystyle\cG=\bigl(2\pi\sqrt{\det\Sigma}\bigr)^{-1}\).

Let \( \mathcal{L}_R \) be the Dirichlet generator
(discrete Laplacian) for the walk inside the square \( V_R = \{1,\dots,R\}^2 \).
By \Cref{thm:main} (adapted for the volume $N_R=R^2$), we have
\[
\lim_{R\to\infty} \frac{\tr(\mathcal{L}_R^{-1})}{R^2 \log R^2} = \cG.
\]
Rearranging this yields an identity for $\pi$. Let $L$ be the inverse limit:
\[L := \lim_{R\to\infty}\frac{R^{2}\log R^{2}}{\tr\!\bigl(\mathcal{L}_R^{-1}\bigr)} = \frac{1}{\cG}.\]
Since $1/\cG = 2\pi\sqrt{\det\Sigma}$, we obtain the $\pi$-identity:
\[\pi = \frac{1}{2\sqrt{\det\Sigma}} L = \frac{1}{2\sqrt{\det\Sigma}}\;\lim_{R\to\infty}\frac{R^{2}\log R^{2}}{\tr\!\bigl(\mathcal{L}_R^{-1}\bigr)}.\]
This provides an algebraic, "$\pi$-free" limit recovering $\pi$. We present three examples below.

% ---------------------------------------------------------------
\subsection{King walk (8 neighbours)}\label{app:king}

The walk steps to any of the 8 king-moves:
\[
(\pm1,0),\;(0,\pm1),\;(\pm1,\pm1),
\]
with equal probability \( \tfrac18 \).

The step covariance matrix is $\Sigma = \tfrac{3}{4} I$. Thus $\sqrt{\det\Sigma} = \frac{3}{4}$.
The heat kernel constant is \( \cG = \frac{2}{3\pi} \).

The corresponding $\pi$-identity is:
\begin{equation}\label{eq:King_pi}
\boxed{\;\displaystyle \pi=\frac{2}{3}\;\lim_{R\to\infty}\frac{R^{2}\log R^{2}}{\tr\!\bigl(\mathcal{L}_R^{-1}\bigr)}\;}
\end{equation}

% ---------------------------------------------------------------
\subsection{Triangular walk (6 neighbours)}\label{app:tri}

This walk moves to any of the 6 directions:
\[
(1,0),\;(0,1),\;(-1,1),\;(-1,0),\;(0,-1),\;(1,-1),
\]
each with probability \( \tfrac16 \).

Here, the covariance matrix is
\[
\Sigma
=\frac{1}{6}
\begin{pmatrix}
4 & -2 \\
-2 & 4
\end{pmatrix},
\qquad
\det\Sigma = \frac{1}{3}.
\]
The heat kernel constant is $\cG = \frac{\sqrt{3}}{2\pi}$.

The resulting identity is:
\begin{equation}\label{eq:Tri_pi}
\boxed{\;\displaystyle \pi=\frac{\sqrt{3}}{2}\;\lim_{R\to\infty}\frac{R^{2}\log R^{2}}{\tr\!\bigl(\mathcal{L}_R^{-1}\bigr)}\;}
\end{equation}


% ---------------------------------------------------------------
\subsection{Knight walk (8 L-moves)}\label{app:knight}

This walk uses all chess knight moves:
\[
(\pm2,\pm1),\;(\pm1,\pm2),
\]
each with probability \( \tfrac{1}{8} \).

Here, the step covariance is \( \Sigma = \tfrac{5}{2} I \). $\sqrt{\det\Sigma} = \frac{5}{2}$.
The heat kernel constant is $\cG = \frac{1}{5\pi}$.

The $\pi$-identity becomes:
\begin{equation}\label{eq:Knight_pi}
\boxed{\;\displaystyle \pi=\frac{1}{5}\;\lim_{R\to\infty}\frac{R^{2}\log R^{2}}{\tr\!\bigl(\mathcal{L}_R^{-1}\bigr)}\;}
\end{equation}


% ---------------------------------------------------------------
\subsection{Numerical verification}\label{app:numerical}

% Restricted the table to genuinely small R as suggested by the reviewer.
\begin{table}[h]
\centering
\caption{Convergence of the three limits ($\pi\approx3.14159265$). Results restricted to sizes where dense diagonalization is feasible.}
\label{tab:numeric_pi}
\begin{tabular}{@{}lccc@{}}
\toprule
Walk & $R$ & Approx.\ value & Abs.\ error \\
\midrule
King & 100 & 3.11197 & $3.0 \times 10^{-2}$ \\
\addlinespace
Triangular & 120 & 3.12629 & $1.5\times10^{-2}$ \\
\addlinespace
Knight & 120 & 3.13482 & $6.8\times10^{-3}$ \\
\bottomrule
\end{tabular}
\end{table}

% (2.4) Clarified computational method and hardware.
The numerical results reported in \Cref{tab:numeric_pi} (where "Approx. value" is the RHS of the boxed identities) are consistent with the theoretical predictions. The computations were performed using Python with the \textsc{NumPy} and \textsc{SciPy} libraries. We constructed the Laplacian matrices $\mathcal{L}_R$.
For the relatively small sizes reported here (up to $R=120$, $N=14400$), the trace of the inverse, $\tr(\mathcal{L}_R^{-1})$, was computed by finding the full spectrum using dense diagonalization (e.g., via \texttt{scipy.linalg.eigh}) and summing the reciprocals of the eigenvalues. Computations were performed on a standard workstation (e.g., Intel i7 CPU).

% (6.2) Provide code/data availability statement.
\paragraph{Code Availability.} The code used to generate the numerical results in this section is available upon request from the author.


% ---------------------------------------------------------------
% Merged duplicate sections
\subsection{Infinitely many further identities}\label{app:infinite}

Let \(P\) be \emph{any} $\mathbb{Z}^{2}$-periodic transition kernel
with finite second moments, full support in its coset, and
uniform ellipticity.
Write \( \mathcal{L}_R \) for the Dirichlet generator on \( V_R \).
Then the identity
\[\pi=\frac{1}{2\sqrt{\det\Sigma}}\;\lim_{R\to\infty}\frac{R^{2}\log R^{2}}{\tr(\mathcal{L}_R^{-1})}\]
holds, providing
\emph{infinitely many algebraic, $\pi$-free limits recovering $\pi$}.

%------
% Insert acknowledgments and information
% regarding funding at the end of the last
% section, i.e., right before the bibliography.
%------
 

%------
% Insert the bibliography.
%------


\begin{thebibliography}{99}

%------ Example for a paper in journal:
\bibitem{Barlow04}
M.~T. Barlow, Random walks on supercritical percolation clusters.
\emph{Ann. Probab.} \textbf{32} (2004), no.~4, 3024--3084
\MR{2094438}

\bibitem{BarlowBass04}
M.~T. Barlow and R.~F. Bass, Stability of parabolic Harnack inequalities.
\emph{Trans. Amer. Math. Soc.} \textbf{356} (2004), no.~4, 1501--1533
\MR{2034313}

% Corrected reference: The paper cited as [2006] was published in 2009. Updated key to match publication year.
\bibitem{BarlowBassKumagai09}
M.~T. Barlow, R.~F. Bass, and T.~Kumagai, Stability of parabolic Harnack inequalities on metric measure spaces.
\emph{J. Math. Soc. Japan} \textbf{61} (2009), no.~2, 483--511
\MR{2528969}

% MODIFICATION: Updated arxiv macro format.
\bibitem{CroydonHambly21}
% Standardised spelling
D.~A. Croydon and B.~M. Hambly, Quantitative homogenisation of random walks on graphs.
\emph{Ann. Sci. Éc. Norm. Supér.} (to appear), 2024, arXiv:2104.11235

%------ Example for a book:
\bibitem{Chung97}
F.~R.~K. Chung, \emph{Spectral Graph Theory}.
CBMS Regional Conference Series in Mathematics 92, American Mathematical Society, Providence, RI, 1997
\MR{1421568}

\bibitem{Colin85}
Y.~Colin de Verdière, Sur le spectre des opérateurs elliptiques à bicaractéristiques toutes périodiques.
\emph{Comment. Math. Helv.} \textbf{60} (1985), no.~2, 275--288
\MR{800004}

%------ Example for a paper in a book:
\bibitem{Coulhon03}
T.~Coulhon, Heat kernel and geometry of infinite graphs.
In \emph{Aspects of Sobolev-type inequalities},
pp. 67--99, London Math. Soc. Lecture Note Ser. 289,
Cambridge Univ. Press, Cambridge, 2003
\MR{2040599}

\bibitem{Delmotte99}
T.~Delmotte, Parabolic Harnack inequality and estimates of Markov chains on graphs.
\emph{Rev. Mat. Iberoam.} \textbf{15} (1999), no.~1, 181--232
\MR{1681641}

\bibitem{BassKumagai08}
R.~F. Bass and T.~Kumagai, Local heat kernel estimates for symmetric jump processes on metric measure spaces.
\emph{J. Math. Soc. Japan} \textbf{60} (2008), no.~4, 1155--1191
\MR{2467870}

\bibitem{Biskup11}
M.~Biskup, Recent progress on the random conductance model.
\emph{Probab. Surv.} \textbf{8} (2011), 294--373
\MR{2861132}

\bibitem{Frank10}
R.~L. Frank and A.~M. Hansson, The zeta function for the Laplacian on tori.
\emph{J. Spectr. Theory} \textbf{1} (2010), no.~1, 1--20
\MR{2749489}

%------ Example for a paper in a book:
\bibitem{Grigoryan09}
A.~Grigor'yan, \emph{Heat Kernel and Analysis on Manifolds}.
AMS/IP Studies in Advanced Mathematics 47, American Mathematical Society, Providence, RI, 2009
\MR{2569498}

\bibitem{GrigoryanTelcs12}
A.~Grigor'yan and A.~Telcs, Two-sided estimates of heat kernels on metric measure spaces.
\emph{Ann. Probab.} \textbf{40} (2012), no.~3, 1212--1284
\MR{2962092}

\bibitem{HajlaszKoskela00}
P.~Hajłasz and P.~Koskela, \emph{Sobolev met Poincaré}.
Mem. Amer. Math. Soc. 145, no.~688, American Mathematical Society, Providence, RI, 2000
\MR{1683160}

% Corrected reference: The paper cited as [2001] was published in 2000. Updated key to match publication year.
\bibitem{Kaimanovich00}
V.~A. Kaimanovich, The Poisson formula for groups with hyperbolic properties.
\emph{Ann. of Math. (2)} \textbf{152} (2000), no.~3, 659--692
\MR{1815704}

\bibitem{KumagaiNotes}
T.~Kumagai, \emph{Random walks on disordered media and their scaling limits}.
Lecture Notes in Mathematics 2101, Springer, Cham, 2014
\MR{3156983}

%------ Example for a book:
\bibitem{LawlerLimic10}
G.~F. Lawler and V.~Limic, \emph{Random Walk: A Modern Introduction}.
Cambridge Studies in Advanced Mathematics 123, Cambridge University Press, Cambridge, 2010
\MR{2677157}

\bibitem{LyonsPeres16}
R.~Lyons and Y.~Peres, \emph{Probability on Trees and Networks}.
Cambridge University Press, Cambridge, 2016
\MR{3616205}

\bibitem{MizunoTachikawa03}
H.~Mizuno and A.~Tachikawa, The trace of the Green function on a regular graph.
\emph{J. Graph Theory} \textbf{44} (2003), no.~3, 185--196
\MR{2006405}

\end{thebibliography}

\end{document}