\documentclass{article}
\usepackage[utf8]{inputenc}
\usepackage{amsmath, amssymb}
\usepackage{geometry}

\title{Linear-Time Decoders for "Good" qLDPC Codes}
\author{Research Overview}
\date{\today}

\begin{document}

\maketitle

\begin{abstract}
This paper reviews the current state of "good" quantum Low-Density Parity-Check (qLDPC) codes and the ongoing challenge of constructing efficient linear-time decoders for them.
\end{abstract}

\section{Domain}
Quantum Error Correction (QEC) \& Codes

\section{Context}
A "good" quantum code is one where the number of logical qubits ($k$) and the code distance ($d$) both scale linearly with the number of physical qubits ($n$). For years, it was open if these even \emph{existed}.

\section{Recent Breakthrough}
In 2022/2023, "asymptotically good" quantum Low-Density Parity-Check (qLDPC) codes were finally constructed (the "quantum Tanner codes").

\section{Current Open Challenge}
While we now know they \emph{exist}, the race is on to construct practical, \textbf{linear-time decoders} for them. Recent work (2024) has proposed some, but optimizing their fault-tolerance thresholds remains a major mathematical engineering challenge.

\end{document}
