\documentclass{article}
\usepackage[utf8]{inputenc}
\usepackage{amsmath, amssymb}
\usepackage{geometry}

\title{Concentration of Measure in QAOA}
\author{Research Overview}
\date{\today}

\begin{document}

\maketitle

\begin{abstract}
This paper examines the performance guarantees of the Quantum Approximate Optimization Algorithm (QAOA) at low depth, focusing on issues related to concentration of measure.
\end{abstract}

\section{Domain}
Quantum Algorithms (QAOA)

\section{The Problem}
The Quantum Approximate Optimization Algorithm (QAOA) is a leading candidate for near-term advantage. A major open mathematical problem is determining the \textbf{performance guarantees} of QAOA at low depth ($p$) for general graphs.

\section{Status}
\textbf{Open.} It is known that at low depth, the algorithm "sees" only a local view of the graph, which limits its performance (due to "overlap gap properties" or "concentration of measure"). Proving exactly \emph{when} QAOA beats classical algorithms like Goemans-Williamson for specific hard instances is an active open battleground.

\end{document}
