\documentclass{article}
\usepackage[utf8]{inputenc}
\usepackage{amsmath, amssymb}
\usepackage{geometry}

\title{The Quantum PCP (Probabilistically Checkable Proof) Conjecture}
\author{Research Overview}
\date{\today}

\begin{document}

\maketitle

\begin{abstract}
This paper discusses the Quantum PCP Conjecture, a potential quantum analogue to the classical PCP theorem, which asks whether approximating the ground state energy of a local Hamiltonian is QMA-hard.
\end{abstract}

\section{Domain}
Quantum Complexity Theory

\section{The Problem}
This is likely the "grand challenge" of quantum complexity. In classical complexity, the PCP theorem states that for any problem in NP, there exists a proof that can be verified by reading only a constant number of bits. The \textbf{Quantum PCP conjecture} asks if there is a quantum analogue: \emph{Is calculating the ground state energy of a local Hamiltonian (a physical energy function) hard to approximate even with a constant error margin?}

\section{Implications}
If true, it would mean that "entanglement" makes finding even \emph{approximate} solutions to quantum physical systems extremely difficult (QMA-hard). It is deeply connected to the stability of entanglement at room temperature.

\section{Status}
\textbf{Open.} A related precursor, the \textbf{NLTS (No Low-Energy Trivial States) Conjecture}, was \textbf{solved} recently (2023) by Anshu, Breuckmann, and Nirkhe, proving that there \emph{are} systems where all low-energy states remain highly entangled. However, the full Quantum PCP conjecture remains unsolved.

\end{document}
