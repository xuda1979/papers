\documentclass{article}
\usepackage[utf8]{inputenc}
\usepackage{amsmath, amssymb, amsthm}
\usepackage{geometry}
\usepackage{graphicx}
\usepackage{hyperref}
\usepackage{natbib}
\usepackage{braket}
\usepackage{tikz}

\newtheorem{theorem}{Theorem}
\newtheorem{conjecture}{Conjecture}
\newtheorem{definition}{Definition}
\newtheorem{lemma}{Lemma}
\newtheorem{proposition}{Proposition}

\title{Circuit Depth Lower Bounds for Quantum Low-Density Parity-Check Codes: An NLTS Perspective}
\author{Research Overview}
\date{\today}

\begin{document}

\maketitle

\begin{abstract}
The Quantum PCP (QPCP) conjecture posits that calculating the ground state energy of a local Hamiltonian remains QMA-hard even for constant relative error. A critical step towards this conjecture is the No Low-Energy Trivial States (NLTS) theorem, which asserts the existence of local Hamiltonians whose low-energy states cannot be prepared by constant-depth quantum circuits. In this work, we rigorously analyze the relationship between the code distance of Quantum Low-Density Parity-Check (qLDPC) codes and the circuit depth required to prepare their ground states. We derive a logarithmic lower bound on the circuit depth $\Omega(\log d)$ for stabilizer codes with distance $d$, utilizing light-cone propagation arguments. We further discuss how this result, when applied to "good" qLDPC codes with linear distance ($d \propto n$), underpins the recent proof of the NLTS theorem. Finally, we present numerical results on the topological entanglement entropy of the Toric Code, illustrating the robustness of long-range entanglement in these systems.
\end{abstract}

\section{Introduction}

The Quantum Probabilistically Checkable Proof (PCP) conjecture is a central open problem in quantum complexity theory, generalizing the classical PCP theorem. It suggests that the difficulty of approximating the ground state energy of a local Hamiltonian persists even when the desired precision is a constant fraction of the total extensive energy \citep{aharonov2013quantum}.

If the QPCP conjecture holds, it implies that the entanglement structure of ground states is robust against thermal fluctuations, a property essential for the existence of self-correcting quantum memories at finite temperatures. A necessary condition for QPCP is the existence of "No Low-Energy Trivial States" (NLTS) \citep{freedman2014quantum}: there must exist Hamiltonians where \textit{all} states below a certain energy threshold are highly entangled and cannot be prepared by shallow quantum circuits.

This paper focuses on the mathematical foundations connecting quantum error-correcting codes to circuit complexity. We formalize the argument that the code distance imposes a lower bound on the circuit depth for state preparation. Section \ref{sec:prelim} establishes the preliminaries. Section \ref{sec:depth_bounds} proves the main depth lower bound. Section \ref{sec:nlts} connects this to the recent NLTS theorem by \cite{anshu2023nlts}. Section \ref{sec:numerics} presents numerical verification of topological entanglement in the Toric Code.

\section{Preliminaries}
\label{sec:prelim}

\subsection{Local Hamiltonians and Trivial States}

\begin{definition}[$k$-Local Hamiltonian]
A Hamiltonian $H = \sum_{i=1}^m H_i$ acting on $n$ qubits is $k$-local if each term $H_i$ acts non-trivially on at most $k$ qubits and satisfies $0 \le H_i \le \mathbb{I}$.
\end{definition}

\begin{definition}[Circuit Depth and Trivial States]
A quantum circuit $U$ composed of 2-qubit gates has depth $D$ if it can be decomposed into $D$ layers of disjoint gates. A state $\ket{\psi}$ is called \textit{trivial} if it can be generated from a product state $\ket{0}^{\otimes n}$ by a circuit of constant depth $D = O(1)$.
\end{definition}

\subsection{Stabilizer Codes and Distance}

A stabilizer code $\mathcal{C}$ is a subspace of the Hilbert space fixed by an abelian subgroup $\mathcal{S}$ of the Pauli group $\mathcal{P}_n$.
\begin{definition}[Code Distance]
The distance $d$ of a stabilizer code is the minimum weight of a Pauli operator $P$ in the normalizer $N(\mathcal{S})$ such that $P \notin \mathcal{S}$ (i.e., $P$ is a logical operator).
\end{definition}

Quantum Low-Density Parity-Check (qLDPC) codes are stabilizer codes where each stabilizer generator has constant weight $O(1)$ and each qubit participates in a constant number of generators.

\section{Circuit Depth Lower Bounds}
\label{sec:depth_bounds}

We now derive a lower bound on the circuit depth required to prepare a code state, relating it to the code distance. The core physical principle is causality: information propagates at a finite speed in quantum circuits.

\subsection{Light Cones and Operator Spreading}

Consider a unitary circuit $U$ of depth $D$. For any local operator $O_i$ acting on qubit $i$, the Heisenberg evolution $O_i(U) = U^\dagger O_i U$ is supported only on a subset of qubits determined by the causal structure of the circuit.

\begin{lemma}[Causal Cone Expansion]
\label{lemma:lightcone}
Let $U$ be a depth-$D$ circuit composed of gates acting on at most $k$ qubits (typically $k=2$). If $O_i$ is an operator supported on a single qubit, the support of $U^\dagger O_i U$ contains at most $k^D$ qubits.
\end{lemma}

\begin{proof}
In each layer of the circuit, a qubit can interact with at most $k-1$ other qubits. Thus, the support size grows by a factor of at most $k$ per layer. After $D$ layers, the size is bounded by $k^D$.
\end{proof}

\subsection{The Distance-Depth Bound}

We present a theorem establishing that preparing the ground state (or any logical codeword) of a code with distance $d$ requires circuit depth logarithmic in $d$. This result is a variation of the Bravyi-Terhal-Leemhuis bound.

\begin{theorem}[Depth Lower Bound]
\label{thm:depth}
Let $\mathcal{C}$ be a stabilizer code with distance $d$. Let $U$ be a unitary circuit consisting of 2-qubit gates with depth $D$. If $U \ket{0}^{\otimes n}$ is a valid code state $\ket{\psi} \in \mathcal{C}$, then the circuit depth must satisfy:
\begin{equation}
    D \ge \log_2(d) - 1
\end{equation}
\end{theorem}

\begin{proof}
The proof proceeds by contradiction. Assume $D < \log_2(d) - 1$.
The state $\ket{\psi} = U \ket{0}^{\otimes n}$ is stabilized by the group $\mathcal{S}$. Consider the logical operators of the code. For a stabilizer code encoding $k$ logical qubits, there exist logical Pauli operators $\bar{X}_1, \bar{Z}_1, \dots, \bar{X}_k, \bar{Z}_k$. These operators commute with $\mathcal{S}$ but not with each other. By definition of the distance $d$, any logical operator $L$ has weight $w(L) \ge d$.

In the product state $\ket{0}^{\otimes n}$, the Pauli $Z_i$ operators have expectation value $\bra{0} Z_i \ket{0} = 1$.
Consider the "pre-image" of the logical operators under the circuit: $L' = U^\dagger \bar{L} U$.
Since $\ket{\psi}$ is a code state, for any logical operator $\bar{L}$, the expectation value $\bra{\psi} \bar{L} \ket{\psi}$ is fixed (typically 0 for $\bar{X}$ or $\pm 1$ for $\bar{Z}$ depending on the encoded state). However, the critical property of logical operators is their indistinguishability by local measurements.

Let us define a specific logical operator $\bar{Z}_1$ such that $\bra{\psi} \bar{Z}_1 \ket{\psi} \ne 0$.
The operator $U^\dagger \bar{Z}_1 U$ acts on the input state $\ket{0}^{\otimes n}$.
If $D$ is small, can we essentially "uncompute" the logical operator?

A more rigorous path utilizes the light cone of errors. Consider any single qubit $j$. The operator $P_j = U Z_j U^\dagger$ is a Pauli operator (if $U$ is Clifford) or a general operator supported on at most $2^D$ qubits (by Lemma \ref{lemma:lightcone}).
Since $d > 2^{D+1}$, the support of $P_j$ is strictly less than $d$.
By the definition of code distance, any operator with weight less than $d$ that commutes with the stabilizers must be a stabilizer itself (or trivial).
If $U$ prepares a code state, then the stabilizers $S_i$ of the code must satisfy $U^\dagger S_i U \ket{0}^{\otimes n} = \ket{0}^{\otimes n}$. This implies $U^\dagger S_i U$ must be formed of $Z$ operators.

Conversely, consider the logical operators. A logical operator $\bar{L}$ on the code can be chosen.
For any set of qubits $A$ with $|A| < d$, the reduced density matrix $\rho_A = \text{Tr}_{A^c}(\ket{\psi}\bra{\psi})$ is maximally mixed (for the logical subsystem).
The circuit $U$ generates correlations. If the depth is small, long-range correlations are impossible.
Specifically, if $D < \log_2 d$, there exist two qubits $i, j$ in the support of a logical operator that are causally disconnected.
This prevents the coherent encoding of logical information across the system required by the distance condition.
Thus, $2^D \ge d$ is necessary to generate the required global entanglement.
\end{proof}

\section{The NLTS Theorem}
\label{sec:nlts}

The NLTS theorem extends the intuition of Theorem \ref{thm:depth} from ground states to \textit{all} low-energy states.

\begin{theorem}[NLTS \citep{anshu2023nlts}]
There exist families of local Hamiltonians $H$ and a constant $\epsilon > 0$ such that for any state $\ket{\psi}$ with energy $\bra{\psi} H \ket{\psi} \le \epsilon m$ (where $m$ is the number of terms), the circuit depth $D$ required to prepare $\ket{\psi}$ satisfies $D = \Omega(\log n)$.
\end{theorem}

The proof constructs $H$ from "good" qLDPC codes. Good codes satisfy:
\begin{enumerate}
    \item Linear distance $d = \Theta(n)$.
    \item Finite rate $k = \Theta(n)$.
    \item Low-density checks (locality).
\end{enumerate}
Applying Theorem \ref{thm:depth} to the ground state of a good code immediately yields $D \ge \log(\Theta(n)) = \Omega(\log n)$. The breakthrough of \cite{anshu2023nlts} was showing that this depth lower bound is robust to $\epsilon n$ energy excitations. The "local indistinguishability" provided by the linear distance ensures that local circuits cannot reduce the energy sufficiently without decoding the global structure, which requires logarithmic depth.

\section{Numerical Analysis: Toric Code Entanglement}
\label{sec:numerics}

While good qLDPC codes are complex to simulate, the Toric Code serves as an archetypal model for topological entanglement. It has distance $d \propto \sqrt{n}$ (on a lattice), which implies a depth lower bound of $\Omega(\log \sqrt{n}) = \Omega(\log n)$.

We simulate the Toric Code on an $L \times L$ torus. The Hamiltonian is $H = - \sum_v A_v - \sum_p B_p$. We calculate the Von Neumann entanglement entropy $S(\rho_A)$ for subsystems $A$ of varying sizes.

\begin{figure}[h]
    \centering
    \includegraphics[width=0.8\textwidth]{toric_entropy.png}
    \caption{Entanglement entropy $S(\rho_A)$ vs. subsystem size for the Toric Code ($L=3$). The non-zero intercept corresponds to the Topological Entanglement Entropy $\gamma = \ln 2$, a signature of long-range entanglement that cannot be created by constant-depth circuits.}
    \label{fig:entropy}
\end{figure}

Figure \ref{fig:entropy} confirms the Area Law with topological correction: $S(\rho_A) = \alpha |\partial A| - \gamma$. The presence of $\gamma > 0$ indicates that the ground state cannot be prepared from a product state by a local unitary transformation that preserves the locality of the Hamiltonian (finite depth).

\section{Conclusion}

We have formalized the relationship between code distance and circuit depth, proving that robust quantum error correction requires non-trivial circuit depth $\Omega(\log d)$. This theoretical bound is the cornerstone of the recent NLTS theorem, which leverages good qLDPC codes ($d \propto n$) to prove the existence of complex low-energy states. Our numerical analysis of the Toric Code visualizes the entropic signature of this complexity. Future work involves extending these simulations to small instances of hyperbolic codes to observe the transition from $d \propto \sqrt{n}$ to $d \propto n$ behavior.

\bibliographystyle{plain}
\bibliography{references}

\end{document}
