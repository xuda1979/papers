% !TEX program = xelatex
\documentclass[aspectratio=169,11pt]{beamer}

% 中文支持
\usepackage{ctex}
\usepackage{fontspec}

% Beamer主题设置
\usetheme{Madrid}
\usecolortheme{seahorse}
\setbeamertemplate{navigation symbols}{}
\setbeamertemplate{footline}[frame number]

% 包
\usepackage{graphicx}
\usepackage{tikz}
\usepackage{booktabs}
\usepackage{multicol}
\usepackage{xcolor}
\usepackage{hyperref}

% 自定义颜色
\definecolor{cmccblue}{RGB}{0,82,147}
\definecolor{cmccgreen}{RGB}{0,150,94}
\definecolor{cmccred}{RGB}{200,30,50}
\definecolor{highlight}{RGB}{255,215,0}

\setbeamercolor{title}{fg=cmccblue}
\setbeamercolor{frametitle}{fg=cmccblue}
\setbeamercolor{structure}{fg=cmccblue}
\setbeamercolor{block title}{bg=cmccblue,fg=white}
\setbeamercolor{block body}{bg=cmccblue!10}

% 标题信息
\title[年终总结]{\textbf{2025年度工作总结与展望}}
\subtitle{中国移动研究院 · 未来院3室}
\author{许达}
\date{2025年12月}
\institute{中国移动研究院}

\begin{document}

% ==================== 标题页 ====================
\begin{frame}
    \titlepage
\end{frame}

% ==================== 目录 ====================
\begin{frame}{汇报提纲}
    \tableofcontents
\end{frame}

% ==================== 第一部分:工作总结 ====================
\section{工作总结}

% ---------- 总体概述 ----------
\begin{frame}{工作总结 · 总体概述}
    \begin{block}{核心方向}
        深耕 \textbf{AI for Quantum}、\textbf{AI for Science}、\textbf{AI for AI} 交叉领域
    \end{block}
    
    \vspace{0.3cm}
    
    \begin{columns}[T]
        \begin{column}{0.48\textwidth}
            \textbf{软件工程成果}
            \begin{itemize}
                \item 开发软件/工具 \textbf{6个}
                \item 代码总量超过 \textbf{13万行}
                \item 软件著作权申请 \textbf{2项}
            \end{itemize}
        \end{column}
        \begin{column}{0.48\textwidth}
            \textbf{科研成果}
            \begin{itemize}
                \item 投稿一流期刊论文 \textbf{2篇}
                \item 内部技术调研报告 \textbf{20+篇}
                \item 架构设计文档 \textbf{多份}
            \end{itemize}
        \end{column}
    \end{columns}
\end{frame}

% ---------- 技术突破1 ----------
\subsection{技术突破}
\begin{frame}{技术突破(1):量子噪声与AI量子纠错系统}
    \begin{block}{项目定位}
        给量子计算机戴上"AI降噪耳机" —— 面向超导/离子阱量子计算装置的AI增强纠错平台
    \end{block}
    
    \vspace{0.2cm}
    
    \begin{columns}[T]
        \begin{column}{0.55\textwidth}
            \textbf{核心创新}
            \begin{itemize}
                \item \textcolor{cmccgreen}{\textbf{架构革新}}:引入多头隐含注意力(Multi-Head Latent Attention)机制
                \item \textcolor{cmccgreen}{\textbf{效率提升}}:模型学会"抓重点",大幅提升推理速度
                \item \textcolor{cmccgreen}{\textbf{通用适配}}:一套代码适配超导/离子阱两种物理平台
            \end{itemize}
        \end{column}
        \begin{column}{0.42\textwidth}
            \textbf{完成情况}
            \begin{itemize}
                \item[$\checkmark$] 超导模块V1版本完成
                \item[$\checkmark$] 端到端纠错仿真链路验证
                \item[$\circ$] 离子阱模块开发中
                \item[$\circ$] 对接北京市科委量子项目
            \end{itemize}
        \end{column}
    \end{columns}
    
    \vspace{0.2cm}
    \begin{center}
        \colorbox{highlight!30}{\textbf{目标:持续完善系统,争取做到业界最好水平}}
    \end{center}
\end{frame}

% ---------- 技术突破2 ----------
\begin{frame}{技术突破(2):AI自动科研多智能体系统}
    \begin{block}{项目定位}
        让AI"像科学家一样思考" —— 智能体贯穿科研全链路
    \end{block}
    
    \vspace{0.2cm}
    
    \textbf{已实现能力}
    \begin{center}
        \tikz{
            \node[draw,rounded corners,fill=cmccblue!20] at (0,0) {选题};
            \draw[->,thick] (0.6,0) -- (1.4,0);
            \node[draw,rounded corners,fill=cmccblue!20] at (2,0) {检索};
            \draw[->,thick] (2.6,0) -- (3.4,0);
            \node[draw,rounded corners,fill=cmccblue!20] at (4,0) {代码实验};
            \draw[->,thick] (4.9,0) -- (5.7,0);
            \node[draw,rounded corners,fill=cmccblue!20] at (6.5,0) {数据分析};
            \draw[->,thick] (7.4,0) -- (8.2,0);
            \node[draw,rounded corners,fill=cmccblue!20] at (9,0) {论文写作};
        }
    \end{center}
    
    \vspace{0.2cm}
    
    \begin{columns}[T]
        \begin{column}{0.55\textwidth}
            \textbf{核心创新}
            \begin{itemize}
                \item \textcolor{cmccgreen}{\textbf{预算审慎推理}}:AI根据问题难度动态调整"思考时间"
                \item \textcolor{cmccgreen}{\textbf{风险敏感搜索}}:结合控制理论,预判探索风险
                \item \textcolor{cmccgreen}{\textbf{测试时计算}}:强迫AI回答前"多想一会儿"
            \end{itemize}
        \end{column}
        \begin{column}{0.42\textwidth}
            \textbf{应用成效}
            \begin{itemize}
                \item[$\checkmark$] V1版本组内及跨部门试用
                \item[$\checkmark$] 辅助完成多篇论文草稿
                \item[$\checkmark$] 显著缩短科研迭代周期
            \end{itemize}
        \end{column}
    \end{columns}
\end{frame}

% ---------- 技术突破3 ----------
\begin{frame}{技术突破(3):量子科研与代码大模型}
    \begin{block}{项目定位}
        打造懂量子物理的"专用大脑" —— 量子领域专业代码大模型
    \end{block}
    
    \vspace{0.3cm}
    
    \begin{columns}[T]
        \begin{column}{0.55\textwidth}
            \textbf{技术路线}
            \begin{itemize}
                \item 高质量量子专用数据管线构建
                \item 参数高效微调(LoRA)降低训练成本
                \item 推理时算法增强科学推理能力
                \item 强化学习推理增强训练
            \end{itemize}
        \end{column}
        \begin{column}{0.42\textwidth}
            \textbf{当前进展}
            \begin{itemize}
                \item[$\checkmark$] 数据管线开发完成
                \item[$\checkmark$] 训练脚本开发完成
                \item[$\circ$] 第一阶段微调训练启动
            \end{itemize}
        \end{column}
    \end{columns}
    
    \vspace{0.3cm}
    \begin{center}
        \colorbox{cmccgreen!20}{\textbf{填补行业空白:解决量子领域缺乏专用代码大模型的痛点}}
    \end{center}
\end{frame}

% ---------- 组织管理创新 ----------
\subsection{组织管理创新}
\begin{frame}{组织管理创新}
    \begin{columns}[T]
        \begin{column}{0.48\textwidth}
            \begin{block}{工程规范建设}
                \begin{itemize}
                    \item 建立规范化代码管理流程
                    \item Git规范与Code Review机制
                    \item 自动化测试体系搭建
                    \item One-Click训练与评测脚本
                \end{itemize}
            \end{block}
            
            \vspace{0.2cm}
            
            \begin{block}{算力环境建设}
                \begin{itemize}
                    \item 搭建团队自有算力环境
                    \item 约5块Ascend 910B算力卡
                    \item 沉淀可复用训练脚本
                    \item 降低新成员入门门槛
                \end{itemize}
            \end{block}
        \end{column}
        \begin{column}{0.48\textwidth}
            \begin{block}{知识沉淀与共享}
                \begin{itemize}
                    \item 内部技术调研报告 \textbf{约20篇}
                    \item AI量子纠错架构设计文档
                    \item AI-Scientist工作流说明
                    \item 定期分享AI领域最新进展
                \end{itemize}
            \end{block}
            
            \vspace{0.2cm}
            
            \begin{block}{协作机制创新}
                \begin{itemize}
                    \item 技术分享会定期举办
                    \item Pair-Programming实践
                    \item 跨部门智能体试用推广
                    \item 帮助同事提升科研效率
                \end{itemize}
            \end{block}
        \end{column}
    \end{columns}
\end{frame}

% ---------- 人员培养 ----------
\subsection{人员培养}
\begin{frame}{人员培养}
    \begin{block}{角色定位}
        作为技术专家,在"量子×AI软件工程"方向发挥带头作用
    \end{block}
    
    \vspace{0.3cm}
    
    \begin{columns}[T]
        \begin{column}{0.48\textwidth}
            \textbf{技术指导}
            \begin{itemize}
                \item 软件架构设计指导
                \item 代码质量控制把关
                \item 工程工具选型建议
                \item 开发流程规范培训
            \end{itemize}
        \end{column}
        \begin{column}{0.48\textwidth}
            \textbf{能力培养方式}
            \begin{itemize}
                \item 技术分享会讲授
                \item Pair-Programming辅导
                \item Code Review反馈指导
                \item 技术文档编写示范
            \end{itemize}
        \end{column}
    \end{columns}
    
    \vspace{0.3cm}
    
    \begin{center}
        \colorbox{cmccblue!15}{
            \textbf{目标:培养"量子+AI"复合型人才,建立常态化技术分享机制}
        }
    \end{center}
\end{frame}

% ==================== 第二部分:问题与反思 ====================
\section{问题与反思}
\begin{frame}{问题与反思}
    \begin{columns}[T]
        \begin{column}{0.48\textwidth}
            \begin{alertblock}{当前短板}
                \begin{itemize}
                    \item 对公司内部业务场景理解需加深
                    \item 科研与业务结合度有提升空间
                    \item 产业化需求理解有待加强
                \end{itemize}
            \end{alertblock}
        \end{column}
        \begin{column}{0.48\textwidth}
            \begin{alertblock}{科研推进难点}
                \begin{itemize}
                    \item 外部商业模型成本高昂
                    \item 大模型训练算力资源紧张
                    \item 多方案对比实验受限
                \end{itemize}
            \end{alertblock}
        \end{column}
    \end{columns}
    
    \vspace{0.4cm}
    
    \begin{block}{针对性改进措施}
        \begin{enumerate}
            \item 将量子与AI技术更紧密对接具体业务问题
            \item 深入研究提高大模型训练效率、降低训练费用的方法
            \item 深入研究推理时算法,提高大模型能力
            \item 开发部署自研科研模型,解决"卡脖子"问题
        \end{enumerate}
    \end{block}
\end{frame}

% ==================== 第三部分:未来工作思考 ====================
\section{未来工作思考}

% ---------- 技术突破思考 ----------
\subsection{技术突破思考}
\begin{frame}{未来工作思考 · 技术突破方向}
    \begin{columns}[T]
        \begin{column}{0.32\textwidth}
            \begin{block}{\small AI量子纠错系统}
                \begin{itemize}
                    \footnotesize
                    \item 完善系统架构与算法
                    \item 提升纠错准确率与速度
                    \item 形成v1.0文档与Benchmark
                    \item 争取做到业界最好水平
                \end{itemize}
            \end{block}
        \end{column}
        \begin{column}{0.32\textwidth}
            \begin{block}{\small 量子大模型}
                \begin{itemize}
                    \footnotesize
                    \item 发布v0.x → v1.0版本
                    \item 量子编程任务达可用水平
                    \item 产出2-3篇算法论文
                    \item 填补行业空白
                \end{itemize}
            \end{block}
        \end{column}
        \begin{column}{0.32\textwidth}
            \begin{block}{\small 科研智能体系统}
                \begin{itemize}
                    \footnotesize
                    \item 接入自研科研推理模型
                    \item 形成标杆级自动化案例
                    \item 推广全院使用
                    \item 提升科研生产力
                \end{itemize}
            \end{block}
        \end{column}
    \end{columns}
    
    \vspace{0.3cm}
    
    \begin{block}{2026年科研目标}
        \begin{itemize}
            \item 完成多篇重要问题论文投稿
            \item 持续保持高质量工程标准
            \item 在"AI for Science"领域建立专业化标准
        \end{itemize}
    \end{block}
\end{frame}

% ---------- 组织管理思考 ----------
\subsection{组织管理与人员培养思考}
\begin{frame}{未来工作思考 · 组织管理与人员培养}
    \begin{columns}[T]
        \begin{column}{0.48\textwidth}
            \begin{block}{组织管理创新思考}
                \begin{itemize}
                    \item \textbf{流程优化}
                    \begin{itemize}
                        \item 完善敏捷开发流程
                        \item 强化跨部门协作机制
                        \item 建立项目里程碑管理
                    \end{itemize}
                    \item \textbf{资源整合}
                    \begin{itemize}
                        \item 算力资源池化管理
                        \item 数据资产规范化
                        \item 工具链标准化建设
                    \end{itemize}
                \end{itemize}
            \end{block}
        \end{column}
        \begin{column}{0.48\textwidth}
            \begin{block}{人员培养思考}
                \begin{itemize}
                    \item \textbf{人才梯队建设}
                    \begin{itemize}
                        \item 培养"量子+AI"复合型人才
                        \item 建立导师制培养机制
                        \item 鼓励技术创新与尝试
                    \end{itemize}
                    \item \textbf{能力提升路径}
                    \begin{itemize}
                        \item 常态化技术分享机制
                        \item 内外部培训结合
                        \item 项目实战锻炼
                    \end{itemize}
                \end{itemize}
            \end{block}
        \end{column}
    \end{columns}
    
    \vspace{0.3cm}
    
    \begin{center}
        \colorbox{cmccgreen!20}{
            \textbf{目标:提升团队整体工程与算法能力,打造高水平技术团队}
        }
    \end{center}
\end{frame}

% ---------- 资源诉求 ----------
\begin{frame}{资源需求与支撑}
    \begin{block}{当前资源状况}
        \begin{itemize}
            \item 算力:约5块Ascend 910B算力卡
            \item 可满足基础实验,但难以支撑多模型并行与大规模实验
        \end{itemize}
    \end{block}
    
    \vspace{0.3cm}
    
    \begin{columns}[T]
        \begin{column}{0.48\textwidth}
            \begin{block}{算力资源需求}
                \begin{itemize}
                    \item 增加算力卡数量
                    \item 支撑多方案并行实验
                    \item 加速模型训练迭代
                \end{itemize}
            \end{block}
        \end{column}
        \begin{column}{0.48\textwidth}
            \begin{block}{其他支撑需求}
                \begin{itemize}
                    \item 加强与业务部门对接
                    \item 真机实验资源协调
                    \item 跨部门协作支持
                \end{itemize}
            \end{block}
        \end{column}
    \end{columns}
\end{frame}

% ==================== 总结页 ====================
\section{总结}
\begin{frame}{总结}
    \begin{center}
        \begin{tikzpicture}
            % 左边:工作总结
            \node[draw,rounded corners,fill=cmccblue!20,minimum width=4.5cm,minimum height=3.5cm,align=center] at (-4,0) {
                \textbf{2025工作总结}\\[0.2cm]
                \footnotesize
                $\bullet$ 三大核心项目推进\\
                $\bullet$ 13万+行代码产出\\
                $\bullet$ 2篇论文投稿\\
                $\bullet$ 工程规范体系建设\\
                $\bullet$ 人才培养机制建立
            };
            
            % 箭头
            \draw[->,very thick,cmccblue] (-1.2,0) -- (1.2,0);
            
            % 右边:未来展望
            \node[draw,rounded corners,fill=cmccgreen!20,minimum width=4.5cm,minimum height=3.5cm,align=center] at (4,0) {
                \textbf{2026工作展望}\\[0.2cm]
                \footnotesize
                $\bullet$ 真机部署验证\\
                $\bullet$ 量子大模型v1.0发布\\
                $\bullet$ 科研智能体推广\\
                $\bullet$ 复合型人才培养\\
                $\bullet$ 高水平论文产出
            };
        \end{tikzpicture}
    \end{center}
    
    \vspace{0.5cm}
    
    \begin{center}
        \Large\textbf{聚焦"量子×大模型×智能体",为未来技术路线奠定关键基座}
    \end{center}
\end{frame}

% ==================== 致谢页 ====================
\begin{frame}
    \begin{center}
        \vspace{2cm}
        {\Huge\textbf{感谢聆听}}
        
        \vspace{1cm}
        
        {\Large 敬请批评指正}
        
        \vspace{1.5cm}
        
        {\normalsize 中国移动研究院 · 未来院3室}\\
        {\normalsize 许达}\\
        {\normalsize 2025年12月}
    \end{center}
\end{frame}

\end{document}
