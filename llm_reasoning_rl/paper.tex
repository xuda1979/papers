\documentclass{article}
\usepackage{geometry}
\usepackage{amsmath, amssymb}
\usepackage{graphicx}
\usepackage{hyperref}
\usepackage{natbib}

\title{Advancing LLM Reasoning through Reinforcement Learning: A Review and Novel Algorithms}
\author{Jules}
\date{\today}

\begin{document}

\maketitle

\begin{abstract}
Large Language Models (LLMs) have demonstrated impressive capabilities in natural language understanding and generation. However, their performance in complex tasks requiring multi-step reasoning, mathematical problem solving, and code generation remains an area of active research. This paper reviews the current state of reinforcement learning (RL) methods applied to LLMs for enhancing these capabilities. We critically analyze existing approaches such as PPO, GRPO, and iterative fine-tuning. Furthermore, we propose novel algorithmic frameworks designed to overcome the limitations of current methods, focusing on improving sample efficiency, stability, and the ability to generalize across diverse reasoning domains.
\end{abstract}

\section{Introduction}
The advent of Large Language Models (LLMs) has revolutionized the field of Artificial Intelligence. Despite their success, LLMs often struggle with tasks that require rigorous logic, such as mathematics and programming. Reinforcement Learning (RL) has emerged as a promising avenue for aligning LLMs with complex objectives and improving their reasoning skills.

\section{Review of Existing Methods}
\subsection{Reinforcement Learning from Human Feedback (RLHF)}
RLHF has been the standard for aligning models. We discuss its application and limitations in reasoning tasks.

\subsection{Process Supervision}
Recent works have highlighted the importance of rewarding the reasoning process rather than just the final outcome.

\section{Proposed Methodology}
We propose a new family of algorithms that leverage [Insert High-Level Idea, e.g., hierarchical RL, intrinsic motivation, or Monte Carlo Tree Search integration] to better guide the model through complex reasoning paths.

\section{Experiments and Results}
[Placeholder for experimental setup and expected results]

\section{Conclusion}
Summary of contributions and future directions.

\bibliographystyle{plain}
\bibliography{references}

\end{document}
