\section{Framework 5: Derived Algebraic Geometry of Gauge Moduli}
\label{sec:derived}
%=============================================================================

\subsection{Derived Enhancement of Configuration Space}

\begin{definition}[Derived Configuration Stack]
Let $\mathbf{R}\mathcal{M}_\Lambda$ denote the \textbf{derived moduli stack} of 
gauge field configurations:
\[
\mathbf{R}\mathcal{M}_\Lambda = [T^*\mathcal{A}_\Lambda / \mathcal{G}_\Lambda]
\]
where $\mathcal{A}_\Lambda = SU(N)^{|\text{edges}|}$ and $\mathcal{G}_\Lambda = SU(N)^{|\text{vertices}|}$.

The derived structure carries:
\begin{enumerate}
\item A $(-1)$-shifted symplectic form (BV-BRST structure)
\item A perfect obstruction theory
\item A virtual fundamental class
\end{enumerate}
\end{definition}

\begin{theorem}[Virtual Localization for Mass Gap]
\label{thm:virtual-localization}
The partition function localizes to:
\[
Z_\Lambda(\beta) = \int_{[\mathbf{R}\mathcal{M}]^{\text{vir}}} e^{-S/\hbar} = 
\sum_{\text{fixed pts}} \frac{e^{-S_{\text{fixed}}/\hbar}}{e(\text{normal})}
\]
where the sum is over critical points (flat connections) and $e(\text{normal})$ 
is the equivariant Euler class of the normal bundle.
\end{theorem}

\begin{proof}
This follows from Kontsevich's virtual localization formula applied to the 
BRST-exact action:
\[
S = \{Q, \Psi\}
\]
where $Q$ is the BRST operator and $\Psi$ is the gauge-fixing fermion.

The localization sum has only the trivial connection contributing at strong 
coupling, giving:
\[
Z_\Lambda \xrightarrow{\beta \to 0} \frac{\text{Vol}(\mathcal{G}_\Lambda)}{\text{Vol}(\mathcal{A}_\Lambda)} 
\cdot (1 + O(\beta))
\]
which is finite and non-zero.
\end{proof}

\subsection{Obstruction to Phase Transition}

\begin{theorem}[Derived Obstruction to Critical Points]
\label{thm:derived-obstruction}
The derived moduli stack $\mathbf{R}\mathcal{M}_\Lambda$ has:
\[
H^i(\mathbf{R}\mathcal{M}, \mathcal{O}) = \begin{cases}
\mathbb{C} & i = 0 \\
0 & i < 0
\end{cases}
\]
for any $\beta > 0$. This implies no phase transition.
\end{theorem}

\begin{proof}
A phase transition would correspond to a jump in the cohomology 
$H^{-1}(\mathbf{R}\mathcal{M}, \mathcal{O})$, representing obstructions.

By deformation invariance of derived geometry, $H^{-1}$ is constant in $\beta$.

At $\beta = \infty$ (classical limit), the moduli space is smooth and 
$H^{-1} = 0$.

By constancy, $H^{-1} = 0$ for all $\beta$, hence no phase transition.
\end{proof}

%=============================================================================



