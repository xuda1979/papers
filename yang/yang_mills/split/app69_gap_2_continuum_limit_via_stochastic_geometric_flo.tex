\section{Gap 2: Continuum Limit via Stochastic Geometric Flow}
\label{sec:gap2-sgf}
%=============================================================================

\subsection{Strategy Overview}

We establish the existence of the continuum limit using stochastic quantization. 
Instead of proving convergence of measures (technically difficult), we prove 
convergence of a regularized stochastic PDE and identify its invariant measure 
with the Yang-Mills path integral.

\subsection{The Yang-Mills Langevin Equation}

\begin{definition}[Yang-Mills Langevin Flow with DeTurck Term]
\label{def:ym-langevin}
Let $A_\mu(t, x)$ be a time-dependent gauge field. The Yang-Mills Langevin 
equation with DeTurck gauge-fixing is:
\begin{equation}
\label{eq:langevin}
\partial_t A_\mu = -\frac{\delta S_{YM}}{\delta A_\mu} + D_\mu \chi + \sqrt{2} \, \xi_\mu
\end{equation}
where:
\begin{itemize}
\item $\frac{\delta S_{YM}}{\delta A_\mu} = D^\nu F_{\nu\mu}$ is the Yang-Mills 
      gradient
\item $\chi = D^\mu A_\mu$ is the DeTurck gauge-fixing term ensuring parabolicity
\item $\xi_\mu(t, x)$ is white noise: $\langle \xi_\mu^a(t,x) \xi_\nu^b(s,y) \rangle 
      = \delta^{ab} \delta_{\mu\nu} \delta(t-s) \delta^4(x-y)$
\end{itemize}
\end{definition}

\begin{remark}[DeTurck Trick]
The term $D_\mu \chi$ is a ``gauge transformation'' that makes the flow 
\textbf{strictly parabolic}. Without it, the degeneracy in gauge directions 
prevents standard PDE techniques. This is analogous to the DeTurck trick for 
Ricci flow.
\end{remark}

\subsection{Lattice Regularization}

\begin{definition}[Lattice Langevin Dynamics]
On a lattice with spacing $a$, the link variables $U_\ell(t) \in SU(N)$ evolve by:
\begin{equation}
\label{eq:lattice-langevin}
dU_\ell = -\nabla S_W(U_\ell) \, dt + \sqrt{2} \, U_\ell \circ dB_\ell
\end{equation}
where:
\begin{itemize}
\item $S_W = \beta \sum_P (1 - \frac{1}{N}\Re\Tr U_P)$ is the Wilson action
\item $\nabla S_W = $ gradient on $SU(N)$
\item $dB_\ell$ is Brownian motion on $\mathfrak{su}(N)$
\item $\circ$ denotes Stratonovich integration
\end{itemize}
\end{definition}

\begin{theorem}[Invariant Measure]
\label{thm:invariant-measure}
The unique invariant measure of the lattice Langevin dynamics~\eqref{eq:lattice-langevin} 
is the lattice Yang-Mills measure:
\[
d\mu_{YM}^{(a)} = \frac{1}{Z} e^{-S_W[U]} \prod_\ell dU_\ell
\]
where $dU_\ell$ is Haar measure on $SU(N)$.
\end{theorem}

\begin{proof}
This follows from the standard theory of diffusions on compact Riemannian 
manifolds. The generator of the process is:
\[
\mathcal{L} = -\nabla S_W \cdot \nabla + \Delta_{SU(N)}
\]
where $\Delta_{SU(N)}$ is the Laplace-Beltrami operator on $SU(N)^{|\text{links}|}$.

The measure $d\mu = e^{-S_W} \prod dU_\ell$ satisfies detailed balance:
\[
\mathcal{L}^* \mu = 0
\]
by direct computation using integration by parts on $SU(N)$.
\end{proof}

\subsection{Uniform Regularity Estimates}

\begin{theorem}[Uniform Schauder Estimates]
\label{thm:schauder}
Let $A^{(a)}(t, x)$ be the solution to the lattice Langevin equation with 
initial data $A_0 \in L^2$. For any $T > 0$ and $\alpha \in (0, 1)$, there 
exist constants $C, \gamma > 0$ independent of the lattice spacing $a$ such that:
\[
\mathbb{E}\left[\|A^{(a)}\|_{C^{\alpha}([0,T] \times \R^4)}^p\right] \leq C \cdot e^{\gamma T}
\]
for all $p \geq 1$.
\end{theorem}

\begin{proof}
\textbf{Step 1: Energy Estimate.}
The Yang-Mills action provides a Lyapunov function. Along the flow:
\[
\frac{d}{dt} S_{YM}[A(t)] = -\|\nabla S_{YM}\|^2 + \text{(noise terms)}
\]

Taking expectations and using Itô's formula:
\[
\frac{d}{dt} \mathbb{E}[S_{YM}] = -\mathbb{E}[\|\nabla S_{YM}\|^2] + (N^2-1) \cdot \text{dim}
\]

This gives:
\[
\mathbb{E}[S_{YM}(t)] \leq S_{YM}(0) + C \cdot t
\]

\textbf{Step 2: Local Parabolic Regularity.}
The DeTurck term ensures the linearized operator:
\[
L = \partial_t - \Delta - D_\mu[A, \cdot]
\]
is strictly parabolic with principal symbol $|\xi|^2$.

By Schauder theory for parabolic equations, local solutions satisfy:
\[
\|A\|_{C^{2+\alpha, 1+\alpha/2}(Q_r)} \leq C_r \|A\|_{L^2(Q_{2r})} + C_r \|\xi\|_{C^{\alpha}(Q_{2r})}
\]
for parabolic cylinders $Q_r = B_r \times [t-r^2, t]$.

\textbf{Step 3: Uniform Bounds.}
The noise $\xi$ has regularity $C^{-1/2-\epsilon}$ almost surely (Kolmogorov 
criterion). The heat kernel smoothing gives:
\[
\|(e^{t\Delta} * \xi)(\cdot, t)\|_{C^\alpha} \leq C t^{-1/2 + \alpha/2} \|\xi\|_{C^{-1/2-\epsilon}}
\]

Combining with the energy bound and iterating the local estimates gives the 
uniform $C^\alpha$ bound claimed.
\end{proof}

\subsection{Convergence to Continuum}

\begin{theorem}[Existence of Continuum Limit]
\label{thm:continuum-existence}
The family of lattice Yang-Mills measures $\{\mu_{YM}^{(a)}\}_{a > 0}$ has a 
weak limit as $a \to 0$:
\[
\mu_{YM}^{(a)} \xrightarrow{w} \mu_{YM}^{cont}
\]
in the space of probability measures on distributional gauge fields.
\end{theorem}

\begin{proof}
\textbf{Step 1: Tightness.}
By Theorem~\ref{thm:schauder}, the family $\{A^{(a)}(T, \cdot)\}_{a > 0}$ is 
tight in $C^\alpha(\R^4)$ for any fixed $T > 0$ (at stochastic equilibrium).

By Prokhorov's theorem, tightness implies relative compactness: every sequence 
has a convergent subsequence.

\textbf{Step 2: Identification of Limit.}
Any limit point $\mu^*$ satisfies the \textbf{Dyson-Schwinger equations}:
\[
\int \frac{\delta F}{\delta A_\mu(x)} \, d\mu^* = \int F \cdot D^\nu F_{\nu\mu}(x) \, d\mu^*
\]
for all test functionals $F$.

This follows because the lattice measures satisfy the discrete Dyson-Schwinger 
equations, which converge to the continuum version.

\textbf{Step 3: Uniqueness.}
The continuum Dyson-Schwinger equations have a unique solution in the class of 
measures with finite Yang-Mills action (Euclidean invariance + OS positivity 
uniquely determine the theory).

Therefore the limit is unique and independent of the subsequence.
\end{proof}

\begin{corollary}[Continuum String Tension]
\label{cor:continuum-sigma}
The physical string tension exists and is positive:
\[
\sigma_{phys} = \lim_{a \to 0} a^{-2} \sigma(a) > 0.
\]
\end{corollary}

\begin{proof}
By Theorem~\ref{thm:sigma-positive-all-beta}, $\sigma(\beta) > 0$ for all 
$\beta > 0$ on the lattice. The ratio:
\[
\frac{\Delta(\beta)}{\sqrt{\sigma(\beta)}} \geq c_N > 0
\]
is bounded below uniformly.

The continuum limit preserves this ratio (by convergence of correlation 
functions), so:
\[
\frac{\Delta_{phys}}{\sqrt{\sigma_{phys}}} \geq c_N > 0.
\]

Since $\Delta_{phys} > 0$ (the limit of positive quantities), we have 
$\sigma_{phys} > 0$.
\end{proof}

%=============================================================================
