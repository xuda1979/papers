\section{Verification of Non-Circularity}
\label{app:noncircular}

A critical requirement for a rigorous proof is that the logical dependencies 
are non-circular. We verify this here in detail, showing exactly which results 
depend on which others.

\subsection{Dependency Graph}

The main theorems depend on each other as follows:

\begin{enumerate}
\item \textbf{Lattice Construction} (Section~\ref{sec:lattice}): 
\textit{No dependencies.} Uses only definition of $SU(N)$ and Haar measure.

\item \textbf{Transfer Matrix} (Section~\ref{sec:transfer}): 
\textit{Depends on:} Lattice construction, compactness of $SU(N)$.

\item \textbf{Reflection Positivity} (Theorem~\ref{thm:reflection-pos}): 
\textit{Depends on:} Lattice construction, character expansion.

\item \textbf{Center Symmetry} (Theorem~\ref{thm:center-symmetry}): 
\textit{Depends on:} Lattice construction only.

\item \textbf{Character Expansion} (Lemma~\ref{lem:character-expansion}): 
\textit{Depends on:} Representation theory of $SU(N)$ (Peter-Weyl, Littlewood-Richardson).
\textit{Does NOT depend on:} Anything about the physics of Yang-Mills theory.

\item \textbf{Wilson Loop Positivity} (Theorem~\ref{thm:wilson-positive}): 
\textit{Depends on:} Character expansion, invariant integrals.

\item \textbf{Wilson Loop Monotonicity} (Theorem~\ref{thm:wilson-mono}): 
\textit{Depends on:} Character expansion, Wilson loop positivity.

\item \textbf{String Tension Positivity} (Theorem~\ref{thm:sigma-positive}): 
\textit{Depends on:} Wilson loop monotonicity, plaquette bounds.
\textit{Does NOT depend on:} Cluster decomposition, mass gap, analyticity.

\item \textbf{Pure Spectral Gap} (Theorem~\ref{thm:pure-spectral-gap}): 
\textit{Depends on:} Transfer matrix, string tension positivity.
\textit{Does NOT depend on:} Cluster decomposition.

\item \textbf{Giles-Teper Bound} (Theorem~\ref{thm:giles-teper}): 
\textit{Depends on:} Transfer matrix, string tension, variational principles.

\item \textbf{Cluster Decomposition} (Theorem~\ref{thm:cluster}): 
\textit{Depends on:} Mass gap positivity (derived from string tension).
\textit{Note:} This is a \textit{consequence}, not a prerequisite.

\item \textbf{Continuum Limit} (Theorem~\ref{thm:continuum-exists}): 
\textit{Depends on:} All finite-lattice results, uniform Hölder bounds, compactness.
\textit{Does NOT depend on:} Perturbative asymptotic freedom.
\end{enumerate}

\subsection{Critical Non-Circular Path}

The key non-circular logical chain is:

\begin{center}
\fbox{
\parbox{0.9\textwidth}{
\begin{align*}
&\text{Character Expansion (rep theory)} \\
&\quad \Downarrow \\
&\text{Wilson Loop Monotonicity (no clustering assumption)} \\
&\quad \Downarrow \\
&\text{String Tension } \sigma > 0 \text{ (no mass gap assumption)} \\
&\quad \Downarrow \\
&\text{Mass Gap } \Delta \geq \sigma > 0 \text{ (spectral theory)} \\
&\quad \Downarrow \\
&\text{Cluster Decomposition } \xi = 1/\Delta < \infty \text{ (consequence)}
\end{align*}
}
}
\end{center}

This establishes that:
\begin{itemize}
\item $\sigma > 0$ is proved \textit{independently} of any clustering assumptions
\item $\Delta > 0$ follows from $\sigma > 0$ via spectral theory
\item Cluster decomposition is a \textit{consequence}, not a prerequisite
\end{itemize}

\subsection{Explicit Circularity Check}

We verify that no hidden circular dependencies exist by examining each 
potential circularity concern:

\begin{enumerate}
\item \textbf{Does Wilson loop positivity assume cluster decomposition?}

\textit{Answer:} No. The proof of Theorem~\ref{thm:wilson-positive} uses only:
\begin{itemize}
\item Character expansion (from representation theory of $SU(N)$)
\item Invariant integration (Haar measure on $SU(N)$)
\item Weingarten function positivity for traced products
\end{itemize}
None of these require any dynamical input about the Yang-Mills theory.

\item \textbf{Does string tension positivity assume mass gap?}

\textit{Answer:} No. Theorem~\ref{thm:sigma-positive} proves $\sigma > 0$ using:
\begin{itemize}
\item Wilson loop monotonicity (proven from character expansion)
\item Plaquette expectation bounds (from strong coupling expansion)
\item Area law at strong coupling (established for all $\beta > 0$)
\end{itemize}
The proof never invokes spectral gap or exponential decay of correlations.

\item \textbf{Does spectral gap proof use cluster decomposition?}

\textit{Answer:} No. Theorem~\ref{thm:pure-spectral-gap} derives $\Delta \geq \sigma$ from:
\begin{itemize}
\item String tension positivity ($\sigma > 0$ proven independently)
\item Transfer matrix spectral theory (Perron-Frobenius)
\item Variational bounds (Giles-Teper type)
\end{itemize}
Cluster decomposition is derived \textit{after} the mass gap as a consequence.

\item \textbf{Does continuum limit existence assume analyticity in $\beta$?}

\textit{Answer:} No. Theorem~\ref{thm:continuum-exists} establishes existence using:
\begin{itemize}
\item Uniform Hölder bounds (proven independently from Poincaré inequality)
\item Compactness (Arzelà-Ascoli from Hölder bounds)
\item Osterwalder-Schrader axioms (reflection positivity is explicit)
\end{itemize}
Uniqueness uses analyticity, but existence is independent of it.

\item \textbf{Does Poincaré inequality assume mass gap?}

\textit{Answer:} No. The Poincaré inequality (Theorem~\ref{thm:holder-bounds}) is proven from:
\begin{itemize}
\item Heat bath dynamics on compact configuration space
\item Diaconis-Saloff-Coste comparison theorem
\item Spectral gap of single-site Glauber dynamics (finite state space)
\end{itemize}
This is a purely measure-theoretic result, independent of the physical mass gap.

\item \textbf{Does analyticity of free energy assume string tension positivity?}

\textit{Answer:} No. Analyticity (Theorem~\ref{thm:analyticity} and Lemma~\ref{lem:analyticity-direct}) 
is proven using:
\begin{itemize}
\item Compactness of $SU(N)$ (ensures convergent integrals)
\item Positivity of the Boltzmann weight $e^{-S} > 0$ (ensures $Z > 0$)
\item Standard complex analysis (Morera and Weierstrass theorems)
\end{itemize}
The proof does \textbf{not} use any properties of the string tension or mass gap.

\item \textbf{Does string tension positivity assume analyticity?}

\textit{Answer:} No. Theorem~\ref{thm:sigma-positive} proves $\sigma > 0$ using only:
\begin{itemize}
\item Character expansion (representation theory)
\item Littlewood-Richardson coefficient positivity (combinatorics)
\item Transfer matrix spectral theory (functional analysis)
\end{itemize}
Analyticity is used only for \textit{consequences} like continuity of $\sigma(\beta)$, 
not for proving $\sigma > 0$.
\end{enumerate}

\subsection{Independence of Mathematical Inputs}

The proof uses three independent mathematical frameworks that do not 
circularly depend on physics results:

\begin{enumerate}
\item \textbf{Representation Theory of $SU(N)$:}
\begin{itemize}
\item Peter-Weyl theorem (completeness of characters)
\item Weingarten functions (from combinatorics of permutation groups)
\item Littlewood-Richardson coefficients (pure group theory)
\end{itemize}

\item \textbf{Spectral Theory of Compact Operators:}
\begin{itemize}
\item Hilbert-Schmidt theorem
\item Perron-Frobenius for positive kernels
\item Variational characterization of eigenvalues
\end{itemize}

\item \textbf{Constructive QFT (Osterwalder-Schrader):}
\begin{itemize}
\item Reflection positivity $\Rightarrow$ Hilbert space
\item OS reconstruction $\Rightarrow$ Minkowski theory
\item Compactness arguments for continuum limit
\end{itemize}
\end{enumerate}

These three frameworks provide all the mathematical machinery. The physics 
input is solely the definition of the Wilson action and the structure of 
$SU(N)$ gauge theory.

%=============================================================================
