\section{Introduction: Alternative Mathematical Methods}
\label{sec:resolution-intro}
%=============================================================================

In Part I and Part II, we established the mass gap for $SU(2)$ and $SU(3)$ 
using the Bessel--Nevanlinna method (Theorems~\ref{thm:bessel-su2} and 
\ref{thm:bessel-su3}) combined with the GKS character expansion 
(Theorem~\ref{thm:sigma-positive}) and the Giles--Teper bound 
(Theorem~\ref{thm:giles-teper}). These proofs are complete and rigorous.

This Part III presents \textbf{alternative mathematical approaches} that provide 
independent verification and additional insight. These methods were developed 
in parallel and offer different perspectives on the same results:

\begin{enumerate}[label=\textbf{Method \arabic*:}, leftmargin=2.5cm]
\item \textbf{Stochastic geometry}: Vortex tension via optimal transport
\item \textbf{Spectral geometry}: Giles-Teper via flux tube analysis  
\item \textbf{Tropical geometry}: GKS inequality via valuation theory
\item \textbf{Concentration inequalities}: Uniform bounds via measure concentration
\end{enumerate}

While the Bessel--Nevanlinna approach (Part I) provides the most direct proof 
for $SU(2)$ and $SU(3)$, these alternative methods extend to general 
compact gauge groups and provide quantitative bounds not available from 
the analytic approach alone.

%=============================================================================



