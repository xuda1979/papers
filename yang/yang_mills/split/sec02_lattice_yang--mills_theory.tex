\section{Lattice Yang--Mills Theory}
\label{sec:lattice}
%=============================================================================

\subsection{The Lattice}

Let $\Lambda_L = (\mathbb{Z}/L\mathbb{Z})^4$ be a four-dimensional periodic 
lattice with $L^4$ sites. We work with lattice spacing $a > 0$, which will 
eventually be taken to zero.

\begin{definition}[Lattice Structure]
The lattice $\Lambda_L$ consists of:
\begin{enumerate}[label=(\roman*)]
\item \textbf{Sites}: $x \in (\mathbb{Z}/L\mathbb{Z})^4$, total $L^4$ sites
\item \textbf{Links (edges)}: Oriented pairs $(x, x+\hat{\mu})$ for $\mu \in \{1,2,3,4\}$, 
total $4L^4$ oriented links
\item \textbf{Plaquettes}: Elementary squares with corners at 
$(x, x+\hat{\mu}, x+\hat{\mu}+\hat{\nu}, x+\hat{\nu})$ for $\mu < \nu$, 
total $6L^4$ plaquettes (choosing orientation)
\end{enumerate}
\end{definition}

\subsection{Gauge Field Configuration}

To each oriented edge (link) $e$ of the lattice, we assign a group element 
$U_e \in SU(N)$. For the reversed edge $-e$, we set $U_{-e} = U_e^{-1}$.

The space of all gauge field configurations is:
\[
\mathcal{C} = \{U : \text{edges} \to SU(N)\} \cong SU(N)^{4L^4}
\]

\begin{remark}[Configuration Space Topology]
The configuration space $\mathcal{C}$ is a compact, connected, simply-connected 
manifold (product of copies of $SU(N)$, which has these properties). This 
compactness is essential for well-definedness of the path integral.
\end{remark}

\subsection{Haar Measure}

\begin{definition}[Haar Measure on $SU(N)$]
The Haar measure $dU$ on $SU(N)$ is the unique left- and right-invariant 
probability measure:
\[
\int_{SU(N)} f(VU) \, dU = \int_{SU(N)} f(UV) \, dU = \int_{SU(N)} f(U) \, dU
\]
for all $V \in SU(N)$ and integrable $f$.
\end{definition}

\begin{lemma}[Haar Measure Properties]
\label{lem:haar-props}
The Haar measure satisfies:
\begin{enumerate}[label=(\roman*)]
\item \textbf{Normalization}: $\int_{SU(N)} dU = 1$
\item \textbf{Inversion invariance}: $\int f(U^{-1}) \, dU = \int f(U) \, dU$
\item \textbf{Character orthogonality}: 
$\int_{SU(N)} \chi_\lambda(U) \overline{\chi_\mu(U)} \, dU = \delta_{\lambda\mu}$
for irreducible characters $\chi_\lambda, \chi_\mu$
\item \textbf{Peter-Weyl theorem}: 
$L^2(SU(N), dU) = \bigoplus_\lambda V_\lambda \otimes V_\lambda^*$
as representations of $SU(N) \times SU(N)$
\end{enumerate}
\end{lemma}

\subsection{Wilson Action}

For each elementary square (plaquette) $p$ with edges $e_1, e_2, e_3, e_4$ 
traversed in order, define the plaquette variable:
\[
W_p = U_{e_1} U_{e_2} U_{e_3}^{-1} U_{e_4}^{-1}
\]

\begin{definition}[Wilson Action]
The Wilson action is:
\[
S_\beta[U] = \frac{\beta}{N} \sum_{\text{plaquettes } p} \Re\Tr(1 - W_p)
\]
where $\beta = 2N/g^2$ is the inverse coupling constant.
\end{definition}

\begin{remark}[Continuum Limit of Wilson Action]
As $a \to 0$ with $A_\mu(x) = (U_{x,\mu} - 1)/(iga)$ held fixed:
\[
\Re\Tr(1 - W_p) = \frac{a^4 g^2}{2N} \Tr(F_{\mu\nu}^2) + O(a^6)
\]
where $F_{\mu\nu} = \partial_\mu A_\nu - \partial_\nu A_\mu + ig[A_\mu, A_\nu]$ 
is the field strength. Thus:
\[
S_\beta[U] \xrightarrow{a \to 0} \frac{1}{4} \int d^4x \, \Tr(F_{\mu\nu}F^{\mu\nu})
\]
the classical Yang-Mills action.
\end{remark}

\subsection{Partition Function and Expectation Values}

The partition function is:
\[
Z_L(\beta) = \int \prod_{\text{edges } e} dU_e \, e^{-S_\beta[U]}
\]
where $dU_e$ is the normalized Haar measure on $SU(N)$.

For any gauge-invariant observable $\mathcal{O}$, the expectation value is:
\[
\langle \mathcal{O} \rangle_\beta = \frac{1}{Z_L(\beta)} 
\int \prod_e dU_e \, \mathcal{O}[U] \, e^{-S_\beta[U]}
\]

\subsection{Gauge Invariance}

\begin{definition}[Gauge Transformation]
A gauge transformation is a map $g : \text{sites} \to SU(N)$. It acts on link 
variables by:
\[
U_{x,\mu}^g = g_x U_{x,\mu} g_{x+\hat{\mu}}^{-1}
\]
\end{definition}

\begin{lemma}[Gauge Invariance of Wilson Action]
The Wilson action is gauge-invariant: $S_\beta[U^g] = S_\beta[U]$ for all 
gauge transformations $g$.
\end{lemma}

\begin{proof}
Under gauge transformation, the plaquette variable transforms as:
\[
W_p^g = g_x W_p g_x^{-1}
\]
(conjugation by $g$ at the base point $x$ of the plaquette). Since the trace 
is invariant under conjugation: $\Tr(W_p^g) = \Tr(g_x W_p g_x^{-1}) = \Tr(W_p)$.
\end{proof}

\begin{definition}[Gauge-Invariant Observable]
An observable $\mathcal{O}[U]$ is gauge-invariant if $\mathcal{O}[U^g] = \mathcal{O}[U]$ 
for all gauge transformations $g$.
\end{definition}

\begin{example}[Wilson Loop]
The Wilson loop $W_C = \frac{1}{N}\Tr\left(\prod_{e \in C} U_e\right)$ along 
any closed contour $C$ is gauge-invariant.
\end{example}

%=============================================================================



