% Packages
\usepackage[utf8]{inputenc}
\usepackage[T1]{fontenc}
\usepackage{amsmath,amsthm,amssymb,amsfonts}
\usepackage{mathtools}
\usepackage{mathrsfs}
\usepackage{enumitem}
\usepackage{calc}
\usepackage[margin=1in]{geometry}
% NOTE: This project compiles a very large master document (hundreds of inputs
% with many theorem-like environments). On some TeX distributions (notably
% MiKTeX with default memory), the LaTeX hook/socket machinery used by hyperref
% can trigger "TeX capacity exceeded" during compilation.
%
% To keep the build stable, we avoid loading hyperref in the master build and
% provide minimal no-op fallbacks for common hyperref macros.
%
% If you want clickable links for a final PDF, switch back to hyperref and/or
% compile with an engine/memory configuration that can handle the full document.
\providecommand{\href}[2]{#2}
\providecommand{\url}[1]{\texttt{#1}}
\providecommand{\hyperref}[2][]{#2}
\providecommand{\autoref}[1]{\ref{#1}}
\providecommand{\pdfbookmark}[3][]{% hyperref-only
  % no-op when hyperref is disabled
}
\providecommand{\texorpdfstring}[2]{#1}
\providecommand{\phantomsection}{}
% NOTE: To keep this very large document within TeX memory limits, we avoid
% hook-heavy/large packages unless they're truly needed.
\usepackage{tcolorbox}
\usepackage{cancel}
\usepackage{slashed}
\usepackage{tikz}
% \usepackage{listings}  % (disabled: high memory; re-enable if code blocks require it)
\usepackage{booktabs}
\usetikzlibrary{positioning,arrows.meta,shapes.geometric}

% Code listing style (only applies if listings is enabled)
%\lstset{
%  basicstyle=\small\ttfamily,
%  breaklines=true,
%  frame=single,
%  numbers=left,
%  numberstyle=\tiny
%}

% Theorem environments
\newtheorem{theorem}{Theorem}[section]
\newtheorem{lemma}[theorem]{Lemma}
\newtheorem{proposition}[theorem]{Proposition}
\newtheorem{corollary}[theorem]{Corollary}
\newtheorem{definition}[theorem]{Definition}
\newtheorem{example}[theorem]{Example}
\newtheorem{conjecture}[theorem]{Conjecture}
\newtheorem{problem}[theorem]{Open Problem}
\newtheorem{construction}[theorem]{Construction}
\newtheorem{strategy}[theorem]{Strategy}
\newtheorem{verification}[theorem]{Verification}
\newtheorem{inequality}[theorem]{Inequality}
\newtheorem{danger}[theorem]{Danger}

\theoremstyle{remark}
\newtheorem{remark}[theorem]{Remark}
\newtheorem{summary}[theorem]{Summary}

% Define \part if missing (some split files use it even under article).
% Make this conditional to avoid clashes if a class/package already defines \part.
\makeatletter
\@ifundefined{part}{
  \newcommand{\part}[1]{%
    \clearpage
    	hispagestyle{plain}%
    \null\vfil
    \begin{center}
      {\Large\bfseries ##1}
    \end{center}
    \vfil
    \clearpage
  }
}{}
\makeatother

% Operators
\DeclareMathOperator{\Tr}{Tr}
\DeclareMathOperator{\Spec}{Spec}
\renewcommand{\Re}{\operatorname{Re}}
\DeclareMathOperator{\Ric}{Ric}
\DeclareMathOperator{\Hess}{Hess}
\DeclareMathOperator{\LSI}{LSI}

% Custom commands for integrated content
\newcommand{\SU}{\mathrm{SU}}
\newcommand{\Hilb}{\mathcal{H}}
\newcommand{\E}{\mathbb{E}}
\newcommand{\Z}{\mathbb{Z}}
\newcommand{\re}{\operatorname{Re}}
\newcommand{\tr}{\operatorname{tr}}
\newcommand{\GaugeGrp}{\mathcal{G}}
\newcommand{\dmu}{d\mu}
\newcommand{\R}{\mathbb{R}}
\newcommand{\C}{\mathbb{C}}
\newcommand{\N}{\mathbb{N}}
\newcommand{\cT}{\mathcal{T}}
\newcommand{\cC}{\mathcal{C}}
\newcommand{\cG}{\mathcal{G}}



