\section{Mathematical Prerequisites}
\label{app:prerequisites}

This appendix summarizes the key mathematical theorems used in the proof.

\subsection{Functional Analysis}

\begin{theorem}[Spectral Theorem for Compact Self-Adjoint Operators]
Let $T : \mathcal{H} \to \mathcal{H}$ be a compact self-adjoint operator on a 
Hilbert space. Then:
\begin{enumerate}[label=(\roman*)]
\item $T$ has a countable set of eigenvalues $\{\lambda_n\}$ with $|\lambda_n| \to 0$
\item Each nonzero eigenvalue has finite multiplicity
\item $\mathcal{H} = \ker(T) \oplus \overline{\text{span}\{e_n : Te_n = \lambda_n e_n\}}$
\item $T = \sum_n \lambda_n |e_n\rangle\langle e_n|$ (spectral decomposition)
\end{enumerate}
\end{theorem}

\begin{theorem}[Jentzsch's Theorem (Generalized Perron-Frobenius)]
Let $T$ be a compact positive integral operator on $L^2(X, \mu)$ with continuous 
strictly positive kernel $K(x,y) > 0$. Then:
\begin{enumerate}[label=(\roman*)]
\item The spectral radius $r(T) > 0$ is an eigenvalue
\item $r(T)$ is simple (multiplicity 1)
\item The eigenfunction for $r(T)$ can be chosen strictly positive
\end{enumerate}
\end{theorem}

\begin{theorem}[Courant-Fischer Min-Max Principle]
For a self-adjoint operator $H$ with eigenvalues $E_0 \leq E_1 \leq E_2 \leq \cdots$:
\[
E_n = \min_{\dim V = n+1} \max_{\psi \in V, \|\psi\|=1} \langle \psi | H | \psi \rangle
\]
\end{theorem}

\subsection{Representation Theory of $SU(N)$}

\begin{theorem}[Peter-Weyl Theorem]
Let $G$ be a compact Lie group with Haar measure $dg$. Then:
\[
L^2(G, dg) = \bigoplus_{\lambda \in \hat{G}} V_\lambda \otimes V_\lambda^*
\]
where $\hat{G}$ is the set of equivalence classes of irreducible representations 
and $V_\lambda$ is the representation space for $\lambda$.
\end{theorem}

\begin{theorem}[Character Orthogonality]
For irreducible representations $\lambda, \mu$ of a compact group $G$:
\[
\int_G \chi_\lambda(g) \overline{\chi_\mu(g)} \, dg = \delta_{\lambda\mu}
\]
where $\chi_\lambda(g) = \Tr(D^\lambda(g))$ is the character.
\end{theorem}

\begin{theorem}[Littlewood-Richardson Rule]
For $SU(N)$ representations labeled by Young diagrams $\lambda, \mu$:
\[
V_\lambda \otimes V_\mu = \bigoplus_\nu N_{\lambda\mu}^\nu V_\nu
\]
where $N_{\lambda\mu}^\nu \in \mathbb{Z}_{\geq 0}$ (non-negative integers).
\end{theorem}

\subsection{Constructive Field Theory}

\begin{theorem}[Osterwalder-Schrader Reconstruction]
Let $\{S_n\}$ be a family of Schwinger functions satisfying:
\begin{enumerate}[label=(OS\arabic*)]
\item Temperedness
\item Euclidean covariance
\item Reflection positivity
\item Symmetry
\item Cluster property
\end{enumerate}
Then there exists a unique Wightman QFT whose Euclidean continuation gives $\{S_n\}$.
\end{theorem}

\begin{theorem}[Griffiths-Ruelle Theorem]
For a lattice system with interaction $\Phi$, the following are equivalent:
\begin{enumerate}[label=(\roman*)]
\item Uniqueness of infinite-volume Gibbs measure
\item Differentiability of pressure as function of parameters
\item Absence of spontaneous symmetry breaking
\end{enumerate}
\end{theorem}

\subsection{Markov Chain Comparison Theorems}

\begin{theorem}[Diaconis-Saloff-Coste Comparison]
Let $P$ and $Q$ be two reversible Markov chains on a finite state space with 
the same stationary distribution $\pi$. If there exists $A > 0$ such that 
for all edges $(x, y)$ of $Q$:
\[
\pi(x) Q(x,y) \leq A \cdot \text{path}_{P}(x, y)
\]
where $\text{path}_P(x,y)$ is the probability flow from $x$ to $y$ in $P$, then:
\[
\text{gap}(Q) \geq \frac{\text{gap}(P)}{A \cdot \ell^*}
\]
where $\ell^*$ is the maximum path length.
\end{theorem}

This theorem is used in the proof of the Poincaré inequality from spectral 
gap (Theorem~\ref{thm:holder-bounds}) to relate the heat bath dynamics gap 
to the transfer matrix gap.




