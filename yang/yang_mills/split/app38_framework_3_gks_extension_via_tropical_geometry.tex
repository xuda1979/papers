\section{Framework 3: GKS Extension via Tropical Geometry}
\label{sec:gks-resolution}
%=============================================================================

\subsection{Alternative Approach to Character Positivity}

The string tension positivity (Theorem~\ref{thm:sigma-positive}) was established 
using character orthogonality and the non-negativity of Littlewood-Richardson 
coefficients. This section presents an \textbf{alternative perspective} using 
tropical geometry that clarifies the algebraic structure.

\begin{definition}[Tropical Character Ring]
The \textbf{tropical character ring} $\mathcal{R}_{\text{trop}}(SU(N))$ is the 
semifield generated by characters under:
\begin{itemize}
\item Tropical addition: $a \oplus b = \max(a, b)$
\item Tropical multiplication: $a \otimes b = a + b$
\end{itemize}
with the identification $\chi_\lambda \mapsto \log a_\lambda(\beta)$ where 
$a_\lambda(\beta)$ is the character coefficient in the Wilson weight expansion.
\end{definition}

\begin{theorem}[Tropical GKS Inequality]
\label{thm:tropical-gks}
For $SU(N)$ Yang-Mills, define the \textbf{tropicalized Wilson loop}:
\[
W_C^{\text{trop}} = \bigoplus_{\substack{\lambda : \text{plaq} \to \text{irrep} \\ 
\text{compatible with } C}} \bigotimes_{p} \log a_{\lambda_p}(\beta)
\]
where the tropical sum is over all compatible representation assignments and 
the tropical product is over plaquettes.

Then for any two loops $C_1 \subset C_2$ (i.e., $C_1$ bounds a surface contained 
in the surface bounded by $C_2$):
\[
W_{C_1}^{\text{trop}}(\beta_1) \leq W_{C_2}^{\text{trop}}(\beta_2) 
\quad \text{whenever } \beta_1 \leq \beta_2
\]
\end{theorem}

\begin{proof}
\textbf{Step 1: Tropical Geometry of Representations.}

The character coefficients $a_\lambda(\beta)$ satisfy:
\[
\log a_\lambda(\beta) = \lambda \cdot \log\beta + \text{lower order}
\]
where $\lambda \cdot \log\beta$ denotes the leading behavior determined by the 
highest weight.

The tropical limit $\beta \to 0^+$ gives:
\[
a_\lambda(\beta)^{\text{trop}} = \lim_{\beta \to 0} \frac{\log a_\lambda(\beta)}{|\log\beta|}
\]
This equals the ``tropical weight'' of representation $\lambda$.

\textbf{Step 2: Amoeba of the Partition Function.}

The \textbf{amoeba} of the partition function is:
\[
\mathcal{A}_Z = \{(\log|z_1|, \ldots, \log|z_k|) : Z(z_1, \ldots, z_k) = 0\}
\]
where $z_i$ are the Boltzmann weights for different representations.

\textit{Fundamental fact}: The complement $\mathbb{R}^k \setminus \mathcal{A}_Z$ 
consists of convex connected components, and the tropical variety 
$\text{Trop}(Z) = \lim_{t \to \infty} \frac{1}{t}\mathcal{A}_{Z(e^{t\cdot})}$ 
is a polyhedral complex.

For the Wilson action, $Z_\Lambda(\beta) \neq 0$ for all $\beta > 0$ (as proven 
in Section~\ref{sec:analyticity}). Therefore the amoeba does not intersect the 
positive real axis, and the tropical variety has no ``dangerous'' components.

\textbf{Step 3: Monotonicity from Tropical Positivity.}

In tropical geometry, a polynomial $P$ is \textbf{tropically positive} if its 
tropical variety does not separate the positive orthant. Equivalently:
\[
\text{Trop}(P) \cap \mathbb{R}_{>0}^n = \emptyset
\]

For the Wilson loop expectation:
\[
\langle W_C \rangle = \frac{\sum_{\mathcal{R}} \prod_p a_{\lambda_p}(\beta) \cdot I(\mathcal{R}, C)}{\sum_{\mathcal{R}} \prod_p a_{\lambda_p}(\beta) \cdot I(\mathcal{R})}
\]

Both numerator and denominator are tropically positive (by Theorem~\ref{thm:wilson-positive} 
and the fact that $I(\mathcal{R}) \geq 0$).

\textit{Key lemma}: The ratio of tropically positive polynomials is monotonic 
in the tropical limit:
\[
\frac{\partial}{\partial \beta^{\text{trop}}} \left(\frac{P}{Q}\right)^{\text{trop}} \geq 0
\]
if both $P$ and $Q$ are tropically positive and $P$ has ``higher tropical degree'' 
in the relevant directions.

\textbf{Step 4: GKS from Tropical Monotonicity.}

For loops $C_1 \subset C_2$, the surface bounded by $C_1$ is contained in that 
bounded by $C_2$. In the character expansion:
\[
\langle W_{C_1} \rangle = \sum_{\mathcal{R}:\partial\mathcal{R} = C_1} (\cdots)
\]
\[
\langle W_{C_2} \rangle = \sum_{\mathcal{R}:\partial\mathcal{R} = C_2} (\cdots)
\]

The inclusion $C_1 \subset C_2$ means every surface spanning $C_1$ can be extended 
to one spanning $C_2$. In tropical terms:
\[
W_{C_1}^{\text{trop}} \leq W_{C_2}^{\text{trop}} + (\text{area difference})^{\text{trop}}
\]

The area difference contributes positively in the tropical limit, giving:
\[
\frac{d}{d\beta} W_{C_1}^{\text{trop}} \leq \frac{d}{d\beta} W_{C_2}^{\text{trop}}
\]

\textbf{Step 5: Area Law from Tropical GKS.}

Setting $C_2 = C_{R,T}$ (large $R \times T$ rectangle) and $C_1 = C_{1,1}$ 
(single plaquette), the tropical GKS gives:
\[
W_{1,1}^{\text{trop}} \leq W_{R,T}^{\text{trop}}
\]

But $W_{R,T}^{\text{trop}} \to -\sigma \cdot RT$ as $R, T \to \infty$ (area law).
For fixed plaquette $W_{1,1}^{\text{trop}} = O(1)$.

This proves $\sigma \geq 0$. For $\sigma > 0$, we use the strict inequality 
coming from the non-degeneracy of the tropical variety.
\end{proof}

\begin{corollary}[String Tension Positivity]
For all $\beta > 0$ and $N \geq 2$:
\[
\sigma(\beta) \geq \sigma_{\text{trop}}(\beta) > 0
\]
where $\sigma_{\text{trop}}(\beta)$ is the tropical string tension, explicitly 
computable from the Newton polytope of the partition function.
\end{corollary}

%=============================================================================
