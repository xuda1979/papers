\documentclass[11pt,a4paper]{article}

% Packages
\usepackage[utf8]{inputenc}
\usepackage[T1]{fontenc}
\usepackage{amsmath,amsthm,amssymb,amsfonts}
\usepackage{mathtools}
\usepackage{mathrsfs}
\usepackage{enumitem}
\usepackage[margin=1in]{geometry}
\usepackage[pdfusetitle,hidelinks]{hyperref}
\usepackage{tcolorbox}

% Theorem environments
\newtheorem{theorem}{Theorem}[section]
\newtheorem{lemma}[theorem]{Lemma}
\newtheorem{proposition}[theorem]{Proposition}
\newtheorem{corollary}[theorem]{Corollary}
\newtheorem{definition}[theorem]{Definition}
\newtheorem{remark}[theorem]{Remark}

% Operators
\DeclareMathOperator{\Tr}{Tr}
\newcommand{\SU}{\mathrm{SU}}
\newcommand{\R}{\mathbb{R}}
\newcommand{\C}{\mathbb{C}}
\newcommand{\Z}{\mathbb{Z}}

\title{Rigorous Status Classification\\
\large Clear Distinction Between Proven, Framework, and Speculative Results}
\author{Generated Solution}
\date{December 17, 2025}

\begin{document}
\maketitle

\section{The Consistency Problem}

The original manuscript had inconsistent claims about rigor level:

\begin{itemize}
\item \textbf{Title:} "A Rigorous Proof for Four-Dimensional Gauge Theory"
\item \textbf{Abstract:} "Complete rigorous proof" 
\item \textbf{Body:} "These are heuristic arguments, not rigorous proofs" (Section 14.5)
\item \textbf{Conclusion:} Mix of "proven" and "conditional" statements
\end{itemize}

This damages scientific credibility. Here's how to fix it:

\section{Proposed Status Classification System}

\subsection{Four Rigour Levels}

\begin{tcolorbox}[colback=green!10!white, colframe=green!60!black]
🟢 \textbf{RIGOROUS:} Meets highest mathematical standards
\begin{itemize}
\item All steps justified by established theorems
\item No gaps in the logic 
\item Could be published in pure math journals
\item \textit{Example:} Cluster expansion for strong coupling
\end{itemize}
\end{tcolorbox}

\begin{tcolorbox}[colback=blue!10!white, colframe=blue!60!black]
🔵 \textbf{FRAMEWORK:} Mathematically sound but conditional 
\begin{itemize}
\item Logic is correct given stated assumptions
\item Assumptions are plausible but unproven
\item Could be published in theoretical physics journals
\item \textit{Example:} Gap transport via LSI methods
\end{itemize}
\end{tcolorbox}

\begin{tcolorbox}[colback=orange!10!white, colframe=orange!60!black]
🟠 \textbf{HEURISTIC:} Educated guess with some justification
\begin{itemize}
\item Contains logical gaps or unverified steps
\item Physical intuition is reasonable
\item Requires further development for rigor
\item \textit{Example:} Weak coupling Gaussian approximation
\end{itemize}
\end{tcolorbox}

\begin{tcolorbox}[colback=red!10!white, colframe=red!60!black]
🔴 \textbf{SPECULATIVE:} Exploratory, possibly wrong
\begin{itemize}
\item Major conceptual leaps
\item May contain category errors
\item Research direction only
\item \textit{Example:} Tropical geometry applications
\end{itemize}
\end{tcolorbox}

\subsection{Revised Title and Abstract}

\textbf{OLD TITLE:} "Mass Gap and Confinement for Adjoint QCD: A Rigorous Proof for Four-Dimensional Gauge Theory"

\textbf{NEW TITLE:} "The Yang-Mills Mass Gap: A Systematic Framework with Rigorous Foundations"

\textbf{OLD ABSTRACT BEGINNING:} "We present a complete rigorous proof..."

\textbf{NEW ABSTRACT BEGINNING:}
\begin{quote}
We present a \textbf{systematic mathematical framework} for proving the Yang-Mills mass gap, combining rigorous foundations with conditional frameworks that address the key technical challenges...
\end{quote}

\section{Section-by-Section Status Update}

\subsection{Rigorous Sections (Keep Claims)}

\begin{enumerate}
\item \textbf{Finite Volume Mass Gap} 🟢
   - Transfer matrix spectrum
   - Perron-Frobenius theorem
   - Compactness arguments
   - \textit{Status:} \textcolor{green}{\textbf{PROVEN}}

\item \textbf{Strong Coupling Regime} 🟢
   - Cluster expansion convergence
   - Uniform bounds for $\beta < \beta_c$
   - Exponential decay of correlations
   - \textit{Status:} \textcolor{green}{\textbf{PROVEN}}

\item \textbf{Center Symmetry} 🟢
   - Polyakov loop vanishing
   - $\mathbb{Z}_N$ symmetry preservation
   - \textit{Status:} \textcolor{green}{\textbf{PROVEN}}
\end{enumerate}

\subsection{Framework Sections (Conditional Claims)}

\begin{enumerate}
\item \textbf{Intermediate Coupling Control} 🔵
   - Hierarchical LSI method
   - Variance-based transport
   - Bootstrap approach
   - \textit{Status:} \textcolor{blue}{\textbf{CONDITIONAL}} on uniform bounds

\item \textbf{RG Bridge Construction} 🔵
   - Multi-scale analysis
   - Gap transport between regimes
   - \textit{Status:} \textcolor{blue}{\textbf{CONDITIONAL}} on individual regime results

\item \textbf{Reflection Positivity Applications} 🔵
   - Giles-Teper bounds
   - String tension positivity
   - \textit{Status:} \textcolor{blue}{\textbf{CONDITIONAL}} on infinite volume survival
\end{enumerate}

\subsection{Heuristic Sections (Downgrade Claims)}

\begin{enumerate}
\item \textbf{Weak Coupling Gaussian Dominance} 🟠
   - Perturbative effective theory
   - $O(1/\beta^2)$ corrections
   - \textit{Status:} \textcolor{orange}{\textbf{HEURISTIC}} - requires gauge fixing control

\item \textbf{Continuum Limit Existence} 🟠
   - Scale setting procedure
   - Physical limit extraction
   - \textit{Status:} \textcolor{orange}{\textbf{HEURISTIC}} - requires non-perturbative AF
\end{enumerate}

\subsection{Speculative Sections (Remove or Disclaimer)}

\begin{enumerate}
\item \textbf{Tropical Geometry Applications} 🔴
   - \textit{Status:} \textcolor{red}{\textbf{REMOVE}} or clearly mark as non-contributory

\item \textbf{Perfectoid Spaces} 🔴
   - \textit{Status:} \textcolor{red}{\textbf{REMOVE}} completely

\item \textbf{Advanced Stochastic Methods} 🔴
   - \textit{Status:} \textcolor{red}{\textbf{SIMPLIFY}} to standard techniques
\end{enumerate}

\section{Revised Conclusion Structure}

Instead of claiming a complete proof, the conclusion should clearly state what has been achieved:

\subsection{What Is Rigourously Established}

\begin{quote}
We have \textbf{rigorously proven}:
\begin{enumerate}
\item Finite-volume mass gaps exist for all $\beta > 0$
\item Strong coupling mass gaps survive the infinite volume limit  
\item Center symmetry ensures confinement in finite volume
\item The logical chain: confinement $\Rightarrow$ string tension $\Rightarrow$ mass gap
\end{enumerate}
\end{quote}

\subsection{What Framework Provides}

\begin{quote}
We have developed a \textbf{systematic framework} that:
\begin{enumerate}
\item Extends control to intermediate coupling via hierarchical methods
\item Transports gaps between different coupling regimes
\item Connects lattice and continuum physics via RG flow
\item Reduces the mass gap problem to specific technical conditions
\end{enumerate}
\end{quote}

\subsection{What Remains Open}

\begin{quote}
The framework is \textbf{conditional} on:
\begin{enumerate}
\item Non-perturbative validity of asymptotic freedom
\item Uniform bounds surviving the infinite volume limit
\item Reflection positivity preservation in the continuum
\item Technical conditions on LSI constants and variance bounds
\end{enumerate}

These represent the \textbf{current frontier} of rigorous gauge theory, not fundamental obstacles.
\end{quote}

\section{Language Guidelines for Revision}

\subsection{Replace Absolute Claims}

\textbf{AVOID:}
\begin{itemize}
\item "We prove..."
\item "It follows that..."
\item "Therefore, the mass gap exists..."
\end{itemize}

\textbf{USE INSTEAD:}
\begin{itemize}
\item "We establish a framework where..."  
\item "Conditional on [assumptions], it follows that..."
\item "If [conditions] hold, then the mass gap exists..."
\end{itemize}

\subsection{Introduce Conditional Statements}

\textbf{FORMULA:}
```latex
\begin{theorem}[Conditional Mass Gap]
\label{thm:conditional-gap}
If conditions (A), (B), (C) hold, then the mass gap exists and satisfies [properties].
\end{theorem}
```

\textbf{ALWAYS FOLLOW WITH:}
```latex
\begin{remark}[Status of Conditions]
Condition (A) is \textcolor{green}{proven}, condition (B) is \textcolor{blue}{framework-level}, condition (C) is \textcolor{orange}{heuristic}.
\end{remark}
```

\section{Adjoint Fermion Interpolation (The Core Physical Argument)}

This is the strongest part of the argument and should be emphasized:

\begin{theorem}[Adjoint QCD Mass Gap - The Physical Core]
The Adjoint QCD interpolation $\SU(N) + $ adjoint Majorana provides a physical bridge where:
\begin{enumerate}
\item At $m = 0$: Theory is supersymmetric and gapped (established)
\item At $m = \infty$: Fermions decouple, giving pure Yang-Mills
\item For all $m > 0$: Adjoint fermions preserve center symmetry
\item No phase transition can occur (no order parameter)
\end{enumerate}

\textit{Status:} \textcolor{blue}{\textbf{FRAMEWORK}} - requires proving analyticity in $m$
\end{theorem}

\textbf{This argument should be front and center, not buried in technical sections.}

\section{Sample Text Revision}

\textbf{BEFORE (Section 1):}
\begin{quote}
This paper provides a complete rigorous proof of the Yang-Mills mass gap for $\SU(N)$ gauge theory...
\end{quote}

\textbf{AFTER:}
\begin{quote}
This paper provides a systematic framework for proving the Yang-Mills mass gap for $\SU(N)$ gauge theory. While the strong coupling regime admits a complete rigorous treatment, the full problem requires framework-level methods that are conditional on established conjectures in gauge theory...
\end{quote}

\begin{tcolorbox}[colback=green!10!white, colframe=green!50!black]
\textbf{Result of This Revision:}
\begin{itemize}
\item \textbf{Honest:} Clear about what's proven vs conditional
\item \textbf{Credible:} No longer claims more than achieved
\item \textbf{Valuable:} Framework is still a major contribution
\item \textbf{Focused:} Highlights the adjoint fermion insight
\end{itemize}

The paper becomes much more scientifically valuable when it honestly represents its achievements.
\end{tcolorbox}

\end{document}