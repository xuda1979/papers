\documentclass[11pt,a4paper]{article}

\usepackage[utf8]{inputenc}
\usepackage[T1]{fontenc}
\usepackage{amsmath,amsthm,amssymb}
\usepackage{enumitem}
\usepackage[margin=1in]{geometry}
\usepackage{tcolorbox}
\usepackage{xcolor}

\newtheorem{theorem}{Issue}
\theoremstyle{remark}
\newtheorem*{assessment}{Assessment}
\newtheorem*{fix}{Required Fix}

\newtcolorbox{okbox}[1]{colback=green!5!white,colframe=green!75!black,title=#1}
\newtcolorbox{warnbox}[1]{colback=yellow!5!white,colframe=yellow!75!black,title=#1}
\newtcolorbox{errorbox}[1]{colback=red!5!white,colframe=red!75!black,title=#1}

\title{Proof Verification Report:\\Checking Against Referee's Requirements}
\author{Critical Analysis}
\date{December 2025}

\begin{document}

\maketitle

\section{Referee's Requirements (Summary)}

The referee specified that a rigorous proof requires:
\begin{enumerate}
\item A \textbf{gauge-invariant disorder parameter} (vortex free energy / 't Hooft loops)
\item A \textbf{rigorous inequality} (Tomboulis-Yaffe type) converting disorder parameter bounds to Wilson area law
\item A \textbf{proof that the disorder parameter bound holds} along the continuum scaling trajectory
\item \textbf{No circular reasoning} or uncontrolled approximations
\end{enumerate}

\section{Verification of Each Component}

%===========================================================================
\subsection{Component 1: Gauge-Invariant Disorder Parameter}
%===========================================================================

\begin{okbox}{PASS: Definition is Gauge-Invariant}
\textbf{What we have:} Definition~\ref{def:vortex-fe} defines
\[
F_v(\Lambda; \beta) := -\log \frac{Z_{\text{twist}}}{Z_{\text{untwist}}}
\]

\textbf{Referee's requirement:} ``Define a gauge-invariant disorder parameter''

\textbf{Verification:} The ratio of partition functions is manifestly gauge-invariant 
since both $Z_{\text{twist}}$ and $Z_{\text{untwist}}$ are gauge-invariant. 
No gauge-fixing is involved. \checkmark
\end{okbox}

%===========================================================================
\subsection{Component 2: Tomboulis-Yaffe Inequality}
%===========================================================================

\begin{warnbox}{PARTIAL: Proof Sketch, Not Complete Rigor}
\textbf{What we have:} Theorem~\ref{thm:tomboulis-yaffe} claims
\[
\sigma(\beta) \geq \frac{f_v(\beta)}{N}
\]

\textbf{Referee's requirement:} ``Use an existing rigorous inequality''

\textbf{Issues with our proof:}
\begin{enumerate}
\item \textbf{Step 2} claims Wilson loops can be written via character orthogonality as
\[
\langle W_C \rangle = \frac{1}{N} \sum_{k=0}^{N-1} e^{-2\pi i k/N} \cdot \frac{Z_{\text{twist}}(z^k)}{Z}
\]
This identity needs more careful justification. The actual relationship between 
Wilson loops and twisted partition functions is more subtle.

\item \textbf{Step 3} uses ``convexity of free energy in twist angle'' without proof.

\item The original Tomboulis-Yaffe paper proves the inequality for $SU(2)$ using 
specific properties. Extension to $SU(N)$ requires careful verification.
\end{enumerate}

\textbf{Assessment:} The inequality is \textit{likely true} and follows established 
patterns, but our proof is a \textit{sketch} rather than complete rigor.
\end{warnbox}

\begin{fix}
Either:
\begin{itemize}
\item Cite Tomboulis-Yaffe directly and note the extension to $SU(N)$ is standard
\item Provide complete proof with all technical details
\end{itemize}
\end{fix}

%===========================================================================
\subsection{Component 3: $f_v(\beta) > 0$ for All $\beta$}
%===========================================================================

\subsubsection{Strong Coupling: $\beta < \beta_0$}

\begin{okbox}{PASS: Strong Coupling Calculation}
Theorem~\ref{thm:fv-strong} correctly calculates
\[
f_v(\beta) = \beta(1 - \cos(2\pi/N)) + O(\beta^2) > 0
\]
using standard strong coupling expansion. This is rigorous. \checkmark
\end{okbox}

\subsubsection{All $\beta$: Topological Obstruction}

\begin{errorbox}{FAIL: Topological Argument Has Gaps}
\textbf{What we claim:} Theorem~\ref{thm:fv-topological} asserts $f_v(\beta) > 0$ 
for all $\beta \in (0, \infty)$.

\textbf{Our argument:}
\begin{enumerate}
\item If $f_v(\beta_0) = 0$, twisted and untwisted ensembles are equivalent
\item This requires measure to be ``insensitive'' to topological sector
\item Only happens at $\beta = 0$ or $\beta = \infty$
\end{enumerate}

\textbf{CRITICAL GAPS:}
\begin{enumerate}
\item \textbf{Gap 1:} The claim ``$f_v = 0$ implies twisted/untwisted ensembles 
are asymptotically equivalent'' is \textit{not proven}. We show 
$Z_{\text{twist}}/Z_{\text{untwist}} \to 1$, but this doesn't mean the 
\textit{local} properties are the same.

\item \textbf{Gap 2:} The dichotomy ``either flat connections or complete disorder'' 
is stated but not proven. There could be intermediate situations.

\item \textbf{Gap 3:} Step 5 claims ``$c(\beta) > 0$ is continuous'' without 
proof. Continuity of free energy in $\beta$ is standard, but strict positivity 
requires an argument.

\item \textbf{Gap 4 (FUNDAMENTAL):} The argument essentially says ``$f_v = 0$ 
would be weird, so $f_v > 0$''. This is physically intuitive but not mathematically 
rigorous. The referee specifically warned against such heuristics.
\end{enumerate}
\end{errorbox}

\begin{fix}
The topological obstruction argument needs substantial strengthening:
\begin{itemize}
\item Either prove a \textbf{quantitative lower bound} on $f_v(\beta)$ valid 
for all $\beta$, or
\item Use \textbf{monotonicity} combined with strong coupling: if $f_v$ is 
monotone and $f_v(\beta_0) > 0$ for some $\beta_0$, conclude about all $\beta$, or
\item Connect to \textbf{established results} (e.g., lattice QCD showing no 
deconfinement transition at $T=0$)
\end{itemize}
\end{fix}

%===========================================================================
\subsection{Component 4: Controlled Scaling}
%===========================================================================

\begin{errorbox}{FAIL: Circular Reasoning}
\textbf{What we claim:} Theorem~\ref{thm:fv-scaling} asserts
\[
\frac{f_v(\beta(a))}{a^2} \to f_v^{\text{phys}} > 0
\]

\textbf{Our argument:}
\begin{enumerate}
\item Dimensional analysis: $f_v^{\text{phys}} = c_N \Lambda_{\text{QCD}}^2$
\item String tension has finite continuum limit
\item Asymptotic freedom gives scaling
\end{enumerate}

\textbf{CRITICAL GAPS:}
\begin{enumerate}
\item \textbf{Circularity:} Step 2 says ``the string tension has a finite 
continuum limit $\sigma_{\text{phys}} > 0$ (established by lattice QCD 
calculations and supported by Theorem~\ref{thm:fv-topological})''. 

But we're \textit{trying to prove} $\sigma_{\text{phys}} > 0$! We cannot 
use it as input.

\item \textbf{Dimensional analysis is not proof:} Saying $f_v^{\text{phys}} = c_N \Lambda^2$ 
by dimensional analysis only shows the \textit{form}, not that $c_N > 0$.

\item \textbf{No control on decay rate:} Even if $f_v(\beta) > 0$ for all $\beta$, 
we need to show $f_v(\beta)$ doesn't decay \textit{faster} than $a^2$ as 
$\beta \to \infty$. This is the core difficulty.
\end{enumerate}
\end{errorbox}

\begin{fix}
This is the \textbf{hardest part}. Possible approaches:
\begin{itemize}
\item \textbf{RG approach:} Show $f_v(\beta)$ satisfies an RG equation with 
controlled fixed point (Balaban-style)
\item \textbf{Correlation inequality:} Prove $f_v(\beta) \geq c \cdot a(\beta)^2$ 
for some $c > 0$ using correlation inequalities
\item \textbf{Semiclassical:} In weak coupling, vortices are thick objects 
with action $\sim 1/g^2 \sim \beta$, so $f_v \sim \beta \cdot a^2$ which 
gives $f_v/a^2 \sim \beta \to \infty$...but this gives the \textit{wrong} 
scaling for a finite limit!
\end{itemize}
\end{fix}

%===========================================================================
\section{Overall Verdict}
%===========================================================================

\begin{errorbox}{VERDICT: Proof is Incomplete}
\textbf{What follows the referee's framework:}
\begin{itemize}
\item[\checkmark] Disorder parameter is gauge-invariant
\item[\checkmark] Tomboulis-Yaffe inequality structure is correct
\item[\checkmark] Strong coupling calculation is rigorous
\end{itemize}

\textbf{What does NOT meet the standard:}
\begin{itemize}
\item[$\times$] Topological obstruction argument (Theorem~\ref{thm:fv-topological}) 
is heuristic, not rigorous
\item[$\times$] Scaling theorem (Theorem~\ref{thm:fv-scaling}) contains circular 
reasoning
\item[$\times$] No quantitative control on $f_v(\beta)$ at weak coupling
\end{itemize}

\textbf{The fundamental issue:} We have not actually proven the ``single missing 
theorem'' the referee identified. We've set up the framework correctly but the 
key step --- proving $f_v(\beta) > 0$ with controlled scaling --- remains 
incomplete.
\end{errorbox}

%===========================================================================
\section{What Would Make It Rigorous}
%===========================================================================

To complete the proof, we need \textbf{ONE} of the following:

\subsection{Option A: Quantitative Lower Bound on $f_v$}

Prove: For all $\beta > 0$,
\[
f_v(\beta) \geq c \cdot \min(1, a(\beta)^2)
\]
for some universal $c > 0$.

\subsection{Option B: No Phase Transition Theorem}

Prove: The free energy density $f(\beta)$ is analytic in $\beta$ for all 
$\beta \in (0, \infty)$, implying no phase transition.

Combined with $f_v(\beta_0) > 0$ at strong coupling and continuity, this gives 
$f_v(\beta) > 0$ for all $\beta$.

\subsection{Option C: Rigorous RG}

Extend Balaban's RG analysis to:
\begin{enumerate}
\item Include vortex observables
\item Show $f_v$ flows to a positive fixed point value
\item Control all error terms
\end{enumerate}

\subsection{Option D: Accept Lattice QCD Evidence}

If we accept numerical lattice QCD results as ``established'', then:
\begin{itemize}
\item No deconfinement transition at $T = 0$ (only at $T_c > 0$)
\item $\sigma_{\text{phys}} > 0$ measured numerically
\end{itemize}

This makes the proof ``conditional on standard lattice QCD'', which is a 
reasonable position but not a \textit{pure} mathematical proof.

%===========================================================================
\section{Conclusion}
%===========================================================================

The proof as written:
\begin{enumerate}
\item \textbf{Correctly follows} the referee's recommended framework (disorder 
parameters + Tomboulis-Yaffe)
\item \textbf{Is rigorous} in strong coupling
\item \textbf{Has gaps} in the key theorems for general $\beta$:
\begin{itemize}
\item Theorem~\ref{thm:fv-topological} (topological obstruction) is heuristic
\item Theorem~\ref{thm:fv-scaling} (scaling) has circular reasoning
\end{itemize}
\end{enumerate}

\textbf{Recommendation:} The paper should clearly state that the disorder 
parameter framework is the \textit{correct approach} (as the referee suggested), 
but the proof of $f_v(\beta) > 0$ for all $\beta$ with controlled scaling is 
still \textbf{conditional} on either:
\begin{itemize}
\item Absence of zero-temperature phase transition, or
\item Numerical lattice QCD evidence
\end{itemize}

\end{document}
