\documentclass[11pt]{article}
\usepackage{microtype}
\usepackage{booktabs}
\usepackage{amsmath,amssymb,amsfonts}
\usepackage{graphicx}
\usepackage{natbib}
\usepackage[hidelinks]{hyperref}
\usepackage[margin=1in]{geometry}
\title{Inference-Time Hybrid Reasoning for Scientific LLM Agents:\\
Hierarchical, Causal, Bayesian, Analogical, and Counterfactual Integration}
\author{David Xu, Ph.D.\\ General Algorithmic Technologies Company}
\date{\today}
\begin{document}
\maketitle

\begin{abstract}
Large language models (LLMs) excel at pattern completion but lack explicit mechanisms for scientific reasoning. We propose an inference-time orchestration framework that integrates five paradigms---hierarchical planning, causal inference, Bayesian belief updating, analogical structure mapping, and counterfactual simulation---as tool calls under LLM control. The LLM decomposes scientific questions into subproblems, invokes specialized tools, and aggregates their outputs into a final answer. We provide theoretical links to probabilistic graphical models and cognitive science, and report real, reproducible results from a tool-centric synthetic benchmark designed to stress each module in isolation. Our results show improved marginal likelihoods for Bayesian updating compared to a plug-in MLE baseline, reduced mean-squared error for backdoor-adjusted causal estimation versus naive regression in confounded settings, near-perfect counterfactual reconstruction in linear-Gaussian structural causal models, and substantially higher accuracy in analogical mapping via structure-preserving permutations than degree heuristics.
\end{abstract}

\section{Introduction}
LLMs trained via next-token prediction implicitly encode broad statistical knowledge, yet they often struggle with structured scientific tasks that require decomposition, causality, uncertainty quantification, analogical generalization, and counterfactual analysis. A promising direction is \emph{inference-time augmentation}: the LLM acts as a planner and controller that decomposes a query and calls specialized reasoning tools, returning their results to compose a final answer \citep{rajaraman2024orchestration, zelikman2022star, yao2022react}. We formalize a unified framework for five reasoning paradigms that are central to scientific method and show tool-level evaluations.
This paper makes three contributions: (i) a general orchestration operator unifying hierarchical, causal, Bayesian, analogical, and counterfactual tools; (ii) a synthetic benchmark that isolates module competence with reproducible metrics; and (iii) theoretical and empirical analysis connecting probabilistic graphical models and cognitive science.

\section{Related Work}
\paragraph{Hierarchical planning and cognitive architectures.}
Cognitive architectures such as Soar and ACT-R demonstrate how hierarchical control, declarative knowledge, and production rules enable complex behavior \citep{lebiere2009actr,newell1990unified}. In LLMs, constrained decoding and planner-executor patterns emulate these ideas \citep{yao2022react}.
\paragraph{Causality and counterfactuals.}
Structural causal models (SCMs) and $do$-calculus provide identifiability conditions and semantics for interventions and counterfactuals \citep{pearl2009causality, peters2017elements}. Counterfactual inference follows the abduction--action--prediction pipeline \citep{pearl2009causality}.
\paragraph{Bayesian reasoning.}
Probabilistic graphical models and probabilistic programming encapsulate uncertainty and belief updating \citep{gelman2013bayesian, koller2009pgm}.
\paragraph{Analogical reasoning.}
Structure mapping theory formalizes analogical transfer via relational correspondences rather than surface similarity \citep{gentner1983structure, falkenhainer1989structure}. Graph-based analogical mapping underlies cross-domain generalization.

\section{Framework}
Let $f_\theta$ denote an LLM and $\mathcal{T}=\{\mathcal{T}_\mathrm{hier},\mathcal{T}_\mathrm{causal},\mathcal{T}_\mathrm{bayes},\mathcal{T}_\mathrm{analog},\mathcal{T}_\mathrm{cf}\}$ denote reasoning tools.
At inference time, we compute
\begin{equation}
y = \mathcal{R}\!\left(f_\theta, \mathcal{T} \mid x\right),
\end{equation}
where $\mathcal{R}$ is an orchestration operator that parses $x$, produces a structured plan, dispatches sub-queries to tools, and composes $y$ with provenance.

\subsection{Hierarchical Planning}
We model decomposition as a DAG $G=(V,E)$ over subgoals with partial order constraints. The planner selects $G$ by maximizing a structured score $S(G\mid x)$ under schema constraints. Constrained decoding \citep{anderson2023guidance} ensures well-formed JSON plans and tool signatures.

\subsection{Causal Inference}
We assume an SCM $\mathcal{M}=(U,V,F,P(U))$ with interventions defined by the $do(\cdot)$ operator. For backdoor-eligible queries, the average causal effect (ACE) is identified by adjustment:
\begin{equation}
\mathrm{ACE} = \sum_{z} \left( \mathbb{E}[Y \mid X{=}1,Z{=}z] - \mathbb{E}[Y \mid X{=}0,Z{=}z] \right) P(Z{=}z),
\end{equation}
and estimated by regression adjustment or inverse-propensity weighting \citep{peters2017elements}.

\subsection{Bayesian Updating}
Given hypothesis $H$ and evidence $E$, posterior beliefs follow Bayes' rule $P(H\!\mid\!E)\propto P(E\!\mid\!H)P(H)$. For conjugate Beta--Binomial models, the posterior is $\mathrm{Beta}(\alpha{+}k,\beta{+}n{-}k)$, and posterior predictive distributions admit closed forms \citep{gelman2013bayesian}.

\subsection{Analogical Mapping}
We formalize analogy as finding a permutation $m$ maximizing relational consistency between graphs $G_A$ and $G_B$:
\begin{equation}
m^\star = \arg\max_{m} \sum_{(u,v)\in E_A} \mathbb{I}\!\left[(m(u),m(v))\in E_B\right].
\end{equation}
This connects to role discovery and graph matching.

\subsection{Counterfactual Simulation}
Given observation $o$, counterfactuals follow abduction--action--prediction \citep{pearl2009causality}. In linear-Gaussian SCMs, conditional means admit closed-form updates, enabling fast counterfactual queries.

\section{Synthetic Benchmark and Metrics}
We design a tool-centric synthetic benchmark that isolates each module:
\begin{enumerate}
\item \textbf{Bayesian:} Beta--Binomial estimation evaluated by per-trial log marginal likelihood vs a plug-in MLE baseline.
\item \textbf{Causal:} Linear Gaussian SCM with confounding; compare naive OLS $Y\!\sim\!X$ to backdoor-adjusted $Y\!\sim\!X{+}Z$ via MSE to the true structural coefficient $a$.
\item \textbf{Counterfactual:} Linear Gaussian SCM counterfactual mean squared error.
\item \textbf{Analogical:} Accuracy of structure-preserving permutation vs degree-based heuristic on directed acyclic graphs.
\end{enumerate}
This benchmark is reproducible and stresses formal competence of the tool modules that an LLM controller would invoke at inference time.

\section{Results}
Table~\ref{tab:results} summarizes empirical results over thousands of instances. Bayesian updating yields higher marginal likelihood than MLE plug-in estimates; backdoor adjustment reduces causal MSE vs naive regression; counterfactual simulation achieves low error under correct model assumptions; structure mapping strongly outperforms degree heuristics.


\begin{table}[t]
\centering
\begin{tabular}{l r}
\toprule
\textbf{Module / Metric} & \textbf{Score} \\ \midrule
Bayesian per-trial log-likelihood (marginal) & -0.0457 \\
MLE per-trial log-likelihood (plug-in) & -0.5748 \\
Delta (Bayes $-$ MLE) & 0.5292 \\
Causal MSE (naive OLS $Y\!\sim\!X$) & 0.352470 \\
Causal MSE (adjusted $Y\!\sim\!X{+}Z$) & 0.002345 \\
Delta MSE (naive $-$ adjusted) & 0.350125 \\
Counterfactual MSE (linear-Gaussian SCM) & 0.000000 \\
Analogical accuracy (structure mapping) & 1.0000 \\
Analogical accuracy (degree heuristic) & 0.7747 \\
\bottomrule
\end{tabular}
\caption{Tool-centric synthetic benchmark. $n_\text{Bayes}=400$ tasks; $n_\text{Causal}=300$ tasks; Counterfactual cases=1800; Analogical queries=1500. Metrics summarized from \texttt{reasoning\_tools\_benchmark\_summary.csv}.}
\label{tab:results}
\end{table}


Running the benchmark writes summary metrics to \texttt{reasoning_tools_benchmark_summary.csv}.
\section{Theory: Why Integration Helps}
\paragraph{Inductive bias complementarity.}
Each module contributes constraints that reduce the effective hypothesis class $\mathcal{H}$, lowering generalization error bounds while preserving expressivity for scientific tasks.
\paragraph{Information perspective.}
Let $R_i$ be module transcripts; the hybrid transcript $R_\mathrm{hyb}$ increases mutual information with the target $Y$ via complementary views, i.e., $I(Y;R_\mathrm{hyb}) \ge \max_i I(Y;R_i)$ under weak conditional independence assumptions.
\paragraph{Error decomposition.}
For estimators $\hat{\theta}_i$ with bias--variance tradeoffs tailored to substructures (causal, probabilistic, relational), orthogonalization of errors reduces aggregate risk when composed with hierarchical planning.

\section{Discussion and Limitations}
Our evaluation is \emph{tool-centric and synthetic}, designed for offline reproducibility. In a full LLM agent, the planner would call these modules on real datasets (e.g., ScienceQA, PubMedQA) and integrate outputs into natural language answers with calibrated uncertainties. Hybrid reasoning incurs additional compute for tool calls and planning; mitigating latency and enforcing resource-aware scheduling remain open challenges. Future work will (i) plug this orchestration into an LLM controller, (ii) evaluate on real benchmarks, and (iii) extend analogical mapping beyond exact permutations to noisy relational correspondences.

\section{Conclusion}
Inference-time hybrid reasoning via hierarchical, causal, Bayesian, analogical, and counterfactual modules forms a principled path toward scientific LLM agents. Our theory and empirical results support the viability of this approach and provide a reproducible foundation for larger-scale studies.

\bibliographystyle{plainnat}
\bibliography{refs}
\end{document}
