\documentclass[11pt, a4paper]{article}
\usepackage[utf8]{inputenc}
\usepackage{geometry}
\usepackage{amsmath, amssymb}
\usepackage{graphicx}
\usepackage{natbib}
\usepackage{hyperref}

\geometry{margin=1in}

\title{The Synthetic Validity Framework: Epistemological Auditing of Machine-Generated Knowledge}
\author{Research Group on Theoretical AI}
\date{\today}

\begin{document}

\maketitle

\begin{abstract}
The use of Large Language Models (LLMs) to simulate human populations ("silico sampling") is rapidly gaining traction in the social sciences. This paper argues that this practice faces a profound "Epistemological Crisis." We present evidence of a "Demographic Representativeness Gap," where LLM simulations of marginalized groups rely on reductive stereotypes rather than accurate cultural priors. We introduce the "Synthetic Validity Framework," a methodology for auditing the sociological fidelity of machine-generated data. Through large-scale "Demographic Alignment Tests" comparing LLM outputs to the World Values Survey, we quantify "Social Identity Bias" and propose rigorous standards for the admissibility of synthetic data in scientific discourse.
\end{abstract}

\section{Introduction}
"Why interview humans when you can prompt GPT-4?" This provocative question is driving a wave of research replacing human subjects with AI proxies \citep{argyle2023out}. While efficient, this approach rests on the unproven assumption that the model's training data—a snapshot of the internet—is an unbiased mirror of humanity.

We contend it is a "Funhouse Mirror." LLMs capture the \textit{dominant} discourse but often flatten the variance of minority opinions, a phenomenon we term "Opinion Collapse." Furthermore, when prompted to adopt a persona (e.g., "Answer as a Muslim woman"), the model often resorts to "Cultural Hallucination"—producing a caricature based on western stereotypes rather than authentic intra-group diversity \citep{santurkar2023whose}.

\section{Theoretical Framework}

\subsection{Epistemological Auditing}
We propose that synthetic data must pass a "validity audit" before use. We define three criteria:
\begin{enumerate}
    \item \textbf{Distributional Fidelity}: Does the variance of the synthetic sample match the variance of the human population?
    \item \textbf{Intersectional Consistency}: Does the model maintain coherence when simulating intersectional identities (e.g., Black conservative women)?
    \item \textbf{Temporal Stability}: Are the simulated opinions robust to minor prompting variations, or are they fragile artifacts?
\end{enumerate}

\section{Methodology}

\subsection{Demographic Alignment Test (DAT)}
We probe Llama-3 and GPT-4 with 500 questions from the World Values Survey (WVS). We prompt the models with 50 distinct demographic personas.
We calculate the \textbf{Stereotype Amplification Score (SAS)}:
\begin{equation}
    SAS(G) = \frac{\text{Var}(LLM_G)}{\text{Var}(Human_G)}
\end{equation}
If  < 1$, the model is "flattening" the group $. If the mean distance $|| \mu_{LLM} - \mu_{Human} ||$ is large, the model is misaligned.

\section{Preliminary Findings}
Our audit reveals:
\begin{itemize}
    \item \textbf{The WEIRD Bias}: Models default to "Western, Educated, Industrialized, Rich, Democratic" values even when prompted otherwise.
    \item \textbf{Stereotype Caricatures}: When simulating non-western personas, models significantly over-emphasize religious traditionalism compared to real-world data, ignoring secular subgroups within those populations.
\end{itemize}

\section{Conclusion}
The "Sociology of Synthetic Epistemology" is a necessary new field. Without it, we risk polluting the scientific record with "zombie data"—statistically perfect but sociologically hollow. We urge journals to adopt the Synthetic Validity Framework as a prerequisite for publishing LLM-based social science.

\bibliographystyle{plainnat}
\bibliography{references}

\end{document}
