\documentclass[11pt]{amsart}

% Basic theorem environments
\newtheorem{theorem}{Theorem}[section]
\newtheorem{proposition}[theorem]{Proposition}
\newtheorem{lemma}[theorem]{Lemma}
\newtheorem{corollary}[theorem]{Corollary}
\newtheorem{hypothesis}[theorem]{Hypothesis}
\newtheorem{conjecture}[theorem]{Conjecture}

\theoremstyle{definition}
\newtheorem{definition}[theorem]{Definition}
\newtheorem{remark}[theorem]{Remark}
\newtheorem{example}[theorem]{Example}

\title{Towards the Spacetime Penrose Inequality via Jang Deformations and $p$--Harmonic Methods}

\author{Da Xu}

\begin{document}

\maketitle

\begin{abstract}
We describe a program toward a proof of the spacetime Penrose inequality for general asymptotically flat initial data sets satisfying the dominant energy condition. Our approach combines the Jang equation, a conformal deformation to nonnegative scalar curvature, and a $p$--harmonic level-set method extending the work of Agostiniani--Mazzieri--Oronzio to a singular setting. In this paper we formulate and prove several conditional and partial results along this line.

Under a set of explicit analytic hypotheses on the Jang--Lichnerowicz operator, on the smoothing of corners with distributional scalar curvature, and on the behavior of $p$--harmonic potentials on degenerating backgrounds, we derive the spacetime Penrose inequality for a single black hole. The logical dependence on these hypotheses is made completely explicit.

We emphasize that we do \emph{not} claim a complete proof of the spacetime Penrose inequality in full generality here. Several of the analytic inputs needed for such a proof are stated as hypotheses or conjectures and will be developed in subsequent work. The main contribution of the present paper is to isolate these inputs and to show how they fit into a unified geometric framework.
\end{abstract}

\section{Introduction}

In this paper we describe and partially implement a program toward a proof of the spacetime Penrose inequality for general asymptotically flat initial data sets $(M,g,k)$ satisfying the dominant energy condition and containing an outermost apparent horizon.

At the time of writing, the full spacetime Penrose inequality remains open. What is known is the Riemannian (time-symmetric) case $k=0$, established independently by Huisken--Ilmanen and Bray, together with various special and perturbative spacetime results. None of the currently available analytic results on Jang deformations, conformal scalar curvature corrections, or $p$--harmonic level-set flows by itself resolves the general case.

The first version of this manuscript was written under the optimistic working assumption that several analytic ingredients (Fredholm properties of a Jang--Lichnerowicz operator on asymptotically cylindrical backgrounds, quantitative control of corner smoothings with distributional scalar curvature, and Gamma-limits for $p$--harmonic functionals on degenerating manifolds) could be treated as black boxes. In light of subsequent scrutiny, and following the referee's comments, we have revised the paper so that the status of these ingredients is completely explicit. They are now formulated as hypotheses or conjectures, and all downstream geometric statements are correspondingly conditional.

\begin{theorem}[Main theorem]\label{thm:main}
Let $(M,g,k)$ be an asymptotically flat initial data set satisfying the dominant energy condition and containing an outermost marginally outer trapped surface $\Sigma$ with area $|\Sigma|$. Assume that $(M,g,k)$ has a single asymptotically flat end. Then the ADM mass $m_{\mathrm{ADM}}$ of $(M,g,k)$ satisfies
\[
 m_{\mathrm{ADM}} \;\ge\; \sqrt{\frac{|\Sigma|}{16\pi}}.
\]
\end{theorem}

The proof of Theorem~\ref{thm:main} given here is \emph{conditional}: it relies on four analytic hypotheses concerning
\begin{itemize}
  \item the Fredholm and maximum-principle properties of a Jang--Lichnerowicz operator on a manifold with one asymptotically flat end and several asymptotically cylindrical ends;
  \item the smoothing of internal corners while preserving a distributional scalar curvature lower bound and uniform control of the negative part;
  \item the behavior of $p$--harmonic potentials on the resulting family of smooth approximating metrics and their Gamma/Mosco limits as $\varepsilon\to 0$ and $p\to 1$;
  \item the regularity and rigidity theory for weak static vacuum solutions arising in the equality case.
\end{itemize}
These hypotheses are stated precisely in Section~\ref{sec:hypotheses}, and every theorem whose proof uses them is explicitly labeled as such.

The geometric part of the argument---namely, the reduction of the spacetime Penrose inequality to these four analytic hypotheses---is unconditional.
In this sense the present paper should be viewed primarily as an organizing framework and a collection of conditional reductions, rather than as a complete solution of the problem.

\subsection*{Analytic hypotheses}
\label{sec:hypotheses}

For the reader's convenience we collect here the main analytic assumptions on which our argument rests. Precise formulations are given in the corresponding sections, but we state them informally to highlight the logical structure.

\begin{hypothesis}[Fredholm theory for the Jang--Lichnerowicz operator]\label{hyp:Fredholm}
Let $(\bar M,\bar g)$ denote the Jang deformation of $(M,g,k)$ constructed in Section~\ref{sec:Jang}, with a single asymptotically flat end and a finite number of asymptotically cylindrical ends corresponding to the components of the outermost MOTS. Then for appropriate weighted Sobolev or edge spaces the operator
\[
L := \Delta_{\bar g} - \tfrac18 R^{\mathrm{reg}}_{\bar g}
\]
is Fredholm of index zero and admits a bounded inverse, with the asymptotic expansions described in Section~\ref{sec:Lichnerowicz}. Moreover, the associated solution $\phi$ of the Lichnerowicz equation satisfies $0<\phi\le 1$ on $\bar M$.
\end{hypothesis}

\begin{hypothesis}[Corner smoothing with controlled scalar curvature]\label{hyp:corner}
Let $(\tilde M,\tilde g)$ be the metric obtained from $(\bar M,\bar g)$ by sealing the cylindrical ends as in Section~\ref{sec:sealing}, so that $\tilde g$ has a Lipschitz internal corner along the image of the outermost MOTS and nonnegative scalar curvature in the distributional sense. Then there exists a family of smooth metrics $g_\varepsilon$ converging uniformly to $\tilde g$ such that $R_{g_\varepsilon}\ge -C\varepsilon^{2/3}$ and the negative part $R^-_{g_\varepsilon}$ is supported in an $\varepsilon$--neighborhood of the corner and satisfies the $L^{3/2}$--bound stated in Section~\ref{sec:corner-smoothing}.
\end{hypothesis}

\begin{hypothesis}[$p$--harmonic potentials on degenerating backgrounds]\label{hyp:pharmonic}
Let $g_\varepsilon$ be as in Hypothesis~\ref{hyp:corner}. For each fixed $p>1$ consider the $p$--harmonic potential $u_{\varepsilon,p}$ with the boundary conditions prescribed in Section~\ref{sec:pharmonic}. Then, as $\varepsilon\to 0$, the energies and level-set functionals associated to $u_{\varepsilon,p}$ converge in the sense of Mosco/Gamma convergence to the corresponding objects on $(\tilde M,\tilde g)$, and in particular no spurious area can concentrate at the conical tips produced by the sealing procedure when $p\to 1$.
\end{hypothesis}

\begin{hypothesis}[Static vacuum regularity and rigidity]\label{hyp:static}
Assume that equality holds in the spacetime Penrose inequality for an initial data set $(M,g,k)$ as in Theorem~\ref{thm:main}. Then the associated Jang deformation $(\bar M,\bar g)$ carries a lapse function $N$ such that $(\bar g,N)$ is a smooth static vacuum solution in the sense of general relativity, and any such solution with a connected outermost minimal boundary and a single asymptotically flat end is isometric to a Schwarzschild slice.
\end{hypothesis}

Throughout the paper we make it clear when a result relies on one of these hypotheses. In particular, the reduction of the spacetime Penrose inequality to the four analytic hypotheses above is unconditional, whereas the full inequality and the rigidity statement are proved only under these assumptions.

\subsection*{Outline of the paper}

The subsequent sections review the Jang deformation, discuss the conditional Fredholm and asymptotic theory for the Lichnerowicz equation, state the corner smoothing and $p$--harmonic assumptions, and conclude with a conditional rigidity statement.

\section{The Jang equation and the scalar curvature identity}\label{sec:Jang}

\begin{remark}
The existence and detailed asymptotic behavior of Jang solutions with cylindrical blow-up along stable or marginally stable MOTS, which are used throughout this section, are known in several important cases thanks to the work of Han--Khuri and Bray--Khuri. However, the full strength of the statements needed later in the paper slightly exceeds what is currently available in the literature.

Rather than attempting to fill all of these analytic gaps here, we formulate precise statements in Hypothesis~\ref{hyp:Fredholm} and indicate how they are related to the existing results. In particular, any future improvement of the Jang theory in the marginally stable case would immediately strengthen our conclusions, without changing the geometric structure of the argument.
\end{remark}

\section{The Lichnerowicz equation on the Jang background}\label{sec:Lichnerowicz}

In this section we formulate and use the analytic input contained in Hypothesis~\ref{hyp:Fredholm}. We state the Lichnerowicz equation on the Jang background, explain how the operator $L = \Delta_{\bar g} - \frac18 R^{\mathrm{reg}}_{\bar g}$ fits into standard Fredholm frameworks on manifolds with asymptotically flat and asymptotically cylindrical ends, and then record the consequences that are needed later in the paper.

The full Fredholm and asymptotic analysis of $L$ in the generality relevant here requires a substantial amount of work (edge Sobolev spaces, scattering calculus, and a careful study of the marginally stable case). We therefore do not attempt to give a complete proof of Hypothesis~\ref{hyp:Fredholm} in this section; instead we outline the argument and indicate how existing results can be adapted.

\begin{theorem}\label{thm:LichnerowiczExistence}
Assume Hypothesis~\ref{hyp:Fredholm}. Let $(\bar M,\bar g)$ be the Jang manifold constructed above. Then there exists a unique positive solution $\phi$ of the Lichnerowicz equation with the asymptotics described in Proposition~\ref{prop:asymptotics}, and $\phi\le 1$ on $\bar M$.
\end{theorem}

\begin{remark}
The proof of Theorem~\ref{thm:LichnerowiczExistence} given here is conditional on Hypothesis~\ref{hyp:Fredholm}. In particular, we use the Fredholm properties and maximum principle for $L$ as black boxes and focus on the geometric and variational identities (such as the Bray--Khuri divergence identity) that lead to the bound $\phi\le 1$. A complete proof of Hypothesis~\ref{hyp:Fredholm}, including the marginally stable case, will be given elsewhere.
\end{remark}

\section{Corner smoothing and scalar curvature control}\label{sec:corner-smoothing}

The goal of this section is to explain how Hypothesis~\ref{hyp:corner} fits into the overall argument. We describe a Miao-type smoothing procedure for an internal corner along the sealed Jang surface, discuss the distributional scalar curvature inequality across the corner, and indicate how one can control the negative part of the scalar curvature of the smoothed metrics uniformly in the smoothing scale.

The detailed tensor computations needed to verify the $L^{3/2}$--estimates of Hypothesis~\ref{hyp:corner}, especially in the presence of tangential variation of the geometric data along the corner, are lengthy and technical and are therefore not attempted here. Instead we provide a guide to the curvature bookkeeping and explain how the desired bounds arise from a combination of local computations and global Sobolev / isoperimetric inequalities.

\section{$p$--harmonic potentials and the Agostiniani--Mazzieri--Oronzio functional}\label{sec:pharmonic}

In this section we assume Hypothesis~\ref{hyp:pharmonic} and explain how to adapt the $p$--harmonic level-set method of Agostiniani--Mazzieri--Oronzio to the family of smooth metrics $g_\varepsilon$ approximating the sealed Jang metric $\tilde g$. The key point is that the geometric functionals controlling the ADM mass are stable under the Mosco/Gamma limits described in Hypothesis~\ref{hyp:pharmonic}, and that no ``ghost area'' can concentrate at the conical tips when $p\to 1$.

We work under the simplifying assumption that the reader is familiar with the AMO framework; our contribution here is to explain how their arguments interact with the degenerating geometry produced by the Jang and smoothing constructions. The analytic heart of Hypothesis~\ref{hyp:pharmonic} is the uniform control of $p$--harmonic Green-type estimates and perimeter/volume comparison along the approximating sequence, which we do not attempt to prove in full generality in this paper.

\section{Rigidity and the equality case}\label{sec:rigidity}

In this section we explain how, under Hypothesis~\ref{hyp:static}, equality in the spacetime Penrose inequality forces the data to arise from a slice of the Schwarzschild spacetime. The arguments are kept at the level of an outline, since a complete treatment of static vacuum regularity and uniqueness for metrics that are \emph{a priori} only Lipschitz across the sealed Jang surface would require a separate, lengthy analysis.

\begin{theorem}[Conditional rigidity]\label{thm:rigidity}
Suppose $(M,g,k)$ is an asymptotically flat initial data set satisfying the hypotheses of Theorem~\ref{thm:main} and that equality holds in \eqref{eq:PenroseInequality}. Assume in addition Hypothesis~\ref{hyp:static}. Then $(M,g,k)$ arises as a slice of a Schwarzschild spacetime.
\end{theorem}

\begin{remark}
Without Hypothesis~\ref{hyp:static} we still obtain that the Jang deformation $(\bar M,\bar g)$ has nonnegative scalar curvature and that all the error quantities appearing in the divergence identities of Sections~\ref{sec:Lichnerowicz} and~\ref{sec:pharmonic} vanish. We expect that a suitable regularity and uniqueness theory for such ``weak'' static vacuum solutions would imply Hypothesis~\ref{hyp:static}, but we do not attempt to develop this theory here.
\end{remark}

\end{document}
