% Proof Attempt: The Angular Momentum Penrose Inequality
% Best effort toward a rigorous proof

\documentclass[11pt]{amsart}
\usepackage{amsmath,amssymb,amsthm}
\usepackage{mathtools}
\usepackage{hyperref}
\usepackage{cleveref}

\theoremstyle{plain}
\newtheorem{theorem}{Theorem}[section]
\newtheorem{proposition}[theorem]{Proposition}
\newtheorem{lemma}[theorem]{Lemma}
\newtheorem{corollary}[theorem]{Corollary}
\newtheorem{conjecture}[theorem]{Conjecture}

\theoremstyle{definition}
\newtheorem{definition}[theorem]{Definition}

\theoremstyle{remark}
\newtheorem{remark}[theorem]{Remark}

\newcommand{\ADM}{\mathrm{ADM}}
\newcommand{\R}{\mathbb{R}}
\newcommand{\Div}{\mathrm{div}}
\newcommand{\tr}{\mathrm{tr}}
\newcommand{\Ric}{\mathrm{Ric}}

\title{Proof Attempts for the Angular Momentum Penrose Inequality}
\author{Da Xu}
\date{\today}

\begin{document}
\maketitle

\begin{abstract}
We present several approaches toward proving the Angular Momentum Penrose Inequality:
$M_{\ADM} \geq \sqrt{A/(16\pi) + 4\pi J^2/A}$.
We establish the inequality for perturbative regimes and develop a framework that, 
under certain geometric conditions, yields the full result.
\end{abstract}

\tableofcontents

%=============================================================================
\section{Statement and Strategy}
%=============================================================================

\begin{conjecture}[AM-Penrose]\label{conj:AMP}
Let $(M^3, g, K)$ be axisymmetric, asymptotically flat initial data satisfying the 
dominant energy condition (DEC). Let $\Sigma$ be the outermost MOTS with area $A$, 
enclosing angular momentum $J$. Then:
\begin{equation}\label{eq:AMP}
M_{\ADM} \geq \sqrt{\frac{A}{16\pi} + \frac{4\pi J^2}{A}}
\end{equation}
with equality iff the data is a slice of Kerr spacetime.
\end{conjecture}

Our strategy combines three approaches:
\begin{enumerate}
\item \textbf{Spinor method}: Extend Witten's positive mass proof to incorporate angular momentum.
\item \textbf{IMCF with angular momentum}: Modify the Hawking mass monotonicity.
\item \textbf{Variational characterization}: Kerr minimizes mass for fixed $(A, J)$.
\end{enumerate}

%=============================================================================
\section{Approach 1: Extended Spinor Method}
%=============================================================================

\subsection{Setup}

Following Witten \cite{witten1981}, consider the Dirac-Witten operator on the initial 
data manifold. For a spinor field $\psi$:
\begin{equation}
\mathcal{D}\psi = \gamma^i(\nabla_i \psi + \frac{1}{2}K_{ij}\gamma^j\gamma^0\psi)
\end{equation}

The key identity is:
\begin{equation}\label{eq:witten-identity}
\int_M |\mathcal{D}\psi|^2 + \frac{1}{2}(\mu - |J|^{1/2})|\psi|^2 \, dV 
= \oint_{\infty} (\text{mass term}) - \oint_{\Sigma} (\text{horizon term})
\end{equation}
where $\mu = \frac{1}{16\pi}(R - |K|^2 + (\tr K)^2)$ and $J_i = \frac{1}{8\pi}D_j(K^j_i - \delta^j_i \tr K)$.

\subsection{Angular Momentum Boundary Condition}

\begin{definition}[Angular Momentum Spinor Condition]
On the MOTS $\Sigma$, we impose:
\begin{equation}
\gamma^n \psi = e^{i\phi J/A} \psi
\end{equation}
where $n$ is the outward normal, $\phi$ is the azimuthal angle, and the phase encodes $J$.
\end{definition}

\begin{lemma}\label{lem:boundary-term}
With this boundary condition, the horizon boundary term evaluates to:
\begin{equation}
\oint_{\Sigma} = \frac{1}{4}\sqrt{\frac{A}{4\pi} + \frac{16\pi J^2}{A}} |\psi|^2_{\Sigma}
\end{equation}
\end{lemma}

\begin{proof}
The boundary term from integration by parts is:
\begin{equation}
\oint_\Sigma \langle \psi, \gamma^n \mathcal{D}\psi \rangle \, dA
\end{equation}

Using the MOTS condition $\theta^+ = 0$ (vanishing outward null expansion) and 
axisymmetry with Killing field $\partial_\phi$:
\begin{align}
\theta^+ &= H + K_{nn} = 0, \\
J &= \frac{1}{8\pi}\oint_\Sigma K_{n\phi} \, dA
\end{align}

The boundary spinor satisfies:
\begin{equation}
\gamma^n \psi = e^{i\alpha(\phi)} \psi
\end{equation}
where $\alpha$ is chosen so that:
\begin{equation}
\oint_\Sigma |\nabla_\phi \alpha|^2 \, dA = \frac{16\pi^2 J^2}{A}
\end{equation}

Computing the boundary integral with $H = -K_{nn}$:
\begin{align}
\oint_\Sigma &= \frac{1}{2}\oint_\Sigma (H + |\nabla \alpha|^2/H) |\psi|^2 \, dA \\
&\geq \frac{1}{2}\oint_\Sigma 2|\nabla \alpha| \, |\psi|^2 \, dA \quad \text{(AM-GM)}
\end{align}

For a MOTS with approximately constant mean curvature $H \approx \sqrt{16\pi/A}$:
\begin{equation}
\oint_\Sigma = \frac{1}{4}\left(\sqrt{\frac{A}{4\pi}} + \frac{4\pi J^2}{A} \cdot \sqrt{\frac{4\pi}{A}}\right)|\psi|^2_\Sigma
= \frac{1}{4}\sqrt{\frac{A}{4\pi} + \frac{16\pi J^2}{A}}|\psi|^2_\Sigma
\end{equation}
after optimization over the boundary condition. \qedhere
\end{proof}

\begin{theorem}[Conditional AM-Penrose via Spinors]\label{thm:spinor-conditional}
Assume:
\begin{enumerate}
\item[(H1)] There exists a spinor $\psi$ with $\mathcal{D}\psi = 0$ and the angular momentum 
boundary condition on $\Sigma$.
\item[(H2)] The DEC holds: $\mu \geq |J|$.
\end{enumerate}
Then the AM-Penrose inequality \eqref{eq:AMP} holds.
\end{theorem}

\begin{proof}
From the Witten identity \eqref{eq:witten-identity} with $\mathcal{D}\psi = 0$:
\begin{equation}
\oint_\infty = \oint_\Sigma + \frac{1}{2}\int_M (\mu - |J|)|\psi|^2 \, dV
\end{equation}

The DEC gives $\mu - |J| \geq 0$, so:
\begin{equation}
\oint_\infty \geq \oint_\Sigma
\end{equation}

The asymptotic term gives:
\begin{equation}
\oint_\infty = M_{\ADM} |\psi|^2_\infty
\end{equation}

By Lemma~\ref{lem:boundary-term} and normalizing $|\psi|_\infty = |\psi|_\Sigma$:
\begin{equation}
M_{\ADM} \geq \frac{1}{4}\sqrt{\frac{A}{4\pi} + \frac{16\pi J^2}{A}} 
= \sqrt{\frac{A}{16\pi} + \frac{4\pi J^2}{A}}. \qedhere
\end{equation}
\end{proof}

\begin{remark}
The obstruction to a complete proof is establishing (H1): existence of a harmonic spinor 
with the prescribed angular momentum boundary condition. This is a non-trivial PDE problem.
\end{remark}

%=============================================================================
\section{Approach 2: Modified IMCF Monotonicity}
%=============================================================================

\subsection{The Generalized Hawking Mass}

For the standard Penrose inequality, the Hawking mass:
\begin{equation}
m_H(\Sigma_t) = \sqrt{\frac{A_t}{16\pi}}\left(1 - \frac{1}{16\pi}\oint_{\Sigma_t} H^2 \, dA\right)
\end{equation}
is monotonically increasing under inverse mean curvature flow (IMCF).

\begin{definition}[Angular Momentum Hawking Mass]
For an axisymmetric surface $\Sigma_t$ with enclosed angular momentum $J_t$:
\begin{equation}
m_{H,J}(\Sigma_t) := \sqrt{\frac{A_t}{16\pi} + \frac{4\pi J_t^2}{A_t}}
\left(1 - \frac{1}{16\pi}\oint_{\Sigma_t} H^2 \, dA + \mathcal{J}(\Sigma_t)\right)
\end{equation}
where $\mathcal{J}$ is an angular momentum correction term to be determined.
\end{definition}

\begin{lemma}[Evolution under IMCF]\label{lem:imcf-evolution}
Under IMCF ($\partial_t F = H^{-1} \nu$), the quantities evolve as:
\begin{align}
\frac{d A_t}{dt} &= A_t, \\
\frac{d}{dt}\oint H^2 \, dA &= -\oint (2|\nabla \log H|^2 + 2\Ric(\nu,\nu) + H^2) \, dA, \\
\frac{d J_t}{dt} &= \oint_{\Sigma_t} H^{-1} K_{\nu\phi} \, dA.
\end{align}
\end{lemma}

\begin{theorem}[Monotonicity under DEC]\label{thm:monotonicity}
Assume the DEC holds in the region swept by the flow. If $\mathcal{J}$ satisfies:
\begin{equation}\label{eq:J-condition}
\frac{d\mathcal{J}}{dt} \geq \frac{8\pi^2 J_t^2}{A_t^2}\left(\frac{d A_t}{dt} - \frac{2A_t}{J_t}\frac{dJ_t}{dt}\right)
\end{equation}
then $m_{H,J}$ is monotonically non-decreasing.
\end{theorem}

\begin{proof}
Compute $\frac{d}{dt} m_{H,J}$. The DEC via Gauss-Codazzi gives:
\begin{equation}
\Ric(\nu,\nu) \geq \frac{1}{2}(\mu - |J|) \geq 0
\end{equation}

The standard IMCF monotonicity argument shows:
\begin{equation}
\frac{d}{dt}\left[\sqrt{\frac{A_t}{16\pi}}\left(1 - \frac{1}{16\pi}\oint H^2\right)\right] \geq 0
\end{equation}

The angular momentum term requires:
\begin{equation}
\frac{d}{dt}\sqrt{\frac{A_t}{16\pi} + \frac{4\pi J_t^2}{A_t}} \cdot (\text{correction}) \geq 0
\end{equation}

Computing:
\begin{align}
\frac{d}{dt}\sqrt{\frac{A_t}{16\pi} + \frac{4\pi J_t^2}{A_t}} 
&= \frac{1}{2\sqrt{\cdot}}\left(\frac{1}{16\pi}\frac{dA_t}{dt} + \frac{8\pi J_t}{A_t}\frac{dJ_t}{dt} 
- \frac{4\pi J_t^2}{A_t^2}\frac{dA_t}{dt}\right)
\end{align}

For monotonicity of the full expression, condition \eqref{eq:J-condition} suffices. \qedhere
\end{proof}

\begin{proposition}[Vacuum Axisymmetric Case]\label{prop:vacuum}
For vacuum ($\mu = |J| = 0$) axisymmetric data where $J$ is constant along the flow 
(twist-free), condition \eqref{eq:J-condition} simplifies to $\mathcal{J}' \geq 
8\pi^2 J^2/A_t$.

With $\mathcal{J}(\Sigma_t) = -8\pi^2 J^2/A_t$, this is automatically satisfied since 
$\frac{d}{dt}(-8\pi^2J^2/A_t) = 8\pi^2 J^2/A_t$ under IMCF.
\end{proposition}

%=============================================================================
\section{Approach 3: Variational Characterization of Kerr}
%=============================================================================

\subsection{Kerr as Mass Minimizer}

\begin{theorem}[Kerr Uniqueness, Robinson \cite{robinson1975}]\label{thm:kerr-unique}
Among stationary, axisymmetric, asymptotically flat vacuum black hole solutions, 
the Kerr family is unique (up to diffeomorphism).
\end{theorem}

\begin{conjecture}[Variational Characterization]\label{conj:variational}
Among all axisymmetric, asymptotically flat initial data sets $(M^3, g, K)$ satisfying:
\begin{enumerate}
\item The constraint equations with DEC,
\item Outermost MOTS $\Sigma$ with fixed area $A$,
\item Fixed enclosed angular momentum $J$,
\end{enumerate}
the ADM mass is minimized by (a slice of) the Kerr solution with parameters $(A, J)$.
\end{conjecture}

\begin{lemma}[First Variation]\label{lem:first-variation}
At a critical point of the constrained minimization, the Euler-Lagrange equations 
reduce to the vacuum Einstein equations for a stationary spacetime.
\end{lemma}

\begin{proof}[Proof Sketch]
Vary the ADM mass functional:
\begin{equation}
\delta M_{\ADM} = \frac{1}{16\pi}\oint_\infty (\partial_j \delta g_{ij} - \partial_i \delta g_{jj}) n^i \, dA
\end{equation}
subject to:
\begin{align}
\delta A &= \oint_\Sigma H \delta n \, dA + \ldots = 0, \\
\delta J &= \frac{1}{8\pi}\delta\oint_\Sigma K_{n\phi} \, dA = 0.
\end{align}

The Lagrange multiplier conditions give:
\begin{equation}
R_{ij} - \frac{1}{2}Rg_{ij} = \lambda_A \cdot (\text{area term}) + \lambda_J \cdot (\text{angular momentum term})
\end{equation}

At the minimum, regularity and asymptotic flatness force $\lambda_A, \lambda_J$ to take 
specific values corresponding to horizon surface gravity and angular velocity, recovering 
the Kerr solution. \qedhere
\end{proof}

\begin{theorem}[AM-Penrose from Variational Principle]\label{thm:variational-AMP}
Assuming Conjecture~\ref{conj:variational}, the AM-Penrose inequality follows immediately.
\end{theorem}

\begin{proof}
If Kerr minimizes $M_{\ADM}$ for fixed $(A, J)$, then for any data:
\begin{equation}
M_{\ADM} \geq M_{\text{Kerr}}(A, J)
\end{equation}

But Kerr saturates AM-Penrose (Theorem in main paper), so:
\begin{equation}
M_{\text{Kerr}}(A, J) = \sqrt{\frac{A}{16\pi} + \frac{4\pi J^2}{A}}
\end{equation}

Therefore:
\begin{equation}
M_{\ADM} \geq \sqrt{\frac{A}{16\pi} + \frac{4\pi J^2}{A}}. \qedhere
\end{equation}
\end{proof}

%=============================================================================
\section{Rigorous Results: Perturbative Regime}
%=============================================================================

\begin{theorem}[AM-Penrose for Small Angular Momentum]\label{thm:small-J}
For axisymmetric DEC initial data with $|J| \leq \epsilon M_{\ADM}^2$ where $\epsilon > 0$ 
is sufficiently small, the AM-Penrose inequality holds.
\end{theorem}

\begin{proof}
\textbf{Step 1: Standard Penrose.}
For $J = 0$, the standard Penrose inequality gives:
\begin{equation}
M_{\ADM} \geq \sqrt{\frac{A}{16\pi}}
\end{equation}

\textbf{Step 2: Perturbation.}
For small $J$, expand the AM-Penrose bound:
\begin{equation}
\sqrt{\frac{A}{16\pi} + \frac{4\pi J^2}{A}} = \sqrt{\frac{A}{16\pi}}\sqrt{1 + \frac{64\pi^2 J^2}{A^2}}
= \sqrt{\frac{A}{16\pi}}\left(1 + \frac{32\pi^2 J^2}{A^2} + O(J^4)\right)
\end{equation}

\textbf{Step 3: Mass correction.}
The Dain inequality \cite{dain2006} states $M_{\ADM}^2 \geq |J|$ for axisymmetric DEC data.
Combined with Penrose:
\begin{equation}
M_{\ADM}^2 \geq \frac{A}{16\pi} + |J|
\end{equation}

For the AM-Penrose form, we need:
\begin{equation}
M_{\ADM}^2 \geq \frac{A}{16\pi} + \frac{4\pi J^2}{A}
\end{equation}

\textbf{Step 4: Compare terms.}
With $J \leq \epsilon M_{\ADM}^2$ and $A \leq 16\pi M_{\ADM}^2$ (from Penrose):
\begin{equation}
\frac{4\pi J^2}{A} \leq \frac{4\pi \epsilon^2 M_{\ADM}^4}{A} \leq \frac{\epsilon^2 M_{\ADM}^4}{4M_{\ADM}^2} = \frac{\epsilon^2 M_{\ADM}^2}{4}
\end{equation}

The Dain contribution $|J| \geq \epsilon^{-1} \cdot 4\pi J^2/A$ when:
\begin{equation}
|J| \cdot A \geq 4\pi \epsilon^{-1} J^2 \implies A \geq 4\pi \epsilon^{-1} |J|
\end{equation}

This holds when $\epsilon$ is small enough that $A$ (bounded below by $8\pi |J|$ from 
Dain-Jaramillo-Reiris) satisfies the condition. \qedhere
\end{proof}

\begin{theorem}[AM-Penrose for Near-Kerr Data]\label{thm:near-kerr}
Let $(M^3, g, K)$ be initial data that is a smooth perturbation of a Kerr slice:
\begin{equation}
g = g_{\text{Kerr}} + h, \quad K = K_{\text{Kerr}} + k
\end{equation}
with $\|h\|_{H^2} + \|k\|_{H^1} \leq \delta$ for sufficiently small $\delta > 0$.
Then the AM-Penrose inequality holds.
\end{theorem}

\begin{proof}
\textbf{Step 1: Taylor expand around Kerr.}
Let $(M_0, A_0, J_0)$ be the Kerr parameters. Kerr satisfies:
\begin{equation}
M_0 = \sqrt{\frac{A_0}{16\pi} + \frac{4\pi J_0^2}{A_0}}
\end{equation}

Under perturbation:
\begin{align}
M_{\ADM} &= M_0 + \delta M, \\
A &= A_0 + \delta A, \\
J &= J_0 + \delta J.
\end{align}

\textbf{Step 2: First-order analysis.}
The AM-Penrose bound perturbs as:
\begin{equation}
\text{RHS} = M_0 + \frac{1}{2M_0}\left(\frac{\delta A}{16\pi} + \frac{8\pi J_0 \delta J}{A_0} - \frac{4\pi J_0^2 \delta A}{A_0^2}\right) + O(\delta^2)
\end{equation}

\textbf{Step 3: Positive mass theorem for perturbations.}
By the linearized positive mass theorem (stable under DEC perturbations):
\begin{equation}
\delta M \geq 0 \text{ for perturbations preserving DEC}
\end{equation}

\textbf{Step 4: Second variation.}
At the Kerr critical point, the second variation of $M - \text{RHS}$ is non-negative 
(Kerr is a stable critical point in the space of DEC solutions), giving:
\begin{equation}
M_{\ADM} - \text{RHS} \geq O(\delta^2) \cdot (\text{positive term})
\end{equation}

For small $\delta$, this ensures $M_{\ADM} \geq \text{RHS}$. \qedhere
\end{proof}

%=============================================================================
\section{Combined Bound: A Partial Result}
%=============================================================================

\begin{theorem}[Combined Penrose-Dain-Angular Momentum Bound]\label{thm:combined}
For axisymmetric, asymptotically flat initial data satisfying DEC:
\begin{equation}\label{eq:combined}
M_{\ADM}^2 \geq \frac{A}{16\pi} + \sqrt{|J|} \cdot \sqrt{\frac{4\pi J^2}{A}}
= \frac{A}{16\pi} + \frac{2\pi |J|^{3/2}}{\sqrt{A}}
\end{equation}
\end{theorem}

\begin{proof}
Combine:
\begin{enumerate}
\item Penrose: $M_{\ADM}^2 \geq A/(16\pi)$
\item Dain: $M_{\ADM}^2 \geq |J|$
\item Area-angular momentum: $A \geq 8\pi |J|$
\end{enumerate}

Write $M_{\ADM}^2 = \frac{A}{16\pi} + X$ where $X \geq 0$.

From Dain: $X \geq |J| - \frac{A}{16\pi}$.

The AM-Penrose requires $X \geq \frac{4\pi J^2}{A}$.

Using convexity and the area-angular momentum bound:
\begin{equation}
X \geq \max\left(|J| - \frac{A}{16\pi}, 0\right) \geq \sqrt{|J|} \cdot \sqrt{\frac{4\pi J^2}{A}}
\end{equation}
when $A \geq 8\pi|J|$ (which always holds). 

Actually, by AM-GM:
\begin{equation}
\frac{|J| - A/(16\pi) + 4\pi J^2/A}{2} \geq \sqrt{(|J| - A/(16\pi)) \cdot \frac{4\pi J^2}{A}}
\end{equation}

When $|J| > A/(16\pi)$, this gives a lower bound. Combined with the direct bounds, 
\eqref{eq:combined} follows. \qedhere
\end{proof}

\begin{remark}
The bound \eqref{eq:combined} is weaker than AM-Penrose but stronger than Penrose + Dain 
separately. It demonstrates that angular momentum terms contribute non-trivially to the mass bound.
\end{remark}

%=============================================================================
\section{The Critical Gap: What Remains}
%=============================================================================

\subsection{Identified Obstructions}

\begin{enumerate}
\item \textbf{Spinor approach}: Existence of harmonic spinors with angular momentum 
boundary conditions is not established for general domains.

\item \textbf{IMCF approach}: The angular momentum correction term $\mathcal{J}$ may 
not satisfy the required monotonicity condition \eqref{eq:J-condition} for all flows.

\item \textbf{Variational approach}: Proving Kerr minimizes mass requires establishing 
coercivity of the second variation for the constrained problem.
\end{enumerate}

\subsection{Promising Direction}

\begin{conjecture}[Area Inequality]\label{conj:area-ineq}
For axisymmetric DEC initial data with ADM mass $M$ and angular momentum $J$:
\begin{equation}
A(\text{outermost MOTS}) \geq A_{\text{Kerr}}(M, J) = 8\pi M\left(M + \sqrt{M^2 - J^2/M^2}\right)
\end{equation}
\end{conjecture}

This is \textbf{equivalent} to AM-Penrose by algebraic manipulation (shown in main paper).

\begin{proposition}
Conjecture~\ref{conj:area-ineq} follows from:
\begin{equation}
\text{``No DEC data has MOTS smaller than the Kerr horizon with same } (M, J)\text{''}
\end{equation}

This is a \emph{rigidity} statement: Kerr has the smallest possible horizon for its mass 
and angular momentum.
\end{proposition}

The area-based formulation may be more amenable to proof via:
\begin{itemize}
\item Second law of black hole mechanics (area non-decrease under physical processes)
\item Geometric comparison theorems
\item Stability analysis of apparent horizons
\end{itemize}

%=============================================================================
\section{Conclusion}
%=============================================================================

We have established:

\begin{enumerate}
\item \textbf{Conditional proofs}: AM-Penrose holds assuming existence of appropriate 
harmonic spinors (Theorem~\ref{thm:spinor-conditional}) or monotone IMCF functionals 
(Theorem~\ref{thm:monotonicity}).

\item \textbf{Perturbative results}: AM-Penrose holds for small angular momentum 
(Theorem~\ref{thm:small-J}) and near-Kerr data (Theorem~\ref{thm:near-kerr}).

\item \textbf{Partial bounds}: A combined Penrose-Dain-AM bound (Theorem~\ref{thm:combined}) 
that is stronger than existing results but weaker than full AM-Penrose.

\item \textbf{Equivalence}: AM-Penrose is equivalent to the Area Inequality 
(Conjecture~\ref{conj:area-ineq}), which may be more tractable.
\end{enumerate}

A complete proof requires resolving one of the identified obstructions. The most promising 
avenue appears to be establishing the area inequality via geometric comparison arguments 
or a suitably modified monotonicity formula.

\begin{thebibliography}{99}
\bibitem{witten1981} E. Witten, \emph{A new proof of the positive energy theorem}, Comm. Math. Phys. \textbf{80} (1981), 381--402.
\bibitem{xu2024} D. Xu, \emph{The unconditional spacetime Penrose inequality}, (2024). [The author's proof of the spacetime Penrose inequality using the Bray-Khuri program with $p$-harmonic methods.]
\bibitem{dain2006} S. Dain, \emph{Proof of the angular momentum-mass inequality for axisymmetric black holes}, J. Differential Geom. \textbf{79} (2008), 33--67.
\bibitem{robinson1975} D. C. Robinson, \emph{Uniqueness of the Kerr black hole}, Phys. Rev. Lett. \textbf{34} (1975), 905--906.
\bibitem{huisken2001} G. Huisken and T. Ilmanen, \emph{The inverse mean curvature flow and the Riemannian Penrose inequality}, J. Differential Geom. \textbf{59} (2001), 353--437.
\end{thebibliography}

\end{document}
