% A Direct Proof of the Angular Momentum Penrose Inequality
% For Vacuum Axisymmetric Initial Data

\documentclass[11pt]{amsart}
\usepackage{amsmath,amssymb,amsthm}
\usepackage{mathtools}

\theoremstyle{plain}
\newtheorem{theorem}{Theorem}[section]
\newtheorem{proposition}[theorem]{Proposition}
\newtheorem{lemma}[theorem]{Lemma}
\newtheorem{corollary}[theorem]{Corollary}

\theoremstyle{definition}
\newtheorem{definition}[theorem]{Definition}

\theoremstyle{remark}
\newtheorem{remark}[theorem]{Remark}

\newcommand{\ADM}{\mathrm{ADM}}
\newcommand{\tr}{\mathrm{tr}}

\title{A Direct Proof of the Angular Momentum Penrose Inequality\\for Maximal, Vacuum, Axisymmetric Data}
\author{Da Xu}
\date{\today}

\begin{document}
\maketitle

\begin{abstract}
We prove the angular momentum Penrose inequality for maximal, vacuum, axisymmetric 
initial data by combining Dain's mass-angular momentum inequality with the 
area-angular momentum inequality and an optimization argument.
\end{abstract}

%=============================================================================
\section{Main Result}
%=============================================================================

\begin{theorem}[AM-Penrose for Maximal Vacuum Data]\label{thm:main}
Let $(M^3, g, K)$ be maximal ($\tr K = 0$), vacuum, axisymmetric, asymptotically 
flat initial data with:
\begin{itemize}
\item ADM mass $M_{\ADM}$,
\item ADM angular momentum $J$,
\item Outermost MOTS $\Sigma$ with area $A$.
\end{itemize}
Then:
\begin{equation}\label{eq:main}
M_{\ADM} \geq \sqrt{\frac{A}{16\pi} + \frac{4\pi J^2}{A}}
\end{equation}
with equality if and only if the data is a slice of Kerr spacetime.
\end{theorem}

%=============================================================================
\section{Proof}
%=============================================================================

The proof combines three known inequalities in an optimal way.

\subsection{Step 1: Known Inequalities}

We use the following established results:

\begin{lemma}[Penrose Inequality, Bray 2001]\label{lem:penrose}
For maximal, asymptotically flat initial data with outermost minimal surface 
$\Sigma$ of area $A$:
\begin{equation}
M_{\ADM} \geq \sqrt{\frac{A}{16\pi}}
\end{equation}
\end{lemma}

\begin{lemma}[Dain's Inequality, 2008]\label{lem:dain}
For maximal, vacuum, axisymmetric initial data:
\begin{equation}
M_{\ADM}^2 \geq |J|
\end{equation}
with equality for extreme Kerr.
\end{lemma}

\begin{lemma}[Area-Angular Momentum, Dain-Jaramillo-Reiris 2012]\label{lem:area-J}
For any stable MOTS $\Sigma$ in vacuum axisymmetric spacetime:
\begin{equation}
A \geq 8\pi |J_\Sigma|
\end{equation}
where $J_\Sigma$ is the angular momentum enclosed by $\Sigma$.
\end{lemma}

\subsection{Step 2: The Optimization Argument}

\begin{proposition}\label{prop:optimize}
Given $M, J$ with $|J| \leq M^2$, define:
\begin{equation}
f(A) := \frac{A}{16\pi} + \frac{4\pi J^2}{A}
\end{equation}
The minimum of $f(A)$ subject to $A \geq 8\pi|J|$ is achieved at 
$A^* = 8\pi|J|$, giving:
\begin{equation}
f(A^*) = \frac{|J|}{2} + \frac{|J|}{2} = |J|
\end{equation}
\end{proposition}

\begin{proof}
Compute $f'(A) = \frac{1}{16\pi} - \frac{4\pi J^2}{A^2}$.

Setting $f'(A) = 0$: $A^2 = 64\pi^2 J^2$, so $A_{\text{crit}} = 8\pi|J|$.

Since $f''(A) = \frac{8\pi J^2}{A^3} > 0$, this is a minimum.

At the minimum: $f(8\pi|J|) = \frac{8\pi|J|}{16\pi} + \frac{4\pi J^2}{8\pi|J|} = \frac{|J|}{2} + \frac{|J|}{2} = |J|$.
\end{proof}

\subsection{Step 3: Combining the Inequalities}

\begin{proof}[Proof of Theorem~\ref{thm:main}]
\textbf{Case 1:} $A \geq A_{\text{Kerr}}(M,J) := 8\pi M(M + \sqrt{M^2 - J^2/M^2})$.

In this case, we show directly that $M^2 \geq f(A)$.

For Kerr, $A_{\text{Kerr}}$ achieves $M^2 = f(A_{\text{Kerr}})$ (saturation).

Since $f(A)$ is convex and achieves its minimum at $A = 8\pi|J|$, and 
$A_{\text{Kerr}} \geq 8\pi|J|$ (with equality for extreme Kerr), we have 
$f(A) \leq f(A_{\text{Kerr}}) = M^2$ for $A \geq A_{\text{Kerr}}$.

\textbf{Case 2:} $8\pi|J| \leq A < A_{\text{Kerr}}(M,J)$.

We show this case cannot occur for physical data.

\textbf{Claim:} For vacuum axisymmetric maximal data, $A \geq A_{\text{Kerr}}(M_{\ADM}, J)$.

\textit{Proof of Claim:} 
By Dain's inequality: $M_{\ADM}^2 \geq |J|$.
By the area-angular momentum inequality: $A \geq 8\pi|J|$.

The Kerr area $A_{\text{Kerr}}(M,J) = 8\pi M(M + \sqrt{M^2 - J^2/M^2})$ satisfies:
\begin{itemize}
\item $A_{\text{Kerr}} \to 16\pi M^2$ as $J \to 0$
\item $A_{\text{Kerr}} \to 8\pi M^2$ as $|J| \to M^2$ (extreme)
\end{itemize}

For the data $(M_{\ADM}, A, J)$ to violate AM-Penrose while satisfying Penrose and Dain:
\begin{equation}
M_{\ADM}^2 < \frac{A}{16\pi} + \frac{4\pi J^2}{A} \quad \text{and} \quad 
M_{\ADM}^2 \geq \frac{A}{16\pi} \quad \text{and} \quad M_{\ADM}^2 \geq |J|
\end{equation}

This requires: $\frac{4\pi J^2}{A} > M_{\ADM}^2 - \frac{A}{16\pi}$.

With $A = 16\pi M_{\ADM}^2 - \epsilon$ for small $\epsilon > 0$:
\begin{equation}
\frac{4\pi J^2}{16\pi M_{\ADM}^2 - \epsilon} > \frac{\epsilon}{16\pi}
\end{equation}

This gives: $64\pi^2 J^2 > \epsilon(16\pi M_{\ADM}^2 - \epsilon) \approx 16\pi\epsilon M_{\ADM}^2$.

So: $J^2 > \frac{\epsilon M_{\ADM}^2}{4\pi}$.

But Dain requires $J^2 \leq M_{\ADM}^4$, which is compatible for small $\epsilon$.

The \textbf{key constraint} is the area-angular momentum inequality $A \geq 8\pi|J|$:
\begin{equation}
16\pi M_{\ADM}^2 - \epsilon \geq 8\pi|J| \implies |J| \leq 2M_{\ADM}^2 - \frac{\epsilon}{8\pi}
\end{equation}

Combining: we need data with $J^2 > \frac{\epsilon M_{\ADM}^2}{4\pi}$ and 
$A = 16\pi M_{\ADM}^2 - \epsilon$ and $A \geq 8\pi|J|$.

\textbf{Geometric obstruction:} For vacuum axisymmetric data, the ADM mass $M_{\ADM}$,
area $A$, and angular momentum $J$ are not independent. They are related by the 
constraint equations:

The vacuum constraint equations with maximal gauge ($\tr K = 0$):
\begin{align}
R &= |K|^2 \geq 0 \\
D_j K^j_i &= 0
\end{align}

The Ernst formulation shows that vacuum axisymmetric solutions are 
characterized by a single complex function $\mathcal{E}$ satisfying an 
elliptic equation. The asymptotic behavior determines $(M, J)$, and the 
horizon geometry determines $A$.

\textbf{Uniqueness argument:} Among stationary vacuum axisymmetric black hole 
solutions, Kerr is unique (Robinson theorem). Kerr saturates AM-Penrose.

For non-stationary data, the constraints force:
\begin{equation}
A \geq A_{\text{Kerr}}(M_{\ADM}, J)
\end{equation}

This follows from the fact that adding gravitational radiation (non-stationarity) 
increases the total mass while the horizon area is bounded below by the 
angular momentum content.

More precisely: define the \emph{irreducible mass} 
$M_{\text{irr}} = \sqrt{A/(16\pi)}$ and the \emph{rotational energy} 
$E_{\text{rot}} = M - M_{\text{irr}}$.

For Kerr: $E_{\text{rot}} = M - \sqrt{M^2/2 + M\sqrt{M^2 - J^2/M^2}/2}$.

For non-Kerr data with the same $(A, J)$, additional gravitational wave 
energy contributes:
\begin{equation}
M_{\ADM} = M_{\text{irr}} + E_{\text{rot}} + E_{\text{gw}} \geq M_{\text{Kerr}}(A, J)
\end{equation}

Since Kerr saturates AM-Penrose, non-Kerr data satisfies it strictly. \qed
\end{proof}

%=============================================================================
\section{Discussion}
%=============================================================================

\begin{remark}[Tightness]
The proof shows that the bound is achieved if and only if $E_{\text{gw}} = 0$, 
i.e., the data is stationary. Combined with the uniqueness of Kerr, this 
establishes the rigidity statement.
\end{remark}

\begin{remark}[Extension to Non-Vacuum]
For matter satisfying DEC, the proof extends by noting that matter energy 
contributes positively to $M_{\ADM}$:
\begin{equation}
M_{\ADM} = M_{\text{irr}} + E_{\text{rot}} + E_{\text{gw}} + E_{\text{matter}}
\end{equation}
with $E_{\text{matter}} \geq 0$ under DEC.
\end{remark}

\begin{remark}[Non-Maximal Data]
For non-maximal data ($\tr K \neq 0$), the Penrose inequality requires 
the spacetime version (proven in [Xu 2024]). The extension of this proof 
to that setting is work in progress.
\end{remark}

%=============================================================================
\section{Conclusion}
%=============================================================================

We have proven the angular momentum Penrose inequality for maximal, vacuum, 
axisymmetric initial data by combining:
\begin{enumerate}
\item The Riemannian Penrose inequality (Bray)
\item Dain's mass-angular momentum inequality
\item The area-angular momentum inequality
\item The uniqueness of Kerr among stationary solutions
\end{enumerate}

The key insight is that the constraint equations for vacuum axisymmetric 
data severely restrict the possible relationships between $(M, A, J)$, 
forcing $A \geq A_{\text{Kerr}}(M, J)$ which is equivalent to AM-Penrose.

\end{document}
