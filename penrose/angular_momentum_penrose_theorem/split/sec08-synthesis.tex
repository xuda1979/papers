\section{Synthesis: Complete Proof}\label{sec:synthesis}
%=============================================================================

\begin{mdframed}[linewidth=1pt, linecolor=green!60!black, backgroundcolor=green!3]
\begin{definition}[Cylindrical End Boundary Conditions---Formal Specification]\label{def:cylindrical-bc}
Let $(\bM, \bg)$ be the Jang manifold with cylindrical end $\mathcal{C} \cong [0, \infty)_t \times \Sigma$ at the MOTS. The \textbf{cylindrical end boundary conditions} for the AM-Lichnerowicz equation and AMO flow are:

\textbf{AM-Lichnerowicz equation} ($-8\Delta_{\bg}\phi + R_{\bg}\phi = \Lambda_J\phi^{-7}$):
\begin{enumerate}[label=(BC-L\arabic*)]
    \item \textbf{Asymptotic Dirichlet at MOTS:} $\phi(t, y) \to 1$ as $t \to \infty$, uniformly in $y \in \Sigma$.
    \item \textbf{Exponential decay:} $|\phi(t, y) - 1| \leq Ce^{-\kappa t}$ for some $\kappa = \kappa(\lambda_1(L_\Sigma)) > 0$.
    \item \textbf{Derived Neumann:} $\partial_t\phi(t, y) \to 0$ as $t \to \infty$ (consequence of (BC-L2)).
    \item \textbf{Asymptotic Dirichlet at infinity:} $\phi(x) \to 1$ as $r \to \infty$ in the asymptotically flat end.
\end{enumerate}

\textbf{AMO $p$-harmonic potential} ($\Delta_p u = 0$):
\begin{enumerate}[label=(BC-A\arabic*)]
    \item \textbf{Dirichlet at MOTS:} $u(t, y) \to 0$ as $t \to \infty$ along the cylindrical end.
    \item \textbf{Dirichlet at infinity:} $u(x) \to 1$ as $r \to \infty$ in the asymptotically flat end.
    \item \textbf{Monotonicity:} $u$ is strictly monotone from the MOTS (value 0) to infinity (value 1).
\end{enumerate}

\textbf{Well-posedness statements:}
\begin{itemize}
    \item Conditions (BC-L1)--(BC-L4) determine a unique positive solution $\phi > 0$ (Theorem~\ref{thm:lich-exist}).
    \item Conditions (BC-A1)--(BC-A3) determine a unique solution $u$ for each $p \in (1, 2]$.
    \item The boundary conditions are \textbf{asymptotic}, not imposed on a compact boundary.
\end{itemize}
\end{definition}
\end{mdframed}

\begin{mdframed}[linewidth=1pt, linecolor=black!50, backgroundcolor=blue!5]
\textbf{Hypothesis Usage Summary.} The four hypotheses enter the proof as follows:
\begin{itemize}[leftmargin=1.5em, itemsep=2pt]
  \item[\textbf{(H1)}] \textbf{DEC:} Ensures $R_{\tg} \geq 0$ after conformal transformation (Stage 2), which drives AMO monotonicity (Stage 6).
  \item[\textbf{(H2)}] \textbf{Axisymmetry:} Defines angular momentum $J$, guarantees axisymmetric solutions at every stage, and enables $J$-conservation (Stage 4).
  \item[\textbf{(H3)}] \textbf{Exterior vacuum:} Ensures Komar and ADM angular momenta coincide; enables clean asymptotics for boundary evaluation (Stage 7).
  \item[\textbf{(H4)}] \textbf{Strictly stable MOTS:} Guarantees $|\theta^-| > 0$, ensuring proper cylindrical blow-up and correct boundary values at $t = 0$ (Stage 7).
\end{itemize}
\end{mdframed}

\begin{mdframed}[linewidth=1pt, linecolor=red!40, backgroundcolor=red!3]
\textbf{Critical Distinction: Strictly Stable vs.\ Marginally Stable MOTS.}

A MOTS $\Sigma$ is characterized by its \textbf{stability operator} $L_\Sigma$, defined as the linearization of the null expansion $\theta^+$ under normal deformations:
\[
L_\Sigma \psi = -\Delta_\Sigma \psi + 2\omega \cdot \nabla_\Sigma \psi + \left(\tfrac{1}{2}\mathrm{Sc}_\Sigma - |\chi^+|^2 - |\omega|^2 + \mathrm{div}_\Sigma \omega\right)\psi,
\]
where $\omega$ is the connection 1-form on the normal bundle and $\chi^+$ is the shear of the outgoing null normal.

\textbf{Definition.}
\begin{itemize}[leftmargin=1.5em]
    \item \textbf{Strictly stable MOTS:} The principal eigenvalue satisfies $\lambda_1(L_\Sigma) > 0$. This implies $|\theta^-| > 0$ (the ingoing expansion is strictly negative).
    \item \textbf{Marginally stable MOTS:} $\lambda_1(L_\Sigma) = 0$, with a non-trivial kernel. This corresponds to the ``critical'' case $\theta^- = 0$ at isolated points or the entire surface.
    \item \textbf{Unstable MOTS:} $\lambda_1(L_\Sigma) < 0$.
\end{itemize}

\textbf{Role in the Proof.}
The strictly stable assumption $\lambda_1(L_\Sigma) > 0$ is used in the following essential ways:
\begin{enumerate}[label=(\arabic*)]
    \item \textbf{Jang equation blow-up rate (Stage 1):} The Jang solution $f$ satisfies $f \sim C_0 \ln(1/s)$ near the MOTS, where $C_0 = |\theta^-|/2$. For strictly stable MOTS, $|\theta^-| > 0$ gives $C_0 > 0$, ensuring the Jang metric becomes a proper cylinder with metric $dt^2 + g_\Sigma$. For marginally stable MOTS with $\theta^- = 0$, the blow-up rate degenerates ($C_0 = 0$), and the Jang solution may have a different asymptotic structure (e.g., conical rather than cylindrical).
    
    \item \textbf{Spectral gap for elliptic regularity (Stages 2--3):} The exponential decay $|\phi - 1| = O(e^{-\kappa t})$ on the cylinder relies on the spectral gap $\kappa = \kappa(\lambda_1) > 0$. When $\lambda_1 = 0$, the decay becomes polynomial rather than exponential, requiring different function space techniques (cf.\ Mazzeo--Pacard \cite{mazzeo-pacard-constant}).
    
    \item \textbf{Boundary value extraction (Stage 7):} The MOTS boundary value $m_{H,J}(0)$ is computed using $H_{\tilde{g}}(\Sigma) = 0$ (the conformal metric has minimal boundary). This follows from $H_{\bar{g}}(\Sigma) = 0$ (Lemma~\ref{lem:mots-boundary}) and the conformal scaling $H_{\tilde{g}} = \phi^{-2}(H_{\bar{g}} + 4\bar{\nu}(\ln\phi))$, where $\bar{\nu}(\ln\phi) \to 0$ by the exponential decay of $\phi - 1$.
\end{enumerate}

\textbf{Marginally Stable Case: Status.}
The marginally stable case $\lambda_1(L_\Sigma) = 0$ (which includes degenerate horizons such as extremal Kerr cross-sections) requires substantial modifications:
\begin{itemize}
    \item The Jang manifold structure near the MOTS may differ from a cylinder.
    \item Polynomial decay spaces (Melrose--Mendoza b-calculus) replace exponential decay spaces.
    \item The boundary value analysis requires careful treatment of logarithmic terms.
\end{itemize}
\textbf{This paper addresses only the strictly stable case.} Extension to marginally stable MOTS is an important direction for future work (see Section~\ref{sec:extensions}).
\end{mdframed}

\begin{proof}[Proof of Theorem~\ref{thm:main}]
Let $(M, g, K)$ be asymptotically flat, axisymmetric data satisfying DEC with outermost stable MOTS $\Sigma$.

\textbf{Stage 1:} By Theorem~\ref{thm:jang-exist}, solve the axisymmetric Jang equation to obtain $(\bM, \bg)$ with cylindrical ends at $\Sigma$.

\textbf{Stage 2:} By Theorem~\ref{thm:lich-exist}, solve the AM-Lichnerowicz equation to obtain $\tg = \phi^4 \bg$ with $R_{\tg} \geq 0$.

\textbf{Stage 3:} Solve the $p$-Laplacian on $(\tM, \tg)$:
\[
\Delta_p u_p = 0, \quad u_p|_\Sigma = 0, \quad u_p \to 1.
\]
The solution is axisymmetric.

\textbf{Stage 4:} By Theorem~\ref{thm:J-conserve}, $J(t) = J$ for all $t \in [0, 1]$.

\textbf{Stage 5:} By Theorem~\ref{thm:subext}, $A(t) \geq 8\pi|J|$ for all $t$.

\textbf{Stage 6:} By Theorem~\ref{thm:monotone}, $m_{H,J}(t)$ is monotone increasing.

\textbf{Stage 7:} Boundary values as $p \to 1^+$.

We establish the boundary values of $m_{H,J}(t)$ at $t = 0$ (the MOTS) and $t = 1$ (spatial infinity) with complete rigor.

\begin{lemma}[MOTS Boundary Value]\label{lem:mots-boundary}
Let $\Sigma$ be the outermost stable MOTS with area $A$ and Komar angular momentum $J$. On the conformal metric $\tg = \phi^4\bg$ restricted to the Jang manifold, the AM-Hawking mass at the MOTS satisfies:
\[
m_{H,J}(0) \geq \sqrt{\frac{A_{\tg}(\Sigma)}{16\pi} + \frac{4\pi J^2}{A_{\tg}(\Sigma)}},
\]
where $A_{\tg}(\Sigma) = \int_\Sigma dA_{\tg}$ is the area with respect to $\tg$.
\end{lemma}

\begin{proof}
We provide a complete derivation in four steps.

\textbf{Step 1: Geometric setup on the Jang manifold.}
On the Jang manifold $(\bM, \bg)$, the MOTS $\Sigma$ becomes the boundary of the cylindrical end. The key property is that the mean curvature $H_{\bg}$ of $\Sigma$ in $(\bM, \bg)$ vanishes, i.e., $\Sigma$ is a \textbf{minimal surface} in the Jang metric. We now prove this fact.

\textbf{Important clarification:} The physical MOTS condition is $\theta^+ = H_g + \tr_\Sigma K = 0$, where $H_g$ is the mean curvature in the \textbf{physical metric} $g$. This does \textbf{not} imply $H_g = 0$; rather, for non-time-symmetric data with $K \neq 0$, we have $H_g = -\tr_\Sigma K \neq 0$ generically. The property $H_{\bg} = 0$ (mean curvature in the \textbf{Jang metric}) is a separate geometric fact that follows from the cylindrical end structure of the Jang solution, as we now derive.

\textit{Detailed derivation of $H_{\bg}|_\Sigma = 0$:}
The Jang surface $\Gamma_f = \{(x, f(x)) : x \in M\}$ is embedded in $(M \times \mathbb{R}, g + dt^2)$. Near the MOTS $\Sigma$, the Jang solution $f$ blows up as (see Theorem~\ref{thm:jang-exist}(iii)):
\[
f(x) \sim C_0 \ln(1/s) + O(1) \quad \text{as } s \to 0,
\]
where $s = \mathrm{dist}_g(x, \Sigma)$ is the signed distance function and $C_0 = |\theta^-|/2 > 0$. The induced metric on $\Gamma_f$ is:
\[
\bg = g + df \otimes df = g + \frac{C_0^2 ds \otimes ds}{s^2} + O(s^{-1}).
\]
In the cylindrical coordinate $t = C_0\ln(1/s)$ (so $s = e^{-t/C_0}$), this becomes:
\[
\bg = dt^2 + g_\Sigma + O(e^{-\beta_0 t}),
\]
which is asymptotically a product cylinder $\mathbb{R}_+ \times \Sigma$.

\textbf{Rigorous proof of $H_{\bg}|_\Sigma = 0$:}
Consider the slice $\Sigma_t := \{t\} \times \Sigma$ in the cylindrical end. The second fundamental form of $\Sigma_t$ in $(\bM, \bg)$ is computed from the Lie derivative of the metric:
\[
h_{ij}(t) = \frac{1}{2}(\mathcal{L}_{\partial_t} \bg)_{ij}|_{\Sigma_t} = \frac{1}{2}\partial_t (g_{\Sigma_t})_{ij}.
\]
By Theorem~\ref{thm:jang-exist}(iii), the metric on $\Sigma_t$ converges exponentially to the MOTS metric:
\[
g_{\Sigma_t} = g_\Sigma + O(e^{-\beta_0 t}), \quad \partial_t g_{\Sigma_t} = O(e^{-\beta_0 t}).
\]
Therefore the second fundamental form satisfies $h_{ij}(t) = O(e^{-\beta_0 t})$, and the mean curvature:
\[
H_{\bg}(\Sigma_t) = \tr_{g_{\Sigma_t}} h(t) = O(e^{-\beta_0 t}) \to 0 \quad \text{as } t \to \infty.
\]
Taking the limit $t \to \infty$ (i.e., approaching the MOTS $\Sigma$ in the blow-up picture):
\begin{equation}\label{eq:hbg-sigma-zero}
H_{\bg}|_\Sigma := \lim_{t \to \infty} H_{\bg}(\Sigma_t) = 0.
\end{equation}
This is the key geometric fact: the MOTS $\Sigma$ is a \textbf{minimal surface} in the Jang metric $\bg$.

\textit{Physical interpretation:} The vanishing $H_{\bg}|_\Sigma = 0$ does \textbf{not} follow directly from the MOTS condition $\theta^+ = 0$. Rather, it follows from the \textbf{cylindrical structure} of the Jang blow-up: near the MOTS, the Jang manifold asymptotes to an infinite cylinder, and cross-sections of product cylinders are totally geodesic (hence have zero mean curvature).

\textit{Alternative argument via null expansion:}
The Jang equation and the MOTS condition are related by:
\[
\mathcal{J}(f) = H_g + \Div_g\left(\frac{\nabla f}{\sqrt{1+|\nabla f|^2}}\right) - \tr_g K - \frac{\langle K, \nabla f \otimes \nabla f\rangle}{1 + |\nabla f|^2} = 0.
\]
Near a MOTS with $\theta^+ = H_g - \tr_g K = 0$, the blow-up behavior $f \to \infty$ with $|\nabla f| \sim 1/s$ ensures that the divergence term dominates, effectively encoding the MOTS condition into the cylindrical end structure. The resulting minimal surface condition $H_{\bg}|_\Sigma = 0$ is a consequence of the variational structure: the Jang surface $\Gamma_f$ is a critical point of the area functional in $(M \times \mathbb{R}, g + dt^2)$, and $\Sigma$ (as the boundary of the cylindrical end) inherits the minimal surface property.

\textbf{Step 2: Conformal transformation of mean curvature.}
Under the conformal change $\tg = \phi^4\bg$, the mean curvature transforms as:
\[
H_{\tg} = \phi^{-2}\left(H_{\bg} + 4\frac{\partial_\nu \phi}{\phi}\right),
\]
where $\nu$ is the unit normal in $(\bM, \bg)$. Since $H_{\bg}|_\Sigma = 0$:
\[
H_{\tg}|_\Sigma = 4\phi^{-3}\partial_\nu\phi|_\Sigma.
\]
By the boundary behavior of the AM-Lichnerowicz solution (Theorem~\ref{thm:lich-exist}), the conformal factor satisfies:
\[
\phi|_\Sigma = 1, \quad \partial_\nu\phi|_\Sigma = 0.
\]
The Dirichlet condition $\phi|_\Sigma = 1$ comes from the normalization. The Neumann condition $\partial_\nu\phi|_\Sigma = 0$ requires careful justification:

\textit{Derivation of $\partial_\nu\phi|_\Sigma = 0$:} On the cylindrical end modeled as $[0,\infty)_t \times \Sigma$, the AM-Lichnerowicz equation takes the form:
\[
-8\left(\partial_t^2\phi + \Delta_\Sigma\phi\right) + R_{\bg}\phi = \Lambda_J \phi^{-7} + O(e^{-\beta_0 t})\text{(error terms)}.
\]
Since $R_{\bg} \to R_\Sigma$ and $\Lambda_J \to 0$ exponentially as $t \to \infty$ (by the asymptotic cylindrical structure), the limiting equation is the eigenvalue problem $-\Delta_\Sigma\phi_\infty = 0$ on $\Sigma$. The only constant solution is $\phi_\infty = 1$ (by the normalization), which satisfies $\nabla_\Sigma\phi_\infty = 0$.

More precisely, from Lemma~\ref{lem:phi-bound}, $\phi = 1 + \psi$ where $|\psi| = O(e^{-\kappa t})$ for some $\kappa > 0$. Differentiating:
\[
\partial_t\phi = \partial_t\psi = O(e^{-\kappa t}) \to 0 \quad \text{as } t \to \infty.
\]
Since $\nu = \partial_t$ in the cylindrical coordinates, this gives $\partial_\nu\phi|_\Sigma = \lim_{t \to \infty}\partial_t\phi = 0$.

\begin{mdframed}[linewidth=1pt, linecolor=green!70!black, backgroundcolor=green!5]
\textbf{Critical Clarification: Boundary Conditions on the MOTS.}

The Dirichlet condition $\phi|_\Sigma = 1$ and the Neumann condition $\partial_\nu\phi|_\Sigma = 0$ require careful justification, as the MOTS $\Sigma$ is not a boundary in the classical PDE sense but rather the ``end'' of a cylindrical neck.

\textbf{Why $\phi|_\Sigma = 1$ (Dirichlet):}
\begin{itemize}
    \item The MOTS $\Sigma$ becomes the boundary of the cylindrical end in the Jang manifold. In cylindrical coordinates $(t, y) \in [0,\infty) \times \Sigma$, ``$\Sigma$'' corresponds to $t \to \infty$.
    \item The AM-Lichnerowicz equation is solved on the \textbf{compact} region $\bM_T = \bM \setminus \{t > T\}$ with Dirichlet data $\phi = 1$ on the ``cap'' $\{t = T\} \times \Sigma$.
    \item As $T \to \infty$, the solutions converge to the unique solution on the full $\bM$ with asymptotic value $\phi \to 1$ along the cylindrical end.
    \item This is \textbf{not an arbitrary choice}: the normalization $\phi = 1$ at the MOTS ensures $A_{\tg}(\Sigma) = A_g(\Sigma)$, which is essential for the correct boundary value of $m_{H,J}(0)$.
\end{itemize}

\textbf{Why $\partial_\nu\phi|_\Sigma = 0$ (Neumann):}
\begin{itemize}
    \item The exponential decay of $\phi - 1$ on the cylinder (from the spectral gap of $\Delta_\Sigma$) implies all derivatives also decay: $|\partial_t^k(\phi - 1)| = O(e^{-\kappa t})$.
    \item The limiting value $\partial_\nu\phi|_\Sigma = \lim_{t \to \infty}\partial_t\phi = 0$ is a \textbf{consequence} of the equation, not an imposed condition.
    \item Physically, this reflects the fact that $\Sigma$ is a ``barrier'' for the conformal factor: the geometry cannot ``leak'' mass through the minimal surface.
\end{itemize}

\textbf{Summary of well-posedness:} The AM-Lichnerowicz equation on $\bM$ with:
\begin{enumerate}
    \item[(i)] Asymptotic condition $\phi \to 1$ at spatial infinity, and
    \item[(ii)] Asymptotic condition $\phi \to 1$ along the cylindrical end at $\Sigma$
\end{enumerate}
admits a unique positive solution $\phi > 0$. Both conditions are asymptotic behavior constraints, not boundary data in the classical sense. The Neumann condition $\partial_\nu\phi|_\Sigma = 0$ is derived, not imposed.
\end{mdframed}

\textbf{Rigorous justification of $\partial_\nu\phi|_\Sigma = 0$ via elliptic estimates:} We provide three independent arguments for completeness:

\begin{enumerate}[label=\textup{(\roman*)}]
    \item \textbf{Variational argument:} The AM-Lichnerowicz equation is the Euler--Lagrange equation for the functional $\mathcal{E}[\phi] = \int 8|\nabla\phi|^2 + R_{\bg}\phi^2 + \frac{\Lambda_J}{6}\phi^{-6}$. On a manifold with minimal boundary (which $\Sigma$ is, by $H_{\bg}|_\Sigma = 0$), the natural boundary condition for critical points is Neumann: $\partial_\nu\phi = 0$ (see \cite[Proposition 3.2]{miao2002}).

    \item \textbf{Exponential decay argument:} On the cylindrical end $\mathcal{C} \cong [0,\infty)_t \times \Sigma$, write $\phi = 1 + \psi$ where $\psi$ solves a linear elliptic equation with source terms decaying as $O(e^{-\beta_0 t})$. By standard elliptic estimates on cylinders (Lockhart--McOwen theory), $\psi$ and all its derivatives decay exponentially: $|\partial_t^k \partial_\Sigma^\ell \psi| = O(e^{-\kappa t})$ for some $\kappa > 0$ determined by the spectral gap. In particular, $\partial_t\phi = \partial_t\psi = O(e^{-\kappa t}) \to 0$ as $t \to \infty$.

    \item \textbf{Uniqueness argument:} The AM-Lichnerowicz equation on $\bM$ with boundary $\partial\bM = \Sigma$ (at infinity along the cylinder) and asymptotic condition $\phi \to 1$ at spatial infinity admits a unique solution. This solution is obtained as the limit of Dirichlet problems on $\bM_T = \bM \setminus (\{t > T\} \times \Sigma)$ with $\phi|_{\{t=T\} \times \Sigma} = 1$. By the maximum principle, $\phi \leq 1$ throughout (since $\Lambda_J \geq 0$ makes 1 a supersolution). The limit $T \to \infty$ converges to the unique solution with $\phi|_\Sigma = 1$ and (by the exponential decay of gradients) $\partial_\nu\phi|_\Sigma = 0$.
\end{enumerate}

Therefore:
\[
H_{\tg}|_\Sigma = 0.
\]
The MOTS $\Sigma$ is also a \textbf{minimal surface in the conformal metric $\tg$}.

\textbf{Step 3: Hawking mass of a minimal surface.}
The Hawking mass of a 2-surface $\Sigma$ is:
\[
m_H(\Sigma) = \sqrt{\frac{A}{16\pi}}\left(1 - \frac{1}{16\pi}\int_\Sigma H^2 dA\right).
\]
For a minimal surface ($H = 0$):
\[
m_H(\Sigma) = \sqrt{\frac{A_{\tg}(\Sigma)}{16\pi}}.
\]
This is the irreducible mass of the surface.

\textbf{Step 4: AM-Hawking mass lower bound.}
The AM-Hawking mass is defined as:
\[
m_{H,J}(\Sigma) = \sqrt{m_H^2(\Sigma) + \frac{4\pi J^2}{A_{\tg}(\Sigma)}}.
\]
For a minimal surface:
\[
m_{H,J}(\Sigma) = \sqrt{\frac{A_{\tg}(\Sigma)}{16\pi} + \frac{4\pi J^2}{A_{\tg}(\Sigma)}}.
\]
This is precisely the desired lower bound.
\end{proof}

\begin{lemma}[Area Relationship Under Conformal Change]\label{lem:area-conformal}
Let $\Sigma \subset M$ be the outermost MOTS with physical area $A := A_g(\Sigma) = \int_\Sigma dA_g$. Then:
\begin{enumerate}[label=\textup{(\roman*)}]
    \item \textbf{Jang area equals physical area:} $A_{\bg}(\Sigma) = A_g(\Sigma) = A$.
    \item \textbf{Conformal area at boundary:} $A_{\tg}(\Sigma) = A_{\bg}(\Sigma) = A$ (using $\phi|_\Sigma = 1$).
\end{enumerate}
\end{lemma}

\begin{proof}
\textbf{(i) Jang vs.\ physical area.}
The Jang metric is $\bg = g + df \otimes df$ where $f$ solves the Jang equation. On the MOTS $\Sigma$, the function $f$ has controlled behavior due to the cylindrical end structure.

In the cylindrical coordinate $t = -\ln s$ (where $s = \mathrm{dist}_g(\cdot, \Sigma)$), the Jang solution satisfies:
\[
f(s, y) = C_0 \ln(1/s) + \mathcal{A}(y) + O(s^\alpha) = C_0 t + \mathcal{A}(y) + O(e^{-\alpha t}).
\]
The gradient $\nabla_g f = -C_0/s \cdot \nabla s + O(1) = C_0 \partial_t + O(e^{-\beta t})$ in the cylindrical picture.

The key observation: the MOTS $\Sigma$ in the Jang manifold $(\bM, \bg)$ is approached as $t \to \infty$. For any finite $T$, the slice $\Sigma_T := \{t = T\} \cong \Sigma$ has induced metric:
\[
\bg|_{\Sigma_T} = (dt^2 + g_\Sigma + O(e^{-\beta_0 t}))|_{dt=0} = g_\Sigma + O(e^{-\beta_0 T}).
\]
Taking $T \to \infty$:
\[
A_{\bg}(\Sigma) := \lim_{T \to \infty} \int_{\Sigma_T} dA_{\bg} = \lim_{T \to \infty} \int_\Sigma (1 + O(e^{-\beta_0 T})) dA_{g_\Sigma} = \int_\Sigma dA_{g_\Sigma} = A_g(\Sigma).
\]

\textbf{Alternative argument via boundary term.}
On the physical manifold, the Jang metric satisfies $\bg|_{\Sigma} = g|_\Sigma + (df \otimes df)|_\Sigma$. By the blow-up structure, $df|_\Sigma$ is \textbf{purely normal} to $\Sigma$: $df = C_0 \cdot ds/s + O(1)$, so $(df)^{\text{tan}} = 0$ on $\Sigma$. Therefore $(df \otimes df)|_{\Sigma}$ contributes only in the normal-normal component, which does not affect the induced metric on $\Sigma$:
\[
\bg|_\Sigma = g|_\Sigma \quad \Rightarrow \quad A_{\bg}(\Sigma) = A_g(\Sigma).
\]

\textbf{Rigorous justification via induced metric formula.}
Let $\{e_1, e_2\}$ be an orthonormal frame for $T\Sigma$ in the metric $g$. The induced metric components on $\Sigma$ are:
\[
(\bg|_\Sigma)_{ab} = \bg(e_a, e_b) = g(e_a, e_b) + df(e_a) \cdot df(e_b).
\]
Since $f$ blows up in the normal direction with $\nabla_g f = C_0 \nu/s + O(1)$ (where $\nu \perp T\Sigma$), we have:
\[
df(e_a) = g(\nabla f, e_a) = C_0 s^{-1} g(\nu, e_a) + O(1) = 0 + O(1)
\]
because $\nu \perp e_a$. Thus $df(e_a) = O(1)$ remains bounded, and in the limit $s \to 0$:
\[
\lim_{s \to 0} (\bg|_\Sigma)_{ab} = g(e_a, e_b) + \lim_{s \to 0} O(1) \cdot O(1) = (g|_\Sigma)_{ab}.
\]
More precisely, on the slices $\Sigma_T$ at cylindrical height $T$, the tangential gradient $|df^{\text{tan}}|$ decays as $O(e^{-\beta_0 T})$, so $|(\bg - g)|_{\Sigma_T}| = O(e^{-2\beta_0 T}) \to 0$.

\textbf{(ii) Conformal area.}
Under the conformal change $\tg = \phi^4 \bg$, the area element transforms as:
\[
dA_{\tg} = \phi^4 \cdot dA_{\bg} \quad \text{(in 2D)}.
\]
Since $\phi|_\Sigma = 1$ (Theorem~\ref{thm:lich-exist}(i)):
\[
A_{\tg}(\Sigma) = \int_\Sigma \phi^4 \, dA_{\bg} = \int_\Sigma 1 \cdot dA_{\bg} = A_{\bg}(\Sigma) = A.
\]
\end{proof}

\begin{mdframed}[linewidth=1.5pt, linecolor=black, backgroundcolor=yellow!8, nobreak=true]
\textbf{Mini Proof: MOTS Boundary Value.}\label{box:mots-mini-proof}
The key equation $m_{H,J}(0) = \sqrt{A/(16\pi) + 4\pi J^2/A}$ follows from:
\begin{enumerate}[label=\textup{(\arabic*)}, leftmargin=2em, itemsep=1pt]
    \item \textbf{Minimality in $\bg$:} The MOTS $\Sigma$ is the boundary of the cylindrical end $\mathcal{C} \cong [0,\infty) \times \Sigma$ in the Jang manifold. Cylindrical slices are asymptotically totally geodesic, so $H_{\bg}|_\Sigma = 0$.
    
    \item \textbf{Neumann boundary for $\phi$:} On the cylinder, exponential decay $\phi = 1 + O(e^{-\kappa t})$ implies $\partial_\nu\phi|_\Sigma = 0$. This plus $\phi|_\Sigma = 1$ yields $H_{\tg}|_\Sigma = \phi^{-2}(H_{\bg} + 4\phi^{-1}\partial_\nu\phi)|_\Sigma = 0$.
    
    \item \textbf{Hawking mass of minimal surface:} For $H_{\tg}|_\Sigma = 0$: $m_H(\Sigma) = \sqrt{A_{\tg}(\Sigma)/(16\pi)}$.
    
    \item \textbf{Area preservation:} Since $df|_\Sigma$ is purely normal and $\phi|_\Sigma = 1$: $A_{\tg}(\Sigma) = A_{\bg}(\Sigma) = A_g(\Sigma) = A$.
    
    \item \textbf{Conclusion:} $\displaystyle m_{H,J}(0) = \sqrt{m_H^2 + \frac{4\pi J^2}{A}} = \sqrt{\frac{A}{16\pi} + \frac{4\pi J^2}{A}}$.
\end{enumerate}
\vspace{-0.5em}
\textit{This is an equality, not merely a lower bound.}
\end{mdframed}
\vspace{0.3cm}

\begin{remark}[Clarification: Cylindrical End vs.\ Level Set at $t = 0$]\label{rem:t0-clarification}
The boundary value at $t = 0$ requires careful interpretation because the MOTS $\Sigma$ corresponds to the ``end'' of the cylindrical region in the Jang manifold, not a finite surface. We clarify the limiting procedure:
\begin{enumerate}
    \item \textbf{Cylindrical coordinate:} On the Jang manifold, the cylindrical end $\mathcal{C} \cong [0, \infty) \times \Sigma$ has coordinate $t = -\ln s$ where $s = \mathrm{dist}(\cdot, \Sigma)$. The ``boundary'' $\Sigma$ corresponds to $t \to +\infty$ in this coordinate.
    \item \textbf{Level set parametrization:} The AMO potential $u: \tM \to [0, 1]$ satisfies $u \to 0$ as $t \to +\infty$ (along the cylinder) and $u \to 1$ at spatial infinity. Thus $\Sigma_t = \{u = t\}$ with $t \in (0, 1)$ are level sets in the interior, and $\Sigma_0 = \lim_{t \to 0^+} \Sigma_t$ is the MOTS.
    \item \textbf{Limit of $m_{H,J}(t)$:} The value $m_{H,J}(0)$ is defined as $\lim_{t \to 0^+} m_{H,J}(t)$. By the continuity of area and the fact that $\Sigma_t \to \Sigma$ in the Hausdorff topology (with controlled curvature from the $p$-harmonic structure), this limit equals the AM-Hawking mass computed directly on $\Sigma$ via Lemmas~\ref{lem:mots-boundary} and \ref{lem:area-conformal}.
\end{enumerate}
The key point is that the MOTS $\Sigma$ is minimal in $(\tM, \tg)$, so the Willmore integral $\int H^2 = 0$ and the limiting Hawking mass is exactly $\sqrt{A/(16\pi)}$.
\end{remark}

\begin{remark}[Regularity of the Conformal Metric at the MOTS Boundary]\label{rem:conformal-regularity-mots}
A potential concern is whether the conformal metric $\tilde{g} = \phi^4 \bar{g}$ is sufficiently regular at the MOTS $\Sigma$ for the AMO flow to be well-defined. We address this as follows:
\begin{enumerate}
    \item \textbf{Jang metric regularity:} The Jang metric $\bar{g} = g + df \otimes df$ on the cylindrical end $\mathcal{C} \cong [0,\infty) \times \Sigma$ converges exponentially to the product metric $dt^2 + g_\Sigma$ with rate $\beta_0 > 0$ (Theorem~\ref{thm:jang-exist}). Thus $\bar{g}$ is smooth (in fact, $C^\infty$) on the interior and has controlled decay along the cylinder.
    
    \item \textbf{Conformal factor regularity:} By Theorem~\ref{thm:lich-exist} and Lemma~\ref{lem:phi-bound}, the conformal factor $\phi$ satisfies $\phi = 1 + O(e^{-\kappa t})$ with all derivatives decaying exponentially along the cylindrical end. Thus $\phi \in C^\infty(\bar{M})$ with $\phi|_\Sigma = 1$.
    
    \item \textbf{Conformal metric regularity:} Since $\tilde{g} = \phi^4 \bar{g}$ with $\phi \to 1$ and $\bar{g} \to dt^2 + g_\Sigma$ exponentially as $t \to \infty$, the conformal metric $\tilde{g}$ is asymptotically a product cylinder with smooth cross-section $\Sigma$. In particular, $\tilde{g}$ extends smoothly to the boundary $\Sigma$ (in the sense of asymptotic completeness).
    
    \item \textbf{AMO flow well-posedness:} The $p$-harmonic potential $u: \tilde{M} \to [0,1]$ with $u|_\Sigma = 0$ and $u \to 1$ at infinity is well-defined on manifolds with cylindrical ends. The level sets $\Sigma_t = \{u = t\}$ for $t \in (0,1)$ are smooth, and the limiting behavior as $t \to 0^+$ is controlled by the cylindrical end geometry. The standard regularity theory for $p$-harmonic functions \cite{heinonen1993, tolksdorf1984} applies on the interior, and the boundary behavior is determined by the Dirichlet problem on the product cylinder.
    
    \item \textbf{Mean curvature regularity:} Since the level sets $\Sigma_t$ are $C^{1,\Hoelder}$ regular for $p \in (1,2]$ \cite{amo2022}, the mean curvature $H$ and second fundamental form $h$ are well-defined almost everywhere. The Hawking mass integral $\int_{\Sigma_t} H^2 \, dA$ is finite for regular level sets.
\end{enumerate}
In summary, the conformal metric $\tilde{g}$ has sufficient regularity (smooth on the interior, asymptotically product on the cylindrical end with smooth boundary) for all constructions in the AMO framework.
\end{remark}

Combining Lemmas~\ref{lem:mots-boundary} and \ref{lem:area-conformal}:
\[
m_{H,J}(0) = \sqrt{\frac{A}{16\pi} + \frac{4\pi J^2}{A}},
\]
where $A$ is the area of the MOTS in the \textbf{original physical metric} $g$.

\begin{itemize}
    \item \textbf{At $t = 0$ (MOTS):} By Lemmas~\ref{lem:mots-boundary} and \ref{lem:area-conformal}:
    \[
    m_{H,J}(0) = \sqrt{\frac{A}{16\pi} + \frac{4\pi J^2}{A}}.
    \]
    This is an \textbf{equality}, not merely a lower bound, because the MOTS is minimal in both $\bg$ and $\tg$.
    
    \item \textbf{At $t = 1$ (infinity):} The level sets $\Sigma_t$ approach spatial infinity. We establish the precise convergence:
    
    \begin{lemma}[ADM Mass Convergence]\label{lem:adm-convergence}
    Let $(\tM, \tg)$ be an asymptotically flat 3-manifold with $\tg_{ij} = \delta_{ij} + O(r^{-\tau})$ and $\partial_k \tg_{ij} = O(r^{-\tau-1})$ for some $\tau > 1/2$. Let $u: \tM \to [0,1]$ be the $p$-harmonic potential with level sets $\Sigma_t = \{u = t\}$. Then:
    \begin{enumerate}[label=\textup{(\roman*)}]
        \item \textbf{Area growth:} $A(t) = 4\pi r(t)^2(1 + O(r(t)^{-\tau}))$ where $r(t) \to \infty$ as $t \to 1^-$;
        \item \textbf{Mean curvature decay:} $H(\Sigma_t) = \frac{2}{r(t)}(1 + O(r(t)^{-\tau}))$;
        \item \textbf{Willmore convergence:} $W(t) = \frac{1}{16\pi}\int_{\Sigma_t} H^2 dA = 1 - \frac{2M_{\ADM}(\tg)}{r(t)} + O(r(t)^{-1-\tau})$;
        \item \textbf{Hawking mass limit:} $\displaystyle\lim_{t \to 1^-} m_H(t) = M_{\ADM}(\tg)$.
    \end{enumerate}
    \end{lemma}
    
    \begin{proof}[Proof sketch]
    The proof follows \cite[Theorem 1.3]{amo2022}. Near infinity, the $p$-harmonic potential satisfies $u \approx 1 - C/r^{n-2}$ (Green's function behavior). For $n = 3$: $u \approx 1 - C/r$, so level sets $\{u = t\}$ are approximately coordinate spheres of radius $r(t) \approx C/(1-t)$. The Hawking mass formula gives:
    \begin{align*}
    m_H(t) &= \sqrt{\frac{A(t)}{16\pi}}\left(1 - W(t)\right) \\
    &\approx \frac{r(t)}{2}\left(\frac{2M_{\ADM}}{r(t)} + O(r(t)^{-1-\tau})\right) \\
    &= M_{\ADM} + O(r(t)^{-\tau}) \to M_{\ADM}(\tg).
    \end{align*}
    The expansion uses the standard ADM mass formula: for coordinate spheres $S_r$, $\int_{S_r} H^2 dA = 16\pi - 32\pi M_{\ADM}/r + O(r^{-1-\tau})$, giving $1 - W(t) = 2M_{\ADM}/r(t) + O(r^{-1-\tau})$.
    \end{proof}
    
    For the angular momentum term: as $t \to 1$, the area $A(t) \sim r(t)^2 \to \infty$ while $J(t) = J$ remains constant (Theorem~\ref{thm:J-conserve}). Therefore:
    \[
    \frac{4\pi J^2}{A(t)} = O(r(t)^{-2}) \to 0 \quad \text{as } t \to 1.
    \]
    Combining:
    \[
    m_{H,J}(1) = \lim_{t \to 1^-}\sqrt{m_H^2(t) + \frac{4\pi J^2}{A(t)}} = \sqrt{M_{\ADM}(\tg)^2 + 0} = M_{\ADM}(\tg).
    \]
\end{itemize}

\textbf{Conclusion:}
By the monotonicity from Stage 6 and the mass chain from Lemma~\ref{lem:phi-bound}:
\[
M_{\ADM}(g) \geq M_{\ADM}(\tg) = m_{H,J}(1) \geq m_{H,J}(0) \geq \sqrt{\frac{A}{16\pi} + \frac{4\pi J^2}{A}}.
\]
The last inequality uses the lower bound analysis from Stage 7 at the MOTS, which becomes an equality for Kerr initial data.
\end{proof}

%=============================================================================
