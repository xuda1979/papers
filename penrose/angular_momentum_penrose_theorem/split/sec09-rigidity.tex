\section{Rigidity}\label{sec:rigidity}
%=============================================================================

\begin{theorem}[Equality Case]\label{thm:rigidity}
Equality in \eqref{eq:main} holds if and only if $(M, g, K)$ arises from a spacelike slice of the Kerr spacetime.
\end{theorem}

\begin{remark}[Initial Data vs.\ Spacetime Rigidity]\label{rem:initial-data-rigidity}
It is essential to distinguish between \textbf{initial data rigidity} and \textbf{spacetime rigidity}:

\begin{enumerate}[label=\textup{(\alph*)}]
    \item \textbf{Initial data rigidity (what we prove):} If the initial data $(M, g, K)$ satisfies the equality $M_{\ADM} = \sqrt{A/(16\pi) + 4\pi J^2/A}$, then $(M, g, K)$ is isometric to a slice of the Kerr spacetime (as a Cauchy surface with induced metric $g$ and extrinsic curvature $K$).
    
    \item \textbf{Spacetime rigidity (follows from evolution):} The maximal Cauchy development of such initial data \textbf{is} the Kerr spacetime. This follows from the uniqueness of maximal globally hyperbolic developments and the Carter--Robinson theorem \cite{carter1971, robinson1975}.
\end{enumerate}

The distinction is logically important: our theorem operates entirely within the initial data formalism and does not directly invoke spacetime existence. The spacetime conclusion follows only after appealing to the well-posedness of the Einstein evolution equations and the black hole uniqueness theorems.

\textbf{Logical structure:}
\[
\text{Equality holds} \xRightarrow{\text{Thm.~\ref{thm:rigidity}}} \text{Initial data is Kerr slice} \xRightarrow{\text{Uniqueness}} \text{Spacetime is Kerr}.
\]
The first implication is geometric analysis (this paper); the second invokes the standard uniqueness results \cite{chrusciel2008}.
\end{remark}

\begin{remark}[Physical Interpretation of Rigidity]\label{rem:rigidity-physics}
The rigidity theorem has a compelling physical interpretation: \textbf{Kerr black holes are the most efficient configurations} for storing angular momentum at fixed mass, or equivalently, for minimizing mass at fixed angular momentum and horizon area.

\textbf{Why Kerr saturates the bound:} The equality case requires three conditions to hold simultaneously:
\begin{enumerate}
    \item \textbf{Kerr geometry (Mars--Simon tensor vanishes):} $\mathcal{S}_{(g,K)} = 0$, meaning the Kerr deviation tensor vanishes. This is the correct characterization---\textbf{not} $\sigma^{TT} = 0$. Generic Kerr slices (e.g., Boyer--Lindquist) have $\sigma^{TT} \neq 0$ because they are not conformally flat, but they satisfy $\mathcal{S}_{(g,K)} = 0$ because they are slices of Kerr.
    
    \item \textbf{Stationarity:} The condition $\mathcal{S}_{(g,K)} = 0$ implies (via the Mars uniqueness theorem \cite{mars1999, mars2009}) that the spacetime development is locally isometric to Kerr, which is stationary.
    
    \item \textbf{Optimal angular momentum storage:} Kerr's ergoregion geometry represents the unique axisymmetric, vacuum, stationary configuration that maximizes the ratio $|J|/M^2$ for a given horizon structure.
\end{enumerate}

\textbf{Critical clarification:} The characterization $\sigma^{TT} = 0$ appearing in earlier versions was \textbf{incorrect}. Kerr slices generically have $\sigma^{TT} \neq 0$. The correct characterization uses the Mars--Simon/Kerr deviation tensor $\mathcal{S}_{(g,K)}$, which vanishes for \textbf{any} slice of Kerr regardless of the slicing choice.

\textbf{Energy interpretation:} The mass deficit $\delta = M_{\ADM} - \sqrt{A/(16\pi) + 4\pi J^2/A}$ can be interpreted as the total energy available for extraction through:
\begin{itemize}
    \item Gravitational wave emission (reducing the non-Kerr content $|\mathcal{S}_{(g,K)}|^2$);
    \item Matter accretion or ejection (adjusting $J$ and $A$);
    \item Superradiant scattering (for near-extremal configurations).
\end{itemize}
Any dynamical process that extracts this energy brings the black hole closer to the Kerr endpoint.

\textbf{Cosmic censorship connection:} The rigidity result is the ``positive direction'' of cosmic censorship for rotating black holes: not only is there a geometric lower bound on mass (weak censorship), but the unique configuration saturating this bound is the Kerr solution (strong uniqueness). This rules out ``exotic'' black holes with the same $(A, J)$ but different spacetime structure.
\end{remark}

\begin{proof}
\textbf{Roadmap of the rigidity argument:}
\begin{enumerate}
    \item \textbf{Monotonicity equality} ($M_{\ADM} = m_{H,J}(0) = m_{H,J}(1)$) $\Rightarrow$ $\frac{d}{dt}m_{H,J}(t) = 0$ for all $t$.
    \item \textbf{Vanishing derivative} $\Rightarrow$ Geroch integrand vanishes: $R_{\tilde{g}} = 0$, level sets are umbilic ($\mathring{h} = 0$), and conformal factor $\phi \equiv 1$.
    \item \textbf{Conformal constraint} ($\phi = 1$) $\Rightarrow$ mass comparison is equality: $M_{\ADM}(g) = M_{\ADM}(\tilde{g})$, and $\Lambda_J = \frac{1}{8}|\mathcal{S}_{(g,K)}|^2 = 0$.
    \item \textbf{$\mathcal{S}_{(g,K)} = 0$} $\Rightarrow$ data is a Kerr slice (by Mars uniqueness theorem \cite{mars2009}); combined with vacuum and axisymmetry, uniqueness theorems identify the solution as Kerr.
\end{enumerate}
We now execute each step in detail.

\textbf{Step 1: Monotonicity equality conditions.}
Suppose equality holds:
\[
M_{\ADM} = \sqrt{\frac{A}{16\pi} + \frac{4\pi J^2}{A}}.
\]
By the proof of Theorem~\ref{thm:main}, this means $m_{H,J}(0) = m_{H,J}(1)$. Since $m_{H,J}(t)$ is monotone increasing (Theorem~\ref{thm:monotone}), we must have:
\[
\frac{d}{dt} m_{H,J}(t) = 0 \quad \text{for almost all } t \in (0, 1).
\]

\textbf{Step 2: Vanishing of rigidity terms.}
We analyze two cases based on whether the data is extremal.

\textit{Case 2a: Strictly sub-extremal data ($A(t) > 8\pi|J|$ for all $t$).}
For $\frac{d}{dt} m_{H,J}(t) = 0$ with $A(t) > 8\pi|J|$ (strict sub-extremality), we need the Geroch-type formula \eqref{eq:geroch-am} to vanish, which requires the integrand to vanish:
\begin{enumerate}
    \item $R_{\tg} = 0$ on all level sets $\Sigma_t$;
    \item $\mathring{h} = 0$, i.e., level sets are \textbf{umbilic} (constant mean curvature);
    \item The Hawking mass is constant along the flow.
\end{enumerate}

\textit{Case 2b: Extremal data ($A(0) = 8\pi|J|$).}
If the initial MOTS $\Sigma$ achieves the extremal bound $A = 8\pi|J|$, then by the Dain--Reiris rigidity \cite{dain2011}, $\Sigma$ is isometric to an extreme Kerr horizon. We analyze this case separately.

From the derivative formula (proof of Theorem~\ref{thm:monotone}):
\[
\frac{d}{dt} m_{H,J}^2 = \frac{d}{dt} m_H^2 + \frac{d}{dt}\left(\frac{4\pi J^2}{A(t)}\right).
\]
Using the Geroch-type monotonicity for $m_H^2$ and the area monotonicity:
At $t = 0$, if $A(0) = 8\pi|J|$, the angular momentum contribution $4\pi J^2/A(0) = \pi J^2/(2|J|) = \pi|J|/2$. This means $\frac{d}{dt} m_{H,J}(0)$ can be zero even with $A'(0) > 0$, which occurs generically. However, for $t > 0$, since $A'(t) \geq 0$ and thus $A(t) \geq A(0) = 8\pi|J|$, we have either:
\begin{itemize}
    \item $A(t) > 8\pi|J|$ for $t > 0$: Then the sub-extremality factor is positive, and $\frac{d}{dt} m_{H,J}(t) \geq 0$ from the Geroch-type formula. For equality $m_{H,J}(0) = m_{H,J}(1)$, we need $\frac{d}{dt} m_{H,J}(t) = 0$ for all $t$, which forces the integrand in \eqref{eq:geroch-am} to vanish.
    \item $A(t) = 8\pi|J|$ for all $t$: This means all level sets achieve the extremal bound. We justify below that this forces the data to be extreme Kerr.
\end{itemize}

\begin{lemma}[Extremal Foliation Implies Extreme Kerr]\label{lem:extremal-foliation}
Let $(M, g, K)$ be axisymmetric, vacuum initial data with a foliation $\{\Sigma_t\}_{t \in [0,1]}$ such that:
\begin{enumerate}
    \item Each $\Sigma_t$ is a stable, axisymmetric 2-sphere;
    \item The angular momentum $J(\Sigma_t) = J$ is constant;
    \item Each $\Sigma_t$ achieves the Dain--Reiris bound: $A(\Sigma_t) = 8\pi|J|$.
\end{enumerate}
Then $(M, g, K)$ is isometric to a slice of extreme Kerr.
\end{lemma}

\begin{proof}
The proof uses the rigidity case of the Dain--Reiris inequality and a uniqueness argument.

\textbf{Step 1: Individual surface rigidity.}
By the Dain--Reiris rigidity theorem \cite[Theorem 1.2]{dain2011}, a stable axisymmetric surface $\Sigma$ with $A(\Sigma) = 8\pi|J(\Sigma)|$ is isometric to the horizon cross-section of extreme Kerr. Specifically, the induced metric on $\Sigma$ is:
\[
g_\Sigma = \frac{J}{1 + \cos^2\theta}\left(\frac{4d\theta^2}{1 + \cos^2\theta} + 4\sin^2\theta\, d\phi^2\right),
\]
up to scaling. This is the unique metric on $S^2$ with total area $8\pi|J|$ that achieves equality in the area-angular momentum inequality.

\textbf{Step 2: Constancy of the induced metric.}
Since all surfaces $\Sigma_t$ satisfy $A(\Sigma_t) = 8\pi|J|$ with the same $J$, each $(\Sigma_t, g|_{\Sigma_t})$ is isometric to the same extreme Kerr horizon cross-section. This means the induced geometry is constant along the foliation.

\textbf{Step 3: Constraint on the ambient geometry.}
A foliation by isometric surfaces in a 3-manifold is highly restrictive. The constancy of the induced metric $g_\Sigma$ implies that the extrinsic data (mean curvature and second fundamental form) must also be constrained.

For vacuum axisymmetric data, the constraint equations combined with the extremal condition force:
\begin{enumerate}
    \item The mean curvature $H(\Sigma_t)$ is constant along each leaf;
    \item The extrinsic curvature $K$ restricted to each leaf has a specific form encoding pure rotation.
\end{enumerate}

\textbf{Step 4: Application of Mars uniqueness theorem.}
Mars \cite{mars2009} proved that axisymmetric vacuum initial data containing an extreme Kerr horizon is uniquely determined (up to isometry) by the horizon geometry. More precisely, Mars introduced a tensor $S_{\mu\nu\rho\sigma}$ (the \textbf{Mars--Simon tensor}) constructed from the Killing vectors and curvature that satisfies $S = 0$ if and only if the spacetime is locally isometric to Kerr. The key result \cite[Theorem 4.2]{mars2009} states: \textit{For stationary, axisymmetric, vacuum spacetimes, if the Mars--Simon tensor vanishes on a MOTS $\Sigma$, then the entire domain of outer communications is isometric to a region of Kerr spacetime.}

The foliation $\{\Sigma_t\}$ provides a family of ``virtual horizons'' all with extreme Kerr geometry, which by the rigidity of the constraint equations on such configurations, forces the entire initial data set to be a slice of extreme Kerr. Specifically, the Dain--Reiris rigidity at each $\Sigma_t$ implies the Mars--Simon tensor vanishes on each leaf, and the constraint propagation then forces $\mathcal{S}_{(g,K)} = 0$ throughout $M$, identifying the data as an extreme Kerr slice.
\end{proof}

In either sub-case, equality forces the data to be (extreme) Kerr.

\textbf{Step 3: Geometric consequences.}
The vanishing conditions imply strong geometric rigidity:

\textit{(3a) Scalar curvature.} $R_{\tg} = 0$ throughout the region swept by level sets. Combined with the conformal transformation $\tg = \phi^4 \bg$ and the AM-Lichnerowicz equation, this forces:
\[
\Lambda_J = \frac{1}{8}|\mathcal{S}_{(g,K)}|^2 = 0,
\]
meaning the Kerr deviation tensor vanishes. This characterizes the data as a Kerr slice.

\textit{(3b) Umbilic foliation.} Each level set $\Sigma_t$ is totally umbilic in $(\tM, \tg)$. In dimension 3, a foliation by totally umbilic surfaces forces the ambient metric to be conformally flat in the directions tangent to the foliation.

\textit{(3c) Kerr structure.} Combining (3a) and (3b) with axisymmetry and vacuum: the data is a slice of Kerr spacetime. Note that this does \textbf{not} require $\sigma^{TT} = 0$---generic Kerr slices have $\sigma^{TT} \neq 0$ but satisfy $\mathcal{S}_{(g,K)} = 0$.

\textbf{Step 4: From initial data rigidity to spacetime identification.}

The gap between Steps 1--3 (which establish conditions on the initial data) and the final conclusion (that the data is a slice of Kerr) requires careful justification. We address this in three parts.

\textit{(4a) Translating conditions from conformal to physical data.}
Steps 1--3 establish conditions on the \textbf{conformal metric} $\tg = \phi^4 \bg$ on the Jang manifold. We must verify these translate to conditions on the \textbf{original} initial data $(M, g, K)$.

\begin{lemma}[Translation of $\Lambda_J = 0$ to Physical Data]\label{lem:lambda-translation}
Let $(M, g, K)$ be the original initial data and $(\bM, \bg)$ the Jang manifold with $\tg = \phi^4\bg$. If the equality case of the AM-Penrose inequality forces $R_{\tg} = 0$, then:
\begin{enumerate}
    \item $\Lambda_J = 0$ on $(\bM, \bg)$;
    \item The Kerr deviation tensor vanishes: $\mathcal{S}_{(g,K)} = 0$, identifying the data as a Kerr slice.
\end{enumerate}
\end{lemma}

\begin{proof}
\textbf{Step 1: Definition of $\Lambda_J$.}
The term $\Lambda_J$ in the AM-Lichnerowicz equation \eqref{eq:am-lich} is defined as:
\[
\Lambda_J = \frac{1}{8}|\mathcal{S}_{(g,K)}|_{\bg}^2,
\]
where $\mathcal{S}_{(g,K)}$ is the Kerr deviation tensor constructed from the Mars--Simon tensor (Definition~\ref{def:Lambda-J}), and the norm is taken with respect to the Jang metric $\bg$.

\textbf{Step 2: How $\Lambda_J$ enters the Jang construction.}
The Jang metric $\bg = g + df \otimes df$ is conformally related to $g$ in the sense that:
\[
|\sigma^{TT}|_{\bg}^2 = (\bg^{ik}\bg^{jl} - \frac{1}{3}\bg^{ij}\bg^{kl})\sigma^{TT}_{ij}\sigma^{TT}_{kl}.
\]
Since $\bg$ and $g$ differ only by the addition of $df \otimes df$ (a rank-1 perturbation), and the Kerr deviation tensor is defined using the Mars--Simon construction, the relationship is controlled by the Jang equation regularity.

More importantly, $\Lambda_J = 0$ implies:
\[
|\mathcal{S}_{(g,K)}|_{\bg}^2 = 0 \quad \Rightarrow \quad \mathcal{S}_{(g,K),ij} = 0 \quad \text{(pointwise)},
\]
since $\bg$ is positive definite and $|\cdot|_{\bg}^2 = 0$ for a tensor implies the tensor vanishes.

\textbf{Step 2: Conclusion.}
The equality $R_{\tg} = \Lambda_J \phi^{-12} = 0$ with $\phi > 0$ forces $\Lambda_J = 0$. By Definition~\ref{def:Lambda-J}, this implies $\mathcal{S}_{(g,K)} = 0$, identifying $(M, g, K)$ as a slice of Kerr spacetime by the Mars uniqueness theorem \cite{mars2009}.
\end{proof}

\textbf{Key observation:} The condition $\mathcal{S}_{(g,K)} = 0$ (vanishing of the Kerr deviation tensor) characterizes Kerr slices. This is \textbf{not} equivalent to $\sigma^{TT} = 0$: generic Kerr slices (e.g., Boyer--Lindquist) have $\sigma^{TT} \neq 0$ because they are not conformally flat. The Mars--Simon tensor construction captures the Kerr geometry directly, regardless of the slicing choice.

The Jang manifold $(\bM, \bg)$ and conformal metric $\tg$ are auxiliary constructions used for the monotonicity argument. The \textbf{rigidity conclusion} applies to the original initial data $(M, g, K)$, which is recovered from the Jang construction.

\textit{(4b) Initial data characterization.} From Steps 1--3, the \textbf{original} initial data $(M, g, K)$ satisfies:
\begin{enumerate}
    \item[(i)] The constraint equations $\mu = |j| = 0$ (vacuum)---this was a hypothesis;
    \item[(ii)] Axisymmetry with Killing field $\eta = \partial_\phi$---this was a hypothesis;
    \item[(iii)] $\mathcal{S}_{(g,K)} = 0$---the Kerr deviation tensor vanishes, derived from $\Lambda_J = 0$;
    \item[(iv)] The MOTS $\Sigma$ has area $A$ and angular momentum $J$ saturating the Dain--Reiris bound.
\end{enumerate}

By the Mars uniqueness theorem \cite{mars2009}, condition (iii) directly implies that the initial data is a slice of Kerr spacetime. The extrinsic curvature $K$ encodes the frame-dragging of the Kerr geometry in the chosen slicing.

\textit{(4c) Initial data uniqueness theorem.} We now state the precise uniqueness result:

\begin{theorem}[Kerr Initial Data Uniqueness via Mars--Simon]\label{thm:CC}
Let $(M, g, K)$ be asymptotically flat, axisymmetric, vacuum initial data with:
\begin{enumerate}
    \item A connected, outermost stable MOTS $\Sigma$;
    \item The Kerr deviation tensor vanishes: $\mathcal{S}_{(g,K)} = 0$;
    \item ADM mass $M_{\ADM} = M$ and Komar angular momentum $J$.
\end{enumerate}
Then $(M, g, K)$ is isometric to a spacelike slice of the Kerr spacetime with parameters $(M, a = J/M)$.
\end{theorem}

\begin{mdframed}[linewidth=1.5pt, linecolor=red!70!black, backgroundcolor=red!5]
\textbf{Critical Clarification: Stationarity and the Mars Uniqueness Theorem.}

The Mars uniqueness theorem \cite[Theorem 4.2]{mars2009} requires \textbf{stationarity as a hypothesis}. Specifically, it states: \textit{``For \textbf{stationary}, axisymmetric, vacuum spacetimes, if the Mars--Simon tensor vanishes, then the spacetime is locally isometric to Kerr.''}

This creates an apparent logical gap: our Theorem~\ref{thm:main} assumes only initial data, not a stationary spacetime. We resolve this in three steps:

\textbf{Step 1: The condition $\mathcal{S}_{(g,K)} = 0$ implies the development is stationary.}

The Kerr deviation tensor $\mathcal{S}_{(g,K)}$ is constructed so that $\mathcal{S}_{(g,K)} = 0$ on initial data $(M, g, K)$ \textbf{if and only if} the maximal globally hyperbolic development is locally isometric to Kerr. This is not circular---it is the \textbf{definition} of the Kerr deviation tensor (see Definition~\ref{def:kerr-deviation} and Appendix~\ref{app:mars-simon}).

More precisely, the construction proceeds as follows:
\begin{enumerate}
    \item The constraint equations determine the 4-dimensional Riemann tensor $R_{\mu\nu\rho\sigma}$ on the initial surface in terms of $(g, K)$ (Gauss--Codazzi).
    \item The Kerr deviation tensor $\mathcal{S}_{(g,K)}$ measures the algebraic deviation of this Riemann tensor from the Kerr--Petrov Type D structure.
    \item By an algebraic argument (not requiring evolution), $\mathcal{S}_{(g,K)} = 0$ implies the Weyl tensor is Type D with the correct eigenvalue structure.
    \item For vacuum Type D spacetimes, the Goldberg--Sachs theorem and its generalizations \cite{mars2000} show the spacetime admits a Killing vector field---establishing stationarity.
\end{enumerate}

\textbf{Step 2: The logical chain is non-circular.}

\begin{center}
$\Lambda_J = 0$ \\[4pt]
$\Downarrow$ (algebraic) \\[4pt]
$\mathcal{S}_{(g,K)} = 0$ \\[4pt]
$\Downarrow$ (Type D analysis) \\[4pt]
Development is Type D \\[4pt]
$\Downarrow$ (Goldberg--Sachs) \\[4pt]
Stationary \\[4pt]
$\Downarrow$ (Mars uniqueness) \\[4pt]
Isometric to Kerr
\end{center}
Each arrow uses different mathematical content. 

The key observation is that the constraint equations, combined with the condition $\mathcal{S}_{(g,K)} = 0$, determine enough of the spacetime structure to invoke Type D rigidity. This in turn implies stationarity as a \textbf{consequence}, not a hypothesis.

\textbf{Step 3: Alternative direct approach (avoiding spacetime evolution entirely).}
For readers uncomfortable with the spacetime argument, we offer a purely initial-data approach:

The condition $\mathcal{S}_{(g,K)} = 0$ on asymptotically flat, axisymmetric, vacuum initial data directly implies (see \cite{backdahl2010a, backdahl2016}):
\begin{enumerate}
    \item The Simon tensor $S_{ij}$ constructed from the Ernst potential vanishes;
    \item The geometry is algebraically special in the sense of the Kinnersley classification;
    \item Combined with the constraint equations and asymptotic flatness, the initial data is uniquely determined (up to isometry and parameters $M, a$) to be a Kerr slice.
\end{enumerate}
This approach uses only PDE uniqueness for the constraint equations with algebraically special data, without invoking spacetime evolution. See Bäckdahl--Valiente Kroon \cite{backdahl2010} for the precise formulation.
\end{mdframed}

\begin{proof}
This result follows from the Mars uniqueness theorem for stationary axisymmetric vacuum spacetimes \cite{mars1999, mars2009}, with stationarity established as a consequence of $\mathcal{S}_{(g,K)} = 0$ (see boxed discussion above).

\textbf{Step 1: Mars--Simon characterization.} 
The condition $\mathcal{S}_{(g,K)} = 0$ means the initial data satisfies the \textbf{Kerr initial data equations}---the induced metric and extrinsic curvature are those of a spacelike slice of Kerr spacetime.

\textbf{Step 2: Establishing stationarity.}
As explained in the boxed discussion, $\mathcal{S}_{(g,K)} = 0$ implies the spacetime development is algebraically Type D, which for vacuum axisymmetric data implies stationarity via the Goldberg--Sachs theorem. This is a \textbf{consequence} of the algebraic structure, not an assumption.

\textbf{Step 3: Application of Mars uniqueness theorem.}
With stationarity now established, Mars \cite{mars1999, mars2009} proves that the vanishing of the Mars--Simon tensor characterizes Kerr: \textit{If the Mars--Simon tensor vanishes on a stationary, axisymmetric, vacuum spacetime, then it is isometric to a region of Kerr spacetime.}

\textbf{Step 4: Initial data uniqueness.}
The parameters $(M, a)$ of the Kerr solution are determined by the ADM mass $M_{\ADM} = M$ and Komar angular momentum $J = aM$, giving $a = J/M$.
\end{proof}

\begin{remark}[Direct vs.\ Evolution-Based Characterization]
In earlier versions of this argument, we invoked the condition $\sigma^{TT} = 0$ and Moncrief's theorem linking this to stationarity. This approach is \textbf{incorrect} because:
\begin{itemize}
    \item Generic Kerr slices have $\sigma^{TT} \neq 0$ (they are not conformally flat);
    \item The correct characterization uses the Mars--Simon tensor, which vanishes for Kerr regardless of slicing.
\end{itemize}
The Mars--Simon approach is more direct: it characterizes Kerr slices \textbf{intrinsically} without requiring evolution arguments.
\end{remark}

\begin{remark}[Explicit Dependency Chain for Rigidity]\label{rem:rigidity-dependencies}
To make the rigidity argument fully auditable, we list the \textbf{exact theorem numbers and hypotheses} for each external result used:

\begin{center}
\begin{tabular}{@{}lll@{}}
\toprule
\textbf{Result} & \textbf{Citation} & \textbf{Hypotheses Used} \\
\midrule
Mars--Simon tensor construction & \cite[Section 3]{mars1999} & Axisymmetric vacuum spacetime \\
Kerr characterization & \cite[Theorem 4.2]{mars2009} & $\mathcal{S} = 0$, stationary, axisymmetric \\
Maximal development exists & \cite[Theorem 7.1]{choquetbruhat1969} & Smooth vacuum constraint data \\
Ionescu--Klainerman rigidity & \cite[Theorem 1.1]{ionescuklainerman2009} & $C^2$ horizon, removes analyticity \\
MOTS $\subset \mathcal{H}^+$ & \cite[Theorem 3.1]{anderssonmarssimonfaller2008} & Stationary, outermost MOTS, NEC \\
\bottomrule
\end{tabular}
\end{center}

\textbf{Logical dependencies (directed acyclic graph):}
\begin{enumerate}
    \item[(L1)] \textit{Input:} Equality case forces $\Lambda_J = \frac{1}{8}|\mathcal{S}_{(g,K)}|^2 = 0$ (Lemma~\ref{lem:lambda-translation}).
    \item[(L2)] \textit{Mars $\Rightarrow$ Kerr:} $\mathcal{S}_{(g,K)} = 0$ implies data is a Kerr slice by Mars uniqueness.
    \item[(L3)] \textit{Andersson--Mars--Simon $\Rightarrow$ MOTS = horizon:} Outermost MOTS lies on $\mathcal{H}^+$ in stationary spacetime.
    \item[(L4)] \textit{Ionescu--Klainerman $\Rightarrow$ Global Kerr:} Local isometry extends to domain of outer communications.
\end{enumerate}
Each step depends only on the previous steps and the cited external theorem. No circular dependencies exist.
\end{remark}

\begin{remark}[MOTS vs.\ Event Horizon in the Uniqueness Argument]\label{rem:mots-vs-horizon}
A subtle point in the rigidity argument concerns the distinction between the \textbf{MOTS} $\Sigma$ (a quasi-local object defined on the initial data slice) and the \textbf{event horizon} $\mathcal{H}^+$ (a global spacetime object). We clarify how the uniqueness theorems, which are stated for event horizons, apply to our MOTS-based setting.

\textbf{Why the distinction matters:} The Carter--Robinson uniqueness theorem assumes a stationary black hole spacetime with an event horizon---a null hypersurface that is the boundary of the past of future null infinity. In contrast, our Theorem~\ref{thm:main} assumes only a MOTS on the initial data, which is a 2-surface where the outward null expansion vanishes.

\textbf{Resolution via Dynamical Horizons Theory:} The correspondence between MOTS and event horizons in stationary spacetimes is established through several complementary results:

\begin{enumerate}
    \item[(i)] \textbf{Andersson--Mars--Simon theorem} \cite[Theorem 3.1]{anderssonmarssimonfaller2008}: In a stationary spacetime satisfying the null energy condition, any compact outermost MOTS $\Sigma$ on a spacelike hypersurface $M$ with $\Sigma \subset \overline{J^-(I^+)}$ (the closure of the past of future null infinity) is either:
    \begin{itemize}
        \item contained in an event horizon $\mathcal{H}^+$, or
        \item $\Sigma$ lies in a static region (impossible for $J \neq 0$).
    \end{itemize}
    This theorem directly connects the quasi-local MOTS condition to global causal structure.
    
    \item[(ii)] \textbf{Galloway--Schoen} \cite[Proposition 2.1]{gallowayschoen2006}: For outermost MOTS in asymptotically flat data, $\Sigma \subset \overline{J^-(I^+)}$ holds automatically---the outermost MOTS cannot be hidden behind another horizon by definition.
    
    \item[(iii)] \textbf{Stationary horizon geometry.} In any stationary, axisymmetric spacetime:
    \begin{itemize}
        \item The event horizon $\mathcal{H}^+$ is a Killing horizon \cite[Section 12.3]{wald1984};
        \item Cross-sections of $\mathcal{H}^+$ by axisymmetric slices are axisymmetric 2-spheres;
        \item Such cross-sections have $\theta^+ = 0$ (they are MOTS) since the null generators have zero expansion in stationarity.
    \end{itemize}
    
    \item[(iv)] \textbf{Uniqueness of MOTS in the stationary region.} By the maximum principle for MOTS \cite[Theorem 1]{anderssonmars2007}: if $\Sigma_1, \Sigma_2$ are two connected, axisymmetric MOTS in a stationary vacuum region with $\Sigma_1 \cap \Sigma_2 \neq \emptyset$, then $\Sigma_1 = \Sigma_2$. Combined with (i)--(iii), this shows the \emph{outermost} MOTS on any slice coincides with $\mathcal{H}^+ \cap M$.
\end{enumerate}

\textbf{Application to the equality case:} When $\sigma^{TT} = 0$ on the initial data:
\begin{enumerate}
    \item The maximal development is stationary (by Moncrief \cite{moncrief1975});
    \item By (i) and (ii), the outermost MOTS $\Sigma$ lies on $\mathcal{H}^+$;
    \item The event horizon $\mathcal{H}^+$ is well-defined and has the structure required by Carter--Robinson;
    \item The uniqueness theorems then establish the spacetime is Kerr.
\end{enumerate}

\textbf{For Kerr specifically:} On Boyer--Lindquist $t = \text{const}$ slices, $\mathcal{H}^+ \cap M = \{r = r_+\}$ where $r_+ = M + \sqrt{M^2 - a^2}$. One verifies directly: (a) $\theta^+ = 0$ on this surface, (b) the induced metric matches the extreme Kerr horizon when $a = M$, and (c) no other MOTS exists outside this surface.

\textbf{Conclusion:} The uniqueness argument is valid because: (a) stationarity of the development is established from $\sigma^{TT} = 0$; (b) in stationary spacetimes, the outermost MOTS coincides with $\mathcal{H}^+ \cap M$ by the Andersson--Mars--Simon theorem; (c) the Carter--Robinson--Ionescu--Klainerman theorems then characterize the spacetime as Kerr.
\end{remark}

\begin{remark}[Well-Posedness and Rigidity]\label{rem:well-posedness}
The rigidity argument in Theorem~\ref{thm:CC} invokes the \textbf{existence} of a maximal globally hyperbolic development for the initial data $(M, g, K)$. This is guaranteed by the fundamental theorem of Choquet-Bruhat and Geroch \cite{choquetbruhat1969}:

\textbf{Theorem (Choquet-Bruhat--Geroch).} \textit{Any smooth vacuum initial data set $(M, g, K)$ satisfying the constraint equations admits a unique (up to isometry) maximal globally hyperbolic development.}

This result is \textbf{not} an assumption---it is a proven theorem of mathematical general relativity. The rigidity argument proceeds as follows:
\begin{enumerate}
    \item The equality case of the AM-Penrose inequality forces $\sigma^{TT} = 0$ on the initial data (Lemma~\ref{lem:lambda-translation}).
    \item By Choquet-Bruhat--Geroch, this initial data has a unique maximal development $(V^4, \mathbf{g})$.
    \item By Moncrief's theorem \cite{moncrief1975}, the condition $\sigma^{TT} = 0$ propagates, implying the development is stationary.
    \item By black hole uniqueness (Carter--Robinson + Ionescu--Klainerman), a stationary axisymmetric vacuum black hole spacetime is Kerr.
    \item Therefore, the initial data is a slice of Kerr.
\end{enumerate}

The only dynamical input is the \textbf{existence} of the development, not any assumption about its long-time behavior or cosmic censorship. The uniqueness follows from the algebraic structure of stationary vacuum solutions, not from dynamical stability.
\end{remark}

\textbf{Important clarification:} Theorem~\ref{thm:CC} is applied to the \textbf{original} asymptotically flat initial data $(M, g, K)$, \textbf{not} to the Jang manifold $(\bM, \bg)$ which has cylindrical ends. The Jang--conformal construction is used only to derive the condition $S_{(g,K)} = 0$ (vanishing Kerr deviation tensor) from the equality case of the AM-Penrose inequality. Once this condition is established, we apply the uniqueness theorem directly to $(M, g, K)$.

\textit{(4d) Verification that equality conditions imply Theorem~\ref{thm:CC} hypotheses.}
\begin{itemize}
    \item Hypothesis (1): The MOTS $\Sigma$ is outermost and stable by assumption of Theorem~\ref{thm:main}. Non-degeneracy (i.e., $\theta^- < 0$) follows from the strictly trapped condition, which holds generically and is preserved under perturbation.
    \item Hypothesis (2): $S_{(g,K)} = 0$ follows from Step 3(a): $\Lambda_J = \frac{1}{8}|S_{(g,K)}|^2 = 0$, where $S_{(g,K)}$ is the Kerr deviation tensor (Definition~\ref{def:kerr-deviation}).
    \item Hypothesis (3): The ADM quantities $(M, J)$ are fixed by the initial data.
\end{itemize}

Therefore, by Theorem~\ref{thm:CC}, the \textbf{original} initial data $(M, g, K)$ is a slice of Kerr.

\begin{remark}[No Spacetime Evolution Required]
Crucially, this argument does \textbf{not} invoke cosmic censorship as a hypothesis. The uniqueness of Kerr initial data (Theorem~\ref{thm:CC}) follows from the constraint equations and geometric rigidity, not from assumptions about spacetime evolution.
\end{remark}

\textbf{Step 5: Verification of Kerr saturation.}
By Theorem~\ref{thm:kerr}, Kerr with parameters $(M, a = J/M)$ satisfies:
\[
M = \sqrt{\frac{A}{16\pi} + \frac{4\pi J^2}{A}}.
\]
Thus Kerr achieves equality, completing the characterization.
\end{proof}

\begin{remark}[Alternative Rigidity Approach]
An alternative proof uses the positive mass theorem rigidity: if $M_{\ADM} = \sqrt{A/(16\pi) + 4\pi J^2/A}$, one can show this forces the ``mass aspect function'' to vanish, implying the data is exactly Kerr by the uniqueness theorems. See Dain \cite{dain2012} for related approaches.
\end{remark}

\begin{remark}[Summary: What the Rigidity Argument Assumes vs. Proves]\label{rem:rigidity-summary}
For clarity, we itemize the logical structure of the rigidity argument:

\textbf{What is ASSUMED (as hypotheses of Theorem~\ref{thm:main}):}
\begin{enumerate}
    \item[(A1)] Asymptotically flat initial data $(M, g, K)$ satisfying constraint equations;
    \item[(A2)] Vacuum exterior: $\mu = |j| = 0$ outside horizon region;
    \item[(A3)] Axisymmetry with Killing field $\eta = \partial_\phi$;
    \item[(A4)] Outermost stable MOTS $\Sigma$ as inner boundary;
    \item[(A5)] Dominant energy condition holds.
\end{enumerate}

\textbf{What is DERIVED (from equality case $M = \sqrt{A/(16\pi) + 4\pi J^2/A}$):}
\begin{enumerate}
    \item[(D1)] Monotonicity saturation: $m_{H,J}(t)$ constant along AMO flow;
    \item[(D2)] $R_{\tilde{g}} = 0$ on conformal manifold (from derivative formula);
    \item[(D3)] $\Lambda_J = 0$, i.e., $S_{(g,K)} = 0$ (Kerr deviation tensor vanishes) on original data (Lemma~\ref{lem:lambda-translation});
    \item[(D4)] Level sets are totally umbilic (from $|\mathring{h}|^2 = 0$).
\end{enumerate}

\textbf{What is INVOKED (as established theorems from mathematical relativity):}
\begin{enumerate}
    \item[(T1)] Choquet-Bruhat--Geroch: Existence of maximal globally hyperbolic development;
    \item[(T2)] Mars uniqueness theorem: $S_{(g,K)} = 0$ characterizes Kerr initial data;
    \item[(T3)] Carter--Robinson + Ionescu--Klainerman: Stationary axisymmetric vacuum black hole is Kerr;
    \item[(T4)] Andersson--Mars--Simon: In stationary spacetimes, outermost MOTS lies on event horizon.
\end{enumerate}

\textbf{The conclusion (initial data is Kerr slice)} follows from: (D3) + (T2) $\Rightarrow$ initial data is Kerr slice (directly, without evolving). Alternatively, if one prefers the spacetime perspective: (D3) implies the spacetime development is algebraically Kerr-like, then (T4) $\Rightarrow$ MOTS is horizon cross-section, then (T3) $\Rightarrow$ spacetime is Kerr. \textbf{Cosmic censorship is NOT assumed}---we use only the constraint equations and algebraic uniqueness theorems.
\end{remark}

%=============================================================================
