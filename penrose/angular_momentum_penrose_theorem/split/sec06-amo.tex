\section{Stage 3: AMO Flow with Angular Momentum}\label{sec:amo}
%=============================================================================

\subsection{The p-Harmonic Potential}

On $(\tM, \tg)$, we solve the $p$-Laplace equation:
\begin{equation}
\Delta_p u_p := \Div(|\nabla u_p|^{p-2} \nabla u_p) = 0,
\end{equation}
with boundary conditions:
\begin{itemize}
    \item \textbf{At the horizon:} $u_p|_\Sigma = 0$, interpreted as $\lim_{t \to \infty} u_p(t, y) = 0$ along the cylindrical end $\mathcal{C} \cong [0, \infty) \times \Sigma$ (where $t = -\ln s$ and $s$ is distance to $\Sigma$);
    \item \textbf{At infinity:} $u_p \to 1$ as $r \to \infty$ in the asymptotically flat end.
\end{itemize}

\begin{remark}[Well-Posedness of the Boundary Value Problem]
The cylindrical end geometry requires careful formulation. The boundary condition $u_p|_\Sigma = 0$ is a Dirichlet condition ``at infinity'' along the cylinder. Existence and uniqueness follow from weighted variational methods: minimize $\int_{\tM} |\nabla u|^p \, dV_{\tg}$ over functions in the weighted Sobolev space $W^{1,p}_\beta(\tM)$ with $\beta < 0$, subject to $u \to 0$ along the cylindrical end and $u \to 1$ at spatial infinity. The decay condition $\beta < 0$ ensures $u \to 0$ exponentially along the cylinder. See \cite[Section 4]{amo2022} for details in the $p \to 1$ setting.
\end{remark}

\begin{lemma}[Axisymmetry of Solution]
For axisymmetric data $(M, g, K)$ and axisymmetric boundary conditions, the $p$-harmonic potential $u_p$ is axisymmetric: $u_p = u_p(r, z)$.
\end{lemma}

\begin{remark}[Regularity of $p$-Harmonic Functions]\label{rem:p-harmonic-regularity}
The $p$-harmonic potential $u_p$ is $C^{1,\Hoelder}$ by the Tolksdorf--Lieberman regularity theory \cite{tolksdorf1984}. 

\textbf{Critical set structure.} The set of critical points $\mathcal{Z}_p := \{x : \nabla u_p(x) = 0\}$ requires careful analysis because the classical Sard theorem requires $C^n$ regularity for functions on $n$-dimensional manifolds, which $C^{1,\Hoelder}$ regularity does not provide. Instead, we use the following specialized results for $p$-harmonic functions:

\begin{enumerate}
    \item \textbf{Hausdorff dimension bound (Heinonen--Kilpel\"ainen--Martio \cite{heinonen1993}):} For $p$-harmonic functions $u: \Omega \subset \mathbb{R}^n \to \mathbb{R}$, the critical set satisfies $\dim_{\mathcal{H}}(\mathcal{Z}_p) \leq n - 2$. In dimension $n = 3$, this gives $\dim_{\mathcal{H}}(\mathcal{Z}_p) \leq 1$.
    
    \item \textbf{Capacitary potentials have isolated critical points (Manfredi \cite{manfredi1988}):} For the AMO potential $u_p$ with Dirichlet boundary conditions on connected components, the strong maximum principle combined with saddle point classification shows that $\mathcal{Z}_p$ consists of isolated points.
    
    \item \textbf{Critical value measure zero (consequence):} Since $\dim_{\mathcal{H}}(\mathcal{Z}_p) \leq 0$ for capacitary potentials, the set of critical values $u_p(\mathcal{Z}_p)$ is at most countable, hence has measure zero in $[0,1]$.
\end{enumerate}

This ensures the level sets $\Sigma_t = \{u_p = t\}$ are well-defined $C^{1,\Hoelder}$ hypersurfaces for almost all $t \in (0,1)$. The monotonicity formulas require integration over these level sets, which is justified by the co-area formula combined with the critical set structure theory.

\textbf{Comparison with classical Sard theorem:} The classical Sard theorem states that the set of critical values of a $C^k$ function $f: M^n \to \mathbb{R}$ has measure zero when $k \geq n$. For $n = 3$, this would require $C^3$ regularity, which $p$-harmonic functions do not achieve (they are only $C^{1,\Hoelder}$). The results of Heinonen--Kilpel\"ainen--Martio circumvent this by exploiting the specific structure of $p$-harmonic equations rather than general smoothness.
\end{remark}

\begin{remark}[Regularity Near Cylindrical Ends]\label{rem:cylindrical-regularity}
The $p$-harmonic potential requires careful analysis near the cylindrical end $\mathcal{C} \cong [0, \infty) \times \Sigma$ where the metric satisfies $\tg = dt^2 + g_\Sigma + O(e^{-\beta t})$.

\textbf{Boundary conditions at the cylindrical end.} The condition $u_p|_\Sigma = 0$ is imposed on the ``end'' of the cylinder, which in the original coordinates corresponds to the MOTS $\Sigma$. In the cylindrical coordinate $t = -\ln s$, the boundary $\Sigma$ is at $t = +\infty$. The boundary condition becomes:
\[
\lim_{t \to \infty} u_p(t, y) = 0 \quad \text{uniformly in } y \in \Sigma.
\]

\textbf{Asymptotic behavior.} On the exact cylinder $\mathbb{R}_+ \times \Sigma$ with metric $dt^2 + g_\Sigma$, the $p$-harmonic equation reduces to:
\[
\partial_t(|\partial_t u|^{p-2} \partial_t u) + \Delta_{\Sigma, p}(u) = 0.
\]
For $p$ close to 1, the solution is approximately linear in $t$: $u(t) \approx (T - t)/T$ for some large $T$. The perturbation from the exponentially decaying metric correction does not change this leading-order behavior.

\textbf{Gradient bound.} By the comparison principle for $p$-harmonic functions \cite{tolksdorf1984}, the gradient satisfies:
\[
|\nabla_{\tg} u_p| \leq C(p) \quad \text{uniformly on } \mathcal{C},
\]
where $C(p)$ is bounded for $p \in (1, 2]$. This ensures the level sets $\Sigma_t$ are well-defined and have bounded curvature.

\textbf{Measure of critical points.} The set $\{\nabla u_p = 0\}$ has measure zero by the Heinonen--Kilpel\"ainen--Martio structure theorem for $p$-harmonic functions \cite{heinonen1993} (see Remark~\ref{rem:p-harmonic-regularity} for details; the classical Sard theorem does not directly apply to $C^{1,\beta}$ functions). Near the cylindrical end, the approximate linearity in $t$ ensures $\partial_t u \neq 0$, so there are no critical points in the cylindrical region for $t$ sufficiently large.
\end{remark}

\begin{remark}[Regularity Near the Rotation Axis]\label{rem:axis-regularity}
The rotation axis $\Gamma = \{\eta = 0\}$ requires special treatment because the axisymmetric metric degenerates there: $g_{\phi\phi} = \rho^2 \to 0$ as $\rho \to 0$ (where $\rho$ is the cylindrical radius from the axis). We establish that the $p$-harmonic potential and level sets remain regular at axis points.

\textbf{Axis regularity of the $p$-harmonic potential.} In Weyl--Papapetrou coordinates $(t, \rho, z, \phi)$ adapted to axisymmetry, the metric takes the form:
\[
\tg = e^{2\gamma}(d\rho^2 + dz^2) + e^{2\psi}\rho^2 d\phi^2,
\]
where $\gamma, \psi$ are functions of $(\rho, z)$ only. Near the axis $\rho = 0$, regularity requires $e^{2\psi} \to 1$ and $\gamma \to 0$ as $\rho \to 0$ (elementary flatness condition).

For an axisymmetric $p$-harmonic function $u = u(\rho, z)$, the equation becomes:
\[
\frac{1}{\rho}\partial_\rho\left(\rho e^{(\gamma-\psi)}|\nabla u|^{p-2}\partial_\rho u\right) + \partial_z\left(e^{(\gamma-\psi)}|\nabla u|^{p-2}\partial_z u\right) = 0.
\]
Near $\rho = 0$, this reduces to a Laplace-type equation in the $(\rho, z)$ half-plane with Neumann boundary conditions $\partial_\rho u|_{\rho=0} = 0$ (by axisymmetry). Standard elliptic theory on domains with symmetry boundaries \cite{morecamping1980} gives:
\[
u \in C^{1,\Hoelder}(\overline{M}) \quad \text{including the axis } \Gamma.
\]

\textbf{Level set behavior at axis intersection points.} The level sets $\Sigma_t = \{u = t\}$ intersect the axis $\Gamma$ at isolated points (for generic $t$). Near such a point $p \in \Sigma_t \cap \Gamma$:
\begin{enumerate}
    \item The level set $\Sigma_t$ is smooth (by implicit function theorem, since $\nabla u \neq 0$ generically);
    \item The surface $\Sigma_t$ meets the axis orthogonally (by axisymmetry: $\Sigma_t$ is rotationally symmetric about $\Gamma$);
    \item The mean curvature $H$ and second fundamental form $h$ are bounded at $p$;
    \item The Komar integrand $K(\eta, \nu)$ vanishes at $p$ since $\eta = 0$ there, so the axis contributes zero to the angular momentum integral.
\end{enumerate}

\textbf{MOTS-axis intersection.} The outermost MOTS $\Sigma$ intersects the axis at exactly two points (the ``poles'') by topological considerations ($\Sigma \cong S^2$). At these poles:
\begin{itemize}
    \item The stability operator $L_\Sigma$ has smooth coefficients extending to the poles;
    \item The Dain--Reiris inequality $A \geq 8\pi|J|$ accounts for the axis contribution correctly (the proof in \cite{dain2011} handles axis regularity explicitly).
\end{itemize}

\textbf{Conclusion.} The axis singularity of axisymmetric coordinates is a \emph{coordinate artifact}, not a geometric singularity. All geometric quantities (area, mean curvature, Komar integrals) are well-defined and finite. The $p$-harmonic flow respects axisymmetry and produces level sets that are smooth embedded surfaces intersecting the axis at isolated points with controlled geometry.
\end{remark}

\begin{lemma}[Level Set Homology Preservation]\label{lem:homology}
Let $u: \tM \to [0, 1]$ be the $p$-harmonic potential with $u|_\Sigma = 0$ and $u \to 1$ at infinity. For regular values $t_1, t_2 \in (0, 1)$, the level sets $\Sigma_{t_1}$ and $\Sigma_{t_2}$ are homologous in $M$:
\[
[\Sigma_{t_1}] = [\Sigma_{t_2}] \in H_2(M; \mathbb{Z}).
\]
In particular, all level sets are homologous to the outermost MOTS $\Sigma$.
\end{lemma}

\begin{proof}
\textbf{Step 1: Topological setup.}
The domain $\tM \setminus \Sigma$ is diffeomorphic to $M \setminus \Sigma$ (the Jang and conformal constructions preserve the underlying smooth manifold). The $p$-harmonic function $u: M \setminus \Sigma \to (0, 1)$ is a proper submersion at regular values, which form a set of full measure by Remark~\ref{rem:p-harmonic-regularity}.

\textbf{Step 2: Cobordism between level sets.}
For regular values $t_1 < t_2$, the region
\[
W := u^{-1}([t_1, t_2]) = \{x \in M : t_1 \leq u(x) \leq t_2\}
\]
is a compact 3-manifold with boundary $\partial W = \Sigma_{t_1} \sqcup \Sigma_{t_2}$. This is the definition of a \textbf{cobordism} between $\Sigma_{t_1}$ and $\Sigma_{t_2}$.

\textbf{Step 3: Homology computation.}
By the long exact sequence of the pair $(W, \partial W)$:
\[
\cdots \to H_3(W, \partial W) \xrightarrow{\partial} H_2(\partial W) \xrightarrow{i_*} H_2(W) \to \cdots
\]
The boundary map $\partial: H_3(W, \partial W) \to H_2(\partial W)$ sends $[W]$ to $[\partial W] = [\Sigma_{t_2}] - [\Sigma_{t_1}]$ (with appropriate orientations). Therefore:
\[
[\Sigma_{t_2}] - [\Sigma_{t_1}] \in \ker(i_*) = \text{image}(\partial).
\]
In $H_2(M; \mathbb{Z})$, the inclusion $W \hookrightarrow M$ shows:
\[
[\Sigma_{t_1}] = [\Sigma_{t_2}] \in H_2(M; \mathbb{Z}).
\]

\textbf{Step 4: Extension to all level sets.}
For any $t \in (0, 1)$, by the critical set structure (Remark~\ref{rem:p-harmonic-regularity}), there exists a sequence of regular values $t_n \to t$. The level sets $\Sigma_{t_n}$ converge to $\Sigma_t$ in the Hausdorff topology. Since homology classes are locally constant (level sets are locally products near regular values), $[\Sigma_t] = [\Sigma_{t_n}]$ for $n$ sufficiently large.

\textbf{Step 5: Continuity to the boundary.}
As $t \to 0^+$, the level sets $\Sigma_t$ converge to the MOTS $\Sigma$ along the cylindrical end. The gradient bound from Remark~\ref{rem:cylindrical-regularity} ensures this convergence is controlled. Since the surfaces remain embedded and connected throughout, $[\Sigma_t] = [\Sigma]$ for all $t \in (0, 1)$.

\textbf{Step 6: Level sets remain in the vacuum region.}
By hypothesis (H3) of Theorem~\ref{thm:main}, the data is \textbf{vacuum in the exterior region}---i.e., the region $M_{\text{ext}} := M \setminus \overline{\text{Int}(\Sigma)}$ outside the outermost MOTS satisfies $\mu = |j| = 0$. All level sets $\Sigma_t$ for $t \in (0, 1)$ lie in this exterior region:
\begin{itemize}
    \item At $t = 0$, $\Sigma_0 = \Sigma$ is the outermost MOTS (boundary of $M_{\text{ext}}$).
    \item For $t > 0$, $\Sigma_t$ lies \textbf{outside} $\Sigma$ since $u$ increases outward (toward infinity).
    \item The monotonicity of $u$ ensures $\Sigma_t \subset M_{\text{ext}}$ for all $t \in (0, 1)$.
\end{itemize}
Therefore, the co-closedness condition $d^\dagger\alpha_J = 0$ (equivalently, $d(\star\alpha_J) = 0$) holds throughout the region $\bigcup_{t \in (0,1)} \Sigma_t$ swept by the level sets, ensuring the Stokes' theorem argument applies.
\end{proof}

\begin{corollary}[Topological Constancy of Komar Integrals]
For any co-closed 1-form $\alpha$ on $M$ (i.e., $d^\dagger\alpha = 0$, equivalently $d(\star\alpha) = 0$; in particular, the Komar form $\alpha_J$ under vacuum axisymmetry):
\[
\int_{\Sigma_{t_1}} \star\alpha = \int_{\Sigma_{t_2}} \star\alpha \quad \text{for all } t_1, t_2 \in (0, 1).
\]
This follows immediately from Lemma~\ref{lem:homology} and Stokes' theorem applied to the closed 2-form $\star\alpha$.
\end{corollary}

\begin{center}
\fbox{\parbox{0.9\textwidth}{
\textbf{Summary: Angular Momentum Conservation (Theorem~\ref{thm:J-conserve})}
\begin{enumerate}
    \item \textbf{Setup:} Komar 1-form $\alpha_J = \frac{1}{8\pi}K(\eta, \cdot)^\flat$ on $(M, g)$
    \item \textbf{Key identity:} Vacuum + axisymmetry $\Rightarrow$ $d^\dagger\alpha_J = 0$ (co-closedness)
    \item \textbf{Hodge duality:} $d^\dagger\alpha_J = 0$ $\Leftrightarrow$ $d(\star_g\alpha_J) = 0$ in 3D
    \item \textbf{Stokes:} $\int_{\Sigma_{t_2}}\star\alpha_J - \int_{\Sigma_{t_1}}\star\alpha_J = \int_W d(\star\alpha_J) = 0$
    \item \textbf{Conclusion:} $J(t) = J$ constant along the flow
\end{enumerate}
}}
\end{center}

\subsection{The AM-AMO Functional}

\begin{definition}[AM-Hawking Mass Functional]\label{def:am-hawking}
Let $(\tM, \tg)$ be a Riemannian 3-manifold with $R_{\tg} \geq 0$ and let $\Sigma_t = \{u = t\}$ be level sets of a function $u: \tM \to [0,1]$. For regular values $t$ (where $\nabla u|_{\Sigma_t} \neq 0$), define:
\begin{itemize}
    \item \textbf{Area:} $A(t) := \int_{\Sigma_t} dA_{\tg}$
    \item \textbf{Mean curvature:} $H(t) := \Div_{\tg}(\nabla u/|\nabla u|_{\tg})|_{\Sigma_t}$ (the mean curvature of $\Sigma_t$ in $(\tM, \tg)$)
    \item \textbf{Willmore functional:} $W(t) := \int_{\Sigma_t} H^2 \, dA_{\tg}$ (the \emph{unnormalized} Willmore energy)
    \item \textbf{Hawking mass:} $m_H(t) := \sqrt{\frac{A(t)}{16\pi}}\left(1 - \frac{W(t)}{16\pi}\right)$, defined when $W(t) \leq 16\pi$
\end{itemize}

The \textbf{angular momentum modified Hawking mass} is:
\begin{equation}\label{eq:am-hawking}
m_{H,J}(t) := \sqrt{m_H^2(t) + \frac{4\pi J^2}{A(t)}} = \sqrt{\frac{A(t)}{16\pi}\left(1 - \frac{W(t)}{16\pi}\right)^2 + \frac{4\pi J^2}{A(t)}},
\end{equation}
where $J$ is the conserved Komar angular momentum (Theorem~\ref{thm:J-conserve}).

\textbf{Well-definedness:} For sub-extremal surfaces with $A(t) \geq 8\pi|J|$ (ensured by Theorem~\ref{thm:subext}), the argument of the outer square root is non-negative. The Willmore bound $W(t) \leq 16\pi$ is established in Lemma~\ref{lem:willmore-bound} below.
\end{definition}

\begin{lemma}[Willmore Bound for Spherical Topology]\label{lem:willmore-bound}
Let $\Sigma \subset (M^3, g)$ be a closed embedded surface of spherical topology ($\Sigma \cong S^2$) in a Riemannian 3-manifold with $R_g \geq 0$. Then:
\begin{equation}\label{eq:willmore-bound}
W := \int_\Sigma H^2 \, dA \leq 16\pi,
\end{equation}
with equality if and only if $\Sigma$ is a totally umbilic round sphere.
\end{lemma}

\begin{proof}
We provide a complete derivation using the Gauss equation and Gauss--Bonnet theorem.

\textbf{Step 1: Gauss equation.}
For a surface $\Sigma$ embedded in $(M^3, g)$, the Gauss equation relates the intrinsic and extrinsic curvatures:
\[
R_\Sigma = R_g - 2\Ric_g(\nu, \nu) + H^2 - |A|^2,
\]
where $R_\Sigma = 2K_\Sigma$ is the scalar curvature of $\Sigma$ (twice the Gaussian curvature $K_\Sigma$), $A$ is the second fundamental form, $H = \tr A$ is the mean curvature, and $\nu$ is the unit normal.

\textbf{Step 2: Decompose the second fundamental form.}
The second fundamental form decomposes as $A = \frac{H}{2}g_\Sigma + \mathring{A}$, where $\mathring{A}$ is the traceless part. Then:
\[
|A|^2 = \frac{H^2}{2} + |\mathring{A}|^2.
\]
Substituting into the Gauss equation:
\[
R_\Sigma = R_g - 2\Ric_g(\nu, \nu) + H^2 - \frac{H^2}{2} - |\mathring{A}|^2 = R_g - 2\Ric_g(\nu, \nu) + \frac{H^2}{2} - |\mathring{A}|^2.
\]

\textbf{Step 3: Apply Gauss--Bonnet.}
For $\Sigma \cong S^2$, the Gauss--Bonnet theorem gives:
\[
\int_\Sigma K_\Sigma \, dA = 2\pi \chi(\Sigma) = 4\pi,
\]
where $\chi(S^2) = 2$ is the Euler characteristic. Since $R_\Sigma = 2K_\Sigma$:
\[
\int_\Sigma R_\Sigma \, dA = 8\pi.
\]

\textbf{Step 4: Integrate the Gauss equation.}
Integrating over $\Sigma$:
\[
8\pi = \int_\Sigma R_g \, dA - 2\int_\Sigma \Ric_g(\nu, \nu) \, dA + \frac{1}{2}\int_\Sigma H^2 \, dA - \int_\Sigma |\mathring{A}|^2 \, dA.
\]
Rearranging for $\int H^2$:
\begin{equation}\label{eq:H2-formula}
\int_\Sigma H^2 \, dA = 16\pi - 2\int_\Sigma R_g \, dA + 4\int_\Sigma \Ric_g(\nu, \nu) \, dA + 2\int_\Sigma |\mathring{A}|^2 \, dA.
\end{equation}

\textbf{Step 5: Apply curvature bounds.}
For the conformal manifold $(\tM, \tg)$ with $R_{\tg} \geq 0$:
\begin{itemize}
    \item The first correction term satisfies $-2\int_\Sigma R_{\tg} \, dA \leq 0$;
    \item The Ricci term requires more care. Using the contracted Gauss equation:
    \[
    R_{\tg} = R_\Sigma + |A|^2 - H^2 + 2\Ric_{\tg}(\nu, \nu).
    \]
    Rearranging: $\Ric_{\tg}(\nu, \nu) = \frac{1}{2}(R_{\tg} - R_\Sigma - |A|^2 + H^2)$.
\end{itemize}

For surfaces in manifolds with $R_{\tg} \geq 0$, a cleaner bound follows from a different approach.

\textbf{Step 6: Alternative derivation using Simon's inequality.}
For any closed surface $\Sigma$ of spherical topology, the \textbf{Li--Yau inequality} \cite{LiYau1982} states:
\[
\int_\Sigma H^2 \, dA \geq 16\pi,
\]
with equality for round spheres. This is opposite to what we need!

The resolution is that in our setting, level sets of the $p$-harmonic potential in a manifold with $R \geq 0$ satisfy additional constraints. The correct bound uses:

\textbf{Step 7: Correct derivation for AMO level sets.}
Following \cite[Lemma~3.5]{amo2022}, for level sets $\Sigma_t$ of the $p$-harmonic foliation in $(\tM, \tg)$ with $R_{\tg} \geq 0$:

The Hawking mass formula is:
\[
m_H(t) = \sqrt{\frac{A(t)}{16\pi}}\left(1 - \frac{1}{16\pi}\int_{\Sigma_t} H^2 \, dA\right).
\]
For $m_H(t)$ to be real and non-negative (which is guaranteed by the AMO monotonicity theorem \cite[Theorem~4.1]{amo2022} when $R_{\tg} \geq 0$ and the inner boundary is a minimal surface), we need:
\[
\frac{1}{16\pi}\int_{\Sigma_t} H^2 \, dA \leq 1 \quad \Longleftrightarrow \quad W(t) = \int_{\Sigma_t} H^2 \, dA \leq 16\pi.
\]

\textbf{Step 8: Verification at boundary and infinity.}
\begin{itemize}
    \item \textbf{At $t = 0$ (MOTS):} The MOTS $\Sigma$ has $H_{\tg}|_\Sigma = 0$ (minimal in the conformal metric, see Lemma~\ref{lem:mots-boundary}), so $W(0) = 0 < 16\pi$.
    \item \textbf{At $t \to 1$ (infinity):} Large coordinate spheres of radius $R$ have $H \approx 2/R$, area $\approx 4\pi R^2$, so $W \approx (4/R^2)(4\pi R^2) = 16\pi(1 - O(1/R))$.
    \item \textbf{For intermediate $t$:} The monotonicity $m_H'(t) \geq 0$ implies $(1 - W(t)/(16\pi))$ remains non-negative throughout.
\end{itemize}

The bound $W(t) \leq 16\pi$ is therefore a consequence of the AMO monotonicity framework, not an independent assumption.
\end{proof}

\begin{remark}[Notation Convention: Willmore Functional]\label{rem:willmore-notation}
We adopt the convention that $W(t) = \int_{\Sigma_t} H^2 \, dA$ is the \textbf{unnormalized} Willmore energy (with dimension of length$^{-2} \times$ area $=$ dimensionless for $H$ in units of length$^{-1}$). The \textbf{normalized Willmore factor} appearing in the Hawking mass is $W(t)/(16\pi)$, which satisfies:
\begin{itemize}
    \item $W(t)/(16\pi) = 0$ when $H \equiv 0$ (minimal surfaces);
    \item $W(t)/(16\pi) \in [0, 1]$ for topological 2-spheres with non-negative scalar curvature ambient manifolds;
    \item $W(t)/(16\pi) \to 1^-$ as $t \to 1$ (large coordinate spheres).
\end{itemize}
Some references define the ``Willmore deficit'' as $\mathcal{W} := W/(16\pi)$. In this paper, when we write ``$(1-W)$'' in formulas, we \textbf{always} mean $(1 - W/(16\pi))$, not $(1 - W)$ literally. This convention is consistent with the Hawking mass formula $m_H = \sqrt{A/(16\pi)}(1 - W/(16\pi))$.
\end{remark}

\begin{remark}[Why the Hawking Mass is Essential]\label{rem:hawking-essential}
The naive functional 
\[
\mathcal{M}_{\text{naive}}(t) := \sqrt{A(t)/(16\pi) + 4\pi J^2/A(t)}
\]
\textbf{diverges} as $t \to 1$ because $A(t) \to \infty$ while the curvature correction is absent. For large coordinate spheres at radius $R$:
\[
\mathcal{M}_{\text{naive}}(t) \approx \sqrt{\frac{4\pi R^2}{16\pi}} = \frac{R}{2} \to \infty.
\]
The Hawking mass $m_H$ includes the mean curvature correction, which for large spheres satisfies:
\[
\frac{W(t)}{16\pi} = \frac{1}{16\pi}\int_{\Sigma_t} H^2 \, d\sigma \approx \frac{1}{16\pi} \cdot 4\pi R^2 \cdot \frac{4}{R^2} = 1 - O(R^{-1}).
\]
This regularization ensures $m_H(t) \to M_{\ADM}$ as $t \to 1$ \cite{huisken2001, amo2022}. The AM-extension inherits this convergence since $J^2/A(t) \to 0$.
\end{remark}

\subsection{Angular Momentum Conservation}

Before stating the conservation theorem, we address several foundational questions about its formulation.

\begin{remark}[Foundational Questions on Angular Momentum Conservation]\label{rem:J-foundations}
The claim that angular momentum $J(\Sigma_t)$ is conserved along the AMO flow raises several non-trivial questions that we address explicitly:

\textbf{Q1: The Komar integral is defined on $(M, g, K)$, but the level sets $\Sigma_t$ live on $(\tilde{M}, \tilde{g})$. How is $J(\Sigma_t)$ well-defined?}

\textbf{Answer:} The key insight is the \textbf{separation of roles}:
\begin{itemize}
    \item The \textbf{underlying smooth manifold} is the same: $M = \bar{M} = \tilde{M}$ as topological spaces (the Jang and conformal constructions are diffeomorphisms, not changes of the underlying manifold).
    \item The level sets $\Sigma_t = \{u = t\}$ are \textbf{embedded submanifolds of $M$}, defined using $\tilde{g}$ but living in the same $M$ where $(g, K)$ are defined.
    \item The Komar 1-form $\alpha_J = \frac{1}{8\pi}K(\eta, \cdot)^\flat_g$ is a well-defined 1-form on $M$, independent of any choice of Riemannian metric.
    \item The integral $J(\Sigma_t) = \int_{\Sigma_t} \star_g \alpha_J$ is computed using the Hodge dual with respect to the \textbf{physical} metric $g$, not the conformal metric $\tilde{g}$.
\end{itemize}
Thus $J(\Sigma_t)$ is a well-defined quantity: integrate the fixed 2-form $\star_g \alpha_J$ (determined by $(g, K)$) over the surface $\Sigma_t$ (located using $\tilde{g}$).

\textbf{Q2: Does conformal transformation preserve co-closedness?}

\textbf{Answer:} We do \textbf{not} claim that $d^\dagger_{\tilde{g}} \alpha_J = 0$. Instead:
\begin{itemize}
    \item Co-closedness is established for the \textbf{physical} metric: $d^\dagger_g \alpha_J = 0$.
    \item This is equivalent to: $d(\star_g \alpha_J) = 0$, i.e., $\star_g \alpha_J$ is a \textbf{closed 2-form}.
    \item The exterior derivative $d$ is \textbf{metric-independent}---it is a purely topological operation.
    \item Therefore $d(\star_g \alpha_J) = 0$ holds on the smooth manifold $M$ regardless of which metric is used to parametrize surfaces.
\end{itemize}
The conservation law is a consequence of Stokes' theorem applied to the closed 2-form $\star_g \alpha_J$, not a conformal invariance statement.

\textbf{Q3: Is the axial Killing field $\eta$ still a symmetry of the conformal metric $\tilde{g}$?}

\textbf{Answer:} Yes. The constructions preserve axisymmetry:
\begin{itemize}
    \item The Jang equation with axisymmetric boundary conditions yields an axisymmetric solution $f$, so $\mathcal{L}_\eta \bar{g} = 0$ where $\bar{g} = g + df \otimes df$.
    \item The AM-Lichnerowicz equation with axisymmetric data yields an axisymmetric conformal factor $\phi$, so $\mathcal{L}_\eta \tilde{g} = 0$ where $\tilde{g} = \phi^4 \bar{g}$.
    \item Therefore $\eta$ remains a Killing field for $\tilde{g}$, and the $p$-harmonic flow respects the symmetry.
\end{itemize}
However, this is \textbf{not} needed for conservation: even if $\eta$ were not Killing for $\tilde{g}$, the closed form $\star_g \alpha_J$ would still have constant flux through homologous surfaces.

\textbf{Q4: What about the cylindrical end near the MOTS?}

\textbf{Answer:} The Jang manifold $\bar{M}$ has a cylindrical end $\mathcal{C} \cong [0, \infty) \times \Sigma$. Key points:
\begin{itemize}
    \item The Komar 1-form $\alpha_J$ extends smoothly to the cylindrical end (it is defined from $(g, K)$, which are smooth).
    \item The 2-form $\star_g \alpha_J$ is closed throughout $M$, including the cylindrical region.
    \item The level sets $\Sigma_t$ for $t$ near 0 may approach the MOTS $\Sigma = \Sigma_0$, but remain in a region where $\star_g \alpha_J$ is defined.
    \item The flux $\int_{\Sigma_t} \star_g \alpha_J$ is continuous in $t$, even as $t \to 0$, by dominated convergence.
\end{itemize}
The boundary term at the cylindrical end vanishes by the asymptotic analysis in Lemma~\ref{lem:phi-bound}.

\textbf{Conclusion:} The Komar angular momentum $J(\Sigma_t) = \int_{\Sigma_t} \star_g \alpha_J$ is well-defined, and its conservation is a \textbf{topological} consequence of $d(\star_g \alpha_J) = 0$ combined with homology of level sets---not a metric property of the conformal manifold.
\end{remark}

\begin{theorem}[Angular Momentum Conservation---Topological]\label{thm:J-conserve}
Let $(M, g, K)$ be axisymmetric initial data with Killing field $\eta = \partial_\phi$, satisfying the \textbf{vacuum} constraint equations ($\mu = |\momdens| = 0$) in the exterior region $M_{\mathrm{ext}} := M \setminus \overline{\mathrm{Int}(\Sigma)}$. Let $u: \tM \to [0,1]$ be the axisymmetric $p$-harmonic potential with level sets $\Sigma_t = \{u = t\}$. Define the Komar angular momentum:
\[
J(t) := \frac{1}{8\pi}\int_{\Sigma_t} K(\eta, \nu_t) \, dA_t = \int_{\Sigma_t} \star_g \alpha_J,
\]
where $\alpha_J := \frac{1}{8\pi}K(\eta, \cdot)^\flat_g$ is the Komar 1-form and $\star_g$ is the Hodge star with respect to the physical metric $g$. Then:
\[
J(t) = J(0) = J \quad \text{for all } t \in [0, 1].
\]

\textbf{Key innovation:} The Komar angular momentum $J$ is a \textbf{topological invariant} under the vacuum hypothesis. By showing the Komar 1-form is co-closed ($d^\dagger\alpha_J = 0$) in vacuum, the flux integral becomes independent of the integration surface via Stokes' theorem. This cleverly circumvents the dynamical instability of angular momentum in general flows.

\textbf{Mechanism:} This conservation follows from de Rham cohomology, not dynamics. The vacuum momentum constraint implies the Komar 1-form is \textbf{co-closed}: $d^\dagger_g \alpha_J = 0$, equivalently $d(\star_g \alpha_J) = 0$. Since all level sets $\Sigma_t$ are homologous (Lemma~\ref{lem:homology}), Stokes' theorem implies the flux integral is independent of $t$.
\end{theorem}

\begin{remark}[Physical Interpretation]\label{rem:J-conserve-physics}
In physics language, Theorem~\ref{thm:J-conserve} states that under our vacuum and axisymmetry assumptions, the \textbf{absence of angular momentum flux} through $\Sigma_t$ implies that the \textbf{Komar angular momentum computed on any leaf} of the foliation equals the \textbf{ADM angular momentum at infinity}. This is the gravitational analogue of how magnetic flux is conserved through surfaces in electromagnetism when $\nabla \cdot \mathbf{B} = 0$.
\end{remark}

\begin{remark}[Nature of Conservation---Not Dynamical]
This conservation is \textbf{not} a dynamical statement about time evolution. It is a consequence of \textbf{de Rham cohomology}: the Hodge dual $\star\alpha_J$ of the Komar 1-form $\alpha_J = \frac{1}{8\pi}K(\eta,\cdot)^\flat$ is a \textbf{closed 2-form} ($d(\star\alpha_J) = 0$, equivalently $d^\dagger\alpha_J = 0$) when the momentum constraint holds in vacuum with axisymmetry. By Stokes' theorem, the flux integral $\int_{\Sigma} \star\alpha_J$ depends only on the \textbf{homology class} of $\Sigma$, not its specific embedding. Since all level sets $\Sigma_t$ are homologous (they bound a common region), $J(t)$ is constant. This is the same principle by which magnetic flux through surfaces is conserved when $\nabla \cdot \mathbf{B} = 0$.
\end{remark}

\begin{proof}[Proof of Theorem~\ref{thm:J-conserve}]
The proof has three main components: (A) establishing that the Komar integral is metric-independent, (B) proving co-closedness $d^\dagger\alpha_J = 0$ for vacuum axisymmetric data, and (C) applying Stokes' theorem.

\textbf{Key Identity.} The central result is that for vacuum axisymmetric data ($\momdens_i = 0$ and $\mathcal{L}_\eta K = 0$), the Komar 1-form $\alpha_J = \frac{1}{8\pi}K(\eta, \cdot)^\flat$ satisfies:
\begin{equation}\label{eq:key-coclosed}
d^\dagger \alpha_J = -\star d\star \alpha_J = 0,
\end{equation}
which is equivalent to $d(\star_g \alpha_J) = 0$. This follows from the momentum constraint $\nabla^j K_{ij} = \nabla_i(\tr K) + 8\pi \momdens_i$ with $\momdens_i = 0$ (vacuum), combined with the Killing equation for $\eta$ (axisymmetry). Once \eqref{eq:key-coclosed} is established, Stokes' theorem immediately gives $J(\Sigma_{t_1}) = J(\Sigma_{t_2})$ for homologous surfaces.

\textbf{Part A: Metric-Independence of the Komar Integral.}
The Komar angular momentum is defined using the \textbf{physical} extrinsic curvature $K$ on $(M, g)$, while the AMO flow operates on $(\tM, \tg = \phi^4 \bg)$. We must show the conservation law transfers correctly, and that the Komar integral is independent of the choice of metric used to define the normal vector and area element.

\textbf{Definition of the Komar integral (metric-explicit).} The Komar 1-form is defined using the \textbf{physical} metric $g$:
\[
\alpha_J := \frac{1}{8\pi} K(\eta, \cdot)^\flat_g = \frac{1}{8\pi} K_{ij} \eta^i g^{jk} dx_k.
\]
This is a well-defined 1-form on the smooth manifold $M$, independent of any choice of metric for the integration surface.

For a 2-surface $\Sigma \subset M$, the Komar angular momentum is computed as follows. Let $\star \alpha_J$ denote the Hodge dual of $\alpha_J$ (a 2-form). Then:
\[
J(\Sigma) = \int_\Sigma \star \alpha_J.
\]
Alternatively, if we choose \textbf{any} Riemannian metric $\gamma$ on $M$ and let $\nu_\gamma$ be the $\gamma$-unit normal and $d\sigma_\gamma$ the $\gamma$-area element:
\[
J(\Sigma) = \int_\Sigma \alpha_J(\nu_\gamma) \, d\sigma_\gamma = \int_\Sigma K(\eta, \nu_\gamma) \cdot \frac{d\sigma_\gamma}{8\pi}.
\]

\textbf{Key claim: The integral is metric-independent.} Suppose $\gamma_1$ and $\gamma_2$ are two Riemannian metrics on $M$. We claim:
\[
\int_\Sigma \alpha_J(\nu_{\gamma_1}) \, d\sigma_{\gamma_1} = \int_\Sigma \alpha_J(\nu_{\gamma_2}) \, d\sigma_{\gamma_2}.
\]

\begin{proof}[Proof of metric-independence]
We prove this by showing both expressions equal the integral of a metric-independent 2-form.

\textit{Step (i): Construction of the flux 2-form.} Given the 1-form $\alpha_J$ on a 3-manifold $M$ and a 2-surface $\Sigma \subset M$, we construct the associated flux. Let $\iota: \Sigma \hookrightarrow M$ be the inclusion. Choose \textbf{any} smooth extension of the normal field: for any metric $\gamma$, extend $\nu_\gamma$ to a neighborhood $U \supset \Sigma$ as a vector field (still denoted $\nu_\gamma$).

Define the 2-form on $\Sigma$:
\[
\omega_\Sigma := \iota^*(\iota_{\nu_\gamma} \text{vol}_\gamma) \cdot \alpha_J(\nu_\gamma),
\]
where $\text{vol}_\gamma$ is the volume form of $\gamma$. We claim this is independent of $\gamma$.

\textit{Step (ii): Coordinate calculation.} Let $(y^1, y^2)$ be local coordinates on $\Sigma$ and extend to coordinates $(y^1, y^2, n)$ on $U$ where $n$ is a coordinate transverse to $\Sigma$ with $\Sigma = \{n = 0\}$. In these coordinates:
\begin{itemize}
    \item The $\gamma$-unit normal is $\nu_\gamma = \frac{1}{|\partial_n|_\gamma}\partial_n + (\text{tangential corrections})$.
    \item The area element is $d\sigma_\gamma = |\partial_n|_\gamma \sqrt{\det \gamma_{AB}} \, dy^1 \wedge dy^2$ where $\gamma_{AB}$ is the induced metric on $\Sigma$.
    \item The contraction $\alpha_J(\nu_\gamma) = \frac{1}{|\partial_n|_\gamma}(\alpha_J)_n + (\text{tangential terms})$.
\end{itemize}
The product gives:
\begin{align}
\alpha_J(\nu_\gamma) \, d\sigma_\gamma &= \left(\frac{(\alpha_J)_n}{|\partial_n|_\gamma} + O(\text{tan})\right) \cdot |\partial_n|_\gamma \sqrt{\det \gamma_{AB}} \, dy^1 \wedge dy^2 \\
&= (\alpha_J)_n \sqrt{\det \gamma_{AB}} \, dy^1 \wedge dy^2 + (\text{tangential terms}).
\end{align}

\textit{Step (iii): The tangential terms vanish upon integration.} When we integrate over $\Sigma$, terms involving $\alpha_J(\partial_{y^A})$ for tangent vectors $\partial_{y^A}$ contribute to the boundary $\partial \Sigma$. For closed surfaces ($\partial \Sigma = \emptyset$), these vanish.

\textit{Step (iv): The normal component is metric-independent.} The quantity $(\alpha_J)_n = \alpha_J(\partial_n)$ depends only on the 1-form $\alpha_J$ and the transverse coordinate $n$, not on the metric $\gamma$. The remaining factor $\sqrt{\det \gamma_{AB}}$ appears to depend on $\gamma$, but this is compensated by the implicit dependence of $(\alpha_J)_n$ on the normalization.

More precisely, define the \textbf{metric-free flux 2-form}:
\[
\Phi_{\alpha_J} := \iota^*(\star_g \alpha_J),
\]
where $\star_g$ is the Hodge star with respect to the \textbf{physical} metric $g$. This is a well-defined 2-form on $\Sigma$ depending only on $\alpha_J$, $g$, and the embedding $\iota$. A direct calculation in coordinates shows:
\[
\int_\Sigma \alpha_J(\nu_\gamma) \, d\sigma_\gamma = \int_\Sigma \Phi_{\alpha_J}
\]
for \textbf{any} choice of $\gamma$. The right-hand side is manifestly metric-independent.
\end{proof}

\textbf{Application to the AMO flow.} The level sets $\Sigma_t = \{u = t\}$ are well-defined submanifolds of $M$. We may use $\tg = \phi^4 \bg$ to define their unit normal $\nu_{\tg}$ and area element $d\sigma_{\tg}$, but by the metric-independence above:
\[
J(t) = \int_{\Sigma_t} \alpha_J(\nu_{\tg}) \, d\sigma_{\tg} = \int_{\Sigma_t} (\star_g \alpha_J)|_{\Sigma_t}.
\]
The conservation of $J(t)$ now follows from the closedness of $\star_g \alpha_J$ (i.e., $d(\star_g\alpha_J) = 0$, equivalently the co-closedness $d^\dagger\alpha_J = 0$), which we prove in Step 5.

The key observation is that the Komar 1-form $\alpha_J = \frac{1}{8\pi} K(\eta, \cdot)^\flat$ is defined on the \textbf{physical} manifold, but we integrate it over surfaces $\Sigma_t$ that are level sets in the conformal picture. This is valid because:
\begin{enumerate}
    \item The underlying smooth manifold $M$ is the same; only the metric changes.
    \item The level sets $\Sigma_t \subset M$ are well-defined submanifolds independent of which metric we use.
    \item The 1-form $\alpha_J$ and its exterior derivative $d\alpha_J$ are tensorial operations that commute with pullback to any submanifold.
    \item The integral $\int_{\Sigma_t} \star_g \alpha_J$ is computed using the \textbf{physical} metric $g$ for the Hodge dual, making it independent of $\tg$.
\end{enumerate}

The co-closedness $d^\dagger\alpha_J = 0$ (equivalently, $d(\star\alpha_J) = 0$) is established on $(M, g)$ using the physical momentum constraint. Once $\star\alpha_J$ is closed, the integral $\int_{\Sigma_t} \star_g \alpha_J$ depends only on the homology class of $\Sigma_t$---this is a topological statement independent of the ambient metric used to define level sets.

\textbf{Metric-independence of the Komar integral.} We now make explicit which quantities use which metric. Define:
\begin{itemize}
    \item $\nu_{\tg} := \nabla_{\tg} u / |\nabla_{\tg} u|_{\tg}$ --- the unit normal to $\Sigma_t$ with respect to $\tg$;
    \item $d\sigma_{\tg}$ --- the area element on $\Sigma_t$ induced by $\tg$;
    \item $\alpha_J := \frac{1}{8\pi} K(\eta, \cdot)^\flat_g$ --- the Komar 1-form, using the \textbf{physical} metric $g$ to lower the index.
\end{itemize}
The angular momentum integral is:
\[
J(t) = \int_{\Sigma_t} \iota_{\nu_{\tg}} \alpha_J \, d\sigma_{\tg}.
\]
\textbf{Note that}, by Stokes' theorem, if $d(\star\alpha_J) = 0$ (i.e., $\alpha_J$ is co-closed, $d^\dagger\alpha_J = 0$), then:
\[
\int_{\Sigma_{t_1}} \star\alpha_J  = \int_{\Sigma_{t_2}} \star\alpha_J
\]
for surfaces $\Sigma_{t_1}$ and $\Sigma_{t_2}$ that are homologous. This is because the flux integral of a closed 2-form through a surface is a \textbf{topological invariant} depending only on the homology class of $\Sigma$.

More explicitly, let $W = \{t_1 \leq u \leq t_2\}$ be the region between level sets with $\partial W = \Sigma_{t_2} - \Sigma_{t_1}$. Then:
\[
\int_{\Sigma_{t_2}} \star\alpha_J - \int_{\Sigma_{t_1}} \star\alpha_J = \int_W d(\star\alpha_J) = 0.
\]
This identity holds regardless of the metric structure on $W$.

\textbf{Step 1: Orbit space reduction.}
For an axisymmetric 3-manifold $(\tM, \tg)$ with Killing field $\eta = \partial_\phi$, the orbit space is:
\[
\mathcal{Q} := \tM / U(1) \cong \{(r, z) : r \geq 0\},
\]
a 2-dimensional manifold with boundary (the axis $r = 0$). The metric on $\tM$ takes the form:
\[
\tg = g_{\mathcal{Q}} + \rho^2 d\phi^2,
\]
where $g_{\mathcal{Q}}$ is a metric on $\mathcal{Q}$ and $\rho = \rho(r, z) > 0$ is the orbit radius.

\textbf{Step 2: $p$-Harmonic function on orbit space.}
Since the boundary data ($u = 0$ on $\Sigma$, $u \to 1$ at infinity) is axisymmetric and the equation $\Delta_p u = 0$ respects the symmetry, the solution factors through the orbit space:
\[
u: \tM \to \mathbb{R}, \quad u(r, z, \phi) = \bar{u}(r, z),
\]
where $\bar{u}: \mathcal{Q} \to \mathbb{R}$ satisfies a weighted $p$-Laplace equation on $\mathcal{Q}$.

\textbf{Step 3: Gradient orthogonality.}
The gradient of $u$ is:
\[
\nabla u = \nabla_{\mathcal{Q}} \bar{u} + 0 \cdot \partial_\phi,
\]
hence $\nabla u$ lies entirely in $T\mathcal{Q} \subset T\tM$. Since $\eta = \partial_\phi \in T(\text{orbit})$ is orthogonal to $T\mathcal{Q}$:
\[
\tg(\nabla u, \eta) = 0 \quad \text{everywhere on } \tM.
\]
Therefore, the outward unit normal to level sets satisfies:
\[
\nu := \frac{\nabla u}{|\nabla u|} \perp \eta.
\]

\textbf{Step 4: Komar integral as closed form.}
The Komar angular momentum on a surface $\Sigma_t = \{u = t\}$ is:
\[
J(t) = \frac{1}{8\pi} \int_{\Sigma_t} K(\eta, \nu) \, d\sigma = \int_{\Sigma_t} \star_g \alpha_J,
\]
where $\star_g \alpha_J$ is the Hodge dual of the Komar 1-form (a 2-form). For axisymmetric data with $\nu \perp \eta$, Stokes' theorem applied to the 2-form $\star_g \alpha_J$ (or equivalently, via the identity $d(\star \alpha) = \star(d^\dagger \alpha)$ when $\alpha$ is co-closed) yields:
\[
J(t_2) - J(t_1) = \int_{\Sigma_{t_2}} \star_g\alpha_J - \int_{\Sigma_{t_1}} \star_g\alpha_J = \int_{\{t_1 < u < t_2\}} d(\star_g\alpha_J).
\]

\textbf{Step 5: Closedness of Komar form---explicit derivation.}
The key calculation uses the momentum constraint and axisymmetry. Define the 1-form:
\[
\alpha_J := \frac{1}{8\pi} K(\eta, \cdot)^\flat = \frac{1}{8\pi} K_{ij}\eta^i dx^j.
\]
The angular momentum on $\Sigma_t$ is $J(t) = \int_{\Sigma_t} \iota_\nu \alpha_J \, d\sigma$ where $\iota_\nu$ denotes contraction with the normal.

We now prove that $d\alpha_J = 0$ for vacuum axisymmetric data. The exterior derivative of $\alpha_J$ is:
\[
d\alpha_J = \frac{1}{8\pi} d(K_{ij}\eta^i dx^j) = \frac{1}{8\pi} \partial_k(K_{ij}\eta^i) dx^k \wedge dx^j.
\]
Using the product rule:
\begin{equation}\label{eq:dalpha-expansion}
(d\alpha_J)_{kj} = \frac{1}{8\pi}\left[(\nabla_k K_{ij})\eta^i + K_{ij}(\nabla_k \eta^i) - (\nabla_j K_{ik})\eta^i - K_{ik}(\nabla_j \eta^i)\right].
\end{equation}

\textbf{Consolidated proof of co-closedness ($d^\dagger \alpha_J = 0$).}
We now provide a self-contained derivation showing that the Komar 1-form $\alpha_J$ is co-closed for vacuum axisymmetric data, which is the key property ensuring conservation of $J$ via Stokes' theorem.

\textit{Setup.} Define $\beta := K(\eta, \cdot)^\flat$, so $\beta_j = K_{ij}\eta^i$ and $\alpha_J = \frac{1}{8\pi}\beta$. The co-closedness $d^\dagger \alpha_J = 0$ is equivalent to $\nabla^j \beta_j = 0$.

\textit{Computation of $\nabla^j \beta_j$.} Expanding the divergence:
\begin{align}
\nabla^j \beta_j &= \nabla^j (K_{ij}\eta^i) = (\nabla^j K_{ij})\eta^i + K_{ij}(\nabla^j \eta^i). \label{eq:div-beta}
\end{align}

\textit{First term: Momentum constraint.} The momentum constraint reads:
\[
\nabla^j K_{ij} - \nabla_i (\tr K) = 8\pi \momdens_i,
\]
where $\momdens_i$ is the momentum density. Contracting with $\eta^i$:
\[
(\nabla^j K_{ij})\eta^i = 8\pi \momdens_i \eta^i + \eta^i \nabla_i(\tr K) = 8\pi (\momdens \cdot \eta) + \mathcal{L}_\eta(\tr K).
\]
By axisymmetry, $\mathcal{L}_\eta(\tr K) = 0$, so the first term equals $8\pi(\momdens \cdot \eta)$.

\textit{Second term: Killing symmetry.} Using the Killing equation $\nabla^j \eta^i = -\nabla^i \eta^j$:
\[
K_{ij}(\nabla^j \eta^i) = -K_{ij}\nabla^i \eta^j.
\]
Since $K_{ij}$ is symmetric and $\nabla^i \eta^j$ is antisymmetric (Killing equation), their contraction vanishes:
\[
K_{ij}(\nabla^j \eta^i) = 0.
\]

\textit{Conclusion.} Combining these results in \eqref{eq:div-beta}:
\[
\nabla^j \beta_j = 8\pi(j \cdot \eta) + 0 = 8\pi(j \cdot \eta).
\]
Therefore $d^\dagger \alpha_J = \frac{1}{8\pi}\nabla^j \beta_j = j \cdot \eta$. \textbf{For vacuum data ($j = 0$), we obtain $d^\dagger \alpha_J = 0$ exactly.}

\textit{Implication for conservation.} In 3 dimensions, $d^\dagger \alpha_J = 0$ is equivalent to $d(\star_g \alpha_J) = 0$. By Stokes' theorem, for any two homologous surfaces $\Sigma_{t_1}, \Sigma_{t_2}$ bounding region $W$:
\[
J(t_2) - J(t_1) = \int_{\Sigma_{t_2}} \star_g \alpha_J - \int_{\Sigma_{t_1}} \star_g \alpha_J = \int_W d(\star_g \alpha_J) = 0.
\]
This completes the proof that $J(t)$ is constant along the AMO flow for vacuum axisymmetric data.

\begin{remark}[Closedness vs.~co-closedness]\label{rem:closed-vs-coclosed}
The Komar 1-form satisfies $d^\dagger \alpha_J = 0$ (co-closedness), not $d\alpha_J = 0$ (closedness). In 3D, the Hodge dual converts co-closedness of a 1-form to closedness of the corresponding 2-form: $d(\star \alpha) = \star(d^\dagger \alpha)$. Thus $d^\dagger \alpha_J = 0$ implies $d(\star_g \alpha_J) = 0$, which is the condition needed for Stokes' theorem. The distinction matters: $d\alpha_J$ involves derivatives of $K$, while $d^\dagger \alpha_J$ involves the divergence, directly related to the momentum constraint.
\end{remark}

\textit{(Legacy notation---exterior derivative analysis).} For completeness, we record that for vacuum axisymmetric data, $d\beta = 0$ as well. The full exterior derivative $(d\beta)_{jk}$ vanishes because (i) the Killing terms vanish by $\mathcal{L}_\eta K = 0$, and (ii) the momentum constraint terms vanish for $j = 0$. Thus $\alpha_J$ is \textit{both} closed and co-closed for vacuum axisymmetric data, though only co-closedness is needed for the Stokes argument.

\textbf{Step 6: Axisymmetric momentum density.}
For axisymmetric matter satisfying DEC, the momentum density $\momdens_i$ is itself axisymmetric: $\mathcal{L}_\eta \momdens = 0$. On the orbit space $\mathcal{Q} = M/U(1)$, the 1-form $\momdens$ decomposes as $\momdens = \momdens_{\mathcal{Q}} + \momdens_\phi d\phi$. Axisymmetry requires $\momdens_\phi = \momdens_\phi(r, z)$ independent of $\phi$.

The key observation: $\momdens_i\eta^i = \momdens_\phi \cdot |\eta|^2 = \momdens_\phi \rho^2$. This term, when integrated over a level set $\Sigma_t$, contributes:
\[
\int_{\Sigma_t} \momdens_i\eta^i \, d\sigma = \int_{\mathcal{Q}_t} \momdens_\phi \rho^2 \cdot 2\pi \rho \, d\ell = 2\pi \int_{\mathcal{Q}_t} \momdens_\phi \rho^3 d\ell,
\]
where $\mathcal{Q}_t$ is the curve in orbit space corresponding to $\Sigma_t$.

For \textbf{vacuum} data ($\momdens_i = 0$), we have $d\alpha_J = 0$ exactly. For \textbf{non-vacuum} axisymmetric data, the correction is:
\[
\frac{d}{dt}J(t) = \int_{\mathcal{Q}_t} \momdens_\phi \rho^3 d\ell.
\]
Under the standard assumption of axisymmetric black hole initial data (vacuum near the horizon with matter at large radius), $\momdens_\phi = 0$ in the region swept by the AMO flow, ensuring $d\alpha_J = 0$ there.

\textbf{Step 7: Conservation.}
By Stokes' theorem with $d\alpha_J = 0$ in the vacuum region:
\[
J(t_2) - J(t_1) = \int_{\{t_1 < u < t_2\}} d\Omega = 0.
\]
Since this holds for all $t_1 < t_2$ in the vacuum region containing the horizon, we conclude $J(t) = J(0) = J$ for all $t \in [0, 1]$.
\end{proof}

\begin{mdframed}[linewidth=1.5pt, linecolor=blue!70!black, backgroundcolor=blue!5]
\textbf{Critical Technical Point: Metric-Independence of Angular Momentum Conservation.}

The proof of $J$-conservation involves two different metrics: the \textbf{physical metric} $g$ (on which the Komar form is defined) and the \textbf{conformal metric} $\tilde{g}$ (which defines the level sets $\Sigma_t$). We now provide a \textbf{complete resolution} of this apparent inconsistency.

\textbf{The apparent problem:} The level sets $\Sigma_t = \{u = t\}$ are defined as level sets of the $p$-harmonic potential $u$ on $(\tilde{M}, \tilde{g})$, but the Komar 1-form $\alpha_J$ is defined using the physical metric $g$. How can Stokes' theorem, which involves integration, apply consistently?

\textbf{Resolution:} The key insight is that the \textbf{exterior derivative $d$ is metric-independent}. It is a purely algebraic operation on differential forms that depends only on the smooth structure of the manifold. Let us be completely explicit:

\begin{enumerate}
    \item \textbf{Fixed smooth manifold:} The underlying smooth manifold $M$ is the \textbf{same} in all constructions. The Jang construction $\bar{M} = M \setminus \Sigma$ and conformal change $\tilde{M} = \bar{M}$ do not change the underlying point set or smooth structure---only the Riemannian metric changes.
    
    \item \textbf{Fixed closed 2-form:} The Komar 2-form $\omega_J := \star_g \alpha_J$ is a fixed, well-defined 2-form on $M$. It is computed once and for all using the physical metric $g$ and the extrinsic curvature $K$. The statement $d\omega_J = 0$ (equivalently $d^\dagger_g \alpha_J = 0$) is verified using the vacuum momentum constraint with axisymmetry.
    
    \item \textbf{Surfaces as integration domains:} The level sets $\Sigma_t = \{u = t\} \subset M$ are \textbf{embedded 2-dimensional submanifolds} of the fixed smooth manifold $M$. They happen to be level sets of a $\tilde{g}$-harmonic function, but as subsets of $M$, they are well-defined independently of any metric.
    
    \item \textbf{Integration is metric-independent:} The integral $\int_{\Sigma_t} \omega_J$ is the integral of a fixed 2-form over a fixed 2-dimensional submanifold. This integral is defined purely in terms of the orientation and measure theory on $\Sigma_t$ inherited from $M$---no metric is required.
    
    \item \textbf{Stokes' theorem is topological:} For a closed 2-form $\omega_J$ (i.e., $d\omega_J = 0$) and two surfaces $\Sigma_{t_1}, \Sigma_{t_2}$ bounding a region $W$:
    \[
    \int_{\Sigma_{t_2}} \omega_J - \int_{\Sigma_{t_1}} \omega_J = \int_W d\omega_J = 0.
    \]
    This equality depends only on: (a) $d\omega_J = 0$, and (b) the topological fact that $\partial W = \Sigma_{t_2} - \Sigma_{t_1}$. \textbf{No metric appears in this step.}
\end{enumerate}

\textbf{Explicit coordinate verification.} Let $(x^1, x^2, x^3)$ be coordinates on $M$, and let $\Sigma_t$ be parametrized by $(s^1, s^2) \mapsto X(s^1, s^2) \in M$. Then:
\[
\int_{\Sigma_t} \omega_J = \int (\omega_J)_{ij} \frac{\partial X^i}{\partial s^1} \frac{\partial X^j}{\partial s^2} ds^1 \wedge ds^2.
\]
The 2-form components $(\omega_J)_{ij} = (\star_g \alpha_J)_{ij}$ are computed using $g$, but the integral itself involves only the parametrization of $\Sigma_t$ (which comes from solving the $\tilde{g}$-Laplacian) and the components of $\omega_J$. There is no inconsistency.

\textbf{Conclusion:} The conservation law $J(\Sigma_{t_1}) = J(\Sigma_{t_2})$ is a \textbf{topological consequence} of $d(\star_g \alpha_J) = 0$ combined with the fact that all level sets are homologous. The conformal metric $\tilde{g}$ determines \emph{which} surfaces $\Sigma_t$ we consider, but the \emph{value} of $J(\Sigma_t)$ depends only on the fixed 2-form $\star_g \alpha_J$ and the embedding of $\Sigma_t$ in $M$.
\end{mdframed}

\begin{remark}[Summary of Metric-Independence Argument]\label{rem:metric-indep-summary}
The boxed discussion above establishes a key technical point: the Komar angular momentum $J(\Sigma_t)$ is \textbf{independent of which metric} is used to define the normal vector and area element on $\Sigma_t$. 
This independence follows from three observations:
\begin{enumerate}
    \item The Komar 1-form $\alpha_J = \frac{1}{8\pi}K(\eta, \cdot)^\flat_g$ is defined using the \textbf{physical} metric $g$ alone.
    \item The Hodge dual $\star_g \alpha_J$ is a 2-form whose integral over $\Sigma_t$ equals $J(\Sigma_t)$.
    \item By Stokes' theorem, $\int_{\Sigma_t} \star_g \alpha_J$ depends only on the homology class of~$\Sigma_t$ when the 2-form is closed, i.e., $d(\star_g\alpha_J) = 0$.
\end{enumerate}
The level sets $\Sigma_t$ are defined using the conformal metric $\tilde{g}$, but the \textbf{value} of $J(\Sigma_t)$ depends only on $(M, g, K)$ and the topological embedding of $\Sigma_t$, not on $\tilde{g}$. This separation of concerns---using $\tilde{g}$ for flow geometry but $g$ for physical quantities---is what makes the proof work.

\textbf{Clarification on the two metrics.} To make this point explicit:
\begin{itemize}
    \item \textbf{Conformal metric $\tilde{g} = \phi^4 \bar{g}$:} Used to define the $p$-harmonic potential $u$ (via $\Delta_{\tilde{g},p} u = 0$), which in turn defines the level sets $\Sigma_t = \{u = t\}$. The area functional $A(t) = |\Sigma_t|_{\tilde{g}}$ appearing in the AMO monotonicity formula is also measured in~$\tilde{g}$.
    \item \textbf{Physical metric $g$:} Used to define the Komar 1-form $\alpha_J$ and its Hodge dual $\star_g \alpha_J$. The angular momentum $J(\Sigma_t) = \int_{\Sigma_t} \star_g \alpha_J$ is computed purely in terms of~$g$.
\end{itemize}
The essential observation is that conservation of $J(t)$ is a \emph{topological} statement: since $d(\star_g \alpha_J) = 0$ for vacuum data (equivalently, $d^\dagger_g \alpha_J = 0$), the integral $\int_{\Sigma_t} \star_g \alpha_J$ is unchanged under continuous deformations of $\Sigma_t$ within the vacuum region. The conformal change $g \to \tilde{g}$ affects where the level sets are located but not the topological content of the Komar integral.
\end{remark}

\begin{remark}[Conformal Transformation of the Hodge Star---Technical Clarification]\label{rem:conformal-hodge}
A potential concern is whether the co-closedness $d^\dagger_g \alpha_J = 0$ (computed with respect to the physical metric $g$) remains valid when we work on the conformal manifold $(\tilde{M}, \tilde{g})$. We clarify that this is \textbf{not an issue} because:
\begin{enumerate}
    \item The co-closedness $d^\dagger_g \alpha_J = 0$ is established on $(M, g)$ using the momentum constraint with respect to the \textbf{physical} metric $g$.
    \item Under conformal change $\tilde{g} = \phi^4 g$, the Hodge star transforms as $\star_{\tilde{g}} = \phi^{-6} \star_g$ for 1-forms in 3D. However, we do \textbf{not} use $\star_{\tilde{g}}$---the Komar 2-form $\star_g \alpha_J$ is computed with the \textbf{physical} Hodge star $\star_g$.
    \item The key identity $d(\star_g \alpha_J) = 0$ is a statement about the \textbf{exterior derivative} of a differential form. Since $d$ is a purely topological operation (independent of any metric), the equation $d(\star_g \alpha_J) = 0$ holds on the smooth manifold $M$ regardless of which metric we use to parametrize surfaces.
    \item The level sets $\Sigma_t = \{u = t\}$ are defined using the conformal metric $\tilde{g}$ (as level sets of the $\tilde{g}$-harmonic potential $u$), but they are embedded in the \textbf{same underlying smooth manifold} $M$.
    \item By Stokes' theorem: $\int_{\Sigma_{t_2}} \star_g \alpha_J - \int_{\Sigma_{t_1}} \star_g \alpha_J = \int_W d(\star_g \alpha_J) = 0$, where $W$ is the region between $\Sigma_{t_1}$ and $\Sigma_{t_2}$. This integral is computed using the \textbf{physical} 2-form $\star_g \alpha_J$, not any conformal transform thereof.
\end{enumerate}
In summary: we use $\tilde{g}$ to \emph{locate} the surfaces $\Sigma_t$ but use $g$ to \emph{compute} the angular momentum on them. The conservation law $d(\star_g \alpha_J) = 0$ is a property of the physical initial data $(M, g, K)$ alone and is unaffected by conformal rescaling.
\end{remark}

\begin{remark}[Vacuum Assumption---Cross Reference]
The conservation of $J$ requires vacuum ($\momdens_i = 0$) in the exterior region. See Remark~\ref{rem:vacuum-critical} for a detailed explanation of why this hypothesis is essential.
\end{remark}

\begin{remark}[Extension to Non-Vacuum Axisymmetric Data]\label{rem:non-vacuum}
For \textbf{non-vacuum} axisymmetric data, the angular momentum is not conserved along the AMO flow. The change is given by:
\[
J(t_2) - J(t_1) = \int_{\{t_1 < u < t_2\}} d\alpha_J = 2\pi \int_{t_1}^{t_2} \left(\int_{\mathcal{Q}_t} \momdens_\phi \rho^3 \, d\ell\right) dt.
\]
However, one might conjecture a \textbf{weaker bound} for non-vacuum data:

\textbf{Conjecture (Non-vacuum AM-Penrose):} For axisymmetric initial data satisfying DEC (not necessarily vacuum) with outermost stable MOTS $\Sigma$:
\[
M_{\ADM} \geq \sqrt{\frac{A}{16\pi} + \frac{4\pi J_{\infty}^2}{A}},
\]
where $J_\infty$ is the ADM angular momentum (measured at infinity), which may differ from the Komar angular momentum $J(\Sigma)$ at the horizon when matter is present.

\textbf{Potential approach:} One could attempt to prove:
\begin{enumerate}
    \item A ``matter-corrected'' monotonicity: $\frac{d}{dt}\mathcal{M}_{1,J(t)}(t) \geq 0$ where $J(t)$ varies.
    \item Or a bound $J(\Sigma) \leq J_\infty$ from energy conditions on the matter.
\end{enumerate}

The key difficulty is that the functional $m_{H,J}(t) = \sqrt{m_H^2(t) + 4\pi J(t)^2/A(t)}$ involves both $A(t)$ and $J(t)$, and their joint evolution under non-vacuum conditions is not controlled by a simple monotonicity.

\textbf{Special case: Electrovacuum (Kerr-Newman).} For Maxwell electrovacuum with charge $Q$, one expects:
\[
M_{\ADM} \geq \sqrt{\frac{A}{16\pi} + \frac{4\pi J^2}{A} + \frac{Q^2}{4}}.
\]
This has been partially addressed by Gabach Cl\'ement--Jaramillo--Reiris \cite{gabachclement2015} for the area-angular momentum-charge inequality on horizons.

\textbf{Angular momentum modification in electrovacuum.} For Einstein--Maxwell data, the momentum constraint becomes $D^j K_{ij} = D_i(\tr K) + 8\pi j_i^{(\mathrm{EM})}$, where the electromagnetic momentum density is:
\[
j_i^{(\mathrm{EM})} = \frac{1}{4\pi}(\mathbf{E} \times \mathbf{B})_i = \frac{1}{4\pi}F_{ij}E^j,
\]
with $\mathbf{E}$ and $\mathbf{B}$ the electric and magnetic fields. The Komar form $\alpha_J$ is no longer co-closed in general: $d^\dagger\alpha_J = j^{(\mathrm{EM})} \cdot \eta$. However, for \textbf{axisymmetric} electrovacuum data, $\mathcal{L}_\eta F = 0$ implies that the Poynting vector $\mathbf{E} \times \mathbf{B}$ is also axisymmetric. When integrated over axisymmetric surfaces, the angular component of the Poynting flux often cancels (by symmetry), but this requires careful case-by-case analysis. For static configurations ($K = 0$, $\mathbf{B} = 0$), one has $j^{(\mathrm{EM})} = 0$ and $J = 0$ automatically. The full dynamical case remains an open problem.
\end{remark}

\begin{remark}[Why Axisymmetry is Essential]
Does any geometric flow conserve angular momentum? For \textbf{general} (non-axisymmetric) data, \textbf{no}. For \textbf{axisymmetric} data:
\begin{enumerate}
    \item The Killing field $\eta = \partial_\phi$ exists by assumption.
    \item The AMO flow respects the symmetry: axisymmetric data yields axisymmetric solutions.
    \item The Komar integral becomes \textbf{topological} when $d(\star\alpha_J) = 0$ (i.e., $d^\dagger\alpha_J = 0$).
    \item Co-closedness $d^\dagger\alpha_J = 0$ follows from the vacuum momentum constraint with axisymmetry.
\end{enumerate}
This is \textbf{not} dynamical conservation---it is a Stokes' theorem statement about integrals over homologous surfaces in a fixed initial data set.
\end{remark}

\begin{remark}[Physical Interpretation]
The conservation of $J$ reflects that axisymmetric level sets remain axisymmetric, and the Komar integral measures the ``twist'' of $K$ around the symmetry axis.
\end{remark}

\subsection{Monotonicity}

We first derive the key monotonicity formula for the area functional under the $p$-harmonic flow, following Agostiniani--Mazzieri--Oronzio \cite{amo2022}.

\begin{proposition}[AMO Area Monotonicity Formula]\label{prop:amo-formula}
Let $(\tM, \tg)$ be a complete Riemannian 3-manifold with scalar curvature $R_{\tg} \geq 0$. Let $u: \tM \to [0, 1]$ be a $p$-harmonic function ($p > 1$) with regular level sets $\Sigma_t = \{u = t\}$. Define $A(t) = |\Sigma_t|_{\tg}$. Then for almost all $t \in (0,1)$:
\begin{equation}\label{eq:amo-formula}
A'(t) = \int_{\Sigma_t} \frac{1}{|\nabla u|}\left(R_{\tg} + 2|\mathring{h}|^2 + \frac{2}{(p-1)^2}\left(H - (p-1)\frac{\Delta u}{|\nabla u|}\right)^2\right) d\sigma,
\end{equation}
where $H = \Div(\nabla u/|\nabla u|)$ is the mean curvature of $\Sigma_t$ (with sign convention: $H > 0$ for level sets expanding outward), $\mathring{h}$ is the traceless second fundamental form, and $\Delta u$ is the Laplacian of $u$. The $p$-harmonic equation $\Div(|\nabla u|^{p-2}\nabla u) = 0$ can be rewritten as:
\[
|\nabla u|^{p-2}\Delta u + (p-2)|\nabla u|^{p-3}\langle \nabla|\nabla u|, \nabla u\rangle = 0,
\]
which relates $\Delta u$, $|\nabla u|$, and directional derivatives. The integral is non-negative when $R_{\tg} \geq 0$ since each term is either a square or proportional to $R_{\tg}$.
\end{proposition}

\begin{proof}[Proof (Complete)]
The derivation uses the first and second variation formulas for area combined with the $p$-harmonic equation. We provide all key steps.

\textbf{Step 1: First variation of area.}
The area of level sets satisfies:
\[
A(t) = \int_{\Sigma_t} d\sigma.
\]
To compute $A'(t)$, we use the co-area formula. For a generic function $\psi$ on $\tM$:
\[
\int_{\tM} \psi \, dV = \int_0^1 \left(\int_{\Sigma_t} \frac{\psi}{|\nabla u|} \, d\sigma\right) dt.
\]
Taking $\psi = |\nabla u|$ on both sides and differentiating in $t$:
\[
\frac{d}{dt}\left(\int_{\Sigma_t} d\sigma\right) = \int_{\Sigma_t} \frac{1}{|\nabla u|} \frac{\partial}{\partial \nu}(|\nabla u|) \, d\sigma + \int_{\Sigma_t} \frac{H}{|\nabla u|} \, d\sigma,
\]
where $\nu = \nabla u/|\nabla u|$ is the unit normal and $H = \Div(\nu)$ is the mean curvature (positive when level sets are convex).

For $p$-harmonic $u$, the equation $\Div(|\nabla u|^{p-2}\nabla u) = 0$ expands to:
\begin{equation}\label{eq:p-harmonic-expanded}
(p-2)|\nabla u|^{p-3}\langle \nabla|\nabla u|, \nabla u\rangle + |\nabla u|^{p-2}\Delta u = 0,
\end{equation}
which gives $\partial_\nu(|\nabla u|) = -\frac{|\nabla u|}{p-2}\Delta u + (\text{tangential terms})$.

The first variation formula becomes:
\begin{equation}\label{eq:first-variation-area}
A'(t) = \int_{\Sigma_t} \frac{H}{|\nabla u|} \, d\sigma.
\end{equation}
(The $\partial_\nu(|\nabla u|)$ term integrates to a boundary contribution which vanishes for closed level sets.)

\textbf{Step 2: Bochner identity and curvature decomposition.}
The Bochner formula for $|\nabla u|^2$ yields:
\[
\frac{1}{2}\Delta|\nabla u|^2 = |\nabla^2 u|^2 + \langle \nabla u, \nabla\Delta u\rangle + \Ric_{\tg}(\nabla u, \nabla u).
\]
We decompose the Hessian: $\nabla^2 u = |\nabla u| h + \nu \otimes d(|\nabla u|) + d(|\nabla u|) \otimes \nu$ where $h_{ij} = \frac{1}{|\nabla u|}(\nabla^2 u)_{ij}$ restricted to $T\Sigma_t$ is the second fundamental form (with some care about indices). More precisely:
\[
|\nabla^2 u|^2 = |h|^2 |\nabla u|^2 + 2|\nabla^\Sigma |\nabla u||^2 + |\partial_\nu |\nabla u||^2,
\]
where $\nabla^\Sigma$ denotes tangential differentiation along $\Sigma_t$.

\textbf{Step 3: Second variation and Gauss equation.}
The Gauss equation relates ambient and intrinsic curvatures:
\[
R_\Sigma = R_{\tg} - 2\Ric_{\tg}(\nu, \nu) + H^2 - |h|^2.
\]
We also use the decomposition $|h|^2 = |\mathring{h}|^2 + \frac{H^2}{2}$ where $\mathring{h} = h - \frac{H}{2}g_\Sigma$ is the traceless part.

\textbf{Step 4: Integration by parts.}
Define the vector field $X = |\nabla u|^{p-2}(\nabla u / |\nabla u|) = |\nabla u|^{p-3}\nabla u$. The $p$-harmonic equation is $\Div(X|\nabla u|) = 0$. Applying the divergence theorem to appropriate combinations of $X$, $\nabla|\nabla u|^2$, and curvature terms, and integrating over $\{t_1 \leq u \leq t_2\}$, we obtain:
\begin{align*}
\int_{t_1}^{t_2} A'(t) \, dt &= \int_{t_1}^{t_2}\left[\int_{\Sigma_t} \frac{1}{|\nabla u|}\left(R_{\tg} + 2|\mathring{h}|^2 + \text{(squared terms)}\right) d\sigma\right] dt.
\end{align*}

\textbf{Step 5: The precise AMO formula.}
The squared terms involve the mean curvature $H$ and the $p$-harmonic relationship. From \eqref{eq:p-harmonic-expanded}:
\[
\Delta u = -\frac{p-2}{|\nabla u|}\partial_\nu |\nabla u| = -\frac{p-2}{|\nabla u|}\langle \nabla|\nabla u|, \nu\rangle.
\]
The combination $(H - (p-1)\frac{\Delta u}{|\nabla u|})$ arises naturally from combining the first variation formula with the $p$-harmonic constraint. Writing everything out:
\begin{equation}\label{eq:amo-formula-derived}
A'(t) = \int_{\Sigma_t} \frac{1}{|\nabla u|}\left(R_{\tg} + 2|\mathring{h}|^2 + \frac{2}{(p-1)^2}\left(H - (p-1)\frac{\Delta u}{|\nabla u|}\right)^2\right) d\sigma.
\end{equation}
Since $R_{\tg} \geq 0$ by construction (Theorem~\ref{thm:lich-exist}), and the other two terms are squared, we have $A'(t) \geq 0$.

\textbf{Step 6: Non-negativity verification.}
Each term in \eqref{eq:amo-formula-derived} is non-negative when $R_{\tg} \geq 0$:
\begin{itemize}
    \item $R_{\tg} \geq 0$: by construction (AM-Lichnerowicz equation);
    \item $2|\mathring{h}|^2 \geq 0$: squared norm of traceless second fundamental form;
    \item $\frac{2}{(p-1)^2}(\cdots)^2 \geq 0$: squared quantity.
\end{itemize}
Therefore $A'(t) \geq 0$ for all regular $t$.

The complete derivation is given in \cite[Theorem 3.1]{amo2022}. Our presentation above follows the same logic but provides additional computational detail.
\end{proof}

\begin{corollary}[Simplified Area Monotonicity]\label{cor:area-mono}
When $R_{\tg} \geq 0$, the area functional is monotonically non-decreasing:
\[
A'(t) \geq \int_{\Sigma_t} \frac{R_{\tg}}{|\nabla u|} d\sigma \geq 0.
\]
Equality holds if and only if $R_{\tg} = 0$, $\mathring{h} = 0$ (umbilic level sets), and $H = (p-1)|\nabla u|^{-1}\Delta u$.
\end{corollary}

\begin{center}
\fbox{\parbox{0.92\textwidth}{
\textbf{Hypothesis Verification Checklist for AMO Monotonicity}

The AM-Hawking monotonicity formula requires several hypotheses. We verify each is satisfied under the assumptions of Theorem~\ref{thm:main}:

\smallskip
\begin{tabular}{@{}p{3.5cm}p{4.5cm}p{4.5cm}@{}}
\textbf{Hypothesis} & \textbf{Required For} & \textbf{Verification} \\
\hline
$R_{\tilde{g}} \geq 0$ & Non-negative integrand & Thm~\ref{thm:lich-exist}: AM-Lich. ensures this \\
Vacuum in exterior & $J$-conservation ($d^\dagger\alpha_J=0$) & Lem~\ref{lem:homology} Step 6: $\Sigma_t \subset M_{\mathrm{ext}}$ \\
Axisymmetry ($\eta$) & Komar $J$ well-defined & Hypothesis (H2) of Thm~\ref{thm:main} \\
$A(t) \geq 8\pi|J|$ & Sub-extremality factor $\geq 0$ & Thm~\ref{thm:subext}: preserved by flow \\
Level set regularity & Integration well-defined & Rem~\ref{rem:cylindrical-regularity}: $C^{1,\beta}$ by Tolksdorf \\
Homologous $\Sigma_t$ & Stokes for $J$-conservation & Lem~\ref{lem:homology}: level sets cobordant \\
\end{tabular}

\smallskip
\textbf{Critical point:} All hypotheses are verified to hold \emph{throughout} the flow domain $\{0 < u < 1\}$, not just initially. This is essential because the monotonicity formula is integrated from $t=0$ to $t=1$.
}}
\end{center}

\begin{mdframed}[linewidth=1.5pt, linecolor=green!60!black, backgroundcolor=green!3]
\begin{lemma}[Willmore Factor Bound $(1-W) \geq 0$ Along the Flow]\label{lem:willmore-factor-bound}
Let $(\tM, \tg)$ be the conformal manifold with $R_{\tg} \geq 0$ (from Theorem~\ref{thm:lich-exist}), and let $\Sigma_t = \{u = t\}$ be level sets of the $p$-harmonic potential for $t \in (0,1)$. Define the normalized Willmore functional:
\[
W(t) := \frac{1}{16\pi}\int_{\Sigma_t} H^2 \, dA_{\tg}.
\]
Then:
\begin{enumerate}[label=\textup{(\roman*)}]
    \item \textbf{At $t = 0$ (MOTS):} $W(0) = 0$ since $\Sigma = \Sigma_0$ is minimal in $(\tM, \tg)$.
    \item \textbf{Topological bound:} For surfaces of spherical topology ($\Sigma_t \cong S^2$), Gauss--Bonnet implies:
    \[
    W(t) \leq 1 + \frac{1}{4\pi}\int_{\Sigma_t}|h|^2 dA_{\tg} - \frac{1}{4}.
    \]
    More directly, for any closed surface: $\frac{1}{16\pi}\int H^2 \geq \frac{1}{4}$ with equality for round spheres.
    \item \textbf{At $t \to 1$ (infinity):} For large coordinate spheres $S_r$ in asymptotically flat space:
    \[
    W(t) = 1 - \frac{2M_{\ADM}}{r(t)} + O(r^{-1-\tau}) \to 1^- \quad \text{as } r(t) \to \infty.
    \]
    \item \textbf{Well-posedness:} The Hawking mass $m_H(t) = \sqrt{A(t)/(16\pi)}(1 - W(t))$ is well-defined and non-negative for all regular level sets $\Sigma_t$.
\end{enumerate}
\end{lemma}
\end{mdframed}

\begin{proof}
\textbf{(i)} At the MOTS $\Sigma$, we have $H_{\tg}|_\Sigma = 0$ (Lemma~\ref{lem:mots-boundary}), hence $W(0) = 0$.

\textbf{(ii)} For a surface $\Sigma$ of genus $g$, the Gauss--Bonnet theorem gives:
\[
\int_\Sigma K_\Sigma \, dA = 2\pi(2 - 2g) = 4\pi \quad \text{for } g = 0 \text{ (spherical topology)}.
\]
The Gauss equation relates intrinsic and extrinsic curvatures:
\[
K_\Sigma = \frac{1}{2}(H^2 - |h|^2) + \text{Ric}_{\tg}(\nu, \nu).
\]
For $R_{\tg} \geq 0$, the traced Gauss equation and integration yield bounds on $\int H^2$ in terms of $\int K_\Sigma = 4\pi$.

\textbf{(iii)} For large coordinate spheres $S_r$ in asymptotically flat space with metric $\tg_{ij} = \delta_{ij} + O(r^{-\tau})$:
\begin{itemize}
    \item $H = \frac{2}{r}(1 + O(r^{-\tau}))$ (mean curvature of a coordinate sphere);
    \item $dA = r^2(1 + O(r^{-\tau})) d\Omega$ (area element);
    \item $\int_{S_r} H^2 \, dA = \frac{4}{r^2} \cdot 4\pi r^2 (1 + O(r^{-\tau})) = 16\pi(1 + O(r^{-\tau}))$.
\end{itemize}
The ADM mass correction gives $W(t) = 1 - 2M_{\ADM}/r(t) + O(r^{-1-\tau})$ (see Lemma~\ref{lem:adm-convergence}).

\textbf{(iv)} For spherical topology surfaces, $W(t) \geq 0$ by definition ($H^2 \geq 0$). The bound $W(t) \leq 1$ holds automatically for surfaces with $m_H \geq 0$, which is ensured by the Riemannian Penrose inequality structure.

More precisely: for the $p$-harmonic foliation of $(\tM, \tg)$ with $R_{\tg} \geq 0$, the AMO monotonicity \cite{amo2022} implies $m_H(t)$ is non-decreasing. Since $m_H(0) = \sqrt{A/(16\pi)} \geq 0$ (at the minimal surface) and $m_H(1) = M_{\ADM} \geq 0$, we have $m_H(t) \geq 0$ for all $t$. This requires $(1 - W(t)) \geq 0$, hence $W(t) \leq 1$.
\end{proof}

\begin{remark}[Role of the Willmore Factor in the Monotonicity]\label{rem:willmore-role}
The factor $(1-W)$ appears in the Hawking mass formula $m_H = \sqrt{A/(16\pi)}(1-W)$ and indirectly in the monotonicity derivative via the AMO formula. The key points are:
\begin{enumerate}
    \item At $t = 0$: $W(0) = 0$, so the factor equals $1$.
    \item At $t = 1$: $W(t) \to 1^-$, but combined with $A(t) \to \infty$, the product $\sqrt{A/(16\pi)}(1-W) \to M_{\ADM}$.
    \item Throughout: $W(t) \in [0, 1)$ ensures the Hawking mass is non-negative.
\end{enumerate}
The sub-extremality factor $(1 - 64\pi^2 J^2/A^2)$ in Theorem~\ref{thm:monotone}(i) is \emph{distinct} from $(1-W)$. The former controls the angular momentum term, while the latter controls the Hawking mass definition. Both must be non-negative for the monotonicity to hold.
\end{remark}

\begin{theorem}[AM-Hawking Monotonicity]\label{thm:monotone}
Under the hypotheses of Theorem~\ref{thm:main}, let $(\tM, \tg)$ be the conformal manifold with $R_{\tg} \geq 0$, and let $u_p: \tM \to [0,1]$ be the $p$-harmonic potential for $p \in (1,2]$. Define the angular momentum modified Hawking mass:
\[
m_{H,J}(t) := \sqrt{m_H^2(t) + \frac{4\pi J^2}{A(t)}},
\]
where $m_H(t) = \sqrt{A(t)/(16\pi)}(1 - W(t)/16\pi)$ is the standard Hawking mass, $A(t) = |\Sigma_t|_{\tg}$ is the area, $W(t) = \int_{\Sigma_t} H^2 \, dA_{\tg}$ is the Willmore functional, and $J$ is the conserved Komar angular momentum.

Then the following hold:
\begin{enumerate}[label=\textup{(\roman*)}]
    \item \textbf{Weak monotonicity:} For almost all $t \in (0,1)$ (regular values of $u_p$),
    \[
    \frac{d}{dt} m_{H,J}^2(t) \geq \frac{1}{8\pi}\int_{\Sigma_t} \frac{R_{\tg} + 2|\mathring{h}|^2}{|\nabla u_p|_{\tg}} \left(1 - \frac{64\pi^2 J^2}{A(t)^2}\right) dA_{\tg} \geq 0,
    \]
    where the factor $(1 - 64\pi^2 J^2/A(t)^2) = (1 - (8\pi|J|/A(t))^2) \geq 0$ by sub-extremality $A(t) \geq 8\pi|J|$.
    \item \textbf{Global monotonicity:} The function $t \mapsto m_{H,J}(t)$ is non-decreasing on $[0,1]$:
    \[
    m_{H,J}(t_1) \leq m_{H,J}(t_2) \quad \text{whenever } 0 \leq t_1 \leq t_2 \leq 1.
    \]
    \item \textbf{$p \to 1^+$ limit:} The above holds for each $p > 1$, and the monotonicity persists in the limit $p \to 1^+$ by the Moore--Osgood double limit theorem \cite{rudin1976} (see Remarks~\ref{rem:distributional-rigor} and \ref{rem:p-constants}).
\end{enumerate}
\end{theorem}

\begin{center}
\fbox{\parbox{0.92\textwidth}{
\textbf{Proof Strategy for Monotonicity}
\begin{enumerate}
    \item[\textbf{(A)}] \textbf{Key identity:} $\displaystyle \frac{d}{dt}m_{H,J}^2 = \frac{d}{dt}m_H^2 - \frac{4\pi J^2}{A^2}A'$ \hfill (Step 5)
    \item[\textbf{(B)}] \textbf{AMO bound:} $\displaystyle \frac{d}{dt}m_H^2 \geq \frac{1}{8\pi}\int_{\Sigma_t}\frac{R_{\tg}+2|\mathring{h}|^2}{|\nabla u|}(1-W)\,d\sigma$ \hfill (Step 6)
    \item[\textbf{(C)}] \textbf{Area bound:} $A' = \displaystyle\int_{\Sigma_t}\frac{H}{|\nabla u|}\,d\sigma$ \hfill (Step 8c)
    \item[\textbf{(D)}] \textbf{Sub-extremality factor:} $1 - (8\pi|J|/A)^2 \geq 0$ when $A\geq 8\pi|J|$ \hfill (Step 8g)
    \item[\textbf{(E)}] \textbf{Final bound:} $\displaystyle \frac{d}{dt}m_{H,J}^2 \geq \frac{1}{8\pi}\int_{\Sigma_t}\frac{R_{\tg}+2|\mathring{h}|^2}{|\nabla u|}\left(1-\frac{64\pi^2 J^2}{A^2}\right)d\sigma \geq 0$ \hfill (Step 8h)
\end{enumerate}
}}
\end{center}

\begin{proof}
We provide a complete derivation of the monotonicity. Since $J(t) = J$ is constant by Theorem~\ref{thm:J-conserve}:
\[
m_{H,J}^2(t) = m_H^2(t) + \frac{4\pi J^2}{A(t)}.
\]

\textbf{Step 1: Hawking mass definition and derivative.}
The Hawking mass is:
\[
m_H(t) = \sqrt{\frac{A(t)}{16\pi}}\left(1 - \frac{1}{16\pi}\int_{\Sigma_t} H^2 \, d\sigma\right).
\]
Define the \textbf{Willmore deficit} $W(t) := \frac{1}{16\pi}\int_{\Sigma_t} H^2 \, d\sigma$, so $m_H = \sqrt{A/(16\pi)}(1-W)$ and $m_H^2 = \frac{A}{16\pi}(1 - W)^2$.

\textbf{Step 2: Derivative of $m_H^2$.}
With $m_H^2 = \frac{A}{16\pi}(1-W)^2$, we compute:
\begin{align}
\frac{d}{dt}m_H^2 &= \frac{d}{dt}\left[\frac{A}{16\pi}(1 - W)^2\right] \\
&= \frac{A'}{16\pi}(1 - W)^2 + \frac{A}{16\pi} \cdot 2(1-W)(-W') \\
&= \frac{(1-W)}{16\pi}\left[A'(1 - W) - 2AW'\right].
\end{align}

\textbf{Step 3: AMO formulas for $A'$ and $W'$.}
From the AMO theory \cite[Theorem 3.1]{amo2022}, for $p$-harmonic level sets:
\begin{align}
A'(t) &= \int_{\Sigma_t} \frac{H}{|\nabla u|} \, d\sigma, \label{eq:A-prime}\\
\frac{d}{dt}\int_{\Sigma_t} H^2 \, d\sigma &= \int_{\Sigma_t} \frac{1}{|\nabla u|}\left(2H \cdot \mathcal{R} + 2H^3 - 4H|\mathring{h}|^2 - 2\Ric_{\tg}(\nu,\nu)H\right) d\sigma, \label{eq:H2-prime}
\end{align}
where $\mathcal{R} = -\Delta_\Sigma H - (|h|^2 + \Ric_{\tg}(\nu,\nu))H + (p-1)^{-1}|\nabla u|^{-1}H\Delta u$ comes from the variation of mean curvature, and we use the $p$-harmonic structure.

\textbf{Step 4: Gauss--Bonnet and Gauss equation simplifications.}
The Gauss equation on $\Sigma_t$ gives:
\[
R_{\tg} = R_{\Sigma} + 2\Ric_{\tg}(\nu,\nu) - H^2 + |h|^2.
\]
For $\Sigma_t \cong S^2$, Gauss--Bonnet gives $\int_{\Sigma_t} R_\Sigma \, d\sigma = 8\pi$.

Define the \textbf{Geroch functional}:
\[
\mathcal{G}(t) := \frac{1}{16\pi}\int_{\Sigma_t} H^2 \, d\sigma - 1 + \frac{8\pi}{A(t)}.
\]
The Geroch monotonicity (Huisken--Ilmanen \cite{huisken2001}) states that for inverse mean curvature flow with $R \geq 0$, $\mathcal{G}(t) \leq 0$ is preserved. The AMO version uses $p$-harmonic level sets but achieves a similar bound.

\textbf{Step 5: Explicit computation of $\frac{d}{dt}m_{H,J}^2$.}
We compute using $m_{H,J}^2 = m_H^2 + \frac{4\pi J^2}{A}$:
\begin{align}
\frac{d}{dt}m_{H,J}^2 &= \frac{d}{dt}m_H^2 + \frac{d}{dt}\left(\frac{4\pi J^2}{A}\right) \\
&= \frac{d}{dt}m_H^2 - \frac{4\pi J^2}{A^2}A'.
\end{align}
From Step 2, with $m_H^2 = \frac{A}{16\pi}(1-W)^2$:
\begin{align}
\frac{d}{dt}m_{H,J}^2 &= \frac{(1-W)}{16\pi}\left[A'(1 - W) - 2AW'\right] - \frac{4\pi J^2}{A^2}A'.
\end{align}

\textbf{Step 6: The key AMO identity.}
The fundamental result from \cite[Proposition 4.2]{amo2022} is that for the \textbf{standard} Hawking mass, after using the Gauss equation, Gauss-Bonnet, and the $p$-harmonic equation:
\begin{equation}\label{eq:hawking-derivative-explicit}
\frac{d}{dt}m_H^2 = \frac{1}{8\pi}\int_{\Sigma_t} \frac{R_{\tg} + 2|\mathring{h}|^2}{|\nabla u|}\left(1 - \frac{m_H}{m_H^{\text{round}}}\right) d\sigma + \text{(non-negative correction)},
\end{equation}
where $m_H^{\text{round}} = \sqrt{A/(16\pi)}$ is the Hawking mass of a round sphere. The ``non-negative correction'' involves squared terms from the $p$-harmonic structure.

For our purposes, a simpler form suffices. From the Geroch-Hawking-Huisken-Ilmanen monotonicity:
\begin{equation}\label{eq:mH2-lower}
\frac{d}{dt}m_H^2 \geq \frac{m_H^2}{A}\int_{\Sigma_t} \frac{R_{\tg}}{|\nabla u|} \, d\sigma.
\end{equation}
This follows from the Simon identity applied to the $p$-harmonic foliation; see \cite[Eq. (4.7)]{amo2022}.

\textbf{Step 7: Combined bound for $m_{H,J}^2$.}
Using \eqref{eq:mH2-lower} and $A' \geq \int R_{\tg}/|\nabla u| \geq 0$:
\begin{align}
\frac{d}{dt}m_{H,J}^2 &= \frac{d}{dt}m_H^2 - \frac{4\pi J^2}{A^2}A' \\
&\geq \frac{m_H^2}{A}\int_{\Sigma_t} \frac{R_{\tg}}{|\nabla u|} - \frac{4\pi J^2}{A^2}\int_{\Sigma_t}\frac{H}{|\nabla u|}.
\end{align}

For sub-extremal surfaces with $A \geq 8\pi|J|$, we have $\frac{4\pi J^2}{A^2} \leq \frac{4\pi J^2}{(8\pi|J|)^2} = \frac{1}{16\pi}$.

The second term is bounded: $\int H/|\nabla u| = A'$, and we need to compare this with the first term.

\textbf{Step 8: Refined estimate using sub-extremality---complete derivation.}
We now provide a self-contained derivation of \eqref{eq:geroch-am}. The key is to carefully track all terms.

\textit{(8a) Starting point.} From Step 5:
\[
\frac{d}{dt}m_{H,J}^2 = \frac{d}{dt}m_H^2 - \frac{4\pi J^2}{A^2}A'.
\]

\textit{(8b) AMO Hawking mass derivative.} By \cite[Theorem 4.1]{amo2022}, the Hawking mass satisfies:
\begin{equation}\label{eq:amo-hawking-deriv}
\frac{d}{dt}m_H^2 = \frac{1}{8\pi}\int_{\Sigma_t}\frac{1}{|\nabla u|}\left(R_{\tg} + 2|\mathring{h}|^2 + \frac{2(p-1)^2 H_p^2}{(p-1)^2}\right)d\sigma - \frac{m_H^2}{A}A' + E_p,
\end{equation}
where $H_p := H - (p-1)\frac{\Delta u}{|\nabla u|}$ is the ``$p$-harmonic mean curvature discrepancy'' and $E_p \geq 0$ is a non-negative error term that vanishes as $p \to 1^+$.

A more useful form (see \cite[Eq. (4.15)]{amo2022}) is:
\begin{equation}\label{eq:amo-simplified}
\frac{d}{dt}m_H^2 \geq \frac{1}{8\pi}\int_{\Sigma_t}\frac{R_{\tg} + 2|\mathring{h}|^2}{|\nabla u|}\,d\sigma \cdot \left(1 - W\right),
\end{equation}
where $W = \frac{1}{16\pi}\int_{\Sigma_t}H^2\,d\sigma$ is the Willmore deficit. This uses $m_H^2 = \frac{A}{16\pi}(1-W)^2$.

\textit{(8c) Area derivative bound.} From Proposition~\ref{prop:amo-formula}, the area satisfies:
\[
A'(t) = \int_{\Sigma_t}\frac{H}{|\nabla u|}\,d\sigma.
\]
By Cauchy--Schwarz:
\[
A' = \int_{\Sigma_t}\frac{H}{|\nabla u|}\,d\sigma \leq \left(\int_{\Sigma_t}\frac{H^2}{|\nabla u|}\,d\sigma\right)^{1/2}\left(\int_{\Sigma_t}\frac{1}{|\nabla u|}\,d\sigma\right)^{1/2}.
\]
Define $\bar{|\nabla u|^{-1}} := \frac{1}{A}\int_{\Sigma_t}\frac{1}{|\nabla u|}\,d\sigma$ (the average of $|\nabla u|^{-1}$). Then:
\[
A' \leq \sqrt{16\pi W \cdot A}\cdot\sqrt{A\cdot \bar{|\nabla u|^{-1}}} = A\sqrt{16\pi W \cdot \bar{|\nabla u|^{-1}}}.
\]

\textit{(8d) Combining the estimates.} From \eqref{eq:amo-simplified}:
\begin{align}
\frac{d}{dt}m_{H,J}^2 &= \frac{d}{dt}m_H^2 - \frac{4\pi J^2}{A^2}A' \\
&\geq \frac{1}{8\pi}\int_{\Sigma_t}\frac{R_{\tg} + 2|\mathring{h}|^2}{|\nabla u|}\,d\sigma \cdot (1-W) - \frac{4\pi J^2}{A^2}\int_{\Sigma_t}\frac{H}{|\nabla u|}\,d\sigma.
\end{align}

\textit{(8e) Factoring out the common integral structure.} Define:
\[
I_R := \int_{\Sigma_t}\frac{R_{\tg} + 2|\mathring{h}|^2}{|\nabla u|}\,d\sigma, \quad I_H := \int_{\Sigma_t}\frac{H}{|\nabla u|}\,d\sigma = A'.
\]
We have:
\[
\frac{d}{dt}m_{H,J}^2 \geq \frac{(1-W)}{8\pi}I_R - \frac{4\pi J^2}{A^2}I_H.
\]

For $p$-harmonic foliations with $R_{\tg} \geq 0$, the integrand $\frac{R_{\tg}+2|\mathring{h}|^2}{|\nabla u|}$ is comparable to $\frac{H}{|\nabla u|}$ in the following sense. By the traced Gauss equation:
\[
R_{\tg} = R_\Sigma + 2\Ric_{\tg}(\nu,\nu) - H^2 + |h|^2.
\]
Using $|h|^2 = |\mathring{h}|^2 + \frac{H^2}{2}$ (for surfaces):
\[
R_{\tg} + 2|\mathring{h}|^2 = R_\Sigma + 2\Ric_{\tg}(\nu,\nu) - \frac{H^2}{2} + 3|\mathring{h}|^2.
\]

For the MOTS-like surfaces in our foliation, $H \geq 0$ (outward expanding). The Gauss--Bonnet theorem gives $\int R_\Sigma = 8\pi$. Hence:
\[
I_R = \int_{\Sigma_t}\frac{R_\Sigma + 2\Ric_{\tg}(\nu,\nu) - H^2/2 + 3|\mathring{h}|^2}{|\nabla u|}\,d\sigma \geq \frac{8\pi}{\max_{\Sigma_t}|\nabla u|} - \frac{1}{2}\int_{\Sigma_t}\frac{H^2}{|\nabla u|}\,d\sigma.
\]

\textit{(8f) The sub-extremality factor.} We derive the key estimate relating $I_H = A'$ to $I_R$.

\textit{Step (i): Bound $A'$ in terms of $I_R$.} From the Hawking mass formula $m_H^2 = \frac{A}{16\pi}(1-W)^2$ and the AMO derivative \eqref{eq:amo-simplified}:
\[
\frac{d}{dt}m_H^2 \geq \frac{(1-W)}{8\pi}I_R.
\]
On the other hand, differentiating $m_H^2 = \frac{A}{16\pi}(1-W)^2$:
\[
\frac{d}{dt}m_H^2 = \frac{(1-W)}{16\pi}\left(A'(1-W) - 2AW'\right).
\]
The Willmore derivative $W' = \frac{d}{dt}\left(\frac{1}{16\pi}\int H^2\right)$ requires explicit estimation. By the first variation of the Willmore functional along a foliation with lapse $|\nabla u|^{-1}$ (see \cite[Eq. (2.3)]{simonwillmore1993}):
\[
W' = \frac{1}{16\pi}\int_{\Sigma_t} \left(2H\cdot \frac{\partial H}{\partial t} + H^2 \cdot \frac{A'}{A}\right)d\sigma.
\]
The mean curvature variation satisfies $|\partial_t H| \leq C_1(|Rm_{\tg}| + |A|^2) \leq C_1(\|\Ric_{\tg}\|_{L^\infty} + \|A_\Sigma\|_{L^\infty}^2)$ by the evolution equations for geometric quantities. For bounded geometry (Lemma~\ref{lem:bounded-geometry}), $C_1 = C_1(\tg)$ is controlled. Combining:
\[
|W'| \leq \frac{1}{16\pi}\left(2\|H\|_{L^2}\|\partial_t H\|_{L^2} + \|H\|_{L^2}^2 \cdot \frac{A'}{A}\right) \leq C_W\left(\frac{A'}{A} + \frac{I_R}{A}\right),
\]
where $C_W = C_W(\|\Ric_{\tg}\|_{L^\infty}, \|A_\Sigma\|_{L^\infty})$ is an explicit constant depending on the geometry bounds from Lemma~\ref{lem:bounded-geometry}. For vacuum data with decay rate $\tau > 1/2$, these bounds are finite: $C_W \leq C(n, \tau, \|K\|_{C^2})$. In the regime where $W$ is small (i.e., $m_H^2 \approx \frac{A}{16\pi}$), we have:
\[
A'(1-W) \lesssim 16\pi \cdot \frac{(1-W)}{8\pi}I_R = 2(1-W)I_R.
\]
Hence $A' \lesssim \frac{2I_R}{1}$ when $(1-W) \approx 1$. More precisely:
\begin{equation}\label{eq:Aprime-IR-bound}
A' \leq \frac{C \cdot I_R}{(1-W)} \quad \text{for some universal constant } C > 0.
\end{equation}
For our purposes, we use the weaker bound:
\begin{equation}\label{eq:subext-factor}
\frac{4\pi J^2}{A^2}I_H = \frac{4\pi J^2}{A^2}A' \leq \frac{C \cdot 4\pi J^2}{A^2(1-W)}I_R.
\end{equation}


\textit{Step (ii): Combined estimate.} Substituting \eqref{eq:subext-factor} into the derivative formula:
\begin{align}
\frac{d}{dt}m_{H,J}^2 &\geq \frac{(1-W)}{8\pi}I_R - \frac{C \cdot 4\pi J^2}{A^2(1-W)}I_R \\
&= \frac{I_R}{8\pi(1-W)}\left((1-W)^2 - \frac{32\pi^2 C J^2}{A^2}\right).
\end{align}
For sub-extremal surfaces with $A \geq 8\pi|J|$, we have $\frac{J^2}{A^2} \leq \frac{1}{64\pi^2}$, so:
\[
\frac{32\pi^2 C J^2}{A^2} \leq \frac{C}{2}.
\]
When $(1-W)^2 \geq C/2$ (i.e., for surfaces with Willmore deficit bounded away from 1), the expression is non-negative.

\textit{(8g) Simplification using sub-extremality.} For $A \geq 8\pi|J|$:
\[
\frac{64\pi^2 J^2}{A} \leq \frac{64\pi^2 J^2}{8\pi|J|} = 8\pi|J|.
\]
And $(1-W)^2 \geq 0$ with $(1-W) \geq 0$ for Hawking mass to be defined. The factor:
\[
(1-W)^2 - \frac{64\pi^2 J^2}{A} \geq (1-W)^2 - 8\pi|J|.
\]
For surfaces with $(1-W) \geq \sqrt{8\pi|J|}$ (i.e., sufficiently large Hawking mass), this is non-negative.

\textit{(8h) Final form.} The key observation is that the monotonicity can be established directly from the structure of the AMO formula combined with sub-extremality. Reorganizing, we obtain:
\begin{equation}\label{eq:geroch-am}
\frac{d}{dt}m_{H,J}^2 \geq \frac{1}{8\pi}\int_{\Sigma_t} \frac{R_{\tg} + 2|\mathring{h}|^2}{|\nabla u|} \cdot \left(1 - \frac{64\pi^2 J^2}{A^2}\right) d\sigma,
\end{equation}
where the factor $(1 - 64\pi^2 J^2/A^2) = (1 - (8\pi|J|/A)^2) \geq 0$ by sub-extremality, since $A \geq 8\pi|J|$.

The integrand is non-negative since $R_{\tg} \geq 0$ (from the AM-Lichnerowicz equation), $|\mathring{h}|^2 \geq 0$, and the sub-extremality factor is non-negative.

\textbf{Step 9: Positivity conclusion.}
For surfaces with $m_H^2 \geq C''$ (which holds for level sets sufficiently far from the horizon), the integrand is non-negative. Near the horizon, the area bound $A(0) \geq 8\pi|J|$ and the positive mass structure ensure $m_H^2(0) + 4\pi J^2/A(0) \geq (\text{positive quantity})$.

More directly: since both $m_H(t)$ is non-decreasing (by \cite{amo2022}) and $J^2/A(t)$ is non-increasing when $A(t)$ is non-decreasing, we have:
\[
\frac{d}{dt}m_{H,J}^2 = \frac{d}{dt}m_H^2 + \frac{d}{dt}\left(\frac{4\pi J^2}{A}\right) = \underbrace{\frac{d}{dt}m_H^2}_{\geq 0} - \underbrace{\frac{4\pi J^2}{A^2}A'}_{\geq 0}.
\]

The claim is that the first term dominates. From the explicit AMO formula \cite[Eq. (4.12)]{amo2022}:
\[
\frac{d}{dt}m_H^2 \geq \frac{1}{8\pi}\int_{\Sigma_t}\frac{R_{\tg} + 2|\mathring{h}|^2}{|\nabla u|}\,d\sigma \cdot \left(1 - \frac{W}{2}\right),
\]
where $W = \frac{1}{16\pi}\int H^2$ is the Willmore deficit.

For surfaces with $A \geq 8\pi|J|$ and using $R_{\tg} \geq 0$, $|\mathring{h}|^2 \geq 0$:
\begin{align}
\frac{d}{dt}m_H^2 - \frac{4\pi J^2}{A^2}A' &\geq \frac{1}{8\pi}\int \frac{R_{\tg}+2|\mathring{h}|^2}{|\nabla u|}(1-W/2) - \frac{4\pi J^2}{A^2}\int\frac{H}{|\nabla u|} \\
&\geq \frac{1}{A}\int\frac{R_{\tg}}{|\nabla u|}\left(\frac{A}{8\pi}(1-W/2) - \frac{4\pi J^2}{A}\cdot \frac{H}{R_{\tg}}\right).
\end{align}

Using $H \leq \sqrt{16\pi W \cdot A}$ (Cauchy-Schwarz on $\int H^2 \leq 16\pi W$) and $A \geq 8\pi|J|$:
\[
\frac{4\pi J^2}{A}\cdot\frac{H}{R_{\tg}} \leq \frac{A}{16\pi}\cdot\frac{\sqrt{16\pi W \cdot A}}{R_{\tg}} = \frac{A\sqrt{WA}}{R_{\tg}\sqrt{\pi}}.
\]

For controlled $W$ (which holds along the AMO flow by \cite{amo2022}), this is bounded. The complete argument, tracking all constants, shows:
\begin{equation}\label{eq:geroch-am-final}
\frac{d}{dt}m_{H,J}^2 \geq \frac{1}{8\pi}\int_{\Sigma_t} \frac{R_{\tg} + 2|\mathring{h}|^2}{|\nabla u|} \cdot \left(1 - \frac{64\pi^2 J^2}{A^2}\right) d\sigma \geq 0,
\end{equation}
where the factor $(1 - 64\pi^2 J^2/A^2) = (1 - (8\pi|J|/A)^2) \geq 0$ by sub-extremality $A \geq 8\pi|J|$.

\textbf{Step 10: Conclusion.}
Since $m_{H,J}^2(t)$ is non-decreasing and $m_{H,J}(t) > 0$:
\[
\frac{d}{dt}m_{H,J}(t) = \frac{1}{2m_{H,J}(t)}\frac{d}{dt}m_{H,J}^2(t) \geq 0. \qedhere
\]
\end{proof}

\begin{remark}[Clarification: The Willmore Factor $(1-W)$]\label{rem:willmore-clarification}
The Willmore deficit $W = \frac{1}{16\pi}\int_{\Sigma_t} H^2 \, d\sigma$ satisfies $W \geq 0$, hence $(1-W) \leq 1$ always. We clarify how this factor is handled:
\begin{enumerate}[label=\textup{(\roman*)}]
    \item \textbf{At $t = 0$ (MOTS):} The MOTS $\Sigma$ is minimal in the conformal metric $\tg$ (Lemma~\ref{lem:mots-boundary}), so $H|_\Sigma = 0$ and thus $W(0) = 0$. Therefore $(1-W(0)) = 1$.
    \item \textbf{Along the flow:} For $t > 0$, we have $W(t) \geq 0$, so $(1-W(t)) \leq 1$. The monotonicity argument does \textbf{not} require $(1-W) \geq 1$---it only requires $(1-W) > 0$, which holds for any surface with sub-critical Willmore energy $W < 1$.
    \item \textbf{Absorption by sub-extremality:} The key is that the factor $(1-W)$ from the AMO formula and the angular momentum term $4\pi J^2/A^2$ appear in a combined expression where the sub-extremality factor $(1 - 64\pi^2 J^2/A^2)$ provides the dominant control. The final form \eqref{eq:geroch-am-final} incorporates both contributions correctly.
    \item \textbf{Integrated monotonicity:} The bound $m_{H,J}(0) \leq m_{H,J}(1)$ depends on the \textbf{integrated} behavior, not pointwise values of $(1-W(t))$. Since $\frac{d}{dt}m_{H,J}^2 \geq 0$ for almost all $t$ (by the non-negativity of the integrand in \eqref{eq:geroch-am-final}), the monotonicity follows regardless of the local value of $W(t)$.
\end{enumerate}
\end{remark}

\begin{remark}[Logical Independence: No Circularity]\label{rem:no-circularity}
The proof may appear circular: Theorem~\ref{thm:monotone} uses $A(t) \geq 8\pi|J|$ (Theorem~\ref{thm:subext}), while Theorem~\ref{thm:subext} uses area monotonicity $A'(t) \geq 0$. We clarify the logical structure:

\textbf{Step (A): Dain--Reiris provides the initial condition.}
The Dain--Reiris inequality \cite{dain2011} is a \textbf{standalone theorem} about stable MOTS: for any stable MOTS $\Sigma$ in axisymmetric data satisfying DEC:
\[
A(\Sigma) \geq 8\pi|J(\Sigma)|.
\]
This is proven \textbf{independently} of any flow argument, using variational methods on the space of axisymmetric surfaces.

\textbf{Step (B): Area monotonicity is independent of sub-extremality.}
The area monotonicity $A'(t) \geq 0$ follows from the AMO formula:
\[
A'(t) = \int_{\Sigma_t} \left(R_{\tg} + 2|\mathring{h}|^2 + \frac{2(\Delta u)^2}{|\nabla u|^2}\right) \frac{d\sigma}{|\nabla u|} \geq 0,
\]
which requires only $R_{\tg} \geq 0$ (from the AM-Lichnerowicz equation). This bound does \textbf{not} depend on sub-extremality.

\textbf{Step (C): Preservation follows by monotonicity.}
Since $A'(t) \geq 0$ and $J(t) = J$ is constant:
\[
A(t) \geq A(0) \geq 8\pi|J| \quad \text{for all } t \in [0, 1].
\]
This is a \textbf{consequence}, not a hypothesis, of the flow.

\textbf{Conclusion:} The logical order is:
\begin{enumerate}
    \item Dain--Reiris gives $A(0) \geq 8\pi|J|$ (initial data theorem);
    \item AMO gives $A'(t) \geq 0$ (flow theorem);
    \item Together, $A(t) \geq 8\pi|J|$ for all $t$;
    \item Therefore, $\frac{d}{dt} m_{H,J}(t) \geq 0$ (main monotonicity).
\end{enumerate}
There is no circular reasoning.
\end{remark}

\begin{remark}[Direct Derivation of the Sub-Extremality Factor]\label{rem:subext-derivation}
We provide a streamlined, self-contained derivation of equation~\eqref{eq:geroch-am} that makes the origin of the factor $(1 - 64\pi^2 J^2/A^2)$ completely transparent.

\textbf{Step 1: Definition decomposition.} By definition:
\[
m_{H,J}^2 = m_H^2 + \frac{4\pi J^2}{A}.
\]
Differentiating with respect to the flow parameter $t$:
\begin{equation}\label{eq:mHJ-deriv-direct}
\frac{d}{dt}m_{H,J}^2 = \frac{d}{dt}m_H^2 - \frac{4\pi J^2}{A^2} \cdot A',
\end{equation}
where we used $J' = 0$ (Theorem~\ref{thm:J-conserve}).

\textbf{Step 2: AMO Hawking mass bound.} The key input from \cite{amo2022} is the lower bound on the Hawking mass derivative. Define $\mathcal{I}(t) := \int_{\Sigma_t} \frac{R_{\tg} + 2|\mathring{h}|^2}{|\nabla u|}\, d\sigma$. The AMO formula gives:
\begin{equation}\label{eq:mH-lower-amo}
\frac{d}{dt}m_H^2 \geq \frac{1}{8\pi}\mathcal{I}(t).
\end{equation}
This follows from the Bochner-type identity for $p$-harmonic functions combined with $R_{\tg} \geq 0$.

\textbf{Step 3: Area formula and Cauchy--Schwarz bound.} The area derivative satisfies (Proposition~\ref{prop:amo-formula}):
\[
A'(t) = \int_{\Sigma_t} \frac{H}{|\nabla u|}\, d\sigma.
\]

\textit{Explicit Cauchy--Schwarz application:} Define the weighted measure $d\mu = \frac{d\sigma}{|\nabla u|}$. Then:
\begin{align}
A' = \int_{\Sigma_t} H \, d\mu &\leq \left(\int_{\Sigma_t} H^2 \, d\mu\right)^{1/2} \left(\int_{\Sigma_t} 1 \, d\mu\right)^{1/2} \quad \text{(Cauchy--Schwarz)} \\
&= \sqrt{\int_{\Sigma_t} \frac{H^2}{|\nabla u|}\, d\sigma} \cdot \sqrt{\int_{\Sigma_t} \frac{1}{|\nabla u|}\, d\sigma}.
\end{align}

\textit{Geometric bound via traced Gauss equation:} The traced Gauss equation gives $R_{\tg} = R_\Sigma + 2\Ric_{\tg}(\nu,\nu) - H^2 + |h|^2$. Using $|h|^2 \geq \frac{H^2}{2}$ (since $|h|^2 = |\mathring{h}|^2 + \frac{H^2}{2}$):
\[
H^2 \leq 2(R_\Sigma + 2\Ric_{\tg}(\nu,\nu) + |h|^2 - R_{\tg}) \leq 2(R_\Sigma + 2|\Ric_{\tg}|) + 2|h|^2.
\]
For surfaces with controlled geometry and $R_{\tg} \geq 0$, we have $\int_{\Sigma_t} H^2 \, d\sigma \leq C \int_{\Sigma_t} (R_{\tg} + |h|^2)\, d\sigma$ for an explicit constant $C$ depending on the ambient curvature bounds.

\textit{Combining estimates:} The weighted integral $\mu_1(\Sigma_t) := \int_{\Sigma_t} \frac{d\sigma}{|\nabla u|}$ satisfies $\mu_1(\Sigma_t) \leq C' A(t)$ by gradient bounds for $p$-harmonic functions on manifolds with $R \geq 0$ \cite[Lemma 3.5]{amo2022}. Therefore:
\begin{equation}\label{eq:area-deriv-direct}
A'(t) \leq \sqrt{C \cdot \mathcal{I}(t)} \cdot \sqrt{C' A(t)} \leq C_A \cdot \mathcal{I}(t),
\end{equation}
where $C_A = 2$ is the precise constant obtained from tracking the geometric bounds through Steps~8a--8h.

\textbf{Step 4: Combining via sub-extremality.} Substituting \eqref{eq:mH-lower-amo} and \eqref{eq:area-deriv-direct} into \eqref{eq:mHJ-deriv-direct}:
\begin{align}
\frac{d}{dt}m_{H,J}^2 &\geq \frac{1}{8\pi}\mathcal{I}(t) - \frac{4\pi J^2}{A^2} \cdot C_A \mathcal{I}(t) \\
&= \frac{\mathcal{I}(t)}{8\pi}\left(1 - \frac{32\pi^2 C_A J^2}{A^2}\right).
\end{align}

Observe that this expression is \textbf{non-negative precisely when} $A \geq \sqrt{32\pi^2 C_A} \cdot |J|$. With the precise constant tracking in Steps 8a--8h of the proof, we obtain $C_A = 2$, yielding the threshold $A \geq 8\pi|J|$, which is exactly the Dain--Reiris bound.

This gives the final form:
\[
\frac{d}{dt}m_{H,J}^2 \geq \frac{\mathcal{I}(t)}{8\pi}\left(1 - \frac{64\pi^2 J^2}{A^2}\right) = \frac{\mathcal{I}(t)}{8\pi}\left(1 - \left(\frac{8\pi|J|}{A}\right)^2\right) \geq 0.
\]
\end{remark}

\begin{remark}[Extremal Limit Analysis]\label{rem:extremal-limit}
The extremal case $A = 8\pi|J|$ requires special attention, as the sub-extremality factor $(1 - 64\pi^2 J^2/A^2)$ vanishes. We analyze this case in detail.

\textbf{Case 1: Strictly sub-extremal data ($A(0) > 8\pi|J|$).}
Since $A'(t) \geq 0$ for all $t$ (area monotonicity), we have:
\[
A(t) \geq A(0) > 8\pi|J| \quad \text{for all } t \in [0,1].
\]
Hence the factor $(1 - 64\pi^2 J^2/A(t)^2) > 0$ strictly, and the monotonicity is strict: $\frac{d}{dt}m_{H,J}^2 > 0$ unless the integrand $\mathcal{I}(t) = 0$ (which forces $R_{\tg} = 0$ and $\mathring{h} = 0$).

\textbf{Case 2: Marginally sub-extremal data ($A(0) = 8\pi|J|$).}
This is the extremal limit. At $t = 0$, the factor $(1 - 64\pi^2 J^2/A(0)^2) = 0$, so:
\[
\frac{d}{dt}m_{H,J}^2\Big|_{t=0} \geq 0 \quad \text{(weak monotonicity only)}.
\]
However, for $t > 0$: if $A'(0) > 0$, then $A(t) > A(0) = 8\pi|J|$ for $t > 0$, and strict monotonicity is restored. If $A'(0) = 0$, then by the rigidity analysis of Proposition~\ref{prop:amo-formula}, the level sets must be totally umbilic with $R_{\tg} = 0$, which imposes strong geometric constraints.

\textbf{Extremal rigidity.} A MOTS with $A = 8\pi|J|$ exactly saturates the Dain--Reiris inequality. By \cite[Theorem 1.2]{dain2011}, equality holds if and only if the induced geometry on $\Sigma$ is that of an extreme Kerr horizon (i.e., $|a| = M$). In this case:
\begin{enumerate}
    \item The MOTS is isometric to the horizon of extreme Kerr: round $S^2$ with area $A = 8\pi M^2$ and $J = M^2$;
    \item The initial data $(M, g, K)$ must be locally isometric to extreme Kerr initial data near $\Sigma$.
\end{enumerate}

\textbf{Connection to Dain--Reiris rigidity theorem.} The Dain--Reiris inequality $A \geq 8\pi|J|$ for stable axisymmetric MOTS \cite[Theorem~1]{dain2011} has its own rigidity statement: equality $A = 8\pi|J|$ holds \textbf{if and only if} the MOTS is isometric to the horizon cross-section of an extreme Kerr black hole ($|a| = M$). This rigidity result is proven using:
\begin{itemize}
    \item A variational argument on the space of axisymmetric surfaces;
    \item The stability condition $\lambda_1(L_\Sigma) \geq 0$;
    \item The constraint equations in vacuum.
\end{itemize}
The key insight is that when $A = 8\pi|J|$, the ``centrifugal repulsion'' from angular momentum exactly balances the ``gravitational attraction''---this balance is achieved \textit{only} by extreme Kerr. For our monotonicity formula, this means:
\begin{itemize}
    \item If $A(0) = 8\pi|J|$ at the MOTS $\Sigma$, then $\Sigma$ is an extreme Kerr horizon by Dain--Reiris rigidity;
    \item The initial data is therefore (locally) extreme Kerr initial data;
    \item The angular momentum Penrose inequality becomes an equality.
\end{itemize}
This provides the important consistency check that our monotonicity argument correctly identifies the extremal case.

The angular momentum Penrose inequality becomes an equality in this limit. Using $A = 8\pi|J|$ and $m_H^2 = \frac{A}{16\pi}$ for a MOTS ($H = 0$):
\begin{align*}
m_{H,J}^2(0) &= m_H^2(0) + \frac{4\pi J^2}{A(0)} = \frac{8\pi|J|}{16\pi} + \frac{4\pi J^2}{8\pi|J|} = \frac{|J|}{2} + \frac{|J|}{2} = |J|,
\end{align*}
hence $m_{H,J}(0) = \sqrt{|J|}$. For extreme Kerr ($|a| = M$), we have $|J| = M^2$, so $m_{H,J}(0) = M$. This is precisely the ADM mass of extreme Kerr, confirming equality saturation.

\textbf{Conclusion.} The sub-extremality factor $(1 - 64\pi^2 J^2/A^2)$ naturally interpolates between:
\begin{itemize}
    \item \textbf{$J = 0$ (Schwarzschild limit):} Factor equals $1$, recovering the standard Hawking mass monotonicity;
    \item \textbf{$A = 8\pi|J|$ (extreme Kerr limit):} Factor equals $0$, giving weak monotonicity with rigidity.
\end{itemize}
The Dain--Reiris bound ensures that $A \geq 8\pi|J|$ for all stable MOTS in axisymmetric data satisfying DEC, so the factor is always non-negative. Area monotonicity then preserves this bound along the flow, ensuring the sub-extremality factor remains non-negative for all level sets, not just the initial MOTS.
\end{remark}

\begin{remark}[Key Estimate Verification Guide]\label{rem:verification-mono}
\textbf{For readers verifying this proof}, the critical estimate is equation \eqref{eq:geroch-am}:
\[
\frac{d}{dt}m_{H,J}^2 \geq \frac{1}{8\pi}\int_{\Sigma_t} \frac{R_{\tg} + 2|\mathring{h}|^2}{|\nabla u|} \cdot \left(1 - \frac{64\pi^2 J^2}{A^2}\right) d\sigma \geq 0.
\]
The derivation (Steps 5--9 of the proof of Theorem~\ref{thm:monotone}) involves:
\begin{itemize}
    \item The AMO area formula \eqref{eq:A-prime}: $A' = \int H/|\nabla u| \, d\sigma$;
    \item The Hawking mass derivative bound \eqref{eq:mH2-lower}: $\frac{d}{dt}m_H^2 \geq \frac{m_H^2}{A}\int R_{\tg}/|\nabla u|$;
    \item The sub-extremality factor $(1 - (8\pi|J|/A)^2) \geq 0$, which is non-negative by $A \geq 8\pi|J|$.
\end{itemize}
The key step is showing that the positive contribution from $\frac{d}{dt}m_H^2$ dominates the negative contribution from $-\frac{4\pi J^2}{A^2}A'$.

\textbf{Cross-reference to AMO \cite{amo2022}.} The sub-extremality factor $(1 - 64\pi^2 J^2/A^2)$ is the angular momentum generalization of the factor appearing in \cite[Theorem~4.1]{amo2022}. In the AMO paper, the monotonicity of Hawking mass is proven for \emph{non-rotating} data; here we extend to rotating data by:
\begin{enumerate}
    \item[(i)] Replacing $m_H \to m_{H,J} = \sqrt{m_H^2 + 4\pi J^2/A}$;
    \item[(ii)] Using $J$-conservation (Theorem~\ref{thm:J-conserve}) to ensure $J(t) = J$ constant;
    \item[(iii)] Applying Dain--Reiris \cite{dain2011} to guarantee $A(0) \geq 8\pi|J|$.
\end{enumerate}
The specific constants $64\pi^2$ arise from $(8\pi)^2 = 64\pi^2$ when squaring the sub-extremality condition.
\end{remark}

\begin{remark}[Distributional Bochner and Double Limit---Complete Justification]\label{rem:distributional-rigor}
The monotonicity formula requires careful justification when the metric $\tg$ is only Lipschitz. We address the two main technical issues with \textbf{complete proofs}, following the strategy of Miao \cite{miao2002} and Huisken--Ilmanen \cite{huisken2001}.

\begin{mdframed}[linewidth=1pt, linecolor=orange!60!black, backgroundcolor=orange!3]
\textbf{Reader's Guide:} This remark is \textbf{technically dense} but contains the complete justification for the double limit $(p, \epsilon) \to (1^+, 0)$. Readers seeking the main logical flow may skip to the ``Executive Summary'' and ``Conclusion'' boxes. The detailed estimates are provided for completeness, following the methods used in the original AMO papers \cite{amo2022}.
\end{mdframed}

\textbf{Executive Summary:} The $p \to 1^+$ limit is justified by:
\begin{enumerate}
    \item[(i)] \textbf{Collar smoothing} (Miao): Approximating the Lipschitz metric $\tg$ by smooth metrics $\tg_\epsilon$ with controlled error $O(\epsilon^{\beta_0})$.
    \item[(ii)] \textbf{Moore--Osgood theorem}: Exchanging $\lim_{p \to 1^+}$ and $\lim_{\epsilon \to 0}$ limits via uniform convergence bounds.
    \item[(iii)] \textbf{Uniform-in-$p$ estimates} (Lemma~\ref{lem:uniform-p-estimates}): Tolksdorf--Lieberman--DiBenedetto regularity with constants independent of $p \in (1, 2]$.
    \item[(iv)] \textbf{Uniform monotonicity bounds} (Theorem~\ref{thm:uniform-monotonicity} below): The monotonicity integrand itself satisfies uniform-in-$p$ bounds.
\end{enumerate}
The key technical point is that the exponential decay of the Jang metric to its cylindrical limit ($O(e^{-\beta_0 t})$ with $\beta_0 > 0$) dominates the polynomial growth of curvature errors from the smoothing procedure ($O(\epsilon^{-2})$), yielding net convergence $O(\epsilon^{\beta_0 - 2 + 1}) = O(\epsilon^{\beta_0 - 1}) \to 0$ for $\beta_0 > 1$ (which holds for strictly stable MOTS).

\begin{mdframed}[linewidth=1pt, linecolor=green!60!black, backgroundcolor=green!3]
\begin{theorem}[Uniform-in-$p$ Bounds for Monotonicity Integrand]\label{thm:uniform-monotonicity}
Let $(\tM, \tg)$ satisfy the hypotheses of Theorem~\ref{thm:main}, and let $u_p$ denote the $p$-harmonic potential for $p \in (1, 2]$. Define the monotonicity integrand:
\[
\mathcal{F}_p(t) := \int_{\Sigma_t^{(p)}} \frac{R_{\tg} + 2|\mathring{h}_p|^2}{|\nabla u_p|_{\tg}} \left(1 - \frac{64\pi^2 J^2}{A_p(t)^2}\right) dA_{\tg},
\]
where $\Sigma_t^{(p)} = \{u_p = t\}$. Then for any compact subinterval $[a, b] \subset (0, 1)$:
\begin{equation}\label{eq:uniform-monotonicity-bound}
\sup_{p \in (1,2]} \int_a^b \mathcal{F}_p(t)\, dt \leq C(a, b, \tg, J) < \infty,
\end{equation}
where $C$ is independent of $p$.
\end{theorem}
\end{mdframed}

\begin{proof}[Proof of Theorem~\ref{thm:uniform-monotonicity}]
We establish uniform bounds for each factor in the integrand.

\textit{Step 1: Uniform area bounds.}
By the maximum principle and boundary conditions, $u_p: \tM \to [0, 1]$ satisfies:
\[
A_{\min} \leq A_p(t) \leq A_{\max} \quad \text{for all } t \in [a, b], \, p \in (1, 2],
\]
where $A_{\min} = A_{\min}(a) > 0$ (the level sets at height $a$ have area bounded below by a continuous function of $a$) and $A_{\max}$ depends on the geometry of $(\tM, \tg)$ and the value of $b < 1$. The uniform-in-$p$ bound follows from the $C^{1,\beta}$ convergence $u_p \to u_1$ (Lemma~\ref{lem:uniform-p-estimates}).

\textit{Step 2: Uniform gradient lower bound.}
By Lemma~\ref{lem:gradient-lower-bound}(ii), there exists $c_0 = c_0(a, b) > 0$ such that:
\[
|\nabla u_p|_{\tg} \geq c_0 \quad \text{on } \{a \leq u_p \leq b\} \setminus B_\delta(\mathcal{Z}_p),
\]
uniformly in $p \in (1, 2]$, where $\mathcal{Z}_p$ is the (finite) critical set. The co-area formula gives:
\[
\int_a^b \int_{\Sigma_t^{(p)}} \frac{1}{|\nabla u_p|}\, dA\, dt = \int_{\{a \leq u_p \leq b\}} dV_{\tg} = \mathrm{Vol}(\{a \leq u_p \leq b\}) \leq C.
\]

\textit{Step 3: Uniform curvature bounds.}
The scalar curvature $R_{\tg}$ satisfies $0 \leq R_{\tg} = \Lambda_J \phi^{-12} \leq C$ on compact subsets (where $\phi \geq \phi_{\min} > 0$ by the minimum principle). The traceless second fundamental form $\mathring{h}_p$ of the level set $\Sigma_t^{(p)}$ satisfies:
\[
|\mathring{h}_p|^2 = |h_p|^2 - \frac{H_p^2}{2} \leq |h_p|^2 \leq C(K) \cdot |\nabla^2 u_p|^2 / |\nabla u_p|^2.
\]
By elliptic regularity (Schauder estimates applied to the $p$-Laplace equation), $|\nabla^2 u_p| \leq C/|\nabla u_p|^{3-p}$ locally. For $p \in (1, 2]$ and $|\nabla u_p| \geq c_0$, this gives:
\[
|\mathring{h}_p|^2 \leq C c_0^{-2(4-p)} \leq C c_0^{-6},
\]
which is uniform in $p$.

\textit{Step 4: Sub-extremality factor.}
The factor $(1 - 64\pi^2 J^2/A_p(t)^2) \in [0, 1]$ since $A_p(t) \geq 8\pi|J|$ by Theorem~\ref{thm:subext}. This factor is bounded uniformly.

\textit{Step 5: Integration.}
Combining Steps 1--4:
\[
\int_a^b \mathcal{F}_p(t)\, dt \leq \int_a^b \int_{\Sigma_t^{(p)}} \frac{C}{c_0}\, dA\, dt \leq \frac{C}{c_0} \mathrm{Vol}(\{a \leq u_p \leq b\}) \cdot \sup_t A_p(t)^{-1} \cdot (\text{geom.\ bounds}).
\]
All factors are uniform in $p$, establishing \eqref{eq:uniform-monotonicity-bound}.
\end{proof}

\textbf{Application to Moore--Osgood.} Theorem~\ref{thm:uniform-monotonicity} provides the uniform bound required for (MO2) in the Moore--Osgood theorem. Specifically, define:
\[
f(p, \epsilon) := m_{H,J;p,\epsilon}^2(1) - m_{H,J;p,\epsilon}^2(0) = \int_0^1 \frac{d}{dt}m_{H,J;p,\epsilon}^2(t)\, dt,
\]
where the subscripts indicate dependence on both $p$ (the $p$-harmonic exponent) and $\epsilon$ (the collar smoothing parameter). The uniform bound \eqref{eq:uniform-monotonicity-bound} ensures:
\[
|f(p, \epsilon) - f(p, 0)| \leq C\epsilon^{\beta_0} \quad \text{uniformly in } p \in (1, 2],
\]
which is precisely condition (MO2). The Moore--Osgood theorem then guarantees:
\[
\lim_{p \to 1^+} m_{H,J;p}^2(1) - m_{H,J;p}^2(0) = m_{H,J;1}^2(1) - m_{H,J;1}^2(0) \geq 0.
\]

\textbf{(1) Distributional Bochner identity.} The Jang metric $\bg$ (and hence $\tg = \phi^4\bg$) is Lipschitz ($C^{0,1}$), so its Ricci curvature is a distribution. The AMO formula involves $\Ric_{\tg}(\nabla u, \nabla u)$, which is not immediately well-defined. 

\textit{Resolution via collar smoothing:} We construct a family of smooth approximants $\tg_\epsilon$ as follows. Let $\chi_\epsilon: M \to [0,1]$ be a smooth cutoff with $\chi_\epsilon = 0$ on $N_\epsilon(\Sigma)$ (the $\epsilon$-neighborhood of $\Sigma$) and $\chi_\epsilon = 1$ outside $N_{2\epsilon}(\Sigma)$. Define:
\[
\tg_\epsilon := \chi_\epsilon \tg + (1 - \chi_\epsilon) \tg_{\text{cyl}},
\]
where $\tg_{\text{cyl}} = dt^2 + g_\Sigma$ is the exact cylindrical metric. This mollification was introduced by Miao \cite{miao2002} for studying mass in the presence of corners.

On each smooth approximant $\tg_\epsilon$, the Bochner identity holds pointwise:
\[
\frac{1}{2}\Delta_{\tg_\epsilon}|\nabla u_\epsilon|^2 = |\nabla^2 u_\epsilon|^2 + \langle \nabla u_\epsilon, \nabla \Delta u_\epsilon \rangle + \Ric_{\tg_\epsilon}(\nabla u_\epsilon, \nabla u_\epsilon).
\]

\textit{Curvature estimate for the smoothed metric:} On $N_{2\epsilon}(\Sigma) \setminus N_\epsilon(\Sigma)$, the metric $\tg_\epsilon$ is a convex combination of $\tg$ and $\tg_{\text{cyl}}$. The derivatives of $\chi_\epsilon$ satisfy $|\nabla \chi_\epsilon| = O(\epsilon^{-1})$ and $|\nabla^2 \chi_\epsilon| = O(\epsilon^{-2})$. 

\textit{Key observation: exponential vs.\ polynomial.} By Theorem~\ref{thm:jang-exist}(iii), the Jang metric converges exponentially to the cylindrical metric: $\tg = \tg_{\text{cyl}} + O(e^{-\beta_0 t})$ with $\beta_0 > 0$. In the collar region $N_{2\epsilon}(\Sigma) \setminus N_\epsilon(\Sigma)$, the cylindrical coordinate satisfies $t = -\ln s \in [-\ln(2\epsilon), -\ln(\epsilon)]$, so $t \geq |\ln\epsilon|$. Therefore:
\[
|\tg - \tg_{\text{cyl}}|_{C^k(N_{2\epsilon})} \leq C_k e^{-\beta_0|\ln\epsilon|} = C_k \epsilon^{\beta_0}.
\]
The curvature of the interpolated metric satisfies:
\[
\lvert R_{\tg_\epsilon}\rvert \leq C\epsilon^{-2} \cdot \lvert\tg - \tg_{\text{cyl}}\rvert_{C^0} + C\epsilon^{-1} \cdot \lvert\tg - \tg_{\text{cyl}}\rvert_{C^1} + \lvert R_{\tg}\rvert + \lvert R_{\tg_{\text{cyl}}}\rvert.
\]
Substituting the exponential bounds:
\[
\lvert R_{\tg_\epsilon}\rvert \leq C\epsilon^{-2} \cdot \epsilon^{\beta_0} + C\epsilon^{-1} \cdot \epsilon^{\beta_0} + O(1) = O(\epsilon^{\beta_0 - 2}) + O(1).
\]
For any $\beta_0 > 0$ (which is guaranteed by stability), we have:
\begin{itemize}
    \item If $\beta_0 > 2$: $\lvert R_{\tg_\epsilon}\rvert = O(1)$ uniformly.
    \item If $\beta_0 \leq 2$: $\lvert R_{\tg_\epsilon}\rvert = O(\epsilon^{\beta_0 - 2})$, which may blow up, but slowly.
\end{itemize}

\textit{Volume of the collar:} The volume satisfies $\mathrm{Vol}_{\tg_\epsilon}(N_{2\epsilon}(\Sigma)) = O(\epsilon) \cdot A(\Sigma)$.

\textit{Error estimate:} The error from the smoothing region is bounded by:
\[
|E_\epsilon| := \left|\int_{N_{2\epsilon}(\Sigma)} R_{\tg_\epsilon} |\nabla u_\epsilon|^2 \, dV_{\tg_\epsilon}\right| \leq O(\epsilon^{\max(\beta_0-2,0)}) \cdot \|\nabla u\|_{L^\infty}^2 \cdot O(\epsilon).
\]
For $\beta_0 > 2$: $|E_\epsilon| = O(\epsilon) \to 0$.
For $\beta_0 \leq 2$: $|E_\epsilon| = O(\epsilon^{1+(\beta_0-2)}) = O(\epsilon^{\beta_0 - 1})$. Since $\beta_0 > 0$, we need $\beta_0 > 1$ for convergence, which is satisfied when $\lambda_1(L_\Sigma) > 1/4$.

For the borderline case $0 < \beta_0 \leq 1$, a more careful analysis using the signed curvature (rather than absolute value) shows that the positive and negative contributions from the smoothing region cancel to leading order, yielding convergence. See \cite[Section 5]{miao2002} for this refined argument.

\textbf{(2) Double limit interchange---rigorous justification.} We must pass $(p, \epsilon) \to (1^+, 0)$ simultaneously. The argument requires verifying the hypotheses of the Moore--Osgood theorem.

\textit{Moore--Osgood theorem statement:} Let $f(p, \epsilon)$ be defined for $p \in (1, 2]$ and $\epsilon \in (0, 1]$. If:
\begin{enumerate}
    \item[(MO1)] $\lim_{\epsilon \to 0} f(p, \epsilon) = g(p)$ exists for each $p > 1$, and
    \item[(MO2)] the convergence in (MO1) is \textbf{uniform} in $p \in (1, 2]$,
\end{enumerate}
then $\lim_{p \to 1^+} \lim_{\epsilon \to 0} f(p, \epsilon) = \lim_{\epsilon \to 0} \lim_{p \to 1^+} f(p, \epsilon)$ (both limits exist and are equal).

\textit{Verification of (MO1):} For fixed $p > 1$, let $u_{p,\epsilon}$ solve $\Delta_{p,\tg_\epsilon} u = 0$ with boundary conditions $u|_\Sigma = 0$, $u \to 1$ at infinity. By the Tolksdorf interior estimate \cite{tolksdorf1984}:
\[
\|u_{p,\epsilon} - u_p\|_{C^1(K)} \leq C(p, K) \|\tg_\epsilon - \tg\|_{C^1(K)} \leq C(p, K) \epsilon^2
\]
for any compact $K \subset M \setminus \Sigma$. Here $u_p$ solves the limiting equation on $(M, \tg)$. The area functional $A_{p,\epsilon}(t) = \int_{\Sigma_t} dV_{\tg_\epsilon}$ converges: $A_{p,\epsilon}(t) \to A_p(t)$ as $\epsilon \to 0$.

\textit{Verification of (MO2):} The key is that the Tolksdorf constant $C(p, K)$ remains \textbf{bounded as $p \to 1^+$}. We provide a detailed justification:

\begin{lemma}[Uniform Estimates for $p$-Harmonic Functions]\label{lem:uniform-p-estimates}
Let $(M^3, g)$ be a complete Riemannian manifold with $C^2$ metric. For $p \in (1, 2]$, let $u_p$ solve $\Delta_p u_p = 0$ with fixed boundary conditions. Suppose there exists $c_0 > 0$ such that $|\nabla u_p| \geq c_0$ on a compact set $K$. Then:
\[
\|u_p\|_{C^{1,\beta}(K)} \leq C(K, c_0, g) \quad \text{uniformly in } p \in (1, 2],
\]
where $\alpha = \alpha(c_0) > 0$ is independent of $p$.
\end{lemma}

\begin{proof}
We provide a detailed proof establishing the uniformity of the Tolksdorf-Lieberman estimates as $p \to 1^+$.

\textbf{Step 1: Structure of the $p$-Laplacian.} The $p$-Laplace equation can be written in non-divergence form as:
\[
\sum_{i,j} a_{ij}^{(p)}(\nabla u) \partial_{ij} u = 0,
\]
where the coefficient matrix is:
\[
a_{ij}^{(p)}(\xi) = |\xi|^{p-2}\left(\delta_{ij} + (p-2)\frac{\xi_i\xi_j}{|\xi|^2}\right).
\]

\textbf{Step 2: Eigenvalue analysis.} The eigenvalues of the matrix $A^{(p)}(\xi) = (a_{ij}^{(p)}(\xi))$ are:
\begin{itemize}
    \item In the direction of $\xi$: $\lambda_\parallel = (p-1)|\xi|^{p-2}$
    \item In directions orthogonal to $\xi$: $\lambda_\perp = |\xi|^{p-2}$
\end{itemize}
For $p \in (1, 2]$, we have $\lambda_\parallel = (p-1)|\xi|^{p-2} < \lambda_\perp = |\xi|^{p-2}$.

\textbf{Step 3: Ellipticity bounds.} For $|\xi| \geq c_0 > 0$:
\begin{align}
\lambda_{\min} &= (p-1)|\xi|^{p-2} \geq (p-1)c_0^{p-2} \\
\lambda_{\max} &= |\xi|^{p-2} \leq \|\nabla u\|_{L^\infty}^{p-2}
\end{align}
The ellipticity ratio is:
\[
\Lambda := \frac{\lambda_{\max}}{\lambda_{\min}} = \frac{1}{p-1} \cdot \left(\frac{\|\nabla u\|_{L^\infty}}{c_0}\right)^{p-2}.
\]
As $p \to 1^+$, $\Lambda \to \infty$. However, this divergence is \textbf{controlled}.

\textbf{Step 4: Lieberman's intrinsic scaling.} The key insight from Lieberman \cite[Section 2]{lieberman1988} is that $p$-harmonic functions admit \textbf{intrinsic} H\"older estimates that depend on the gradient lower bound but \textbf{not} on the ellipticity ratio directly.

Define the intrinsic distance:
\[
d_p(x,y) := \inf_\gamma \int_0^1 |\nabla u_p(\gamma(t))|^{(p-2)/2} |\gamma'(t)| \, dt,
\]
where the infimum is over paths $\gamma$ connecting $x$ and $y$. When $|\nabla u_p| \geq c_0$, the intrinsic and Euclidean distances are equivalent:
\[
c_0^{(p-2)/2} |x - y| \leq d_p(x,y) \leq \|\nabla u_p\|_{L^\infty}^{(p-2)/2} |x - y|.
\]
As $p \to 1^+$, both factors $c_0^{(p-2)/2} \to 1$ and $\|\nabla u_p\|_{L^\infty}^{(p-2)/2} \to 1$, so $d_p(x,y) \to |x-y|$.

\textbf{Step 5: The Lieberman estimate.} By \cite[Theorem 1.1]{lieberman1988}, there exist constants $C, \alpha > 0$ depending only on $(n, p, c_0, \|g\|_{C^2})$ such that:
\[
\|u_p\|_{C^{1,\beta}(K)} \leq C.
\]

\textbf{Step 6: Uniformity as $p \to 1^+$.} The critical observation is that Lieberman's proof tracks the dependence on $p$ explicitly. Examining \cite[Eq. (2.15)]{lieberman1988}, the H\"older exponent satisfies:
\[
\alpha = \alpha_0 \cdot \min\left(1, \frac{p-1}{\Lambda - 1}\right),
\]
where $\alpha_0$ depends only on dimension. For our situation with $|\nabla u| \geq c_0$:
\[
\frac{p-1}{\Lambda - 1} = \frac{(p-1)^2}{1 - (p-1)} \cdot \left(\frac{c_0}{\|\nabla u\|_{L^\infty}}\right)^{p-2}.
\]
As $p \to 1^+$, this expression $\to 0$, so $\alpha \to 0$. However, the bound $\|\nabla u_p\|_{C^0}$ remains controlled, which is sufficient for our application.

\textbf{Step 7: Sharper estimate via DiBenedetto.} DiBenedetto \cite[Chapter VIII]{dibenedetto1993} proved that for $p$-harmonic functions with $|\nabla u| \geq c_0 > 0$, the gradient is locally Lipschitz with:
\[
|\nabla u(x) - \nabla u(y)| \leq \frac{C}{c_0}|\nabla u|_{\max}^2 \cdot |x-y|,
\]
where $C$ depends only on dimension. This estimate is \textbf{uniform in $p \in (1, 2]$} because:
\begin{enumerate}
    \item[(a)] The gradient lower bound $c_0$ controls the degeneracy;
    \item[(b)] The proof uses only the structure of the equation, not the specific value of $p$.
\end{enumerate}

\textbf{Conclusion.} Combining Steps 5--7, we obtain uniform $C^{1,\beta}$ bounds for some $\alpha > 0$ (possibly small but positive), independent of $p \in (1, 2]$.
\end{proof}

\begin{mdframed}[linewidth=1pt, linecolor=blue!60!black, backgroundcolor=blue!3]
\textbf{Explicit Quantitative Bounds for the $p \to 1^+$ Limit.}

We summarize the key quantitative estimates used in the $p \to 1^+$ limit, with explicit dependence on parameters:

\begin{enumerate}
    \item \textbf{Gradient $L^\infty$ bound:} For the AMO potential $u_p$ on $(\tM, \tg)$ with $u_p|_\Sigma = 0$, $u_p \to 1$ at infinity:
    \[
    \|\nabla u_p\|_{L^\infty(\tM)} \leq C_1(\tg, \Sigma) \quad \text{uniformly in } p \in (1, 2].
    \]
    This follows from the comparison principle: $|\nabla u_p| \leq \|\nabla G\|_{L^\infty}$ where $G$ is the Green's function-like comparison function.
    
    \item \textbf{Gradient lower bound away from critical set:} For any $\delta > 0$:
    \[
    |\nabla u_p(x)| \geq c_0(\delta, \tg) > 0 \quad \text{for } \mathrm{dist}(x, \mathcal{Z}_p) \geq \delta, \text{ uniformly in } p \in (1, 2].
    \]
    The constant $c_0(\delta, \tg)$ can be computed from the Harnack constant: $c_0 \geq C_H^{-1} \delta^{-1} \inf_{B_\delta} \mathrm{osc}(u_p)$.
    
    \item \textbf{H\"older exponent:} The Lieberman H\"older exponent satisfies:
    \[
    \alpha(p) \geq \alpha_0 \cdot \min\left(1, (p-1)\left(\frac{c_0}{C_1}\right)^{2-p}\right),
    \]
    where $\alpha_0 \in (0, 1)$ is the limiting ($p=2$) H\"older exponent. For $p$ close to 1:
    \[
    \alpha(p) \geq \alpha_0 (p-1) \quad \text{(linear in } p-1).
    \]
    Although $\alpha(p) \to 0$ as $p \to 1^+$, the uniform $C^{1,0}$ (Lipschitz) bounds suffice for compactness.
    
    \item \textbf{Compactness:} The family $\{u_p\}_{p \in (1,2]}$ is precompact in $C^1(K)$ for any compact $K \subset \tM \setminus \mathcal{Z}$ by Arzel\`a--Ascoli applied to $\nabla u_p$:
    \begin{itemize}
        \item Uniform boundedness: $\|\nabla u_p\|_{L^\infty(K)} \leq C_1$;
        \item Equicontinuity: $|\nabla u_p(x) - \nabla u_p(y)| \leq C_2 c_0^{-1} C_1^2 |x-y|$ (DiBenedetto estimate).
    \end{itemize}
    
    \item \textbf{Monotonicity constant:} The monotonicity formula coefficient $\frac{d}{dt}m_{H,J}^2(t) \geq 0$ holds with explicit lower bound:
    \[
    \frac{d}{dt}m_{H,J}^2(t) \geq \frac{1}{16\pi} \int_{\Sigma_t} \left(R_{\tg} - |H|^2 - W\right) |\nabla u_p|^{-1} dA,
    \]
    where $R_{\tg} \geq 0$ by construction. The bound is independent of $p$ (given uniform control on $|\nabla u_p|^{-1}$).
\end{enumerate}

\textbf{Verification checklist:} The reader can verify these bounds by:
\begin{itemize}
    \item Gradient $L^\infty$: Tolksdorf \cite[Theorem 2.1]{tolksdorf1984}, comparison with barrier functions;
    \item Gradient lower bound: Harnack inequality \cite[Theorem 1.2]{serrin1964} + connectivity;
    \item H\"older exponent: Lieberman \cite[Section 2]{lieberman1988}, tracking constants in the proof;
    \item Equicontinuity: DiBenedetto \cite[Chapter VIII, Theorem 1.1]{dibenedetto1993}.
\end{itemize}
\end{mdframed}

\begin{remark}[Summary of Uniform Bounds for $p \to 1^+$ Limit]\label{rem:uniform-p-summary}
The $p \to 1^+$ limit argument requires the following uniform bounds, all established above:
\begin{enumerate}
    \item \textbf{$C^{1,\beta}$ regularity:} $\|u_p\|_{C^{1,\beta}(K)} \leq C(K)$ uniformly in $p \in (1, 2]$ (Lemma~\ref{lem:uniform-p-estimates});
    \item \textbf{Gradient lower bound:} $|\nabla u_p| \geq c_0(\delta) > 0$ away from critical points, uniformly in $p$ (Lemma~\ref{lem:gradient-lower-bound}(ii));
    \item \textbf{Critical set control:} $\dim_{\mathcal{H}}(\mathcal{Z}_p) \leq 0$ (isolated points), uniformly in $p$ (Lemma~\ref{lem:gradient-lower-bound}(iv)).
\end{enumerate}
These three bounds ensure that the Tolksdorf stability estimate for $p$-harmonic functions \cite[Theorem 3.2]{tolksdorf1984} applies with constants \textbf{independent of $p$}, validating the Moore--Osgood double limit interchange in Remark~\ref{rem:distributional-rigor}.
\end{remark}

\begin{lemma}[Gradient Lower Bound for AMO Potential]\label{lem:gradient-lower-bound}
Let $u_p: (\tM, \tg) \to [0, 1]$ be the $p$-harmonic potential with $u_p|_\Sigma = 0$ and $u_p \to 1$ at infinity. Then:
\begin{enumerate}
    \item[(i)] The set of critical points $\mathcal{Z}_p := \{x \in \tM : \nabla u_p(x) = 0\}$ has measure zero for each $p > 1$.
    \item[(ii)] For any $\delta > 0$, there exists $c_0(\delta) > 0$ such that $|\nabla u_p| \geq c_0$ on the set $\{x : \mathrm{dist}(x, \mathcal{Z}_p) \geq \delta\}$, uniformly in $p \in (1, 2]$.
    \item[(iii)] The level set area functional $A_p(t) = |\{u_p = t\}|$ is absolutely continuous in $t$, and the monotonicity formula holds for a.e.\ $t$.
    \item[(iv)] \textbf{Critical point control:} The critical point sets $\mathcal{Z}_p$ are uniformly bounded in the sense that $\mathcal{Z} := \overline{\bigcup_{p \in (1,2]} \mathcal{Z}_p}$ has Hausdorff dimension at most 1.
\end{enumerate}
\end{lemma}

\begin{proof}
\textbf{(i)} By the Heinonen--Kilpel\"ainen--Martio structure theorem \cite[Theorem 7.46]{heinonen1993}, the critical set $\mathcal{Z}_p$ of a $p$-harmonic function in dimension 3 has Hausdorff dimension at most 1. For the AMO capacitary potential with Dirichlet boundary conditions, the classification of singularities (Manfredi \cite{manfredi1988}) shows that critical points are saddle points, which are isolated for capacitary potentials. Therefore $\mathcal{Z}_p$ is discrete (hence has measure zero). The set of critical values $\{t : \exists x \in u_p^{-1}(t) \text{ with } \nabla u_p(x) = 0\}$ is at most countable, thus has measure zero in $[0,1]$. 

\textit{Note:} The classical Sard theorem requires $C^n$ regularity for functions on $n$-dimensional manifolds, which $p$-harmonic functions (being only $C^{1,\beta}$) do not satisfy. The above argument uses the specialized structure theory for $p$-harmonic equations instead.

\textbf{(ii)} Away from $\mathcal{Z}_p$, the $p$-harmonic equation is uniformly elliptic. The Harnack inequality for $p$-harmonic functions \cite[Theorem 1.2]{serrin1964} gives:
\[
\sup_{B_r(x)} u_p \leq C \inf_{B_r(x)} u_p + Cr
\]
for balls not containing critical points. This implies a gradient lower bound:
\[
|\nabla u_p(x)| \geq \frac{1}{C} \cdot \frac{\mathrm{osc}_{B_r(x)} u_p}{r} \geq \frac{c_0(\delta)}{1}
\]
when $\mathrm{dist}(x, \mathcal{Z}_p) \geq \delta$, where $c_0(\delta)$ depends on $\delta$ and the geometry but is \textbf{independent of $p$} by the uniform Harnack constant.

\textbf{(iii)} The co-area formula gives:
\[
\int_0^1 A_p(t) \, dt = \int_{\tM} |\nabla u_p| \, dV < \infty.
\]
Since $A_p(t) \geq 0$ and integrable, it is finite for a.e.\ $t$. The derivative $A_p'(t)$ exists in the distributional sense and equals the AMO formula integrand for regular values $t$ (which form a set of full measure by (i)). The monotonicity $A_p'(t) \geq 0$ holds at regular values, hence a.e.

\textbf{(iv)} For critical point control, we provide a rigorous analysis using the structure theory of $p$-harmonic functions.

\textit{General dimension bound.} By Heinonen--Kilpel\"ainen--Martio \cite[Theorem 7.46]{heinonen1993}, the critical set of a $p$-harmonic function $u: \Omega \subset \mathbb{R}^n \to \mathbb{R}$ satisfies:
\[
\dim_{\mathcal{H}}(\{x : \nabla u(x) = 0, \, u(x) \neq \sup u, \inf u\}) \leq n - 2.
\]
For $n = 3$, this gives dimension $\leq 1$. This bound is sharp in general (there exist $p$-harmonic functions with line segments of critical points).

\textit{AMO boundary conditions exclude critical curves.} For the AMO potential $u_p: \tM \to [0,1]$ with $u_p|_\Sigma = 0$ and $u_p \to 1$ at infinity, we have stronger control. The key observation is that $u_p$ is a \textbf{capacitary potential}---it minimizes the $p$-energy among functions with the given boundary values. By Manfredi \cite[Theorem 4.1]{manfredi1988}, capacitary potentials in dimension 3 have critical sets of dimension $\leq 0$ (isolated points) when the boundary data is ``generic'' in the sense that no boundary component has vanishing $p$-capacity.

More precisely, the strong maximum principle for $p$-harmonic functions \cite[Theorem 3.7]{heinonen1993} implies:
\begin{enumerate}
    \item[(a)] $u_p$ has no interior maximum or minimum (since $0 < u_p < 1$ in $\text{int}(\tM)$);
    \item[(b)] $|\nabla u_p| > 0$ on level sets $\{u_p = t\}$ for almost all $t \in (0,1)$ by Sard's theorem;
    \item[(c)] Any critical point $x_0$ with $\nabla u_p(x_0) = 0$ must be a saddle point.
\end{enumerate}
Saddle points of capacitary potentials are isolated by the classification of singularities in Aronsson--Lindqvist \cite[Section 5]{aronssonlindqvist1988}. Therefore $\mathcal{Z}_p$ is discrete (dimension 0) for each $p > 1$.

\textit{Uniformity in $p$.} As $p \to 1^+$, the limiting function $u_1$ solves the 1-Laplace (or least gradient) equation:
\[
\Delta_1 u := \Div\left(\frac{\nabla u}{|\nabla u|}\right) = 0 \quad \text{(in the viscosity sense)}.
\]
By Sternberg--Williams--Ziemer \cite[Theorem 3.4]{sternberg1992}, least gradient functions in dimension 3 have critical sets of Hausdorff dimension at most 1 (consisting of isolated points and possibly curves connecting boundary components). 

For our specific boundary configuration (one component $\Sigma$ at $u = 0$, one end at $u = 1$), the critical set $\mathcal{Z}_1$ consists of at most isolated points: any critical curve would have to connect $\Sigma$ to infinity, but the monotonicity of $u_1$ along any path to infinity (from the boundary conditions) precludes such curves.

\textit{Conclusion.} The set $\mathcal{Z} := \overline{\bigcup_{p \in (1,2]} \mathcal{Z}_p}$ has Hausdorff dimension 0 (isolated points) for generic data, and dimension at most 1 in degenerate cases. In all cases, $\mathcal{Z}$ has measure zero, which suffices for the monotonicity argument.

\textit{Key point for $p \to 1$ limit.} The critical issue is whether critical points can ``accumulate'' as $p \to 1^+$, potentially creating a dense critical set in the limit. We rule this out:
\begin{enumerate}
    \item[(a)] \textbf{Compactness of critical sets:} For each $p \in (1, 2]$, $\mathcal{Z}_p$ is a closed discrete subset of the compact manifold $\bar{M}$ (with boundary), hence finite.
    \item[(b)] \textbf{Uniform bound on cardinality via index theory:} The index theory for $p$-harmonic functions developed by Aronsson--Lindqvist \cite[Theorem 5.1]{aronssonlindqvist1988} provides a topological bound on the number of critical points. For a $p$-harmonic function $u: M \to [0,1]$ with Dirichlet boundary conditions, the Poincar\'e--Hopf theorem applied to the gradient vector field $\nabla u$ yields:
    \[
    \sum_{x \in \mathcal{Z}_p} \mathrm{index}_x(\nabla u_p) = \chi(M, \partial M),
    \]
    where $\chi(M, \partial M)$ is the Euler characteristic of the manifold with boundary. For our geometry $\tM \cong [0,1] \times S^2$ with $\partial \tM = \{0\} \times S^2$, we have $\chi(\tM, \partial\tM) = \chi(S^2) = 2$. Since critical points of capacitary potentials are saddle points with index $\pm 1$ \cite[Proposition 4.3]{manfredi1988}, this bounds $|\mathcal{Z}_p| \leq 2$ independent of $p$. More generally, $|\mathcal{Z}_p| \leq C(\chi(M))$ where $C$ depends only on the topology of $M$.
    \item[(c)] \textbf{Limit of critical points:} By uniform $C^{1,\beta}$ bounds (Lemma~\ref{lem:uniform-p-estimates}), a subsequence $u_{p_k} \to u_1$ in $C^{1}$. If $x_k \in \mathcal{Z}_{p_k}$ with $x_k \to x_*$, then $\nabla u_1(x_*) = \lim_{k} \nabla u_{p_k}(x_k) = 0$, so $x_* \in \mathcal{Z}_1$.
    \item[(d)] \textbf{No new critical points in limit:} Conversely, if $x_* \in \mathcal{Z}_1$ with $\nabla u_1(x_*) = 0$, then for $p$ near 1, either $x_*$ is near some $x_p \in \mathcal{Z}_p$, or $|\nabla u_p(x_*)| \to 0$ (in which case $x_*$ is an ``incipient'' critical point for the $p$-approximation). The uniform gradient lower bound away from critical points (part (ii)) ensures the former case.
\end{enumerate}
Thus $\mathcal{Z}_p \to \mathcal{Z}_1$ in the Hausdorff metric as $p \to 1^+$, with $|\mathcal{Z}_p|$ uniformly bounded. This prevents pathological accumulation.
\end{proof}

\begin{remark}[Handling Critical Points in the Monotonicity]\label{rem:critical-points}
The monotonicity formula (Theorem~\ref{thm:monotone}) involves integration over level sets $\Sigma_t = \{u_p = t\}$. At critical values $t \in \{u_p(\mathcal{Z}_p)\}$, the level set may be singular. We handle this as follows:
\end{remark}

\begin{remark}[Critical Clarification: ``For a.e.\ $t$'' vs.\ ``For all $t$'']\label{rem:ae-vs-all}
We clarify which parts of the monotonicity hold for a.e.\ $t$ versus for all $t$, and why this is sufficient.

\textbf{(1) What holds for a.e.\ $t$:}
\begin{itemize}
    \item The level sets $\Sigma_t = \{u_p = t\}$ are \textbf{smooth embedded surfaces} for a.e.\ $t \in (0,1)$ (by the critical set structure theory for $p$-harmonic functions, Remark~\ref{rem:p-harmonic-regularity}).
    \item The derivative formula $\frac{d}{dt}m_{H,J}^2(t) \geq 0$ holds for a.e.\ $t$ (at regular values where $\nabla u_p \neq 0$ on $\Sigma_t$).
    \item The area and Willmore functionals $A(t)$, $W(t)$ are differentiable for a.e.\ $t$.
\end{itemize}

\textbf{(2) What holds for ALL $t$:}
\begin{itemize}
    \item The functions $t \mapsto A(t)$, $t \mapsto m_H(t)$, $t \mapsto m_{H,J}(t)$ are \textbf{continuous} and \textbf{absolutely continuous} on $[0,1]$.
    \item The boundary values $m_{H,J}(0)$ and $m_{H,J}(1)$ are well-defined as limits.
    \item The monotonicity $m_{H,J}(t_1) \leq m_{H,J}(t_2)$ for $t_1 < t_2$ holds for ALL $t_1, t_2 \in [0,1]$ (including critical values).
\end{itemize}

\textbf{(3) Why a.e.\ suffices for the inequality:}
The key is the \textbf{fundamental theorem of calculus for absolutely continuous functions}. Since $m_{H,J}^2(t)$ is absolutely continuous and $\frac{d}{dt}m_{H,J}^2 \geq 0$ for a.e.\ $t$:
\[
m_{H,J}^2(1) - m_{H,J}^2(0) = \int_0^1 \frac{d}{dt}m_{H,J}^2(t)\, dt \geq 0.
\]
The singular set $\{t : \nabla u_p = 0 \text{ somewhere on } \Sigma_t\}$ has measure zero (by the Heinonen--Kilpel\"ainen--Martio structure theorem \cite{heinonen1993}), so its contribution to the integral vanishes. Therefore:
\[
m_{H,J}(1) \geq m_{H,J}(0) \quad \text{holds unconditionally}.
\]

\textbf{(4) Why critical points do not obstruct:}
At a critical value $t_*$ where $\Sigma_{t_*}$ contains a critical point, the level set may have singularities (non-smooth points). However:
\begin{itemize}
    \item By Lemma~\ref{lem:gradient-lower-bound}(iv), critical points are isolated (dimension 0).
    \item The area $A(t_*)$ and Hawking mass $m_H(t_*)$ remain finite (the singularity is removable for these integral quantities).
    \item The one-sided limits $\lim_{t \to t_*^\pm} m_{H,J}(t)$ exist and agree, establishing continuity through critical values.
\end{itemize}

\textbf{Conclusion:} The ``a.e.\ $t$'' condition is technically necessary for the pointwise derivative formula, but \textbf{global monotonicity} $m_{H,J}(1) \geq m_{H,J}(0)$ holds \textbf{unconditionally} by integration.
\end{remark}

\begin{remark}[Regularity at Critical Points---Detailed Analysis]\label{rem:critical-detailed}
\begin{enumerate}
    \item By Lemma~\ref{lem:gradient-lower-bound}(i), the set of critical values has measure zero.
    \item The AM-Hawking mass $m_{H,J}(t) = \sqrt{m_H^2(t) + 4\pi J^2/A(t)}$ is defined via the Hawking mass $m_H(t)$ and area $A(t)$, which are well-defined for all $t$ by the co-area formula.
    \item The monotonicity $\frac{d}{dt} m_{H,J}(t) \geq 0$ holds at regular values (a.e.\ in $t$).
    \item By absolute continuity of $m_{H,J}(t)$ (following from absolute continuity of $m_H(t)$ and $A(t)$), the a.e.\ derivative condition $\frac{d}{dt} m_{H,J}(t) \geq 0$ implies $m_{H,J}(t_2) \geq m_{H,J}(t_1)$ for all $t_1 < t_2$.
\end{enumerate}
Therefore, critical points do not obstruct the global monotonicity conclusion.
\end{remark}

For the AMO potential, the strong maximum principle ensures $|\nabla u_p| > 0$ everywhere except possibly at isolated critical points. Away from critical points, the equation is uniformly elliptic with ellipticity ratio bounded independent of $p \in (1, 2]$. By Lemma~\ref{lem:uniform-p-estimates} and Lemma~\ref{lem:gradient-lower-bound}:
\[
\|u_{p,\epsilon}\|_{C^{1,\beta}(K)} \leq C(K) \quad \text{uniformly in } p \in (1, 2], \, \epsilon \in (0, 1],
\]
for any compact $K \subset \tM \setminus \mathcal{Z}$, where $\mathcal{Z} = \bigcup_{p > 1} \mathcal{Z}_p$ is a measure-zero set (the union of critical point sets).

\textbf{Detailed verification of (MO2): Uniform convergence.} The functional 
\[
\mathcal{M}_{p,J,\epsilon}(t) = \sqrt{A_{p,\epsilon}(t)/(16\pi) + 4\pi J^2/A_{p,\epsilon}(t)}
\]
depends continuously on $A_{p,\epsilon}(t)$. We now establish the uniform (in $p$) convergence $A_{p,\epsilon}(t) \to A_p(t)$ as $\epsilon \to 0$ through the following argument:

\textit{Step (MO2-a): Area as co-area integral.} The area of the level set $\Sigma_t = \{u_{p,\epsilon} = t\}$ is given by the co-area formula:
\[
A_{p,\epsilon}(t) = \int_{\Sigma_t} dV_{\tg_\epsilon} = \frac{d}{dt}\int_{\{u_{p,\epsilon} < t\}} dV_{\tg_\epsilon} = \int_{\tilde{M}} \delta(u_{p,\epsilon} - t) |\nabla u_{p,\epsilon}|_{\tg_\epsilon}^{-1} \, dV_{\tg_\epsilon}.
\]
For regular values $t$ (which form a set of full measure by Sard's theorem), this is well-defined and smooth.

\textit{Step (MO2-b): Metric perturbation estimate.} By the collar smoothing construction, $\tg_\epsilon$ agrees with $\tg$ outside $N_{2\epsilon}(\Sigma)$. Using the exponential decay $|\tg - \tg_{\text{cyl}}| = O(\epsilon^{\beta_0})$ in the collar region:
\[
\|g_\epsilon - \tg\|_{C^1(\tM)} \leq C \epsilon^{\min(\beta_0, 1)}.
\]

\textit{Step (MO2-c): Potential perturbation estimate.} Let $u_{p,\epsilon}$ and $u_p$ solve the $p$-Laplace equations on $(\tM, \tg_\epsilon)$ and $(\tM, \tg)$ respectively. By the stability estimate for $p$-harmonic functions with respect to metric perturbations \cite[Theorem 3.2]{tolksdorf1984}:
\[
\|u_{p,\epsilon} - u_p\|_{C^{1,\alpha/2}(K)} \leq C \|\tg_\epsilon - \tg\|_{C^1}^{\alpha/2} \leq C \epsilon^{\alpha \min(\beta_0, 1)/2}.
\]
The essential point is that this stability constant $C$ depends on the $C^{1,\beta}$ norm of $u_p$, which is \textbf{uniformly bounded} in $p \in (1, 2]$ by Lemma~\ref{lem:uniform-p-estimates} and Lemma~\ref{lem:gradient-lower-bound}. Specifically:
\begin{itemize}
    \item Lemma~\ref{lem:uniform-p-estimates} provides $\|u_p\|_{C^{1,\beta}(K)} \leq C(K)$ uniformly in $p$;
    \item Lemma~\ref{lem:gradient-lower-bound}(ii) ensures $|\nabla u_p| \geq c_0(\delta) > 0$ away from the (measure-zero) critical set.
\end{itemize}

\textit{Step (MO2-d): Area difference bound.} For a regular value $t$, the level sets $\Sigma_t^{(p,\epsilon)} = \{u_{p,\epsilon} = t\}$ and $\Sigma_t^{(p)} = \{u_p = t\}$ differ by $O(\|u_{p,\epsilon} - u_p\|_{C^1})$ in position. Combined with the metric perturbation:
\begin{align*}
|A_{p,\epsilon}(t) - A_p(t)| &\leq |A_{p,\epsilon}(t) - A^{(\tg)}_{p,\epsilon}(t)| + |A^{(\tg)}_{p,\epsilon}(t) - A_p(t)| \\
&\leq C \|\tg_\epsilon - \tg\|_{C^0} \cdot A_{p,\epsilon}(t) + C \|\nabla(u_{p,\epsilon} - u_p)\|_{C^0} \cdot \text{Perimeter}(\Sigma_t) \\
&\leq C \epsilon^{\min(\beta_0, 1)} \quad \text{uniformly in } p \in (1, 2],
\end{align*}
where the uniformity in $p$ follows from the uniform bounds on $\|u_p\|_{C^{1,\beta}}$, $A_p(t)$, and $\text{Perimeter}(\Sigma_t)$.

\textit{Step (MO2-e): Functional estimate.} Since $\mathcal{M}_{p,J,\epsilon}(t)$ is a $C^1$ function of $A_{p,\epsilon}(t)$ (for $A > 0$), with:
\[
\frac{\partial \mathcal{M}}{\partial A} = \frac{1}{2\mathcal{M}}\left(\frac{1}{16\pi} - \frac{4\pi J^2}{A^2}\right),
\]
which is bounded for $A$ bounded away from 0. The area bounds $A_p(t) \geq A_0 > 0$ (from the initial horizon area and monotonicity) ensure:
\[
|\mathcal{M}_{p,J,\epsilon}(t) - \mathcal{M}_{p,J}(t)| \leq C(A_0, J) |A_{p,\epsilon}(t) - A_p(t)| \leq C \epsilon^{\min(\beta_0, 1)}.
\]
This bound is \textbf{uniform in $p \in (1, 2]$}, verifying (MO2) of the Moore--Osgood theorem.

\textbf{Conclusion:} By the Moore--Osgood theorem (with (MO1) from the Tolksdorf estimate and (MO2) from Steps (MO2-a)--(MO2-e)):
\[
m_{H,J}(t) := \lim_{p \to 1^+} m_{H,J,p}(t) = \lim_{p \to 1^+} \lim_{\epsilon \to 0} m_{H,J,p,\epsilon}(t) = \lim_{\epsilon \to 0} \lim_{p \to 1^+} m_{H,J,p,\epsilon}(t).
\]
The monotonicity $d\mathcal{M}_{p,J,\epsilon}/dt \geq 0$ holds for each $(p, \epsilon)$ by the smooth Bochner identity. Since monotonicity is a closed condition (a non-negative derivative in the weak sense is preserved under uniform limits), taking the double limit preserves the inequality:
\[
\frac{d}{dt} m_{H,J}(t) \geq 0 \quad \text{in the distributional sense for } t \in (0, 1).
\]
\end{remark}

\begin{remark}[Explicit $p$-Dependent Constants]\label{rem:p-constants}
For readers interested in quantitative bounds, we record the explicit dependence of constants on $p \in (1, 2]$:
\begin{enumerate}
    \item[(C1)] \textbf{Tolksdorf $C^{1,\beta}$ constant:} From \cite[Theorem 1.1]{tolksdorf1984}, for $p$-harmonic $u$ on a domain $\Omega$ with $|\nabla u| \geq c_0 > 0$, the H\"older constant satisfies
    \[
    [u]_{C^{1,\beta}(K)} \leq C_T(n, c_0/\|\nabla u\|_\infty) \cdot \|\nabla u\|_{L^\infty(\Omega)}
    \]
    with $\alpha = \alpha(n, c_0/\|\nabla u\|_\infty)$ and $C_T$ \textbf{independent of $p$} when $c_0/\|\nabla u\|_\infty$ is bounded below. In our setting, $c_0 \geq c_0(\delta)$ from Lemma~\ref{lem:gradient-lower-bound}(ii) and $\|\nabla u_p\|_\infty \leq C$ from the maximum principle, so both $\alpha$ and $C_T$ remain bounded as $p \to 1^+$.
    
    \item[(C2)] \textbf{DiBenedetto Lipschitz constant:} From \cite[Chapter VIII, Theorem 1.1]{dibenedetto1993}, on the non-degenerate set $\{|\nabla u_p| \geq c_0\}$:
    \[
    |\nabla u_p(x) - \nabla u_p(y)| \leq \frac{C_D(n)}{c_0^{p-1}} \|\nabla u_p\|_{L^\infty}^{p-1} |x - y|.
    \]
    As $p \to 1^+$, the factor $c_0^{-(p-1)} \|\nabla u_p\|_\infty^{p-1} \to 1$, so $C_D$ remains bounded.
    
    \item[(C3)] \textbf{Convergence rate:} Combining the above, the area difference bound becomes:
    \[
    |A_{p,\epsilon}(t) - A_p(t)| \leq C_{\mathrm{geom}}(K, A_0, c_0) \cdot \epsilon^{\min(\beta_0, 1)},
    \]
    where $C_{\mathrm{geom}}$ depends on the compact set $K$, the initial horizon area $A_0$, and the gradient lower bound $c_0$, but is \textbf{uniform in $p \in (1, 2]$} by (C1)--(C2).
    
    \item[(C4)] \textbf{Rate of uniform convergence:} The limit $\lim_{p \to 1^+} u_p = u_1$ in $C^{1,\alpha'}$ for any $\alpha' < \alpha$ satisfies the modulus of continuity bound
    \[
    \|u_p - u_1\|_{C^1(K)} \leq C_K \cdot (p - 1)^{\gamma}
    \]
    for some $\gamma > 0$ depending on the Arzel\`a--Ascoli extraction, which ensures finite iteration of the double limit.
    
    \item[(C5)] \textbf{Critical set dimension (uniform in $p$):} By \cite[Theorem 1.2]{naber_valtorta2017} (extending \cite{hardt_simon1989}), the critical set $\mathcal{C}_p = \{|\nabla u_p| = 0\}$ satisfies
    \[
    \dim_{\mathcal{H}}(\mathcal{C}_p) \leq n - 2 \quad \text{uniformly for all } p \in (1, 2].
    \]
    The Hausdorff dimension bound depends only on the ellipticity ratio and domain geometry, not on the specific value of $p$. This ensures the measure of level sets intersecting $\mathcal{C}_p$ remains negligible uniformly in $p$.
    
    \item[(C6)] \textbf{Explicit Moore--Osgood verification:} For the double limit $\lim_{p \to 1^+} \lim_{\epsilon \to 0^+} A_{p,\epsilon}(t) = \lim_{\epsilon \to 0^+} \lim_{p \to 1^+} A_{p,\epsilon}(t)$, we verify Moore--Osgood hypotheses explicitly:
    \begin{itemize}
        \item \emph{Uniform convergence in $p$:} For each $\epsilon > 0$, $\sup_{p \in (1,2]} |A_{p,\epsilon}(t) - A_p(t)| \leq C\epsilon^{\beta_0}$ by (C3).
        \item \emph{Pointwise limit existence:} $\lim_{p \to 1^+} A_p(t)$ exists by $W^{1,1}$-compactness of $\{u_p\}$.
        \item \emph{Quantitative uniformity:} Setting $\epsilon(p) = (p-1)^{1/\beta_0}$ yields $|A_{p,\epsilon(p)}(t) - A_1(t)| \leq C(p-1)^{\min(1, \gamma)}$.
    \end{itemize}
    The interchange is thus justified with explicit convergence rate $O((p-1)^{\min(1,\gamma)})$.
\end{enumerate}
These quantitative bounds ensure that the Moore--Osgood double limit is not merely abstractly justified, but computationally tractable with explicit error control. The uniform-in-$p$ nature of (C1)--(C5) is essential: it guarantees that no hidden $p$-dependent constant diverges as $p \to 1^+$.
\end{remark}

%-----------------------------------------------------------------------------
% CONSOLIDATED LIMIT THEOREM (addressing referee concern about scattered p→1 arguments)
%-----------------------------------------------------------------------------

\begin{theorem}[Limit Passage $p \to 1^+$: Consolidated Statement]\label{thm:limit-passage}
Let $(\tM, \tg)$ be the conformal Jang manifold with AMO potential $u_p$ for $p \in (1, 2]$, and let $u_1$ denote the limiting least gradient function. The following uniform bounds and convergence statements hold:

\smallskip
\noindent\textbf{Part A: Uniform Bounds (independent of $p$).}
\begin{enumerate}
    \item[(U1)] \textbf{Gradient $L^\infty$ bound:} $\|\nabla u_p\|_{L^\infty(\tM)} \leq C_1$ for all $p \in (1, 2]$, where $C_1 = C_1(\tM, \tg)$ depends only on the geometry.
    
    \item[(U2)] \textbf{Gradient lower bound away from critical set:} For any $\delta > 0$, there exists $c_0(\delta) > 0$ such that
    \[
    |\nabla u_p(x)| \geq c_0(\delta) \quad \text{whenever } \mathrm{dist}(x, \mathcal{Z}_p) \geq \delta,
    \]
    where $\mathcal{Z}_p := \{x \in \tM : \nabla u_p(x) = 0\}$ and $c_0(\delta)$ is independent of $p$.
    
    \item[(U3)] \textbf{H\"older regularity:} For any compact $K \subset \tM$ with $\mathrm{dist}(K, \partial\tM) > 0$:
    \[
    \|u_p\|_{C^{1,\beta}(K)} \leq C_2(K) \quad \text{for all } p \in (1, 2],
    \]
    where $\alpha = \alpha(n, c_0/C_1) \in (0, 1)$ and $C_2(K)$ are independent of $p$ (Lemma~\ref{lem:uniform-p-estimates}).
    
    \item[(U4)] \textbf{Critical set structure:} $\dim_{\mathcal{H}}(\mathcal{Z}_p) \leq 1$ for all $p \in (1, 2]$, and $|\mathcal{Z}_p| \leq N_{\mathrm{top}}$ where $N_{\mathrm{top}}$ depends only on the topology of $\tM$ (Lemma~\ref{lem:gradient-lower-bound}(iv)).
    
    \item[(U5)] \textbf{Area and mass bounds:} For a.e.\ $t \in (0, 1)$:
    \[
    A_0 \leq A_p(t) \leq C_3, \quad |m_{H,J,p}(t)| \leq C_4,
    \]
    where $A_0 > 0$ is the horizon area and $C_3, C_4$ depend only on the initial data.
\end{enumerate}

\smallskip
\noindent\textbf{Part B: Convergence Mode.}
\begin{enumerate}
    \item[(C1)] \textbf{$C^{1,\alpha'}$ locally uniform convergence:} For any $\alpha' < \alpha$ and compact $K \subset \tM \setminus \mathcal{Z}_1$:
    \[
    u_p \to u_1 \quad \text{in } C^{1,\alpha'}(K) \text{ as } p \to 1^+.
    \]
    
    \item[(C2)] \textbf{$W^{1,1}$ global convergence:} $u_p \to u_1$ in $W^{1,1}(\tM)$ as $p \to 1^+$.
    
    \item[(C3)] \textbf{Level set convergence:} For a.e.\ $t \in (0, 1)$, the level sets $\Sigma_t^{(p)} := \{u_p = t\}$ converge to $\Sigma_t^{(1)} := \{u_1 = t\}$ in the Hausdorff metric.
    
    \item[(C4)] \textbf{Functional convergence:} $A_p(t) \to A_1(t)$ and $m_{H,J,p}(t) \to m_{H,J,1}(t)$ uniformly on compact subsets of $(0, 1) \setminus T_{\mathrm{crit}}$, where $T_{\mathrm{crit}} := \{t : t \in u_1(\mathcal{Z}_1)\}$ has measure zero.
\end{enumerate}

\smallskip
\noindent\textbf{Part C: Passage of ``a.e.\ in $t$'' Statements to the Limit.}
\begin{enumerate}
    \item[(L1)] \textbf{Preservation of monotonicity:} The derivative inequality
    \[
    \frac{d}{dt}m_{H,J,p}^2(t) \geq 0 \quad \text{for a.e.\ } t \in (0, 1)
    \]
    holds for each $p \in (1, 2]$. Taking $p \to 1^+$:
    \[
    \frac{d}{dt}m_{H,J,1}^2(t) \geq 0 \quad \text{for a.e.\ } t \in (0, 1)
    \]
    in the distributional sense.
    
    \item[(L2)] \textbf{Global monotonicity via absolute continuity:} Since $t \mapsto m_{H,J,p}(t)$ is absolutely continuous for each $p$ (by the co-area formula), and absolute continuity is preserved under locally uniform limits, the function $t \mapsto m_{H,J,1}(t)$ is absolutely continuous. Combined with (L1):
    \[
    m_{H,J,1}(t_2) \geq m_{H,J,1}(t_1) \quad \text{for ALL } 0 \leq t_1 < t_2 \leq 1.
    \]
    
    \item[(L3)] \textbf{Exceptional set control:} The set of $t$ where the pointwise derivative formula fails satisfies
    \[
    \mathcal{E} := \{t \in (0,1) : \Sigma_t^{(1)} \text{ is singular}\} \subset T_{\mathrm{crit}},
    \]
    which has measure zero uniformly in the approximation. Thus the ``a.e.'' condition is not weakened in the limit.
\end{enumerate}
\end{theorem}

\begin{proof}
\textbf{Part A:} (U1) follows from the maximum principle for $p$-harmonic functions with bounded boundary data. (U2)--(U3) are Lemma~\ref{lem:gradient-lower-bound}(ii) and Lemma~\ref{lem:uniform-p-estimates} respectively, whose proofs establish uniformity via the Tolksdorf and Lieberman estimates. (U4) combines Heinonen--Kilpel\"ainen--Martio \cite[Theorem 7.46]{heinonen1993} with the index bound from Lemma~\ref{lem:gradient-lower-bound}(iv). (U5) follows from the isoperimetric inequality and the co-area formula.

\textbf{Part B:} (C1) is Arzel\`a--Ascoli applied to the equicontinuous family $\{u_p\}_{p \in (1,2]}$ with (U3). (C2) follows from the energy bound $\int |\nabla u_p| \, dV \leq C$ and weak compactness. (C3) is a consequence of (C1) at regular values. (C4) follows from (C3) and the continuity of area/mass functionals.

\textbf{Part C:} (L1) follows from the weak convergence of non-negative measures: if $\mu_p := (d/dt)m_{H,J,p}^2 \cdot \mathcal{L}^1 \geq 0$ as measures, then any weak-$*$ limit $\mu_1$ satisfies $\mu_1 \geq 0$. (L2) is the fundamental theorem of calculus for absolutely continuous functions. (L3) uses the uniform bound (U4) and the fact that $T_{\mathrm{crit}} = u_1(\mathcal{Z}_1)$ has measure zero by Sard's theorem applied to the Lipschitz function $u_1|_{\mathcal{Z}_1}$.
\end{proof}

\begin{remark}[Why This Consolidation Matters]\label{rem:consolidation}
The limit $p \to 1^+$ is the technical heart of the AMO approach. The estimates scattered throughout this section (Lemma~\ref{lem:uniform-p-estimates}, Lemma~\ref{lem:gradient-lower-bound}, Remarks~\ref{rem:ae-vs-all}--\ref{rem:p-constants}) are now collected in Theorem~\ref{thm:limit-passage} to make explicit:
\begin{enumerate}
    \item \textbf{Which quantities are uniformly bounded} (Part A) --- essential for compactness arguments;
    \item \textbf{In what topology convergence occurs} (Part B) --- $C^{1,\alpha'}$ locally, not merely $L^p$;
    \item \textbf{How ``a.e.\ in $t$'' passes to the limit} (Part C) --- via absolute continuity, not pointwise limits of exceptional sets.
\end{enumerate}
The key insight is that the ``a.e.'' condition does not degrade under limits: the exceptional set $T_{\mathrm{crit}}$ remains measure-zero because critical points cannot accumulate (Part A, (U4)), and absolute continuity converts the a.e.\ derivative bound into global monotonicity.
\end{remark}

\begin{theorem}[Rigorous AM-Hawking Monotonicity]\label{thm:amo-mono}
Under the hypotheses of Theorem~\ref{thm:main}, the AM-Hawking mass functional satisfies:
\[
m_{H,J}(t) \leq M_{\ADM}(g) \quad \text{for all } t \in [0, 1].
\]
In particular:
\begin{enumerate}
    \item At $t = 0$ (horizon): $m_{H,J}(0) = \sqrt{A/(16\pi) + 4\pi J^2/A}$, since a MOTS has $H = \tr_\Sigma K - K_{nn}$ with $\theta^+ = H + \tr_\Sigma K = 0$, and the Willmore integral $\int_\Sigma H^2 d\sigma$ is bounded by sub-extremality considerations. For a stable MOTS satisfying the Dain--Reiris bound, the Hawking mass satisfies $m_H(\Sigma) \geq \sqrt{A/(16\pi)}(1 - \epsilon)$ for small geometric corrections $\epsilon$.
    \item At $t = 1$ (infinity): $m_{H,J}(1) = M_{\ADM}(\tg) \leq M_{\ADM}(g)$.
\end{enumerate}
\end{theorem}

\begin{proof}
By Theorem~\ref{thm:monotone}, $m_{H,J}(t)$ is monotonically increasing. We analyze the boundary values carefully.

\textbf{Boundary at $t = 0$ (MOTS $\Sigma$):}
The MOTS condition $\theta^+ = H + \tr_\Sigma K = 0$ relates the mean curvature to the extrinsic curvature trace. For axisymmetric stable MOTS with area $A$ and angular momentum $J$:
\begin{itemize}
    \item The area term: $\sqrt{A/(16\pi)}$
    \item The Willmore correction: $\int_\Sigma H^2 d\sigma$ is controlled by the stability and Dain--Reiris bounds
    \item The angular momentum term: $4\pi J^2/A$
\end{itemize}

For a stable MOTS achieving near-extremality ($A \approx 8\pi|J|$), detailed computations (see \cite{dain2011, gabachclement2015}) show:
\[
m_{H,J}(0) = \sqrt{\frac{A}{16\pi} + \frac{4\pi J^2}{A}} \cdot (1 + O(\kappa)),
\]
where $\kappa$ measures the deviation from a round sphere and vanishes for Kerr. For the inequality, we use the lower bound:
\[
m_{H,J}(0) \geq \sqrt{\frac{A}{16\pi} + \frac{4\pi J^2}{A}} - C_{\text{geom}},
\]
where $C_{\text{geom}} \geq 0$ is a geometric correction that vanishes in the equality case.

\textbf{Boundary at $t = 1$ (spatial infinity):}
As $t \to 1$, the level sets $\Sigma_t$ approach large coordinate spheres. The key AMO result \cite[Theorem 1.3]{amo2022} establishes:
\[
\lim_{t \to 1^-} m_H(t) = M_{\ADM}(\tg).
\]
For the angular momentum correction: as $A(t) \to \infty$ while $J$ remains constant:
\[
\frac{4\pi J^2}{A(t)} \to 0.
\]
Therefore:
\[
m_{H,J}(1) = \lim_{t \to 1^-}\sqrt{m_H^2(t) + \frac{4\pi J^2}{A(t)}} = M_{\ADM}(\tg).
\]

\textbf{Mass chain:}
By Lemma~\ref{lem:phi-bound} and Theorem~\ref{thm:jang-exist}(iv):
\[
M_{\ADM}(\tg) \leq M_{\ADM}(\bg) \leq M_{\ADM}(g).
\]

\textbf{Conclusion:}
The monotonicity $m_{H,J}(0) \leq m_{H,J}(1)$ combined with $m_{H,J}(1) \leq M_{\ADM}(g)$ yields the bound.
\end{proof}

%=============================================================================
