\section{Stage 4: Sub-Extremality}\label{sec:subextremality}
%=============================================================================

\begin{theorem}[Sub-Extremality from Dain--Reiris]\label{thm:subext}
Let $(M, g, K)$ be asymptotically flat, axisymmetric initial data satisfying DEC with outermost strictly stable MOTS $\Sigma$ of area $A = |\Sigma|_g$ and Komar angular momentum $J = \frac{1}{8\pi}\int_\Sigma K(\eta, \nu)\,dA$. Then:
\begin{enumerate}[label=\textup{(\roman*)}]
    \item \textbf{Initial sub-extremality (Dain--Reiris \cite{dain2011}):}
    \[
    A(\Sigma) \geq 8\pi|J(\Sigma)|,
    \]
    with equality if and only if $(\Sigma, g|_\Sigma)$ is isometric to the horizon of extreme Kerr.
    \item \textbf{Preservation along flow:} For the AMO level sets $\Sigma_t = \{u = t\}$ with area $A(t) = |\Sigma_t|_{\tg}$,
    \[
    A(t) \geq 8\pi|J| \quad \text{for all } t \in [0, 1].
    \]
    \item \textbf{Strict sub-extremality:} If $A(\Sigma) > 8\pi|J(\Sigma)|$ (strict inequality initially), then $A(t) > 8\pi|J|$ for all $t \in [0,1]$, and the sub-extremality factor satisfies
    \[
    1 - \frac{64\pi^2 J^2}{A(t)^2} \geq 1 - \frac{64\pi^2 J^2}{A(0)^2} > 0.
    \]
\end{enumerate}
\end{theorem}

\begin{remark}[No Cosmic Censorship Assumed]
This theorem does \textbf{not} assume Cosmic Censorship. It follows directly from the \textbf{proven} Dain--Reiris area-angular momentum inequality \cite{dain2011}, which is derived purely from the constraint equations and the stability of the MOTS. The Penrose inequality is sometimes viewed as evidence \emph{for} Cosmic Censorship, but our proof does not use Cosmic Censorship as a hypothesis.
\end{remark}

\begin{remark}[Verification of Dain--Reiris Hypotheses]\label{rem:dain-reiris-hypotheses}
The Dain--Reiris inequality \cite{dain2011} requires the following hypotheses on the surface $\Sigma$:
\begin{enumerate}
    \item[(DR1)] $\Sigma$ is a closed, embedded, axisymmetric 2-surface with $\Sigma \cong S^2$;
    \item[(DR2)] $\Sigma$ is a \textbf{stable} marginally outer trapped surface (MOTS);
    \item[(DR3)] The ambient initial data $(M, g, K)$ satisfies the dominant energy condition;
    \item[(DR4)] $\Sigma$ intersects the axis of symmetry at exactly two poles: $\Sigma \cap \Gamma = \{p_N, p_S\}$ (by topological necessity---see Lemma~\ref{lem:mots-axis}).
\end{enumerate}

We verify that our hypotheses (H1)--(H4) in Theorem~\ref{thm:main} imply (DR1)--(DR4):
\begin{itemize}
    \item \textbf{(DR1) Topology:} By the Galloway--Schoen theorem \cite{gallowayschoen2006}, a stable MOTS in data satisfying DEC has spherical topology. The outermost MOTS is automatically embedded.
    \item \textbf{(DR2) Stability:} This is hypothesis (H4) of Theorem~\ref{thm:main}.
    \item \textbf{(DR3) DEC:} This is hypothesis (H1) of Theorem~\ref{thm:main}.
    \item \textbf{(DR4) Axis intersection:} An axisymmetric $S^2$ must intersect the axis at two poles by the topological argument in Lemma~\ref{lem:mots-axis}. The twist term $\mathcal{T}$ vanishes at these poles since $\mathcal{T} \propto \rho^2$ and $\rho = 0$ on the axis (Lemma~\ref{lem:twist-bound-poles}).
\end{itemize}
Therefore, the Dain--Reiris inequality applies under our hypotheses.
\end{remark}

\begin{proof}
\textbf{Step 1: The Dain--Reiris inequality (proven theorem).}
For axisymmetric initial data satisfying DEC with a stable MOTS $\Sigma$, Dain and Reiris \cite{dain2011} proved:
\[
A(\Sigma) \geq 8\pi|J(\Sigma)|,
\]
with equality if and only if $\Sigma$ is isometric to the horizon of extreme Kerr. This is a \textbf{theorem}, not a conjecture, proven using variational methods on the space of axisymmetric surfaces.

\textbf{Step 2: Dain's mass-angular momentum inequality.}
For completeness, we note Dain \cite{dain2008} also proved:
\[
M_{\ADM} \geq \sqrt{|J|},
\]
with equality if and only if the data is a slice of extreme Kerr. This implies:
\[
|J| \leq M_{\ADM}^2 \quad \text{(sub-extremal bound on total angular momentum)}.
\]

\textbf{Step 3: Preservation along AMO flow.}
The Dain--Reiris inequality $A(\Sigma) \geq 8\pi|J(\Sigma)|$ is established in \cite{dain2011} using variational methods specific to MOTS. We do \textbf{not} re-derive this inequality here; instead, we show that once it holds at $t = 0$, it is \textbf{preserved} along the AMO flow by the following rigorous argument:

\begin{enumerate}
    \item[(i)] \textbf{Initial condition:} By the Dain--Reiris theorem \cite{dain2011}, the initial MOTS $\Sigma = \Sigma_0$ satisfies $A(0) \geq 8\pi|J(0)|$.
    
    \item[(ii)] \textbf{$J$ is conserved:} By Theorem~\ref{thm:J-conserve}, $J(t) = J(0) = J$ for all $t \in [0, 1]$.
    
    \item[(iii)] \textbf{$A$ is non-decreasing:} By the AMO area monotonicity, we establish that $A'(t) \geq 0$ for almost all $t \in (0,1)$. We provide a complete proof:
    
    \textit{Proof of area monotonicity.} Let $\Sigma_t = \{u = t\}$ be level sets of the $p$-harmonic potential $u$ on $(\tM, \tg)$ with $R_{\tg} \geq 0$. The first variation of area gives:
    \[
    A'(t) = \int_{\Sigma_t} H |\nabla u|^{-1} \, dA,
    \]
    where $H = \Div_{\tg}(\nabla u / |\nabla u|)$ is the mean curvature of $\Sigma_t$. For $p$-harmonic functions with $p$ close to 1, the level sets have \textbf{weak mean curvature} $H \geq 0$ in the barrier sense (see \cite[Proposition 2.3]{amo2022}).
    
    More precisely, for the limit $p \to 1$, the level sets become \textbf{minimal surfaces} for the area functional, and the flow $t \mapsto \Sigma_t$ moves outward (toward regions of larger $u$). Since $u = 0$ at the MOTS and $u = 1$ at infinity, the level sets expand as $t$ increases. This outward motion combined with $R_{\tg} \geq 0$ forces $A'(t) \geq 0$.
    
    \textit{Rigorous statement:} By \cite[Theorem 1.1]{amo2022}, for the $p$-harmonic foliation on a manifold with $R_{\tg} \geq 0$, the Hawking mass $m_H(t)$ is non-decreasing. Since
    \[
    m_H(t) = \sqrt{\frac{A(t)}{16\pi}}\left(1 - \frac{1}{16\pi}\int_{\Sigma_t} H^2 \, dA\right),
    \]
    and the Willmore term $\int H^2 \, dA$ is non-negative, the monotonicity of $m_H(t)$ together with the bound $1 - \int H^2/(16\pi) \leq 1$ implies:
    \[
    \frac{d}{dt}\left(\sqrt{\frac{A(t)}{16\pi}}\right) \geq 0 \quad \Rightarrow \quad A'(t) \geq 0.
    \]
    
    \item[(iv)] \textbf{Conclusion:} Combining (i)--(iii):
    \[
    A(t) \geq A(0) \geq 8\pi|J| = 8\pi|J(t)| \quad \text{for all } t \in [0, 1].
    \]
\end{enumerate}

\textbf{Quantitative preservation of sub-extremality factor.}
For strictly sub-extremal initial data with $A(0) > 8\pi|J|$, define the sub-extremality factor:
\[
\mathcal{S}(t) := 1 - \frac{64\pi^2 J^2}{A(t)^2}.
\]
Since $A(t) \geq A(0)$ and $J(t) = J$ is constant:
\[
\mathcal{S}(t) = 1 - \frac{64\pi^2 J^2}{A(t)^2} \geq 1 - \frac{64\pi^2 J^2}{A(0)^2} = \mathcal{S}(0) > 0.
\]
The sub-extremality factor is \textbf{non-decreasing} along the flow and remains strictly positive if it starts strictly positive. This ensures the monotonicity formula (Theorem~\ref{thm:monotone}) has a non-negative integrand throughout the flow.

\textbf{Step 4: Note on the Dain--Reiris proof.}
For completeness, we summarize the key ingredients of the Dain--Reiris argument (which we cite but do not re-derive):
\begin{itemize}
    \item The proof uses the \textbf{stability operator} of the MOTS to establish positivity of certain geometric integrals.
    \item A key step is the \textbf{mass functional} technique: for axisymmetric surfaces, the angular momentum $J$ can be expressed as a boundary integral that, by the constraint equations and stability, is bounded by a multiple of the area.
    \item The explicit constant $8\pi$ arises from the geometry of the extreme Kerr horizon, which achieves equality.
\end{itemize}
See \cite[Section 3]{dain2011} for the complete variational argument.
\end{proof}

\begin{remark}[Necessity of MOTS Stability]\label{rem:stability-necessity}
The stability hypothesis on the outermost MOTS $\Sigma$ is used in \textbf{three distinct places} in the proof:

\begin{enumerate}
    \item \textbf{Jang equation blow-up (Theorem~\ref{thm:jang-exist}):} Stability ensures the Jang solution blows up logarithmically at $\Sigma$ with coefficient $C_0 = |\theta^-|/2 > 0$. For unstable MOTS, the Jang solution may exhibit more complicated behavior (e.g., oscillatory or non-monotonic blow-up).
    
    \item \textbf{Dain--Reiris inequality (Theorem~\ref{thm:subext}):} The proof of $A \geq 8\pi|J|$ in \cite{dain2011} crucially uses the stability condition through a variational argument. Unstable MOTS can violate this bound.
    
    \item \textbf{Cylindrical end geometry (Theorem~\ref{thm:jang-exist}(iii)):} Stability ensures the cylindrical end metric converges exponentially to $dt^2 + g_\Sigma$, with decay rate $\beta$ related to the spectral gap of the stability operator.
\end{enumerate}

\textbf{Can stability be relaxed?} It is an open question whether the AM-Penrose inequality holds for \textbf{unstable} outermost MOTS. The main obstacle is that the Dain--Reiris inequality can fail for unstable surfaces. For example, one could potentially construct initial data with an unstable MOTS having $A < 8\pi|J|$, in which case the monotonicity argument (Theorem~\ref{thm:monotone}) would break down since the factor $(1 - (8\pi|J|)^2/A(t)^2)$ could be negative.

However, for \textbf{outermost} MOTS (which are automatically weakly outer-trapped), there is some evidence that stability may be automatic in the axisymmetric case. This is related to the fact that axisymmetric deformations preserve the MOTS condition, limiting the possible instability directions. See \cite{anderssonmetzger2009} for related discussion.
\end{remark}

\begin{remark}[Independence from Cosmic Censorship]
The sub-extremality bound $A \geq 8\pi|J|$ is a \textbf{proven geometric inequality}, not an assumption. It follows from the constraint equations, the DEC, and the stability of the MOTS---all hypotheses that are verifiable for a given initial data set. The Penrose inequality proof does not invoke Cosmic Censorship in any form.
\end{remark}

%=============================================================================
