\section{Stage 4: Sub-Extremality}\label{sec:subextremality}
%=============================================================================

\begin{mdframed}[linewidth=1pt, linecolor=orange!60!black, backgroundcolor=orange!3]
\textbf{Logical Ordering and Non-Circularity.}

This section establishes the sub-extremality bound $A(t) \geq 8\pi|J|$ for AMO level sets. To prevent circularity, we explicitly state the logical dependencies:

\textbf{What is already established at this stage:}
\begin{enumerate}[label=(\arabic*)]
    \item \textbf{Stage 1 (Section~\ref{sec:jang}):} The Jang manifold $(\bar{M}, \bar{g})$ exists with cylindrical end at the MOTS $\Sigma$.
    \item \textbf{Stage 2 (Section~\ref{sec:lichnerowicz}):} The conformal factor $\phi > 0$ exists, giving $(\tilde{M}, \tilde{g})$ with $R_{\tilde{g}} \geq 0$.
    \item \textbf{Stage 3 (Section~\ref{sec:amo}):} The AMO $p$-harmonic potential $u_p$ exists, foliating $\tilde{M}$ by level sets $\{\Sigma_t\}_{t \in [0,1]}$.
    \item \textbf{Angular momentum conservation (Theorem~\ref{thm:J-conserve} in Section~\ref{sec:amo}):} $J(\Sigma_t) = J$ for all $t$.
\end{enumerate}

\textbf{What this section proves:}
\begin{itemize}
    \item The initial sub-extremality $A(\Sigma) \geq 8\pi|J(\Sigma)|$ follows from the \textbf{Dain--Reiris theorem} \cite{dain2011}, which depends \textbf{only on the initial data} $(M, g, K)$.
    \item The preservation along the flow uses the $J$-conservation from Stage 4 and the area lower bound from the Dain--Reiris theorem applied to each $\Sigma_t$.
\end{itemize}

\textbf{What this section does NOT use:}
\begin{itemize}
    \item The sub-extremality bound does \textbf{not} use the AMO monotonicity (Stage 6).
    \item The sub-extremality bound does \textbf{not} use the final inequality $m_{H,J}(0) \leq M_{\mathrm{ADM}}$.
\end{itemize}
This ordering ensures no circular reasoning: sub-extremality is an \textbf{input} to the mass monotonicity formula, not an output.
\end{mdframed}

\begin{theorem}[Sub-Extremality from Dain--Reiris]\label{thm:subext}
Let $(M, g, K)$ be asymptotically flat, axisymmetric initial data satisfying DEC with outermost strictly stable MOTS $\Sigma$ of area $A = |\Sigma|_g$ and Komar angular momentum $J = \frac{1}{8\pi}\int_\Sigma K(\eta, \nu)\,dA$. Then:
\begin{enumerate}[label=\textup{(\roman*)}]
    \item \textbf{Initial sub-extremality (Dain--Reiris \cite{dain2011}):}
    \[
    A(\Sigma) \geq 8\pi|J(\Sigma)|,
    \]
    with equality if and only if $(\Sigma, g|_\Sigma)$ is isometric to the horizon of extreme Kerr.
    \item \textbf{Preservation along flow:} For the AMO level sets $\Sigma_t = \{u = t\}$ with area $A(t) = |\Sigma_t|_{\tg}$,
    \[
    A(t) \geq 8\pi|J| \quad \text{for all } t \in [0, 1].
    \]
    \item \textbf{Strict sub-extremality:} If $A(\Sigma) > 8\pi|J(\Sigma)|$ (strict inequality initially), then $A(t) > 8\pi|J|$ for all $t \in [0,1]$, and the sub-extremality factor satisfies
    \[
    1 - \frac{64\pi^2 J^2}{A(t)^2} \geq 1 - \frac{64\pi^2 J^2}{A(0)^2} > 0.
    \]
\end{enumerate}
\end{theorem}

\begin{remark}[No Cosmic Censorship Assumed]
This theorem does \textbf{not} assume Cosmic Censorship. It follows directly from the \textbf{proven} Dain--Reiris area-angular momentum inequality \cite{dain2011}, which is derived purely from the constraint equations and the stability of the MOTS. The Penrose inequality is sometimes viewed as evidence \emph{for} Cosmic Censorship, but our proof does not use Cosmic Censorship as a hypothesis.
\end{remark}

\begin{remark}[Verification of Dain--Reiris Hypotheses]\label{rem:dain-reiris-hypotheses}
The Dain--Reiris inequality \cite{dain2011} requires the following hypotheses on the surface $\Sigma$:
\begin{enumerate}
    \item[(DR1)] $\Sigma$ is a closed, embedded, axisymmetric 2-surface with $\Sigma \cong S^2$;
    \item[(DR2)] $\Sigma$ is a \textbf{stable} marginally outer trapped surface (MOTS);
    \item[(DR3)] The ambient initial data $(M, g, K)$ satisfies the dominant energy condition;
    \item[(DR4)] $\Sigma$ intersects the axis of symmetry at exactly two poles: $\Sigma \cap \Gamma = \{p_N, p_S\}$ (by topological necessity---see Lemma~\ref{lem:mots-axis}).
\end{enumerate}

We verify that our hypotheses (H1)--(H4) in Theorem~\ref{thm:main} imply (DR1)--(DR4):
\begin{itemize}
    \item \textbf{(DR1) Topology:} By the Galloway--Schoen theorem \cite{gallowayschoen2006}, a stable MOTS in data satisfying DEC has spherical topology. The outermost MOTS is automatically embedded.
    \item \textbf{(DR2) Stability:} This is hypothesis (H4) of Theorem~\ref{thm:main}.
    \item \textbf{(DR3) DEC:} This is hypothesis (H1) of Theorem~\ref{thm:main}.
    \item \textbf{(DR4) Axis intersection:} An axisymmetric $S^2$ must intersect the axis at two poles by the topological argument in Lemma~\ref{lem:mots-axis}. The twist term $\mathcal{T}$ vanishes at these poles since $\mathcal{T} \propto \rho^2$ and $\rho = 0$ on the axis (Lemma~\ref{lem:twist-bound-poles}).
\end{itemize}
Therefore, the Dain--Reiris inequality applies under our hypotheses.
\end{remark}

\begin{proof}
\textbf{Step 1: The Dain--Reiris inequality (proven theorem).}
For axisymmetric initial data satisfying DEC with a stable MOTS $\Sigma$, Dain and Reiris \cite{dain2011} proved:
\[
A(\Sigma) \geq 8\pi|J(\Sigma)|,
\]
with equality if and only if $\Sigma$ is isometric to the horizon of extreme Kerr. This is a \textbf{theorem}, not a conjecture, proven using variational methods on the space of axisymmetric surfaces.

\textbf{Step 2: Dain's mass-angular momentum inequality.}
For completeness, we note Dain \cite{dain2008} also proved:
\[
M_{\ADM} \geq \sqrt{|J|},
\]
with equality if and only if the data is a slice of extreme Kerr. This implies:
\[
|J| \leq M_{\ADM}^2 \quad \text{(sub-extremal bound on total angular momentum)}.
\]

\textbf{Step 3: Preservation along AMO flow.}
The Dain--Reiris inequality $A(\Sigma) \geq 8\pi|J(\Sigma)|$ is established in \cite{dain2011} using variational methods specific to MOTS. We do \textbf{not} re-derive this inequality here; instead, we show that once it holds at $t = 0$, it is \textbf{preserved} along the AMO flow by the following rigorous argument:

\begin{enumerate}
    \item[(i)] \textbf{Initial condition:} By the Dain--Reiris theorem \cite{dain2011}, the initial MOTS $\Sigma = \Sigma_0$ satisfies $A(0) \geq 8\pi|J(0)|$.
    
    \item[(ii)] \textbf{$J$ is conserved:} By Theorem~\ref{thm:J-conserve}, $J(t) = J(0) = J$ for all $t \in [0, 1]$.
    
    \item[(iii)] \textbf{$A$ is non-decreasing:} By the AMO area monotonicity, we establish that $A'(t) \geq 0$ for almost all $t \in (0,1)$. We provide a complete proof:
    
    \textit{Proof of area monotonicity.} Let $\Sigma_t = \{u = t\}$ be level sets of the $p$-harmonic potential $u$ on $(\tM, \tg)$ with $R_{\tg} \geq 0$. By Proposition~\ref{prop:amo-formula}, the AMO formula gives:
    \begin{equation}\label{eq:amo-formula-subext}
    A'(t) = \int_{\Sigma_t} \frac{1}{|\nabla u|}\left(R_{\tg} + 2|\mathring{h}|^2 + \frac{2}{(p-1)^2}\left(H - (p-1)\frac{\Delta u}{|\nabla u|}\right)^2\right) d\sigma.
    \end{equation}
    Each term in the integrand is non-negative:
    \begin{itemize}
        \item $R_{\tg} \geq 0$ by Theorem~\ref{thm:lich-exist} (AM-Lichnerowicz equation);
        \item $|\mathring{h}|^2 \geq 0$ (squared norm of traceless second fundamental form);
        \item The third term is a squared quantity, hence non-negative.
    \end{itemize}
    Since $|\nabla u| > 0$ on regular level sets (which comprise all but a measure-zero set of $t$ values by Sard's theorem), we conclude $A'(t) \geq 0$ for a.e.\ $t \in (0,1)$.
    
    \textit{Remarks on the derivation:}
    \begin{enumerate}
        \item The AMO formula \eqref{eq:amo-formula-subext} is derived using the Bochner identity, the $p$-harmonic equation, and integration by parts---see \cite[Theorem 3.1]{amo2022} or the self-contained derivation in Proposition~\ref{prop:amo-formula}.
        \item The condition $R_{\tg} \geq 0$ is \textbf{essential}: the conformal transformation $\tg = \phi^4 \bg$ with $\phi$ solving the AM-Lichnerowicz equation ensures $R_{\tg} = \Lambda_J \phi^{-12} \geq 0$. Without this, $R_{\tg}$ could be negative and area monotonicity would fail.
        \item For the limit $p \to 1^+$, the level sets approximate minimal surfaces, and the squared term involving $H$ and $\Delta u$ vanishes. The bound $A'(t) \geq \int_{\Sigma_t} R_{\tg}/|\nabla u|\, d\sigma \geq 0$ remains valid.
    \end{enumerate}
    
    \item[(iv)] \textbf{Conclusion:} Combining (i)--(iii):
    \[
    A(t) \geq A(0) \geq 8\pi|J| = 8\pi|J(t)| \quad \text{for all } t \in [0, 1].
    \]
\end{enumerate}

\textbf{Quantitative preservation of sub-extremality factor.}
For strictly sub-extremal initial data with $A(0) > 8\pi|J|$, define the sub-extremality factor:
\[
\mathcal{S}(t) := 1 - \frac{64\pi^2 J^2}{A(t)^2}.
\]
Since $A(t) \geq A(0)$ and $J(t) = J$ is constant:
\[
\mathcal{S}(t) = 1 - \frac{64\pi^2 J^2}{A(t)^2} \geq 1 - \frac{64\pi^2 J^2}{A(0)^2} = \mathcal{S}(0) > 0.
\]
The sub-extremality factor is \textbf{non-decreasing} along the flow and remains strictly positive if it starts strictly positive. This ensures the monotonicity formula (Theorem~\ref{thm:monotone}) has a non-negative integrand throughout the flow.

\textbf{Step 4: Note on the Dain--Reiris proof.}
For completeness, we summarize the key ingredients of the Dain--Reiris argument (which we cite but do not re-derive):
\begin{itemize}
    \item The proof uses the \textbf{stability operator} of the MOTS to establish positivity of certain geometric integrals.
    \item A key step is the \textbf{mass functional} technique: for axisymmetric surfaces, the angular momentum $J$ can be expressed as a boundary integral that, by the constraint equations and stability, is bounded by a multiple of the area.
    \item The explicit constant $8\pi$ arises from the geometry of the extreme Kerr horizon, which achieves equality.
\end{itemize}
See \cite[Section 3]{dain2011} for the complete variational argument.
\end{proof}

\begin{remark}[Necessity of MOTS Stability]\label{rem:stability-necessity}
The stability hypothesis on the outermost MOTS $\Sigma$ is used in \textbf{three distinct places} in the proof:

\begin{enumerate}
    \item \textbf{Jang equation blow-up (Theorem~\ref{thm:jang-exist}):} Stability ensures the Jang solution blows up logarithmically at $\Sigma$ with coefficient $C_0 = |\theta^-|/2 > 0$. For unstable MOTS, the Jang solution may exhibit more complicated behavior (e.g., oscillatory or non-monotonic blow-up).
    
    \item \textbf{Dain--Reiris inequality (Theorem~\ref{thm:subext}):} The proof of $A \geq 8\pi|J|$ in \cite{dain2011} relies on the stability condition through a variational argument. Unstable MOTS can violate this bound.
    
    \item \textbf{Cylindrical end geometry (Theorem~\ref{thm:jang-exist}(iii)):} Stability ensures the cylindrical end metric converges exponentially to $dt^2 + g_\Sigma$, with decay rate $\beta$ related to the spectral gap of the stability operator.
\end{enumerate}

\textbf{Can stability be relaxed?} It is an open question whether the AM-Penrose inequality holds for \textbf{unstable} outermost MOTS. The main obstacle is that the Dain--Reiris inequality can fail for unstable surfaces. For example, one could potentially construct initial data with an unstable MOTS having $A < 8\pi|J|$, in which case the monotonicity argument (Theorem~\ref{thm:monotone}) would break down since the factor $(1 - (8\pi|J|)^2/A(t)^2)$ could be negative.

However, for \textbf{outermost} MOTS (which are automatically weakly outer-trapped), there is some evidence that stability may be automatic in the axisymmetric case. This is related to the fact that axisymmetric deformations preserve the MOTS condition, limiting the possible instability directions. See \cite{anderssonmetzger2009} for related discussion.
\end{remark}

\begin{remark}[Independence from Cosmic Censorship]
The sub-extremality bound $A \geq 8\pi|J|$ is a \textbf{proven geometric inequality}, not an assumption. It follows from the constraint equations, the DEC, and the stability of the MOTS---all hypotheses that are verifiable for a given initial data set. The Penrose inequality proof does not invoke Cosmic Censorship in any form.
\end{remark}

\begin{mdframed}[linewidth=1pt, linecolor=blue!60!black, backgroundcolor=blue!3]
\begin{remark}[Clarification: Bootstrap Structure of the Sub-Extremality Argument]\label{rem:bootstrap-clarification}
A careful reader may wonder about a potential circularity: the Dain--Reiris inequality is proven for MOTS, but the level sets $\Sigma_t$ for $t > 0$ are \textbf{not} MOTS. How can we claim $A(t) \geq 8\pi|J|$ for all $t$?

\textbf{Resolution:} The argument is \textbf{not} a re-application of Dain--Reiris at each $t$. Instead:
\begin{enumerate}
    \item[(1)] \textbf{Initial bound (Dain--Reiris):} At $t = 0$, the outermost MOTS $\Sigma_0 = \Sigma$ satisfies $A(0) \geq 8\pi|J|$ by the Dain--Reiris theorem \cite{dain2011}. This is a \textbf{one-time application} at the MOTS only.
    
    \item[(2)] \textbf{Angular momentum conservation:} By Theorem~\ref{thm:J-conserve}, $J(t) = J$ is constant for all $t \in [0,1]$. This uses Stokes' theorem and the vacuum condition in the exterior.
    
    \item[(3)] \textbf{Area monotonicity (proven independently):} By the AMO theory \cite{amo2022}, the area $A(t)$ of level sets is non-decreasing: $A(t) \geq A(0)$. This follows from the $p$-harmonic structure and $R_{\tg} \geq 0$, and does \textbf{not} require $\Sigma_t$ to be a MOTS.
    
    \item[(4)] \textbf{Conclusion (algebraic):} Combining (1), (2), (3):
    \[
    A(t) \geq A(0) \geq 8\pi|J| = 8\pi|J(t)|.
    \]
    No circularity exists because the Dain--Reiris inequality is used only at $t = 0$, and the preservation for $t > 0$ follows from the independent monotonicity of area.
\end{enumerate}

\textbf{Key point:} The Dain--Reiris inequality and the AMO area monotonicity are \textbf{logically independent} theorems. Dain--Reiris applies to MOTS and uses MOTS-specific variational arguments. AMO area monotonicity applies to level sets of $p$-harmonic functions and uses the Bochner technique with $R_{\tg} \geq 0$. The sub-extremality preservation is the \textbf{combination} of these two independent results.
\end{remark}
\end{mdframed}

%=============================================================================
