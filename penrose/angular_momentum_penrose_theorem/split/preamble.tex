% The Angular Momentum Penrose Inequality
% Formatted for submission to Communications in Mathematical Physics
% December 2025
% 
% This file contains the preamble (packages, commands, settings)
% Include this file with % The Angular Momentum Penrose Inequality
% Formatted for submission to Communications in Mathematical Physics
% December 2025
% 
% This file contains the preamble (packages, commands, settings)
% Include this file with % The Angular Momentum Penrose Inequality
% Formatted for submission to Communications in Mathematical Physics
% December 2025
% 
% This file contains the preamble (packages, commands, settings)
% Include this file with % The Angular Momentum Penrose Inequality
% Formatted for submission to Communications in Mathematical Physics
% December 2025
% 
% This file contains the preamble (packages, commands, settings)
% Include this file with \input{preamble} before \begin{document}

% CMP uses standard article class with specific formatting
\usepackage[a4paper, margin=2.5cm]{geometry}
\usepackage{amsmath,amssymb,amsthm}
\usepackage{mathtools}
\usepackage[T1]{fontenc}
\usepackage{hyperref}
\usepackage{cleveref}
\usepackage{tikz}
\usepackage{booktabs}
\usepackage{enumitem}
\usepackage[hypcap=false]{caption}
\usepackage[expansion=false,protrusion=true,verbose=silent]{microtype}
\usepackage{mdframed}
\usepackage{lineno}
\usepackage{cite}

% Double spacing for review (CMP requirement)
\usepackage{setspace}
\doublespacing

% Line numbers for review
\linenumbers

\hypersetup{
    colorlinks=true,
    linkcolor=blue,
    citecolor=blue,
    urlcolor=blue,
    pdftitle={The Angular Momentum Penrose Inequality},
    pdfauthor={Da Xu},
    pdfkeywords={Penrose inequality, angular momentum, Kerr spacetime, MOTS, Jang equation}
}

% Theorem environments (CMP style)
\theoremstyle{plain}
\newtheorem{theorem}{Theorem}[section]
\newtheorem{conjecture}[theorem]{Conjecture}
\newtheorem{proposition}[theorem]{Proposition}
\newtheorem{lemma}[theorem]{Lemma}
\newtheorem{corollary}[theorem]{Corollary}

\theoremstyle{definition}
\newtheorem{definition}[theorem]{Definition}
\newtheorem{example}[theorem]{Example}

\theoremstyle{remark}
\newtheorem{remark}[theorem]{Remark}

% Custom commands (same as main paper)
\newcommand{\ADM}{\mathrm{ADM}}
\newcommand{\R}{\mathbb{R}}
\newcommand{\MOTS}{\mathrm{MOTS}}
\newcommand{\DEC}{\mathrm{DEC}}
\newcommand{\tr}{\mathrm{tr}}
\newcommand{\Ric}{\mathrm{Ric}}
\newcommand{\Rm}{\mathrm{Rm}}
\newcommand{\Div}{\mathrm{div}}
\newcommand{\tM}{\tilde{M}}
\newcommand{\tg}{\tilde{g}}
\newcommand{\bg}{\bar{g}}
\newcommand{\bM}{\bar{M}}
\newcommand{\momdens}{\boldsymbol{j}}
\newcommand{\Jang}{J}
% Notation clarification: We use distinct symbols to avoid confusion:
%   - \Hoelder (α_H) for the Hölder exponent in C^{k,α} spaces (a regularity parameter in (0,1))
%   - \Komarform (α_J) for the Komar 1-form encoding angular momentum
% Both are denoted α in standard literature but appear in different contexts.
\newcommand{\Hoelder}{\alpha_H}
\newcommand{\Komarform}{\alpha_J}

% Allow page breaks in equations
\allowdisplaybreaks

% Help with overfull boxes
\emergencystretch=1em
\tolerance=1000

\setcounter{tocdepth}{2}
\raggedbottom
 before \begin{document}

% CMP uses standard article class with specific formatting
\usepackage[a4paper, margin=2.5cm]{geometry}
\usepackage{amsmath,amssymb,amsthm}
\usepackage{mathtools}
\usepackage[T1]{fontenc}
\usepackage{hyperref}
\usepackage{cleveref}
\usepackage{tikz}
\usepackage{booktabs}
\usepackage{enumitem}
\usepackage[hypcap=false]{caption}
\usepackage[expansion=false,protrusion=true,verbose=silent]{microtype}
\usepackage{mdframed}
\usepackage{lineno}
\usepackage{cite}

% Double spacing for review (CMP requirement)
\usepackage{setspace}
\doublespacing

% Line numbers for review
\linenumbers

\hypersetup{
    colorlinks=true,
    linkcolor=blue,
    citecolor=blue,
    urlcolor=blue,
    pdftitle={The Angular Momentum Penrose Inequality},
    pdfauthor={Da Xu},
    pdfkeywords={Penrose inequality, angular momentum, Kerr spacetime, MOTS, Jang equation}
}

% Theorem environments (CMP style)
\theoremstyle{plain}
\newtheorem{theorem}{Theorem}[section]
\newtheorem{conjecture}[theorem]{Conjecture}
\newtheorem{proposition}[theorem]{Proposition}
\newtheorem{lemma}[theorem]{Lemma}
\newtheorem{corollary}[theorem]{Corollary}

\theoremstyle{definition}
\newtheorem{definition}[theorem]{Definition}
\newtheorem{example}[theorem]{Example}

\theoremstyle{remark}
\newtheorem{remark}[theorem]{Remark}

% Custom commands (same as main paper)
\newcommand{\ADM}{\mathrm{ADM}}
\newcommand{\R}{\mathbb{R}}
\newcommand{\MOTS}{\mathrm{MOTS}}
\newcommand{\DEC}{\mathrm{DEC}}
\newcommand{\tr}{\mathrm{tr}}
\newcommand{\Ric}{\mathrm{Ric}}
\newcommand{\Rm}{\mathrm{Rm}}
\newcommand{\Div}{\mathrm{div}}
\newcommand{\tM}{\tilde{M}}
\newcommand{\tg}{\tilde{g}}
\newcommand{\bg}{\bar{g}}
\newcommand{\bM}{\bar{M}}
\newcommand{\momdens}{\boldsymbol{j}}
\newcommand{\Jang}{J}
% Notation clarification: We use distinct symbols to avoid confusion:
%   - \Hoelder (α_H) for the Hölder exponent in C^{k,α} spaces (a regularity parameter in (0,1))
%   - \Komarform (α_J) for the Komar 1-form encoding angular momentum
% Both are denoted α in standard literature but appear in different contexts.
\newcommand{\Hoelder}{\alpha_H}
\newcommand{\Komarform}{\alpha_J}

% Allow page breaks in equations
\allowdisplaybreaks

% Help with overfull boxes
\emergencystretch=1em
\tolerance=1000

\setcounter{tocdepth}{2}
\raggedbottom
 before \begin{document}

% CMP uses standard article class with specific formatting
\usepackage[a4paper, margin=2.5cm]{geometry}
\usepackage{amsmath,amssymb,amsthm}
\usepackage{mathtools}
\usepackage[T1]{fontenc}
\usepackage{hyperref}
\usepackage{cleveref}
\usepackage{tikz}
\usepackage{booktabs}
\usepackage{enumitem}
\usepackage[hypcap=false]{caption}
\usepackage[expansion=false,protrusion=true,verbose=silent]{microtype}
\usepackage{mdframed}
\usepackage{lineno}
\usepackage{cite}

% Double spacing for review (CMP requirement)
\usepackage{setspace}
\doublespacing

% Line numbers for review
\linenumbers

\hypersetup{
    colorlinks=true,
    linkcolor=blue,
    citecolor=blue,
    urlcolor=blue,
    pdftitle={The Angular Momentum Penrose Inequality},
    pdfauthor={Da Xu},
    pdfkeywords={Penrose inequality, angular momentum, Kerr spacetime, MOTS, Jang equation}
}

% Theorem environments (CMP style)
\theoremstyle{plain}
\newtheorem{theorem}{Theorem}[section]
\newtheorem{conjecture}[theorem]{Conjecture}
\newtheorem{proposition}[theorem]{Proposition}
\newtheorem{lemma}[theorem]{Lemma}
\newtheorem{corollary}[theorem]{Corollary}

\theoremstyle{definition}
\newtheorem{definition}[theorem]{Definition}
\newtheorem{example}[theorem]{Example}

\theoremstyle{remark}
\newtheorem{remark}[theorem]{Remark}

% Custom commands (same as main paper)
\newcommand{\ADM}{\mathrm{ADM}}
\newcommand{\R}{\mathbb{R}}
\newcommand{\MOTS}{\mathrm{MOTS}}
\newcommand{\DEC}{\mathrm{DEC}}
\newcommand{\tr}{\mathrm{tr}}
\newcommand{\Ric}{\mathrm{Ric}}
\newcommand{\Rm}{\mathrm{Rm}}
\newcommand{\Div}{\mathrm{div}}
\newcommand{\tM}{\tilde{M}}
\newcommand{\tg}{\tilde{g}}
\newcommand{\bg}{\bar{g}}
\newcommand{\bM}{\bar{M}}
\newcommand{\momdens}{\boldsymbol{j}}
\newcommand{\Jang}{J}
% Notation clarification: We use distinct symbols to avoid confusion:
%   - \Hoelder (α_H) for the Hölder exponent in C^{k,α} spaces (a regularity parameter in (0,1))
%   - \Komarform (α_J) for the Komar 1-form encoding angular momentum
% Both are denoted α in standard literature but appear in different contexts.
\newcommand{\Hoelder}{\alpha_H}
\newcommand{\Komarform}{\alpha_J}

% Allow page breaks in equations
\allowdisplaybreaks

% Help with overfull boxes
\emergencystretch=1em
\tolerance=1000

\setcounter{tocdepth}{2}
\raggedbottom
 before \begin{document}

% CMP uses standard article class with specific formatting
\usepackage[a4paper, margin=2.5cm]{geometry}
\usepackage{amsmath,amssymb,amsthm}
\usepackage{mathtools}
\usepackage[T1]{fontenc}
\usepackage{hyperref}
\usepackage{cleveref}
\usepackage{tikz}
\usepackage{booktabs}
\usepackage{enumitem}
\usepackage[hypcap=false]{caption}
\usepackage[expansion=false,protrusion=true,verbose=silent]{microtype}
\usepackage{mdframed}
\usepackage{lineno}
\usepackage{cite}

% Double spacing for review (CMP requirement)
\usepackage{setspace}
\doublespacing

% Line numbers for review
\linenumbers

\hypersetup{
    colorlinks=true,
    linkcolor=blue,
    citecolor=blue,
    urlcolor=blue,
    pdftitle={The Angular Momentum Penrose Inequality},
    pdfauthor={Da Xu},
    pdfkeywords={Penrose inequality, angular momentum, Kerr spacetime, MOTS, Jang equation}
}

% Theorem environments (CMP style)
\theoremstyle{plain}
\newtheorem{theorem}{Theorem}[section]
\newtheorem{conjecture}[theorem]{Conjecture}
\newtheorem{proposition}[theorem]{Proposition}
\newtheorem{lemma}[theorem]{Lemma}
\newtheorem{corollary}[theorem]{Corollary}

\theoremstyle{definition}
\newtheorem{definition}[theorem]{Definition}
\newtheorem{example}[theorem]{Example}

\theoremstyle{remark}
\newtheorem{remark}[theorem]{Remark}

% Custom commands (same as main paper)
\newcommand{\ADM}{\mathrm{ADM}}
\newcommand{\R}{\mathbb{R}}
\newcommand{\MOTS}{\mathrm{MOTS}}
\newcommand{\DEC}{\mathrm{DEC}}
\newcommand{\tr}{\mathrm{tr}}
\newcommand{\Ric}{\mathrm{Ric}}
\newcommand{\Rm}{\mathrm{Rm}}
\newcommand{\Div}{\mathrm{div}}
\newcommand{\tM}{\tilde{M}}
\newcommand{\tg}{\tilde{g}}
\newcommand{\bg}{\bar{g}}
\newcommand{\bM}{\bar{M}}
\newcommand{\momdens}{\boldsymbol{j}}
\newcommand{\Jang}{J}
% Notation clarification: We use distinct symbols to avoid confusion:
%   - \Hoelder (α_H) for the Hölder exponent in C^{k,α} spaces (a regularity parameter in (0,1))
%   - \Komarform (α_J) for the Komar 1-form encoding angular momentum
% Both are denoted α in standard literature but appear in different contexts.
\newcommand{\Hoelder}{\alpha_H}
\newcommand{\Komarform}{\alpha_J}

% Allow page breaks in equations
\allowdisplaybreaks

% Help with overfull boxes
\emergencystretch=1em
\tolerance=1000

\setcounter{tocdepth}{2}
\raggedbottom
