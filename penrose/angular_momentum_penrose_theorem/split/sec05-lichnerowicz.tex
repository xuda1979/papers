\section{Stage 2: AM-Lichnerowicz Equation}\label{sec:lichnerowicz}
%=============================================================================

\subsection{The Conformal Equation}

On the Jang manifold $(\bM, \bg)$, we solve a modified Lichnerowicz equation that accounts for angular momentum. The cylindrical end structure from Theorem~\ref{thm:jang-exist} requires Lockhart--McOwen weighted Sobolev spaces for Fredholm theory.

\begin{definition}[Weighted Sobolev Spaces on Cylindrical Ends]\label{def:weighted-sobolev-cyl}
Let $(\bM, \bg)$ have cylindrical ends $\mathcal{C} \cong [0,\infty) \times \Sigma$ with coordinate $t$ and cross-section $(\Sigma, g_\Sigma)$. For $k \in \mathbb{N}_0$, $p \in [1,\infty)$, and weight $\beta \in \mathbb{R}$, define the weighted Sobolev space:
\[
W^{k,p}_\beta(\bM) := \{u \in W^{k,p}_{\mathrm{loc}}(\bM) : \|u\|_{W^{k,p}_\beta} < \infty\},
\]
where the norm on the cylindrical end is:
\[
\|u\|_{W^{k,p}_\beta(\mathcal{C})}^p := \sum_{j=0}^{k} \int_0^\infty \int_\Sigma e^{-\beta p t} |\nabla^j u|^p \, dA_{g_\Sigma} \, dt,
\]
with $|\nabla^j u|$ denoting the norm of the $j$-th covariant derivative. In the asymptotically flat end, the standard weighted norm from Definition~\ref{def:weighted-holder} applies.

A function $u \in W^{k,p}_\beta$ with $\beta < 0$ decays as $t \to \infty$ on the cylindrical end: $|u(t, \cdot)| = O(e^{\beta t}) \to 0$. For $\beta > 0$, such functions may grow. The Lockhart--McOwen theory \cite{lockhartmccowen1985} shows that the Laplacian $\Delta_{\bg}: W^{k+2,p}_\beta \to W^{k,p}_\beta$ is Fredholm when $\beta$ avoids the \textbf{indicial roots}---values determined by the spectrum of the cross-sectional Laplacian $\Delta_\Sigma$.
\end{definition}

\begin{remark}[Compatibility of Function Spaces]\label{rem:space-compatibility}
The Jang manifold $(\bM, \bg)$ has two distinct asymptotic regions requiring different function space frameworks:
\begin{enumerate}[label=(\roman*)]
    \item \textbf{Asymptotically flat end:} Weighted H\"older spaces $C^{k,\Hoelder}_{-\tau}$ with polynomial weight $r^{-\tau}$ (Definition~\ref{def:weighted-holder});
    \item \textbf{Cylindrical end:} Weighted Sobolev spaces $W^{k,p}_\beta$ with exponential weight $e^{\beta t}$ (Definition~\ref{def:weighted-sobolev-cyl}).
\end{enumerate}
These frameworks are compatible on the transition region $\{R_0 \leq r \leq 2R_0\}$ (equivalently $\{0 \leq t \leq T_0\}$) in the following sense: by Sobolev embedding, $W^{k+1,2}_\beta \hookrightarrow C^{k,\Hoelder}$ locally, and both norms are equivalent (up to constants depending on $R_0$) on the compact overlap region. This allows elliptic estimates to be ``glued'' across the transition using standard partition-of-unity arguments. The key point is that the Fredholm index is determined by the asymptotic behavior at both ends, not the transition region.
\end{remark}

\begin{definition}[AM-Lichnerowicz Operator]\label{def:am-lich}
The angular-momentum-modified Lichnerowicz equation is:
\begin{equation}\label{eq:am-lich}
L_{AM}[\phi] := -8\Delta_{\bg}\phi + R_{\bg}\phi - \Lambda_J \phi^{-7} = 0,
\end{equation}
where $\Lambda_J = \frac{1}{8}|\mathcal{S}_{(g,K)}|^2_{\bg} \geq 0$ is the Kerr deviation contribution (Definition~\ref{def:Lambda-J}). The \textbf{negative} sign in front of $\Lambda_J$ ensures that the conformal scalar curvature $R_{\tg} = \Lambda_J \phi^{-12} \geq 0$.

\textbf{Key property:} For Kerr initial data, $\Lambda_J = 0$ (since $\mathcal{S}_{(g,K)} = 0$), and the equation reduces to the standard Lichnerowicz equation $-8\Delta_{\bg}\phi + R_{\bg}\phi = 0$.
\end{definition}

\begin{remark}[Sign Convention Verification]\label{rem:sign-verification}
We verify the sign conventions in the AM-Lichnerowicz equation:
\begin{enumerate}[label=\textup{(\roman*)}]
    \item \textbf{Conformal transformation formula:} Under $\tg = \phi^4 \bg$, the scalar curvatures are related by:
    \[
    R_{\tg} = \phi^{-4} R_{\bg} - 8\phi^{-5}\Delta_{\bg}\phi = \phi^{-5}(R_{\bg}\phi - 8\Delta_{\bg}\phi).
    \]
    \item \textbf{AM-Lichnerowicz rearrangement:} From \eqref{eq:am-lich}:
    \[
    -8\Delta_{\bg}\phi + R_{\bg}\phi = \Lambda_J \phi^{-7} \quad \Rightarrow \quad R_{\tg} = \phi^{-5} \cdot \Lambda_J \phi^{-7} = \Lambda_J \phi^{-12}.
    \]
    \item \textbf{Positivity:} Since $\Lambda_J = \frac{1}{8}|\mathcal{S}_{(g,K)}|^2 \geq 0$ and $\phi > 0$, we have $R_{\tg} \geq 0$ automatically.
    \item \textbf{Strict positivity:} $R_{\tg} > 0$ where $\mathcal{S}_{(g,K)} \neq 0$, i.e., where the data deviates from Kerr geometry.
    \item \textbf{Equality case:} For Kerr data, $\Lambda_J = 0$, so $R_{\tg} = 0$ and the monotonicity integrand vanishes.
\end{enumerate}
The convention matches the standard Lichnerowicz equation $-8\Delta\phi + R\phi = 0$ (for $R_{\tg} = 0$), with the $\Lambda_J \phi^{-7}$ term producing positive conformal scalar curvature for non-Kerr data.
\end{remark}

\begin{mdframed}[linewidth=1pt, linecolor=black!50, backgroundcolor=yellow!5, nobreak=false, skipabove=\baselineskip, skipbelow=\baselineskip]
\begin{lemma}[Well-Definedness of $\Lambda_J$]\label{lem:Lambda-J-welldef}
The angular momentum source term $\Lambda_J = \frac{1}{8}|\mathcal{S}_{(g,K)}|^2_{\bg}$ is a \textbf{well-defined, coordinate-independent, non-negative scalar function} on any asymptotically flat, axisymmetric vacuum initial data $(M^3, g, K)$. We provide two equivalent constructions: an \textbf{intrinsic algebraic definition} (simpler, self-contained) and a \textbf{PDE-based extension} (connecting to Mars--Simon theory).

\textbf{Construction A: Intrinsic Algebraic Definition (Primary).}
\begin{enumerate}[label=\textup{(\roman*)}]
    \item \textbf{Electric and magnetic Weyl tensors:} Define \textbf{algebraically} from $(g, K)$:
    \begin{align*}
    E_{ij} &:= R_{ij} - \tfrac{1}{3}Rg_{ij} + (\tr K)K_{ij} - K_{ik}K^k{}_j, \\
    B_{ij} &:= \epsilon_i{}^{kl}\nabla_k K_{lj}.
    \end{align*}
    These are intrinsic to $(g, K)$---no embedding or evolution required.
    
    \item \textbf{Reference Kerr tensors via asymptotic expansion:} For asymptotically flat data with ADM mass $M$ and Komar angular momentum $J$, define the \textbf{reference Kerr Weyl tensors} $E^{\mathrm{Kerr}}_{ij}(M,J)$ and $B^{\mathrm{Kerr}}_{ij}(M,J)$ by the \textbf{explicit asymptotic series}:
    \begin{align*}
    E^{\mathrm{Kerr}}_{ij} &= \frac{M}{r^3}\left(\delta_{ij} - 3\hat{r}_i\hat{r}_j\right) + \frac{3Ma^2}{r^5}\left(\text{spin-2 harmonics}\right) + O(r^{-6}), \\
    B^{\mathrm{Kerr}}_{ij} &= \frac{3Ma}{r^4}\epsilon_{(i}{}^{kl}\hat{r}_{j)}\hat{r}_k \hat{z}_l + O(r^{-5}),
    \end{align*}
    where $a = J/M$, $\hat{r} = x/r$, and $\hat{z}$ is the symmetry axis unit vector. These are the unique symmetric trace-free tensors with Kerr asymptotics determined by $(M, J)$.
    
    \item \textbf{Kerr deviation tensor:} Define pointwise:
    \[
    \mathcal{S}_{(g,K),ij} := (E_{ij} - E^{\mathrm{Kerr}}_{ij}) + i(B_{ij} - B^{\mathrm{Kerr}}_{ij}).
    \]
    This is well-defined for $r > r_0$ (exterior region) where the asymptotic expansions converge.
    
    \item \textbf{Angular momentum source:} Define
    \[
    \Lambda_J := \frac{1}{8}|\mathcal{S}_{(g,K)}|^2_{\bg} = \frac{1}{8}\left(|E - E^{\mathrm{Kerr}}|^2 + |B - B^{\mathrm{Kerr}}|^2\right).
    \]
\end{enumerate}

\textbf{Key properties} (immediate from the construction):
\begin{enumerate}[label=\textup{(\alph*)}]
    \item $\Lambda_J \geq 0$ everywhere (squared norm);
    \item $\Lambda_J$ is coordinate-independent (tensor norm);
    \item $\Lambda_J = O(r^{-4-2\tau})$ for asymptotically flat data with decay $\tau > 1/2$;
    \item For Kerr slices: $E = E^{\mathrm{Kerr}}$ and $B = B^{\mathrm{Kerr}}$ exactly, so $\Lambda_J = 0$;
    \item $\Lambda_J = 0$ iff $(M, g, K)$ is a Kerr slice (Theorem~\ref{thm:kerr-characterization}).
\end{enumerate}

\textbf{Construction B: PDE Extension (for Interior Region).}
For the strong-field region (e.g., near the MOTS where the asymptotic expansion may not converge), we extend $E^{\mathrm{Kerr}}$ and $B^{\mathrm{Kerr}}$ to all of $M$ via:
\begin{enumerate}[label=\textup{(\roman*)}]
    \item \textbf{Constraint propagation:} The constraint equations for vacuum data imply the \textbf{Codazzi--Mainardi identity}:
    \[
    \nabla^j E_{ij} = \epsilon_{ijk}K^{jl}B^k{}_l, \qquad \nabla^j B_{ij} = -\epsilon_{ijk}K^{jl}E^k{}_l.
    \]
    This is a \textbf{determined system} (not elliptic in the standard sense, but constrained by the Bianchi identity).
    
    \item \textbf{Well-posedness via harmonic analysis:} Following \cite{backdahl2010a, backdahl2010b}, the reference tensors $(E^{\mathrm{Kerr}}, B^{\mathrm{Kerr}})$ satisfying the Codazzi--Mainardi system with Kerr asymptotics are \textbf{uniquely determined} by $(M, J)$. The proof uses:
    \begin{itemize}
        \item Decomposition into spherical harmonics on large spheres;
        \item The Codazzi--Mainardi system determines the radial evolution of each harmonic mode;
        \item Uniqueness follows from the decay conditions at infinity.
    \end{itemize}
    
    \item \textbf{Smoothness:} The extended tensors $(E^{\mathrm{Kerr}}, B^{\mathrm{Kerr}})$ are smooth on $M \setminus \Sigma$ and continuous up to $\Sigma$.
\end{enumerate}

\medskip
\noindent\textcolor{green!60!black}{\rule{\linewidth}{1pt}}
\smallskip

\noindent\textbf{Rigorous Well-Posedness of the Interior Extension (Construction B).}

We provide a complete proof that the reference tensors $(E^{\mathrm{Kerr}}, B^{\mathrm{Kerr}})$ extend uniquely to the interior region. The argument has three parts.

\textit{Part 1: System structure.} The Codazzi--Mainardi system for symmetric trace-free tensors $(E, B)$ is:
\begin{align}
\nabla^j E_{ij} &= \epsilon_{ijk}K^{jl}B^k{}_l =: F_i(E, B, K), \label{eq:CM-E}\\
\nabla^j B_{ij} &= -\epsilon_{ijk}K^{jl}E^k{}_l =: G_i(E, B, K). \label{eq:CM-B}
\end{align}
Writing $(E, B) = (E^{\mathrm{Kerr}}, B^{\mathrm{Kerr}})$, the system becomes a \emph{linear} first-order system with coefficients depending on $(g, K)$.

\textit{Part 2: Spherical harmonic decomposition.} On a sphere $S_r$ of radius $r$, decompose:
\[
E_{ij}|_{S_r} = \sum_{\ell \geq 2} \sum_{|m| \leq \ell} E_{\ell m}(r) Y^{\ell m}_{ij}(\theta, \phi), \quad
B_{ij}|_{S_r} = \sum_{\ell \geq 2} \sum_{|m| \leq \ell} B_{\ell m}(r) Y^{\ell m}_{ij}(\theta, \phi),
\]
where $Y^{\ell m}_{ij}$ are the symmetric trace-free tensor spherical harmonics. The Codazzi--Mainardi system \eqref{eq:CM-E}--\eqref{eq:CM-B} becomes a system of ODEs for the radial coefficients:
\begin{equation}\label{eq:radial-ODE}
\frac{d}{dr}\begin{pmatrix} E_{\ell m} \\ B_{\ell m} \end{pmatrix} = A_\ell(r) \begin{pmatrix} E_{\ell m} \\ B_{\ell m} \end{pmatrix} + \text{(lower order in } \ell),
\end{equation}
where $A_\ell(r)$ is a matrix depending on $(\ell, r, g, K)$.

\textit{Part 3: Well-posedness via ODE theory.} 
\begin{enumerate}[label=(\alph*)]
    \item \textbf{Boundary data at infinity:} The Kerr asymptotics determine $(E^{\mathrm{Kerr}}_{\ell m}, B^{\mathrm{Kerr}}_{\ell m})|_{r=\infty}$ for each mode. Explicitly:
    \[
    E^{\mathrm{Kerr}}_{2,0}(r) = \frac{M}{r^3}(1 + O(r^{-2})), \quad B^{\mathrm{Kerr}}_{2,0}(r) = \frac{3Ma}{r^4}(1 + O(r^{-1})),
    \]
    with higher modes decaying faster.
    
    \item \textbf{Inward integration:} Given boundary values at $r = r_0$ (sufficiently large), the ODE system \eqref{eq:radial-ODE} has a unique solution by Picard--Lindel\"of. The solution extends smoothly to any $r > 0$ away from the axis.
    
    \item \textbf{Axis regularity:} For axisymmetric data ($m = 0$ modes only), the tensor spherical harmonics have the form $Y^{\ell 0}_{ij} \propto P_\ell(\cos\theta) \times (\text{angular structure})$. The Legendre functions $P_\ell$ are smooth at the poles $\theta = 0, \pi$, so $(E^{\mathrm{Kerr}}, B^{\mathrm{Kerr}})$ extend smoothly to the axis.
    
    \item \textbf{Near-MOTS behavior:} As $r \to r_{\mathrm{MOTS}}$, the coefficients $A_\ell(r)$ remain bounded (they depend on $g, K$, which are smooth up to the MOTS). The ODE solution $(E^{\mathrm{Kerr}}_{\ell m}, B^{\mathrm{Kerr}}_{\ell m})$ therefore extends continuously to $r = r_{\mathrm{MOTS}}$.
\end{enumerate}

\textit{Uniqueness:} Suppose $(\tilde{E}, \tilde{B})$ is another extension satisfying \eqref{eq:CM-E}--\eqref{eq:CM-B} with the same Kerr asymptotics. Then the difference $(\delta E, \delta B) := (\tilde{E} - E^{\mathrm{Kerr}}, \tilde{B} - B^{\mathrm{Kerr}})$ satisfies the homogeneous system with zero boundary data at infinity. By ODE uniqueness (backward from $r = \infty$), $(\delta E, \delta B) = (0, 0)$.

\smallskip
\noindent\textcolor{green!60!black}{\rule{\linewidth}{1pt}}
\medskip

\textbf{Explicit interior construction procedure:}
The reference Kerr tensors $E^{\mathrm{Kerr}}_{ij}$ and $B^{\mathrm{Kerr}}_{ij}$ can be extended to the interior region $\{r < r_0\}$ via the following steps:
\begin{enumerate}[label=(\alph*)]
    \item \textbf{Boundary data:} On a large sphere $S_{r_0}$, compute the asymptotic values $(E^{\mathrm{Kerr}}, B^{\mathrm{Kerr}})|_{S_{r_0}}$ from the explicit Kerr formulas.
    \item \textbf{Inward integration:} Solve the Codazzi--Mainardi system inward from $S_{r_0}$ toward the MOTS. Since the system is first-order in the radial direction (after harmonic decomposition), this is a well-posed ODE for each harmonic mode.
    \item \textbf{Regularity at axis:} The axisymmetry condition $\mathcal{L}_\eta E^{\mathrm{Kerr}} = \mathcal{L}_\eta B^{\mathrm{Kerr}} = 0$ constrains the harmonic modes to those compatible with axis regularity (only $m = 0$ azimuthal modes for the scalar quantities).
\end{enumerate}
The resulting $(E^{\mathrm{Kerr}}, B^{\mathrm{Kerr}})$ are smooth throughout $M \setminus \Sigma$ and satisfy the Codazzi--Mainardi equations by construction.

\textbf{Equivalence:} Constructions A and B agree in the overlap region $\{r > r_0\}$ by uniqueness of the asymptotic expansion. The PDE extension provides values in the interior.
\end{lemma}

\begin{proof}[Proof]
We verify the key claims.

\textbf{Step 1: Intrinsic definition of $(E, B)$.}
The formulas for $E_{ij}$ and $B_{ij}$ involve only:
\begin{itemize}
    \item The Ricci tensor $R_{ij}$ (determined by $g$);
    \item The extrinsic curvature $K_{ij}$ (given data);
    \item The Levi-Civita connection $\nabla$ of $g$.
\end{itemize}
No embedding into a spacetime is required---these are the \textbf{Gauss--Codazzi projections} of the spacetime Weyl tensor onto the initial surface, but computed intrinsically. For $(g, K) \in C^{k,\Hoelder}_{-\tau} \times C^{k-1,\Hoelder}_{-\tau-1}$ with $k \geq 3$, we have $(E, B) \in C^{k-2,\Hoelder}_{-\tau-2}(M; S^2_0 T^*M)$.

\textbf{Step 2: Asymptotic matching.} For asymptotically flat data with ADM mass $M$ and Komar angular momentum $J$, the reference Kerr tensors are:
\begin{align*}
E^{\mathrm{Kerr}}_{ij} &= \frac{M}{r^3}\left(\delta_{ij} - 3\hat{r}_i\hat{r}_j\right) + \frac{3Ma^2}{r^5}\left(\text{spin-2 harmonics}\right) + O(r^{-6}), \\
B^{\mathrm{Kerr}}_{ij} &= \frac{3Ma}{r^4}\epsilon_{(i}{}^{kl}\hat{r}_{j)}\hat{r}_k \hat{z}_l + O(r^{-5}),
\end{align*}
where $a = J/M$, $\hat{r} = x/r$, and $\hat{z}$ is the symmetry axis unit vector. The Kerr deviation tensor is then:
\[
\mathcal{S}_{(g,K),ij} = (E_{ij} - E^{\mathrm{Kerr}}_{ij}) + i(B_{ij} - B^{\mathrm{Kerr}}_{ij}).
\]
For asymptotically flat data, the deviation tensor satisfies:
\[
|\mathcal{S}_{(g,K)}|_g = O(r^{-\tau-2}) - O(r^{-3}) = O(r^{-2-\tau})
\]
(the slower decay dominates). Therefore:
\[
\Lambda_J = \frac{1}{8}|\mathcal{S}_{(g,K)}|^2_g = O(r^{-4-2\tau}).
\]

\textbf{Step 3: Characterization of Kerr.}
The condition $\mathcal{S}_{(g,K)} = 0$ is equivalent to $(E, B) = (E^{\mathrm{Kerr}}, B^{\mathrm{Kerr}})$. By the Mars--Simon tensor characterization \cite{mars1999, simon1984}, this holds iff the data is a Kerr slice. The key theorem is:

\begin{quote}
\textbf{Theorem} (Mars \cite{mars1999}, B\"ackdahl--Valiente Kroon \cite{backdahl2010a}): \textit{Let $(M, g, K)$ be asymptotically flat, axisymmetric, vacuum initial data. Then $\mathcal{S}_{(g,K)} = 0$ if and only if $(M, g, K)$ is isometric to a spacelike slice of the Kerr spacetime.}
\end{quote}

This is proven by showing that $\mathcal{S}_{(g,K)} = 0$ implies the Simon tensor vanishes \cite{simon1984}, which characterizes Kerr among stationary axisymmetric vacuum spacetimes. The initial data version follows from the Killing Initial Data (KID) framework \cite{beigchrusciel1996}.

\textbf{Step 4: Regularity on Jang manifold.}
On the Jang manifold $(\bM, \bg)$, the norm $|\mathcal{S}_{(g,K)}|_{\bg}$ is computed using the Jang metric. Since $\bg = g + df \otimes df$ with $|df| < \infty$ away from $\Sigma$, the norm $|\cdot|_{\bg}$ is equivalent to $|\cdot|_g$ on compact sets. The decay $\Lambda_J = O(r^{-4-2\tau})$ ensures integrability.
\end{proof}
\end{mdframed}

\begin{lemma}[Regularity of $\Lambda_J$ in Weighted Spaces]\label{lem:Lambda-J-regularity}
Let $(M, g, K)$ be asymptotically flat, axisymmetric vacuum initial data with decay rate $\tau > 1/2$ and outermost strictly stable MOTS $\Sigma$. Let $(\bM, \bg)$ be the Jang manifold with cylindrical end $\mathcal{C} \cong [0,\infty)_t \times \Sigma$. Then the angular momentum source term $\Lambda_J = \frac{1}{8}|\mathcal{S}_{(g,K)}|^2_{\bg}$ satisfies:
\begin{enumerate}[label=\textup{(\roman*)}]
    \item \textbf{Global boundedness:} $\Lambda_J \in L^\infty(\bM)$ with explicit bound
    \[
    \|\Lambda_J\|_{L^\infty(\bM)} \leq C(g, K) < \infty,
    \]
    where $C(g, K)$ depends only on the $C^{2,\Hoelder}$ norms of the initial data.
    
    \item \textbf{Asymptotic decay:} On the asymptotically flat end,
    \[
    \Lambda_J(x) = O(r^{-4-2\tau}) \quad \text{as } r \to \infty.
    \]
    
    \item \textbf{Cylindrical end decay:} On the cylindrical end $\mathcal{C}$,
    \[
    \Lambda_J(t, y) = O(e^{-\beta_0 t}) \quad \text{as } t \to \infty,
    \]
    where $\beta_0 = 2\sqrt{\lambda_1(L_\Sigma)} > 0$ is the cylindrical decay rate from Theorem~\ref{thm:jang-exist}.
    
    \item \textbf{Weighted space membership:} For any $\beta \in (-\beta_0/2, 0)$ and $k \geq 0$:
    \[
    \Lambda_J \in W^{k,2}_\beta(\bM) \cap C^{k,\Hoelder}_{-4-2\tau}(\bM).
    \]
    
    \item \textbf{Interior extension boundedness:} The reference Kerr tensors $(E^{\mathrm{Kerr}}, B^{\mathrm{Kerr}})$ obtained via Construction B (inward ODE integration) satisfy:
    \[
    |E^{\mathrm{Kerr}}|_{\bg} + |B^{\mathrm{Kerr}}|_{\bg} \leq C(M, J, g) < \infty
    \]
    throughout $\bM$, including the region near the MOTS.
\end{enumerate}
\end{lemma}

\begin{proof}
We establish each bound systematically.

\textbf{Step 1: Boundedness of the electric and magnetic Weyl tensors.}
The tensors $E_{ij}$ and $B_{ij}$ (Definition~\ref{def:EB-weyl} in Appendix~\ref{app:mars-simon}) are computed algebraically from $(g, K)$:
\begin{align*}
E_{ij} &= R_{ij} - \tfrac{1}{3}Rg_{ij} + (\tr K)K_{ij} - K_{ik}K^k{}_j, \\
B_{ij} &= \epsilon_i{}^{kl}\nabla_k K_{lj}.
\end{align*}
For initial data $(g, K) \in C^{2,\Hoelder}_{-\tau} \times C^{1,\Hoelder}_{-\tau-1}$ with $\tau > 1/2$:
\begin{itemize}
    \item $R_{ij} \in C^{0,\Hoelder}_{-\tau-2}$ (two derivatives of $g$);
    \item $K_{ik}K^k{}_j \in C^{0,\Hoelder}_{-2\tau-2}$ (products of $K$);
    \item $\nabla_k K_{lj} \in C^{0,\Hoelder}_{-\tau-2}$ (one derivative of $K$).
\end{itemize}
Therefore $|E|_g, |B|_g \in C^{0,\Hoelder}_{-\tau-2}(M)$. On any compact set $K \subset M$, these are bounded: $|E|_g, |B|_g \leq C_K < \infty$.

\textbf{Step 2: Asymptotic decay at infinity.}
The reference Kerr tensors have asymptotic behavior (from the explicit formulas in Lemma~\ref{lem:Lambda-J-welldef}):
\[
|E^{\mathrm{Kerr}}|_g = O(r^{-3}), \quad |B^{\mathrm{Kerr}}|_g = O(r^{-4}).
\]
The actual Weyl tensors satisfy the same decay for asymptotically flat data:
\[
|E|_g = O(r^{-\tau-2}), \quad |B|_g = O(r^{-\tau-2}).
\]
The Kerr deviation $\mathcal{S}_{(g,K)} = (E - E^{\mathrm{Kerr}}) + i(B - B^{\mathrm{Kerr}})$ satisfies:
\[
|\mathcal{S}_{(g,K)}|_g = O(r^{-\tau-2}) - O(r^{-3}) = O(r^{-2-\tau})
\]
(the slower decay dominates). Therefore:
\[
\Lambda_J = \frac{1}{8}|\mathcal{S}_{(g,K)}|^2_g = O(r^{-4-2\tau}).
\]

\textbf{Step 3: Cylindrical end decay.}
On the cylindrical end $\mathcal{C} \cong [0,\infty)_t \times \Sigma$, the Jang metric satisfies $\bg = dt^2 + g_\Sigma + O(e^{-\beta_0 t})$ by Theorem~\ref{thm:jang-exist}(iii). The key observation is that $\Lambda_J$ is computed from the \textbf{physical} initial data $(g, K)$, not the Jang solution $f$ or the conformal factor $\phi$.
\end{proof}

%=============================================================================
% AM-Lichnerowicz Existence Theorem
%=============================================================================

\begin{theorem}[AM-Lichnerowicz Existence]\label{thm:lich-exist}
Let $(\bM, \bg)$ be the Jang manifold from Theorem~\ref{thm:jang-exist}, with cylindrical end $\mathcal{C} \cong [0,\infty)_t \times \Sigma$. Let $\Lambda_J = \frac{1}{8}|\mathcal{S}_{(g,K)}|^2_{\bg} \geq 0$ be the angular momentum source term (Definition~\ref{def:Lambda-J}).

There exists a unique positive solution $\phi > 0$ to the AM-Lichnerowicz equation:
\begin{equation}\label{eq:am-lich-exist}
-8\Delta_{\bg}\phi + R_{\bg}\phi = \Lambda_J \phi^{-7}
\end{equation}
with boundary conditions:
\begin{enumerate}[label=\textup{(\roman*)}]
    \item $\phi|_\Sigma = 1$ (Dirichlet condition at the MOTS);
    \item $\phi \to 1$ as $r \to \infty$ in the asymptotically flat end.
\end{enumerate}

The solution satisfies:
\begin{enumerate}[label=\textup{(\alph*)}]
    \item \textbf{Positivity:} $\phi > 0$ throughout $\bM$ (unconditional);
    \item \textbf{Upper bound:} $\phi \leq 1$ throughout $\bM$ \textbf{conditional} on Lemma~\ref{lem:refined-bk} (not required for main theorem);
    \item \textbf{Asymptotic decay:} $\phi = 1 + O(r^{-\tau})$ at spatial infinity;
    \item \textbf{Cylindrical decay:} $|\phi - 1| = O(e^{-\kappa t})$ along the cylindrical end for some $\kappa > 0$;
    \item \textbf{Regularity:} $\phi \in C^{2,\Hoelder}(\bM)$;
    \item \textbf{Conformal scalar curvature:} The conformal metric $\tg = \phi^4 \bg$ satisfies $R_{\tg} = \Lambda_J \phi^{-12} \geq 0$.
\end{enumerate}

\textbf{Important:} The existence proof via the variational method (Proof A below) requires only $R_{\bg} \geq 0$ (standard Bray--Khuri). The upper bound $\phi \leq 1$ is established via the super-solution method (Proof B) which requires the refined estimate $R_{\bg} \geq 2\Lambda_J$. The main theorem (Theorem~\ref{thm:main}) uses only the unconditional results (a), (c)--(f).
\end{theorem}

\begin{proof}
We provide two independent existence proofs: (A) a variational approach that requires only $R_{\bg} \geq 0$, and (B) a sub/super-solution method that additionally uses the refined bound. The variational approach (A) is the \textbf{primary proof}, ensuring the theorem holds unconditionally.

\begin{mdframed}[linewidth=1pt, linecolor=blue!60!black, backgroundcolor=blue!3]
\textbf{Proof A: Variational Existence (Unconditional---Primary Proof).}

This proof requires only $R_{\bg} \geq 0$ (the classical Bray--Khuri bound under DEC) and does \textbf{not} rely on the refined estimate $R_{\bg} \geq 2\Lambda_J$.
\end{mdframed}

\textbf{Step A1: Functional framework.} Define the energy functional on $W^{1,2}_\beta(\bM)$:
\[
\mathcal{E}[\phi] := \int_{\bM} \left(4|\nabla\phi|^2_{\bg} + \frac{1}{8}R_{\bg}\phi^2 + \frac{\Lambda_J}{6}\phi^{-6}\right) dV_{\bg},
\]
where $\beta < 0$ is chosen so that functions in $W^{1,2}_\beta$ decay exponentially on the cylindrical end. The Euler--Lagrange equation is precisely the AM-Lichnerowicz equation \eqref{eq:am-lich-exist}.

\textbf{Step A2: Coercivity.} Since $R_{\bg} \geq 0$ (Bray--Khuri under DEC), the quadratic terms are non-negative:
\[
\int_{\bM} \left(4|\nabla\phi|^2 + \frac{1}{8}R_{\bg}\phi^2\right) dV_{\bg} \geq 4\|\nabla\phi\|_{L^2}^2.
\]
The $\phi^{-6}$ term provides a barrier preventing $\phi \to 0$: for any sequence $\phi_n \to 0$ in $L^6$, the term $\int \Lambda_J \phi_n^{-6}$ diverges where $\Lambda_J > 0$.

\textbf{Step A3: Lower bound and minimizer.} On the constraint set $\mathcal{C} := \{\phi \in W^{1,2}_\beta : \phi > 0, \, \phi|_\Sigma = 1, \, \phi \to 1 \text{ at } \infty\}$:
\begin{itemize}
    \item $\mathcal{E}[\phi] > -\infty$ since all terms are bounded below (the $\phi^{-6}$ term is positive).
    \item $\mathcal{E}[\phi] < +\infty$ for the test function $\phi \equiv 1$, giving $\mathcal{E}[1] = \frac{1}{8}\int R_{\bg} + \frac{1}{6}\int \Lambda_J < \infty$.
\end{itemize}
By the direct method of calculus of variations, there exists a minimizer $\phi_* \in \mathcal{C}$ with $\mathcal{E}[\phi_*] = \inf_{\mathcal{C}}\mathcal{E}$.

\textbf{Step A4: Positivity of minimizer.} The minimizer satisfies $\phi_* > 0$ everywhere. If $\phi_*(x_0) = 0$ for some $x_0$, then $\int_{\bM} \Lambda_J \phi_*^{-6} = +\infty$ where $\Lambda_J(x_0) > 0$, contradicting $\mathcal{E}[\phi_*] < \infty$. On regions where $\Lambda_J = 0$ (Kerr-like regions), the strong maximum principle for the linearized equation $-8\Delta_{\bg}\phi + R_{\bg}\phi = 0$ with $R_{\bg} \geq 0$ ensures $\phi > 0$.

\textbf{Step A5: Regularity.} The minimizer satisfies the weak form of \eqref{eq:am-lich-exist}. Since $\phi_* > 0$ is bounded away from zero (by Step A4), the nonlinearity $\Lambda_J\phi^{-7}$ is Lipschitz in $\phi$. Standard elliptic regularity (bootstrapping from $W^{1,2}$ to $C^{2,\beta}$) gives $\phi_* \in C^{2,\Hoelder}(\bM)$.

\textbf{Step A6: Exponential decay on cylindrical end.} The boundary condition $\phi|_\Sigma = 1$ is interpreted as $\phi \to 1$ along the cylindrical end. Setting $\psi = \phi - 1$, the linearization on the cylinder is:
\[
-8\partial_t^2\psi - 8\Delta_\Sigma\psi + R_{\bg}\psi = O(e^{-\beta_0 t}),
\]
where $\beta_0 > 0$ is the exponential decay rate of the metric perturbation. By spectral theory on the cylinder (Lockhart--McOwen \cite{lockhartmccowen1985}), solutions with $\psi \to 0$ as $t \to \infty$ satisfy $|\psi| = O(e^{-\kappa t})$ for $\kappa > 0$ depending on the spectral gap of $L_\Sigma$.

\begin{mdframed}[linewidth=1pt, linecolor=gray!60!black, backgroundcolor=gray!5]
\textbf{Proof B: Sub/Super-Solution Method (Conditional on Lemma~\ref{lem:refined-bk}).}

This alternative proof uses the refined bound $R_{\bg} \geq 2\Lambda_J$ to establish $\phi \leq 1$ directly.
\end{mdframed}

\textbf{Step B1: Super-solution.} \textit{Assuming} $R_{\bg} \geq 2\Lambda_J$ (Lemma~\ref{lem:refined-bk}), the constant $\bar{\phi} = 1$ satisfies:
\[
L_{AM}[1] = -8\Delta_{\bg}(1) + R_{\bg}(1) - \Lambda_J (1)^{-7} = R_{\bg} - \Lambda_J \geq 2\Lambda_J - \Lambda_J = \Lambda_J \geq 0.
\]
Thus $\bar{\phi} = 1$ is a super-solution.

\textbf{Step B2: Sub-solution.} For small $\epsilon > 0$, the function $\underline{\phi} = \epsilon$ satisfies:
\[
L_{AM}[\epsilon] = R_{\bg}\epsilon - \Lambda_J \epsilon^{-7} < 0
\]
for sufficiently small $\epsilon$ (since the $\epsilon^{-7}$ term dominates). Thus $\underline{\phi} = \epsilon$ is a sub-solution.

\textbf{Step B3: Existence via monotone iteration.} By the sub/super-solution theorem \cite[Chapter 4]{gilbargtrudinger2001}, there exists a solution $\phi$ with $\epsilon \leq \phi \leq 1$. This directly gives the bound $\phi \leq 1$.

\bigskip

\textbf{Uniqueness (common to both proofs).} Suppose $\phi_1, \phi_2$ are two positive solutions. Setting $w = \phi_1 - \phi_2$ and linearizing:
\[
-8\Delta_{\bg}w + R_{\bg}w + 7\Lambda_J \xi^{-8}w = 0
\]
for some $\xi$ between $\phi_1$ and $\phi_2$. Since $R_{\bg} \geq 0$ and $7\Lambda_J \xi^{-8} \geq 0$, the operator has non-negative zero-th order term, and by the maximum principle with Dirichlet conditions ($w = 0$ on boundaries), we have $w = 0$.

\textbf{Conformal scalar curvature.} By direct calculation using the conformal transformation formula:
\[
R_{\tg} = \phi^{-5}(R_{\bg}\phi - 8\Delta_{\bg}\phi) = \phi^{-5} \cdot \Lambda_J \phi^{-7} = \Lambda_J \phi^{-12} \geq 0.
\]
This holds for any positive solution $\phi$, independent of whether $\phi \leq 1$.
\end{proof}

\begin{lemma}[Conformal Factor Bounds]\label{lem:phi-bound}
The solution $\phi$ from Theorem~\ref{thm:lich-exist} satisfies:
\begin{enumerate}[label=\textup{(\roman*)}]
    \item $\phi \leq 1$ throughout $\bM$ (super-solution bound) \textbf{assuming the refined Bray--Khuri bound $R_{\bg} \geq 2\Lambda_J$ from Lemma~\ref{lem:refined-bk}};
    \item $|\phi - 1| = O(e^{-\kappa t})$ along the cylindrical end $\mathcal{C}$, where $\kappa = 2\sqrt{\lambda_1(L_\Sigma)}$ and $\lambda_1(L_\Sigma)$ is the first eigenvalue of the stability operator;
    \item The mass bound $M_{\ADM}(\tg) \leq M_{\ADM}(\bg) \leq M_{\ADM}(g)$ holds \textbf{unconditionally}, either via part (i) or via the alternative proof in Proposition~\ref{prop:alternative-mass}.
\end{enumerate}
\end{lemma}

\begin{proof}
Part (i) follows from Step 1 of Theorem~\ref{thm:lich-exist} by the maximum principle, \textbf{conditional on the validity of Lemma~\ref{lem:refined-bk}}. See Remark~\ref{rem:conditional-phi-bound} below for discussion of this conditionality.

Part (ii): On the cylindrical end, $\Lambda_J = O(e^{-\beta t})$ for some $\beta > 0$ (since $\Lambda_J$ depends on the physical data $(g,K)$, which is bounded, and the cylindrical coordinate $t \to \infty$ corresponds to the MOTS neighborhood). The linearized equation around $\phi = 1$ becomes:
\[
-8\Delta_{\bg}\psi + R_{\bg}\psi \approx 0 \quad \text{on } \mathcal{C},
\]
where $\psi = \phi - 1$. On the product $[0,\infty) \times \Sigma$, separation of variables shows that solutions decay as $e^{-\kappa t}$ where $\kappa$ is related to the spectrum of the stability operator on $\Sigma$.

Part (iii) follows from the conformal mass formula and the fact that $\phi = 1 + O(r^{-\tau})$ at infinity. \textbf{Two independent proofs are available:}
\begin{itemize}
\item \textbf{Path A (conditional):} If part (i) holds ($\phi \leq 1$), then the conformal mass formula directly gives $M_{\ADM}(\tg) \leq M_{\ADM}(\bg)$; see standard references \cite{bartnik1986}.
\item \textbf{Path B (unconditional):} Proposition~\ref{prop:alternative-mass} (Appendix~\ref{app:supersolution}) provides an integral energy-based proof that requires only $R_{\bg} \geq 0$ (the classical Bray--Khuri bound) and does not use the assumption $\phi \leq 1$.
\end{itemize}
The inequality $M_{\ADM}(\bg) \leq M_{\ADM}(g)$ is the Han--Khuri mass bound \cite{hankhuri2013}, independent of the conformal factor.
\end{proof}

\begin{remark}[Conditional vs.\ Unconditional Statements]\label{rem:conditional-phi-bound}
It is essential to distinguish which parts of Lemma~\ref{lem:phi-bound} are conditional on the refined bound $R_{\bg} \geq 2\Lambda_J$ (Lemma~\ref{lem:refined-bk}):

\textbf{Conditional (requires Lemma~\ref{lem:refined-bk}):}
\begin{itemize}
\item Part (i): The pointwise bound $\phi \leq 1$ throughout $\bM$.
\end{itemize}

\textbf{Unconditional (does NOT require Lemma~\ref{lem:refined-bk}):}
\begin{itemize}
\item Part (ii): The exponential decay $|\phi - 1| = O(e^{-\kappa t})$ along the cylindrical end.
\item Part (iii): The mass chain inequality $M_{\ADM}(\tg) \leq M_{\ADM}(\bg) \leq M_{\ADM}(g)$.
\end{itemize}

\textbf{Why this matters:} The main theorem (Theorem~\ref{thm:main}) depends \emph{only} on part (iii)---the mass chain inequality. Since part (iii) has an unconditional proof (Path B via Proposition~\ref{prop:alternative-mass}), the validity of Theorem~\ref{thm:main} does \textbf{not} depend on whether $\phi \leq 1$ holds or on the refined bound $R_{\bg} \geq 2\Lambda_J$.

The pointwise bound $\phi \leq 1$ (part (i)) is a \emph{stronger} result that, if true, provides additional geometric information about the conformal factor. It is presented here for completeness and because it may be useful for future work.
\end{remark}

\begin{remark}[Key Estimates]\label{rem:verification-lich}
The critical estimates in this section are:
\begin{itemize}
    \item The \textbf{conformal scalar curvature identity} $R_{\tg} = \Lambda_J \phi^{-12} \geq 0$ (Remark~\ref{rem:sign-verification}), ensuring the conformal metric has nonnegative scalar curvature. This is a \textbf{direct calculation} from the conformal transformation formula and the AM-Lichnerowicz equation.
    \item The \textbf{mass chain inequality} $M_{\ADM}(\tg) \leq M_{\ADM}(\bg) \leq M_{\ADM}(g)$ (Lemma~\ref{lem:phi-bound}(iii)). \textbf{Two independent proofs are available:}
    \begin{enumerate}[label=(\alph*)]
    \item \textbf{Conditional proof:} If the refined bound $R_{\bg} \geq 2\Lambda_J$ (Lemma~\ref{lem:refined-bk}) holds, then the maximum principle gives $\phi \leq 1$, which via the conformal mass formula yields $M_{\ADM}(\tg) \leq M_{\ADM}(\bg)$.
    \item \textbf{Unconditional proof:} Proposition~\ref{prop:alternative-mass} (Appendix~\ref{app:supersolution}) establishes the mass inequality using \emph{only} the classical bound $R_{\bg} \geq 0$ (known from Bray--Khuri under DEC) combined with an integral energy identity. This proof does not require $\phi \leq 1$ or the refined bound.
    \end{enumerate}
\end{itemize}
The unconditional proof (Proposition~\ref{prop:alternative-mass}) is the primary approach. The refined bound $R_{\bg} \geq 2\Lambda_J$ is \textbf{not required} for Theorem~\ref{thm:main}, though it provides additional geometric insight if valid. See Section~\ref{subsec:critical-estimates} for further discussion.
\end{remark}
