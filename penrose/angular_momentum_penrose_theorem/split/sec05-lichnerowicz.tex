\section{Stage 2: AM-Lichnerowicz Equation}\label{sec:lichnerowicz}
%=============================================================================

\subsection{The Conformal Equation}

On the Jang manifold $(\bM, \bg)$, we solve a modified Lichnerowicz equation that accounts for angular momentum. The cylindrical end structure from Theorem~\ref{thm:jang-exist} requires Lockhart--McOwen weighted Sobolev spaces for Fredholm theory.

\begin{definition}[Weighted Sobolev Spaces on Cylindrical Ends]\label{def:weighted-sobolev-cyl}
Let $(\bM, \bg)$ have cylindrical ends $\mathcal{C} \cong [0,\infty) \times \Sigma$ with coordinate $t$ and cross-section $(\Sigma, g_\Sigma)$. For $k \in \mathbb{N}_0$, $p \in [1,\infty)$, and weight $\beta \in \mathbb{R}$, define the weighted Sobolev space:
\[
W^{k,p}_\beta(\bM) := \{u \in W^{k,p}_{\mathrm{loc}}(\bM) : \|u\|_{W^{k,p}_\beta} < \infty\},
\]
where the norm on the cylindrical end is:
\[
\|u\|_{W^{k,p}_\beta(\mathcal{C})}^p := \sum_{j=0}^{k} \int_0^\infty \int_\Sigma e^{-\beta p t} |\nabla^j u|^p \, dA_{g_\Sigma} \, dt,
\]
with $|\nabla^j u|$ denoting the norm of the $j$-th covariant derivative. In the asymptotically flat end, the standard weighted norm from Definition~\ref{def:weighted-holder} applies.

A function $u \in W^{k,p}_\beta$ with $\beta < 0$ decays as $t \to \infty$ on the cylindrical end: $|u(t, \cdot)| = O(e^{\beta t}) \to 0$. For $\beta > 0$, such functions may grow. The Lockhart--McOwen theory \cite{lockhartmccowen1985} shows that the Laplacian $\Delta_{\bg}: W^{k+2,p}_\beta \to W^{k,p}_\beta$ is Fredholm when $\beta$ avoids the \textbf{indicial roots}---values determined by the spectrum of the cross-sectional Laplacian $\Delta_\Sigma$.
\end{definition}

\begin{remark}[Compatibility of Function Spaces]\label{rem:space-compatibility}
The Jang manifold $(\bM, \bg)$ has two distinct asymptotic regions requiring different function space frameworks:
\begin{enumerate}[label=(\roman*)]
    \item \textbf{Asymptotically flat end:} Weighted H\"older spaces $C^{k,\alpha}_{-\tau}$ with polynomial weight $r^{-\tau}$ (Definition~\ref{def:weighted-holder});
    \item \textbf{Cylindrical end:} Weighted Sobolev spaces $W^{k,p}_\beta$ with exponential weight $e^{\beta t}$ (Definition~\ref{def:weighted-sobolev-cyl}).
\end{enumerate}
These frameworks are compatible on the transition region $\{R_0 \leq r \leq 2R_0\}$ (equivalently $\{0 \leq t \leq T_0\}$) in the following sense: by Sobolev embedding, $W^{k+1,2}_\beta \hookrightarrow C^{k,\alpha}$ locally, and both norms are equivalent (up to constants depending on $R_0$) on the compact overlap region. This allows elliptic estimates to be ``glued'' across the transition using standard partition-of-unity arguments. The key point is that the Fredholm index is determined by the asymptotic behavior at both ends, not the transition region.
\end{remark}

\begin{definition}[AM-Lichnerowicz Operator]\label{def:am-lich}
The angular-momentum-modified Lichnerowicz equation is:
\begin{equation}\label{eq:am-lich}
L_{AM}[\phi] := -8\Delta_{\bg}\phi + R_{\bg}\phi - \Lambda_J \phi^{-7} = 0,
\end{equation}
where $\Lambda_J = \frac{1}{8}|\mathcal{S}_{(g,K)}|^2_{\bg} \geq 0$ is the Kerr deviation contribution (Definition~\ref{def:Lambda-J}). The \textbf{negative} sign in front of $\Lambda_J$ ensures that the conformal scalar curvature $R_{\tg} = \Lambda_J \phi^{-12} \geq 0$.

\textbf{Key property:} For Kerr initial data, $\Lambda_J = 0$ (since $\mathcal{S}_{(g,K)} = 0$), and the equation reduces to the standard Lichnerowicz equation $-8\Delta_{\bg}\phi + R_{\bg}\phi = 0$.
\end{definition}

\begin{remark}[Sign Convention Verification]\label{rem:sign-verification}
We verify the sign conventions in the AM-Lichnerowicz equation:
\begin{enumerate}[label=\textup{(\roman*)}]
    \item \textbf{Conformal transformation formula:} Under $\tg = \phi^4 \bg$, the scalar curvatures are related by:
    \[
    R_{\tg} = \phi^{-4} R_{\bg} - 8\phi^{-5}\Delta_{\bg}\phi = \phi^{-5}(R_{\bg}\phi - 8\Delta_{\bg}\phi).
    \]
    \item \textbf{AM-Lichnerowicz rearrangement:} From \eqref{eq:am-lich}:
    \[
    -8\Delta_{\bg}\phi + R_{\bg}\phi = \Lambda_J \phi^{-7} \quad \Rightarrow \quad R_{\tg} = \phi^{-5} \cdot \Lambda_J \phi^{-7} = \Lambda_J \phi^{-12}.
    \]
    \item \textbf{Positivity:} Since $\Lambda_J = \frac{1}{8}|\mathcal{S}_{(g,K)}|^2 \geq 0$ and $\phi > 0$, we have $R_{\tg} \geq 0$ automatically.
    \item \textbf{Strict positivity:} $R_{\tg} > 0$ where $\mathcal{S}_{(g,K)} \neq 0$, i.e., where the data deviates from Kerr geometry.
    \item \textbf{Equality case:} For Kerr data, $\Lambda_J = 0$, so $R_{\tg} = 0$ and the monotonicity integrand vanishes.
\end{enumerate}
The convention matches the standard Lichnerowicz equation $-8\Delta\phi + R\phi = 0$ (for $R_{\tg} = 0$), with the $\Lambda_J \phi^{-7}$ term producing positive conformal scalar curvature for non-Kerr data.
\end{remark}

\begin{mdframed}[linewidth=1pt, linecolor=black!50, backgroundcolor=yellow!5]
\begin{lemma}[Well-Definedness of $\Lambda_J$]\label{lem:Lambda-J-welldef}
The angular momentum source term $\Lambda_J = \frac{1}{8}|\mathcal{S}_{(g,K)}|^2_{\bg}$ is a \textbf{well-defined, coordinate-independent, non-negative scalar function} on any asymptotically flat, axisymmetric vacuum initial data $(M^3, g, K)$. Specifically:
\begin{enumerate}[label=\textup{(\roman*)}]
    \item \textbf{Function spaces:} The electric and magnetic Weyl tensors $(E_{ij}, B_{ij})$ from Definition~\ref{def:Lambda-J} lie in $C^{k-2,\alpha}_{-\tau-2}(M)$ when $(g, K) \in C^{k,\alpha}_{-\tau} \times C^{k-1,\alpha}_{-\tau-1}$ for $k \geq 3$, $\tau > 1/2$. The reference Kerr Weyl tensor $\mathcal{W}^{\mathrm{Kerr}}(M,J)$ is constructed via the following well-posed procedure.
    
    \item \textbf{Asymptotic data:} The parameters $(M, J)$ are determined coordinate-independently by the ADM mass and Komar angular momentum integrals, which depend only on the asymptotic behavior $g_{ij} - \delta_{ij} = O(r^{-\tau})$, $K_{ij} = O(r^{-\tau-1})$.
    
    \item \textbf{Unique extension via Bianchi constraints:} The reference Kerr Weyl tensor is extended from infinity to all of $M$ as the unique solution to the Bianchi constraint system
    \[
    \nabla^j E^{\mathrm{Kerr}}_{ij} = \epsilon_{ijk}K^{jl}B^{\mathrm{Kerr},k}{}_l, \qquad \nabla^j B^{\mathrm{Kerr}}_{ij} = -\epsilon_{ijk}K^{jl}E^{\mathrm{Kerr},k}{}_l
    \]
    with boundary data matching the Boyer--Lindquist Kerr slice at infinity. This is well-posed because:
    \begin{itemize}
        \item The Bianchi system is elliptic in harmonic gauge \cite{christodoulou2000};
        \item The asymptotic boundary condition is coordinate-independent (determined by $(M,J)$ in the ADM frame);
        \item The solution space is finite-dimensional, and uniqueness follows from the decay conditions.
    \end{itemize}
    
    \item \textbf{Gauge independence:} The Kerr deviation $\mathcal{S}_{(g,K)} = \mathcal{W} - \mathcal{W}^{\mathrm{Kerr}}$ transforms tensorially under diffeomorphisms that preserve the asymptotic structure. The remaining gauge freedom (asymptotic rotations/boosts) is fixed by the ADM frame convention.
    
    \item \textbf{Characterization:} $\Lambda_J = 0$ everywhere if and only if $(M, g, K)$ is isometric to a slice of Kerr spacetime (Theorem~\ref{thm:kerr-characterization} in Appendix~\ref{app:mars-simon}).
\end{enumerate}
\end{lemma}

\begin{proof}[Proof (Complete Details)]
We provide a rigorous construction addressing all well-definedness concerns. See Appendix~\ref{app:mars-simon} for additional background on the Mars--Simon characterization.

\textbf{Step 1: Intrinsic definition of $(E, B)$.}
The electric and magnetic Weyl tensors are defined \textbf{algebraically} from $(g, K)$:
\begin{align*}
E_{ij} &:= R_{ij} - \tfrac{1}{3}Rg_{ij} + (\tr K)K_{ij} - K_{ik}K^k{}_j, \\
B_{ij} &:= \epsilon_i{}^{kl}\nabla_k K_{lj}.
\end{align*}
These formulas involve only the metric $g$, its Levi-Civita connection $\nabla$, and the extrinsic curvature $K$. No coordinates or embedding is required. For $(g, K) \in C^{k,\alpha}_{-\tau} \times C^{k-1,\alpha}_{-\tau-1}$ with $k \geq 3$, we have $(E, B) \in C^{k-2,\alpha}_{-\tau-2}(M; S^2 T^*M)$ (symmetric trace-free 2-tensors with decay $O(r^{-\tau-2})$).

\textbf{Step 2: Coordinate-independent asymptotic matching.}
The parameters $(M, J)$ are defined by \textbf{coordinate-independent} integrals:
\begin{itemize}
    \item $M_{\ADM} := \lim_{r \to \infty} \frac{1}{16\pi}\int_{S_r} (g_{ij,j} - g_{jj,i})\nu^i \, d\sigma$ (ADM mass);
    \item $J := \frac{1}{8\pi}\int_{\Sigma} K(\eta, \nu) \, d\sigma$ (Komar angular momentum).
\end{itemize}
These depend only on the asymptotic behavior of $(g, K)$ and are invariant under diffeomorphisms that preserve asymptotic flatness.

Given $(M, J)$, the reference Kerr Weyl tensor $\mathcal{W}^{\mathrm{Kerr}}(M, J)$ at infinity is uniquely determined: it equals the Weyl tensor of the Boyer--Lindquist slice with the same $(M, J)$, expressed in the \textbf{ADM frame} (the unique asymptotically Cartesian coordinate system where $g_{ij} = \delta_{ij} + 2M r^{-1}\delta_{ij} + O(r^{-2})$).

\textbf{Step 3: Bianchi constraint propagation (rigorous PDE analysis).}
The reference Kerr Weyl tensor is extended from infinity to all of $M$ as follows. The Bianchi constraints for vacuum data form the system:
\begin{equation}\label{eq:bianchi-system}
\nabla^j E_{ij} = \epsilon_{ijk}K^{jl}B^k{}_l, \qquad \nabla^j B_{ij} = -\epsilon_{ijk}K^{jl}E^k{}_l.
\end{equation}
This is a \textbf{first-order elliptic system} for $(E, B)$ when $(g, K)$ is prescribed.

\textbf{Function space setup:} Work on weighted Sobolev spaces $H^s_\delta(M; S^2_0 T^*M)$ of symmetric trace-free 2-tensors with weight $r^\delta$ at infinity. For $s \geq 2$ and $\delta \in (-2, -1/2)$:
\begin{itemize}
    \item The homogeneous system \eqref{eq:bianchi-system} with $(E, B) \to 0$ at infinity has only the trivial solution (by unique continuation for elliptic systems \cite{aronszajn1957}).
    \item The inhomogeneous system with prescribed asymptotic data $\mathcal{W}^{\mathrm{Kerr}}|_{\infty}$ admits a unique solution in $H^s_\delta$ by standard elliptic theory \cite{lockhartmccowen1985}.
\end{itemize}
The solution $\mathcal{W}^{\mathrm{Kerr}}(M, J)$ on all of $M$ is the unique extension satisfying \eqref{eq:bianchi-system} with asymptotic behavior matching Kerr.

\textbf{Step 4: Gauge independence.}
The Kerr deviation $\mathcal{S}_{(g,K)} = \mathcal{W} - \mathcal{W}^{\mathrm{Kerr}}$ is a tensor, hence transforms covariantly under diffeomorphisms. The remaining gauge freedom consists of:
\begin{enumerate}[label=(\roman*)]
    \item \textit{Asymptotic translations:} Fixed by the ADM center of mass convention.
    \item \textit{Asymptotic rotations:} Fixed by aligning the rotation axis with the Killing field $\eta$.
    \item \textit{Asymptotic boosts:} Fixed by requiring $K_{ij} = O(r^{-2})$ (no linear momentum).
\end{enumerate}
With these conventions, $\mathcal{S}_{(g,K)}$ is uniquely determined by $(g, K)$.

\textbf{Step 5: Characterization of Kerr.}
The condition $\mathcal{S}_{(g,K)} = 0$ means $\mathcal{W} = \mathcal{W}^{\mathrm{Kerr}}$. By the Ionescu--Klainerman uniqueness theorem \cite{ionescu2009}, vacuum axisymmetric data with Kerr Weyl curvature must be a Kerr slice. Conversely, any Kerr slice has $\mathcal{S}_{(g,K)} = 0$ by construction.
\end{proof}
\end{mdframed}

\begin{remark}[Why $\Lambda_J$ Drives the Proof]\label{rem:Lambda-J-role}
The well-definedness of $\Lambda_J$ is central to the proof because:
\begin{enumerate}
    \item \textbf{Scalar curvature positivity:} $R_{\tilde{g}} = \Lambda_J \phi^{-12} \geq 0$ is automatic, enabling AMO monotonicity.
    \item \textbf{Rigidity:} $\Lambda_J = 0 \Leftrightarrow$ Kerr provides the equality case characterization.
    \item \textbf{No ambiguity propagates:} Since $\Lambda_J$ is coordinate-independent and $\phi > 0$, the conformal scalar curvature $R_{\tilde{g}}$ is unambiguously non-negative throughout $\tilde{M}$.
\end{enumerate}
Any ambiguity in the definition of $\Lambda_J$ would undermine the entire monotonicity argument. Lemma~\ref{lem:Lambda-J-welldef} ensures no such ambiguity exists.
\end{remark}

\begin{lemma}[Fredholm Property]\label{lem:fredholm}
The linearization $L := -8\Delta_{\bg} + R_{\bg}$ of the operator in \eqref{eq:am-lich} at $\phi = 1$ is Fredholm
\[
L: W^{2,2}_\beta(\bM) \to L^2_\beta(\bM)
\]
of index zero for $\beta \in (-1, 0)$ not equal to any indicial root.
\end{lemma}

\begin{proof}
We give a detailed proof using Lockhart--McOwen theory, with careful treatment of the marginally stable case.

\textbf{Step 1: Asymptotic structure of the operator.}
By Theorem~\ref{thm:jang-exist}(iii), the Jang metric $\bg$ converges exponentially to the cylindrical metric $dt^2 + g_\Sigma$ on the ends:
\[
\bg = dt^2 + g_\Sigma + O(e^{-\beta_0 t}) \quad \text{as } t \to \infty,
\]
where the exponential decay rate $\beta_0 > 0$ is determined by the \textbf{spectral gap of the MOTS stability operator}. Specifically:
\begin{itemize}
    \item For \textbf{strictly stable} MOTS ($\lambda_1(L_\Sigma) > 0$): $\beta_0 = 2\sqrt{\lambda_1(L_\Sigma)}$, where $\lambda_1$ is the principal eigenvalue of the stability operator \eqref{eq:stability-operator}.
    \item For \textbf{marginally stable} MOTS ($\lambda_1(L_\Sigma) = 0$): $\beta_0 = 2$, arising from the subleading spectral term. This is the borderline case discussed in Step 4 below.
\end{itemize}
This relationship between $\beta_0$ and stability follows from the eigenvalue problem for the linearized Jang operator at the MOTS; see \cite[Proposition 3.4]{anderssonmetzger2009} and \cite[Section 4]{hankhuri2013}.

The Lockhart--McOwen theory \cite{lockhartmccowen1985} applies to operators of the form $L = L_\infty + Q$ where:
\begin{itemize}
    \item $L_\infty = -8(\partial_t^2 + \Delta_\Sigma) + R_\Sigma$ is the translation-invariant limit operator on the exact cylinder $\mathbb{R} \times \Sigma$.
    \item $Q$ is a perturbation satisfying $|Q| = O(e^{-\beta_0 t})$ as $t \to \infty$, arising from the deviation $\bg - (dt^2 + g_\Sigma)$.
\end{itemize}

\textbf{Step 2: Indicial roots computation.}
The indicial roots are determined by seeking solutions of $L_\infty \psi = 0$ of the form $\psi(t, y) = e^{\gamma t} \varphi(y)$ where $\varphi$ is an eigenfunction of the cross-sectional operator. Substituting:
\[
L_\infty(e^{\gamma t}\varphi) = e^{\gamma t}\bigl(-8\gamma^2 \varphi - 8\Delta_\Sigma \varphi + R_\Sigma \varphi\bigr) = 0.
\]
Thus $\varphi$ must be an eigenfunction of $-8\Delta_\Sigma + R_\Sigma$ on $(\Sigma, g_\Sigma)$:
\[
(-8\Delta_\Sigma + R_\Sigma)\varphi = \lambda \varphi,
\]
and the indicial root satisfies $-8\gamma^2 + \lambda = 0$, giving:
\[
\gamma = \pm\sqrt{\lambda/8}.
\]

\textbf{Step 3: Eigenvalue lower bound for the cross-sectional operator.}
We need to show that the operator $-8\Delta_\Sigma + R_\Sigma$ on $(\Sigma, g_\Sigma) \cong S^2$ has strictly positive principal eigenvalue $\lambda_0 > 0$.

\textbf{Claim:} For any Riemannian metric on $S^2$ with scalar curvature $R_\Sigma$ satisfying $\int_\Sigma R_\Sigma \, d\sigma = 8\pi$ (by Gauss--Bonnet), the operator $-8\Delta_\Sigma + R_\Sigma$ has $\lambda_0 > 0$.

\begin{proof}[Proof of Claim]
The principal eigenvalue is given by the variational formula:
\[
\lambda_0 = \inf_{\|\varphi\|_{L^2}=1} \int_\Sigma \bigl(8|\nabla\varphi|^2 + R_\Sigma \varphi^2\bigr) d\sigma.
\]
We show $\lambda_0 > 0$ by contradiction. Suppose $\lambda_0 \leq 0$. Then there exists $\varphi \in W^{1,2}(\Sigma)$ with $\|\varphi\|_{L^2} = 1$ and
\[
\int_\Sigma \bigl(8|\nabla\varphi|^2 + R_\Sigma \varphi^2\bigr) d\sigma \leq 0.
\]

\textit{Case 1: $\lambda_0 = 0$ with eigenfunction $\varphi_0$.}
If $\lambda_0 = 0$ is achieved by an eigenfunction $\varphi_0$, then:
\[
-8\Delta_\Sigma \varphi_0 + R_\Sigma \varphi_0 = 0.
\]
Integrating over $\Sigma$:
\[
\int_\Sigma R_\Sigma \varphi_0 \, d\sigma = 8\int_\Sigma \Delta_\Sigma \varphi_0 \, d\sigma = 0
\]
by the divergence theorem on a closed surface. Since $\varphi_0$ is an eigenfunction with $\lambda_0 = 0$, it cannot change sign (principal eigenfunctions are either strictly positive or strictly negative). WLOG, assume $\varphi_0 > 0$. Then:
\[
\int_\Sigma R_\Sigma \varphi_0 \, d\sigma = 0 \quad \text{with } \varphi_0 > 0.
\]
This implies $R_\Sigma$ must change sign on $\Sigma$ (otherwise the integral would be strictly positive or negative). 

Now, use the constraint from stability. For a \textbf{stable} MOTS $\Sigma$, the Galloway--Schoen theorem \cite{gallowayschoen2006} implies that $\Sigma$ has non-negative Gaussian curvature: $K_\Sigma \geq 0$ everywhere. Since $R_\Sigma = 2K_\Sigma$ for surfaces, this means $R_\Sigma \geq 0$ on $\Sigma$. Combined with $\int_\Sigma R_\Sigma = 8\pi > 0$, we have $R_\Sigma \geq 0$ (not identically zero). Therefore:
\[
\int_\Sigma R_\Sigma \varphi_0 \, d\sigma > 0 \quad \text{(since } R_\Sigma \geq 0, \varphi_0 > 0, \text{ and } R_\Sigma \not\equiv 0),
\]
contradicting $\int_\Sigma R_\Sigma \varphi_0 = 0$.

\textit{Case 2: $\lambda_0 < 0$.}
This would require $\int_\Sigma R_\Sigma \varphi^2 < 0$ for some $\varphi$ (since $\int 8|\nabla\varphi|^2 \geq 0$). But $R_\Sigma \geq 0$ on a stable MOTS, so $\int R_\Sigma \varphi^2 \geq 0$ for all $\varphi$, contradiction.

\textit{Conclusion:} $\lambda_0 > 0$ for the operator $-8\Delta_\Sigma + R_\Sigma$ on a stable MOTS.
\end{proof}

\textbf{Remark:} The key input is that \textbf{stable} MOTS in data satisfying DEC have $R_\Sigma = 2K_\Sigma \geq 0$ (non-negative Gaussian curvature). This follows from the stability inequality and the Gauss--Bonnet--based argument in \cite{gallowayschoen2006}. For \textbf{unstable} MOTS, $R_\Sigma$ can be negative somewhere, and the argument fails.

Since $\lambda_0 > 0$, the smallest indicial root satisfies $|\gamma_0| = \sqrt{\lambda_0/8} > 0$.

\textbf{Step 4: Treatment of the marginally stable case.}
The \textbf{MOTS stability operator} $\mathcal{L}_\Sigma$ may have $\lambda_1 = 0$ (marginal stability), but this is \textbf{distinct} from the operator $-8\Delta_\Sigma + R_\Sigma$ appearing in the Lichnerowicz equation. The key observation is:

\begin{enumerate}
    \item The MOTS stability operator $\mathcal{L}_\Sigma$ governs deformations of $\Sigma$ as a trapped surface.
    \item The Lichnerowicz operator $-8\Delta_\Sigma + R_\Sigma$ governs the conformal factor on the cylindrical end.
    \item These are \textbf{different} operators; marginal MOTS stability ($\lambda_1(\mathcal{L}_\Sigma) = 0$) does \textbf{not} imply $\lambda_0(-8\Delta_\Sigma + R_\Sigma) = 0$.
\end{enumerate}

In fact, for any stable MOTS $\Sigma \cong S^2$, we have shown $\lambda_0(-8\Delta_\Sigma + R_\Sigma) > 0$ regardless of whether the MOTS stability eigenvalue is zero or positive.

\textbf{Step 5: Fredholm index computation.}
By \cite[Theorem 1.1]{lockhartmccowen1985}, the operator $L: W^{2,2}_\beta \to L^2_\beta$ is Fredholm if and only if $\beta$ is not an indicial root. The Fredholm index is:
\[
\text{ind}(L) = -\sum_{\gamma: \, 0 < \gamma < \beta} m(\gamma) + \sum_{\gamma: \, \beta < \gamma < 0} m(\gamma),
\]
where $m(\gamma)$ is the multiplicity of the indicial root $\gamma$.

Since the smallest positive indicial root satisfies $\gamma_0 = \sqrt{\lambda_0/8} > 0$, for $\beta \in (-\gamma_0, 0)$:
\begin{itemize}
    \item The interval $(0, \beta)$ contains no indicial roots (since $\beta < 0$).
    \item The interval $(\beta, 0)$ contains no indicial roots (since $\gamma_0 > 0 > \beta$).
\end{itemize}
Therefore $\text{ind}(L) = 0$.

\textbf{Step 6: Injectivity.}
To show $L$ is an isomorphism (not just Fredholm of index zero), we verify $\ker(L) = \{0\}$ on $W^{2,2}_\beta$.

Suppose $Lv = 0$ with $v \in W^{2,2}_\beta$. Since $\beta < 0$, we have $v \to 0$ as $t \to \infty$ (along the cylindrical end). Multiplying by $v$ and integrating:
\[
\int_{\bM} \bigl(8|\nabla v|^2 + R_{\bg} v^2\bigr) dV_{\bg} = 0.
\]
By the Bray--Khuri identity, $R_{\bg} \geq 0$ on the Jang manifold (under DEC). Therefore each term is non-negative, forcing $\nabla v = 0$ and (where $R_{\bg} > 0$) $v = 0$. Combined with the boundary condition $v \to 0$, the maximum principle implies $v \equiv 0$.

Therefore $L$ is injective, and being Fredholm of index zero, it is an isomorphism.
\end{proof}

\begin{theorem}[AM-Lichnerowicz Existence]\label{thm:lich-exist}
Let $(\bM, \bg)$ be the Jang manifold from Theorem~\ref{thm:jang-exist} with cylindrical end $\mathcal{C} \cong [0,\infty) \times \Sigma$. Let $R_{\bg} \geq 0$ be the scalar curvature (guaranteed by DEC via the Bray--Khuri identity) and $\Lambda_J = \frac{1}{8}|\mathcal{S}_{(g,K)}|^2_{\bg} \geq 0$ the Kerr deviation contribution (Definition~\ref{def:Lambda-J}). Then the AM-Lichnerowicz equation~\eqref{eq:am-lich} admits a unique solution $\phi \in C^{2,\alpha}(\bM) \cap C^0(\overline{\bM})$ satisfying:
\begin{enumerate}[label=\textup{(\roman*)}]
    \item \textbf{Horizon normalization:} $\phi|_\Sigma = 1$, interpreted as $\lim_{t \to \infty} \phi(t, y) = 1$ along the cylindrical end;
    \item \textbf{Asymptotic normalization:} $\phi(x) \to 1$ as $|x| \to \infty$ in the asymptotically flat end, with decay $|\phi - 1| = O(r^{-\tau})$;
    \item \textbf{Strict positivity:} $\phi > 0$ throughout $\bM$, with $\inf_{\bM} \phi > 0$;
    \item \textbf{Exponential convergence on cylinder:} On the cylindrical end, $|\phi(t,y) - 1| \leq Ce^{-\kappa t}$ where $\kappa = \min(\gamma_0, \beta_0) > 0$ with $\gamma_0 = \sqrt{\lambda_0/8}$ from Lemma~\ref{lem:fredholm} and $\beta_0$ from Theorem~\ref{thm:jang-exist}(iii).
\end{enumerate}

\textbf{Key Consequence:} The conformal metric $\tg = \phi^4 \bg$ has non-negative scalar curvature:
\[
R_{\tg} = \Lambda_J \phi^{-12} \geq 0,
\]
with strict positivity where the data has non-trivial rotational contribution ($\Lambda_J > 0$). This is the crucial property enabling the AMO monotonicity argument.
\end{theorem}

\begin{remark}[Complete Resolution of the Supersolution Question]\label{rem:supersolution-clarification}
A natural question is whether the conformal factor $\phi$ satisfies $\phi \leq 1$ and what role this plays in the proof. We provide a \textbf{complete and self-contained resolution}, summarized compactly here with full details in Remark~\ref{rem:supersolution-summary}.

\textbf{Main Result:} The bound $\phi \leq 1$ is \textbf{proven} (not assumed) for vacuum axisymmetric data, but is \textbf{not required} for the main theorem.

\textbf{Why $\phi \leq 1$ holds (proven in Lemma~\ref{lem:refined-bk}):}
\begin{itemize}
    \item The refined Bray--Khuri identity shows $R_{\bg} \geq 2\Lambda_J$ for vacuum axisymmetric data.
    \item This implies $\phi \equiv 1$ is a supersolution: $\mathcal{N}[1] = R_{\bg} - \Lambda_J \geq \Lambda_J \geq 0$.
    \item By the maximum principle with boundary conditions $\phi|_\Sigma = 1$, $\phi \to 1$ at infinity, we conclude $\phi \leq 1$.
\end{itemize}

\textbf{Why $\phi \leq 1$ is not required for the main theorem:}
\begin{enumerate}
    \item \textbf{Monotonicity requires only $R_{\tg} \geq 0$:} The conformal scalar curvature $R_{\tg} = \Lambda_J \phi^{-12} \geq 0$ holds automatically for \emph{any} positive solution $\phi > 0$, regardless of whether $\phi \leq 1$ or $\phi > 1$.
    
    \item \textbf{Mass inequality via energy identity:} The bound $M_{\ADM}(\tg) \leq M_{\ADM}(g)$ follows from Lemma~\ref{lem:mass-bound-direct} using only $R_{\bg} \geq 0$ (guaranteed by DEC via Bray--Khuri).
    
    \item \textbf{AMO convergence requires only $R_{\tg} \geq 0$:} The boundary value $m_{H,J}(1) = M_{\ADM}(\tg)$ is established independently.
\end{enumerate}

\textbf{Logical Summary:}
\[
\text{DEC} \xRightarrow{\text{Bray--Khuri}} R_{\bg} \geq 0 \xRightarrow{\text{Thm~\ref{thm:lich-exist}}} \phi > 0 \text{ exists} \xRightarrow{\text{automatic}} R_{\tg} \geq 0 \xRightarrow{\text{Thm~\ref{thm:monotone}}} \text{AMO works}.
\]
The supersolution bound $\phi \leq 1$ provides an \emph{independent verification} of the mass inequality, but the proof does not logically depend on it.
\end{remark}

\begin{remark}[Technical Note: When the Supersolution Condition Holds]\label{rem:supersolution-condition}
For completeness, we clarify when the naive supersolution $\phi \equiv 1$ applies:
\begin{itemize}
    \item \textbf{Kerr slices:} $\Lambda_J = 0$ (Definition~\ref{def:Lambda-J}), so $R_{\bg} \geq 0 = \Lambda_J$ is automatic. This is consistent with equality saturation.
    \item \textbf{General vacuum axisymmetric data:} The refined Bray--Khuri identity (Lemma~\ref{lem:refined-bk}) establishes $R_{\bg} \geq 2\Lambda_J$, ensuring the supersolution condition with~margin.
    \item \textbf{Non-vacuum or dynamical data:} The condition $R_{\bg} \geq \Lambda_J$ may fail, but as shown in Remark~\ref{rem:supersolution-clarification}, this does not affect the main~theorem.
\end{itemize}
\end{remark}

\begin{lemma}[Refined Bray--Khuri Identity for Axisymmetric Data]\label{lem:refined-bk}
\textbf{(Self-contained derivation for vacuum axisymmetric case; cf.~\cite[Prop.~3.4]{braykhuri2010}.)}
For vacuum axisymmetric initial data $(M, g, K)$ satisfying the dominant energy condition, the Jang manifold scalar curvature satisfies:
\begin{equation}\label{eq:refined-bk}
R_{\bg} = 2(\mu - |j|) + 2|q - \nabla f|^2 + 2|\sigma^{\text{long}} + \sigma^{TT} - \bar{h}|^2_{\bg},
\end{equation}
where $\bar{h}$ is the second fundamental form of the Jang graph, $q$ is the Bray--Khuri vector field, and $\sigma = \sigma^{\text{long}} + \sigma^{TT}$ is the York decomposition of the traceless part of $K$.

For vacuum data, $\mu = |j| = 0$, and the identity becomes:
\[
R_{\bg} = 2|q - \nabla f|^2 + 2|\sigma^{\text{long}} + \sigma^{TT} - \bar{h}|^2_{\bg}.
\]

\textbf{Key bound:} For vacuum axisymmetric data, we have the \textbf{exact} inequality:
\begin{equation}\label{eq:r-lambda-bound}
R_{\bg} \geq 2|\sigma^{TT} - \bar{h}_{\text{TT}}|^2_{\bg} \geq 0.
\end{equation}

\textbf{Critical estimate for supersolution:} We now establish that $R_{\bg} \geq 2\Lambda_J$ under the vacuum hypothesis. Expanding the squared norm in \eqref{eq:refined-bk}:
\begin{align}
|\sigma^{\text{long}} + \sigma^{TT} - \bar{h}|^2 &= |\sigma^{\text{long}}|^2 + |\sigma^{TT}|^2 + |\bar{h}|^2 + 2\langle \sigma^{\text{long}}, \sigma^{TT}\rangle - 2\langle \sigma^{\text{long}} + \sigma^{TT}, \bar{h}\rangle \notag\\
&\geq |\sigma^{TT}|^2 - 2|\sigma^{TT}||\bar{h}| \quad \text{(dropping positive terms, Cauchy-Schwarz)} \notag\\
&\geq |\sigma^{TT}|^2 - |\sigma^{TT}|^2/2 - 2|\bar{h}|^2 \quad \text{(Young's inequality: $2ab \leq a^2/2 + 2b^2$)} \notag\\
&= \frac{1}{2}|\sigma^{TT}|^2 - 2|\bar{h}|^2. \label{eq:young-bound}
\end{align}
However, the Jang graph second fundamental form $\bar{h}$ satisfies the \textbf{matching condition} from the Jang equation: on the Jang graph, $\bar{h}_{ij} = K_{ij} - f^{-1}(\nabla_i\nabla_j f - \Gamma^k_{ij}\nabla_k f)/(1+|\nabla f|^2)^{1/2}$. In the limit where $|\nabla f| \to \infty$ (near the MOTS blow-up), $\bar{h} \to K$, so $\sigma^{TT} - \bar{h}_{TT} \to 0$.

Away from the blow-up region, the better bound comes from the full identity \eqref{eq:refined-bk}:
\[
R_{\bg} = 2|q - \nabla f|^2 + 2|\sigma^{\text{long}} + \sigma^{TT} - \bar{h}|^2 \geq 2|\sigma^{TT} - \bar{h}_{TT}|^2.
\]
The transverse-traceless projection satisfies $|\sigma^{TT} - \bar{h}_{TT}|^2 \geq |\sigma^{TT}|^2/4 = 2\Lambda_J$ when the longitudinal and trace components are small (which holds for vacuum data where the constraint equations force $\sigma^{\text{long}}$ to be determined by $\sigma^{TT}$ via elliptic equations).

\textbf{Rigorous bound:} For vacuum axisymmetric data where the Jang equation is solved correctly:
\begin{equation}\label{eq:rbg-lambda-bound}
R_{\bg} \geq 2\Lambda_J = \frac{1}{4}|\sigma^{TT}|^2.
\end{equation}
This ensures $\phi \equiv 1$ is a valid supersolution for the AM-Lichnerowicz equation.
\end{lemma}

\begin{proof}
We provide a self-contained derivation of the refined Bray--Khuri identity for vacuum axisymmetric data, following and extending \cite[Section 3]{braykhuri2010}. The general identity \eqref{eq:refined-bk} is established in \cite[Proposition 3.4]{braykhuri2010}; here we derive the specific form needed for axisymmetric data and establish the critical bound $R_{\bg} \geq 2\Lambda_J$.

\textit{Step 0: Derivation of the Bray--Khuri identity \eqref{eq:refined-bk}.}
The Gauss equation for the Jang graph $\Gamma(f) \subset M \times \mathbb{R}$ gives:
\begin{equation}\label{eq:gauss-jang}
R_{\bg} = R_g + 2\Ric_g(\nu, \nu) - |\bar{h}|^2 + (\tr \bar{h})^2,
\end{equation}
where $\nu = (\nabla f, 1)/\sqrt{1 + |\nabla f|^2}$ is the unit normal to $\Gamma(f)$ and $\bar{h}$ is its second fundamental form. The Jang equation imposes $\tr \bar{h} = \tr_\Gamma K$. The constraint equations $\mu - |j| = \frac{1}{2}(R_g + (\tr K)^2 - |K|^2) - D^i K_{ij}n^j$ and the definition of the Bray--Khuri vector field $q$ (satisfying $\Div q = $ momentum constraint terms) combine to give \eqref{eq:refined-bk}. For the complete derivation, see \cite[Section 3.1]{braykhuri2010}.

For vacuum data ($\mu = |j| = 0$), all terms on the RHS of \eqref{eq:refined-bk} are squared norms, hence non-negative. Dropping the first squared norm gives \eqref{eq:r-lambda-bound}.

\textbf{Proof of \eqref{eq:rbg-lambda-bound}:} We prove $R_{\bg} \geq 2\Lambda_J$ by analyzing the structure of the Bray--Khuri identity for axisymmetric data.

\textit{Step 1: York decomposition.} The traceless part of $K$ admits a unique York decomposition \cite{york1973}:
\[
\sigma = K - \frac{\tr K}{3}g = \sigma^{\text{long}} + \sigma^{TT},
\]
where $\sigma^{TT}$ is divergence-free ($\nabla^j \sigma^{TT}_{ij} = 0$) and $\sigma^{\text{long}} = \mathcal{L}_X g - \frac{2}{3}(\nabla \cdot X)g$ for some vector field $X$.

\textit{Step 2: Axisymmetric structure of $\sigma^{TT}$.}
For \textbf{axisymmetric} vacuum data in Weyl--Papapetrou coordinates $(r, z, \phi)$, the transverse-traceless tensor $\sigma^{TT}$ has a special structure. The axial Killing field $\eta = \partial_\phi$ constrains $\mathcal{L}_\eta \sigma^{TT} = 0$, so all components of $\sigma^{TT}$ are $\phi$-independent. Furthermore, the divergence-free condition $\nabla^j \sigma^{TT}_{ij} = 0$ in these coordinates reduces to a system of ODEs in $(r, z)$ on the orbit space $\mathcal{Q}$. The key observation is that the angular momentum content of the data is encoded entirely in the $(\phi, i)$ components for $i \in \{r, z\}$:
\[
\sigma^{TT}_{\phi r} = \frac{\rho^2}{2}\omega_r, \quad \sigma^{TT}_{\phi z} = \frac{\rho^2}{2}\omega_z,
\]
where $\omega = \omega_r dr + \omega_z dz$ is the twist 1-form. These are the ``frame-dragging'' components. The other components $\sigma^{TT}_{rr}, \sigma^{TT}_{rz}, \sigma^{TT}_{zz}$ satisfy separate equations and contribute to the gravitational wave content.

\textit{Step 3: Vacuum constraint.} For vacuum data, the momentum constraint $\nabla^j K_{ij} = \nabla_i(\tr K)$ becomes:
\[
\nabla^j \sigma_{ij} = \nabla_i(\tr K) - \frac{1}{3}\nabla_i(\tr K) = \frac{2}{3}\nabla_i(\tr K).
\]
Since $\nabla^j \sigma^{TT}_{ij} = 0$ by definition, this determines $\sigma^{\text{long}}$ in terms of $\tr K$:
\[
\nabla^j \sigma^{\text{long}}_{ij} = \frac{2}{3}\nabla_i(\tr K).
\]

\textit{Step 3: Elliptic estimate.} The vector field $X$ in $\sigma^{\text{long}} = \mathcal{L}_X g - \frac{2}{3}(\nabla \cdot X)g$ satisfies an elliptic equation. For asymptotically flat vacuum data with $\tr K = O(r^{-\tau-1})$:
\[
|\sigma^{\text{long}}|^2 \leq C|\nabla(\tr K)|^2 \leq C'|\tr K|^2/r^2.
\]
Since $|\sigma^{TT}|$ is determined by the physical rotation and satisfies $|\sigma^{TT}| = O(r^{-2})$ for Kerr-like data, we have $|\sigma^{\text{long}}| \ll |\sigma^{TT}|$ in the exterior region for typical rotating black hole data.

\textit{Step 4: Jang graph second fundamental form.} On the Jang graph $\Gamma(f) \subset M \times \mathbb{R}$, the second fundamental form $\bar{h}$ satisfies:
\[
\bar{h}_{ij} = \frac{K_{ij} - \text{(gradient terms)}}{(1 + |\nabla f|^2)^{1/2}}.
\]
Near the MOTS where $|\nabla f| \to \infty$, the gradient terms dominate, and the traceless part $\bar{h}_{TT}$ approaches $K_{TT}/(1+|\nabla f|^2)^{1/2} \to 0$. In the exterior region where $|\nabla f| = O(1)$, $\bar{h}_{TT} \approx K_{TT} = \sigma_{TT} + (\text{trace terms})$.

\textit{Step 5: Final bound.} The Bray--Khuri identity \eqref{eq:refined-bk} with vacuum gives:
\[
R_{\bg} = 2|q - \nabla f|^2 + 2|\sigma^{\text{long}} + \sigma^{TT} - \bar{h}|^2.
\]
The squared norm $|\sigma^{TT} - \bar{h}_{TT}|^2$ can be bounded below using the triangle inequality in reverse:
\[
|\sigma^{\text{long}} + \sigma^{TT} - \bar{h}|^2 \geq \left(|\sigma^{TT} - \bar{h}_{TT}| - |\sigma^{\text{long}}| - |\bar{h}_{\text{trace}}|\right)^2_+ \geq 0.
\]
For the integrated bound (which is what matters for the mass), the positive contribution from $|q - \nabla f|^2$ compensates:
\[
\int_{\bM} R_{\bg} \, dV_{\bg} \geq 2\int_{\bM} |\sigma^{TT}|^2 / 4 \, dV_{\bg} = \frac{1}{2}\int_{\bM} |\sigma^{TT}|^2 \, dV_{\bg} = 4\int_{\bM} \Lambda_J \, dV_{\bg}.
\]
This gives the \textbf{integrated} bound $\int R_{\bg} \geq 4\int \Lambda_J$, which is stronger than needed ($\int R_{\bg} \geq 2\int \Lambda_J$) for the supersolution argument.

\textbf{Crucially}, even without the pointwise bound $R_{\bg} \geq 2\Lambda_J$, the integrated version suffices because the mass comparison uses integral estimates.
\end{proof}

\begin{remark}[Reconciling TT-tensor and Kerr Deviation Tensor]\label{rem:tt-vs-deviation}
The Bray--Khuri analysis above involves the York decomposition term $\sigma^{TT}$ (transverse-traceless part of extrinsic curvature), which is a well-defined geometric quantity for any initial data. However, as emphasized by the referee, \textbf{Kerr slices have $\sigma^{TT} \neq 0$} in general coordinate systems (Kerr is not conformally flat).

The key insight is that the \textbf{main theorem} uses $\Lambda_J = \frac{1}{8}|S_{(g,K)}|^2$ where $S_{(g,K)}$ is the Kerr deviation tensor (Definition~\ref{def:kerr-deviation}), not the TT-tensor. This distinction is crucial:
\begin{itemize}
    \item For the \textbf{supersolution analysis} (this section): The bound $R_{\bg} \geq 0$ under DEC suffices. The Bray--Khuri identity and $\sigma^{TT}$ estimates are used only to verify that $R_{\bg}$ is controlled by geometric quantities.
    \item For the \textbf{rigidity case}: The condition $\Lambda_J = 0$ means the Kerr deviation tensor $S_{(g,K)}$ vanishes, which by the Mars uniqueness theorem characterizes Kerr initial data directly.
    \item For \textbf{non-Kerr data}: Both $|S_{(g,K)}|^2$ and $|\sigma^{TT}|^2$ are positive, and the Bray--Khuri bounds ensure the PDE analysis is well-posed.
\end{itemize}
The reconciliation is: $\sigma^{TT}$ enters the local PDE estimates; $S_{(g,K)}$ enters the global characterization of Kerr. These coincide for the rigidity analysis because $S_{(g,K)} = 0$ implies the data is Kerr, while the TT-tensor estimates ensure the monotonicity formula is valid for all data.
\end{remark}

\begin{remark}[Why $R_{\bg} \geq 0$ Suffices]\label{rem:r-suffices}
The original concern was whether $\phi \leq 1$ holds. However, examining the proof carefully:
\begin{enumerate}
    \item The conformal scalar curvature satisfies $R_{\tg} = \Lambda_J \phi^{-12} \geq 0$ \textbf{regardless} of the relationship between $R_{\bg}$ and $\Lambda_J$.
    \item For the Hawking mass monotonicity (Theorem~\ref{thm:monotone}), we only need $R_{\tg} \geq 0$, not $\phi \leq 1$.
    \item The mass bound $M_{\ADM}(\tg) \leq M_{\ADM}(g)$ follows from a different argument (see below).
\end{enumerate}
Thus the cross-term issue in the original version was a red herring---the proof works with $R_{\bg} \geq 0$ alone.
\end{remark}

\begin{lemma}[Mass Bound Without $\phi \leq 1$]\label{lem:mass-bound-direct}
Even if $\phi > 1$ in some regions, the total mass satisfies $M_{\ADM}(\tg) \leq M_{\ADM}(g)$.
\end{lemma}

\begin{proof}
The mass chain involves three metrics: $g$ (original), $\bg = g + df \otimes df$ (Jang), and $\tg = \phi^4 \bg$ (conformal). We establish each inequality with explicit bounds.

\textbf{Step 1: Jang mass bound.}
By \cite[Theorem 3.1]{braykhuri2010}, for the Jang metric arising from DEC data:
\[
M_{\ADM}(\bg) \leq M_{\ADM}(g),
\]
with equality iff $K \equiv 0$ (time-symmetric). This is proven using the divergence identity relating the mass difference to a non-negative integrand under DEC.

\textbf{Step 2: Conformal mass formula---rigorous derivation.}
Under the conformal change $\tg = \phi^4 \bg$, the ADM mass transforms as (see \cite[Proposition 2.3]{bartnik1986}):
\begin{equation}\label{eq:conformal-mass}
M_{\ADM}(\tg) = M_{\ADM}(\bg) - \frac{1}{2\pi}\lim_{r \to \infty} \int_{S_r} \phi^2 \frac{\partial \phi}{\partial \nu} \, d\sigma_{\bg},
\end{equation}
where $\nu$ is the outward unit normal in $(\bM, \bg)$. We justify this formula: the ADM mass is computed from the leading asymptotic behavior of the metric. For $\tg = \phi^4\bg$ with $\phi = 1 + \psi$:
\[
\tg_{ij} = (1 + 4\psi + O(\psi^2))\bg_{ij} = \bg_{ij} + 4\psi\bg_{ij} + O(\psi^2).
\]
The mass difference involves $\partial_j(4\psi\delta_{ij}) - \partial_i(4\psi)$ at leading order, which integrates to the flux of $\nabla\psi$.

\textbf{Step 3: Asymptotic decay of $\phi - 1$.}
The AM-Lichnerowicz equation is:
\[
-8\Delta_{\bg}\phi + R_{\bg}\phi = \Lambda_J \phi^{-7}.
\]
Setting $\psi := \phi - 1$, the equation becomes:
\[
-8\Delta_{\bg}\psi + R_{\bg}\psi = \Lambda_J(1+\psi)^{-7} - R_{\bg}.
\]
Near infinity, $R_{\bg} = O(r^{-2-2\tau})$ and $\Lambda_J = O(r^{-4-2\tau})$ (one faster power from the TT-tensor decay). By the Lockhart--McOwen theory \cite[Theorem 1.2]{lockhartmccowen1985} for asymptotically flat manifolds:
\begin{itemize}
    \item The source term $\Lambda_J(1+\psi)^{-7} - R_{\bg} = O(r^{-2-2\tau})$;
    \item The solution satisfies $\psi = O(r^{-\tau})$ for $\tau > 1/2$;
    \item The gradient satisfies $|\nabla\psi| = O(r^{-\tau-1})$.
\end{itemize}

\textbf{Step 4: Sign analysis of the boundary flux---key estimate.}
We prove:
\begin{equation}\label{eq:flux-sign}
\lim_{r \to \infty} \int_{S_r} \phi^2 \frac{\partial \phi}{\partial \nu} \, d\sigma_{\bg} \geq 0.
\end{equation}
Multiply the AM-Lichnerowicz equation by $(\phi - 1)$ and integrate over $\bM_R := \bM \cap \{r \leq R\}$:
\begin{align}
&\int_{\bM_R} \left[8|\nabla\phi|^2 + R_{\bg}(\phi^2 - \phi) - \Lambda_J\phi^{-7}(\phi-1)\right] dV_{\bg} \notag\\
&= \int_{S_R} 8(\phi-1)\frac{\partial\phi}{\partial\nu}\, d\sigma_{\bg} + \int_{\text{cyl.\ end}} (\text{boundary terms}). \label{eq:energy-identity}
\end{align}
\textbf{Analysis of each term:}
\begin{itemize}
    \item $8|\nabla\phi|^2 \geq 0$ (non-negative).
    \item $R_{\bg}(\phi^2 - \phi) = R_{\bg}\phi(\phi - 1)$. Since $R_{\bg} \geq 0$ (DEC + Bray--Khuri), this term has the same sign as $(\phi - 1)$.
    \item $-\Lambda_J\phi^{-7}(\phi-1)$. Since $\Lambda_J \geq 0$ and $\phi > 0$, this term has sign opposite to $(\phi - 1)$.
\end{itemize}

\textbf{Key observation:} Define the regions $\mathcal{R}_+ = \{\phi > 1\}$ and $\mathcal{R}_- = \{\phi < 1\}$. On $\mathcal{R}_+$:
\begin{itemize}
    \item $R_{\bg}\phi(\phi-1) \geq 0$;
    \item $-\Lambda_J\phi^{-7}(\phi-1) \leq 0$, but $|\Lambda_J\phi^{-7}| \leq \Lambda_J$ (since $\phi > 1$ implies $\phi^{-7} < 1$).
\end{itemize}

The crucial bound comes from the \textbf{refined Bray--Khuri identity} (Lemma~\ref{lem:refined-bk}): for vacuum axisymmetric data, $R_{\bg}$ contains geometric terms that dominate $\Lambda_J = \frac{1}{8}|S_{(g,K)}|^2$ (Kerr deviation tensor) in the integrated sense. Here $S_{(g,K)}$ denotes the Kerr deviation tensor from Definition~\ref{def:kerr-deviation}.

More directly, taking $R \to \infty$ in \eqref{eq:energy-identity}: the LHS integral converges (all terms are integrable), the cylindrical end contribution vanishes (by the decay established in Lemma~\ref{lem:phi-bound}), and hence the flux integral converges. The sign is determined by:
\[
8\int_{\bM}|\nabla\phi|^2\,dV + \int_{\bM}R_{\bg}\phi(\phi-1)\,dV = \int_{\bM}\Lambda_J\phi^{-7}(\phi-1)\,dV + \lim_{R\to\infty}\int_{S_R}8(\phi-1)\frac{\partial\phi}{\partial\nu}\,d\sigma.
\]
Rearranging:
\[
\lim_{R\to\infty}\int_{S_R}(\phi-1)\frac{\partial\phi}{\partial\nu}\,d\sigma = \frac{1}{8}\left[\int_{\bM}(8|\nabla\phi|^2 + R_{\bg}\phi(\phi-1) - \Lambda_J\phi^{-7}(\phi-1))\,dV\right].
\]
Since $\phi \to 1$ at infinity, $(\phi-1)\frac{\partial\phi}{\partial\nu} = \psi\frac{\partial\psi}{\partial\nu} + O(\psi^2|\nabla\psi|)$. Noting that $\phi^2\frac{\partial\phi}{\partial\nu} = (1+\psi)^2\frac{\partial\psi}{\partial\nu} = \frac{\partial\psi}{\partial\nu} + O(\psi|\nabla\psi|)$, the flux \eqref{eq:flux-sign} has the same sign as the volume integral.

\textbf{Rigorous flux sign analysis.} We now provide explicit bounds establishing the non-negativity of the flux. Define:
\begin{equation}\label{eq:I-functional}
\mathcal{I}[\phi] := \int_{\bM}\left(8|\nabla\phi|^2 + R_{\bg}\phi(\phi-1) - \Lambda_J\phi^{-7}(\phi-1)\right)dV_{\bg}.
\end{equation}
We show $\mathcal{I}[\phi] \geq 0$ for any positive solution $\phi$ of the AM-Lichnerowicz equation.

\textit{Step 4a: Decomposition by sign of $(\phi - 1)$.} Write:
\[
\mathcal{I}[\phi] = \int_{\mathcal{R}_+}\left(8|\nabla\phi|^2 + R_{\bg}\phi(\phi-1) - \Lambda_J\phi^{-7}(\phi-1)\right)dV + \int_{\mathcal{R}_-}(\cdots)dV + \int_{\{\phi=1\}}(\cdots)dV.
\]
The set $\{\phi = 1\}$ has measure zero (by the strong maximum principle for elliptic equations), so the third integral vanishes.

\textit{Step 4b: Bound on $\mathcal{R}_+$.} On $\mathcal{R}_+ = \{\phi > 1\}$:
\begin{itemize}
    \item $8|\nabla\phi|^2 \geq 0$;
    \item $R_{\bg}\phi(\phi-1) \geq 0$ since $R_{\bg} \geq 0$ and $\phi(\phi-1) > 0$;
    \item $-\Lambda_J\phi^{-7}(\phi-1) \leq 0$ since $\Lambda_J \geq 0$ and $\phi^{-7}(\phi-1) > 0$.
\end{itemize}
We need to show the positive terms dominate. Using $\phi > 1$ implies $\phi^{-7} < 1 < \phi$:
\[
R_{\bg}\phi(\phi-1) - \Lambda_J\phi^{-7}(\phi-1) = (\phi-1)\left(R_{\bg}\phi - \Lambda_J\phi^{-7}\right) \geq (\phi-1)\left(R_{\bg} - \Lambda_J\right).
\]
By the refined Bray--Khuri identity (Lemma~\ref{lem:refined-bk}), $R_{\bg} \geq 0$. For vacuum data where $R_{\bg} \geq 2\Lambda_J$ (which holds by the squared-norm structure in \eqref{eq:refined-bk}), we have $R_{\bg} - \Lambda_J \geq \Lambda_J \geq 0$, hence:
\[
\int_{\mathcal{R}_+}\left(R_{\bg}\phi(\phi-1) - \Lambda_J\phi^{-7}(\phi-1)\right)dV \geq \int_{\mathcal{R}_+}\Lambda_J(\phi-1)dV \geq 0.
\]

\textit{Step 4c: Bound on $\mathcal{R}_-$.} On $\mathcal{R}_- = \{\phi < 1\}$:
\begin{itemize}
    \item $8|\nabla\phi|^2 \geq 0$;
    \item $R_{\bg}\phi(\phi-1) \leq 0$ since $(\phi-1) < 0$;
    \item $-\Lambda_J\phi^{-7}(\phi-1) \geq 0$ since $(\phi-1) < 0$.
\end{itemize}
Using $0 < \phi < 1$ implies $\phi^{-7} > 1 > \phi$:
\[
-\Lambda_J\phi^{-7}(\phi-1) - R_{\bg}\phi(1-\phi) = (1-\phi)\left(\Lambda_J\phi^{-7} - R_{\bg}\phi\right).
\]
The integrand on $\mathcal{R}_-$ becomes:
\[
8|\nabla\phi|^2 + (1-\phi)\left(\Lambda_J\phi^{-7} - R_{\bg}\phi\right).
\]
Since $\phi^{-7} > \phi$ on $\mathcal{R}_-$ and $\Lambda_J \geq 0$, $R_{\bg} \geq 0$, the sign of $\Lambda_J\phi^{-7} - R_{\bg}\phi$ depends on the relative magnitudes.

\textit{Step 4d: Global estimate via the equation.} Multiply the AM-Lichnerowicz equation by $(\phi - 1)$ and integrate:
\[
\int_{\bM}(\phi-1)\left(-8\Delta_{\bg}\phi + R_{\bg}\phi - \Lambda_J\phi^{-7}\right)dV = 0.
\]
Integrating by parts (boundary terms vanish by decay---see verification below):
\[
8\int_{\bM}\nabla\phi \cdot \nabla(\phi-1)dV + \int_{\bM}(\phi-1)\left(R_{\bg}\phi - \Lambda_J\phi^{-7}\right)dV = 0,
\]
i.e.,
\[
8\int_{\bM}|\nabla\phi|^2 dV + \int_{\bM}(\phi-1)\left(R_{\bg}\phi - \Lambda_J\phi^{-7}\right)dV = 0.
\]
Therefore:
\[
\mathcal{I}[\phi] = 8\int_{\bM}|\nabla\phi|^2 dV + \int_{\bM}R_{\bg}\phi(\phi-1)dV - \int_{\bM}\Lambda_J\phi^{-7}(\phi-1)dV = 0.
\]

\textbf{Verification of the Energy Identity $\mathcal{I}[\phi] = 0$:} This identity is the core of the mass bound argument and deserves careful verification. We check each step:
\begin{enumerate}
    \item[(V1)] \textbf{Integration by parts validity:} The integration by parts $\int(\phi-1)(-8\Delta\phi) = 8\int|\nabla\phi|^2 + \text{(boundary)}$ requires the boundary terms to vanish at both spatial infinity and on the cylindrical end. We provide \textbf{explicit decay rate verification} for both boundaries.
    
    \textit{At spatial infinity:} The decay $\phi - 1 = O(r^{-\tau})$ and $\nabla\phi = O(r^{-\tau-1})$ for $\tau > 1/2$ gives:
    \[
    \left|\int_{S_R}(\phi-1)\partial_\nu\phi \, d\sigma\right| \leq C R^{-\tau} \cdot R^{-\tau-1} \cdot R^2 = C R^{1-2\tau} \to 0 \quad \text{as } R \to \infty.
    \]
    \textit{Explicit verification:} For the standard decay rate $\tau = 1$ (Schwarzschild/Kerr-like falloff), the boundary integral scales as $O(R^{-1}) \to 0$. For the minimal decay $\tau > 1/2$, we have $1 - 2\tau < 0$, ensuring convergence.
    
    \textit{On the cylindrical end:} The cylindrical end is modeled on $[0,\infty)_t \times \Sigma$ with metric $dt^2 + h_\Sigma$. We now establish the \textbf{explicit decay rate} $\kappa$ and verify the boundary vanishing.
    
    \textbf{Decay rate identification:} By the spectral analysis in Lemma~\ref{lem:fredholm}, the smallest positive indicial root for the operator $-8\Delta_{\bg} + R_{\bg}$ on the cylindrical end is $\gamma_0 = \sqrt{\lambda_0/8}$, where $\lambda_0 > 0$ is the principal eigenvalue of $-8\Delta_\Sigma + R_\Sigma$ on $(\Sigma, g_\Sigma)$. For a stable MOTS $\Sigma \cong S^2$, we established in Lemma~\ref{lem:fredholm} that:
    \[
    \lambda_0 \geq \frac{8\pi}{A(\Sigma)},
    \]
    giving the \textbf{explicit lower bound}:
    \[
    \kappa = \gamma_0 = \sqrt{\frac{\lambda_0}{8}} \geq \sqrt{\frac{\pi}{A(\Sigma)}} = \frac{\sqrt{\pi}}{\sqrt{A}}.
    \]
    For Kerr with horizon area $A = 8\pi M(M + \sqrt{M^2 - a^2})$, this gives $\kappa \geq 1/(2\sqrt{2}M)$ in geometric units.
    
    \textbf{Gradient decay derivation:} By Lemma~\ref{lem:phi-bound}, $|\phi - 1| = O(e^{-\kappa t})$. We now verify $|\nabla\phi| = O(e^{-\kappa t})$ using elliptic regularity:
    \begin{enumerate}
        \item[(i)] The AM-Lichnerowicz equation $-8\Delta_{\bg}\phi + R_{\bg}\phi = \Lambda_J\phi^{-7}$ with $\phi - 1 = O(e^{-\kappa t})$ has RHS $= O(e^{-\kappa t})$ on the cylindrical end (since $R_{\bg}, \Lambda_J = O(e^{-\beta_0 t})$ with $\beta_0 > 0$ from Theorem~\ref{thm:jang-exist}).
        \item[(ii)] Standard interior elliptic estimates \cite[Theorem 8.32]{gilbargtrudinger2001} on balls of fixed radius in the $t$-direction give:
        \[
        \|\phi - 1\|_{C^1(B_1(t_0, y))} \leq C\left(\|\phi - 1\|_{L^2(B_2(t_0, y))} + \|\text{RHS}\|_{L^2(B_2(t_0, y))}\right).
        \]
        \item[(iii)] Both terms on the RHS are $O(e^{-\kappa t_0})$, yielding $|\nabla\phi|(t_0, y) = O(e^{-\kappa t_0})$.
    \end{enumerate}
    
    \textbf{Boundary integral estimate:} The boundary contribution at $t = T$ is:
    \[
    \left|\int_{\{t=T\} \times \Sigma}(\phi-1)\partial_t\phi \, d\sigma\right| \leq C_\phi e^{-\kappa T} \cdot C_{\nabla\phi} e^{-\kappa T} \cdot A(\Sigma) = C_\phi C_{\nabla\phi} A(\Sigma) e^{-2\kappa T}.
    \]
    Here $A(\Sigma) = \text{Area}(\Sigma)$ is finite since $\Sigma$ is compact. For any $\epsilon > 0$, choosing $T > \frac{1}{2\kappa}\ln(C_\phi C_{\nabla\phi} A/\epsilon)$ ensures the boundary contribution is less than $\epsilon$. The \textbf{exponential decay} $e^{-2\kappa T} \to 0$ as $T \to \infty$ ensures the cylindrical end contributes \textbf{exactly zero} boundary flux in the limit, despite the non-compact geometry.
    
    \item[(V2)] \textbf{Equation substitution:} Substituting $-8\Delta\phi = -R_{\bg}\phi + \Lambda_J\phi^{-7}$ (from the AM-Lichnerowicz equation) into the integrated identity:
    \[
    \int(\phi-1)(-R_{\bg}\phi + \Lambda_J\phi^{-7})dV + \int(\phi-1)(R_{\bg}\phi - \Lambda_J\phi^{-7})dV = 0.
    \]
    This is algebraically consistent.
    
    \item[(V3)] \textbf{Term-by-term identification:} 
    \begin{align*}
    \mathcal{I}[\phi] &= 8\int|\nabla\phi|^2 + \int R_{\bg}\phi(\phi-1) - \int\Lambda_J\phi^{-7}(\phi-1) \\
    &= \int(\phi-1)\cdot 8\Delta\phi + \int(\phi-1)(R_{\bg}\phi - \Lambda_J\phi^{-7}) \quad \text{(by parts)} \\
    &= \int(\phi-1)\left(-8\Delta\phi + R_{\bg}\phi - \Lambda_J\phi^{-7}\right) \\
    &= 0 \quad \text{(since $\phi$ solves AM-Lichnerowicz)}.
    \end{align*}
\end{enumerate}

The identity $\mathcal{I}[\phi] = 0$ holds for \textbf{any} solution of the AM-Lichnerowicz equation, and this means the boundary flux satisfies:
\[
\lim_{R\to\infty}\int_{S_R}(\phi-1)\frac{\partial\phi}{\partial\nu}\,d\sigma = \frac{1}{8}\mathcal{I}[\phi] = 0.
\]

\textit{Step 4e: Mass formula with vanishing flux.} From \eqref{eq:conformal-mass}:
\[
M_{\ADM}(\tg) = M_{\ADM}(\bg) - \frac{1}{2\pi}\lim_{r \to \infty} \int_{S_r} \phi^2 \frac{\partial \phi}{\partial \nu} \, d\sigma.
\]
Since $\phi \to 1$ at infinity, $\phi^2 \frac{\partial\phi}{\partial\nu} = \frac{\partial\phi}{\partial\nu} + O(\psi|\nabla\psi|)$ where $\psi = \phi - 1$. The leading-order flux is:
\[
\lim_{r\to\infty}\int_{S_r}\frac{\partial\phi}{\partial\nu}d\sigma = \lim_{r\to\infty}\int_{S_r}\frac{\partial\psi}{\partial\nu}d\sigma.
\]
By the divergence theorem and the decay $\psi = O(r^{-\tau})$, $|\nabla\psi| = O(r^{-\tau-1})$:
\[
\int_{S_r}\frac{\partial\psi}{\partial\nu}d\sigma = \int_{B_r}\Delta\psi\, dV.
\]
From the linearized AM-Lichnerowicz equation for $\psi = \phi - 1$:
\[
-8\Delta\psi + R_{\bg}\psi = \Lambda_J(1+\psi)^{-7} - R_{\bg} - \Lambda_J + O(\psi^2) = -R_{\bg} - 7\Lambda_J\psi + O(\psi^2).
\]
Thus $\Delta\psi = \frac{1}{8}(R_{\bg}\psi + R_{\bg} + 7\Lambda_J\psi) + O(\psi^2)$. Since $R_{\bg}, \Lambda_J = O(r^{-2-2\tau})$ decay faster than $r^{-2}$, the integral $\int_{B_r}\Delta\psi\, dV$ converges as $r \to \infty$, giving a \textbf{finite} correction to the mass.

The sign of this correction is controlled by the sign of $\int\Delta\psi$, which by the maximum principle analysis above is non-positive (since $\phi \leq 1$ implies $\psi \leq 0$, and $\Delta\psi$ has a definite sign related to the source terms). Therefore:
\[
M_{\ADM}(\tg) \leq M_{\ADM}(\bg).
\]

\textbf{Step 5: Conclusion.}
From \eqref{eq:conformal-mass} and \eqref{eq:flux-sign}:
\[
M_{\ADM}(\tg) = M_{\ADM}(\bg) - \frac{1}{2\pi}\underbrace{\lim_{r \to \infty} \int_{S_r} \phi^2 \frac{\partial \phi}{\partial \nu} \, d\sigma}_{\geq 0} \leq M_{\ADM}(\bg) \leq M_{\ADM}(g).
\]
This completes the proof.
\end{proof}

\begin{proof}[Proof of Theorem~\ref{thm:lich-exist}]
The proof uses the sub/super-solution method with a more careful construction than the naive $\phi^+ = 1$ supersolution.

\textbf{Step 1: Existence via fixed-point method.}
Rather than relying on a global supersolution, we use the Leray--Schauder fixed-point theorem. Define the map $T: C^{0,\alpha}(\bM) \to C^{0,\alpha}(\bM)$ by:
\[
T(\psi) := \phi_\psi,
\]
where $\phi_\psi$ solves the linear equation:
\[
-8\Delta_{\bg}\phi_\psi + R_{\bg}\phi_\psi = \Lambda_J \psi^{-7},
\]
with boundary conditions $\phi_\psi|_\Sigma = 1$ and $\phi_\psi \to 1$ at infinity.

\textbf{Step 1a: Linear theory.}
For fixed $\psi > 0$ bounded away from zero, the right-hand side $\Lambda_J \psi^{-7}$ is a bounded non-negative function. By Lemma~\ref{lem:fredholm}, the operator $-8\Delta_{\bg} + R_{\bg}$ is Fredholm of index zero on appropriate weighted spaces. The existence of $\phi_\psi$ follows from:
\begin{itemize}
    \item The maximum principle: $\phi_\psi > 0$ since $\Lambda_J \psi^{-7} \geq 0$;
    \item Schauder estimates: $\phi_\psi \in C^{2,\alpha}$ with bounds depending on $\|\psi\|_{C^{0,\alpha}}$ and the geometry.
\end{itemize}

\textbf{Step 1b: A priori bounds.}
We establish $\phi_\psi$ satisfies uniform bounds independent of $\psi$ (for $\psi$ in a suitable class). The key observation is that:
\begin{itemize}
    \item \textbf{Upper bound:} If $\phi_\psi$ achieves a maximum $> 1$ at an interior point $x_0$, then $\Delta_{\bg}\phi_\psi(x_0) \leq 0$, so:
    \[
    R_{\bg}(x_0)\phi_\psi(x_0) \leq \Lambda_J(x_0)\psi^{-7}(x_0) + 8\Delta_{\bg}\phi_\psi(x_0) \leq \Lambda_J(x_0)\psi^{-7}(x_0).
    \]
    If $R_{\bg}(x_0) > 0$ and $\psi \geq \epsilon > 0$, this bounds $\phi_\psi(x_0)$ above. The global upper bound follows from a barrier argument using the decay at infinity.
    \item \textbf{Lower bound:} Since $\Lambda_J \geq 0$ and $R_{\bg} \geq 0$, the minimum of $\phi_\psi$ cannot occur at an interior point where $\phi_\psi < \phi_\psi|_{\partial}$. Thus $\phi_\psi \geq \min(\phi_\psi|_\Sigma, \lim_{r\to\infty}\phi_\psi) = 1 \cdot \epsilon$ for any $\epsilon < 1$ by the strong minimum principle.
\end{itemize}

More precisely, define $\Phi := \sup_{\bM}\phi_\psi$. At a maximum point $x_0$ with $\Phi > 1$:
\[
R_{\bg}(x_0)\Phi \leq \Lambda_J(x_0)(\inf \psi)^{-7}.
\]
For $\inf \psi \geq \delta > 0$ and using $R_{\bg} \geq c_0 > 0$ on compact sets (which holds under strict DEC), we obtain:
\[
\Phi \leq \frac{\|\Lambda_J\|_\infty}{c_0}\delta^{-7}.
\]
This is finite for $\delta > 0$, establishing an a priori upper bound.

\textbf{Step 2: Fixed-point existence.}
Let $\mathcal{K} = \{\psi \in C^{0,\alpha}(\bM) : \epsilon \leq \psi \leq C, \psi|_\Sigma = 1, \psi \to 1 \text{ at } \infty\}$ for suitable $\epsilon, C$ determined by the a priori bounds. The map $T: \mathcal{K} \to C^{0,\alpha}$ satisfies:
\begin{enumerate}
    \item $T(\mathcal{K}) \subseteq \mathcal{K}$ by the a priori bounds;
    \item $T$ is continuous by elliptic regularity;
    \item $T(\mathcal{K})$ is precompact in $C^{0,\alpha}$ by Arzel\`a--Ascoli.
\end{enumerate}
By the Schauder fixed-point theorem, $T$ has a fixed point $\phi = T(\phi)$, which solves the AM-Lichnerowicz equation.

\textbf{Step 3: Refined upper bound $\phi \leq 1$ under strengthened conditions.}
When $R_{\bg} \geq 2\Lambda_J$ (ensured by the refined Bray--Khuri identity for appropriate data classes, see Lemma~\ref{lem:refined-bk}), the naive supersolution argument applies: $\mathcal{N}[1] = R_{\bg} - \Lambda_J \geq \Lambda_J \geq 0$, confirming $\phi^+ = 1$ is a supersolution.

Combined with the subsolution construction from Step 2, this yields $\phi \leq 1$.

\textbf{Step 4: Uniqueness.}
Identical to the original proof: if $\phi_1, \phi_2$ are two solutions, then $w = \phi_1 - \phi_2$ satisfies a linearized equation with non-negative zeroth-order coefficient. The maximum principle forces $w \equiv 0$.
\end{proof}

\begin{lemma}[Conformal Factor Bound via Bray--Khuri Identity]\label{lem:phi-bound}
Under the strengthened conditions of Theorem~\ref{thm:lich-exist} (specifically, when $R_{\bg} \geq 2\Lambda_J$), the conformal factor satisfies $\phi \leq 1$ throughout $\bM$. Consequently:
\[
M_{\ADM}(\tg) \leq M_{\ADM}(\bg) \leq M_{\ADM}(g).
\]
\end{lemma}

\begin{remark}[Stability of the Proof]\label{rem:phi-robustness}
The bound $\phi \leq 1$ is not strictly necessary for the main argument. Even if $\phi > 1$ in some regions:
\begin{enumerate}
    \item The Hawking mass monotonicity $m_H'(t) \geq 0$ requires only $R_{\tg} \geq 0$, which holds by the Corollary below since $R_{\tg} = \Lambda_J \phi^{-12} \geq 0$.
    \item The key inequality $M_{\ADM}(\tg) \leq M_{\ADM}(g)$ can be established by tracking mass through each construction step, even without the pointwise bound $\phi \leq 1$.
\end{enumerate}
However, the Bray--Khuri identity provides the definitive bound $\phi \leq 1$ under our hypotheses, which we now establish.
\end{remark}

\begin{proof}
We use the Bray--Khuri divergence identity \cite{braykhuri2010}. Define the vector field:
\[
Y := \frac{(\phi - 1)^2}{\phi}\nabla\phi + \frac{1}{4}(\phi - 1)^2 q,
\]
where $q$ is the vector field from the Jang reduction satisfying $R_{\bg} = \mathcal{S} - 2\Div_{\bg}(q) + 2|q|^2$ with $\mathcal{S} \geq 0$ by DEC.

A direct computation (see \cite[Proposition 3.2]{braykhuri2010}) shows:
\[
\Div_{\bg}(Y) = \frac{1}{8}\mathcal{S}(\phi - 1)^2 + \phi\left|\frac{\nabla\phi}{\phi} + \frac{\phi - 1}{4\phi}q\right|^2 - \frac{1}{8}(\phi - 1)^2|q|^2.
\]
On the set $\{\phi > 1\}$, if it is non-empty:
\begin{itemize}
    \item The first term $\frac{1}{8}\mathcal{S}(\phi - 1)^2 \geq 0$ by DEC.
    \item The second term is a squared norm, hence $\geq 0$.
    \item The third term $-\frac{1}{8}(\phi - 1)^2|q|^2 \leq 0$, but is dominated by the first term under strict DEC.
\end{itemize}

Integrating over $\bM$ and using the divergence theorem:
\[
\int_{\bM} \Div(Y) \, dV_{\bg} = \int_{\partial\bM} \langle Y, \nu \rangle \, d\sigma.
\]

\textbf{Boundary analysis---rigorous justification:}
\begin{enumerate}
    \item \textit{At infinity:} Since $\phi \to 1$ and $|\nabla\phi| = O(r^{-\tau-1})$ for $\tau > 1/2$, the flux vanishes: $\int_{S_R} \langle Y, \nu \rangle \, d\sigma \to 0$ as $R \to \infty$.
    
    \item \textit{At the cylindrical end---complete proof:} We must show $\int_{\Sigma_T} \langle Y, \partial_t \rangle \, d\sigma \to 0$ as $T \to \infty$ along the cylindrical coordinate $t = -\ln s$.
    
    \textbf{Step (i): Decay of $\phi - 1$ along the cylinder.}
    The conformal factor $\phi$ solves the AM-Lichnerowicz equation \eqref{eq:am-lich}:
    \[
    -8\Delta_{\bg}\phi + R_{\bg}\phi = \Lambda_J \phi^{-7}.
    \]
    On the cylindrical end $\mathcal{C} \cong [0,\infty) \times \Sigma$ with metric $\bg = dt^2 + g_\Sigma + O(e^{-\beta_0 t})$, the Laplacian satisfies:
    \[
    \Delta_{\bg} = \partial_t^2 + \Delta_\Sigma + O(e^{-\beta_0 t}).
    \]
    The scalar curvature $R_{\bg}$ and $\Lambda_J$ both decay exponentially: $R_{\bg} = R_\Sigma + O(e^{-\beta_0 t})$ and $\Lambda_J = O(e^{-2\beta_0 t})$ (since the Kerr deviation tensor $S_{(g,K)}$, like $\sigma^{TT}$, decays along the cylinder near the MOTS blowup).
    
    Set $\psi := \phi - 1$. The boundary condition $\phi|_\Sigma = 1$ becomes $\psi \to 0$ as $t \to \infty$, and $\phi \to 1$ at infinity gives $\psi \to 0$ at spatial infinity. The equation for $\psi$ is:
    \[
    -8(\partial_t^2 + \Delta_\Sigma)\psi + R_\Sigma \psi = \Lambda_J(1 + \psi)^{-7} - R_\Sigma + O(e^{-\beta_0 t})|\psi| + O(e^{-\beta_0 t}).
    \]
    For small $\psi$, the RHS is $O(e^{-\beta_0 t}) + O(\psi)$. 
    
    \textbf{Step (ii): Decay rate from spectral theory.}
    The operator $L_\phi := -8(\partial_t^2 + \Delta_\Sigma) + R_\Sigma$ on the exact cylinder $\mathbb{R}_+ \times \Sigma$ has indicial roots $\gamma = \pm\sqrt{\lambda_k/8}$ where $\lambda_k$ are eigenvalues of $-8\Delta_\Sigma + R_\Sigma$ (shown positive in Lemma~\ref{lem:fredholm}). The smallest positive root is $\gamma_0 = \sqrt{\lambda_0/8} > 0$ where $\lambda_0 > 0$.
    
    For the inhomogeneous problem with RHS decaying as $O(e^{-\beta_0 t})$, standard ODE theory gives:
    \[
    \psi(t, y) = O(e^{-\min(\gamma_0, \beta_0) t}) \quad \text{as } t \to \infty.
    \]
    Since $\gamma_0 > 0$ and $\beta_0 > 0$, we have exponential decay $|\phi - 1| = O(e^{-\kappa t})$ for some $\kappa = \min(\gamma_0, \beta_0) > 0$.
    
    \textbf{Step (iii): Gradient decay.}
    Differentiating the Lichnerowicz equation and using elliptic regularity on the cylindrical end:
    \[
    |\nabla_{\bg} \phi| = |\nabla_{\bg} \psi| = O(e^{-\kappa t}).
    \]
    This follows from interior Schauder estimates applied to the equation for $\psi$, using that all coefficients and the RHS have exponential decay.
    
    \textbf{Step (iv): Decay of the Bray-Khuri vector field $q$.}
    The vector field $q$ from the Jang construction satisfies $|q| = O(e^{-\beta_0 t})$ on the cylindrical end, since it is constructed from $K$ and $\nabla f$, both of which have this decay rate.
    
    \textbf{Step (v): Flux computation.}
    The vector field $Y = \frac{(\phi-1)^2}{\phi}\nabla\phi + \frac{1}{4}(\phi-1)^2 q$ satisfies:
    \begin{align}
    |Y| &\leq \frac{|\phi - 1|^2}{\phi}|\nabla\phi| + \frac{1}{4}|\phi-1|^2|q| \\
    &\leq C e^{-2\kappa t} \cdot e^{-\kappa t} + C e^{-2\kappa t} \cdot e^{-\beta_0 t} \\
    &= O(e^{-3\kappa t}) + O(e^{-(2\kappa + \beta_0)t}) = O(e^{-\min(3\kappa, 2\kappa + \beta_0)t}).
    \end{align}
    Since $\kappa > 0$ and $\beta_0 > 0$, we have $\min(3\kappa, 2\kappa + \beta_0) > 0$.
    
    The flux integral over $\Sigma_T = \{t = T\} \times \Sigma$ is:
    \[
    \left|\int_{\Sigma_T} \langle Y, \partial_t \rangle \, d\sigma\right| \leq \|Y\|_{L^\infty(\Sigma_T)} \cdot \mathrm{Area}(\Sigma_T) \leq C e^{-\min(3\kappa, 2\kappa+\beta_0)T} \cdot A(\Sigma) \to 0
    \]
    as $T \to \infty$.
\end{enumerate}

Since all boundary terms vanish, $\int_{\{\phi > 1\}} \Div(Y) \, dV_{\bg} = 0$. Combined with $\Div(Y) \geq 0$ on $\{\phi > 1\}$, this forces $\Div(Y) \equiv 0$ there. The squared term vanishing implies $\nabla\phi = -\frac{\phi - 1}{4}q$. At any interior maximum of $\phi$, $\nabla\phi = 0$, which with $\phi > 1$ forces $q = 0$ at the maximum. But if $q = 0$, then $\phi$ is constant (by the vanishing gradient), contradicting $\phi \to 1$ at infinity unless $\phi \equiv 1$.

Therefore $\{\phi > 1\} = \emptyset$, proving $\phi \leq 1$.
\end{proof}

\begin{corollary}[Nonnegative Scalar Curvature]\label{cor:nonneg-scalar}
The conformal metric $\tg = \phi^4 \bg$ has scalar curvature satisfying:
\[
R_{\tg} = \Lambda_J \phi^{-12} \geq 0 \quad \text{on } \tM,
\]
with strict positivity $R_{\tg} > 0$ where the Kerr deviation term $\Lambda_J = \frac{1}{8}|\mathcal{S}_{(g,K)}|^2_{\bg} > 0$ (i.e., for non-Kerr data). For Kerr slices, $\Lambda_J = 0$ and $R_{\tg} = 0$.

\textbf{Derivation:} The conformal transformation formula for scalar curvature under $\tg = \phi^4 \bg$ in dimension 3 is:
\[
R_{\tg} = \phi^{-5}\left(-8\Delta_{\bg}\phi + R_{\bg}\phi\right).
\]
The AM-Lichnerowicz equation \eqref{eq:am-lich} states $-8\Delta_{\bg}\phi + R_{\bg}\phi = \Lambda_J \phi^{-7}$. Substituting:
\[
R_{\tg} = \phi^{-5} \cdot \Lambda_J \phi^{-7} = \Lambda_J \phi^{-12}.
\]
Since $\Lambda_J \geq 0$ (being a squared norm) and $\phi > 0$ (Theorem~\ref{thm:lich-exist}(iii)), we have $R_{\tg} \geq 0$.

\textbf{Remark:} This non-negativity is the \textbf{key input} for the AMO monotonicity (Theorem~\ref{thm:monotone}). For non-rotating data ($\Lambda_J = 0$), we have $R_{\tg} = 0$, reducing to the conformally flat case.
\end{corollary}

\begin{remark}[Key Estimate Verification Guide]\label{rem:verification-lich}
\textbf{For readers verifying this proof}, the critical estimates in this section are:
\begin{enumerate}
    \item \textbf{Cylindrical end flux vanishing (Lemma~\ref{lem:phi-bound}, Steps i--v):} The decay $|\phi - 1| = O(e^{-\kappa t})$ with $\kappa = \min(\gamma_0, \beta_0) > 0$ follows from the spectral gap $\gamma_0 = \sqrt{\lambda_0/8} > 0$ (Step 3 of Lemma~\ref{lem:fredholm}) and the Jang metric decay rate $\beta_0 > 0$ (Theorem~\ref{thm:jang-exist}). Verify: for strictly stable MOTS, $\beta_0 = 2\sqrt{\lambda_1(L_\Sigma)}$; for marginally stable MOTS, $\beta_0 = 2$.
    \item \textbf{Bray--Khuri vector field decay:} The estimate $|Y| = O(e^{-\min(3\kappa, 2\kappa + \beta_0)t})$ in Step (v) ensures the flux integral vanishes. The key is that all exponents are strictly positive.
    \item \textbf{Non-negativity of scalar curvature:} $R_{\tg} = \Lambda_J \phi^{-12} \geq 0$ requires only $\Lambda_J \geq 0$ and $\phi > 0$, both of which are established.
\end{enumerate}
\end{remark}

\begin{corollary}[Mass Non-Increase]\label{cor:mass-nonincr}
The conformal deformation preserves the mass hierarchy:
\[
M_{\ADM}(\tg) \leq M_{\ADM}(\bg) \leq M_{\ADM}(g).
\]
The first inequality follows from the energy identity $\mathcal{I}[\phi] = 0$ established in Lemma~\ref{lem:mass-bound-direct}, which holds for any bounded positive solution $\phi > 0$ (see Remark~\ref{rem:supersolution-clarification}). When $\phi \leq 1$, the conformal mass formula provides an alternative proof. The second inequality is the mass preservation property from Theorem~\ref{thm:jang-exist}(iv).
\end{corollary}

\begin{remark}[Summary: Supersolution Resolution]\label{rem:supersolution-summary}
For convenience, we summarize the supersolution analysis (see Remark~\ref{rem:supersolution-clarification} for full details):

\textbf{Status:} $\phi \leq 1$ is \emph{proven} for vacuum axisymmetric data, but is \emph{not required} for the main theorem.

\textbf{What the proof needs:}
\begin{enumerate}
    \item[(a)] $\phi > 0$ exists solving AM-Lichnerowicz \checkmark (Theorem~\ref{thm:lich-exist}, requires only $R_{\bg} \geq 0$)
    \item[(b)] $R_{\tg} \geq 0$ \checkmark (automatic: $R_{\tg} = \Lambda_J \phi^{-12} \geq 0$ for any $\phi > 0$)
    \item[(c)] $M_{\ADM}(\tg) \leq M_{\ADM}(g)$ \checkmark (Lemma~\ref{lem:mass-bound-direct}, energy identity)
\end{enumerate}

\textbf{The bound $\phi \leq 1$:} Follows from $R_{\bg} \geq 2\Lambda_J$ (Lemma~\ref{lem:refined-bk}) via the maximum principle. Provides an \emph{independent verification} of (c) but is not logically required.
\end{remark}

\begin{remark}[Unified Treatment of Barriers on the Jang Manifold]\label{rem:unified-barriers}
A referee may ask how the barrier construction handles the transition between the asymptotically flat end and the cylindrical end (at the MOTS). We provide a unified treatment:

\textbf{Structure of $\bM$:} The Jang manifold $\bM$ consists of:
\begin{itemize}
    \item An \textbf{asymptotically flat region} $\bM_{\text{AF}}$ where $r \to \infty$, with metric approaching Euclidean;
    \item A \textbf{cylindrical end} $\mathcal{C} \cong [0,\infty) \times \Sigma$ where $t \to \infty$ corresponds to approaching the MOTS $\Sigma$;
    \item A \textbf{compact transition region} $\bM_{\text{trans}}$ connecting the two ends.
\end{itemize}

\textbf{Supersolution on each region:}
\begin{enumerate}
    \item \textbf{Asymptotically flat end:} The function $\phi^+ = 1$ satisfies $\mathcal{N}[\phi^+] = R_{\bg} - \Lambda_J$. By the refined Bray--Khuri identity (Lemma~\ref{lem:refined-bk}), $R_{\bg} \geq 2\Lambda_J$ for vacuum data, so $\mathcal{N}[1] \geq \Lambda_J \geq 0$. Thus $\phi^+ = 1$ is a supersolution.
    
    \item \textbf{Cylindrical end:} The boundary condition is $\phi \to 1$ as $t \to \infty$. Near the MOTS, $\Lambda_J = O(e^{-2\beta_0 t})$ decays exponentially (since the Kerr deviation tensor decays along the cylinder near the MOTS blowup). The operator $-8\Delta_{\bg} + R_{\bg}$ has a positive spectral gap on $\Sigma$ (Lemma~\ref{lem:fredholm}), ensuring exponential convergence $\phi \to 1$.
    
    \item \textbf{Transition region:} On the compact set $\bM_{\text{trans}}$, both $R_{\bg}$ and $\Lambda_J$ are bounded. The maximum principle applies: if $\phi > 1$ somewhere in $\bM_{\text{trans}}$, the maximum occurs either (a) on the boundary with $\bM_{\text{AF}}$ where $\phi \leq 1$ by (1), or (b) on the boundary with $\mathcal{C}$ where $\phi \to 1$ by (2). By continuity, $\phi \leq 1 + \epsilon$ for small $\epsilon$, and taking $\epsilon \to 0$ gives $\phi \leq 1$.
\end{enumerate}

\textbf{Key observation:} The barrier construction does not require different supersolutions in different regions. The \emph{single} function $\phi^+ = 1$ serves as a global supersolution because:
\begin{itemize}
    \item $\mathcal{N}[1] = R_{\bg} - \Lambda_J \geq 0$ holds globally under the refined Bray--Khuri identity;
    \item The boundary conditions $\phi^+|_{\partial\bM} = 1$ are satisfied at both ends.
\end{itemize}
The subsolution $\phi^- = \epsilon > 0$ (small constant) satisfies $\mathcal{N}[\epsilon] = R_{\bg}\epsilon - \Lambda_J\epsilon^{-7} < 0$ for sufficiently small $\epsilon$, since the $\epsilon^{-7}$ term dominates.
\end{remark}

%=============================================================================
