\section{Key AMO Estimates for Hawking Mass Monotonicity}\label{app:amo-estimates}
%=============================================================================

This appendix provides a self-contained summary of the key estimates from the Agostiniani--Mazzieri--Oronzio (AMO) framework \cite{amo2022} used in the monotonicity proof (Theorem~\ref{thm:monotone}). While the full theory is developed in \cite{amo2022}, we collect the essential bounds here for the reader's convenience.

\subsection{The \texorpdfstring{$p$}{p}-Harmonic Foliation}

\begin{definition}[$p$-Harmonic Potential]\label{def:p-harmonic-app}
Let $(\tM, \tg)$ be a complete Riemannian 3-manifold with boundary $\Sigma = \partial\tM$ and one asymptotically flat end. For $p \in (1, 2]$, the \textbf{$p$-harmonic potential} $u_p: \tM \to [0,1]$ is the solution to:
\begin{equation}\label{eq:p-harmonic-app}
\begin{cases}
\Div_{\tg}(|\nabla u_p|^{p-2}\nabla u_p) = 0 & \text{in } \tM \setminus \Sigma, \\
u_p|_\Sigma = 0, \\
u_p(x) \to 1 & \text{as } x \to \infty.
\end{cases}
\end{equation}
The level sets $\Sigma_t := \{u_p = t\}$ for regular values $t \in (0,1)$ define a foliation of $\tM$.
\end{definition}

\begin{proposition}[Existence and Regularity {\cite[Theorem~2.3]{amo2022}}]\label{prop:p-harmonic-exist-app}
Under the hypotheses of Theorem~\ref{thm:main}, the $p$-harmonic potential $u_p$ exists uniquely and satisfies:
\begin{enumerate}[label=\textup{(\roman*)}]
    \item $u_p \in C^{1,\Hoelder}_{\mathrm{loc}}(\tM)$ for $\Hoelder = \Hoelder(p) > 0$;
    \item $|\nabla u_p| > 0$ almost everywhere (no critical points in the interior);
    \item Level sets $\Sigma_t$ are $C^{1,\Hoelder}$ embedded surfaces for a.e.\ $t$;
    \item As $p \to 1^+$, the level sets converge to the weak IMCF foliation of Huisken--Ilmanen.
\end{enumerate}
\end{proposition}

\subsection{First Variation Formulas}

The following formulas govern the evolution of geometric quantities along the $p$-harmonic foliation.

\begin{proposition}[Area and Willmore Evolution {\cite[Proposition~3.2]{amo2022}}]\label{prop:area-willmore-app}
Let $A(t)$ and $W(t) = \frac{1}{16\pi}\int_{\Sigma_t} H^2\,d\sigma$ be the area and normalized Willmore functional. Then:
\begin{align}
A'(t) &= \int_{\Sigma_t} \frac{H}{|\nabla u_p|}\,d\sigma, \label{eq:area-deriv-app} \\
\frac{d}{dt}\int_{\Sigma_t} H^2\,d\sigma &= \int_{\Sigma_t} \frac{2H\mathcal{L}_\nu H + H^3 - 2H|\mathring{h}|^2}{|\nabla u_p|}\,d\sigma, \label{eq:willmore-deriv-app}
\end{align}
where $\nu = \nabla u_p/|\nabla u_p|$ is the unit normal and $\mathcal{L}_\nu H$ denotes the Lie derivative of mean curvature.
\end{proposition}

\subsection{The Key Hawking Mass Bound}

\begin{theorem}[Hawking Mass Monotonicity {\cite[Theorem~4.1]{amo2022}}]\label{thm:amo-hawking-app}
Let $(\tM, \tg)$ satisfy $R_{\tg} \geq 0$. The Hawking mass $m_H(t) = \sqrt{A(t)/(16\pi)}(1 - W(t))$ satisfies:
\begin{equation}\label{eq:amo-key-bound}
\frac{d}{dt}m_H^2 \geq \frac{1}{8\pi}\int_{\Sigma_t} \frac{R_{\tg} + 2|\mathring{h}|^2}{|\nabla u_p|}\,d\sigma \cdot (1 - W(t)).
\end{equation}
In particular, when $R_{\tg} \geq 0$:
\begin{enumerate}[label=\textup{(\roman*)}]
    \item $\frac{d}{dt}m_H^2 \geq 0$ (weak monotonicity);
    \item $\frac{d}{dt}m_H^2 \geq \frac{1}{8\pi}\int_{\Sigma_t} \frac{R_{\tg}}{|\nabla u_p|}\,d\sigma$ when $W(t) \leq 1/2$;
    \item $m_H(t) \to M_{\ADM}(\tg)$ as $t \to 1^-$.
\end{enumerate}
\end{theorem}

\begin{proof}[Proof sketch]
The derivation uses:
\begin{enumerate}
    \item The $p$-harmonic equation $\Div(|\nabla u|^{p-2}\nabla u) = 0$ to simplify variation formulas;
    \item The Gauss equation: $R_{\tg} = R_\Sigma + 2\Ric_{\tg}(\nu,\nu) - H^2 + |h|^2$;
    \item Gauss--Bonnet: $\int_{\Sigma_t} R_\Sigma\,d\sigma = 8\pi$ for $\Sigma_t \cong S^2$;
    \item Simon's identity relating $\mathcal{L}_\nu H$ to the traceless second fundamental form.
\end{enumerate}
Combining these with careful analysis of the boundary terms at $p \to 1^+$ yields \eqref{eq:amo-key-bound}. See \cite[Section~4]{amo2022} for the complete derivation.
\end{proof}

\subsection{Application to AM-Hawking Mass}

\begin{corollary}[AM-Hawking Bound Used in Main Proof]\label{cor:am-hawking-bound-app}
For the AM-Hawking mass $m_{H,J}^2 = m_H^2 + 4\pi J^2/A$, when $J$ is constant (Theorem~\ref{thm:J-conserve}) and $A(t) \geq 8\pi|J|$ (sub-extremality):
\begin{equation}\label{eq:am-hawking-bound-app}
\frac{d}{dt}m_{H,J}^2 = \frac{d}{dt}m_H^2 - \frac{4\pi J^2}{A^2}A' \geq \frac{1}{8\pi}\int_{\Sigma_t} \frac{R_{\tg} + 2|\mathring{h}|^2}{|\nabla u_p|}\left(1 - \frac{64\pi^2 J^2}{A^2}\right)d\sigma.
\end{equation}
\end{corollary}

\begin{proof}
Substitute \eqref{eq:amo-key-bound} and \eqref{eq:area-deriv-app} into the identity $\frac{d}{dt}m_{H,J}^2 = \frac{d}{dt}m_H^2 - \frac{4\pi J^2}{A^2}A'$. The sub-extremality factor $(1 - 64\pi^2 J^2/A^2) \geq 0$ arises from comparing the positive curvature term with the negative angular momentum term; see Steps 8a--8h of Theorem~\ref{thm:monotone} for the detailed calculation.
\end{proof}

\begin{remark}[Relationship to Inverse Mean Curvature Flow]
In the limit $p \to 1^+$, the $p$-harmonic foliation converges to the weak inverse mean curvature flow (IMCF) of Huisken--Ilmanen \cite{huisken2001}. The advantage of the $p$-harmonic approach is:
\begin{itemize}
    \item Avoids the ``jumping'' behavior of weak IMCF solutions;
    \item Provides $C^{1,\Hoelder}$ regularity of level sets for $p > 1$;
    \item Allows uniform estimates in the $p \to 1^+$ limit.
\end{itemize}
The monotonicity results hold for each $p \in (1,2]$, and the Moore--Osgood theorem ensures the double limit $(p \to 1^+, t \to 1^-)$ can be exchanged.
\end{remark}

%=============================================================================
