\section{Technical Foundations}\label{app:technical}
%=============================================================================

The analytical foundations of this paper build on established results in geometric analysis:

\begin{enumerate}
    \item \textbf{Twisted Jang Perturbation Theory}: The key observation (Theorem~\ref{thm:jang-exist}, Step 2) is that twist terms scale as $O(s)$ near the MOTS, making them asymptotically negligible compared to the principal curvature terms that diverge as $s^{-1}$. This perturbation structure is compatible with the Han--Khuri barrier construction \cite{hankhuri2013} and the Lockhart--McOwen Fredholm theory \cite{lockhartmccowen1985} used for cylindrical ends.
    
    \item \textbf{Conformal Factor Bounds}: The AM-Lichnerowicz equation (Theorem~\ref{thm:lich-exist}) is analyzed using the Bray--Khuri divergence identity (Lemma~\ref{lem:phi-bound}). The bound $\phi \leq 1$ follows from an integral argument that shows the boundary flux vanishes at both the asymptotic end and the cylindrical end, with explicit decay estimates from the weighted Sobolev framework.
    
    \item \textbf{$p \to 1$ Limit}: The AMO functional monotonicity (Theorem~\ref{thm:amo-mono}) is established for $p > 1$ using the Agostiniani--Mazzieri--Oronzio framework \cite{amo2022}. The sharp inequality emerges in the limit $p \to 1^+$ via Mosco convergence \cite{mosco1969}, which preserves the monotonicity in the distributional sense required for low-regularity metrics.
\end{enumerate}

%=============================================================================
\subsection{Critical Estimates}\label{subsec:critical-estimates-appendix}
%=============================================================================

\begin{mdframed}[linewidth=1.5pt, linecolor=red!70!black, backgroundcolor=red!5]
\textbf{Summary of Key Estimates.} The validity of the main theorem depends on two key technical estimates. We summarize them here.

\medskip
\textbf{Critical Estimate 1: Twist Decay (Theorem~\ref{thm:jang-exist}, Lemma~\ref{lem:twist-bound})}

\textit{Claim:} The twist perturbation term $\mathcal{T}[f]$ in the axisymmetric Jang equation satisfies
\[
|\mathcal{T}[f](x)| \leq C_\mathcal{T} \cdot s(x) \quad \text{as } s \to 0,
\]
where $s = \mathrm{dist}(\cdot, \Sigma)$ is the distance to the MOTS.

\textit{Importance:} This ensures twist is a lower-order perturbation compared to the principal terms ($O(s^{-1})$), allowing the Han--Khuri existence theory to extend to the rotating case.

\textit{Verification:} See Remark~\ref{rem:twist-verification-guide} for detailed checkpoints (V1)--(V4).

\textit{Fallback:} If the decay rate is weaker than $O(s)$, the perturbation argument requires modification. However, the $\rho^2$ factor in \eqref{eq:twist-term} provides significant cushion.

\medskip
\textbf{Critical Estimate 2: Curvature Bound (Lemma~\ref{lem:refined-bk})}

\textit{Claim:} For vacuum initial data, the Jang manifold scalar curvature satisfies
\[
R_{\bg} \geq 2\Lambda_J,
\]
where $\Lambda_J = \frac{1}{8}|\mathcal{S}_{(g,K)}|^2$ is the angular momentum source term.

\textit{Importance:} This bound ensures the conformal factor $\phi \leq 1$, which controls the mass inequality direction.

\textit{Verification:} See Appendix~\ref{app:supersolution} for a detailed proof using the Bray--Khuri identity.

\textit{Fallback:} Proposition~\ref{prop:alternative-mass} provides an \textbf{alternative proof path} that requires only the classical bound $R_{\bg} \geq 0$, bypassing the need for the refined estimate. This makes the main theorem robust against potential failures of the refined bound.

\medskip
\textbf{Summary of Logical Structure:}
\begin{center}
\begin{tikzpicture}[
    node distance=1.5cm and 2cm,
    box/.style={rectangle, draw, rounded corners, text width=3cm, text centered, minimum height=1cm},
    arrow/.style={->, thick}
]
\node[box] (twist) {Twist Decay\\$|\mathcal{T}| = O(s)$};
\node[box, right=of twist] (jang) {Jang Existence\\Thm~\ref{thm:jang-exist}};
\node[box, right=of jang] (lich) {AM-Lichnerowicz\\Thm~\ref{thm:lich-exist}};
\node[box, below=of lich] (mono) {Monotonicity\\Thm~\ref{thm:monotone}};
\node[box, left=of mono] (main) {Main Theorem\\Thm~\ref{thm:main}};
\node[box, below=of twist] (curv) {Curvature Bound\\$R_{\bg} \geq 2\Lambda_J$};
\node[box, right=of curv] (mass) {Mass Inequality\\$M(\tg) \leq M(g)$};

\draw[arrow] (twist) -- (jang);
\draw[arrow] (jang) -- (lich);
\draw[arrow] (lich) -- (mono);
\draw[arrow] (mono) -- (main);
\draw[arrow] (curv) -- (mass);
\draw[arrow] (mass) -- (main);
\draw[arrow, dashed, color=green!60!black] (jang) -- node[left, font=\scriptsize] {Alt.} (mass);
\end{tikzpicture}
\end{center}

The dashed green arrow indicates the alternative proof path (Proposition~\ref{prop:alternative-mass}) that bypasses the refined curvature bound. Both paths lead to the main theorem.
\end{mdframed}

%=============================================================================
\subsection*{Notation Conventions}
\label{subsec:notation-conventions}

\begin{remark}[Notation Disambiguation]\label{rem:notation-disambiguation}
To avoid potential confusion, we distinguish the following uses of related symbols:
\begin{itemize}
    \item $\beta \in (0,1)$: H\"older exponent in function spaces $C^{k,\beta}$ and $C^{k,\beta}_{-\tau}$. This is a regularity parameter. We use $\beta$ (rather than the traditional $\alpha$) to avoid confusion with the Komar form.
    \item $\alpha_J = \frac{1}{8\pi}K(\eta, \cdot)^\flat_g$: The \textbf{Komar 1-form} encoding angular momentum. This is a geometric object defined from the initial data.
\end{itemize}
This convention eliminates any potential ambiguity between regularity exponents and angular momentum-related quantities.
\end{remark}

%=============================================================================
\section*{Glossary of Symbols}
\label{sec:glossary}
%=============================================================================

\begin{small}
\begin{tabular}{@{}lp{10cm}@{}}
\toprule
\textbf{Symbol} & \textbf{Description} \\
\midrule
\multicolumn{2}{@{}l}{\textbf{Abbreviations}} \\
ADM & Arnowitt--Deser--Misner (mass, momentum, angular momentum) \\
DEC & Dominant Energy Condition: $\mu \geq |\momdens|$ \\
MOTS & Marginally Outer Trapped Surface: $\theta^+ = 0$ \\
AMO & Agostiniani--Mazzieri--Oronzio (monotonicity theory) \\
\midrule
\multicolumn{2}{@{}l}{\textbf{Initial Data}} \\
$(M, g, K)$ & Initial data: 3-manifold $M$, Riemannian metric $g$, extrinsic curvature $K$ \\
$M_{\mathrm{ext}}$ & Exterior region: connected component of $M \setminus \Sigma$ containing infinity \\
$M_{\ADM}$ & ADM mass of initial data \\
$J$ & Komar angular momentum (scalar, roman) \\
$\momdens$ & Momentum density vector field from constraint equations (boldface) \\
$\mu$ & Energy density: $\mu = \frac{1}{2}(R_g + (\tr K)^2 - |K|^2)$ \\
$\Sigma$ & Outermost stable MOTS (marginally outer trapped surface) \\
$A$ & Area of $\Sigma$ \\
$\eta = \partial_\phi$ & Axial Killing field \\
$\rho = |\eta|$ & Orbit radius of axial symmetry \\
$\omega$ & Twist 1-form encoding frame-dragging \\
\midrule
\multicolumn{2}{@{}l}{\textbf{Jang--Lichnerowicz Construction}} \\
$(\bM, \bg)$ & Jang manifold with induced metric $\bg = g + df \otimes df$ \\
$f$ & Jang function solving $H_{\Gamma(f)} = \tr_{\Gamma(f)} K$ \\
$(\tM, \tg)$ & Conformal manifold with $\tg = \phi^4 \bg$ \\
$\phi$ & Conformal factor from AM-Lichnerowicz equation \\
$\Lambda_J$ & Angular momentum source term: $\Lambda_J = \frac{1}{8}|S_{(g,K)}|^2$ (Kerr deviation tensor) \\
\midrule
\multicolumn{2}{@{}l}{\textbf{AMO Flow}} \\
$u_p$ & $p$-harmonic potential on $(\tM, \tg)$, satisfying $\Delta_p u_p = 0$ \\
$\Sigma_t = \{u = t\}$ & Level sets of $p$-harmonic potential (defined using $\tg$) \\
$A(t) = |\Sigma_t|_{\tg}$ & Area of level set (measured in $\tg$) \\
$J(t) = J(\Sigma_t)$ & Angular momentum on level set (constant by Theorem~\ref{thm:J-conserve}) \\
$m_H(t)$ & Hawking mass: $\sqrt{A(t)/(16\pi)}(1 - \frac{1}{16\pi}\int_{\Sigma_t} H^2)$ \\
$m_{H,J}(t)$ & AM-Hawking mass: $\sqrt{m_H^2(t) + 4\pi J^2/A(t)}$ \\
$\alpha_J$ & Komar 1-form: $\alpha_J = \frac{1}{8\pi}K(\eta, \cdot)^\flat_g$ \\
\midrule
\multicolumn{2}{@{}l}{\textbf{Function Spaces}} \\
$C^{k,\Hoelder}_{-\tau}$ & Weighted H\"older space with decay $r^{-\tau}$; $\Hoelder \in (0,1)$ is the H\"older exponent \\
$W^{k,2}_\beta$ & Weighted Sobolev space with weight $e^{\beta t}$ \\
$L_\Sigma$ & MOTS stability operator \\
$\lambda_1(L_\Sigma)$ & Principal eigenvalue of stability operator \\
\bottomrule
\end{tabular}
\end{small}

%=============================================================================
\subsection{Boundary Terms on Cylindrical Ends}\label{subsec:boundary-terms-consolidated}
%=============================================================================

\begin{mdframed}[linewidth=1.5pt, linecolor=blue!70!black, backgroundcolor=blue!5]
\textbf{Consolidated Treatment of Boundary/Cylindrical End Terms.}

This subsection provides a \textbf{single, self-contained reference} for the treatment of boundary terms that arise throughout the proof. The Jang manifold $(\bM, \bg)$ is noncompact with two ``ends'':
\begin{enumerate}[label=(\roman*)]
    \item The \textbf{asymptotically flat end} (at spatial infinity, $r \to \infty$);
    \item The \textbf{cylindrical end} (near the MOTS $\Sigma$, coordinate $t \to \infty$ in $\mathcal{C} \cong [0,\infty) \times \Sigma$).
\end{enumerate}
Integration-by-parts identities on $\bM$ produce boundary terms at both ends, which must be controlled.
\end{mdframed}

\textbf{(A) Asymptotically Flat End ($r \to \infty$).}

At spatial infinity, boundary terms vanish due to the decay conditions in Definition~\ref{def:AF}:
\begin{itemize}
    \item \textbf{Metric decay:} $g_{ij} = \delta_{ij} + O(r^{-\tau})$ with $\tau > 1/2$.
    \item \textbf{Conformal factor:} $\phi = 1 + O(r^{-\tau})$ (Theorem~\ref{thm:lich-exist}).
    \item \textbf{Typical boundary integral:} For spheres $S_r$ at radius $r$, integrals of the form $\int_{S_r} \partial_\nu\psi \, dA$ with $\psi = O(r^{-\tau})$ satisfy:
    \[
    \int_{S_r} \partial_\nu\psi \, dA = O(r^{2-\tau-1}) = O(r^{1-\tau}) \to 0 \quad \text{as } r \to \infty \text{ for } \tau > 1.
    \]
    For $\tau \in (1/2, 1)$, more refined cancellation arguments using the constraint equations are needed; see \cite[Proposition 4.1]{bartnik1986}.
\end{itemize}

\textbf{(B) Cylindrical End ($t \to \infty$).}

The cylindrical end $\mathcal{C} \cong [0,\infty)_t \times \Sigma$ requires careful treatment:

\begin{lemma}[Cylindrical Boundary Term Vanishing]\label{lem:cylindrical-boundary-vanishing}
Let $\psi \in W^{2,2}_\beta(\bM)$ with $\beta < 0$, so that $\psi$ and its derivatives decay exponentially as $t \to \infty$ on the cylindrical end. Then for any integration-by-parts identity on $\bM$, the boundary contribution from the cylindrical end vanishes:
\[
\lim_{T \to \infty} \int_{\{t = T\} \times \Sigma} (\text{boundary flux}) = 0.
\]
\end{lemma}

\begin{proof}
Let $\Sigma_T := \{t = T\} \times \Sigma$ be the cylindrical cross-section at height $T$. For $\psi \in W^{2,2}_\beta$ with $\beta < 0$:
\begin{align*}
|\psi|_{\Sigma_T} &= O(e^{\beta T}), \\
|\nabla\psi|_{\Sigma_T} &= O(e^{\beta T}), \\
\text{Area}(\Sigma_T) &= A(\Sigma)(1 + O(e^{-\beta_0 T})) \sim \text{const}.
\end{align*}
A typical boundary flux integral is:
\[
\int_{\Sigma_T} \psi \cdot \partial_\nu\psi \, dA = O(e^{\beta T}) \cdot O(e^{\beta T}) \cdot O(1) = O(e^{2\beta T}) \to 0.
\]
For the conformal factor $\phi$ with $\phi - 1 = O(e^{-\kappa t})$ (Lemma~\ref{lem:phi-bound}(ii)):
\[
\int_{\Sigma_T} (\phi - 1) \partial_t\phi \, dA = O(e^{-\kappa T}) \cdot O(e^{-\kappa T}) \cdot O(1) = O(e^{-2\kappa T}) \to 0.
\]
\end{proof}

\textbf{(C) Application to the Conformal Mass Formula.}

The conformal mass formula (Proposition~\ref{prop:mass-bound-energy} and Proposition~\ref{prop:alternative-mass}) involves:
\[
M_{\ADM}(\tg) - M_{\ADM}(\bg) = -\frac{1}{2\pi}\lim_{r \to \infty} \int_{S_r} \partial_\nu(\phi^2 - 1) \, dA + \lim_{T \to \infty}(\text{cylindrical term}).
\]

\textbf{Asymptotic end:} With $\phi = 1 + O(r^{-\tau})$:
\[
\partial_\nu(\phi^2 - 1) = 2\phi\partial_\nu\phi = 2(1 + O(r^{-\tau})) \cdot O(r^{-\tau-1}) = O(r^{-\tau-1}).
\]
The surface integral is $O(r^{2-\tau-1}) = O(r^{1-\tau}) \to 0$ for $\tau > 1$.

\textbf{Cylindrical end:} By Lemma~\ref{lem:cylindrical-boundary-vanishing}:
\[
\lim_{T \to \infty} \int_{\Sigma_T} \partial_t(\phi^2 - 1) \, dA = \lim_{T \to \infty} O(e^{-2\kappa T}) \cdot O(1) = 0.
\]

\textbf{Conclusion:} The boundary terms vanish at both ends, validating the energy identity in Proposition~\ref{prop:alternative-mass}.

\textbf{(D) Application to the Monotonicity Formula.}

The AMO monotonicity (Theorem~\ref{thm:monotone}) involves integrals over level sets $\Sigma_t = \{u = t\}$ for $t \in (0,1)$. The boundary contributions at $t = 0$ (MOTS) and $t = 1$ (infinity) are:

\textbf{At $t = 0$ (MOTS):} The level sets $\Sigma_t$ approach $\Sigma$ as $t \to 0^+$ in the Hausdorff topology. By Lemma~\ref{lem:mots-boundary}, $\Sigma$ is minimal in $(\tM, \tg)$, so:
\[
m_{H,J}(0) = \sqrt{\frac{A}{16\pi} + \frac{4\pi J^2}{A}} \quad \text{(exact equality)}.
\]

\textbf{At $t = 1$ (infinity):} By Lemma~\ref{lem:adm-convergence}:
\[
m_{H,J}(1) = M_{\ADM}(\tg).
\]

The monotonicity $m_{H,J}(1) \geq m_{H,J}(0)$ follows from the non-negativity of the integrand in Theorem~\ref{thm:monotone}.

\textbf{(E) Sign Verification for Each Boundary Term.}

We summarize the sign of each boundary contribution:

\begin{center}
\begin{tabular}{@{}llcc@{}}
\toprule
\textbf{Identity} & \textbf{Boundary} & \textbf{Value/Sign} & \textbf{Reference} \\
\midrule
Conformal mass & $r \to \infty$ & $0$ (decay) & Prop.~\ref{prop:alternative-mass} \\
Conformal mass & Cylindrical & $0$ (exp.\ decay) & Lemma~\ref{lem:cylindrical-boundary-vanishing} \\
$m_{H,J}$ at MOTS & $t = 0$ & $\sqrt{A/16\pi + 4\pi J^2/A}$ & Lemma~\ref{lem:mots-boundary} \\
$m_{H,J}$ at $\infty$ & $t = 1$ & $M_{\ADM}(\tg)$ & Lemma~\ref{lem:adm-convergence} \\
\bottomrule
\end{tabular}
\end{center}

All boundary terms are either zero (vanishing) or have the correct value for the inequality chain to close.

%=============================================================================
