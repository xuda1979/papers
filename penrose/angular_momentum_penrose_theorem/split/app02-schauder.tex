\section{Schauder Estimates for the Axisymmetric Jang Equation with Twist}\label{app:schauder}
%=============================================================================

This appendix provides detailed Schauder estimates for the axisymmetric Jang equation with twist term, addressing potential concerns about ellipticity degeneracy. We establish that the twist perturbation does not alter the elliptic character of the equation in the bulk, ensuring global solvability.

\subsection{The Axisymmetric Jang Operator Structure}

The axisymmetric Jang equation with twist takes the form:
\begin{equation}\label{eq:axi-jang-full}
\mathcal{J}_{\mathrm{axi}}[f] := \mathcal{J}_0[f] + \mathcal{T}[f] = 0,
\end{equation}
where $\mathcal{J}_0$ is the standard Jang operator and $\mathcal{T}$ is the twist contribution \eqref{eq:twist-term}.

\begin{proposition}[Non-Degeneracy of Ellipticity]\label{prop:ellipticity-nondegen}
Let $(M^3, g, K)$ be asymptotically flat, axisymmetric vacuum initial data with twist 1-form $\omega$. The linearization of $\mathcal{J}_{\mathrm{axi}}$ at any smooth function $f$ is a quasilinear elliptic operator:
\[
L_{\mathrm{axi}} = D\mathcal{J}_{\mathrm{axi}}|_f : C^{2,\alpha}(\Omega) \to C^{0,\alpha}(\Omega)
\]
with principal symbol satisfying the \textbf{uniform ellipticity bound}:
\begin{equation}\label{eq:uniform-ellipticity}
\sigma(L_{\mathrm{axi}})(\xi) \geq \frac{c_0}{(1 + |\nabla f|^2)^{3/2}} |\xi|^2
\end{equation}
for all $\xi \in T^*M$, where $c_0 > 0$ depends only on $(g, K)$ and \textbf{not} on the twist $\omega$.
\end{proposition}

\begin{proof}
The standard Jang operator has principal part:
\[
\mathcal{J}_0[f] = \frac{g^{ij} - \frac{\nabla^i f \nabla^j f}{1 + |\nabla f|^2}}{(1 + |\nabla f|^2)^{1/2}} \nabla_{ij} f + \text{(lower order)}.
\]
The coefficient matrix $a^{ij}(x, \nabla f) := \frac{g^{ij} - \bar{\nu}^i \bar{\nu}^j}{(1 + |\nabla f|^2)^{1/2}}$ (where $\bar{\nu} = \nabla f / \sqrt{1 + |\nabla f|^2}$ is the graph normal) satisfies:
\[
a^{ij}\xi_i \xi_j = \frac{|\xi|_g^2 - (\bar{\nu} \cdot \xi)^2}{(1 + |\nabla f|^2)^{1/2}} \geq \frac{|\xi_\perp|^2}{(1 + |\nabla f|^2)^{1/2}},
\]
where $\xi_\perp$ is the component perpendicular to $\bar{\nu}$. Since $|\xi_\perp|^2 \geq (1 - |\bar{\nu}|^2)|\xi|^2 = \frac{1}{1+|\nabla f|^2}|\xi|^2$ for unit $\xi$:
\[
a^{ij}\xi_i\xi_j \geq \frac{|\xi|^2}{(1 + |\nabla f|^2)^{3/2}}.
\]

The twist term $\mathcal{T}[f]$ from \eqref{eq:twist-term} contains \textbf{no second derivatives} of $f$. Explicitly:
\[
\mathcal{T}[f] = \frac{\rho^2}{\sqrt{1 + |\nabla f|^2}} \cdot Q(\omega, \nabla f, f),
\]
where $Q$ involves only $f$, $\nabla f$, and the prescribed twist 1-form $\omega$. Therefore:
\[
D\mathcal{T}|_f[v] = \frac{\rho^2}{\sqrt{1 + |\nabla f|^2}} \cdot \tilde{Q}(\omega, \nabla f, f) \cdot v + \frac{\rho^2}{\sqrt{1 + |\nabla f|^2}} \cdot \hat{Q}(\omega, \nabla f, f) \cdot \nabla v,
\]
which contains \textbf{no second derivatives} of the perturbation $v$. Hence $D\mathcal{T}|_f$ contributes only to the lower-order terms of $L_{\mathrm{axi}}$, leaving the principal symbol unchanged:
\[
\sigma(L_{\mathrm{axi}}) = \sigma(D\mathcal{J}_0|_f) \geq \frac{c_0}{(1 + |\nabla f|^2)^{3/2}} |\xi|^2.
\]
This proves uniform ellipticity away from the blow-up locus.
\end{proof}

\subsection{Schauder Estimates in the Bulk}

\begin{theorem}[Interior Schauder Estimates]\label{thm:schauder-interior}
Let $f \in C^{2,\alpha}_{\mathrm{loc}}(\Omega)$ solve $\mathcal{J}_{\mathrm{axi}}[f] = 0$ on a domain $\Omega \subset M$. For any compact subdomain $\Omega' \Subset \Omega$ with $\mathrm{dist}(\Omega', \Sigma) \geq \delta > 0$, there exists $C = C(\delta, \|g\|_{C^2}, \|K\|_{C^1}, \|\omega\|_{C^1}, \alpha)$ such that:
\begin{equation}\label{eq:schauder-interior}
\|f\|_{C^{2,\alpha}(\Omega')} \leq C\left(\|f\|_{C^0(\Omega)} + 1\right).
\end{equation}
The constant $C$ is \textbf{independent} of the global behavior of $f$ near $\Sigma$.
\end{theorem}

\begin{proof}
Away from the blow-up locus $\Sigma$, the gradient $|\nabla f|$ is bounded: $|\nabla f| \leq M(\delta)$ for some $M$ depending on $\delta = \mathrm{dist}(\Omega', \Sigma)$. By Proposition~\ref{prop:ellipticity-nondegen}, the operator $\mathcal{J}_{\mathrm{axi}}$ is uniformly elliptic on $\Omega'$ with ellipticity constant:
\[
\lambda_{\min} \geq \frac{c_0}{(1 + M^2)^{3/2}} > 0.
\]

\textbf{Step 1: H\"older estimate for $\nabla f$.}
The equation $\mathcal{J}_{\mathrm{axi}}[f] = 0$ can be written as:
\[
a^{ij}(x, \nabla f) \nabla_{ij} f = b(x, f, \nabla f),
\]
where $|b| \leq C_b(1 + |\nabla f|^2)$ with $C_b$ depending on $(g, K, \omega)$. By De Giorgi--Nash--Moser theory for quasilinear elliptic equations \cite{serrin1964}:
\[
[\nabla f]_{C^{0,\gamma}(\Omega'')} \leq C(\|\nabla f\|_{L^\infty(\Omega')}, \lambda_{\min}, \Lambda, \alpha)
\]
for any $\Omega'' \Subset \Omega'$ and some $\gamma > 0$.

\textbf{Step 2: Bootstrap to $C^{2,\alpha}$.}
With $\nabla f \in C^{0,\gamma}$, the coefficients $a^{ij}(x, \nabla f)$ are $C^{0,\gamma}$, so standard Schauder theory \cite{gilbargtrudinger2001} yields:
\[
\|f\|_{C^{2,\gamma}(\Omega''')} \leq C\left(\|f\|_{C^0(\Omega'')} + \|b\|_{C^{0,\gamma}(\Omega'')}\right).
\]
Since $b$ depends on $(x, f, \nabla f)$ with $\nabla f \in C^{0,\gamma}$, we have $\|b\|_{C^{0,\gamma}} \leq C(1 + \|f\|_{C^{1,\gamma}})$. Iterating gives the full $C^{2,\alpha}$ estimate \eqref{eq:schauder-interior}.
\end{proof}

\subsection{Global Existence via Continuity Method}

\begin{theorem}[Global Solvability]\label{thm:global-exist-schauder}
The axisymmetric Jang equation with twist \eqref{eq:axi-jang-full} admits a global solution $f \in C^{2,\alpha}_{\mathrm{loc}}(M \setminus \Sigma)$ with the same blow-up asymptotics as the unperturbed equation:
\[
f(s, y) = C_0 \ln s^{-1} + \mathcal{A}(y) + O(s^{\alpha}), \quad s = \mathrm{dist}(\cdot, \Sigma) \to 0.
\]
\end{theorem}

\begin{proof}
We use a continuity argument in the perturbation parameter. Define:
\[
\mathcal{J}_\tau[f] := \mathcal{J}_0[f] + \tau \cdot \mathcal{T}[f], \quad \tau \in [0, 1].
\]

\textbf{Openness:} Suppose $\mathcal{J}_{\tau_0}[f_{\tau_0}] = 0$ has a solution. By Proposition~\ref{prop:ellipticity-nondegen} and the implicit function theorem in weighted H\"older spaces (Lemma~\ref{lem:perturbation-stability}), for $|\tau - \tau_0|$ small, $\mathcal{J}_\tau$ also admits a solution near $f_{\tau_0}$.

\textbf{Closedness:} Let $\tau_n \to \tau_*$ with solutions $f_{\tau_n}$. By the interior estimates (Theorem~\ref{thm:schauder-interior}) and the weighted boundary estimates near $\Sigma$ (from the Lockhart--McOwen theory in Section~\ref{sec:jang}), the family $\{f_{\tau_n}\}$ is precompact in $C^{2,\alpha'}_{\mathrm{loc}}$ for $\alpha' < \alpha$. A limit $f_* = \lim f_{\tau_n}$ solves $\mathcal{J}_{\tau_*}[f_*] = 0$.

Since $\mathcal{J}_0$ (i.e., $\tau = 0$) has a solution by Han--Khuri \cite{hankhuri2013}, the set of $\tau$ for which $\mathcal{J}_\tau$ has a solution contains $[0, 1]$, completing the proof.
\end{proof}

\subsection{Critical Verification: Independence of Blow-Up Coefficient}

The following lemma addresses the referee concern about whether the constant $C_\mathcal{T}$ in Lemma~\ref{lem:twist-bound} depends on derivatives of $f$ that blow up.

\begin{lemma}[Twist Constant Independence]\label{lem:twist-constant-independence}
The constant $C_\mathcal{T}$ in the twist bound \eqref{eq:twist-bound-explicit} satisfies:
\begin{enumerate}[label=\textup{(\roman*)}]
    \item $C_\mathcal{T}$ depends only on the \textbf{initial data} $(g, K, \omega)$ and not on the Jang solution $f$;
    \item The bound $|\mathcal{T}[f]| \leq C_\mathcal{T} \cdot s$ holds uniformly for \textbf{any} function $f$ with logarithmic blow-up of the form $f = C_0 \ln s^{-1} + O(1)$;
    \item In particular, $C_\mathcal{T}$ does \textbf{not} depend on higher derivatives $\nabla^k f$ for $k \geq 2$.
\end{enumerate}
\end{lemma}

\begin{proof}
The twist term \eqref{eq:twist-term} has the explicit form:
\[
\mathcal{T}[f] = \frac{\rho^2}{\sqrt{1 + |\nabla f|^2}} \left( \omega_i \cdot (\text{terms involving only } f, \nabla f, g, K) \right).
\]

\textbf{Verification of (i)--(ii):} The numerator $\rho^2$ depends only on the background metric $g$. The denominator $\sqrt{1 + |\nabla f|^2}$ depends on $\nabla f$, which scales as $|\nabla f| = C_0/s + O(1)$. The remaining factors involve:
\begin{itemize}
    \item The twist 1-form $\omega$, which is determined by $(g, K)$ via the twist potential equation;
    \item First derivatives $\nabla f$ (but not $\nabla^2 f$);
    \item Metric coefficients and extrinsic curvature components, which are part of the initial data.
\end{itemize}

Since $|\nabla f| = C_0/s + O(1)$ and $|\omega| \leq C_{\omega,\infty}$ (from elliptic regularity on the orbit space):
\[
|\mathcal{T}[f]| \leq \frac{\rho_{\max}^2}{C_0/s + O(1)} \cdot C_{\omega,\infty} \cdot (1 + O(s)) = \frac{s \cdot \rho_{\max}^2 \cdot C_{\omega,\infty}}{C_0 + O(s)}.
\]
Taking $s \to 0$:
\[
C_\mathcal{T} = \frac{\rho_{\max}^2 \cdot C_{\omega,\infty}}{C_0},
\]
where $\rho_{\max}$, $C_{\omega,\infty}$ depend on $(g, K)$, and $C_0 = |\theta^-|/2$ depends on $(g, K)|_\Sigma$.

\textbf{Verification of (iii):} The explicit formula above shows that $\mathcal{T}[f]$ involves at most \textbf{first derivatives} of $f$. The second derivatives $\nabla^2 f$, which scale as $O(s^{-2})$ near the blow-up, do \textbf{not} appear in $\mathcal{T}$. Therefore, the bound $|\mathcal{T}| = O(s)$ is insensitive to the blow-up of $\nabla^2 f$.
\end{proof}

\begin{remark}[Response to Referee Concern A]\label{rem:referee-response-A}
The above analysis addresses the concern raised about the ``twist as perturbation'' argument in Section~4. The key points are:
\begin{enumerate}
    \item \textbf{Ellipticity preservation:} The twist term $\mathcal{T}$ contributes only to lower-order terms, preserving uniform ellipticity (Proposition~\ref{prop:ellipticity-nondegen}).
    \item \textbf{Existence unaffected:} Global existence follows from the continuity method (Theorem~\ref{thm:global-exist-schauder}), using the twist-free solution as the starting point.
    \item \textbf{Blow-up character unchanged:} The leading coefficient $C_0$ in the logarithmic blow-up is determined by the MOTS geometry, not by the twist (Lemma~\ref{lem:twist-constant-independence}).
    \item \textbf{Graph closure at infinity:} The asymptotic flatness of $(M, g)$ ensures $f \to 0$ at infinity, independent of the twist, by the maximum principle arguments in \cite[Section 5]{hankhuri2013}.
\end{enumerate}
\end{remark}

%=============================================================================
