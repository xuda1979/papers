\section{The Super-Solution Condition and Mass Inequalities}\label{app:supersolution}
%=============================================================================

\begin{remark}[Clarification on the Super-Solution Assumption]\label{rem:supersolution-clarification}
Some readers may question whether the maximum principle bound $\phi \leq 1$ from Theorem~\ref{thm:lich-exist} is essential for the main result. This appendix demonstrates that the bound $\phi \leq 1$ is \textbf{not required} for the mass inequality---the proof can be completed using an energy identity that bypasses the super-solution condition entirely.
\end{remark}

This appendix provides a complete treatment of the super-solution issue raised in Remark~\ref{rem:supersolution-clarification}, demonstrating that the bound $\phi \leq 1$ is \textbf{not required} for the main theorem.

\subsection{The Mass Chain Without \texorpdfstring{$\phi \leq 1$}{phi <= 1}}

The classical conformal approach uses $\phi \leq 1$ to establish $M_{\ADM}(\tg) \leq M_{\ADM}(g)$. We show this bound holds without assuming $\phi \leq 1$.

\begin{proposition}[Mass Bound via Energy Identity]\label{prop:mass-bound-energy}\label{lem:mass-bound-direct}
Let $\phi > 0$ solve the AM-Lichnerowicz equation \eqref{eq:am-lich} with $\phi|_\Sigma = 1$ and $\phi \to 1$ at infinity. Then:
\[
M_{\ADM}(\tg) \leq M_{\ADM}(\bg) \leq M_{\ADM}(g),
\]
regardless of whether $\phi \leq 1$ or $\phi > 1$ in intermediate regions.
\end{proposition}

\begin{proof}
\textbf{Step 1: Second inequality.}
The bound $M_{\ADM}(\bg) \leq M_{\ADM}(g)$ is the Han--Khuri mass bound \cite[Theorem 3.1]{hankhuri2013}, independent of the conformal factor.

\textbf{Step 2: First inequality via the energy identity.}
Define $\psi := \phi - 1$, so $\psi|_\Sigma = 0$ and $\psi \to 0$ at infinity. The AM-Lichnerowicz equation gives:
\[
-8\Delta_{\bg}\psi + R_{\bg}\psi = \Lambda_J \phi^{-7} - R_{\bg}(1) + 8\Delta_{\bg}(1) = \Lambda_J \phi^{-7} - R_{\bg}.
\]
Multiply by $\psi$ and integrate over $\bM$:
\[
8\int_{\bM} |\nabla\psi|^2 \, dV_{\bg} + \int_{\bM} R_{\bg} \psi^2 \, dV_{\bg} = \int_{\bM} (\Lambda_J \phi^{-7} - R_{\bg}) \psi \, dV_{\bg}.
\]

\textbf{Step 3: Sign analysis.}
The LHS is:
\[
8\int |\nabla\psi|^2 + \int R_{\bg} \psi^2 \geq 8\int |\nabla\psi|^2 \geq 0
\]
(using $R_{\bg} \geq 0$ from DEC via Bray--Khuri).

\begin{mdframed}[linewidth=1pt, linecolor=red!60!black, backgroundcolor=red!3]
\begin{lemma}[Refined Bray--Khuri Scalar Curvature Bound]\label{lem:refined-bk}
For vacuum initial data satisfying the dominant energy condition, the Jang manifold scalar curvature satisfies:
\[
R_{\bg} \geq 2\Lambda_J,
\]
where $\Lambda_J = \frac{1}{8}|\mathcal{S}_{(g,K)}|^2_{\bg}$ is the angular momentum source term.
\end{lemma}

\textbf{Status:} This lemma represents a \textbf{key technical estimate} that requires independent verification. We provide a detailed heuristic derivation below, but note that:
\begin{enumerate}[label=(\roman*)]
\item \textbf{This estimate is NOT required for Theorem~\ref{thm:main}.} The alternative proof in Proposition~\ref{prop:alternative-mass} establishes the mass inequality using only $R_{\bg} \geq 0$.
\item \textbf{The constant $2$ in the inequality requires careful justification.} The proof below tracks all geometric factors, but independent confirmation is valuable.
\item \textbf{This estimate is of independent geometric interest.} If valid, it provides a refined understanding of how angular momentum affects the Jang manifold curvature.
\end{enumerate}

\begin{proof}[Heuristic Derivation of Lemma~\ref{lem:refined-bk}]
The proof proceeds in four steps, establishing the relationship between the Jang manifold scalar curvature and the Kerr deviation tensor. \textbf{This derivation should be viewed as a detailed heuristic rather than a complete rigorous proof.}

\textbf{Step 1: The Bray--Khuri identity.}
The classical Bray--Khuri identity \cite[Theorem 2.1]{braykhuri2010} relates the scalar curvature of the Jang manifold $(\bM, \bg)$ to the constraint quantities of the initial data $(M, g, K)$. For the Jang metric $\bg = g + df \otimes df$, the scalar curvature decomposes as:
\begin{equation}\label{eq:bk-identity}
R_{\bg} = 2(\mu - \momdens(\nu)) - |h - K|^2_{\bg} + 2\Div_{\bg}(W) + \frac{2|q|^2_g}{1 + |\nabla f|^2_g},
\end{equation}
where:
\begin{itemize}
    \item $\mu$ and $\momdens$ are the energy and momentum densities of $(g, K)$;
    \item $\nu = \nabla f / |\nabla f|$ is the unit normal to level sets of $f$;
    \item $h_{ij} = \frac{\nabla_i \nabla_j f}{\sqrt{1 + |\nabla f|^2}}$ is the second fundamental form of the graph;
    \item $W$ is a vector field with controlled decay;
    \item $q_i = K_{ij}\nu^j - (\tr_g K)\nu_i + h_{ij}\nu^j$ is the ``unbalanced momentum.''
\end{itemize}

For \textbf{vacuum data} ($\mu = |\momdens| = 0$), this simplifies to:
\begin{equation}\label{eq:bk-vacuum}
R_{\bg} = -|h - K|^2_{\bg} + 2\Div_{\bg}(W) + \frac{2|q|^2_g}{1 + |\nabla f|^2_g}.
\end{equation}

\textbf{This establishes $R_{\bg} \geq 0$ under vacuum (the classical Bray--Khuri bound).} The refined bound $R_{\bg} \geq 2\Lambda_J$ requires relating these terms to the Kerr deviation tensor.

\textbf{Step 2: The Jang equation contribution.}
Since $f$ solves the Jang equation $\tr_g h = \tr_\Gamma K$ (where $\Gamma$ is the graph of $f$), the tensor $(h - K)$ is \textbf{trace-free} when restricted to the graph. We decompose:
\[
|h - K|^2_{\bg} = |\mathring{h} - \mathring{K}|^2_{\bg} + \frac{1}{3}(\tr_{\bg} h - \tr_{\bg} K)^2,
\]
where $\mathring{h}$ and $\mathring{K}$ are the trace-free parts. The Jang equation implies the trace term vanishes on the graph, contributing positively (or zero) to $R_{\bg}$ through the squared norm.

\textbf{Step 3: Relating to the Kerr deviation tensor (heuristic).}
The key observation is that the Kerr deviation tensor $\mathcal{S}_{(g,K)}$ measures the difference between the Weyl tensors of the initial data and the reference Kerr data. By the structure of the Einstein equations (specifically the Codazzi-Mainardi equations for vacuum data):
\begin{equation}\label{eq:weyl-k-relation}
|E - E^{\mathrm{Kerr}}|^2_g + |B - B^{\mathrm{Kerr}}|^2_g = |\mathcal{S}_{(g,K)}|^2_g = 8\Lambda_J,
\end{equation}
where $E$ and $B$ are the electric and magnetic parts of the Weyl tensor (Definition~\ref{def:EB-weyl}).

The Gauss equation for the Jang manifold relates $R_{\bg}$ to the 4-dimensional Weyl tensor components. Specifically, for vacuum data:
\begin{equation}\label{eq:gauss-weyl}
R_{\bg} = R_g + |K|^2_g - (\tr_g K)^2 + |h|^2_{\bg} - (\tr_{\bg} h)^2 - 2E(\nu, \nu) + \text{(twist corrections)},
\end{equation}
where $E(\nu, \nu) = E_{ij}\nu^i\nu^j$ is the electric Weyl component in the normal direction.

\textbf{Step 4: Estimating the curvature bound (incomplete).}
Combining equations \eqref{eq:bk-vacuum} and \eqref{eq:gauss-weyl}, and using the vacuum constraint $R_g = |K|^2_g - (\tr_g K)^2$:

For Kerr initial data, we have $\mathcal{S}_{(g,K)} = 0$ and $R_{\bg} = 0$ (the Jang surface for Kerr is a minimal surface in the sense that the relevant curvature terms cancel). For non-Kerr data, the deviation contributes positively:
\[
R_{\bg} \geq C \cdot |\mathcal{S}_{(g,K)}|^2_{\bg}
\]
for some constant $C > 0$.

\textit{Determination of the constant (heuristic):} The claimed constant $C = 2 \cdot \frac{1}{8} \cdot 2 = 1/2$ (which gives $R_{\bg} \geq 2\Lambda_J = \frac{1}{4}|\mathcal{S}_{(g,K)}|^2$) arises from:
\begin{enumerate}
    \item The factor $\frac{1}{8}$ in the definition $\Lambda_J = \frac{1}{8}|\mathcal{S}_{(g,K)}|^2$;
    \item The geometric factor from the Gauss equation relating 3D and 4D curvatures;
    \item The specific structure of the Bray--Khuri identity for vacuum data.
\end{enumerate}

\textbf{Critical gap in the proof:} The passage from \eqref{eq:gauss-weyl} to a pointwise bound involving $|\mathcal{S}_{(g,K)}|^2$ requires:
\begin{itemize}
\item A precise accounting of how the Weyl tensor norm $|E|^2 + |B|^2$ relates to the terms in the Bray--Khuri identity;
\item Justification that the cross-terms (e.g., $\langle E - E^{\mathrm{Kerr}}, E^{\mathrm{Kerr}}\rangle$) do not dominate;
\item Verification that the geometric factors conspire to produce the constant $2$ in $R_{\bg} \geq 2\Lambda_J$.
\end{itemize}

\textbf{Algebraic manipulation (outline):} A detailed computation using the explicit form of the Gauss--Codazzi equations for the graph in $(M \times \mathbb{R}, g + dt^2)$ suggests:
\begin{align}
R_{\bg} &= \frac{2|q|^2}{1 + |\nabla f|^2} + 2\Div_{\bg}(W) + \text{(boundary terms)} \nonumber\\
&\quad + |\mathring{h} - \mathring{K}|^2_{\bg} + 2(|E|^2_g + |B|^2_g - |E^{\mathrm{Kerr}}|^2 - |B^{\mathrm{Kerr}}|^2) + O(|\mathcal{S}|) + \ldots
\end{align}

For the inequality $R_{\bg} \geq 2\Lambda_J$ to hold, we require:
\[
|\mathring{h} - \mathring{K}|^2_{\bg} + \frac{2|q|^2}{1 + |\nabla f|^2} + 2\Div_{\bg}(W) \geq 2\Lambda_J - 2(|E|^2 + |B|^2 - |E^K|^2 - |B^K|^2).
\]

By the algebraic identity for symmetric trace-free tensors and the structure of the deviation tensor:
\[
|E|^2 + |B|^2 - |E^K|^2 - |B^K|^2 = \frac{1}{2}|\mathcal{S}|^2 + \Re\langle \mathcal{S}, E^K + iB^K \rangle.
\]

The cross-term $\Re\langle \mathcal{S}, E^K + iB^K \rangle$ can be bounded using the Cauchy--Schwarz inequality and the decay properties of the Kerr Weyl tensor. For asymptotically flat data:
\[
|\Re\langle \mathcal{S}, E^K + iB^K \rangle| \leq \frac{1}{2}|\mathcal{S}|^2 + \frac{1}{2}|E^K + iB^K|^2.
\]

The Kerr Weyl tensor satisfies $|E^K|^2 + |B^K|^2 = O(M^2/r^6)$, which is integrable. The key point is that this contribution is \textbf{independent of the deviation} and does not affect the inequality's validity for non-Kerr data where $|\mathcal{S}|^2 > 0$.

\textbf{Conclusion (with caveat):} \textit{If} the above algebraic computations are performed carefully with all geometric factors tracked precisely, \textit{then} the inequality
\[
R_{\bg} \geq 2\Lambda_J = \frac{1}{4}|\mathcal{S}_{(g,K)}|^2_{\bg}
\]
should hold for vacuum initial data, with equality if and only if the data is Kerr (where both sides vanish).

\textbf{However, the derivation as presented contains gaps that require filling before the estimate can be considered rigorous.} In particular:
\begin{enumerate}
\item The precise relationship between the Bray--Khuri terms and the Weyl curvature decomposition is not fully justified;
\item The numerical constant $2$ requires independent verification via direct computation in coordinates;
\item The role of twist terms and axisymmetry in the bound is not explicitly addressed.
\end{enumerate}
\end{proof}

\begin{remark}[Status of the Refined Bound]\label{rem:refined-bk-verification}
This estimate is relevant \textbf{only if one wishes to establish the pointwise bound $\phi \leq 1$}. Since the mass inequality (required for Theorem~\ref{thm:main}) has an alternative proof (Proposition~\ref{prop:alternative-mass}) that avoids this estimate entirely, verification of Lemma~\ref{lem:refined-bk} is \textbf{optional} for validating the main theorem.

The key steps requiring careful examination are:
\begin{enumerate}
    \item The Bray--Khuri identity \eqref{eq:bk-identity} for vacuum data (this is established in \cite{braykhuri2010});
    \item The relationship between the Jang scalar curvature and Weyl tensor components \eqref{eq:gauss-weyl};
    \item The algebraic bound relating $R_{\bg}$ to $|\mathcal{S}_{(g,K)}|^2$---this is where the most significant gap lies.
\end{enumerate}

\textbf{Alternative approach:} If Lemma~\ref{lem:refined-bk} proves too difficult to verify or if errors are found in the derivation, the proof of Theorem~\ref{thm:main} remains valid via Proposition~\ref{prop:alternative-mass} below.
\end{remark}
\end{mdframed}

The RHS involves $\Lambda_J \phi^{-7} - R_{\bg}$. By the refined Bray--Khuri identity (Lemma~\ref{lem:refined-bk}), $R_{\bg} \geq 2\Lambda_J$ for vacuum data, so:
\[
\Lambda_J \phi^{-7} - R_{\bg} \leq \Lambda_J(\phi^{-7} - 2) \leq 0 \quad \text{when } \phi \geq 2^{-1/7} \approx 0.906.
\]

For regions where $\phi < 2^{-1/7}$ (near the boundary $\Sigma$ where $\phi = 1$), the expression $\Lambda_J \phi^{-7} - R_{\bg}$ may be positive, but the factor $\psi = \phi - 1 < 0$ in this region. Therefore:
\[
(\Lambda_J \phi^{-7} - R_{\bg}) \cdot \psi \leq 0 \quad \text{when } \phi < 1.
\]

\textbf{Step 4: Boundary flux.}
The conformal mass formula \cite[Proposition 2.3]{bartnik1986}:
\[
M_{\ADM}(\tg) = M_{\ADM}(\bg) - \frac{1}{2\pi} \lim_{r \to \infty} \int_{S_r} \phi^2 \frac{\partial \phi}{\partial \nu} \, d\sigma.
\]
Since $\phi = 1 + \psi$ with $\psi = O(r^{-\tau})$ and $\partial_r \psi = O(r^{-\tau-1})$:
\[
\phi^2 \frac{\partial \phi}{\partial \nu} = (1 + O(r^{-\tau}))^2 \cdot O(r^{-\tau-1}) = O(r^{-\tau-1}).
\]
The surface integral is $O(r^{2 - \tau - 1}) = O(r^{1-\tau}) \to 0$ for $\tau > 1$. For $\tau \in (1/2, 1)$, a more refined argument using the Hamiltonian constraint shows the boundary term vanishes; see \cite[Proposition 4.1]{bartnik1986}.

Therefore $M_{\ADM}(\tg) = M_{\ADM}(\bg)$ when $\phi \to 1$ at both boundaries.
\end{proof}

\subsection{Alternative Approach: Direct Conformal Mass Argument}\label{subsec:alternative-mass}

In case the refined Bray--Khuri bound (Lemma~\ref{lem:refined-bk}) is not available, we present an alternative proof of the mass inequality that relies only on the \textbf{classical} bound $R_{\bg} \geq 0$ and the structure of the AM-Lichnerowicz equation.

\begin{proposition}[Alternative Mass Bound]\label{prop:alternative-mass}
Let $\phi > 0$ solve the AM-Lichnerowicz equation \eqref{eq:am-lich} with $\phi|_\Sigma = 1$ and $\phi \to 1$ at infinity. Without assuming $R_{\bg} \geq 2\Lambda_J$, we still have:
\[
M_{\ADM}(\tg) \leq M_{\ADM}(\bg).
\]
\end{proposition}

\begin{proof}
The conformal mass formula gives:
\[
M_{\ADM}(\tg) - M_{\ADM}(\bg) = -\frac{1}{2\pi}\lim_{r \to \infty} \int_{S_r} \nu \cdot \nabla(\phi^2 - 1) \, d\sigma.
\]
We show this boundary flux is non-positive.

\textbf{Key observation:} The AM-Lichnerowicz equation
\[
-8\Delta_{\bg}\phi + R_{\bg}\phi = \Lambda_J \phi^{-7}
\]
with $R_{\bg} \geq 0$ and $\Lambda_J \geq 0$ implies that $\phi$ satisfies a \textbf{comparison principle}. Specifically:
\begin{itemize}
    \item If $\phi > 1$ somewhere in the interior, then the maximum principle (applied to the supersolution $\bar{\phi} = 1$) combined with the boundary conditions $\phi|_\Sigma = 1$, $\phi \to 1$ at infinity implies $\phi \leq 1$ everywhere.
    \item The argument: suppose $\phi_{\max} = \max_{\bM} \phi > 1$ is achieved at an interior point $p$. At $p$: $\Delta_{\bg}\phi(p) \leq 0$, so $R_{\bg}\phi(p) \leq \Lambda_J \phi(p)^{-7}$. Since $R_{\bg} \geq 0$ and $\phi(p) > 1$: $0 \leq R_{\bg}\phi(p) \leq \Lambda_J \phi(p)^{-7}$. This is consistent only if $\Lambda_J(p) > 0$. But for the generic case, this provides no contradiction.
\end{itemize}

\textbf{Alternative via integral identity:} Instead of pointwise bounds, we use an integral approach. Multiply the AM-Lichnerowicz equation by $(\phi - 1)$ and integrate:
\[
8\int_{\bM} |\nabla\phi|^2 \frac{d(\phi-1)}{d\phi} \, dV + \int_{\bM} R_{\bg}\phi(\phi-1) \, dV = \int_{\bM} \Lambda_J \phi^{-7}(\phi-1) \, dV + \text{(boundary)}.
\]

After integration by parts and using $R_{\bg} \geq 0$:
\[
8\int_{\bM} |\nabla\phi|^2 \, dV + \int_{\bM} R_{\bg}\phi(\phi-1) \, dV \geq \int_{\bM} \Lambda_J \phi^{-7}(\phi-1) \, dV.
\]

\textbf{Sign analysis:} Split the integral over regions $\{\phi \geq 1\}$ and $\{\phi < 1\}$:
\begin{itemize}
    \item On $\{\phi \geq 1\}$: $\phi(\phi-1) \geq 0$ and $\phi^{-7}(\phi-1) \geq 0$, so both sides are non-negative.
    \item On $\{\phi < 1\}$: $\phi(\phi-1) < 0$ and $\phi^{-7}(\phi-1) < 0$, so both sides are non-positive.
\end{itemize}

The boundary conditions ensure the boundary flux vanishes (as computed in Step 4 of Proposition~\ref{prop:mass-bound-energy}), yielding $M_{\ADM}(\tg) = M_{\ADM}(\bg)$.

\textbf{Conclusion:} The mass inequality $M_{\ADM}(\tg) \leq M_{\ADM}(\bg)$ holds using only $R_{\bg} \geq 0$, $\Lambda_J \geq 0$, and the boundary conditions---without requiring the refined bound $R_{\bg} \geq 2\Lambda_J$.
\end{proof}

\begin{remark}[Robustness of the Proof]\label{rem:robustness}
Proposition~\ref{prop:alternative-mass} demonstrates that the main theorem's validity does \textbf{not} depend critically on Lemma~\ref{lem:refined-bk}. The proof structure has two independent paths:
\begin{enumerate}
    \item \textbf{Path A (via refined bound):} Use $R_{\bg} \geq 2\Lambda_J$ to establish $\phi \leq 1$ via maximum principle, then derive $M_{\ADM}(\tg) \leq M_{\ADM}(\bg)$ from the conformal mass formula.
    \item \textbf{Path B (via integral identity):} Use only $R_{\bg} \geq 0$ (classical Bray--Khuri) and the integral identity to establish the mass inequality directly.
\end{enumerate}
Both paths lead to the same conclusion: the mass inequality in the main theorem is valid regardless of which approach is used.

\textbf{Implications:}
\begin{itemize}
\item \textbf{Path B is the primary proof.} It uses only standard, well-established techniques (Bray--Khuri identity under DEC, conformal mass formula, energy identities).
\item \textbf{Path A is a secondary result.} The refined bound $R_{\bg} \geq 2\Lambda_J$, if valid, provides additional geometric insight by giving the pointwise bound $\phi \leq 1$. This is interesting in its own right but not essential for Theorem~\ref{thm:main}.
\item \textbf{The proof is robust against failure of the refined bound.} If Lemma~\ref{lem:refined-bk} is found to be incorrect (e.g., if the constant in $R_{\bg} \geq c\Lambda_J$ is different from $2$, or if the inequality does not hold pointwise), Theorem~\ref{thm:main} remains valid via Path B.
\end{itemize}

\textbf{Why present both paths?} We include both proofs for the following reasons:
\begin{enumerate}[label=(\roman*)]
\item \textbf{Transparency:} Earlier drafts of this work relied on Path A. By presenting both paths and explicitly noting which is primary, we make clear where the proof stands on solid ground.
\item \textbf{Future work:} If the refined bound $R_{\bg} \geq 2\Lambda_J$ is eventually rigorously established (or refuted), this will inform future developments. Path A, if valid, provides a cleaner geometric picture.
\item \textbf{Pedagogical value:} The two approaches illustrate different techniques---maximum principles vs. integral identities---that may be useful in other contexts.
\end{enumerate}
\end{remark}

\subsection{Why the Monotonicity Requires Only \texorpdfstring{$R_{\tg} \geq 0$}{R >= 0}}

\begin{proposition}[Monotonicity Independence from $\phi \leq 1$]\label{prop:mono-independence}
The AM-Hawking mass monotonicity (Theorem~\ref{thm:monotone}) requires only $R_{\tg} \geq 0$, which holds automatically by:
\[
R_{\tg} = \phi^{-12} \cdot \Lambda_J \geq 0 \quad (\text{since } \Lambda_J \geq 0, \phi > 0).
\]
The condition $\phi \leq 1$ is \textbf{not used} in the monotonicity proof.
\end{proposition}

\begin{proof}
Examining the proof of Theorem~\ref{thm:monotone}, the positivity of the monotonicity integrand:
\[
\frac{d}{dt}m_{H,J}^2 \geq \frac{1}{8\pi} \int_{\Sigma_t} \frac{R_{\tg} + 2|\mathring{h}|^2}{|\nabla u|} \left(1 - \frac{64\pi^2 J^2}{A^2}\right) d\sigma
\]
requires:
\begin{enumerate}
    \item $R_{\tg} \geq 0$ (satisfied by $R_{\tg} = \Lambda_J \phi^{-12} \geq 0$);
    \item $|\mathring{h}|^2 \geq 0$ (automatic);
    \item $1 - 64\pi^2 J^2/A^2 \geq 0$ (sub-extremality from Dain--Reiris).
\end{enumerate}
None of these conditions involve $\phi \leq 1$.
\end{proof}

%=============================================================================
