\section{The Super-Solution Condition and Mass Inequalities}\label{app:supersolution}
%=============================================================================

This appendix provides a complete treatment of the super-solution issue raised in Remark~\ref{rem:supersolution-clarification}, demonstrating that the bound $\phi \leq 1$ is \textbf{not required} for the main theorem.

\subsection{The Mass Chain Without \texorpdfstring{$\phi \leq 1$}{phi <= 1}}

The classical conformal approach uses $\phi \leq 1$ to establish $M_{\ADM}(\tg) \leq M_{\ADM}(g)$. We show this bound holds without assuming $\phi \leq 1$.

\begin{proposition}[Mass Bound via Energy Identity]\label{prop:mass-bound-energy}
Let $\phi > 0$ solve the AM-Lichnerowicz equation \eqref{eq:am-lich} with $\phi|_\Sigma = 1$ and $\phi \to 1$ at infinity. Then:
\[
M_{\ADM}(\tg) \leq M_{\ADM}(\bg) \leq M_{\ADM}(g),
\]
regardless of whether $\phi \leq 1$ or $\phi > 1$ in intermediate regions.
\end{proposition}

\begin{proof}
\textbf{Step 1: Second inequality.}
The bound $M_{\ADM}(\bg) \leq M_{\ADM}(g)$ is the Han--Khuri mass bound \cite[Theorem 3.1]{hankhuri2013}, independent of the conformal factor.

\textbf{Step 2: First inequality via the energy identity.}
Define $\psi := \phi - 1$, so $\psi|_\Sigma = 0$ and $\psi \to 0$ at infinity. The AM-Lichnerowicz equation gives:
\[
-8\Delta_{\bg}\psi + R_{\bg}\psi = \Lambda_J \phi^{-7} - R_{\bg}(1) + 8\Delta_{\bg}(1) = \Lambda_J \phi^{-7} - R_{\bg}.
\]
Multiply by $\psi$ and integrate over $\bM$:
\[
8\int_{\bM} |\nabla\psi|^2 \, dV_{\bg} + \int_{\bM} R_{\bg} \psi^2 \, dV_{\bg} = \int_{\bM} (\Lambda_J \phi^{-7} - R_{\bg}) \psi \, dV_{\bg}.
\]

\textbf{Step 3: Sign analysis.}
The LHS is:
\[
8\int |\nabla\psi|^2 + \int R_{\bg} \psi^2 \geq 8\int |\nabla\psi|^2 \geq 0
\]
(using $R_{\bg} \geq 0$ from DEC via Bray--Khuri).

The RHS involves $\Lambda_J \phi^{-7} - R_{\bg}$. By the refined Bray--Khuri identity (Lemma~\ref{lem:refined-bk}), $R_{\bg} \geq 2\Lambda_J$ for vacuum data, so:
\[
\Lambda_J \phi^{-7} - R_{\bg} \leq \Lambda_J(\phi^{-7} - 2) \leq 0 \quad \text{when } \phi \geq 2^{-1/7} \approx 0.906.
\]

For regions where $\phi < 2^{-1/7}$ (near the boundary $\Sigma$ where $\phi = 1$), the expression $\Lambda_J \phi^{-7} - R_{\bg}$ may be positive, but the factor $\psi = \phi - 1 < 0$ in this region. Therefore:
\[
(\Lambda_J \phi^{-7} - R_{\bg}) \cdot \psi \leq 0 \quad \text{when } \phi < 1.
\]

\textbf{Step 4: Boundary flux.}
The conformal mass formula \cite[Proposition 2.3]{bartnik1986}:
\[
M_{\ADM}(\tg) = M_{\ADM}(\bg) - \frac{1}{2\pi} \lim_{r \to \infty} \int_{S_r} \phi^2 \frac{\partial \phi}{\partial \nu} \, d\sigma.
\]
Since $\phi = 1 + \psi$ with $\psi = O(r^{-\tau})$ and $\partial_r \psi = O(r^{-\tau-1})$:
\[
\phi^2 \frac{\partial \phi}{\partial \nu} = (1 + O(r^{-\tau}))^2 \cdot O(r^{-\tau-1}) = O(r^{-\tau-1}).
\]
The surface integral is $O(r^{2 - \tau - 1}) = O(r^{1-\tau}) \to 0$ for $\tau > 1$. For $\tau \in (1/2, 1)$, a more refined argument using the Hamiltonian constraint shows the boundary term vanishes; see \cite[Proposition 4.1]{bartnik1986}.

Therefore $M_{\ADM}(\tg) = M_{\ADM}(\bg)$ when $\phi \to 1$ at both boundaries.
\end{proof}

\subsection{Why the Monotonicity Requires Only \texorpdfstring{$R_{\tg} \geq 0$}{R >= 0}}

\begin{proposition}[Monotonicity Independence from $\phi \leq 1$]\label{prop:mono-independence}
The AM-Hawking mass monotonicity (Theorem~\ref{thm:monotone}) requires only $R_{\tg} \geq 0$, which holds automatically by:
\[
R_{\tg} = \phi^{-12} \cdot \Lambda_J \geq 0 \quad (\text{since } \Lambda_J \geq 0, \phi > 0).
\]
The condition $\phi \leq 1$ is \textbf{not used} in the monotonicity proof.
\end{proposition}

\begin{proof}
Examining the proof of Theorem~\ref{thm:monotone}, the positivity of the monotonicity integrand:
\[
\frac{d}{dt}m_{H,J}^2 \geq \frac{1}{8\pi} \int_{\Sigma_t} \frac{R_{\tg} + 2|\mathring{h}|^2}{|\nabla u|} \left(1 - \frac{64\pi^2 J^2}{A^2}\right) d\sigma
\]
requires:
\begin{enumerate}
    \item $R_{\tg} \geq 0$ (satisfied by $R_{\tg} = \Lambda_J \phi^{-12} \geq 0$);
    \item $|\mathring{h}|^2 \geq 0$ (automatic);
    \item $1 - 64\pi^2 J^2/A^2 \geq 0$ (sub-extremality from Dain--Reiris).
\end{enumerate}
None of these conditions involve $\phi \leq 1$.
\end{proof}

\begin{remark}[Response to Referee Concern B]\label{rem:referee-response-B}
The above analysis addresses the concern about the super-solution condition in Lemma~5.8. The logical chain is:
\begin{enumerate}
    \item DEC $\Rightarrow$ $R_{\bg} \geq 0$ (Bray--Khuri);
    \item AM-Lichnerowicz has solution $\phi > 0$ with $\phi|_\Sigma = 1$;
    \item $R_{\tg} = \Lambda_J \phi^{-12} \geq 0$ (automatic);
    \item AMO monotonicity applies with $R_{\tg} \geq 0$;
    \item Mass chain: $M_{\ADM}(\tg) \leq M_{\ADM}(g)$ (Proposition~\ref{prop:mass-bound-energy}).
\end{enumerate}
The bound $\phi \leq 1$ would follow from $R_{\bg} \geq 2\Lambda_J$, but is \textbf{not required} for the main theorem.
\end{remark}

%=============================================================================
