% Title, Abstract, and Table of Contents
% This file should be included after \begin{document}

\begin{center}
{\LARGE\bfseries The Angular Momentum Penrose Inequality}

\vspace{0.3cm}

{\large A Proof via the Extended Jang--Conformal--AMO Method}

\vspace{1cm}

{\large Da Xu}

\vspace{0.3cm}

{\itshape China Mobile Research Institute\\
Beijing 100053, China\\
E-mail: xuda@chinamobile.com}

\vspace{1cm}

December 21, 2025

\end{center}

\vspace{1cm}
\begin{abstract}
We prove the \textbf{Angular Momentum Penrose Inequality}: for asymptotically flat, axisymmetric initial data $(M^3, g, K)$ satisfying the dominant energy condition with vacuum in the exterior region, and containing an outermost strictly stable marginally outer trapped surface (MOTS) $\Sigma$ of area $A$ and Komar angular momentum $J$,
\[
M_{\ADM} \geq \sqrt{\frac{A}{16\pi} + \frac{4\pi J^2}{A}},
\]
with equality if and only if the data arises from a slice of the Kerr spacetime.

The proof introduces a four-stage \textbf{Jang--conformal--AMO method}: (1)~solve an axisymmetric Jang equation with twist as a lower-order perturbation; (2)~solve an angular-momentum-modified Lichnerowicz equation; (3)~establish angular momentum conservation via Stokes' theorem for closed 2-forms; (4)~apply the Dain--Reiris sub-extremality bound. The key innovation is the \textbf{AM-Hawking mass} $m_{H,J}(t) := \sqrt{m_H^2(t) + 4\pi J^2/A(t)}$, which is monotonically non-decreasing along the $p$-harmonic flow and converges to $M_{\ADM}$.

As an application, we prove the \textbf{Charged Penrose Inequality} $M_{\ADM} \geq M_{\mathrm{irr}} + Q^2/(4M_{\mathrm{irr}})$ for non-rotating Einstein--Maxwell data.
\end{abstract}

\tableofcontents
