% Attempt to remove vacuum hypothesis: New constraint geometry approach
% 
% Goal: Prove M^2 >= A/(16π) + 4πJ²/A using only:
%   (C1) M² >= A/(16π)         [Hawking - DEC only]
%   (C3) A >= 8π|J|            [Dain-Reiris - stability only, NO vacuum needed]
%
% Key observation: From (C1) and (C3):
%   M² >= A/(16π) >= 8π|J|/(16π) = |J|/2
%
% This gives M² >= |J|/2, weaker than Dain's M² >= |J|.
%
% Question: Can we still prove AM-Penrose with only (C1) and (C3)?

% Define: Q(M,A,J) = M² - A/(16π) - 4πJ²/A
% Target: Q >= 0

% New approach: Directly optimize Q over the weaker constraint set
%   C' = {(M,A,J) : M² >= A/(16π), A >= 8π|J|}

% Parameterize:
%   Let A = 8πα|J| with α >= 1  (from C3)
%   Let M² = A/(16π) + B = α|J|/2 + B with B >= 0  (from C1)

% Then:
%   Q = M² - A/(16π) - 4πJ²/A
%     = (α|J|/2 + B) - α|J|/2 - 4πJ²/(8πα|J|)
%     = B - |J|/(2α)

% For Q >= 0, we need: B >= |J|/(2α)

% The constraint set C' does NOT require any relation between B and J!
% 
% Counterexample: Take α = 1, B = 0:
%   A = 8π|J|, M² = |J|/2
%   Q = 0 - |J|/2 = -|J|/2 < 0   VIOLATION!

% This proves: Vacuum hypothesis CANNOT be trivially removed.

% ===========================================================================
% KEY INSIGHT: The algebraic structure
% ===========================================================================

% The AM-Penrose inequality is equivalent to:
%   M² >= A/(16π) + 4πJ²/A
%
% Rearranging:
%   M² - A/(16π) >= 4πJ²/A
%   A(16πM² - A) >= 64π²J²
%   16πM²·A - A² >= 64π²J²

% Let x = A. This is a quadratic in x:
%   -x² + 16πM²·x - 64π²J² >= 0
%   x² - 16πM²·x + 64π²J² <= 0

% The quadratic x² - 16πM²·x + 64π²J² has roots at:
%   x = 8π(M² ± √(M⁴ - J²))
%
% For real roots, need M⁴ >= J², i.e., M² >= |J|  ← THIS IS DAIN!

% ===========================================================================
% CONCLUSION
% ===========================================================================

% Without Dain's inequality M² >= |J|:
% - The quadratic has no real roots when M² < |J|
% - The quadratic is POSITIVE for all A when M² < |J|
% - This means AM-Penrose is VIOLATED when M² < |J|

% Therefore, the vacuum hypothesis (which enables Dain) is NECESSARY 
% for the current proof structure.

% ===========================================================================
% THE ONLY WAY TO REMOVE VACUUM
% ===========================================================================

% To remove vacuum, one must prove Dain's inequality M² >= |J| 
% under DEC alone, without vacuum.
%
% This would require a new proof of Dain's inequality that:
% 1. Does not use the vacuum structure of the twist equations
% 2. Works for general DEC matter
%
% This is an OPEN PROBLEM in mathematical relativity.
