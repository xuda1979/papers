% THE SPACETIME APPROACH: A 4D PROOF OF PENROSE
%
% The key insight: current proofs are "3+1" methods.
% A sign-independent proof might need to be genuinely 4-dimensional.

\documentclass[12pt]{article}
\usepackage{amsmath,amsthm,amssymb}
\usepackage{mathrsfs}
\newtheorem{theorem}{Theorem}
\newtheorem{lemma}{Lemma}
\newtheorem{proposition}{Proposition}
\newtheorem{corollary}{Corollary}
\newtheorem{conjecture}{Conjecture}
\newtheorem{remark}{Remark}
\newtheorem{definition}{Definition}
\newtheorem{problem}{Problem}
\newtheorem{claim}{Claim}
\newtheorem{principle}{Principle}
\newtheorem{insight}{Key Insight}

\begin{document}

\title{The Spacetime Approach:\\
Toward a 4D Proof of the Penrose Inequality}
\author{Mathematical Development}
\date{\today}
\maketitle

\section{The Core Insight}

\begin{insight}
The favorable jump condition $\tr_\Sigma k \ge 0$ is an artifact of the 3+1 
decomposition. It has no intrinsic 4D meaning.

A proof that avoids this condition must use the SPACETIME structure, not just 
the initial data slice.
\end{insight}

\section{What We Have Access To}

\subsection{The Initial Data}

$(M^3, g, k)$ with:
\begin{itemize}
    \item $R_g - |k|^2 + (\tr k)^2 = 16\pi\mu$ (Hamiltonian constraint)
    \item $\div(k - (\tr k)g) = 8\pi J$ (momentum constraint)
    \item DEC: $\mu \ge |J|$
\end{itemize}

\subsection{The Spacetime Development}

The maximal globally hyperbolic development $(V^4, g_{\mu\nu})$ with:
\begin{itemize}
    \item Einstein equations: $G_{\mu\nu} = 8\pi T_{\mu\nu}$
    \item DEC: $T_{\mu\nu}V^\mu W^\nu \ge 0$ for causal $V, W$
    \item $(M, g, k)$ embeds as a Cauchy surface
\end{itemize}

\subsection{Key 4D Quantities}

\begin{enumerate}
    \item \textbf{Null expansions} $\theta^\pm$ (intrinsic to $\Sigma$ in spacetime)
    \item \textbf{Bondi mass} $M_B(u)$ at null infinity (if it exists)
    \item \textbf{Hawking mass} $M_H[\Sigma]$ (quasi-local)
    \item \textbf{Domain of dependence} $D(\Omega)$ of a region $\Omega \subset M$
\end{enumerate}

\section{The Null Approach}

\subsection{Idea}

From the trapped surface $\Sigma$, shoot null geodesics and track how the 
geometry evolves. The mass at infinity should bound the initial area.

\subsection{The Outgoing Null Hypersurface}

Let $\mathcal{N}^+$ be the outgoing null hypersurface generated by null geodesics 
from $\Sigma$ with tangent $\ell^+ = n + \nu$.

Along $\mathcal{N}^+$, the expansion $\theta^+$ evolves by Raychaudhuri:
\[
\frac{d\theta^+}{d\lambda} = -\frac{1}{2}(\theta^+)^2 - |\sigma^+|^2 - R_{\mu\nu}\ell^{+\mu}\ell^{+\nu}
\]

NEC: $R_{\mu\nu}\ell^\mu\ell^\nu = 8\pi T_{\mu\nu}\ell^\mu\ell^\nu \ge 0$.

So: $\frac{d\theta^+}{d\lambda} \le -\frac{1}{2}(\theta^+)^2$.

\subsection{The Focusing Problem}

For trapped surfaces, $\theta^+_0 \le 0$.

If $\theta^+_0 < 0$, the equation $\dot{\theta} \le -\theta^2/2$ implies:
\[
\theta^+(\lambda) \le \frac{\theta^+_0}{1 + \frac{1}{2}\theta^+_0\lambda}
\]

This blows up to $-\infty$ at $\lambda = -2/\theta^+_0 > 0$.

\textbf{The null geodesics FOCUS before reaching infinity!}

This is actually the SINGULARITY theorem—trapped surfaces lead to geodesic incompleteness.

\subsection{The Resolution: Work Backwards}

Instead of going from $\Sigma$ to infinity, go from INFINITY back to $\Sigma$.

\begin{principle}[Backwards Light Cone]
Consider the past light cone of future null infinity $\mathscr{I}^+$.
This "scans" the spacetime from outside in.
\end{principle}

\section{The Bondi Mass Approach}

\subsection{Setup}

Assume the spacetime is asymptotically flat with null infinity $\mathscr{I}^+$.

The Bondi mass $M_B(u)$ at retarded time $u$ satisfies:
\[
\frac{dM_B}{du} = -\frac{1}{4\pi}\int_{\mathscr{I}^+_u} |N|^2 \, dA \le 0
\]

where $N$ is the news function (gravitational radiation).

So $M_B$ is DECREASING toward the future (mass-energy radiates away).

\subsection{The Mass Hierarchy}

\[
M_{\mathrm{ADM}} \ge M_B(u) \ge M_B(+\infty) = M_{\mathrm{final}}
\]

If the spacetime settles to a stationary state, $M_{\mathrm{final}}$ is the 
final black hole mass.

\subsection{Connecting to Trapped Surfaces}

\begin{claim}[Penrose's Original Argument]
If $\Sigma$ is trapped and cosmic censorship holds:
\begin{enumerate}
    \item A black hole forms with event horizon $\mathcal{H}$
    \item Area theorem: $A(\mathcal{H}) \ge A(\Sigma)$
    \item Final state: Kerr with $M_{\mathrm{final}} \ge \sqrt{A(\mathcal{H})/(16\pi)}$
    \item Mass loss: $M_{\mathrm{ADM}} \ge M_{\mathrm{final}}$
\end{enumerate}

Combining: $M_{\mathrm{ADM}} \ge \sqrt{A(\Sigma)/(16\pi)}$.
\end{claim}

\textbf{The problem}: This assumes cosmic censorship (unproven)!

\section{Avoiding Cosmic Censorship}

\subsection{The Challenge}

Cosmic censorship says singularities are hidden behind horizons.
Without it, we can't guarantee an event horizon exists.

\subsection{A Weaker Assumption}

\begin{definition}[Weak Cosmic Censorship for Initial Data]
Given initial data $(M, g, k)$ with trapped surface $\Sigma$, there EXISTS 
a future development containing a black hole region.
\end{definition}

This is weaker than full cosmic censorship—we only need ONE development with 
a horizon, not that ALL developments have horizons.

\subsection{Using the Outermost MOTS}

The outermost MOTS $\Sigma^*$ on the initial slice serves as a "proxy" for 
the event horizon.

It's a well-defined object on the initial data, no evolution needed!

\textbf{The AMO approach}: Prove $M_{\mathrm{ADM}} \ge M_H[\Sigma^*]$.

\section{The Geroch Monotonicity (4D Version)}

\subsection{The 3D Geroch Mass}

For surfaces in a time-symmetric slice ($k = 0$), the Geroch mass:
\[
m_G[\Sigma] = \sqrt{\frac{A}{16\pi}}\left(1 - \frac{1}{16\pi}\int_\Sigma H^2\right)
\]

is monotone under inverse mean curvature flow (when $H > 0$).

\subsection{The 4D Generalization}

\begin{definition}[Spacetime Geroch Mass]
For a spacelike 2-surface $\Sigma$ in spacetime:
\[
m_G^{4D}[\Sigma] = \sqrt{\frac{A}{16\pi}}\left(1 - \frac{1}{16\pi}\int_\Sigma \theta^+\theta^-\right)
\]
\end{definition}

\textbf{Properties}:
\begin{itemize}
    \item For time-symmetric ($\theta^+ = -\theta^- = H$): reduces to $m_G^{3D}$
    \item For MOTS ($\theta^+ = 0$): $m_G^{4D} = \sqrt{A/(16\pi)}$
    \item For trapped ($\theta^+\theta^- > 0$): $m_G^{4D} < \sqrt{A/(16\pi)}$
\end{itemize}

\subsection{Is $m_G^{4D}$ Monotone?}

Consider a foliation of spacetime by 2-surfaces $\Sigma_t$.

\textbf{Question}: Under what conditions is $\frac{d m_G^{4D}}{dt} \ge 0$?

This requires computing $\frac{d}{dt}(\theta^+\theta^-)$ along the foliation.

The evolution equations are:
\begin{align}
    \frac{d\theta^+}{dt} &= -\frac{1}{2}(\theta^+)^2 - |\sigma^+|^2 - R_{\mu\nu}\ell^+\ell^+ + \ldots \\
    \frac{d\theta^-}{dt} &= -\frac{1}{2}(\theta^-)^2 - |\sigma^-|^2 - R_{\mu\nu}\ell^-\ell^- + \ldots
\end{align}

The product $\theta^+\theta^-$ evolves complicatedly...

\section{A New Monotonicity: The Expansion Product}

\subsection{Definition}

Let $P = \theta^+\theta^-$ (the "expansion product").

For trapped surfaces: $P > 0$.
For untrapped surfaces: $P < 0$ (opposite signs).
For MOTS: $P = 0$.

\subsection{Evolution of $P$}

\[
\frac{dP}{d\lambda} = \theta^- \frac{d\theta^+}{d\lambda} + \theta^+ \frac{d\theta^-}{d\lambda}
\]

Using Raychaudhuri:
\begin{align}
    \frac{dP}{d\lambda} &= \theta^-\left(-\frac{(\theta^+)^2}{2} - |\sigma^+|^2 - R_{++}\right) \\
    &\quad + \theta^+\left(-\frac{(\theta^-)^2}{2} - |\sigma^-|^2 - R_{--}\right)
\end{align}

Since $\theta^\pm < 0$ for trapped surfaces and all other terms are $\le 0$ (by NEC):
\[
\frac{dP}{d\lambda} \ge 0 \quad \text{(for trapped surfaces)}
\]

\textbf{The expansion product INCREASES along outgoing null geodesics!}

\subsection{Implication}

As we flow outward, $P = \theta^+\theta^-$ increases from $P_0 > 0$ toward... what?

If the flow reaches a MOTS ($\theta^+ = 0$), then $P \to 0$.

But $P$ is increasing, not decreasing!

\textbf{Resolution}: The flow might reach infinity (where $P \to 0$ from above) 
or hit a singularity.

\section{The Light Cone Cut Approach}

\subsection{Setup}

Let $\mathcal{N}^-$ be the PAST null cone of a point $p$ near future timelike infinity $i^+$.

$\mathcal{N}^-$ intersects the initial data slice $M$ in some surface $\Sigma_p$.

As $p \to i^+$, the surfaces $\Sigma_p$ sweep out toward spatial infinity.

\subsection{The Mass at Each Cut}

Each cut $\Sigma_p$ has a Hawking mass $M_H[\Sigma_p]$.

\begin{conjecture}[Light Cone Monotonicity]
As $p \to i^+$, $M_H[\Sigma_p] \to M_{\mathrm{ADM}}$ and the approach is monotone.
\end{conjecture}

If true, this gives: $M_{\mathrm{ADM}} = \lim_{p\to i^+} M_H[\Sigma_p] \ge M_H[\Sigma]$ 
for any $\Sigma$ in the sequence.

\subsection{Connecting to Trapped Surfaces}

The challenge: trapped surfaces might NOT be in the "light cone cut" family.

\textbf{Idea}: Show that the Hawking mass of ANY trapped surface is bounded by 
the limit of the light cone sequence.

\section{The Area Radius Function}

\subsection{Definition}

For a 2-surface $\Sigma$, the \textbf{area radius} is:
\[
r_A[\Sigma] = \sqrt{\frac{A(\Sigma)}{4\pi}}
\]

This equals the radius of a round sphere with the same area.

\subsection{The Schwarzschild Comparison}

In Schwarzschild with mass $M$:
\begin{itemize}
    \item Horizon at $r = 2M$, so $r_A = 2M$
    \item Penrose inequality: $M \ge r_A(\Sigma)/2 = \sqrt{A/(16\pi)}$
\end{itemize}

\subsection{A 4D Area Radius}

\begin{definition}[Spacetime Area Radius]
For a 2-surface $\Sigma$ in spacetime, define:
\[
\mathcal{R}[\Sigma] = r_A \cdot f(\theta^+, \theta^-)
\]
where $f$ is a correction factor depending on the null expansions.
\end{definition}

\textbf{Goal}: Find $f$ such that $\mathcal{R}$ is monotone under some evolution 
and $\mathcal{R} \to 2M_{\mathrm{ADM}}$ at infinity.

\subsection{Constraints on $f$}

\begin{enumerate}
    \item $f(0, \theta^-) = 1$ (for MOTS, no correction needed)
    \item $f(\theta, -\theta) = 1$ (for time-symmetric, same as 3D)
    \item $f$ should be smooth and positive for trapped surfaces
\end{enumerate}

\textbf{Candidate}: $f = \frac{2|\theta^-|}{|\theta^+| + |\theta^-|}$.

Check:
- For MOTS: $f = 2|\theta^-|/|\theta^-| = 2$... too big!
- For symmetric ($\theta^+ = \theta^-$): $f = 2|\theta|/(2|\theta|) = 1$ ✓

Hmm, the MOTS case is off by a factor of 2.

\section{A Breakthrough Idea: The Trapped Surface Tube}

\subsection{The Construction}

Given a trapped surface $\Sigma_0$, consider the "tube" of marginally trapped 
surfaces emanating from $\Sigma_0$ in the spacetime development.

This is related to the DYNAMICAL HORIZON concept.

\begin{definition}[Trapped Surface Tube]
The trapped surface tube $\mathcal{T}$ is the union of all MOTS through points 
in the domain of dependence of $\Sigma_0$.
\end{definition}

\subsection{Properties}

On a dynamical horizon:
\begin{itemize}
    \item Each slice is a MOTS ($\theta^+ = 0$)
    \item Area can increase (for future DH) or decrease (for past DH)
    \item There's a "flux law" relating area change to matter/radiation flux
\end{itemize}

\subsection{The Area Bound}

\begin{theorem}[Dynamical Horizon Area Law]
For a future dynamical horizon $\mathcal{H}$:
\[
A_{\mathrm{final}} - A_{\mathrm{initial}} = \int_{\mathcal{H}} (\text{matter flux}) \ge 0
\]
by NEC.
\end{theorem}

This says: the area of the MOTS increases as we move into the future!

\subsection{Connecting to Penrose}

If $\Sigma_0$ is trapped (not just marginally), it lies INSIDE a MOTS $\Sigma^*$.

By the dynamical horizon area law: $A(\Sigma^*_{\mathrm{final}}) \ge A(\Sigma^*_{\mathrm{initial}}) \ge A(\Sigma_0)$.

The final state (if Kerr): $M_{\mathrm{final}} = \sqrt{A_{\mathrm{final}}/(16\pi)}$.

So: $M_{\mathrm{ADM}} \ge M_{\mathrm{final}} = \sqrt{A_{\mathrm{final}}/(16\pi)} \ge \sqrt{A(\Sigma_0)/(16\pi)}$.

\textbf{This works WITHOUT the favorable jump condition!}

\section{The Remaining Gap}

The argument above uses:
\begin{enumerate}
    \item Existence of a dynamical horizon (mild cosmic censorship)
    \item Final state is Kerr (strong assumption)
    \item No mass is radiated to infinity (or we track Bondi mass)
\end{enumerate}

\textbf{The gap}: Steps 1 and 2 are not proven in general!

\subsection{Weakening the Assumptions}

\textbf{Weaker version}:

Instead of "final state is Kerr," use "final state has $M \ge \sqrt{A/(16\pi)}$."

This follows from the RIGIDITY of the Penrose inequality: equality only for 
Schwarzschild. For any non-Schwarzschild state, strict inequality holds.

\textbf{Even weaker}: Just show $M_{\mathrm{ADM}} \ge M_B(u) \ge \sqrt{A(u)/(16\pi)}$ 
for all $u$, where $A(u)$ is the apparent horizon area at time $u$.

\section{Conclusion: The 4D Path}

The spacetime approach offers a path to the Penrose inequality that:
\begin{itemize}
    \item Uses the dynamical horizon structure
    \item Exploits the area increase law (under NEC)
    \item Avoids the 3+1 decomposition artifacts
\end{itemize}

\textbf{The key remaining challenge}: Make the argument rigorous without 
assuming strong cosmic censorship or Kerr final state.

\textbf{Most promising}: The dynamical horizon area law + careful tracking of 
mass through Bondi mass.

\end{document}
