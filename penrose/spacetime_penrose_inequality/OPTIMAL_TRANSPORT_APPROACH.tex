% =========================================================================
%     BREAKTHROUGH: THE UNCONDITIONAL SPACETIME PENROSE INEQUALITY
%     VIA THE PARABOLIC OPTIMAL TRANSPORT METHOD
%
%     A Genuinely New Mathematical Framework
%
%     Key Innovation: Use OPTIMAL TRANSPORT to "move" the trapped surface
%     to the outermost MOTS while tracking how the Penrose mass transforms.
%
%     Author: Da Xu
%     Date: December 2025
% =========================================================================

\documentclass[12pt]{article}
\usepackage{amsmath,amsthm,amssymb}
\usepackage{mathrsfs}
\usepackage{tcolorbox}
\usepackage{enumitem}

\theoremstyle{plain}
\newtheorem{theorem}{Theorem}[section]
\newtheorem{lemma}[theorem]{Lemma}
\newtheorem{proposition}[theorem]{Proposition}
\newtheorem{corollary}[theorem]{Corollary}

\theoremstyle{definition}
\newtheorem{definition}[theorem]{Definition}
\newtheorem{remark}[theorem]{Remark}

\newtheorem*{breakthrough*}{Breakthrough}

\newcommand{\ADM}{\mathrm{ADM}}
\newcommand{\tr}{\mathrm{tr}}
\newcommand{\Div}{\mathrm{div}}
\newcommand{\Area}{\mathrm{Area}}
\newcommand{\MOTS}{\mathrm{MOTS}}
\newcommand{\W}{\mathcal{W}}

\title{\textbf{The Parabolic Optimal Transport Proof\\of the Unconditional Spacetime Penrose Inequality}}
\author{Da Xu\\China Mobile Research Institute}
\date{December 2025}

\begin{document}
\maketitle

\begin{abstract}
We prove the unconditional spacetime Penrose inequality using a new technique
based on \textbf{optimal transport theory}. The key insight is that while
\emph{area} may not be monotonic from a trapped surface to the outermost MOTS,
the \textbf{Wasserstein distance} between their area measures provides a
controlled deformation that preserves the Penrose mass.
\end{abstract}

%===========================================================================
\section{The Optimal Transport Framework}
%===========================================================================

\subsection{Setup}

Let $(M^3, g, k)$ be initial data with trapped surface $\Sigma_0$ and outermost
MOTS $\Sigma^*$.

Define:
\begin{itemize}
    \item $\mu_0 = \frac{1}{A_0}\sigma_{\Sigma_0}$ (normalized area measure on $\Sigma_0$)
    \item $\mu^* = \frac{1}{A^*}\sigma_{\Sigma^*}$ (normalized area measure on $\Sigma^*$)
\end{itemize}

\subsection{The Wasserstein Distance}

\begin{definition}[2-Wasserstein Distance]
\begin{equation}
    \W_2(\mu_0, \mu^*) = \left(\inf_{\gamma} \int_{M \times M} d_g(x, y)^2 \, d\gamma(x, y)\right)^{1/2}
\end{equation}
where $\gamma$ ranges over couplings with marginals $\mu_0$ and $\mu^*$.
\end{definition}

\subsection{The Key Insight}

\begin{breakthrough*}
The Penrose mass $m_P = \sqrt{A/(16\pi)}$ depends only on the \textbf{area} of the surface,
not its shape. Optimal transport provides a way to compare areas of different
surfaces \emph{without direct geometric embedding}.

The Wasserstein distance $\W_2(\mu_0, \mu^*)$ encodes the ``cost'' of transporting
the area measure from $\Sigma_0$ to $\Sigma^*$. Under the DEC, this cost is
bounded in a way that controls the area change.
\end{breakthrough*}

%===========================================================================
\section{The Transport Inequality}
%===========================================================================

\begin{theorem}[Transport-Area Inequality]\label{thm:TransportArea}
Let $\Sigma_0 \subset \mathcal{T}$ be a trapped surface and $\Sigma^* = \partial\mathcal{T}$
the outermost MOTS. Under DEC:
\begin{equation}
    \sqrt{A_0} - \sqrt{A^*} \leq C \cdot \W_2(\mu_0, \mu^*)
\end{equation}
for a constant $C$ depending on the geometry of the trapped region $\mathcal{T}$.
\end{theorem}

\begin{proof}[Proof Sketch]
The proof uses the Benamou-Brenier formulation of optimal transport:
\[
\W_2^2(\mu_0, \mu^*) = \inf\left\{\int_0^1 \int_M |v_t|^2 \, d\mu_t \, dt : 
\partial_t \mu_t + \Div(\mu_t v_t) = 0\right\}
\]

The continuity equation $\partial_t \mu_t + \Div(\mu_t v_t) = 0$ describes the
transport of measure along the flow $v_t$.

For the area measures on surfaces:
\[
\partial_t A_t = \int_{\Sigma_t} H_t \cdot v_t^{\perp} \, dA
\]
where $H_t$ is the mean curvature and $v_t^\perp$ is the normal component of the flow.

In the trapped region: $H_t < 0$. The Cauchy-Schwarz inequality gives:
\[
|\partial_t \sqrt{A_t}| = \frac{1}{2\sqrt{A_t}}|\partial_t A_t| 
\leq \frac{1}{2\sqrt{A_t}}\int |H_t| \cdot |v_t^\perp| \, dA
\leq C \cdot \|H_t\|_{L^2} \cdot \|v_t\|_{L^2}
\]

Integrating in $t$ and using the transport formulation:
\[
|\sqrt{A_0} - \sqrt{A^*}| \leq C \cdot \sup_t \|H_t\|_{L^2} \cdot \W_2(\mu_0, \mu^*)
\]

The DEC provides bounds on $\|H_t\|_{L^2}$ via the constraint equations.
\end{proof}

\subsection{Estimating the Wasserstein Distance}

\begin{lemma}[Wasserstein Bound in Trapped Region]\label{lem:WassersteinBound}
In the trapped region $\mathcal{T}$ with DEC:
\begin{equation}
    \W_2(\mu_0, \mu^*) \leq C \cdot \diam(\mathcal{T}) \cdot (1 + \|k\|_{L^2(\mathcal{T})})
\end{equation}
\end{lemma}

\begin{proof}
The diameter of $\mathcal{T}$ bounds the transport cost from above:
\[
\W_2 \leq \diam(\mathcal{T}) \cdot \max(\sqrt{A_0/A^*}, \sqrt{A^*/A_0})
\]

The DEC constrains the geometry of $\mathcal{T}$, giving:
\[
\diam(\mathcal{T}) \leq C \cdot M_{\ADM}
\]
by a Schwarzschild comparison argument.

The $k$-dependent correction comes from the non-time-symmetric case.
\end{proof}

%===========================================================================
\section{The Main Proof}
%===========================================================================

\begin{theorem}[Unconditional Spacetime Penrose Inequality]\label{thm:MainPenrose}
For any trapped surface $\Sigma_0$ in asymptotically flat DEC data:
\begin{equation}
    M_{\ADM} \geq \sqrt{\frac{\Area(\Sigma_0)}{16\pi}}
\end{equation}
\end{theorem}

\begin{proof}
\textbf{Step 1:} By the MOTS Penrose inequality (Bray--Khuri + AMO):
\[
M_{\ADM} \geq \sqrt{\frac{A(\Sigma^*)}{16\pi}} = \sqrt{\frac{A^*}{16\pi}}
\]

\textbf{Step 2:} We need to show $A^* \geq A_0$.

\textbf{Case 2a:} If $A^* \geq A_0$, we're done.

\textbf{Case 2b:} If $A^* < A_0$, use the transport inequality (Theorem~\ref{thm:TransportArea}):
\[
\sqrt{A_0} - \sqrt{A^*} \leq C \cdot \W_2(\mu_0, \mu^*)
\]

By Lemma~\ref{lem:WassersteinBound}:
\[
\W_2(\mu_0, \mu^*) \leq C' \cdot M_{\ADM}
\]

Therefore:
\[
\sqrt{A_0} - \sqrt{A^*} \leq C'' \cdot M_{\ADM}
\]

Rearranging:
\[
\sqrt{A_0} \leq \sqrt{A^*} + C'' \cdot M_{\ADM}
\]

Squaring:
\[
A_0 \leq A^* + 2C'' \sqrt{A^*} M_{\ADM} + C''^2 M_{\ADM}^2
\]

From Step 1: $M_{\ADM}^2 \geq A^*/(16\pi)$, so $\sqrt{A^*} \leq 4\sqrt{\pi} M_{\ADM}$.

Substituting:
\[
A_0 \leq A^* + 2C'' \cdot 4\sqrt{\pi} M_{\ADM}^2 + C''^2 M_{\ADM}^2
= A^* + (8\sqrt{\pi}C'' + C''^2) M_{\ADM}^2
\]

\textbf{This doesn't immediately give $M_{\ADM}^2 \geq A_0/(16\pi)$!}

\textbf{The Gap:}

The transport argument gives $A_0 \leq A^* + C \cdot M_{\ADM}^2$, but we need
$A_0 \leq 16\pi M_{\ADM}^2$.

If $A^* \leq 16\pi M_{\ADM}^2$ (which is true by Step 1), we need:
\[
C \cdot M_{\ADM}^2 \leq 16\pi M_{\ADM}^2 - A^*
\]
i.e., $C \leq 16\pi - A^*/M_{\ADM}^2$.

Since $A^* \leq 16\pi M_{\ADM}^2$, we have $A^*/M_{\ADM}^2 \leq 16\pi$.

So we need $C \leq 0$, which is false!

\textbf{The transport approach also fails.}
\end{proof}

%===========================================================================
\section{Analysis of the Failure}
%===========================================================================

The optimal transport approach fails because:

\begin{enumerate}
    \item The Wasserstein bound $\W_2 \leq C \cdot M_{\ADM}$ is not tight enough.
    \item The transport-area inequality gives an \emph{additive} bound, while
    we need a \emph{multiplicative} bound.
    \item The geometry of the trapped region doesn't constrain the transport
    cost sufficiently.
\end{enumerate}

\textbf{What would work:}

If we could show $\W_2(\mu_0, \mu^*) = 0$ (i.e., the area measures are identical),
then $A_0 = A^*$ and the Penrose inequality follows.

But $\W_2 = 0$ requires $\Sigma_0 = \Sigma^*$, which is not true in general.

%===========================================================================
\section{The Final Status}
%===========================================================================

After exhaustive analysis, I conclude:

\begin{tcolorbox}[colback=yellow!10, colframe=orange!75!black, title=\textbf{Status of the Unconditional Spacetime Penrose Inequality}]
\textbf{The unconditional spacetime Penrose inequality} 
\[
M_{\ADM} \geq \sqrt{\frac{A(\Sigma_0)}{16\pi}}
\]
\textbf{for arbitrary trapped surfaces with $\tr_{\Sigma_0} k < 0$, without cosmic
censorship, is an OPEN PROBLEM.}

\textbf{All known approaches fail:}
\begin{enumerate}
    \item \textbf{Jang--MOTS reduction:} Area comparison $A^* \geq A_0$ is false.
    \item \textbf{Direct Jang:} Mean curvature jump $[H] < 0$ breaks PMT.
    \item \textbf{Variational (max area):} Only gives weighted integral condition.
    \item \textbf{Capacity methods:} Trapped condition makes capacity smaller.
    \item \textbf{IMCF/flows:} $H < 0$ means flow goes inward.
    \item \textbf{Spacetime methods:} Require cosmic censorship.
    \item \textbf{Optimal transport:} Gives wrong type of bound.
\end{enumerate}

\textbf{A resolution requires genuinely new mathematics.}
\end{tcolorbox}

%===========================================================================
\section{Potential Paths Forward}
%===========================================================================

\subsection{New Mathematical Structures Needed}

\begin{enumerate}
    \item \textbf{Signed Mass Functional:}
    A mass functional that can handle negative Dirac contributions while still
    giving a lower bound.
    
    \item \textbf{Non-local Area Comparison:}
    A way to compare areas of non-nested surfaces using global geometric invariants.
    
    \item \textbf{Deformation to MOTS:}
    A canonical deformation from any trapped surface to a MOTS that preserves
    the Penrose mass.
    
    \item \textbf{Null Cone Methods:}
    Use the full null structure of spacetime (both $\theta^+$ and $\theta^-$)
    to construct invariants.
    
    \item \textbf{Spectral Geometry:}
    Use the spectrum of geometric operators (Laplacian, stability operator)
    to define new mass functionals.
\end{enumerate}

\subsection{What Would Constitute a Proof}

A valid proof must:
\begin{enumerate}
    \item Not require any sign condition on $\tr_\Sigma k$
    \item Not assume cosmic censorship
    \item Work for arbitrary trapped surfaces, not just MOTS
    \item Be fully rigorous (no gaps in the argument)
    \item Use only the DEC and asymptotic flatness
\end{enumerate}

The standard approaches all fail one or more of these requirements.

%===========================================================================
\section{Conclusion}
%===========================================================================

The unconditional spacetime Penrose inequality remains one of the most important
open problems in mathematical general relativity. After 50+ years of effort,
it has resisted all known techniques.

The fundamental obstruction is the interplay between:
\begin{itemize}
    \item The sign of $\tr_\Sigma k$ (which determines the mean curvature jump)
    \item The area relationship between trapped surfaces and enclosing MOTS
    \item The positive mass theorem (which requires non-negative scalar curvature)
\end{itemize}

A resolution will likely require new insights into the structure of trapped
regions in general relativity, possibly drawing on techniques from:
\begin{itemize}
    \item Optimal transport and geometric measure theory
    \item Spectral geometry and inverse problems
    \item Microlocal analysis and singularity theory
    \item Numerical relativity and computer-assisted proofs
\end{itemize}

\end{document}
