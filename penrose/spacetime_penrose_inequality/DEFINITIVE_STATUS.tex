%==============================================================================
%     DEFINITIVE STATUS: THE SPACETIME PENROSE INEQUALITY
%==============================================================================
\documentclass[11pt]{article}
\usepackage{amsmath,amssymb,amsthm}
\usepackage[margin=1in]{geometry}
\usepackage{xcolor}
\usepackage{tcolorbox}

\theoremstyle{plain}
\newtheorem{theorem}{Theorem}
\newtheorem{proposition}[theorem]{Proposition}
\theoremstyle{definition}
\newtheorem{definition}[theorem]{Definition}
\newtheorem{remark}[theorem]{Remark}

\newtcolorbox{resultbox}[1][]{colback=green!5!white,colframe=green!60!black,
    fonttitle=\bfseries,title=#1}
\newtcolorbox{openbox}[1][]{colback=red!5!white,colframe=red!60!black,
    fonttitle=\bfseries,title=#1}

\title{\textbf{Definitive Status Report}\\[0.3cm]
\large The Spacetime Penrose Inequality: What We Know}
\author{Mathematical Analysis}
\date{\today}

\begin{document}
\maketitle

%==============================================================================
\section{The Spacetime Penrose Inequality}
%==============================================================================

\begin{definition}[Spacetime Penrose Inequality]
Let $(M^3, g, k)$ be asymptotically flat initial data satisfying the dominant energy condition. Let $\Sigma$ be a closed trapped surface ($\theta^+ \le 0$, $\theta^- < 0$). The \textbf{Spacetime Penrose Inequality} states:
\[
M_{\mathrm{ADM}} \ge \sqrt{\frac{A(\Sigma)}{16\pi}}.
\]
\end{definition}

%==============================================================================
\section{Established Results (Rigorous)}
%==============================================================================

\begin{resultbox}[Theorem 1: Stable Outermost MOTS]
Let $(M^3, g, k)$ be AF initial data with DEC and $\tau > 1$. Let $\Sigma^*$ be the outermost stable MOTS (apparent horizon). Then:
\[
M_{\mathrm{ADM}} \ge \sqrt{\frac{A(\Sigma^*)}{16\pi}}.
\]
\textbf{Status: PROVED} via Jang equation + AMO method + stability implies $[H] \ge 0$.
\end{resultbox}

\begin{resultbox}[Theorem 2: Favorable Jump Condition]
Let $\Sigma_0$ be a trapped surface with $\mathrm{tr}_{\Sigma_0} k \ge 0$ (favorable jump). Then:
\[
M_{\mathrm{ADM}} \ge \sqrt{\frac{A(\Sigma_0)}{16\pi}}.
\]
\textbf{Status: PROVED} via direct application of Jang--AMO to $\Sigma_0$.
\end{resultbox}

\begin{resultbox}[Theorem 3: Conditional (Cosmic Censorship)]
Under weak cosmic censorship plus standard assumptions (NEC, no white hole, Kerr final state), for ANY trapped surface:
\[
M_{\mathrm{ADM}} \ge \sqrt{\frac{A(\Sigma)}{16\pi}}.
\]
\textbf{Status: PROVED} conditionally via spacetime methods.
\end{resultbox}

%==============================================================================
\section{Open Problem}
%==============================================================================

\begin{openbox}[OPEN: Unconditional Penrose Inequality]
For trapped surfaces $\Sigma_0$ with $\mathrm{tr}_{\Sigma_0} k < 0$ (unfavorable jump):
\[
M_{\mathrm{ADM}} \ge \sqrt{\frac{A(\Sigma_0)}{16\pi}} \quad \text{???}
\]
\textbf{Status: OPEN} — no rigorous proof exists.
\end{openbox}

%==============================================================================
\section{Why the Open Problem Is Hard}
%==============================================================================

\begin{proposition}[The Obstruction]
For trapped surfaces with $\mathrm{tr}_\Sigma k < 0$:
\begin{enumerate}
    \item The Jang equation blows up at $\Sigma$ (barrier argument works).
    \item BUT the mean curvature jump $[H] = \mathrm{tr}_\Sigma k < 0$.
    \item The distributional scalar curvature is: $R = R^{\mathrm{reg}} + 2[H]\delta_\Sigma$.
    \item The negative Dirac mass breaks Miao smoothing.
    \item No known method produces $R_{\hat{g}} \ge 0$.
\end{enumerate}
\end{proposition}

\textbf{Failed approaches:}
\begin{enumerate}
    \item \textbf{Area Monotonicity} $A(\Sigma^*) \ge A(\Sigma_0)$: The proof attempt contains errors; likely FALSE.
    \item \textbf{Maximum Area Trapped Surface}: The variational argument gives only a weighted integral condition, not pointwise $[H] \ge 0$.
    \item \textbf{Modified Jang Equation}: No compatible existence theory.
    \item \textbf{Spacetime methods without cosmic censorship}: The area theorem gives wrong direction.
\end{enumerate}

%==============================================================================
\section{The Critical Gap in the Paper}
%==============================================================================

\textbf{Theorem 3.1 (Maximum Area Trapped Surface), Step 4, Case 2:}

The paper claims that for an unstable MOTS maximizing area:
\[
\int_\Sigma (\mathrm{tr}_\Sigma k) \cdot \phi_1 \, dA \ge 0 \quad \Longrightarrow \quad \mathrm{tr}_\Sigma k \ge 0 \text{ pointwise}.
\]

\textbf{THIS IS FALSE.}

A weighted integral being nonnegative does NOT imply the integrand is pointwise nonnegative. The Miao smoothing requires $[H] \ge 0$ \textbf{pointwise}, which is NOT established.

%==============================================================================
\section{Correct Statements for the Paper}
%==============================================================================

\begin{theorem}[Corrected Main Theorem]
Let $(M^3, g, k)$ be AF initial data with DEC and $\tau > 1$.
\begin{enumerate}
    \item[(a)] For the outermost stable MOTS $\Sigma^*$: $M_{\mathrm{ADM}} \ge \sqrt{A(\Sigma^*)/16\pi}$.
    \item[(b)] For trapped surfaces with $\mathrm{tr}_\Sigma k \ge 0$: $M_{\mathrm{ADM}} \ge \sqrt{A(\Sigma)/16\pi}$.
    \item[(c)] Under cosmic censorship: holds for all trapped surfaces.
\end{enumerate}
The unconditional result for arbitrary trapped surfaces is NOT established.
\end{theorem}

%==============================================================================
\section{Conclusion}
%==============================================================================

The \textbf{unconditional spacetime Penrose inequality} for arbitrary trapped surfaces without sign conditions remains one of the major open problems in mathematical general relativity.

What IS proved:
\begin{itemize}
    \item Penrose inequality for \textbf{apparent horizons} (outermost stable MOTS)
    \item Penrose inequality for trapped surfaces with \textbf{favorable jump}
    \item Penrose inequality under \textbf{cosmic censorship}
\end{itemize}

What IS NOT proved:
\begin{itemize}
    \item Penrose inequality for trapped surfaces with $\mathrm{tr}_\Sigma k < 0$ (unfavorable jump)
    \item Unconditional result without any sign condition or cosmic censorship
\end{itemize}

\textbf{This requires genuinely new mathematical ideas.}

\end{document}
