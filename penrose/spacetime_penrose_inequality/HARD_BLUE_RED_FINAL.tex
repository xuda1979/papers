%% HARD_BLUE_RED_FINAL.tex
%%
%% HARD Adversarial Attack on Penrose 1973 + WCC
%% No hand-waving allowed
%%
%% December 2025

\documentclass[11pt]{amsart}
\usepackage{amsmath,amssymb,amsthm}
\usepackage{xcolor}
\usepackage{tcolorbox}

\tcbuselibrary{theorems,skins}

\newtcolorbox{redteam}{
    colback=red!5!white,
    colframe=red!75!black,
    title={\textbf{RED TEAM ATTACK}}
}

\newtcolorbox{blueteam}{
    colback=blue!5!white,
    colframe=blue!75!black,
    title={\textbf{BLUE TEAM DEFENSE}}
}

\newtcolorbox{fatal}{
    colback=red!20!white,
    colframe=red!90!black,
    title={\textbf{FATAL FLAW?}}
}

\newtcolorbox{verdict}{
    colback=yellow!10!white,
    colframe=orange!75!black,
    title={\textbf{VERDICT}}
}

\newtheorem{theorem}{Theorem}
\newtheorem{lemma}[theorem]{Lemma}

\newcommand{\ADM}{\mathrm{ADM}}
\newcommand{\Area}{\mathrm{Area}}

\title{HARD Blue/Red Attack:\\Penrose 1973 with WCC}
\author{}
\date{December 2025}

\begin{document}
\maketitle

\begin{abstract}
We perform a rigorous adversarial attack on the claimed proof of Penrose 1973 under WCC. No hand-waving is accepted. Every step must be justified or marked as a gap.
\end{abstract}

%% ============================================================================
\section{The Claim Under Attack}
%% ============================================================================

\textbf{Claimed Theorem:} Under NEC + WCC, for any trapped surface $\Sigma$:
\[
M_{\ADM} \ge \sqrt{\frac{\Area(\Sigma)}{16\pi}}
\]

\textbf{Claimed Proof Chain:}
\begin{enumerate}
    \item $\Area(\Sigma) \le \Area(\Sigma^*)$ (area dominance)
    \item $M_{\ADM} \ge \sqrt{\Area(\Sigma^*)/(16\pi)}$ (MOTS Penrose)
    \item Combine: $M_{\ADM} \ge \sqrt{\Area(\Sigma)/(16\pi)}$
\end{enumerate}

%% ============================================================================
\section{Attack Round 1: MOTS Existence}
%% ============================================================================

\begin{redteam}
\textbf{Attack 1.1:} You claim outermost MOTS $\Sigma^*$ exists. Cite the exact theorem with hypotheses.

What if the trapped surface is in a spacetime where MOTS doesn't exist on that Cauchy surface?
\end{redteam}

\begin{blueteam}
\textbf{Defense 1.1:}

\textbf{Theorem (Andersson-Metzger 2009):} Let $(\mathcal{C}, g, k)$ be asymptotically flat initial data. If there exists a closed surface $\Sigma$ with $\theta^+ < 0$ (weakly trapped), then there exists an outermost MOTS $\Sigma^*$ enclosing $\Sigma$.

\textbf{Hypotheses:}
\begin{itemize}
    \item $\mathcal{C}$ is a complete 3-manifold
    \item $(g, k)$ satisfy constraint equations
    \item Asymptotically flat: $g_{ij} = \delta_{ij} + O(r^{-1})$, $k_{ij} = O(r^{-2})$
    \item No boundary (or suitable boundary conditions)
\end{itemize}

These are satisfied for initial data in asymptotically flat spacetimes.

\textbf{Reference:} Andersson-Metzger, ``The area of horizons and the trapped region,'' CMP 2009.
\end{blueteam}

\begin{verdict}
\textbf{DEFENDED.} MOTS existence is a rigorous theorem under standard hypotheses.
\end{verdict}

%% ============================================================================
\section{Attack Round 2: MOTS Penrose Inequality}
%% ============================================================================

\begin{redteam}
\textbf{Attack 2.1:} You claim $M_{\ADM} \ge \sqrt{\Area(\Sigma^*)/(16\pi)}$ for outermost MOTS. 

What is the EXACT theorem? Does it require DEC or just NEC? Single or multiple MOTS components?
\end{redteam}

\begin{blueteam}
\textbf{Defense 2.1:}

\textbf{Theorem (MOTS Penrose):} Let $(\mathcal{C}, g, k)$ be asymptotically flat initial data satisfying DEC (not just NEC). Let $\Sigma^*$ be the outermost MOTS (possibly disconnected with components $\Sigma^*_1, \ldots, \Sigma^*_n$). Then:
\[
M_{\ADM} \ge \sqrt{\frac{\sum_i \Area(\Sigma^*_i)}{16\pi}}
\]

\textbf{Proof methods:}
\begin{enumerate}
    \item Jang equation + Riemannian PI (Bray-Khuri approach)
    \item Direct spacetime methods (various)
\end{enumerate}

\textbf{CAUTION:} This requires DEC, which is STRONGER than NEC!
\end{blueteam}

\begin{redteam}
\textbf{Attack 2.2:} AHA! Penrose 1973 assumes NEC, but MOTS Penrose requires DEC!

DEC: $T_{\mu\nu}V^\mu$ is future causal for future timelike $V$

NEC: $T_{\mu\nu}\ell^\mu\ell^\nu \ge 0$ for null $\ell$

DEC $\Rightarrow$ NEC, but NEC $\not\Rightarrow$ DEC.

Your proof has an INCONSISTENT energy condition!
\end{redteam}

\begin{blueteam}
\textbf{Defense 2.2:}

You're correct that MOTS Penrose requires DEC while Hawking Area Theorem only needs NEC.

\textbf{Resolution:} Upgrade Penrose 1973 to assume DEC (not just NEC).

DEC is physically reasonable — it says energy doesn't flow faster than light.

Most matter models (dust, perfect fluids, electromagnetic field) satisfy DEC.

\textbf{Revised statement:} Penrose 1973 under DEC + WCC.

This is a minor strengthening of hypotheses, not a fatal flaw.
\end{blueteam}

\begin{verdict}
\textbf{PARTIALLY DEFENDED.} Must assume DEC (not just NEC). This is a correction to the statement.
\end{verdict}

%% ============================================================================
\section{Attack Round 3: Area Dominance}
%% ============================================================================

\begin{redteam}
\textbf{Attack 3.1 (CRITICAL):} You claim $\Area(\Sigma) \le \Area(\Sigma^*)$.

PROVE IT. No hand-waving, no ``physical intuition,'' no ``genericity.''

Give me a mathematical proof or admit it's an assumption.
\end{redteam}

\begin{blueteam}
\textbf{Defense 3.1:}

I cannot prove this from first principles.

\textbf{What we know:}
\begin{enumerate}
    \item $\Sigma$ is inside the region bounded by $\Sigma^*$ (by definition of outermost)
    \item $\Sigma$ has $\theta^+ < 0$
    \item $\Sigma^*$ has $\theta^+ = 0$
\end{enumerate}

\textbf{What we DON'T have:}
\begin{enumerate}
    \item A monotonicity formula connecting $\Area(\Sigma)$ to $\Area(\Sigma^*)$
    \item A variational characterization of trapped surfaces
    \item A geometric inequality bounding area of surfaces with $\theta^+ < 0$
\end{enumerate}

\textbf{Honest answer:} Area dominance is an ASSUMPTION, not a theorem.
\end{blueteam}

\begin{fatal}
\textbf{AREA DOMINANCE IS NOT PROVEN.}

Blue Team admits this is an assumption, not derived from WCC + DEC.

This is the critical gap in the proof.
\end{fatal}

\begin{redteam}
\textbf{Attack 3.2:} Can you at least give a COUNTEREXAMPLE or show it's hard?

Construct a trapped surface with area LARGER than the enclosing MOTS.
\end{redteam}

\begin{blueteam}
\textbf{Defense 3.2:}

\textbf{Why counterexamples are hard to construct:}

The trapped condition $\theta^+ = H + P < 0$ constrains the surface.

For a highly crinkled surface:
\begin{itemize}
    \item Mean curvature $H$ oscillates (positive in valleys, negative on peaks)
    \item To have $\theta^+ < 0$ everywhere, need $H + P < 0$ at every point
    \item This limits how ``crinkled'' the surface can be
\end{itemize}

\textbf{Spherical symmetry:} In Schwarzschild, all 2-spheres inside the horizon have area $\le 16\pi M^2 = \Area(\text{horizon})$. Easy to verify: $A(r) = 4\pi r^2$, maximized at $r = 2M$.

\textbf{Near-MOTS:} For $\Sigma$ close to $\Sigma^*$, perturbation theory gives $\Area(\Sigma) < \Area(\Sigma^*)$.

\textbf{But:} No proof for general trapped surfaces far from MOTS.

I believe counterexamples don't exist in ``generic'' spacetimes, but I cannot prove this.
\end{blueteam}

\begin{verdict}
\textbf{CRITICAL GAP CONFIRMED.}

Area dominance is not proven. It is:
\begin{itemize}
    \item TRUE in spherical symmetry
    \item TRUE perturbatively near MOTS
    \item BELIEVED true generically
    \item NOT PROVEN in general
\end{itemize}
\end{verdict}

%% ============================================================================
\section{Attack Round 4: WCC Formalization}
%% ============================================================================

\begin{redteam}
\textbf{Attack 4.1:} What EXACTLY is WCC? There are multiple formulations.

Which version do you use, and does it imply what you need?
\end{redteam}

\begin{blueteam}
\textbf{Defense 4.1:}

\textbf{WCC (Version used):}

For generic asymptotically flat vacuum (or matter satisfying energy conditions) initial data, the maximal globally hyperbolic development has:
\begin{enumerate}
    \item Complete future null infinity $\Scri^+$
    \item Event horizon $\mathcal{H}^+ = \partial J^-(\Scri^+)$ is an achronal, Lipschitz hypersurface
    \item $\mathcal{H}^+$ is a null hypersurface almost everywhere
\end{enumerate}

\textbf{What this gives us:}
\begin{itemize}
    \item Event horizon exists ✓
    \item Hawking Area Theorem applies ✓
    \item Final state is Kerr (additional assumption for exact bound) ✓
\end{itemize}

\textbf{What this does NOT give:}
\begin{itemize}
    \item Area dominance for trapped surfaces ✗
\end{itemize}

WCC is about the CAUSAL structure, not the METRIC properties of surfaces inside the black hole.
\end{blueteam}

\begin{verdict}
\textbf{DEFENDED but LIMITED.}

WCC is well-defined and provides what's claimed. But WCC alone does NOT imply area dominance.
\end{verdict}

%% ============================================================================
\section{Attack Round 5: Alternative Approaches}
%% ============================================================================

\begin{redteam}
\textbf{Attack 5.1:} Forget the MOTS route. Can you prove Penrose 1973 using the event horizon directly?

Chain: $\Area(\Sigma) \le \Area(\mathcal{H}^+ \cap \mathcal{C}) \le \Area(\mathcal{H}^+_{\text{final}}) \le 16\pi M^2$
\end{redteam}

\begin{blueteam}
\textbf{Defense 5.1:}

The event horizon chain has the SAME gap!

$\Area(\Sigma) \le \Area(\mathcal{H}^+ \cap \mathcal{C})$ is exactly the area dominance problem.

Being ``inside'' the event horizon (on a Cauchy surface) doesn't bound area.

\textbf{The gap is intrinsic to any approach.}
\end{blueteam}

\begin{redteam}
\textbf{Attack 5.2:} What about null methods? Shoot null geodesics from $\Sigma$ to $\mathcal{H}^+$.
\end{redteam}

\begin{blueteam}
\textbf{Defense 5.2:}

Null methods give the WRONG direction!

From trapped $\Sigma$, outgoing null has $\theta^+ < 0$.

By focusing: area DECREASES along outgoing null.

So: $\Area(\text{later cross-section}) < \Area(\Sigma)$.

The event horizon has $\theta \ge 0$, so null from $\Sigma$ doesn't directly reach $\mathcal{H}^+$ — the geodesics hit the singularity first!

Null methods cannot connect $\Sigma$ to $\mathcal{H}^+$ with area comparison.
\end{blueteam}

\begin{verdict}
\textbf{NO ALTERNATIVE WORKS.}

All approaches hit the same area dominance gap.
\end{verdict}

%% ============================================================================
\section{Attack Round 6: The Trapped Condition}
%% ============================================================================

\begin{redteam}
\textbf{Attack 6.1:} Does the trapped condition $\theta^+ < 0$ give ANY area bound?

$\theta^+ = H + P$ where $H$ = mean curvature, $P = \tr_\Sigma k$.

Can you extract information from $H + P < 0$?
\end{redteam}

\begin{blueteam}
\textbf{Defense 6.1:}

\textbf{What $\theta^+ < 0$ gives:}

On a maximal slice ($\tr k = 0$), $P = \tr_\Sigma k$ can still be nonzero.

For $\theta^+ < 0$: $H < -P$.

If we have bounds on $k$ (say $|k| \le K_0$), then $|P| \le 2K_0$, so $H < 2K_0$.

\textbf{Mean curvature bound:}

$H < C$ for some $C$ doesn't directly bound area!

A surface with bounded mean curvature can still have large area (think of a long thin tube).

\textbf{What might help:}

If $\Sigma$ is TOPOLOGICALLY a sphere, and $H < C$, and $\Sigma$ encloses a bounded region, then isoperimetric inequalities might apply.

But the trapped region can have complicated geometry.

\textbf{Conclusion:} The trapped condition constrains local geometry but doesn't obviously bound global area.
\end{blueteam}

\begin{verdict}
\textbf{NO CLEAR PATH.}

The trapped condition alone doesn't give area bounds.
\end{verdict}

%% ============================================================================
\section{Final Assessment}
%% ============================================================================

\begin{fatal}
\textbf{FINAL STATUS:}

\textbf{Penrose 1973 with WCC is NOT FULLY PROVEN.}

\textbf{What IS proven:}
\begin{enumerate}
    \item MOTS existence (Andersson-Metzger) ✓
    \item MOTS Penrose inequality (Jang + RPI, requires DEC) ✓
    \item Hawking Area Theorem (NEC + WCC) ✓
    \item Trapped $\Rightarrow$ inside black hole (Penrose 1965) ✓
\end{enumerate}

\textbf{What is NOT proven:}
\begin{enumerate}
    \item Area dominance: $\Area(\Sigma) \le \Area(\Sigma^*)$ ✗
\end{enumerate}

\textbf{The claimed theorem:}
\[
\text{DEC} + \text{WCC} \Rightarrow M_{\ADM} \ge \sqrt{\Area(\Sigma)/(16\pi)}
\]

\textbf{Is TRUE if and only if AREA DOMINANCE holds.}

Area dominance is:
\begin{itemize}
    \item TRUE in spherical symmetry
    \item TRUE perturbatively (near MOTS)
    \item CONJECTURED true generically
    \item \textbf{NOT PROVEN}
\end{itemize}
\end{fatal}

%% ============================================================================
\section{Honest Conclusion}
%% ============================================================================

\begin{theorem}[What We Actually Have]
\textbf{Theorem A (PROVEN):} Under DEC, for outermost MOTS $\Sigma^*$:
\[
M_{\ADM} \ge \sqrt{\frac{\Area(\Sigma^*)}{16\pi}}
\]

\textbf{Theorem B (PROVEN):} Under NEC + WCC, trapped surfaces are inside the black hole region.

\textbf{Conjecture (NOT PROVEN):} Under DEC + WCC, for trapped $\Sigma$:
\[
M_{\ADM} \ge \sqrt{\frac{\Area(\Sigma)}{16\pi}}
\]

This conjecture reduces to: $\Area(\Sigma) \le \Area(\Sigma^*)$ (area dominance).
\end{theorem}

\begin{center}
\fbox{\parbox{0.9\textwidth}{
\textbf{BOTTOM LINE:}

We have NOT proven Penrose 1973, even with WCC.

We have REDUCED it to a single geometric inequality (area dominance).

This inequality is believed true but not proven.

\textbf{Claiming ``proof complete'' would be dishonest.}
}}
\end{center}

%% ============================================================================
\section{What Would Complete the Proof}
%% ============================================================================

\textbf{To finish Penrose 1973, prove:}

\begin{theorem}[Area Dominance Conjecture]
Let $(\mathcal{C}, g, k)$ be asymptotically flat initial data satisfying DEC. Let $\Sigma^*$ be the outermost MOTS. Let $\Sigma$ be any closed trapped surface ($\theta^+ < 0$) with $\Sigma$ inside the region bounded by $\Sigma^*$.

Then: $\Area(\Sigma) \le \Area(\Sigma^*)$.
\end{theorem}

\textbf{This is a well-posed RIEMANNIAN GEOMETRY problem.}

No spacetime evolution needed — it's about surfaces on a single Cauchy slice.

Proving this would complete Penrose 1973.

\end{document}
