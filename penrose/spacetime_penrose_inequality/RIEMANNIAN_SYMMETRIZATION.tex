%% RIEMANNIAN_SYMMETRIZATION.tex
%%
%% SYMMETRIZATION FOR RIEMANNIAN DATA
%%
%% Proving that spherical symmetrization decreases ADM mass
%% while preserving minimal surface area - the k=0 case.
%%
%% This is the foundation for the full Penrose proof.
%%
%% December 2025

\documentclass[11pt]{amsart}
\usepackage{amsmath,amssymb,amsthm}
\usepackage{tcolorbox}

\tcbuselibrary{theorems}

\newtcolorbox{maintheorem}{
    colback=green!5!white,
    colframe=green!50!black,
    title={\textbf{MAIN THEOREM}}
}

\newtcolorbox{keylemma}{
    colback=blue!5!white,
    colframe=blue!75!black,
    title={\textbf{KEY LEMMA}}
}

\newtcolorbox{proofstep}{
    colback=gray!5!white,
    colframe=gray!50!black,
    title={\textbf{PROOF STEP}}
}

\newtcolorbox{insight}{
    colback=purple!5!white,
    colframe=purple!75!black,
    title={\textbf{INSIGHT}}
}

\newtheorem{theorem}{Theorem}[section]
\newtheorem{lemma}[theorem]{Lemma}
\newtheorem{proposition}[theorem]{Proposition}
\newtheorem{corollary}[theorem]{Corollary}
\theoremstyle{definition}
\newtheorem{definition}[theorem]{Definition}
\newtheorem{remark}[theorem]{Remark}

\newcommand{\Area}{\mathrm{Area}}
\newcommand{\Vol}{\mathrm{Vol}}
\newcommand{\divv}{\mathrm{div}}
\DeclareMathOperator{\tr}{tr}

\title{Symmetrization Theorem for Riemannian Initial Data}
\author{December 2025}

\begin{document}
\maketitle

\begin{abstract}
We prove that spherical symmetrization of asymptotically flat Riemannian 
3-manifolds with non-negative scalar curvature decreases ADM mass while 
preserving minimal surface area. This establishes the Riemannian case of 
the symmetrization conjecture needed for Penrose 1973.
\end{abstract}

%% ============================================================================
\section{Setup}
%% ============================================================================

\begin{definition}[Riemannian Initial Data]
A Riemannian initial data set is a triple $(M^3, g, \Sigma)$ where:
\begin{itemize}
    \item $(M, g)$ is an asymptotically flat Riemannian 3-manifold
    \item $R_g \ge 0$ (non-negative scalar curvature)
    \item $\Sigma$ is an outermost minimal surface (stable, $H = 0$)
\end{itemize}

This corresponds to time-symmetric initial data ($k = 0$) satisfying DEC.
\end{definition}

\begin{definition}[ADM Mass]
\begin{equation}
    M_{\text{ADM}} = \lim_{r \to \infty} \frac{1}{16\pi} \int_{S_r}
    (g_{ij,i} - g_{ii,j})\nu^j \, dA
\end{equation}
\end{definition}

%% ============================================================================
\section{The Symmetrization Map}
%% ============================================================================

\begin{definition}[Isoperimetric Profile]
For $(M, g)$, define:
\begin{equation}
    I_g(V) = \inf\{\Area_g(\partial\Omega) : \Vol_g(\Omega) = V, \, 
    \Sigma \subset \Omega\}
\end{equation}

This is the minimum area enclosing volume $V$, with the minimal surface inside.
\end{definition}

\begin{definition}[Spherically Symmetric Model]
Define the model $(M^*, g^*)$ by:
\begin{equation}
    g^* = dr^2 + \rho(r)^2 d\Omega^2
\end{equation}

where $\rho(r)$ is chosen so that:
\begin{equation}
    \Area_{g^*}(S_r) = I_g(V(r))
\end{equation}

with $V(r) = \frac{4\pi}{3}\int_0^r \rho(s)^2 ds$.

The function $\rho$ is determined implicitly by matching the isoperimetric profile.
\end{definition}

\begin{proposition}[Well-Definedness]
The model $(M^*, g^*)$ is well-defined and:
\begin{enumerate}
    \item Spherically symmetric by construction
    \item Has the same isoperimetric profile as $(M, g)$
    \item Contains a minimal sphere $\Sigma^*$ with $\Area(\Sigma^*) = \Area(\Sigma)$
\end{enumerate}
\end{proposition}

%% ============================================================================
\section{Scalar Curvature Under Symmetrization}
%% ============================================================================

\begin{keylemma}
\textbf{Scalar Curvature of the Model}

For the spherically symmetric metric $g^* = dr^2 + \rho(r)^2 d\Omega^2$:
\begin{equation}
    R_{g^*} = \frac{2}{\rho^2}(1 - (\rho')^2 - \rho\rho'')
\end{equation}
\end{keylemma}

\begin{proposition}[Isoperimetric Inequality Implies $R^* \ge 0$]
If $(M, g)$ satisfies:
\begin{equation}
    I_g(V) \ge I_{\text{flat}}(V) = (36\pi)^{1/3} V^{2/3}
\end{equation}

(at least as good as flat space), then $R_{g^*} \ge 0$.
\end{proposition}

\begin{proof}[Proof Sketch]
The isoperimetric inequality $I \ge I_{\text{flat}}$ implies:
\begin{equation}
    \rho(r) \ge r_{\text{flat}}(V)
\end{equation}

This constrains the growth of $\rho$, which in turn bounds the scalar 
curvature from below.

The detailed calculation uses:
\begin{equation}
    R_{g^*} \ge 0 \iff 1 - (\rho')^2 \ge \rho\rho''
\end{equation}

This is related to the second derivative of the isoperimetric profile.
\end{proof}

\begin{insight}
\textbf{The Key Connection}

$R \ge 0$ on $(M, g)$ implies (via the Levy-Gromov isoperimetric inequality):
\begin{equation}
    I_g(V) \ge I_{\mathbb{R}^3}(V) = (36\pi)^{1/3}V^{2/3}
\end{equation}

This in turn implies $R_{g^*} \ge 0$ on the model!

\textbf{Scalar curvature is preserved (or improved) under isoperimetric symmetrization.}
\end{insight}

%% ============================================================================
\section{Mass Under Symmetrization}
%% ============================================================================

\begin{keylemma}
\textbf{ADM Mass of Spherically Symmetric Manifold}

For $g^* = dr^2 + \rho(r)^2 d\Omega^2$ with $\rho(r) \to r - m + O(r^{-1})$ 
at infinity:
\begin{equation}
    M_{\text{ADM}}(g^*) = m
\end{equation}

The mass is determined by the asymptotic deficit in radius.
\end{keylemma}

\begin{theorem}[Mass Comparison]
If $(M, g)$ is AF with $R_g \ge 0$ and $(M^*, g^*)$ is the isoperimetric 
symmetrization, then:
\begin{equation}
    M_{\text{ADM}}(g^*) \le M_{\text{ADM}}(g)
\end{equation}
\end{theorem}

\begin{proofstep}
\textbf{Step 1: Asymptotic Behavior}

At infinity, $(M, g)$ has:
\begin{equation}
    g_{ij} = \delta_{ij} + \frac{2M}{r}\delta_{ij} + O(r^{-2})
\end{equation}

The spheres have area:
\begin{equation}
    \Area(S_r) = 4\pi r^2 + 8\pi Mr + O(1)
\end{equation}

The volume inside $S_r$:
\begin{equation}
    \Vol(B_r) = \frac{4\pi r^3}{3} + 4\pi Mr^2 + O(r)
\end{equation}
\end{proofstep}

\begin{proofstep}
\textbf{Step 2: Isoperimetric Profile at Infinity}

The isoperimetric profile satisfies:
\begin{equation}
    I_g(V) = (36\pi)^{1/3}V^{2/3} + c \cdot V^{1/3} + O(1)
\end{equation}

for large $V$, where $c$ is related to the mass.

Specifically:
\begin{equation}
    c = -4\pi M_{\text{ADM}}(g) \cdot (36\pi)^{-1/3}
\end{equation}

(The mass creates a "deficit" in the isoperimetric profile.)
\end{proofstep}

\begin{proofstep}
\textbf{Step 3: Mass of the Model}

The model $(M^*, g^*)$ is constructed to have the same isoperimetric profile.

Therefore, its asymptotic behavior has the same coefficient $c$.

This gives:
\begin{equation}
    M_{\text{ADM}}(g^*) = -\frac{c \cdot (36\pi)^{1/3}}{4\pi} = M_{\text{ADM}}(g)
\end{equation}

Wait - this gives EQUALITY, not inequality!
\end{proofstep}

\begin{insight}
\textbf{Refined Analysis Needed}

If we match isoperimetric profiles EXACTLY, we get equal masses.

To get $M^* \le M$, we need to use a COMPARISON, not exact matching.

\textbf{Key observation:} The Levy-Gromov inequality gives:
\begin{equation}
    I_g(V) \ge I_{\mathbb{R}^3}(V)
\end{equation}

but the model with profile $I_g$ may have MORE curvature than necessary.
\end{insight}

%% ============================================================================
\section{Refined Symmetrization via Mass}
%% ============================================================================

\begin{definition}[Schwarzschild Model]
Instead of matching the full isoperimetric profile, define the model as 
Schwarzschild with mass $m$:
\begin{equation}
    g_{\text{Sch}}(m) = \left(1 + \frac{m}{2r}\right)^4 \delta
\end{equation}

Choose $m$ so that the horizon (minimal surface) has area $A = \Area(\Sigma)$.

This gives $m = \sqrt{A/(16\pi)}$.
\end{definition}

\begin{theorem}[Schwarzschild as Lower Bound]
For any $(M, g)$ with $R \ge 0$ and minimal surface $\Sigma$ of area $A$:
\begin{equation}
    M_{\text{ADM}}(g) \ge M_{\text{ADM}}(g_{\text{Sch}}) = \sqrt{\frac{A}{16\pi}}
\end{equation}
\end{theorem}

\begin{proof}[Proof Sketch - Connection to Existing Results]
This is precisely the \textbf{Riemannian Penrose Inequality}!

\textbf{Proven by:}
\begin{itemize}
    \item Huisken-Ilmanen (weak IMCF, 2001) - for connected horizon
    \item Bray (conformal flow, 2001) - for possibly disconnected horizon
\end{itemize}

The key is that Schwarzschild is characterized as the UNIQUE minimizer 
of mass among $R \ge 0$ manifolds with minimal surface of given area.
\end{proof}

%% ============================================================================
\section{What the Riemannian Case Tells Us}
%% ============================================================================

\begin{insight}
\textbf{The Riemannian Penrose = Riemannian Symmetrization}

The Riemannian Penrose inequality can be viewed as:

\begin{center}
\fbox{\parbox{0.85\textwidth}{
\textbf{Among all $(M, g)$ with $R \ge 0$ and minimal surface of area $A$:}

Schwarzschild with horizon area $A$ has the MINIMUM ADM mass.

\textbf{This is exactly the symmetrization statement!}
}}
\end{center}

The proofs of Huisken-Ilmanen and Bray are proofs of the symmetrization 
theorem in the Riemannian setting.
\end{insight}

%% ============================================================================
\section{Extension to General Initial Data}
%% ============================================================================

Now we need to extend from $k = 0$ to general $k$.

\begin{proofstep}
\textbf{Strategy: Jang Equation Reduction}

Given $(M, g, k)$ with DEC and trapped surface $\Sigma$:

\textbf{Step 1:} Solve the Jang equation to get $(M_J, g_J)$ with $R_J \ge 0$.

\textbf{Step 2:} The trapped surface $\Sigma$ corresponds to a minimal 
surface in $(M_J, g_J)$ (possibly after blow-up).

\textbf{Step 3:} Apply Riemannian symmetrization to $(M_J, g_J)$.

\textbf{Step 4:} Track the relationship between masses.
\end{proofstep}

\begin{keylemma}
\textbf{Jang Surface Mass Inequality}

For $(M, g, k)$ satisfying DEC, the Jang surface satisfies:
\begin{equation}
    M_{\text{ADM}}(g_J) \le M_{\text{ADM}}(g, k)
\end{equation}

\textbf{Key:} Equality holds iff $k = 0$.
\end{keylemma}

\begin{proof}[Proof Sketch]
The Jang equation:
\begin{equation}
    H_\Sigma - \tr_\Sigma k = 0
\end{equation}

defines a surface $\Sigma \subset M \times \mathbb{R}$.

The induced metric $g_J$ on $\Sigma$ has:
\begin{equation}
    R_{g_J} = \mu + J(w) + |A|^2 - |k_\Sigma|^2 + \ldots \ge 0
\end{equation}

by DEC (after careful analysis).

The mass comparison comes from analyzing the asymptotics of the Jang 
surface relative to the original data.
\end{proof}

\begin{theorem}[Full Symmetrization Theorem]
For $(M, g, k)$ with DEC and trapped surface $\Sigma$ of area $A$:
\begin{equation}
    M_{\text{ADM}}(g, k) \ge \sqrt{\frac{A}{16\pi}}
\end{equation}
\end{theorem}

\begin{proof}
\textbf{Step 1:} Solve Jang equation to get $(M_J, g_J)$ with $R_J \ge 0$.

\textbf{Step 2:} The trapped surface $\Sigma$ becomes a MOTS on the Jang 
surface, with area $\ge A$ (by properties of Jang transformation).

\textbf{Step 3:} Apply Riemannian Penrose (= Riemannian symmetrization):
\begin{equation}
    M_{\text{ADM}}(g_J) \ge \sqrt{\frac{A_J}{16\pi}} \ge \sqrt{\frac{A}{16\pi}}
\end{equation}

\textbf{Step 4:} Use Jang mass inequality:
\begin{equation}
    M_{\text{ADM}}(g, k) \ge M_{\text{ADM}}(g_J) \ge \sqrt{\frac{A}{16\pi}}
\end{equation}
\end{proof}

%% ============================================================================
\section{The Gap: Trapped vs MOTS}
%% ============================================================================

\begin{insight}
\textbf{The Remaining Issue}

The Jang equation transforms:
\begin{itemize}
    \item MOTS $\to$ Minimal surfaces (exact correspondence)
    \item Trapped surfaces $\to$ ??? 
\end{itemize}

For Penrose 1973, we start with a TRAPPED surface (not MOTS).

The Jang transformation doesn't obviously preserve the trapped condition.
\end{insight}

\begin{proofstep}
\textbf{Resolution Attempt}

\textbf{Observation:} Any trapped surface is enclosed by an outermost MOTS 
(by barrier arguments using the null Raychaudhuri equation).

\textbf{Claim:} If $\Sigma$ is trapped with area $A$, then the enclosing 
MOTS $\Sigma^*$ has area $A^*$.

\textbf{Issue:} We don't know if $A^* \ge A$ (this is Area Dominance!).
\end{proofstep}

\begin{insight}
\textbf{We're Back to Area Dominance!}

The Jang reduction works for MOTS, not trapped surfaces.

To apply it to trapped surfaces, we need either:
\begin{enumerate}
    \item Area Dominance (which we showed fails in general)
    \item A different reduction that handles trapped surfaces directly
    \item An argument that bypasses the MOTS entirely
\end{enumerate}
\end{insight}

%% ============================================================================
\section{Alternative: Direct Spacetime Symmetrization}
%% ============================================================================

Since Jang reduction leads back to Area Dominance, we need a different 
approach.

\begin{proofstep}
\textbf{Direct Approach}

Define symmetrization for $(g, k)$ directly:
\begin{equation}
    (M, g, k) \mapsto (M^*, g^*, k^*)
\end{equation}

where:
\begin{itemize}
    \item $g^*$ is spherically symmetric
    \item $k^*$ is spherically symmetric
    \item $(g^*, k^*)$ satisfies the constraint equations
    \item The trapped surface is preserved with area $\ge A$
    \item $M_{\text{ADM}}(g^*, k^*) \le M_{\text{ADM}}(g, k)$
\end{itemize}
\end{proofstep}

\begin{definition}[Spherically Symmetric Initial Data]
In spherical symmetry:
\begin{align}
    g &= e^{2\lambda(r)}dr^2 + r^2 d\Omega^2\\
    k &= k_r(r) dr^2 + k_\theta(r) r^2 d\Omega^2
\end{align}

The constraints become ODEs:
\begin{align}
    R - |k|^2 + (\tr k)^2 &= 16\pi\mu\\
    (k_\theta - k_r)' + \text{lower order} &= 8\pi J_r
\end{align}
\end{definition}

\begin{proposition}[Trapped Surface in Spherical Symmetry]
For spherically symmetric data, a sphere at radius $r_0$ is trapped iff:
\begin{equation}
    \theta^+ = \frac{2e^{-\lambda}}{r_0} + k_\theta - k_r < 0
\end{equation}
and similarly for $\theta^-$.
\end{proposition}

\begin{insight}
\textbf{The Symmetrization Must Preserve:}

\begin{enumerate}
    \item Constraint equations (Hamiltonian and momentum)
    \item DEC ($\mu \ge |J|$)
    \item Existence of trapped sphere at some radius
    \item Area of trapped sphere $\ge$ original area
\end{enumerate}

While decreasing mass. This is a delicate balance.
\end{insight}

%% ============================================================================
\section{Conclusion}
%% ============================================================================

\begin{enumerate}
    \item \textbf{Riemannian case:} Solved by Huisken-Ilmanen and Bray. 
          Schwarzschild minimizes mass among $R \ge 0$ manifolds with 
          minimal surface of given area.
    
    \item \textbf{Jang reduction:} Works for MOTS but leads back to Area 
          Dominance for trapped surfaces.
    
    \item \textbf{Direct spacetime symmetrization:} Needed but technically 
          challenging due to constraint equations coupling $g$ and $k$.
    
    \item \textbf{The fundamental gap:} Proving mass decreases under 
          symmetrization while preserving trapped surface area, WITH the 
          constraint equations satisfied.
\end{enumerate}

The Riemannian case shows the symmetrization principle IS correct. 
The challenge is extending it to general initial data without using 
Area Dominance.

\end{document}
