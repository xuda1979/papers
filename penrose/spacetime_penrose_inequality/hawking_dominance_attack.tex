% ATTACKING THE HAWKING MASS DOMINANCE LEMMA
%
% The key missing piece: Show that sup M_H >= sqrt(A_0/16pi)
% for any trapped surface Sigma_0 in the trapped region.

\documentclass[12pt]{article}
\usepackage{amsmath,amsthm,amssymb}
\usepackage{mathrsfs}
\newtheorem{theorem}{Theorem}
\newtheorem{lemma}{Lemma}
\newtheorem{proposition}{Proposition}
\newtheorem{corollary}{Corollary}
\newtheorem{conjecture}{Conjecture}
\newtheorem{remark}{Remark}
\newtheorem{definition}{Definition}
\newtheorem{problem}{Problem}
\newtheorem{claim}{Claim}

\begin{document}

\title{Attacking the Hawking Mass Dominance Lemma}
\author{Mathematical Development}
\date{\today}
\maketitle

\section{The Target Lemma}

\begin{lemma}[Hawking Mass Dominance]
Let $\mathcal{T}$ be the trapped region in initial data $(M, g, k)$ satisfying DEC.
For any trapped surface $\Sigma_0 \subset \mathcal{T}$:
\[
\sup_{\Sigma \subset \mathcal{T}, \text{trapped}} M_H[\Sigma] \ge \sqrt{\frac{A(\Sigma_0)}{16\pi}}
\]
\end{lemma}

\section{Why This is Non-Trivial}

The Hawking mass is:
\[
M_H[\Sigma] = \sqrt{\frac{A}{16\pi}}\left(1 - \frac{1}{16\pi}\int_\Sigma H^2 \, dA\right)
\]

For a trapped surface, we have $\theta^+ \le 0$, $\theta^- < 0$, giving:
\[
H = \frac{1}{2}(\theta^+ + \theta^-) < 0 \quad \text{(in the interior)}
\]

So $H \ne 0$ in general, and $M_H < M_P = \sqrt{A/(16\pi)}$.

The lemma says: even though each individual surface has $M_H < M_P$, the SUPREMUM 
over all trapped surfaces still dominates the Penrose mass of any particular $\Sigma_0$.

\section{Approach 1: Direct Comparison}

\subsection{Setup}

Let $\Sigma_0$ be a trapped surface with:
\begin{itemize}
    \item Area $A_0$
    \item Mean curvature $H_0$
    \item $\tr_{\Sigma_0} k = \kappa_0$ (could be negative)
\end{itemize}

The Hawking mass of $\Sigma_0$:
\[
M_H[\Sigma_0] = \sqrt{\frac{A_0}{16\pi}}\left(1 - \frac{\langle H_0^2 \rangle A_0}{16\pi}\right)
\]

\subsection{When is $M_H[\Sigma_0] \ge \sqrt{A_0/(16\pi)}$?}

This requires:
\[
1 - \frac{\langle H_0^2 \rangle A_0}{16\pi} \ge 1
\]
which gives $\langle H_0^2 \rangle \le 0$, impossible unless $H_0 = 0$.

So for trapped surfaces (with $H \ne 0$), we ALWAYS have $M_H < M_P$.

\textbf{The supremum must come from a DIFFERENT surface, not $\Sigma_0$ itself!}

\subsection{Can We Find a Better Surface?}

We need $\Sigma$ with:
\[
M_H[\Sigma] = \sqrt{\frac{A_\Sigma}{16\pi}}\left(1 - \frac{\langle H_\Sigma^2 \rangle A_\Sigma}{16\pi}\right) \ge \sqrt{\frac{A_0}{16\pi}}
\]

This is equivalent to:
\[
A_\Sigma \left(1 - \frac{\langle H_\Sigma^2 \rangle A_\Sigma}{16\pi}\right)^2 \ge A_0
\]

\textbf{Case 1}: $A_\Sigma > A_0$ significantly, with $H_\Sigma$ not too large.

\textbf{Case 2}: $A_\Sigma \approx A_0$ but $|H_\Sigma| \ll |H_0|$ (nearly minimal).

\section{Approach 2: Constructing a Nearly Minimal Trapped Surface}

\subsection{The Idea}

Find a trapped surface $\Sigma$ with $H \approx 0$ (close to minimal) inside $\mathcal{T}$.

If such a surface exists with $A(\Sigma) \ge A_0$, we're done.

\subsection{Existence of Nearly Minimal Trapped Surfaces}

\begin{question}
Given the trapped region $\mathcal{T}$, does there exist a trapped surface 
$\Sigma \subset \mathcal{T}$ with arbitrarily small $\int_\Sigma H^2 \, dA$?
\end{question}

\textbf{Analysis}:

For $H$ small, we need $\theta^+ \approx -\theta^- \approx -\tr_\Sigma k$.

The trapped conditions become:
\begin{align}
    \theta^+ &= H + \tr k \le 0 \Rightarrow H \le -\tr k \\
    \theta^- &= H - \tr k < 0 \Rightarrow H < \tr k
\end{align}

For both to hold: $-|\tr k| < H < |\tr k|$ when $\tr k > 0$, or $H < -|\tr k|$ when $\tr k < 0$.

If $\tr k > 0$: We can have $H$ anywhere in $(-\tr k, \tr k)$, including $H = 0$.
\textbf{So in the favorable case, minimal trapped surfaces can exist!}

If $\tr k < 0$: We need $H < \tr k < 0$, so $H < 0$ necessarily.
The smallest $|H|$ is achieved when $H \to (\tr k)^-$.

\textbf{In the unfavorable case, $H$ cannot be zero!}

\subsection{The Minimum Mean Curvature in Unfavorable Case}

For a surface with $\tr_\Sigma k = -\kappa < 0$ (unfavorable):
\[
|H| > \kappa = |\tr_\Sigma k|
\]

The mean curvature is BOUNDED BELOW by the trace of $k$!

\section{Approach 3: Area Growth Argument}

\subsection{The Idea}

Even if $|H|$ can't be small, maybe we can find surfaces with LARGE area to compensate.

We need: $A\left(1 - \frac{\langle H^2 \rangle A}{16\pi}\right)^2 \ge A_0$.

Let $\alpha = \frac{\langle H^2 \rangle A}{16\pi}$. Then we need $A(1-\alpha)^2 \ge A_0$.

If $\alpha$ is fixed (determined by the trapped condition), we need $A \ge \frac{A_0}{(1-\alpha)^2}$.

\subsection{How Large Can $A$ Get?}

The outermost MOTS $\Sigma^*$ has area $A^*$. All trapped surfaces are inside $\Sigma^*$.

\textbf{But}: Can trapped surfaces inside $\Sigma^*$ have area $> A^*$?

Yes! If the trapped region has "tentacles" or is non-convex, interior surfaces 
can bulge into the concavities.

\subsection{Maximum Area Trapped Surface}

\begin{definition}
$A_{\max} = \sup\{A(\Sigma) : \Sigma \subset \mathcal{T}, \text{trapped}\}$
\end{definition}

\textbf{Questions}:
\begin{enumerate}
    \item Is $A_{\max}$ finite?
    \item Is $A_{\max}$ achieved?
    \item What is the relationship between $A_{\max}$ and $A^*$?
\end{enumerate}

\subsection{Finiteness of $A_{\max}$}

Since $\mathcal{T}$ is bounded (contained in an asymptotically flat end), any 
surface $\Sigma \subset \mathcal{T}$ has bounded area.

So $A_{\max} < \infty$. $\checkmark$

\subsection{Achievability}

By compactness of the space of surfaces (in appropriate topology), the supremum 
should be achieved by some surface $\Sigma_{\max}$.

$\Sigma_{\max}$ might be:
\begin{itemize}
    \item Smooth and strictly trapped
    \item A MOTS (marginally trapped)
    \item A singular/weak limit
\end{itemize}

\section{Approach 4: Variational Analysis of Maximum Area}

\subsection{The Constrained Problem}

Maximize $A(\Sigma)$ subject to $\theta^+[\Sigma] \le 0$, $\theta^-[\Sigma] < 0$.

\subsection{First Variation}

Under normal variation $\phi\nu$:
\[
\delta_\phi A = \int_\Sigma \phi H \, dA
\]

For an interior maximizer: $\delta_\phi A = 0$ for all $\phi$, giving $H = 0$ (minimal).

But a minimal surface with $\theta^+ \le 0$ and $\theta^- < 0$ requires:
\begin{align}
    H + \tr k &\le 0 \Rightarrow \tr k \le 0 \\
    H - \tr k &< 0 \Rightarrow \tr k > 0
\end{align}

This is impossible! So the maximizer CANNOT be an interior critical point.

\textbf{The maximizer must saturate a constraint}: either $\theta^+ = 0$ (MOTS) or be at a boundary.

\subsection{Conclusion from Variational Analysis}

\begin{claim}
The maximum area trapped surface is the outermost MOTS $\Sigma^*$.
\end{claim}

\textbf{Wait}: This contradicts the earlier observation that inner surfaces can 
have larger area in the unfavorable case!

Let me reconsider...

\subsection{The Resolution}

The variational analysis shows that among SMOOTH trapped surfaces, the maximizer 
satisfies $H = 0$ at interior points.

But in the unfavorable case, $H \ne 0$ for trapped surfaces.

So either:
\begin{enumerate}
    \item The maximizer is the MOTS (constraint active)
    \item The maximizer is a singular surface (limiting case)
\end{enumerate}

\textbf{Key insight}: In the unfavorable case, the "large area" trapped surfaces 
might be MOTS-like (marginally trapped), not strictly trapped!

\section{Approach 5: The Interpolation Argument}

\subsection{Setup}

Consider a family of surfaces $\Sigma_t$ for $t \in [0, 1]$ with:
\begin{itemize}
    \item $\Sigma_0 = \Sigma_0$ (the given trapped surface)
    \item $\Sigma_1 = \Sigma^*$ (the outermost MOTS)
    \item $\Sigma_t$ interpolates "outward" from $\Sigma_0$ to $\Sigma^*$
\end{itemize}

\subsection{Track Hawking Mass}

$M_H[\Sigma_t]$ varies as $t$ goes from $0$ to $1$.

At $t = 0$: $M_H[\Sigma_0] < M_P[\Sigma_0]$ (since $H_0 \ne 0$).

At $t = 1$: $M_H[\Sigma^*] = M_P[\Sigma^*](1 - \epsilon)$ for some $\epsilon > 0$.

\subsection{Can $M_H$ Achieve $M_P[\Sigma_0]$ Along the Path?}

We need: $\sqrt{A_t}\left(1 - \frac{\langle H_t^2 \rangle A_t}{16\pi}\right) \ge \sqrt{A_0}$ for some $t$.

Define $f(t) = A_t^{1/2}\left(1 - \frac{\langle H_t^2 \rangle A_t}{16\pi}\right)$.

We have:
\begin{itemize}
    \item $f(0) < \sqrt{A_0}$ (interior of MOTS has reduced Hawking mass)
    \item $f(1) = \sqrt{A^*}\left(1 - \frac{(H^*)^2 A^*}{16\pi}\right)$
\end{itemize}

\textbf{The question}: Is $\max_t f(t) \ge \sqrt{A_0}$?

\subsection{Continuity Argument}

If $f$ is continuous and $f(0) < \sqrt{A_0}$ but we believe $\max f \ge \sqrt{A_0}$, 
then there must be some $t^* > 0$ where $f(t^*) \ge \sqrt{A_0}$.

But why should $\max f \ge \sqrt{A_0}$ be true?

\section{Approach 6: Energy Argument}

\subsection{The Physical Picture}

The ADM mass is the total energy. The trapped surface "contains" a black hole 
worth of energy.

\textbf{Intuition}: The energy "enclosed" by $\Sigma_0$ should be at least $M_P[\Sigma_0]$.

\subsection{Quasi-Local Energy}

Various quasi-local energies have been defined:
\begin{itemize}
    \item Hawking mass $M_H$
    \item Brown-York energy
    \item Wang-Yau energy
    \item etc.
\end{itemize}

Maybe one of these satisfies: $E[\Sigma_0] \ge M_P[\Sigma_0]$ for trapped surfaces?

\subsection{The Wang-Yau Mass}

The Wang-Yau mass $E_{WY}[\Sigma, X, T]$ depends on an isometric embedding $X$ 
into $\mathbb{R}^{3,1}$ and a time function $T$.

For optimal choice of $T$:
\[
E_{WY}[\Sigma] = \frac{1}{8\pi}\int_\Sigma (H_0 - |H|) \, dA
\]
where $H_0$ is the mean curvature of $X(\Sigma)$ in $\mathbb{R}^3$.

\textbf{Properties}:
\begin{itemize}
    \item $E_{WY} \ge 0$ if $\Sigma$ has positive Gauss curvature
    \item $E_{WY}$ reduces to ADM mass at infinity
\end{itemize}

\textbf{For trapped surfaces}: The analysis is unclear because $|H|$ might be large.

\section{A New Idea: The Weighted Hawking Mass}

\subsection{Definition}

\begin{definition}[Weighted Hawking Mass]
\[
M_H^\omega[\Sigma] = \sqrt{\frac{A}{16\pi}}\left(1 - \frac{1}{16\pi}\int_\Sigma \omega H^2 \, dA\right)
\]
where $\omega: \Sigma \to [0, 1]$ is a weight function.
\end{definition}

With $\omega = 1$, this is the standard Hawking mass.
With $\omega = 0$, this gives $M_H^0 = M_P = \sqrt{A/(16\pi)}$.

\subsection{Optimization Over Weights}

\begin{claim}
If we can choose $\omega$ appropriately for each surface, we might get a better mass.
\end{claim}

The issue is that $\omega$ should be intrinsic to $\Sigma$, not arbitrary.

\subsection{A Natural Choice}

Let $\omega = \frac{|\theta^-|}{|\theta^+| + |\theta^-|}$ (measures "how trapped").

For strongly trapped surfaces (both $|\theta^\pm|$ large): $\omega \approx 1/2$.
For MOTS ($\theta^+ = 0$): $\omega = 1$.

This gives more weight to the MOTS-like part of trapping.

\section{Summary: The Gap Remains}

We have NOT proven the Hawking Mass Dominance Lemma.

\textbf{The core difficulty}: In the unfavorable case, the mean curvature $H$ is 
bounded away from zero, so $M_H < M_P$ always.

\textbf{Possible resolutions}:
\begin{enumerate}
    \item The lemma is TRUE but requires a sophisticated comparison argument.
    \item The lemma is FALSE, and the Penrose inequality needs modification.
    \item A different quasi-local mass (not Hawking) satisfies the dominance.
\end{enumerate}

\section{Counterexample Attempt}

\subsection{Setup}

Try to construct initial data where:
\begin{itemize}
    \item $\Sigma_0$ is trapped with area $A_0$
    \item The outermost MOTS $\Sigma^*$ has $M_H[\Sigma^*] < \sqrt{A_0/(16\pi)}$
    \item No trapped surface has Hawking mass $\ge \sqrt{A_0/(16\pi)}$
\end{itemize}

If such data exists, the lemma is FALSE.

\subsection{A Candidate}

Take Schwarzschild initial data and perturb $k$ to create an "unfavorable" situation.

Original: $k = 0$ (time-symmetric), $\Sigma^* = $ the throat, $H^* = 0$.

Perturbed: $k \ne 0$ with $\tr_{\Sigma_0} k < 0$ for some interior surface $\Sigma_0$.

The MOTS might shift and have $H^* \ne 0$, reducing $M_H[\Sigma^*]$.

\textbf{Question}: Can we make $M_H[\Sigma^*] < M_P[\Sigma_0]$?

This requires: $\sqrt{A^*}(1 - \epsilon^*) < \sqrt{A_0}$ where $\epsilon^* = \frac{H^{*2}A^*}{16\pi}$.

If $A^* < A_0$ (inner surface is larger) AND $\epsilon^*$ is not too small, this could happen!

\section{Conclusion}

The Hawking Mass Dominance Lemma is the KEY to removing the favorable jump condition.

Current status:
\begin{itemize}
    \item NOT PROVEN
    \item No counterexample found either
    \item Most likely requires new ideas
\end{itemize}

\textbf{Next steps}:
\begin{enumerate}
    \item Study explicit examples (Schwarzschild perturbations)
    \item Investigate other quasi-local masses
    \item Consider if a modified inequality is the true statement
\end{enumerate}

\end{document}
