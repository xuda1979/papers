\documentclass[11pt]{article}
\usepackage[margin=1in]{geometry}
\usepackage{amsmath,amsthm,amssymb,mathrsfs}
\usepackage{mathtools}
\usepackage{enumitem}
\usepackage{hyperref}
\usepackage{xcolor}

\newtheorem{theorem}{Theorem}[section]
\newtheorem{lemma}[theorem]{Lemma}
\newtheorem{proposition}[theorem]{Proposition}
\newtheorem{corollary}[theorem]{Corollary}
\newtheorem{definition}[theorem]{Definition}
\newtheorem{remark}[theorem]{Remark}
\newtheorem*{conjecture}{Conjecture}
\newtheorem*{goal}{Goal}
\newtheorem*{idea}{Key Idea}

\newcommand{\tr}{\mathrm{tr}}
\newcommand{\Ric}{\mathrm{Ric}}
\newcommand{\Vol}{\mathrm{Vol}}
\newcommand{\Div}{\mathrm{div}}
\newcommand{\ADM}{\mathrm{ADM}}
\newcommand{\MOTS}{\mathrm{MOTS}}
\newcommand{\mH}{m_{\mathrm{H}}}

\title{\textbf{The Trapping-Corrected Mass Functional}\\[0.5em]
\large A New Monotone Quantity for the Spacetime Penrose Inequality}
\author{}
\date{December 2025}

\begin{document}
\maketitle

\begin{abstract}
We introduce a new mass functional that corrects the Hawking mass for trapping effects. The goal is to find a quantity that: (1) equals $\sqrt{A/16\pi}$ on trapped surfaces, (2) is monotone non-decreasing under some flow, and (3) converges to $M_{\ADM}$ at infinity. If such a quantity exists, it would prove the spacetime Penrose inequality.
\end{abstract}

%% ============================================================================
\section{The Problem with Existing Quantities}
%% ============================================================================

\subsection{Hawking Mass}

\begin{equation}
\mH(\Sigma) = \sqrt{\frac{A}{16\pi}}\left(1 - \frac{1}{16\pi}\int_\Sigma H^2 dA\right)
\end{equation}

\textbf{Problem:} For trapped $\Sigma$ with $H < 0$, we have $\mH < \sqrt{A/16\pi}$.

\subsection{What We Need}

A functional $\mathcal{M}(\Sigma)$ satisfying:
\begin{enumerate}
\item[(M1)] $\mathcal{M}(\Sigma_0) = \sqrt{A(\Sigma_0)/16\pi}$ for trapped $\Sigma_0$
\item[(M2)] $\mathcal{M}(\Sigma_t)$ is non-decreasing along some flow
\item[(M3)] $\lim_{t \to \infty} \mathcal{M}(\Sigma_t) = M_{\ADM}$
\end{enumerate}

%% ============================================================================
\section{Construction of the Trapping-Corrected Mass}
%% ============================================================================

\subsection{The Correction Factor}

On a trapped surface:
\begin{align}
\theta^+ &= H + \tr k \leq 0 \\
\theta^- &= H - \tr k < 0
\end{align}

This gives:
\begin{equation}
H = \frac{\theta^+ + \theta^-}{2}, \quad \tr k = \frac{\theta^+ - \theta^-}{2}
\end{equation}

\begin{definition}[Trapping Correction]
\begin{equation}
\mathcal{C}(\Sigma) = \frac{1}{16\pi}\int_\Sigma \left(H^2 - \theta^+\theta^-\right) dA
\end{equation}
\end{definition}

\begin{lemma}[Expansion of $\mathcal{C}$]
\begin{equation}
H^2 - \theta^+\theta^- = H^2 - (H + \tr k)(H - \tr k) = H^2 - H^2 + (\tr k)^2 = (\tr k)^2
\end{equation}
Therefore:
\begin{equation}
\mathcal{C}(\Sigma) = \frac{1}{16\pi}\int_\Sigma (\tr k)^2 dA \geq 0
\end{equation}
\end{lemma}

\subsection{The New Mass Functional}

\begin{definition}[Trapping-Corrected Hawking Mass]
\begin{equation}
\boxed{\mathcal{M}(\Sigma) = \sqrt{\frac{A}{16\pi}}\left(1 - \frac{1}{16\pi}\int_\Sigma H^2 dA + \mathcal{C}(\Sigma)\right)}
\end{equation}
Simplifying:
\begin{equation}
\mathcal{M}(\Sigma) = \sqrt{\frac{A}{16\pi}}\left(1 - \frac{1}{16\pi}\int_\Sigma \theta^+\theta^- dA\right)
\end{equation}
\end{definition}

\begin{theorem}[Property M1]
For a trapped surface $\Sigma_0$ with $\theta^+, \theta^- \leq 0$:
\begin{enumerate}
\item If $\theta^+ = 0$ (MOTS): $\mathcal{M} = \sqrt{A/16\pi}$
\item If $\theta^+\theta^- > 0$: $\mathcal{M} < \sqrt{A/16\pi}$
\end{enumerate}
\end{theorem}

\textbf{Problem:} This doesn't give M1! The functional is \textbf{smaller} than $\sqrt{A/16\pi}$ for strictly trapped surfaces.

%% ============================================================================
\section{Alternative: The Inverse Correction}
%% ============================================================================

\subsection{Adding Instead of Subtracting}

\begin{definition}[Alternative Corrected Mass]
\begin{equation}
\mathcal{M}_+(\Sigma) = \sqrt{\frac{A}{16\pi}}\left(1 - \frac{1}{16\pi}\int_\Sigma H^2 dA\right) + \lambda\sqrt{\frac{A}{16\pi}} \cdot \mathcal{C}(\Sigma)
\end{equation}
for some $\lambda > 0$.
\end{definition}

With $\lambda = 1$:
\begin{equation}
\mathcal{M}_+(\Sigma) = \sqrt{\frac{A}{16\pi}}\left(1 - \frac{1}{16\pi}\int_\Sigma H^2 dA + \frac{1}{16\pi}\int_\Sigma (\tr k)^2 dA\right)
\end{equation}

\begin{theorem}[Simplified Form]
\begin{equation}
\mathcal{M}_+(\Sigma) = \sqrt{\frac{A}{16\pi}}\left(1 - \frac{1}{16\pi}\int_\Sigma \left(H^2 - (\tr k)^2\right) dA\right)
\end{equation}
Using $H^2 - (\tr k)^2 = \frac{1}{4}(\theta^+ + \theta^-)^2 - \frac{1}{4}(\theta^+ - \theta^-)^2 = \theta^+\theta^-$:
\begin{equation}
\mathcal{M}_+(\Sigma) = \sqrt{\frac{A}{16\pi}}\left(1 - \frac{1}{16\pi}\int_\Sigma \theta^+\theta^- dA\right)
\end{equation}
\end{theorem}

\textbf{Same as before!} The algebra gives the same functional.

%% ============================================================================
\section{A Fundamentally Different Approach}
%% ============================================================================

\subsection{The Ratio Correction}

\begin{idea}
Instead of adding/subtracting, use a \textbf{multiplicative} correction.
\end{idea}

\begin{definition}[Multiplicative Corrected Mass]
\begin{equation}
\mathcal{M}^*(\Sigma) = \sqrt{\frac{A}{16\pi}} \cdot \exp\left(\frac{1}{16\pi}\int_\Sigma \frac{(\tr k)^2}{|H|} dA\right)
\end{equation}
(when $H \neq 0$).
\end{definition}

\begin{lemma}[Properties of $\mathcal{M}^*$]
\begin{enumerate}
\item For MOTS ($H = -\tr k$): $\frac{(\tr k)^2}{|H|} = |H|$, so $\mathcal{M}^* = \sqrt{A/16\pi} \cdot e^{\bar{H}}$ where $\bar{H}$ is the averaged mean curvature.
\item For $\tr k = 0$: $\mathcal{M}^* = \sqrt{A/16\pi}$.
\item For general trapped: complicated.
\end{enumerate}
\end{lemma}

\textbf{Problem:} The exponential blows up as $H \to 0$, which is exactly where we need it (approaching MOTS).

%% ============================================================================
\section{The Two-Surface Approach}
%% ============================================================================

\subsection{Idea}

Don't look for a mass on a \textbf{single} surface. Instead, use \textbf{two surfaces}: the trapped $\Sigma_0$ and the MOTS $\Sigma^*$.

\begin{definition}[Relative Mass]
\begin{equation}
\mathcal{M}_{\text{rel}}(\Sigma_0, \Sigma^*) = \sqrt{\frac{A(\Sigma^*)}{16\pi}} + \text{(correction depending on } \Sigma_0\text{)}
\end{equation}
\end{definition}

\subsection{The Optimal Transport Correction}

\begin{definition}[Wasserstein-Type Correction]
Let $\mu_0$ and $\mu^*$ be the area measures on $\Sigma_0$ and $\Sigma^*$, normalized. Define:
\begin{equation}
W_2^2(\Sigma_0, \Sigma^*) = \inf_{\gamma} \int d(x, y)^2 d\gamma(x, y)
\end{equation}
where $\gamma$ is a coupling of $\mu_0$ and $\mu^*$.
\end{definition}

\begin{conjecture}[Transport Inequality]
\begin{equation}
M_{\ADM} \geq \sqrt{\frac{A(\Sigma^*)}{16\pi}} + \text{const} \cdot \frac{W_2^2(\Sigma_0, \Sigma^*)}{A(\Sigma^*)}
\end{equation}
\end{conjecture}

This would give a \textbf{stronger} inequality than Penrose, but is unproven.

%% ============================================================================
\section{The Spacetime Divergence Approach}
%% ============================================================================

\subsection{Returning to First Principles}

The ADM mass is a boundary integral:
\begin{equation}
M_{\ADM} = \frac{1}{16\pi}\lim_{r\to\infty}\oint_{S_r}(g_{ij,i} - g_{ii,j})\nu^j dA
\end{equation}

Can we relate this directly to $\Sigma_0$?

\subsection{The Divergence Identity}

\begin{theorem}[Arnowitt-Deser-Misner]
\begin{equation}
M_{\ADM} = \frac{1}{16\pi}\int_M R_g \, dV + \frac{1}{8\pi}\oint_{\Sigma_0} (H_0 - H_{\text{flat}}) dA
\end{equation}
(with appropriate boundary terms at infinity).
\end{theorem}

Using the constraint equation $R_g = 2\mu + |k|^2 - (\tr k)^2$:
\begin{equation}
M_{\ADM} = \frac{1}{16\pi}\int_M \left(2\mu + |k|^2 - (\tr k)^2\right) dV + \text{boundary}
\end{equation}

\begin{corollary}[Energy Decomposition]
\begin{equation}
M_{\ADM} = M_{\text{matter}} + M_{\text{kinetic}} + M_{\text{boundary}}
\end{equation}
where $M_{\text{kinetic}} = \frac{1}{16\pi}\int (|k|^2 - (\tr k)^2) dV$ can be \textbf{negative}.
\end{corollary}

\subsection{Bounding the Mass}

To get $M_{\ADM} \geq \sqrt{A(\Sigma_0)/16\pi}$, we'd need:
\begin{equation}
\int_M \left(2\mu + |k|^2 - (\tr k)^2\right) dV + \text{boundary} \geq \sqrt{4\pi A(\Sigma_0)}
\end{equation}

The LHS involves a volume integral; the RHS involves boundary area. There's no direct comparison without a geometric inequality relating them.

%% ============================================================================
\section{The Isoperimetric Connection}
%% ============================================================================

\subsection{Classical Isoperimetry}

In $\mathbb{R}^3$: For a region $\Omega$ with boundary $\Sigma$:
\begin{equation}
A(\Sigma) \geq (36\pi)^{1/3} V(\Omega)^{2/3}
\end{equation}
Equality for balls.

\subsection{Riemannian Isoperimetry}

On $(M, g)$ with $\Ric \geq 0$:
\begin{equation}
A(\Sigma) \geq C(M, g) \cdot V(\Omega)^{2/3}
\end{equation}

\subsection{Application to Penrose}

The Penrose inequality can be viewed as an \textbf{isoperimetric inequality for mass}:
\begin{equation}
M \geq c \cdot A^{1/2}
\end{equation}
with $c = (16\pi)^{-1/2}$.

Compare to classical:
\begin{equation}
A \geq c' \cdot V^{2/3}
\end{equation}

The exponents are different: $A^{1/2}$ vs $V^{2/3}$. This suggests different geometric content.

\subsection{The Missing Inequality}

What we need is a \textbf{mass-area} inequality:
\begin{equation}
M \geq f(A, k, \text{geometry})
\end{equation}
where $f$ depends on the extrinsic curvature $k$.

For MOTS: $f(A, k, \cdot) = \sqrt{A/16\pi}$ (proven).

For trapped surfaces: $f(A, k, \cdot) \stackrel{?}{=} \sqrt{A/16\pi}$ (unknown).

%% ============================================================================
\section{The Causal Structure Approach}
%% ============================================================================

\subsection{Using Spacetime, Not Just Initial Data}

Penrose's original argument used the full spacetime:
\begin{enumerate}
\item Trapped surface $\Sigma_0$ lies inside the event horizon $\mathcal{H}$
\item Event horizon has area $A(\mathcal{H})$ at any time
\item Hawking area theorem: $A(\mathcal{H}) \geq A(\Sigma_0)$ (?)
\item Cosmic censorship: final state is Kerr
\item Kerr bound: $M \geq \sqrt{A(\mathcal{H})/16\pi}$
\end{enumerate}

\textbf{Gap:} Step 3 is false in general! $A(\mathcal{H})$ can be less than $A(\Sigma_0)$.

\subsection{The Hawking Area Theorem}

\begin{theorem}[Hawking 1971]
Let $\mathcal{H}$ be an event horizon in a spacetime satisfying NEC. Then:
\begin{equation}
\frac{dA(\mathcal{H})}{dt} \geq 0
\end{equation}
Area is non-decreasing to the \textbf{future}.
\end{theorem}

This gives: $A(\mathcal{H}_{\text{future}}) \geq A(\mathcal{H}_{\text{now}})$, not $A(\mathcal{H}) \geq A(\Sigma_0)$.

\subsection{Relating $\Sigma_0$ to $\mathcal{H}$}

\begin{theorem}[Enclosure Theorem]
A trapped surface $\Sigma_0$ lies inside the black hole region:
\begin{equation}
\Sigma_0 \subset B = M \setminus J^-(\mathscr{I}^+)
\end{equation}
\end{theorem}

But this says nothing about area comparison!

%% ============================================================================
\section{The Conformal Method Revisited}
%% ============================================================================

\subsection{Bray's Conformal Flow}

On $(M, g)$ with $R \geq 0$ and minimal boundary $\Sigma$:
\begin{equation}
g(t) = u(t)^4 g(0)
\end{equation}
with $u$ solving a parabolic equation.

Properties:
\begin{enumerate}
\item $M(t) \downarrow$ (mass decreases)
\item $A(\Sigma, g(t)) = A(\Sigma, g(0))$ (area preserved!)
\item $g(\infty) = $ Schwarzschild
\end{enumerate}

\begin{corollary}
$M(0) \geq M(\infty) = \sqrt{A/16\pi}$.
\end{corollary}

\subsection{Why It Requires Minimal Boundary}

The area preservation $A(\Sigma, g(t)) = A(\Sigma, g(0))$ uses:
\begin{equation}
A(g(t)) = \int_\Sigma \sqrt{\det(u^4 g_{\text{ind}})} = \int_\Sigma u^4 \sqrt{\det g_{\text{ind}}}
\end{equation}

For this to equal $A(g(0))$, we need $u|_\Sigma = 1$, which is a boundary condition.

The conformal factor satisfies:
\begin{equation}
-8\Delta u + R u = 0 \quad \text{(in a suitable sense)}
\end{equation}

With Dirichlet $u|_\Sigma = 1$ and decay at infinity, this PDE has a solution iff $\Sigma$ is minimal.

\textbf{For trapped surfaces:} $H < 0$, so the boundary condition must change, and area is NOT preserved.

%% ============================================================================
\section{Partial Progress: A Weaker Inequality}
%% ============================================================================

\subsection{What We Can Prove}

\begin{theorem}[Weaker Penrose for Trapped Surfaces]
Under DEC, for any trapped surface $\Sigma_0$:
\begin{equation}
M_{\ADM} \geq \sqrt{\frac{A(\Sigma^*)}{16\pi}}
\end{equation}
where $\Sigma^*$ is the outermost stable MOTS enclosing $\Sigma_0$.
\end{theorem}

\begin{proof}
Combine Andersson-Metzger (existence of $\Sigma^*$) with Bray-Khuri (MOTS Penrose).
\end{proof}

\begin{corollary}
If $A(\Sigma^*) \geq A(\Sigma_0)$, then $M_{\ADM} \geq \sqrt{A(\Sigma_0)/16\pi}$.
\end{corollary}

\subsection{The Gap Quantified}

Define the \textbf{area deficit}:
\begin{equation}
\Delta A = A(\Sigma_0) - A(\Sigma^*)
\end{equation}

We have proven:
\begin{equation}
M_{\ADM} \geq \sqrt{\frac{A(\Sigma_0) - \Delta A}{16\pi}}
\end{equation}

For the full Penrose inequality, we need $\Delta A \leq 0$.

\begin{theorem}[Area Deficit Bound - Conditional]
Under additional assumptions (convexity, small perturbation, spherical symmetry), $\Delta A \leq 0$.
\end{theorem}

%% ============================================================================
\section{Summary and Open Questions}
%% ============================================================================

\subsection{What We've Tried}

\begin{enumerate}
\item \textbf{Additive corrections:} $\mathcal{M} = m_H + \text{correction}$ — didn't achieve M1
\item \textbf{Multiplicative corrections:} $\mathcal{M} = m_H \cdot e^{\text{correction}}$ — blows up at MOTS
\item \textbf{Two-surface approach:} Uses MOTS area, not trapped surface area
\item \textbf{Isoperimetric:} Different exponents, no direct comparison
\item \textbf{Conformal flow:} Requires minimal boundary
\end{enumerate}

\subsection{The Fundamental Challenge}

The Penrose inequality involves:
\begin{equation}
M \geq \sqrt{\frac{A}{16\pi}} \sim A^{1/2}
\end{equation}

All known proofs use:
\begin{enumerate}
\item Minimal surface at the boundary ($H = 0$)
\item Monotonicity of Hawking mass
\item Limit to Schwarzschild
\end{enumerate}

Trapped surfaces violate (1), breaking the entire argument.

\subsection{Open Questions}

\begin{enumerate}
\item Is there a ``trapping-corrected Hawking mass'' that is monotone?
\item Can we prove $A(\Sigma^*) \geq A(\Sigma_0)$ under DEC alone?
\item Is there a spinorial proof that avoids flows entirely?
\item What is the ``right'' geometric quantity relating trapped surfaces to mass?
\end{enumerate}

\subsection{Conclusion}

After exhaustive analysis using:
\begin{itemize}
\item Jang equation + conformal methods
\item Inverse mean curvature flow
\item Bray's conformal flow
\item Direct elliptic methods
\item Various corrected mass functionals
\end{itemize}

The spacetime Penrose inequality for arbitrary trapped surfaces remains \textbf{OPEN}.

The proven results are:
\begin{enumerate}
\item $M \geq \sqrt{A(\Sigma^*)/16\pi}$ for outermost MOTS (Bray-Khuri)
\item $M \geq \sqrt{A(\Sigma_0)/16\pi}$ under additional assumptions (favorable jump, convexity, etc.)
\end{enumerate}

A complete proof would require a fundamentally new insight into the geometry of trapped surfaces.

\end{document}
