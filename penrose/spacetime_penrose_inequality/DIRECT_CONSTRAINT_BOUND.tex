%% DIRECT_CONSTRAINT_BOUND.tex
%%
%% NEW ATTACK: Direct Bound from Constraint Equations
%%
%% Key Idea: Bypass area dominance entirely by bounding A(Σ) directly from M_ADM
%%
%% December 2025

\documentclass[11pt]{amsart}
\usepackage{amsmath,amssymb,amsthm}
\usepackage{xcolor}
\usepackage{tcolorbox}

\tcbuselibrary{theorems}

\newtcolorbox{keyidea}{
    colback=green!5!white,
    colframe=green!75!black,
    title={\textbf{KEY IDEA}}
}

\newtcolorbox{redattack}{
    colback=red!5!white,
    colframe=red!75!black,
    title={\textbf{RED TEAM ATTACK}}
}

\newtcolorbox{breakthrough}{
    colback=purple!5!white,
    colframe=purple!75!black,
    title={\textbf{POTENTIAL BREAKTHROUGH}}
}

\newtheorem{theorem}{Theorem}[section]
\newtheorem{lemma}[theorem]{Lemma}
\newtheorem{proposition}[theorem]{Proposition}
\newtheorem{corollary}[theorem]{Corollary}
\newtheorem{definition}[theorem]{Definition}

\newcommand{\ADM}{\mathrm{ADM}}
\newcommand{\Area}{\mathrm{Area}}
\newcommand{\tr}{\mathrm{tr}}
\newcommand{\divergence}{\mathrm{div}}
\newcommand{\bR}{\mathbb{R}}

\title{Direct Constraint Bound on Trapped Surface Area\\
\large Bypassing Area Dominance}
\author{}
\date{December 2025}

\begin{document}
\maketitle

\begin{abstract}
We attempt to prove $M_{\ADM} \ge \sqrt{A(\Sigma)/(16\pi)}$ directly from the constraint equations, without using the MOTS or area dominance. The key idea is to relate the trapped surface area directly to global quantities.
\end{abstract}

\tableofcontents

%% ============================================================================
\section{The Strategy}
%% ============================================================================

\begin{keyidea}
Instead of proving:
\begin{equation}
    A(\Sigma) \le A(\Sigma^*) \quad \text{(area dominance)}
\end{equation}
and then using RPI on $\Sigma^*$, we try to prove:
\begin{equation}
    A(\Sigma) \le 16\pi M_{\ADM}^2 \quad \text{(direct bound)}
\end{equation}
directly from the constraint equations.
\end{keyidea}

%% ============================================================================
\section{The Constraint Equations}
%% ============================================================================

On initial data $(M, g, k)$, the constraint equations are:
\begin{align}
    R_g - |k|^2 + (\tr k)^2 &= 2\mu, \label{eq:ham} \\
    \divergence(k - (\tr k)g) &= J. \label{eq:mom}
\end{align}

With DEC: $\mu \ge |J|_g$.

The ADM mass is:
\begin{equation}
    M_{\ADM} = \lim_{r \to \infty} \frac{1}{16\pi} \oint_{S_r} (g_{ij,j} - g_{jj,i}) \nu^i dA
\end{equation}

%% ============================================================================
\section{Approach 1: Energy Integral}
%% ============================================================================

\begin{proposition}[ADM Mass as Volume Integral]
\begin{equation}
    M_{\ADM} = \frac{1}{16\pi} \int_M R_g \, dV + \frac{1}{16\pi} \int_M (|k|^2 - (\tr k)^2) dV + \text{boundary}
\end{equation}
\end{proposition}

\textbf{Idea:} Relate the trapped surface area $A(\Sigma)$ to these integrals.

For a trapped surface, we have:
\begin{equation}
    \theta^+ = H_\Sigma + \tr_\Sigma k < 0, \quad \theta^- = H_\Sigma - \tr_\Sigma k < 0
\end{equation}

Adding: $H_\Sigma < 0$ (mean curvature is negative).

\subsection{Gauss-Bonnet on Trapped Surface}

By Gauss-Bonnet:
\begin{equation}
    \int_\Sigma K_\Sigma dA = 2\pi \chi(\Sigma) = 4\pi \quad \text{(for genus 0)}
\end{equation}
where $K_\Sigma$ is the Gaussian curvature of $\Sigma$.

The Gauss equation gives:
\begin{equation}
    K_\Sigma = \frac{1}{2}R_g|_\Sigma - \frac{1}{2}(R_g - 2\mathrm{Ric}(\nu,\nu)) - \frac{1}{2}(H^2 - |A|^2)
\end{equation}
where $A$ is the second fundamental form and $|A|^2 = A_{ij}A^{ij}$.

This doesn't directly give a bound on $A(\Sigma)$.

%% ============================================================================
\section{Approach 2: Geroch-Type Identity}
%% ============================================================================

Consider the Geroch functional:
\begin{equation}
    m_G(\Sigma) = \sqrt{\frac{A(\Sigma)}{16\pi}} \left(1 - \frac{1}{16\pi} \int_\Sigma H^2 dA\right)
\end{equation}

For a trapped surface with $H < 0$:
\begin{equation}
    m_G(\Sigma) = \sqrt{\frac{A(\Sigma)}{16\pi}} \left(1 - \frac{1}{16\pi} \int_\Sigma H^2 dA\right)
\end{equation}

If $\int_\Sigma H^2 dA < 16\pi$, then $m_G(\Sigma) > 0$.

\textbf{Question:} Can we bound $m_G(\Sigma) \le M_{\ADM}$ directly?

\begin{redattack}
The Geroch monotonicity formula requires:
\begin{enumerate}
    \item A flow connecting $\Sigma$ to infinity
    \item The flow must be well-defined (e.g., IMCF needs $H > 0$)
\end{enumerate}

For trapped surfaces with $H < 0$, IMCF is not applicable.
\end{redattack}

%% ============================================================================
\section{Approach 3: Isoperimetric Inequality}
%% ============================================================================

\begin{keyidea}
In $\bR^3$, the isoperimetric inequality says:
\begin{equation}
    A(\Sigma) \ge (36\pi)^{1/3} V(\Omega)^{2/3}
\end{equation}
where $\Omega$ is the region enclosed by $\Sigma$.

In a manifold with $R_g \ge 0$, this generalizes.

Can we use this to bound $A(\Sigma)$ in terms of the ``volume deficit''?
\end{keyidea}

\begin{proposition}[Isoperimetric Profile and Mass]
On an asymptotically flat manifold $(M, g)$ with $R_g \ge 0$, the isoperimetric profile function satisfies:
\begin{equation}
    I(V) = \inf\{A(\Sigma) : \Vol(\Omega_\Sigma) = V\} \ge (36\pi)^{1/3} V^{2/3} - C \cdot M_{\ADM}
\end{equation}
for some constant $C$.
\end{proposition}

\textbf{Problem:} This gives a \textit{lower} bound on area, not an upper bound.

We need the opposite: $A(\Sigma) \le C \cdot M_{\ADM}^2$.

%% ============================================================================
\section{Approach 4: Inverse Problem}
%% ============================================================================

\begin{keyidea}
Instead of bounding area in terms of mass, bound the \textit{minimum mass} required to have a trapped surface of area $A$.

\textbf{Claim:} If $(M, g, k)$ has a trapped surface $\Sigma$ with $A(\Sigma) = A_0$, then:
\begin{equation}
    M_{\ADM}(g, k) \ge f(A_0)
\end{equation}
for some function $f$.

If $f(A_0) = \sqrt{A_0/(16\pi)}$, this is exactly Penrose.
\end{keyidea}

\subsection{Construction via Constraint Equations}

The trapped condition $\theta^+ < 0$, $\theta^- < 0$ gives:
\begin{align}
    H + \tr_\Sigma k &< 0, \\
    H - \tr_\Sigma k &< 0.
\end{align}

These constrain $H$ and $\tr_\Sigma k$ on the surface.

The constraint equations are:
\begin{align}
    R_g &= |k|^2 - (\tr k)^2 + 2\mu, \\
    \divergence k &= J + d(\tr k).
\end{align}

\textbf{Can we show that having a trapped surface forces the ADM mass to be large?}

\subsection{Heuristic Argument}

A trapped surface has $H < 0$ (concave toward the interior).

By the first variation of area, pushing $\Sigma$ inward \textit{decreases} area.

The region inside $\Sigma$ is ``collapsing'' — it has positive ``expansion energy.''

This energy should contribute to the ADM mass.

\textbf{Quantitative bound:}

The integral:
\begin{equation}
    \int_\Sigma |\theta^-| dA = \int_\Sigma |H - \tr_\Sigma k| dA
\end{equation}
measures the ``trapping strength.''

By Cauchy-Schwarz:
\begin{equation}
    \left(\int_\Sigma |\theta^-| dA\right)^2 \le A(\Sigma) \int_\Sigma (\theta^-)^2 dA
\end{equation}

\textbf{Problem:} We don't have a direct relation between $\int (\theta^-)^2 dA$ and $M_{\ADM}$.

%% ============================================================================
\section{Approach 5: Spinor Witten Identity}
%% ============================================================================

The Witten identity gives:
\begin{equation}
    M_{\ADM} = \frac{1}{4\pi} \int_M |\nabla \psi|^2 + \frac{1}{8\pi} \int_M R_g |\psi|^2 + \frac{1}{4\pi} \int_M \langle \psi, (\divergence k - d\tr k) \cdot \psi \rangle
\end{equation}
for a spinor $\psi$ satisfying the Witten equation $D\psi = 0$.

With DEC, all terms are nonnegative, giving $M_{\ADM} \ge 0$.

\textbf{To get a bound involving $A(\Sigma)$:} Need to choose $\psi$ adapted to $\Sigma$.

\begin{keyidea}
Try the spinor that ``concentrates'' on $\Sigma$.

If $\psi$ is supported near $\Sigma$:
\begin{equation}
    M_{\ADM} \ge \frac{1}{4\pi} \int_{\text{near } \Sigma} |\nabla \psi|^2 + \ldots
\end{equation}

The challenge is to relate $\int |\nabla \psi|^2$ to $A(\Sigma)$.
\end{keyidea}

\subsection{Herzlich-Type Boundary Term}

Herzlich's spinor inequality gives:
\begin{equation}
    M_{\ADM} \ge \sqrt{\frac{A(\Sigma)}{16\pi}} \cdot \text{(spin structure coefficient)}
\end{equation}
when $\Sigma$ is a minimal surface.

For a trapped surface ($H < 0$), there's an additional term:
\begin{equation}
    M_{\ADM} \ge \sqrt{\frac{A(\Sigma)}{16\pi}} - C \int_\Sigma |H| dA
\end{equation}

\textbf{Problem:} The correction term can be larger than the main term.

%% ============================================================================
\section{Approach 6: Bray's Conformal Flow}
%% ============================================================================

Bray's proof of RPI uses a conformal flow that ``inflates'' the minimal surface while preserving $R_g \ge 0$ and decreasing mass.

\textbf{Idea:} Can we run Bray's flow starting from a trapped surface?

\begin{redattack}
Bray's flow requires:
\begin{enumerate}
    \item Starting surface is minimal ($H = 0$)
    \item The surface is outer-minimizing
\end{enumerate}

For trapped surfaces:
\begin{enumerate}
    \item $H < 0$ (not minimal)
    \item Not outer-minimizing
\end{enumerate}

Bray's flow cannot be directly applied.
\end{redattack}

%% ============================================================================
\section{Approach 7: Capacitary Estimate}
%% ============================================================================

The capacity of $\Sigma$ is:
\begin{equation}
    \mathrm{Cap}(\Sigma) = \inf\left\{\int_M |\nabla u|^2 dV : u|_\Sigma = 1, u \to 0 \text{ at } \infty\right\}
\end{equation}

For a round sphere of radius $r$ in $\bR^3$:
\begin{equation}
    \mathrm{Cap}(S_r) = 4\pi r, \quad A(S_r) = 4\pi r^2
\end{equation}

So $A = \pi \cdot \mathrm{Cap}^2$ in the Euclidean case.

\begin{proposition}[Capacity-Area Isoperimetric]
In $(M, g)$ with $R_g \ge 0$:
\begin{equation}
    A(\Sigma) \le c \cdot \mathrm{Cap}(\Sigma)^2
\end{equation}
for some constant $c$ depending on the geometry.
\end{proposition}

\textbf{Relation to mass:}
\begin{equation}
    M_{\ADM} \ge \frac{1}{4\pi} \mathrm{Cap}(\Sigma)
\end{equation}
by the positive mass theorem (the capacity measures the ``amount of mass seen from infinity'').

Combining:
\begin{equation}
    A(\Sigma) \le c \cdot \mathrm{Cap}^2 \le c \cdot (4\pi M_{\ADM})^2 = 16\pi c \cdot M_{\ADM}^2
\end{equation}

\begin{breakthrough}
If the constant $c = 1$, we get:
\begin{equation}
    A(\Sigma) \le 16\pi M_{\ADM}^2
\end{equation}
which is equivalent to $M_{\ADM} \ge \sqrt{A(\Sigma)/(16\pi)}$!

\textbf{The question is: what is the sharp constant $c$?}
\end{breakthrough}

%% ============================================================================
\section{The Capacity Approach: Detailed Analysis}
%% ============================================================================

\subsection{The Capacitary Potential}

Let $u$ be the capacitary potential:
\begin{equation}
    \Delta_g u = 0 \quad \text{in } M \setminus \Omega_\Sigma, \quad u|_\Sigma = 1, \quad u \to 0 \text{ at } \infty
\end{equation}

Then:
\begin{equation}
    \mathrm{Cap}(\Sigma) = \int_{M \setminus \Omega} |\nabla u|^2 dV = -\int_\Sigma \frac{\partial u}{\partial \nu} dA
\end{equation}

\subsection{Isoperimetric for Capacity}

By the co-area formula:
\begin{equation}
    \mathrm{Cap}(\Sigma) = \int_0^1 \frac{1}{\int_{u = t} |\nabla u|^{-1} dA} dt
\end{equation}

The level sets $\{u = t\}$ for $0 < t < 1$ foliate the exterior of $\Sigma$.

By the isoperimetric inequality for each level set:
\begin{equation}
    A(\{u = t\}) \ge ... \text{(depends on enclosed volume)}
\end{equation}

\subsection{Connection to Trapped Surfaces}

For a trapped surface, we can define a \textit{modified} capacity using the null expansion.

Define:
\begin{equation}
    \mathrm{Cap}_\theta(\Sigma) = \inf\left\{\int_M \theta^+ \cdot |\nabla u|^2 dV : u|_\Sigma = 1, u \to 0\right\}
\end{equation}

\textbf{Problem:} $\theta^+$ is only defined on surfaces, not in the bulk.

\textbf{Workaround:} Extend $\theta^+$ using the level sets of some flow.

%% ============================================================================
\section{Main Result: Conditional Capacity Bound}
%% ============================================================================

\begin{theorem}[Capacity-Area for Trapped Surfaces - Conditional]\label{thm:cap-area}
Let $(M, g, k)$ be asymptotically flat initial data satisfying DEC. Let $\Sigma$ be a trapped surface. Assume:
\begin{enumerate}
    \item[(C)] The capacity-area isoperimetric inequality holds with constant $c = 1$:
    \begin{equation}
        A(\Sigma) \le \pi \cdot \mathrm{Cap}(\Sigma)^2
    \end{equation}
\end{enumerate}
Then:
\begin{equation}
    M_{\ADM} \ge \sqrt{\frac{A(\Sigma)}{16\pi}}
\end{equation}
\end{theorem}

\begin{proof}
\textbf{Step 1:} By the positive mass theorem variant:
\begin{equation}
    M_{\ADM} \ge \frac{1}{4\pi} \mathrm{Cap}(\Sigma)
\end{equation}

\textbf{Step 2:} By assumption (C):
\begin{equation}
    A(\Sigma) \le \pi \cdot \mathrm{Cap}(\Sigma)^2
\end{equation}

\textbf{Step 3:} From Step 1: $\mathrm{Cap}(\Sigma) \le 4\pi M_{\ADM}$.

\textbf{Step 4:} Substituting into Step 2:
\begin{equation}
    A(\Sigma) \le \pi \cdot (4\pi M_{\ADM})^2 = 16\pi^3 M_{\ADM}^2
\end{equation}

\textbf{Wait:} This gives $A \le 16\pi^3 M^2$, not $A \le 16\pi M^2$.

\textbf{Error:} The Euclidean relation is $A = 4\pi r^2$ and $\mathrm{Cap} = 4\pi r$, so $A = \frac{1}{4\pi}\mathrm{Cap}^2$.

\textbf{Corrected Step 2:} 
\begin{equation}
    A(\Sigma) \le \frac{1}{4\pi} \mathrm{Cap}(\Sigma)^2
\end{equation}

\textbf{Corrected Step 4:}
\begin{equation}
    A(\Sigma) \le \frac{1}{4\pi} (4\pi M_{\ADM})^2 = 4\pi M_{\ADM}^2
\end{equation}

This gives $M_{\ADM} \ge \sqrt{A/(4\pi)}$, which is STRONGER than Penrose!

\textbf{Problem:} The capacity bound $M \ge \mathrm{Cap}/(4\pi)$ is not correct in general.
\end{proof}

%% ============================================================================
\section{Corrected Capacity Analysis}
%% ============================================================================

\subsection{Correct Relation Between Mass and Capacity}

In Schwarzschild with mass $M$:
\begin{itemize}
    \item ADM mass: $M_{\ADM} = M$
    \item Horizon area: $A = 16\pi M^2$ (at $r = 2M$)
    \item Capacity of horizon: $\mathrm{Cap}(\Sigma) = 4\pi \cdot 2M = 8\pi M$
\end{itemize}

So $\mathrm{Cap} = 8\pi M_{\ADM}$, giving:
\begin{equation}
    M_{\ADM} = \frac{\mathrm{Cap}}{8\pi}
\end{equation}

And:
\begin{equation}
    A = 16\pi M^2 = 16\pi \left(\frac{\mathrm{Cap}}{8\pi}\right)^2 = \frac{\mathrm{Cap}^2}{4\pi}
\end{equation}

\textbf{So the relation $A = \mathrm{Cap}^2/(4\pi)$ is exact for Schwarzschild horizon!}

\subsection{General Inequality}

\begin{conjecture}[Capacity-Area Isoperimetric]
For any closed surface $\Sigma$ in an asymptotically flat $(M, g)$ with $R_g \ge 0$:
\begin{equation}
    A(\Sigma) \le \frac{1}{4\pi} \mathrm{Cap}(\Sigma)^2
\end{equation}
with equality for round spheres in $\bR^3$.
\end{conjecture}

\begin{conjecture}[Mass-Capacity Inequality]
For asymptotically flat $(M, g)$ with $R_g \ge 0$ and outer-minimizing boundary $\Sigma$:
\begin{equation}
    M_{\ADM} \ge \frac{1}{8\pi} \mathrm{Cap}(\Sigma)
\end{equation}
with equality for Schwarzschild.
\end{conjecture}

If both conjectures hold:
\begin{equation}
    A(\Sigma) \le \frac{1}{4\pi} (8\pi M_{\ADM})^2 = 16\pi M_{\ADM}^2
\end{equation}

This is Penrose!

%% ============================================================================
\section{Conclusion}
%% ============================================================================

\begin{tcolorbox}[colback=green!10!white, colframe=green!75!black, title=\textbf{SUMMARY}]
The capacity approach is \textbf{promising} but requires proving:
\begin{enumerate}
    \item Capacity-area isoperimetric: $A \le \mathrm{Cap}^2/(4\pi)$
    \item Mass-capacity inequality: $M \ge \mathrm{Cap}/(8\pi)$
\end{enumerate}

\textbf{Status:}
\begin{itemize}
    \item (1) is related to classical isoperimetric theory — may be provable
    \item (2) is new and connects to positive mass — needs development
\end{itemize}

\textbf{Key insight:} The capacity naturally appears in mass formulas and may provide the ``right'' comparison quantity.

\textbf{Next step:} Prove the capacity-area isoperimetric for surfaces with $R_g \ge 0$.
\end{tcolorbox}

\end{document}
