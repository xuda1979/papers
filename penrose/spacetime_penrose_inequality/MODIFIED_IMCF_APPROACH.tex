% =========================================================================
%     MODIFIED INVERSE MEAN CURVATURE FLOW FOR TRAPPED SURFACES
%
%     Using θ⁺-flow and weak solutions to handle H < 0
%
%     Author: Da Xu
%     Date: December 2025
% =========================================================================

\documentclass[12pt]{article}
\usepackage{amsmath,amsthm,amssymb}
\usepackage{mathrsfs}
\usepackage{tcolorbox}

\theoremstyle{plain}
\newtheorem{theorem}{Theorem}[section]
\newtheorem{lemma}[theorem]{Lemma}
\newtheorem{proposition}[theorem]{Proposition}
\newtheorem{corollary}[theorem]{Corollary}

\theoremstyle{definition}
\newtheorem{definition}[theorem]{Definition}
\newtheorem{remark}[theorem]{Remark}

\newcommand{\ADM}{\mathrm{ADM}}
\newcommand{\tr}{\mathrm{tr}}
\newcommand{\Div}{\mathrm{div}}
\newcommand{\Area}{\mathrm{Area}}
\newcommand{\MOTS}{\mathrm{MOTS}}

\title{\textbf{Modified IMCF Approaches for the Penrose Inequality}}
\author{Da Xu}
\date{December 2025}

\begin{document}
\maketitle

\section{Standard IMCF and Its Failure}

\subsection{The Geroch Monotonicity}

Standard IMCF evolves surfaces by:
\[
    \frac{\partial X}{\partial t} = \frac{\nu}{H}
\]
where $\nu$ is the outward unit normal.

The Hawking mass:
\[
    m_H(\Sigma_t) = \sqrt{\frac{\Area(\Sigma_t)}{16\pi}} \left(1 - \frac{1}{16\pi}\int_{\Sigma_t} H^2 \, dA \right)
\]

satisfies (for $R_g \geq 0$):
\[
    \frac{d m_H}{dt} \geq 0
\]

\subsection{Why It Fails for Trapped Surfaces}

For trapped surfaces, $H < 0$. The flow equation becomes:
\[
    \frac{\partial X}{\partial t} = \frac{\nu}{H} \quad \text{with } H < 0
\]

This means $\frac{\partial X}{\partial t}$ points \textbf{inward}, not outward!

The flow collapses rather than expands.

\section{Approach 1: Inverse Null Expansion Flow}

\subsection{Definition}

Instead of using $H$, use the null expansion $\theta^+$:
\[
    \frac{\partial X}{\partial t} = \frac{\nu}{\theta^+}
\]

For trapped surfaces, $\theta^+ \leq 0$, so this still has the sign problem.

\subsection{Using $|\theta^+|$}

Define the \textbf{inverse modulus flow}:
\[
    \frac{\partial X}{\partial t} = \frac{\nu}{|\theta^+|}
\]

This always points outward.

\begin{lemma}
Under the inverse $|\theta^+|$ flow:
\[
    \frac{d\Area}{dt} = \int_{\Sigma_t} \frac{H}{|\theta^+|} \, dA
\]
\end{lemma}

\textbf{Problem:} When $H < 0$ (trapped region), $\frac{d\Area}{dt} < 0$.

Area still decreases!

\section{Approach 2: The Jang-IMCF Combination}

\subsection{Idea}

\begin{enumerate}
    \item Solve Jang equation to get $(\bar{M}, \bar{g})$ with $R_{\bar{g}} \geq 0$ (away from blow-up)
    \item Apply IMCF on the Jang surface
    \item Handle the blow-up surface carefully
\end{enumerate}

\subsection{IMCF on Jang Surface}

On the Jang surface, MOTS $\Sigma^*$ becomes minimal: $\bar{H} = 0$.

Starting IMCF from $\Sigma^*$ on $(\bar{M}, \bar{g})$:
\begin{itemize}
    \item Flow is well-defined (starts from minimal surface)
    \item Hawking mass is monotonic
    \item Approaches $M_{\ADM}$ at infinity
\end{itemize}

This gives $M_{\ADM} \geq \sqrt{\Area(\Sigma^*)/(16\pi)}$.

\textbf{Gap:} We need $\Area(\Sigma^*) \geq \Area(\Sigma_0)$, which is false.

\section{Approach 3: Weak IMCF (Huisken-Ilmanen)}

\subsection{Level Set Formulation}

Huisken-Ilmanen's weak IMCF uses a level set function $u$ satisfying:
\[
    \Div\left(\frac{\nabla u}{|\nabla u|}\right) = |\nabla u|
\]

The surfaces $\Sigma_t = \{u = t\}$ form a weak IMCF.

\subsection{Starting from Trapped Surfaces}

\begin{definition}
A \textbf{trapped-modified weak IMCF} from $\Sigma_0$ is:
\[
    \Div\left(\frac{\nabla u}{|\nabla u|}\right) = |\nabla u| + f(\theta^+, \theta^-)
\]
where $f$ incorporates the trapped condition.
\end{definition}

\textbf{Attempt:} Choose $f = -\tr_\Sigma k$ to shift from $H$ to $\theta^+$.

The equation becomes:
\[
    \Div\left(\frac{\nabla u}{|\nabla u|}\right) = |\nabla u| - \tr_\Sigma k
\]

\textbf{Problem:} This equation is ill-posed when $\tr_\Sigma k$ varies in sign.

\section{Approach 4: Double-Sided Flow}

\subsection{Idea}

For trapped surfaces with $\theta^+ < 0$ and $\theta^- < 0$:
\begin{itemize}
    \item Inward flow along $\ell$ (outgoing null) is area-decreasing
    \item Outward flow along $n$ (ingoing null) is also area-decreasing
\end{itemize}

Both null directions lead to area decrease!

\subsection{The Optimal Direction}

Define the \textbf{optimal transport direction}:
\[
    V_{\text{opt}} = \alpha \ell + \beta n
\]
where $(\alpha, \beta)$ minimize area decrease rate.

\begin{lemma}
For trapped surfaces:
\[
    \frac{d\Area}{dt} = \int_\Sigma (\alpha \theta^+ + \beta \theta^-) \, dA
\]
Since $\theta^+, \theta^- < 0$, any $\alpha, \beta > 0$ gives $\frac{d\Area}{dt} < 0$.
\end{lemma}

\textbf{Conclusion:} There is no direction that increases area from a trapped surface.

\section{Approach 5: Time-Reversed Flow}

\subsection{Idea}

Run IMCF \textbf{backward in time} from infinity toward $\Sigma_0$.

\begin{definition}
The \textbf{reverse IMCF} is:
\[
    \frac{\partial X}{\partial t} = -\frac{\nu}{H}
\]
starting from large spheres at infinity.
\end{definition}

As $t \to T$, the surfaces should approach $\Sigma_0$.

\subsection{Analysis}

For the reverse flow:
\[
    \frac{dm_H}{dt} \leq 0
\]

So Hawking mass decreases from $M_{\ADM}$ to $m_H(\Sigma_0)$.

We get: $M_{\ADM} \geq m_H(\Sigma_0)$.

\textbf{Problem:} For trapped surfaces:
\[
    m_H(\Sigma_0) = \sqrt{\frac{\Area}{16\pi}}\left(1 - \frac{1}{16\pi}\int H^2 \right)
\]

The term $-\frac{1}{16\pi}\int H^2$ makes $m_H(\Sigma_0) < \sqrt{\Area/(16\pi)}$.

This gives a \textbf{weaker} bound, not the Penrose inequality.

\section{Approach 6: Modified Hawking Mass}

\subsection{Definition}

Define a \textbf{trapped-adapted Hawking mass}:
\[
    \tilde{m}_H(\Sigma) = \sqrt{\frac{\Area}{16\pi}}\left(1 - \frac{1}{16\pi}\int_\Sigma \theta^+ \theta^- \, dA \right)
\]

Note: $\theta^+ \theta^- = H^2 - (\tr_\Sigma k)^2$.

\subsection{Properties}

For trapped surfaces: $\theta^+ \leq 0$, $\theta^- < 0$, so $\theta^+ \theta^- \geq 0$.

Therefore:
\[
    \tilde{m}_H(\Sigma_0) \leq \sqrt{\frac{\Area}{16\pi}}
\]

\textbf{Key Question:} Is $\tilde{m}_H$ monotonic under some flow?

\subsection{Monotonicity Analysis}

Under a general flow $\partial_t X = V\nu$:
\begin{align}
    \frac{d\tilde{m}_H}{dt} &= \frac{\partial \tilde{m}_H}{\partial \Area} \cdot \frac{d\Area}{dt}
    + \frac{\partial \tilde{m}_H}{\partial(\int \theta^+\theta^-)} \cdot \frac{d(\int \theta^+\theta^-)}{dt}
\end{align}

The evolution of $\theta^+ \theta^-$ involves:
\[
    \partial_t(\theta^+ \theta^-) = \theta^- \partial_t \theta^+ + \theta^+ \partial_t \theta^-
\]

Using Raychaudhuri-type equations... this becomes extremely complicated.

\begin{tcolorbox}[colback=yellow!10, colframe=orange!75!black, title=\textbf{Partial Result}]
\textbf{Observation:} The modified Hawking mass $\tilde{m}_H$ satisfies:
\[
    \tilde{m}_H(\Sigma_0) \leq \sqrt{\frac{\Area(\Sigma_0)}{16\pi}}
\]
for trapped surfaces (with equality when $\tr_\Sigma k = 0$).

But we cannot prove $\tilde{m}_H$ is monotonic to $M_{\ADM}$.
\end{tcolorbox}

\section{Approach 7: The θ-Weighted Flow}

\subsection{Definition}

Define a flow weighted by null expansions:
\[
    \frac{\partial X}{\partial t} = \frac{\theta^-}{\theta^+ \theta^-} \nu = \frac{1}{\theta^+} \nu
\]

Wait—this is inverse $\theta^+$ flow, which has sign issues.

\subsection{Alternative: Product Flow}

\[
    \frac{\partial X}{\partial t} = \frac{1}{\sqrt{|\theta^+ \theta^-|}} \nu
\]

For trapped surfaces, $\theta^+ \theta^- > 0$, so this is well-defined.

\begin{lemma}
Under the product flow:
\[
    \frac{d\Area}{dt} = \int_\Sigma \frac{H}{\sqrt{\theta^+ \theta^-}} \, dA
\]
\end{lemma}

Since $H = \frac{\theta^+ + \theta^-}{2}$ and both are negative:
\[
    \frac{H}{\sqrt{\theta^+ \theta^-}} = \frac{\theta^+ + \theta^-}{2\sqrt{\theta^+ \theta^-}} < 0
\]

Area still decreases!

\begin{tcolorbox}[colback=red!10, colframe=red!75!black, title=\textbf{Conclusion: IMCF Variants Fail}]
\textbf{Fundamental Problem:} ALL flow-based methods that move surfaces outward
require positive mean curvature to increase area.

For trapped surfaces with $H < 0$, no outward flow increases area.

\textbf{The trapped condition itself prevents monotonicity.}

This is not a technical issue but a \textbf{geometric obstruction}.
\end{tcolorbox}

\end{document}
