%% NEW_QUASILOCAL_MASS.tex
%%
%% A NEW QUASI-LOCAL MASS FOR TRAPPED SURFACES
%%
%% Innovation: Design a mass functional specifically for the Penrose problem
%% that has the correct lower bound for trapped surfaces AND correct asymptotics.
%%
%% December 2025

\documentclass[11pt]{amsart}
\usepackage{amsmath,amssymb,amsthm}
\usepackage{tcolorbox}

\tcbuselibrary{theorems}

\newtcolorbox{innovation}{
    colback=green!5!white,
    colframe=green!50!black,
    title={\textbf{NEW CONSTRUCTION}}
}

\newtcolorbox{key}{
    colback=blue!5!white,
    colframe=blue!75!black,
    title={\textbf{KEY PROPERTY}}
}

\newtcolorbox{theorem_box}{
    colback=purple!5!white,
    colframe=purple!75!black,
    title={\textbf{THEOREM}}
}

\newtheorem{theorem}{Theorem}
\newtheorem{lemma}[theorem]{Lemma}
\newtheorem{proposition}[theorem]{Proposition}
\newtheorem{corollary}[theorem]{Corollary}
\theoremstyle{definition}
\newtheorem{definition}[theorem]{Definition}
\newtheorem{remark}[theorem]{Remark}

\newcommand{\Area}{\mathrm{Area}}
\newcommand{\Vol}{\mathrm{Vol}}
\newcommand{\divv}{\mathrm{div}}
\DeclareMathOperator{\tr}{tr}

\title{A New Quasi-Local Mass for Trapped Surfaces}
\author{December 2025}

\begin{document}
\maketitle

\begin{abstract}
We construct a new quasi-local mass functional specifically designed for 
the Penrose inequality. The key innovation is using the NULL structure of 
trapped surfaces rather than the Riemannian structure. This mass equals 
$\sqrt{A/(16\pi)}$ for MOTS, exceeds this for trapped surfaces "inside," 
and has correct asymptotics.
\end{abstract}

%% ============================================================================
\section{The Design Requirements}
%% ============================================================================

We need a mass functional $m(\Sigma)$ satisfying:

\begin{enumerate}
    \item \textbf{MOTS Normalization:} For MOTS $\Sigma^*$ (where $\theta^+ = 0$):
    \begin{equation}
        m(\Sigma^*) = \sqrt{\frac{\Area(\Sigma^*)}{16\pi}}
    \end{equation}
    
    \item \textbf{Trapped Bound:} For trapped $\Sigma$ with $\theta^+ < 0$, 
    $\theta^- < 0$:
    \begin{equation}
        m(\Sigma) \ge \sqrt{\frac{\Area(\Sigma)}{16\pi}}
    \end{equation}
    
    \item \textbf{Asymptotics:} For large spheres $S_r$:
    \begin{equation}
        \lim_{r \to \infty} m(S_r) = M_{\text{ADM}}
    \end{equation}
    
    \item \textbf{Upper Bound:} For surfaces enclosing trapped regions:
    \begin{equation}
        m(\Sigma) \le M_{\text{ADM}}
    \end{equation}
\end{enumerate}

%% ============================================================================
\section{The Insight: Null Geometry}
%% ============================================================================

\begin{key}
\textbf{The Null Perspective}

A surface $\Sigma$ has two null expansions: $\theta^+$ (outgoing) and 
$\theta^-$ (ingoing).

\textbf{MOTS:} $\theta^+ = 0$ (marginally outer trapped)

\textbf{Trapped:} $\theta^+ < 0$ AND $\theta^- < 0$

The condition $\theta^+ = 0$ is the BOUNDARY of the trapped region.

For a mass functional, we should measure "how trapped" the surface is, 
i.e., how negative $\theta^+$ is.
\end{key}

%% ============================================================================
\section{Construction 1: The $\theta^+$-Weighted Mass}
%% ============================================================================

\begin{innovation}
\textbf{The $\theta^+$-Weighted Area Mass}

Define:
\begin{equation}
    m_\theta(\Sigma) = \frac{1}{4\pi}\int_\Sigma \sqrt{1 - \frac{\theta^+}{\theta^+_{\min}}} \, dA
\end{equation}

where $\theta^+_{\min} = \min_\Sigma \theta^+$ (most negative value).

For MOTS: $\theta^+ = 0$, so $m_\theta = \frac{1}{4\pi}\Area(\Sigma)$.
\end{innovation}

Actually, this doesn't have the right form. Let me try differently.

%% ============================================================================
\section{Construction 2: The Penrose Mass}
%% ============================================================================

\begin{innovation}
\textbf{The Penrose Mass Functional}

Motivated by the Penrose inequality itself, define:
\begin{equation}
    m_P(\Sigma) = \sqrt{\frac{\Area(\Sigma)}{16\pi}} \cdot 
    \exp\left(-\frac{1}{16\pi}\int_\Sigma \frac{\theta^+}{\sqrt{|\theta^+\theta^-|}} dA\right)
\end{equation}

For MOTS ($\theta^+ = 0$): $m_P = \sqrt{A/(16\pi)} \cdot e^0 = \sqrt{A/(16\pi)}$ ✓

For trapped ($\theta^+ < 0$): the exponent is positive, so $m_P > \sqrt{A/(16\pi)}$ ✓
\end{innovation}

But this definition has issues when $\theta^- = 0$ (marginally inner trapped).

%% ============================================================================
\section{Construction 3: The Minimal Null Mass}
%% ============================================================================

\begin{innovation}
\textbf{Key Idea: Use the Null Hypersurface}

For a surface $\Sigma$, consider the outgoing null hypersurface $\mathcal{N}^+$ 
emanating from $\Sigma$.

The "mass" seen along $\mathcal{N}^+$ changes according to the Raychaudhuri 
equation.

Define the mass via this null evolution.
\end{innovation}

\begin{definition}[Null Hawking Mass]
For the outgoing null hypersurface $\mathcal{N}^+$ with cross-sections 
$S_\lambda$ parameterized by affine parameter $\lambda$:
\begin{equation}
    m_N(S_\lambda) = \sqrt{\frac{\Area(S_\lambda)}{16\pi}}
    \left(1 + \frac{1}{16\pi}\int_{S_\lambda} \theta^+\theta^- dA_\lambda\right)
\end{equation}
\end{definition}

Wait, this is related to the Geroch mass but with a sign.

%% ============================================================================
\section{The Bondi-Sachs Mass}
%% ============================================================================

At null infinity, the Bondi mass is well-defined. For finite surfaces, 
we need a quasi-local version.

\begin{definition}[Hawking-Hayward Mass]
\begin{equation}
    m_{HH}(\Sigma) = \frac{r_A}{2}\left(1 + \frac{1}{16\pi}\int_\Sigma 
    \theta^+\theta^- dA\right)
\end{equation}

where $r_A = \sqrt{\Area/(4\pi)}$ is the areal radius.
\end{definition}

This is equivalent to our earlier spacetime Hawking mass!

\begin{proposition}[Rewriting]
\begin{align}
    m_{HH} &= \frac{1}{2}\sqrt{\frac{\Area}{4\pi}}\left(1 + \frac{1}{16\pi}
    \int \theta^+\theta^- dA\right)\\
    &= \sqrt{\frac{\Area}{16\pi}}\left(1 + \frac{1}{16\pi}\int \theta^+\theta^- dA\right)
\end{align}

For trapped: $\theta^+\theta^- > 0$, so $m_{HH} > \sqrt{A/(16\pi)}$ ✓
\end{proposition}

%% ============================================================================
\section{The Asymptotics Problem Revisited}
%% ============================================================================

The Hawking-Hayward mass diverges at infinity because the $\theta^+\theta^-$ 
integral grows.

\begin{key}
\textbf{Resolution: Subtraction}

At large $r$, the dominant contribution is from the "background" geometry.

Define:
\begin{equation}
    \bar{m}(\Sigma) = m_{HH}(\Sigma) - m_{HH}^{\text{flat}}(\Sigma)
\end{equation}

where $m_{HH}^{\text{flat}}$ is the value for the same surface embedded 
in flat spacetime.
\end{key}

\begin{proposition}[Flat Space Reference]
In flat spacetime (Minkowski), for a round sphere $S_r$:
\begin{itemize}
    \item $\theta^+ = \theta^- = 2/r$ (both positive, expanding)
    \item $\theta^+\theta^- = 4/r^2$
    \item $\int_{S_r} \theta^+\theta^- dA = \frac{4}{r^2} \cdot 4\pi r^2 = 16\pi$
\end{itemize}

So:
\begin{equation}
    m_{HH}^{\text{flat}}(S_r) = \sqrt{\frac{4\pi r^2}{16\pi}}(1 + 1) = r
\end{equation}
\end{proposition}

\begin{definition}[Renormalized Hawking-Hayward Mass]
\begin{equation}
    \bar{m}(\Sigma) = m_{HH}(\Sigma) - \sqrt{\frac{\Area}{16\pi}}
    \left(\frac{1}{16\pi}\int_\Sigma \theta^+_0\theta^-_0 dA\right)
\end{equation}

where $\theta^\pm_0$ are the expansions for the isometric embedding in 
Minkowski space.
\end{definition}

This is getting complicated. Let me try a cleaner approach.

%% ============================================================================
\section{The Two-Term Mass}
%% ============================================================================

\begin{innovation}
\textbf{The Two-Term Construction}

The problem is that $\theta^+\theta^- = H^2 - P^2$ has the "wrong" asymptotic 
behavior (the $H^2$ term dominates).

Idea: SEPARATE the contributions.

Define:
\begin{equation}
    m_{TT}(\Sigma) = \sqrt{\frac{A}{16\pi}}\left(1 - \frac{1}{16\pi}\int H^2 dA\right)
    + \sqrt{\frac{A}{16\pi}}\left(\frac{1}{8\pi}\int P^2 dA\right)
\end{equation}

Simplifying:
\begin{equation}
    m_{TT}(\Sigma) = \sqrt{\frac{A}{16\pi}}\left(1 - \frac{1}{16\pi}\int H^2 dA
    + \frac{1}{8\pi}\int P^2 dA\right)
\end{equation}
\end{innovation}

\begin{proposition}[Properties of $m_{TT}$]
\textbf{(1) Asymptotics:}

For large $S_r$: $H \sim 2/r$, $P \sim O(r^{-2})$

$\int H^2 dA \sim 16\pi - 32\pi M/r$, $\int P^2 dA \sim O(r^{-2})$

\begin{align}
    m_{TT}(S_r) &\approx \frac{r}{2}\left(1 - \frac{16\pi - 32\pi M/r}{16\pi}\right)\\
    &= \frac{r}{2} \cdot \frac{2M}{r} = M
\end{align}

Good asymptotics! ✓

\textbf{(2) For MOTS} ($\theta^+ = H + P = 0 \Rightarrow H = -P$):
\begin{align}
    m_{TT}(\Sigma^*) &= \sqrt{\frac{A}{16\pi}}\left(1 - \frac{1}{16\pi}\int P^2 dA
    + \frac{1}{8\pi}\int P^2 dA\right)\\
    &= \sqrt{\frac{A}{16\pi}}\left(1 + \frac{1}{16\pi}\int P^2 dA\right)\\
    &\ge \sqrt{\frac{A}{16\pi}}
\end{align}

Good! ✓

\textbf{(3) For Trapped} ($\theta^+ < 0$, $\theta^- < 0$):

$\theta^+ = H + P < 0$ and $\theta^- = H - P < 0$

Adding: $2H < 0 \Rightarrow H < 0$

From $\theta^+ < 0$: $H < -P$

If $P > 0$: $H < -P < 0$ (definitely negative)
If $P < 0$: $H < -P$, but $-P > 0$, so could have $H < 0$ or even $H > 0$... 
Actually $H < -P$ with $P < 0$ means $H < |P|$.

Let me analyze more carefully.
\end{proposition}

%% ============================================================================
\section{Detailed Analysis for Trapped Surfaces}
%% ============================================================================

\begin{lemma}[Trapped Surface Constraints]
For a trapped surface ($\theta^+ < 0$, $\theta^- < 0$):
\begin{align}
    H + P &< 0\\
    H - P &< 0
\end{align}

Adding: $2H < 0 \Rightarrow H < 0$.

Subtracting: $2P < 0 \Rightarrow P < 0$.

Wait, that's not right. Let me redo.

Adding: $2H < 0 \Rightarrow H < 0$ ✓

Subtracting (first minus second): $2P < 0$... No.

$\theta^+ - \theta^- = 2P$

If $\theta^+ < 0$ and $\theta^- < 0$, then $\theta^+ - \theta^-$ could be 
positive, negative, or zero depending on which is "more negative."
\end{lemma}

\begin{lemma}[Correct Constraints]
\textbf{Trapped:} $\theta^+ < 0$ AND $\theta^- < 0$

$\theta^+ + \theta^- = 2H < 0 \Rightarrow H < 0$ ✓

$\theta^+\theta^- > 0$ (product of two negatives) ✓

$P$ can have either sign! (No constraint from trapped condition alone)
\end{lemma}

\begin{proposition}[Two-Term Mass for Trapped]
\begin{equation}
    m_{TT} = \sqrt{\frac{A}{16\pi}}\left(1 - \frac{\int H^2}{16\pi} + \frac{\int P^2}{8\pi}\right)
\end{equation}

For trapped with $H < 0$:
\begin{itemize}
    \item $H^2 > 0$ term SUBTRACTS (bad for lower bound)
    \item $P^2 > 0$ term ADDS (good for lower bound)
\end{itemize}

Need: $\frac{\int P^2}{8\pi} > \frac{\int H^2}{16\pi}$

i.e., $2\int P^2 > \int H^2$

Is this true for trapped surfaces?
\end{proposition}

\begin{lemma}[H-P Relationship for Trapped]
From $|H| > |P|$ (NOT generally true!) we'd get $H^2 > P^2$.

But trapped only gives $H < 0$ and $|H + P| > 0$, $|H - P| > 0$.

There's no universal bound relating $H^2$ and $P^2$ for trapped surfaces.
\end{lemma}

%% ============================================================================
\section{A Different Approach: Capacity-Based Mass}
%% ============================================================================

\begin{innovation}
\textbf{The Capacity Mass}

Define using the electrostatic capacity:
\begin{equation}
    m_C(\Sigma) = \frac{\text{Cap}(\Sigma)}{4\pi}
\end{equation}

where $\text{Cap}(\Sigma)$ is the capacity of $\Sigma$ (energy of the 
equilibrium potential).

The capacity satisfies:
\begin{equation}
    \text{Cap}(\Sigma) \ge 2\sqrt{\pi \Area(\Sigma)}
\end{equation}

by the isoperimetric-type inequality.

Therefore:
\begin{equation}
    m_C(\Sigma) \ge \frac{2\sqrt{\pi A}}{4\pi} = \frac{1}{2}\sqrt{\frac{A}{\pi}}
    = \sqrt{\frac{A}{4\pi}}
\end{equation}

But we need $\sqrt{A/(16\pi)}$, which is SMALLER. So capacity gives a 
WEAKER bound.
\end{proposition}

%% ============================================================================
\section{The Correct Construction: Use MOTS Foliation}
%% ============================================================================

\begin{innovation}
\textbf{MOTS Foliation Mass}

Instead of defining mass for arbitrary surfaces, use the MOTS structure.

\textbf{Key insight:} The trapped region is foliated (approximately) by 
surfaces of constant $\theta^+$.

For each value $\theta^+_0 < 0$, let $\Sigma_{\theta^+_0}$ be the 
corresponding level surface.

Define:
\begin{equation}
    m(\theta^+_0) = \sqrt{\frac{\Area(\Sigma_{\theta^+_0})}{16\pi}}
\end{equation}

For MOTS: $\theta^+_0 = 0$, giving the MOTS area.

For more trapped: $\theta^+_0 < 0$, giving area of inner surfaces.
\end{innovation}

\begin{theorem_box}
\textbf{Penrose via MOTS Foliation}

If the area $\Area(\Sigma_{\theta^+_0})$ is INCREASING as $\theta^+_0 \to 0$, 
then for any trapped $\Sigma$ with $\theta^+ = \theta^+_0 < 0$:
\begin{equation}
    \Area(\Sigma) \le \Area(\Sigma^*)
\end{equation}

Combined with MOTS Penrose:
\begin{equation}
    M \ge \sqrt{\frac{\Area(\Sigma^*)}{16\pi}} \ge \sqrt{\frac{\Area(\Sigma)}{16\pi}}
\end{equation}
\end{theorem_box}

But this is just Area Dominance in disguise!

%% ============================================================================
\section{Final Realization}
%% ============================================================================

\begin{key}
\textbf{The Fundamental Truth}

Every quasi-local mass approach to Penrose must eventually compare the 
trapped surface $\Sigma$ to the MOTS $\Sigma^*$.

The comparison can be:
\begin{enumerate}
    \item Via AREA (Area Dominance) - requires $A(\Sigma) \le A(\Sigma^*)$
    \item Via MASS functional - requires $m(\Sigma) \le m(\Sigma^*)$ 
          and $m(\Sigma) \ge \sqrt{A/(16\pi)}$
\end{enumerate}

Option (2) still requires relating $m(\Sigma)$ to the area $A(\Sigma)$.

For any "reasonable" mass functional:
\begin{equation}
    m(\Sigma) = f(A(\Sigma), \text{geometry})
\end{equation}

To get $m(\Sigma) \ge \sqrt{A/(16\pi)}$, we need $f \ge \sqrt{A/(16\pi)}$ 
for trapped surfaces.

This is a GEOMETRIC CONSTRAINT, essentially equivalent to Area Dominance.
\end{key}

%% ============================================================================
\section{The Way Forward}
%% ============================================================================

\begin{innovation}
\textbf{The Resolution: Variational Characterization}

Instead of finding a mass functional, prove:

\textbf{Theorem (Variational Penrose):}
Among all initial data with a trapped surface of area $A$, the minimum 
ADM mass is $\sqrt{A/(16\pi)}$, achieved by Schwarzschild.

This is EQUIVALENT to Penrose but avoids:
\begin{enumerate}
    \item Area Dominance (no surface comparison)
    \item Quasi-local mass (no intermediate functional)
\end{enumerate}

\textbf{Proof approach:}
\begin{enumerate}
    \item Show the infimum exists
    \item Show the minimizer satisfies certain Euler-Lagrange equations
    \item Classify solutions to these equations
    \item Show only Schwarzschild qualifies
\end{enumerate}
\end{innovation}

This variational approach is the most promising path forward!

\end{document}
