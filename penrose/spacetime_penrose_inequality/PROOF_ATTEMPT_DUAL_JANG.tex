\documentclass[11pt]{article}
\usepackage{amsmath,amssymb,amsthm}
\usepackage[margin=1in]{geometry}
\usepackage{tcolorbox}
\usepackage{xcolor}

\newtheorem{theorem}{Theorem}[section]
\newtheorem{lemma}[theorem]{Lemma}
\newtheorem{proposition}[theorem]{Proposition}
\newtheorem{corollary}[theorem]{Corollary}
\newtheorem{definition}[theorem]{Definition}
\newtheorem{remark}[theorem]{Remark}
\newtheorem{conjecture}[theorem]{Conjecture}
\newtheorem{openproblem}{Open Problem}

\newtcolorbox{gap}{colback=red!5!white, colframe=red!75!black, title=\textbf{GAP}}
\newtcolorbox{success}{colback=green!5!white, colframe=green!75!black, title=\textbf{RIGOROUS}}

\title{\textbf{Proof Attempt: Dual Jang Equation Method}\\
\large Toward the Unconditional Spacetime Penrose Inequality}
\author{Research Notes}
\date{December 2025}

\begin{document}
\maketitle

\begin{abstract}
We attempt to prove the spacetime Penrose inequality for arbitrary trapped surfaces using a \textbf{dual Jang equation} that blows up at surfaces where $\theta^- = 0$ rather than $\theta^+ = 0$. The key insight is that combining the standard and dual Jang equations yields information about the mean curvature $H$, which has definite sign for trapped surfaces.
\end{abstract}

\tableofcontents

%==============================================================================
\section{The Dual Jang Equation}
%==============================================================================

\subsection{Setup}

Let $(M^3, g, k)$ be asymptotically flat initial data satisfying the dominant energy condition. Let $\Sigma_0$ be a closed trapped surface with $\theta^+ \leq 0$ and $\theta^- < 0$.

\begin{definition}[Standard Jang Equation]
The \textbf{standard Jang equation} for $f: M \to \mathbb{R}$ is:
\begin{equation}
\mathcal{J}(f) := H_\Gamma - \mathrm{tr}_\Gamma k = 0
\end{equation}
where $\Gamma = \{(x, f(x))\} \subset M \times \mathbb{R}$ is the graph with \textbf{upward} unit normal
\begin{equation}
N = \frac{1}{\sqrt{1+|\nabla f|^2}}(-\nabla f, 1)
\end{equation}
\end{definition}

\begin{lemma}[Jang as Null Expansion]
The Jang operator satisfies:
\begin{equation}
\mathcal{J}(f) = \theta^+_\Gamma
\end{equation}
the outgoing null expansion of the graph $\Gamma$.
\end{lemma}

\begin{success}
This is standard - see Schoen-Yau 1981, Bray-Khuri 2010.
\end{success}

\begin{definition}[Dual Jang Equation]
The \textbf{dual Jang equation} for $f^*: M \to \mathbb{R}$ is:
\begin{equation}
\mathcal{J}^*(f^*) := H_{\Gamma^*} + \mathrm{tr}_{\Gamma^*} k = 0
\end{equation}
where $\Gamma^* = \{(x, f^*(x))\}$ with \textbf{downward} unit normal
\begin{equation}
N^* = \frac{1}{\sqrt{1+|\nabla f^*|^2}}(\nabla f^*, -1)
\end{equation}
\end{definition}

\begin{lemma}[Dual Jang as Null Expansion]
The dual Jang operator satisfies:
\begin{equation}
\mathcal{J}^*(f^*) = \theta^-_{\Gamma^*}
\end{equation}
the ingoing null expansion of the graph $\Gamma^*$.
\end{lemma}

\begin{proof}
The null expansions are $\theta^\pm = H \mp \mathrm{tr}_\Gamma k$ with respect to the upward normal. With the downward normal, we have $H \to -H$ and $k \to -k$ in the trace (due to normal flip), giving:
\begin{equation}
\theta^-_{\Gamma^*} = -H_{\Gamma^*} - (-\mathrm{tr}_{\Gamma^*} k) = -H_{\Gamma^*} + \mathrm{tr}_{\Gamma^*} k
\end{equation}
Wait, this needs more care. Let me reconsider.

With the standard convention, for a surface with unit normal $\nu$:
\begin{align}
\theta^+ &= H + \mathrm{tr}_\Sigma k \\
\theta^- &= H - \mathrm{tr}_\Sigma k
\end{align}
where $H$ is the mean curvature with respect to $\nu$.

For the dual Jang with opposite normal orientation, the mean curvature changes sign: $H^* = -H$. But $\mathrm{tr}_\Sigma k$ depends on the surface, not the normal choice.

Actually, let's define things more carefully. Set:
\begin{equation}
\mathcal{J}^*(f^*) := -H_{\Gamma^*} + \mathrm{tr}_{\Gamma^*} k = -\theta^+_{\Gamma^*}
\end{equation}
where $H_{\Gamma^*}$ uses the same normal convention as $\mathcal{J}$.

Then $\mathcal{J}^* = 0$ means $\theta^+ = 0$, which is the same as standard Jang. That's not useful.

\textbf{Better approach:} Define dual Jang as solving for $\theta^- = 0$:
\begin{equation}
\mathcal{J}^*(f^*) := H_{\Gamma^*} - \mathrm{tr}_{\Gamma^*} k = \theta^-_{\Gamma^*} = 0
\end{equation}
This blows up where $\theta^- = 0$ (past MOTS), not where $\theta^+ = 0$.
\end{proof}

\begin{success}
The dual Jang equation $\mathcal{J}^* = \theta^- = 0$ is well-defined.
\end{success}

%==============================================================================
\section{Existence Theory for Dual Jang}
%==============================================================================

\begin{theorem}[Existence of Dual Jang Solution]
Let $(M^3, g, k)$ be asymptotically flat with a future trapped surface $\Sigma_0$ ($\theta^+ \leq 0$, $\theta^- < 0$). Then there exists a solution $f^*: M \setminus \Sigma_{\text{past}} \to \mathbb{R}$ to $\mathcal{J}^*(f^*) = 0$, where $\Sigma_{\text{past}}$ is the outermost surface with $\theta^- = 0$ (if it exists).
\end{theorem}

\begin{gap}
\textbf{This theorem requires proof.} The existence theory for the standard Jang equation (Schoen-Yau, Eichmair, Metzger) does not directly transfer. Key issues:
\begin{enumerate}
\item The barrier construction uses $\theta^+ = 0$ surfaces (MOTS). For dual Jang, we need $\theta^- = 0$ surfaces (past MOTS).
\item Past MOTS may not exist in general spacetimes.
\item The blow-up analysis differs: $f^* \to -\infty$ as we approach $\theta^- = 0$.
\end{enumerate}
\end{gap}

\begin{remark}[When Past MOTS Exist]
In time-symmetric data ($k = 0$), we have $\theta^+ = \theta^- = H$, so MOTS and past MOTS coincide. For general $k \neq 0$:
\begin{itemize}
\item Future trapped: $\theta^+ \leq 0$, $\theta^- < 0$
\item Past trapped: $\theta^+ < 0$, $\theta^- \leq 0$
\item MOTS: $\theta^+ = 0$
\item Past MOTS: $\theta^- = 0$
\end{itemize}
A spacetime can have MOTS without past MOTS, and vice versa.
\end{remark}

%==============================================================================
\section{The Combined Approach}
%==============================================================================

\subsection{Sum of Jang Equations}

\begin{proposition}[Sum Formula]
If $f$ solves $\mathcal{J}(f) = 0$ and $f^*$ solves $\mathcal{J}^*(f^*) = 0$ on the same region, then:
\begin{equation}
\mathcal{J}(f) + \mathcal{J}^*(f^*) = \theta^+_\Gamma + \theta^-_{\Gamma^*} = 0
\end{equation}
\end{proposition}

This doesn't directly give us $H$, since $\Gamma \neq \Gamma^*$ in general.

\subsection{Average Jang Function}

\begin{definition}[Average Jang]
Define the \textbf{average Jang function}:
\begin{equation}
\bar{f} := \frac{1}{2}(f + f^*)
\end{equation}
and the \textbf{average graph} $\bar{\Gamma} = \{(x, \bar{f}(x))\}$.
\end{definition}

\begin{lemma}[Average Operator]
The average Jang operator satisfies:
\begin{equation}
\frac{1}{2}(\mathcal{J}(\bar{f}) + \mathcal{J}^*(\bar{f})) = H_{\bar{\Gamma}}
\end{equation}
where $H_{\bar{\Gamma}}$ is the mean curvature of $\bar{\Gamma}$.
\end{lemma}

\begin{proof}
\begin{align}
\mathcal{J}(\bar{f}) &= H_{\bar{\Gamma}} + \mathrm{tr}_{\bar{\Gamma}} k = \theta^+_{\bar{\Gamma}} \\
\mathcal{J}^*(\bar{f}) &= H_{\bar{\Gamma}} - \mathrm{tr}_{\bar{\Gamma}} k = \theta^-_{\bar{\Gamma}}
\end{align}
Adding:
\begin{equation}
\mathcal{J}(\bar{f}) + \mathcal{J}^*(\bar{f}) = 2H_{\bar{\Gamma}}
\end{equation}
\end{proof}

\begin{success}
This is a straightforward calculation.
\end{success}

\subsection{The Key Question}

\begin{openproblem}[Average Jang Bound]
If $f$ solves $\mathcal{J}(f) = 0$ and $f^*$ solves $\mathcal{J}^*(f^*) = 0$, what can we say about:
\begin{equation}
\mathcal{J}(\bar{f}) + \mathcal{J}^*(\bar{f}) = 2H_{\bar{\Gamma}}
\end{equation}
Can we bound this in terms of $\theta^\pm$ of the original trapped surface?
\end{openproblem}

\begin{gap}
The average $\bar{f} = (f + f^*)/2$ does NOT satisfy either Jang equation. We cannot directly use the machinery developed for Jang solutions.

\textbf{Key obstacle:} Even if $\mathcal{J}(f) = 0$ and $\mathcal{J}^*(f^*) = 0$, the average $\bar{f}$ satisfies a \textbf{nonlinear} combination:
\begin{equation}
2H_{\bar{\Gamma}} = \mathcal{J}(\bar{f}) + \mathcal{J}^*(\bar{f}) \neq \mathcal{J}(f) + \mathcal{J}^*(f^*) = 0
\end{equation}
because $H_\Gamma$ is nonlinear in $f$.
\end{gap}

%==============================================================================
\section{Alternative: Product Structure}
%==============================================================================

\begin{proposition}[Product of Null Expansions]
For any surface $\Sigma$:
\begin{equation}
\theta^+ \cdot \theta^- = (H + \mathrm{tr}_\Sigma k)(H - \mathrm{tr}_\Sigma k) = H^2 - (\mathrm{tr}_\Sigma k)^2
\end{equation}
\end{proposition}

\begin{corollary}[Sign of Product]
For a trapped surface ($\theta^+ \leq 0$, $\theta^- < 0$):
\begin{equation}
\theta^+ \cdot \theta^- \geq 0
\end{equation}
with equality iff $\theta^+ = 0$ (MOTS).
\end{corollary}

\begin{lemma}[Mean Curvature Bound]
For a trapped surface:
\begin{equation}
H^2 \geq (\mathrm{tr}_\Sigma k)^2 - \theta^+\theta^-
\end{equation}
If $\theta^+\theta^- \geq (\mathrm{tr}_\Sigma k)^2$, then this gives no constraint. But actually:
\begin{equation}
H^2 = (\mathrm{tr}_\Sigma k)^2 + \theta^+\theta^- - 2\,\mathrm{tr}_\Sigma k \cdot H
\end{equation}
Hmm, this is getting circular. Let me try differently.
\end{lemma}

\begin{proposition}[Direct Relations]
From $\theta^\pm = H \pm \mathrm{tr}_\Sigma k$:
\begin{align}
H &= \frac{\theta^+ + \theta^-}{2} < 0 \quad \text{(trapped)} \\
\mathrm{tr}_\Sigma k &= \frac{\theta^+ - \theta^-}{2}
\end{align}
The sign of $\mathrm{tr}_\Sigma k$ depends on whether $|\theta^+| \gtrless |\theta^-|$.
\end{proposition}

%==============================================================================
\section{Scalar Curvature of Dual Jang Metric}
%==============================================================================

\begin{theorem}[Dual Jang Scalar Curvature]
Let $\bar{g}^*$ be the induced metric on the dual Jang graph $\Gamma^*$. Then:
\begin{equation}
R_{\bar{g}^*} = 2(\mu - J(\nu^*)) + 2|q^*|^2 - 2\,\mathrm{Div}_{\bar{g}^*}(q^*) + \text{(boundary terms at } \theta^- = 0 \text{)}
\end{equation}
where $\nu^*$ is the unit normal to $\Gamma^*$ in $M$ and $q^*$ is the dual Jang connection 1-form.
\end{theorem}

\begin{gap}
\textbf{The explicit formula needs derivation.} The standard Jang scalar curvature formula (Schoen-Yau) is:
\begin{equation}
R_{\bar{g}} = 2(\mu - J(\nu)) - 2|\hat{k} - \bar{k}|^2 + 2|q|^2 - 2\,\mathrm{Div}(q)
\end{equation}
The dual version should have similar structure but with $\theta^-$ replacing $\theta^+$ in the boundary terms.
\end{gap}

%==============================================================================
\section{Assessment}
%==============================================================================

\begin{tcolorbox}[colback=yellow!10!white, colframe=orange!75!black, title=\textbf{STATUS: INCOMPLETE}]
The Dual Jang approach has the following gaps:

\textbf{Gap 1:} Existence theory for dual Jang equation (requires past MOTS as barriers)

\textbf{Gap 2:} Average Jang $\bar{f} = (f+f^*)/2$ does not satisfy either equation

\textbf{Gap 3:} Scalar curvature formula for dual Jang metric needs derivation

\textbf{Gap 4:} Even with both Jang solutions, unclear how to combine them to get favorable sign

\textbf{Conclusion:} The dual Jang approach is \textbf{not sufficient} by itself. It provides new structure but does not directly solve the sign problem.
\end{tcolorbox}

\end{document}
