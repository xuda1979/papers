\documentclass[11pt]{article}
\usepackage{amsmath,amsthm,amssymb,mathrsfs}
\usepackage[margin=1in]{geometry}

\newtheorem{theorem}{Theorem}[section]
\newtheorem{lemma}[theorem]{Lemma}
\newtheorem{proposition}[theorem]{Proposition}
\newtheorem{corollary}[theorem]{Corollary}
\newtheorem{definition}[theorem]{Definition}
\newtheorem{conjecture}[theorem]{Conjecture}
\theoremstyle{remark}
\newtheorem{remark}[theorem]{Remark}
\newtheorem*{claim}{Claim}

\newcommand{\R}{\mathbb{R}}
\newcommand{\Sig}{\Sigma}
\newcommand{\tp}{\theta^+}
\newcommand{\tm}{\theta^-}
\newcommand{\Madm}{M_{\mathrm{ADM}}}
\newcommand{\Mirr}{M_{\mathrm{irr}}}

\title{\textbf{New Monotone Quantities for Trapped Surfaces}\\
\large Searching for the Right Functional}
\author{Working Document}
\date{December 2025}

\begin{document}
\maketitle

\begin{abstract}
We systematically search for monotone quantities that could prove the Penrose inequality 
for trapped surfaces. We analyze area-derived functionals, capacity-based quantities, 
and spectral invariants. We find one promising candidate and analyze its properties.
\end{abstract}

\section{Requirements for a Monotone Quantity}

\subsection{What We Need}

A functional $F: \{\text{2-surfaces}\} \to \R$ such that:

\textbf{(M1) Initial bound:} $F(\Sig_0) \geq c_1 \sqrt{A(\Sig_0)}$ for trapped $\Sig_0$

\textbf{(M2) Monotonicity:} $F(\Sig_t) \geq F(\Sig_0)$ along some flow

\textbf{(M3) Terminal bound:} $F(\Sig^*) \leq c_2 \Madm$ for MOTS $\Sig^*$

Then: $c_2 \Madm \geq F(\Sig^*) \geq F(\Sig_0) \geq c_1\sqrt{A(\Sig_0)}$.

\subsection{Why Area Fails}

Area satisfies (M1) trivially: $A(\Sig_0) = \sqrt{A(\Sig_0)}^2 \geq \sqrt{A(\Sig_0)}$ for $A \geq 1$.

Area satisfies (M3): For MOTS, $A(\Sig^*) \leq 16\pi\Madm^2$ (Penrose for MOTS).

But area FAILS (M2): For trapped surfaces, area \emph{decreases} along all smooth flows.

\section{Candidate 1: Weighted Area}

\subsection{Definition}

Let $w: M \to (0, \infty)$ be a positive weight function. Define:
\begin{equation}
F_w(\Sig) = \int_\Sig w \, dA
\end{equation}

\subsection{Evolution}

Under mean curvature flow $\partial_t \Sig = H\nu$:
\begin{equation}
\frac{d}{dt}F_w(\Sig) = \int_\Sig (w H^2 + H \partial_\nu w) \, dA
\end{equation}

For this to be positive when $H < 0$, we need:
\[
w H^2 + H \partial_\nu w > 0 \implies \partial_\nu w < w|H|
\]

\textbf{Problem:} This requires $w$ to decrease outward faster than $|H|$, but then 
$w \to 0$ at infinity, making $F_w$ useless for comparing to ADM mass.

\subsection{Exponential Weight}

Try $w = e^{\phi}$ where $\phi$ solves some PDE.

\[
\frac{d}{dt}F_w = \int_\Sig e^\phi (H^2 + H\partial_\nu\phi) \, dA
\]

Need $H^2 + H\partial_\nu\phi > 0$, i.e., $\partial_\nu\phi > -|H|$ when $H < 0$.

If $\phi$ increases outward (natural for potential), then $\partial_\nu\phi > 0$ and 
$H^2 + H\partial_\nu\phi = H(H + \partial_\nu\phi)$. 

For $H < 0$: need $H + \partial_\nu\phi < 0$, i.e., $\partial_\nu\phi < |H|$.

So we need: $0 < \partial_\nu\phi < |H|$ to make the expression positive.

\textbf{This is possible in principle but requires $\phi$ to be constructed adaptively.}

\section{Candidate 2: Hawking Mass}

\subsection{Definition}

The Hawking mass of a 2-surface $\Sig$:
\begin{equation}
m_H(\Sig) = \sqrt{\frac{A(\Sig)}{16\pi}}\left(1 - \frac{1}{16\pi}\int_\Sig H^2 \, dA\right)
\end{equation}

\subsection{Properties}

For a MOTS with $H = 0$: $m_H(\Sig^*) = \sqrt{A(\Sig^*)/(16\pi)}$.

For a trapped surface: $H < 0$, so $H^2 > 0$, and:
\[
m_H(\Sig_0) = \sqrt{\frac{A(\Sig_0)}{16\pi}}\left(1 - \frac{1}{16\pi}\int H^2 \, dA\right) < \sqrt{\frac{A(\Sig_0)}{16\pi}}
\]

\textbf{The Hawking mass of a trapped surface is LESS than} $\sqrt{A/(16\pi)}$.

\subsection{Monotonicity}

Under inverse mean curvature flow (where it exists):
\begin{equation}
\frac{d}{dt}m_H \geq 0
\end{equation}
if the ambient space has non-negative scalar curvature.

\textbf{But:} IMCF doesn't exist for $H < 0$.

\subsection{Modification: Null Hawking Mass}

Define the null Hawking mass using $\tp$ instead of $H$:
\begin{equation}
m_H^+(\Sig) = \sqrt{\frac{A(\Sig)}{16\pi}}\left(1 - \frac{1}{16\pi}\int_\Sig (\tp)^2 \, dA\right)
\end{equation}

For MOTS ($\tp = 0$): $m_H^+(\Sig^*) = \sqrt{A/(16\pi)}$.

For trapped surfaces ($\tp < 0$): $m_H^+(\Sig_0) < \sqrt{A/(16\pi)}$.

\textbf{Same problem: monotonicity fails.}

\section{Candidate 3: Isoperimetric Deficit}

\subsection{Definition}

The isoperimetric deficit:
\begin{equation}
\delta(\Sig) = A(\Sig) - 4\pi r(\Sig)^2
\end{equation}
where $r(\Sig)$ is some ``effective radius.''

Different choices:
\begin{itemize}
\item $r = \sqrt{A/(4\pi)}$ (area radius): gives $\delta = 0$ always (trivial)
\item $r = $ capacity/$(4\pi)$: $\delta = A - (\text{Cap})^2/(4\pi)$
\item $r = $ ADM mass: $\delta = A - 4\pi\Madm^2$
\end{itemize}

For the last choice, Penrose inequality is $\delta \leq 0$ for horizon cross-sections.

\subsection{For Trapped Surfaces}

We want $A(\Sig_0) \leq 16\pi\Madm^2$, i.e., $\delta = A - 16\pi\Madm^2 \leq 0$.

This is exactly the Penrose inequality! No new content.

\section{Candidate 4: Capacity}

\subsection{Definition}

The harmonic capacity:
\begin{equation}
\text{Cap}(\Sig) = \inf_{u} \int_{M \setminus \Omega_\Sig} |\nabla u|^2 \, dV
\end{equation}
where $u|_\Sig = 1$ and $u \to 0$ at infinity.

For round spheres in $\R^3$: $\text{Cap}(S_r) = 4\pi r$.

\subsection{Capacity-Area Inequality}

\begin{theorem}[Classical]
For any surface $\Sig$ in $\R^3$:
\begin{equation}
\text{Cap}(\Sig) \geq \sqrt{4\pi A(\Sig)}
\end{equation}
with equality for round spheres.
\end{theorem}

\subsection{Capacity-Mass Inequality}

In asymptotically flat manifolds with $R \geq 0$:
\begin{equation}
\Madm \geq \frac{\text{Cap}(\Sig)}{4\pi}
\end{equation}

\subsection{Combining}

For trapped $\Sig_0$ in $(M, g)$ with $R \geq 0$:
\[
\Madm \geq \frac{\text{Cap}(\Sig_0)}{4\pi} \geq \frac{\sqrt{4\pi A(\Sig_0)}}{4\pi} = \frac{\sqrt{A(\Sig_0)}}{\sqrt{4\pi}} = \sqrt{\frac{A(\Sig_0)}{4\pi}}
\]

\textbf{This gives} $\Madm \geq \sqrt{A/(4\pi)}$, not $\sqrt{A/(16\pi)}$.

The constant is OFF by factor of 2!

\subsection{Why the Factor of 4}

Penrose inequality: $\Madm \geq \sqrt{A/(16\pi)}$ involves the Schwarzschild relation $M = r/2$ for $A = 4\pi r^2$.

Capacity inequality: $\text{Cap} = 4\pi r$ for sphere of radius $r$.

So $\text{Cap}/(4\pi) = r = 2M$, giving $\Madm = \text{Cap}/(8\pi)$.

The standard capacity-mass bound $\Madm \geq \text{Cap}/(4\pi)$ is weaker by factor of 2.

\textbf{Question:} Is there a tighter capacity-mass bound?

\section{Candidate 5: Modified Capacity}

\subsection{The $p$-Capacity}

Define the $p$-capacity:
\begin{equation}
\text{Cap}_p(\Sig) = \inf_u \int_{M \setminus \Omega_\Sig} |\nabla u|^p \, dV
\end{equation}

For $p = 1$: this is the perimeter (total variation), related to area.

For $p = 2$: harmonic capacity.

For $p \to \infty$: related to Cheeger constant.

\subsection{Limiting Behavior}

As $p \to 1$:
\begin{equation}
\text{Cap}_p(\Sig)^{1/(p-1)} \to A(\Sig)
\end{equation}
(up to constants).

As $p \to 2$:
\begin{equation}
\text{Cap}_p(\Sig) \to \text{Cap}(\Sig)
\end{equation}

\subsection{Optimal $p$}

There should be a $p^* \in (1, 2)$ such that:
\begin{equation}
\Madm = c(p^*) \cdot \text{Cap}_{p^*}(\Sig)^{1/(3-p^*)}
\end{equation}
with the right constant for the Penrose inequality.

\textbf{Problem:} Finding $p^*$ and proving the corresponding inequality requires 
new estimates that are not currently known.

\section{Candidate 6: Spectral Quantity}

\subsection{The First Eigenvalue}

Let $\lambda_1(\Sig)$ be the first Dirichlet eigenvalue of the Laplacian on the 
region $\Omega_\Sig$ enclosed by $\Sig$:
\begin{equation}
-\Delta u = \lambda_1 u, \quad u|_\Sig = 0, \quad u > 0 \text{ in } \Omega_\Sig
\end{equation}

\subsection{Faber-Krahn Inequality}

\begin{theorem}[Faber-Krahn]
Among all domains of given volume, the ball minimizes $\lambda_1$:
\begin{equation}
\lambda_1(\Omega) \geq \lambda_1(B) = \frac{c_n}{V(\Omega)^{2/n}}
\end{equation}
\end{theorem}

For $n = 3$: $\lambda_1 \geq c_3 / V^{2/3}$.

\subsection{Relation to Mass}

There's no direct relation between $\lambda_1$ and ADM mass in general.

\textbf{However:} On Schwarzschild, the quasi-local mass inside radius $r$ is $M$ (constant), 
and $\lambda_1$ of the ball of radius $r$ scales as $r^{-2}$.

No useful bound emerges.

\section{Candidate 7: Willmore Energy}

\subsection{Definition}

The Willmore energy:
\begin{equation}
W(\Sig) = \int_\Sig H^2 \, dA
\end{equation}

For spheres: $W(S_r) = 4\pi \cdot (2/r)^2 \cdot 4\pi r^2 = 16\pi$.

\subsection{Willmore Inequality}

\begin{theorem}[Willmore]
For any closed surface $\Sig$ in $\R^3$:
\begin{equation}
W(\Sig) \geq 16\pi
\end{equation}
with equality for round spheres.
\end{theorem}

\subsection{For Trapped Surfaces}

For trapped $\Sig_0$ with $H < 0$:
\[
W(\Sig_0) = \int H^2 \, dA \geq 16\pi
\]

So: $\int H^2 \, dA \geq 16\pi$, giving $\langle H^2 \rangle \cdot A \geq 16\pi$, 
hence $|H|_{\text{rms}} \geq 4\sqrt{\pi/A}$.

\textbf{This bounds $|H|$ from below, not $A$ from above.} Not useful for Penrose.

\section{Candidate 8: The Promising Direction}

\subsection{Renormalized Hawking Mass}

Define:
\begin{equation}
\tilde{m}(\Sig) = m_H(\Sig) \cdot e^{\int_0^t f(\Sig_s) \, ds}
\end{equation}
where $f$ is chosen to make $\tilde{m}$ monotone.

\subsection{Required $f$}

Under IMCF (where it exists):
\[
\frac{d m_H}{dt} = m_H \cdot g(\Sig_t)
\]
for some $g$ depending on curvature.

If $g < 0$ (Hawking mass decreasing), we need $f > |g|$ to compensate.

\textbf{The challenge:} Finding $f$ that works for trapped surfaces where IMCF doesn't exist.

\subsection{A New Proposal}

\begin{definition}[Trapping-Renormalized Mass]
\begin{equation}
\mathcal{M}(\Sig) := \sqrt{\frac{A(\Sig)}{16\pi}} \cdot \exp\left(\frac{1}{A}\int_\Sig \frac{|\tp|}{|H|} \, dA\right)
\end{equation}
when $H \neq 0$ everywhere on $\Sig$.
\end{definition}

\textbf{Properties:}

For MOTS ($\tp = 0$): $\mathcal{M}(\Sig^*) = \sqrt{A/(16\pi)} = \Mirr$.

For trapped surfaces: $|\tp|/|H| = |H + \tr k|/|H| = |1 + \tr k/H|$.

If $\tr k/H > 0$ (same sign): $|\tp|/|H| < 1$.
If $\tr k/H < 0$ (opposite sign): $|\tp|/|H| > 1$.

\begin{conjecture}
For trapped surfaces in spacetimes satisfying DEC:
\begin{equation}
\mathcal{M}(\Sig_0) \leq \Madm
\end{equation}
\end{conjecture}

\textbf{Status:} This is a new quantity. Its monotonicity properties are unknown. 
The conjecture would imply Penrose inequality since $\mathcal{M} \geq \sqrt{A/(16\pi)}$ 
when the exponential factor $\geq 1$.

\section{Analysis of the New Quantity}

\subsection{When Does $\mathcal{M} \geq \sqrt{A/(16\pi)}$?}

The exponential factor $\exp\left(\frac{1}{A}\int \frac{|\tp|}{|H|} \, dA\right) \geq 1$ always.

So $\mathcal{M}(\Sig) \geq \sqrt{A/(16\pi)}$ always!

\textbf{This is good for (M1).}

\subsection{Terminal Value}

For MOTS: $\tp = 0$, so $|\tp|/|H| = 0$, giving $\mathcal{M} = \sqrt{A/(16\pi)}$.

Combined with Penrose for MOTS: $\Madm \geq \sqrt{A(\Sig^*)/(16\pi)} = \mathcal{M}(\Sig^*)$.

\textbf{This is good for (M3).}

\subsection{Monotonicity?}

The key question: Is $\mathcal{M}$ monotone along some flow?

Under a flow from $\Sig_0$ (trapped) to $\Sig^*$ (MOTS):
\begin{itemize}
\item $A$ may decrease then increase, or just decrease
\item $|\tp|/|H|$ goes from some positive value to 0
\item The exponential factor decreases from $> 1$ to $1$
\end{itemize}

For $\mathcal{M}$ to be monotone increasing, we need:
\[
\frac{d}{dt}\log\mathcal{M} = \frac{1}{2}\frac{\dot{A}}{A} + \frac{d}{dt}\left(\frac{1}{A}\int \frac{|\tp|}{|H|}\right) \geq 0
\]

The first term is negative (area decreases).
The second term involves complicated derivatives of $\tp$, $H$, and $A$.

\textbf{There is no obvious reason for this to be positive.}

\section{Conclusion}

\textbf{Summary of candidates:}

\begin{center}
\begin{tabular}{|l|c|c|c|}
\hline
Quantity & (M1) Initial & (M2) Monotone & (M3) Terminal \\
\hline
Area $A$ & Yes & \textbf{No} & Yes \\
Weighted Area & Unclear & Possible & Unclear \\
Hawking Mass & No ($<$ needed) & Yes (IMCF) & Yes \\
Capacity & Yes (weaker) & Yes & Yes (weaker const) \\
$p$-Capacity & Unknown & Unknown & Unknown \\
Spectral & No relation & N/A & No relation \\
Willmore & No (wrong direction) & No & No \\
$\mathcal{M}$ (new) & Yes & \textbf{Unknown} & Yes \\
\hline
\end{tabular}
\end{center}

\textbf{The new quantity $\mathcal{M}$} satisfies (M1) and (M3) but monotonicity is unproven.

\textbf{If $\mathcal{M}$ is monotone along some flow, Penrose 1973 is solved.}

\textbf{Open Problems:}
\begin{enumerate}
\item Find a flow under which $\mathcal{M}$ is monotone
\item Or prove $\mathcal{M}$ is NOT monotone and the approach fails
\item Or find a better quantity satisfying all three conditions
\end{enumerate}

\end{document}
