% =========================================================================
%     NOVEL EXPLORATION 3: THE SPACETIME HARMONIC ENERGY
%
%     A new functional combining 4D and 3D perspectives
%
%     Author: Da Xu
%     Date: December 2025
% =========================================================================

\documentclass[12pt]{article}
\usepackage{amsmath,amsthm,amssymb}
\usepackage{mathrsfs}
\usepackage{tcolorbox}

\theoremstyle{plain}
\newtheorem{theorem}{Theorem}[section]
\newtheorem{lemma}[theorem]{Lemma}
\newtheorem{proposition}[theorem]{Proposition}
\newtheorem{corollary}[theorem]{Corollary}
\newtheorem{conjecture}[theorem]{Conjecture}

\theoremstyle{definition}
\newtheorem{definition}[theorem]{Definition}
\newtheorem{remark}[theorem]{Remark}
\newtheorem{observation}[theorem]{Key Observation}

\newcommand{\ADM}{\mathrm{ADM}}
\newcommand{\tr}{\mathrm{tr}}
\newcommand{\Div}{\mathrm{div}}
\newcommand{\Area}{\mathrm{Area}}

\title{\textbf{Novel Exploration 3: Spacetime Harmonic Energy}}
\author{Da Xu}
\date{December 2025}

\begin{document}
\maketitle

\section{Motivation: Combining Null Structure}

The trapped condition involves BOTH null expansions:
\[
    \theta^+ = H + \tr_\Sigma k \leq 0, \quad \theta^- = H - \tr_\Sigma k < 0
\]

Previous approaches use either $H$ or $\tr_\Sigma k$ separately.

\textbf{New idea:} Build functionals that intrinsically use both null structures.

\section{The Null Energy Functional}

\subsection{Definition}

\begin{definition}
The \textbf{null energy} of a surface $\Sigma$ is:
\[
    E_\pm(\Sigma) = \int_\Sigma |\theta^\pm|^2 \, dA
\]
\end{definition}

For trapped surfaces: $E_+, E_- > 0$.

\subsection{The Combined Null Energy}

\begin{definition}
The \textbf{combined null energy} is:
\[
    E_{\text{null}}(\Sigma) = \sqrt{E_+(\Sigma) \cdot E_-(\Sigma)} = \sqrt{\int |\theta^+|^2 \, dA \cdot \int |\theta^-|^2 \, dA}
\]
\end{definition}

By Cauchy-Schwarz:
\[
    E_{\text{null}}(\Sigma) \geq \left|\int \theta^+\theta^- \, dA\right|
\]

\subsection{Relation to Geometry}

\[
    \int \theta^+\theta^- \, dA = \int (H^2 - (\tr_\Sigma k)^2) \, dA
\]

For MOTS ($\theta^+ = 0$): $E_{\text{null}} = 0$.

For trapped: $E_{\text{null}} > 0$.

\section{The Harmonic Radius}

\subsection{Definition}

\begin{definition}
The \textbf{harmonic radius} of $\Sigma$ is defined by:
\[
    r_H(\Sigma) = \sqrt{\frac{\Area(\Sigma)}{4\pi}}
\]
This is the radius of a round sphere with the same area.
\end{definition}

\subsection{The Harmonic Mass}

\begin{definition}
The \textbf{harmonic mass} is:
\[
    m_{\text{harm}}(\Sigma) = \frac{r_H(\Sigma)}{2}\left(1 - \frac{E_{\text{null}}(\Sigma)}{(4\pi)^2}\right)
\]
\end{definition}

\subsection{Analysis}

For Schwarzschild horizon ($r = 2M$):
\begin{itemize}
    \item $r_H = 2M$
    \item $\theta^+ = 0$ (MOTS), so $E_{\text{null}} = 0$
    \item $m_{\text{harm}} = \frac{2M}{2}(1 - 0) = M$ ✓
\end{itemize}

For trapped surfaces inside:
\begin{itemize}
    \item $r_H < 2M$
    \item $E_{\text{null}} > 0$
    \item $m_{\text{harm}} < \frac{r_H}{2} < M$
\end{itemize}

\section{A New Approach: The Trapping Potential}

\subsection{Observation}

The trapped condition can be written as:
\[
    |\tr_\Sigma k| > |H|
\]
on surfaces where $H$ and $\tr_\Sigma k$ have opposite signs.

\subsection{The Trapping Function}

\begin{definition}
The \textbf{trapping function} on $\Sigma$ is:
\[
    T(p) = |\tr_\Sigma k(p)|^2 - H(p)^2 = -\theta^+(p)\theta^-(p)
\]
\end{definition}

For trapped surfaces: $T > 0$ everywhere (since $\theta^+\theta^- > 0$ and we take negative).

Wait, let me reconsider. $\theta^+\theta^- > 0$ for trapped, so $-\theta^+\theta^- < 0$.

But $(\tr_\Sigma k)^2 - H^2 = -\theta^+\theta^-$ also.

So: $(\tr_\Sigma k)^2 - H^2 < 0$ for trapped surfaces? Let me check:
\[
    \theta^+\theta^- = (H + \tr k)(H - \tr k) = H^2 - (\tr k)^2
\]

For trapped: $\theta^+, \theta^- < 0$, so $\theta^+\theta^- > 0$.

Therefore: $H^2 - (\tr k)^2 > 0$, i.e., $H^2 > (\tr k)^2$.

So $|H| > |\tr_\Sigma k|$ for trapped surfaces!

\begin{observation}
For trapped surfaces: $|H| > |\tr_\Sigma k|$.

This means the mean curvature dominates!
\end{observation}

This is interesting because our obstruction was about $\tr_\Sigma k < 0$.

But actually, for trapped surfaces, $H$ is MORE negative than $\tr_\Sigma k$ in some sense.

\section{Revisiting the Sign}

Let's be very careful:
\begin{align}
    \theta^+ &= H + \tr_\Sigma k \leq 0 \\
    \theta^- &= H - \tr_\Sigma k < 0
\end{align}

Adding: $2H = \theta^+ + \theta^- < 0$, so $H < 0$. ✓

Subtracting: $2\tr_\Sigma k = \theta^+ - \theta^-$.

If $\theta^+ < 0$ and $\theta^- < 0$:
\begin{itemize}
    \item If $|\theta^+| > |\theta^-|$: $\tr_\Sigma k < 0$
    \item If $|\theta^+| < |\theta^-|$: $\tr_\Sigma k > 0$
    \item If $|\theta^+| = |\theta^-|$: $\tr_\Sigma k = 0$
\end{itemize}

\textbf{So the sign of $\tr_\Sigma k$ depends on which null expansion is ``more negative''!}

\section{The Null Ratio Approach}

\subsection{Definition}

\begin{definition}
The \textbf{null ratio} is:
\[
    \rho = \frac{|\theta^+|}{|\theta^-|} = \frac{-\theta^+}{-\theta^-} = \frac{\theta^+}{\theta^-}
\]
\end{definition}

For trapped surfaces: $\rho > 0$.

\subsection{Sign of $\tr_\Sigma k$ in Terms of $\rho$}

\[
    \tr_\Sigma k = \frac{\theta^+ - \theta^-}{2} = \frac{\theta^-(\rho - 1)}{2}
\]

Since $\theta^- < 0$:
\begin{itemize}
    \item $\rho > 1 \Rightarrow \tr_\Sigma k < 0$ (unfavorable)
    \item $\rho < 1 \Rightarrow \tr_\Sigma k > 0$ (favorable)
    \item $\rho = 1 \Rightarrow \tr_\Sigma k = 0$ (time-symmetric)
\end{itemize}

\subsection{The $\rho$-Averaged Area}

\begin{definition}
The \textbf{$\rho$-averaged area} is:
\[
    A_\rho(\Sigma) = \int_\Sigma \frac{2}{1 + \rho} \, dA = \int_\Sigma \frac{2\theta^-}{\theta^+ + \theta^-} \, dA
    = \int_\Sigma \frac{\theta^-}{H} \, dA
\]
\end{definition}

Since $\theta^- < 0$ and $H < 0$: $\theta^-/H > 0$.

For unfavorable ($\rho > 1$): $\frac{2}{1+\rho} < 1$, so $A_\rho < \Area$.

For favorable ($\rho < 1$): $\frac{2}{1+\rho} > 1$, so $A_\rho > \Area$.

\section{A Potential Breakthrough}

\begin{observation}
For unfavorable trapped surfaces ($\tr_\Sigma k < 0$, $\rho > 1$):
\[
    A_\rho(\Sigma) < \Area(\Sigma)
\]

If we could prove:
\[
    M_{\ADM} \geq \sqrt{\frac{A_\rho(\Sigma)}{16\pi}}
\]

This would NOT give Penrose inequality directly!

But... what if there's a complementary bound?
\end{observation}

\subsection{The Complementary Ratio}

Define $\tilde{\rho} = 1/\rho = \theta^-/\theta^+$.

The complementary averaged area:
\[
    A_{\tilde{\rho}}(\Sigma) = \int_\Sigma \frac{2}{1 + \tilde{\rho}} \, dA = \int_\Sigma \frac{2\theta^+}{\theta^+ + \theta^-} \, dA
    = \int_\Sigma \frac{\theta^+}{H} \, dA
\]

\subsection{Sum of Averaged Areas}

\[
    A_\rho + A_{\tilde{\rho}} = \int_\Sigma \left(\frac{\theta^-}{H} + \frac{\theta^+}{H}\right) dA 
    = \int_\Sigma \frac{\theta^+ + \theta^-}{H} \, dA = \int_\Sigma \frac{2H}{H} \, dA = 2\Area(\Sigma)
\]

So:
\[
    A_\rho + A_{\tilde{\rho}} = 2\Area(\Sigma)
\]

\subsection{Geometric Mean}

\[
    \sqrt{A_\rho \cdot A_{\tilde{\rho}}} \leq \frac{A_\rho + A_{\tilde{\rho}}}{2} = \Area(\Sigma)
\]

by AM-GM!

\begin{proposition}
\[
    \sqrt{A_\rho \cdot A_{\tilde{\rho}}} \leq \Area(\Sigma)
\]
with equality when $\rho = 1$ (time-symmetric).
\end{proposition}

\section{The Dual Penrose Approach}

\begin{conjecture}[Dual Penrose]
For trapped surfaces:
\[
    M_{\ADM}^2 \geq \frac{1}{16\pi}\sqrt{A_\rho \cdot A_{\tilde{\rho}}}
\]
\end{conjecture}

If true:
\[
    M_{\ADM}^2 \geq \frac{1}{16\pi}\sqrt{A_\rho \cdot A_{\tilde{\rho}}} \leq \frac{\Area(\Sigma)}{16\pi}
\]

Wait, this gives $M_{\ADM}^2 \leq \frac{\Area}{16\pi}$, which is backwards!

Let me reconsider...

\section{Back to Basics}

The issue is:
\begin{itemize}
    \item Penrose: $M \geq \sqrt{A/(16\pi)}$, i.e., $A \leq 16\pi M^2$
    \item We want to bound $A$ from above
\end{itemize}

Any quantity $\tilde{A} \leq A$ gives:

``$M \geq \sqrt{\tilde{A}/(16\pi)}$'' which is weaker than Penrose.

We need $\tilde{A} \geq A$ (larger), and then prove $M \geq \sqrt{\tilde{A}/(16\pi)}$.

\section{The Expanded Area}

\begin{definition}
The \textbf{trapping-expanded area} is:
\[
    A^*(\Sigma) = \Area(\Sigma) \cdot \frac{\max(\rho, 1)}{\rho}
\]
\end{definition}

For unfavorable ($\rho > 1$): $A^* = \Area \cdot 1 = \Area$.

Hmm, that doesn't expand.

Let me try:
\[
    A^*(\Sigma) = \Area(\Sigma) \cdot \max(1, \rho)
\]

For unfavorable ($\rho > 1$): $A^* = \Area \cdot \rho > \Area$. ✓

\begin{conjecture}[Expanded Penrose]
\[
    M_{\ADM} \geq \sqrt{\frac{\Area(\Sigma)}{16\pi}}
\]
follows from
\[
    M_{\ADM} \geq \sqrt{\frac{A^*(\Sigma)}{16\pi \cdot \max(1,\rho)^2}}
\]
\end{conjecture}

This is just Penrose rewritten. Not useful.

\section{Fundamental Insight}

\begin{tcolorbox}[colback=red!10, colframe=red!75!black]
\textbf{The real insight:}

Local modifications using $\theta^\pm$ cannot strengthen Penrose because:
\begin{itemize}
    \item Any combination of $\theta^\pm$ is determined on $\Sigma$ alone
    \item $M_{\ADM}$ is determined at infinity
    \item The connection requires GLOBAL information
\end{itemize}

The $\rho$-averaged areas are interesting geometrically but don't bridge the local-global gap.
\end{tcolorbox}

\section{A Truly New Direction: The Light-Cone Mass}

\subsection{Definition}

Shoot null rays from $\Sigma$ to infinity. Define:

\begin{definition}
The \textbf{light-cone mass} of $\Sigma$ is:
\[
    m_{\text{LC}}(\Sigma) = \lim_{r \to \infty} \frac{r}{2}\left(1 - \frac{\text{solid angle seen from } \Sigma}{4\pi}\right)
\]
\end{definition}

This measures how much the light cone from $\Sigma$ is focused/defocused by geometry.

\subsection{Properties}

For Schwarzschild with $\Sigma$ at $r = r_0$:
\begin{itemize}
    \item Light rays are bent by mass $M$
    \item At infinity, the solid angle is reduced
    \item $m_{\text{LC}} = M$ (should recover ADM mass)
\end{itemize}

\subsection{For Trapped Surfaces}

Trapped means outgoing light rays converge initially ($\theta^+ < 0$).

But they might eventually diverge after leaving the trapped region!

The total focusing depends on:
\begin{enumerate}
    \item Initial convergence ($\theta^+|_{\Sigma}$)
    \item Matter encountered ($R_{\mu\nu}\ell^\mu\ell^\nu$)
    \item Geometry traversed
\end{enumerate}

\begin{conjecture}
\[
    m_{\text{LC}}(\Sigma_0) \geq \sqrt{\frac{\Area(\Sigma_0)}{16\pi}}
\]
\end{conjecture}

And $m_{\text{LC}}(\Sigma_0) \leq M_{\ADM}$ (mass seen from $\Sigma$ is at most total mass).

If both hold: Penrose!

\section{Conclusion}

\begin{tcolorbox}[colback=blue!10, colframe=blue!75!black]
\textbf{Most Promising New Direction:}

The \textbf{light-cone mass} approach:
\begin{enumerate}
    \item Naturally incorporates null structure
    \item Connects $\Sigma$ to infinity via light rays
    \item Physical interpretation: mass seen by observers on $\Sigma$
\end{enumerate}

\textbf{Needs:} Rigorous definition and analysis of $m_{\text{LC}}$.
\end{tcolorbox}

\end{document}
