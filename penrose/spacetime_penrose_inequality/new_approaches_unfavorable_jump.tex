% NEW MATHEMATICAL APPROACHES TO THE UNFAVORABLE JUMP CASE
%
% This document explores rigorous new approaches to prove the Penrose inequality
% for trapped surfaces with tr_Σ k < 0 (unfavorable jump condition).
%
% CURRENT STATUS: The Area Monotonicity approach is INVALID.
% We need genuinely new mathematics.

\documentclass{article}
\usepackage{amsmath,amsthm,amssymb}
\newtheorem{theorem}{Theorem}
\newtheorem{lemma}{Lemma}
\newtheorem{proposition}{Proposition}
\newtheorem{conjecture}{Conjecture}
\newtheorem{remark}{Remark}
\newtheorem{question}{Question}
\newtheorem{problem}{Problem}
\newtheorem{idea}{Idea}

\begin{document}

\title{New Approaches to the Unfavorable Jump Case}
\date{\today}
\maketitle

\section{The Problem}

Let $(M^3, g, k)$ be asymptotically flat initial data satisfying DEC.
Let $\Sigma_0$ be a closed trapped surface with:
\begin{itemize}
    \item $\theta^+ = H + \tr_\Sigma k \le 0$
    \item $\theta^- = H - \tr_\Sigma k < 0$
    \item $\tr_{\Sigma_0} k < 0$ (unfavorable jump condition)
\end{itemize}

\textbf{Goal:} Prove $M_{\mathrm{ADM}} \ge \sqrt{A(\Sigma_0)/(16\pi)}$.

\textbf{The obstacle:} The Jang equation approach produces a distributional scalar curvature:
\[
R_{\bar{g}} = R^{\mathrm{reg}} + 2[H]\delta_\Sigma
\]
where $[H] = \tr_\Sigma k < 0$. This negative Dirac mass breaks the positive mass argument.

\section{Failed Approaches}

\subsection{Area Monotonicity (INVALID)}
The claim: reduce to the outermost MOTS $\Sigma^*$ where $[H] \ge 0$, then use $A(\Sigma^*) \ge A(\Sigma_0)$.

\textbf{Why it fails:}
\begin{itemize}
    \item The proof in Section 3 of the paper is not rigorous (contains "Wait—" and contradictions).
    \item The inequality $A(\Sigma^*) \ge A(\Sigma_0)$ is likely FALSE in general.
    \item No clean monotonicity principle connects $\Sigma_0$ to $\Sigma^*$.
\end{itemize}

\subsection{Spacetime Area Comparison (WRONG DIRECTION)}
Using the Hawking area theorem: area is non-increasing along generators of a null hypersurface with $\theta \le 0$.

\textbf{Why it fails:}
Since $\theta^+ < 0$ on $\Sigma_0$, outgoing null rays converge. This means area DECREASES as you move outward along null rays—the wrong direction for our needs.

\section{New Approach 1: Modified Jang Equation with Corrector}

\begin{idea}[Absorbing the negative jump]
Instead of the standard Jang equation, consider a \textbf{modified} equation that absorbs the negative contribution.
\end{idea}

The generalized Jang equation is:
\[
\mathcal{J}[f] := \left(g^{ij} - \frac{f^i f^j}{1 + |Df|^2}\right)\left(D_i D_j f - k_{ij}\right) = 0.
\]

\textbf{Modification:} Add a source term $\psi$ to compensate for $[H] < 0$:
\[
\mathcal{J}[f] = \psi
\]
where $\psi$ is chosen so that the resulting distributional scalar curvature satisfies:
\[
R_{\bar{g}}^{\mathrm{new}} = R^{\mathrm{reg}} + 2[H]\delta_\Sigma + 2\psi\delta_\Sigma \ge 0.
\]

This requires $\psi \ge -[H] = -\tr_\Sigma k > 0$.

\textbf{Problem:} How to choose $\psi$ compatibly with existence theory?

\textbf{Key question:} Does modifying the Jang equation break the DEC → $R_{\bar{g}} \ge 0$ identity?

\section{New Approach 2: Penrose-like Inequality via Null Expansion}

\begin{idea}[Direct use of trapped condition]
The trapped condition $\theta^\pm < 0$ is STRONGER than needed for the Jang approach. Perhaps it can be exploited directly.
\end{idea}

On a trapped surface:
\[
\theta^+ = H + \tr_\Sigma k \le 0, \quad \theta^- = H - \tr_\Sigma k < 0.
\]

This gives:
\[
H \le -\tr_\Sigma k \quad \text{and} \quad H < \tr_\Sigma k.
\]

Combining: $-\tr_\Sigma k \ge H > \tr_\Sigma k$, which forces... wait, this is only compatible if $\tr_\Sigma k < 0$.

For $\tr_\Sigma k < 0$: we have $H < \tr_\Sigma k < 0$, so $H < 0$ (mean-convex toward interior).

\textbf{Geometric interpretation:}
- $\Sigma_0$ is mean-convex toward the interior
- Both null expansions are negative (strictly trapped)

\textbf{Question:} Can we bound ADM mass using only the trapped condition without the Jang reduction?

\subsection{Geroch-Huisken-Ilmanen monotonicity}

In the Riemannian case ($k = 0$), IMCF from a minimal surface $\Sigma$ gives:
\[
\frac{d}{dt}\left(\sqrt{A(\Sigma_t)/16\pi} - M_H(\Sigma_t)\right) \ge 0
\]
where $M_H$ is the Hawking mass.

For $k \neq 0$, the analogous object would involve the null expansions.

\textbf{Question:} Is there a flow in the spacetime setting that gives monotonicity?

\section{New Approach 3: Weak Formulation}

\begin{idea}[Distribute the negative mass]
Instead of concentrating the negative contribution $2[H]\delta_\Sigma$ at the interface, find a weak formulation where it is absorbed into the bulk.
\end{idea}

The Jang scalar curvature identity is:
\[
R_{\bar{g}} = 16\pi(\mu - J(\nu)) + (\text{quadratic terms}) + 2[H]\delta_\Sigma.
\]

If we could write the delta function as a limit of smooth functions:
\[
[H]\delta_\Sigma = \lim_{\epsilon \to 0} [H] \cdot \rho_\epsilon
\]
where $\rho_\epsilon$ is a smooth approximation supported in a neighborhood of $\Sigma$, then perhaps:
\[
R^{\mathrm{reg}}_{\bar{g}} + 2[H]\rho_\epsilon \ge 0
\]
for small $\epsilon$, using the fact that $R^{\mathrm{reg}} \ge 16\pi(\mu - J(\nu)) \ge 0$ by DEC.

\textbf{Problem:} This is hand-wavy. The issue is that $[H] < 0$ is a genuine geometric obstruction, not just a distributional technicality.

\section{New Approach 4: Optimal Transport / Mass Reallocation}

\begin{idea}[Transport mass to compensate]
The ADM mass is a boundary integral. Perhaps we can "transport" mass from infinity to cancel the negative contribution at $\Sigma$.
\end{idea}

The ADM mass is:
\[
M_{\mathrm{ADM}} = \frac{1}{16\pi}\lim_{r \to \infty} \oint_{S_r} (g_{ij,j} - g_{jj,i}) \nu^i dA.
\]

A positive mass theorem argument shows $M_{\mathrm{ADM}} \ge 0$ when $R_{\bar{g}} \ge 0$ distributionally.

For $[H] < 0$, the distributional curvature is:
\[
\int_{\bar{M}} R_{\bar{g}} \, dV = \int_{\bar{M} \setminus \Sigma} R^{\mathrm{reg}} \, dV + 2\int_\Sigma [H] \, dA.
\]

If $|[H]|$ is small enough, perhaps the positive bulk contribution dominates?

\textbf{Quantitative question:} Under DEC, can we bound $\int_\Sigma |[H]| dA$ in terms of the ADM mass?

\section{New Approach 5: Bootstrap from Physical Arguments}

\begin{idea}[Use the physics]
The Penrose inequality is motivated by the cosmic censorship conjecture. Perhaps a spacetime evolution argument can prove it.
\end{idea}

Heuristically:
\begin{enumerate}
    \item $\Sigma_0$ is a trapped surface at time $t = 0$.
    \item By cosmic censorship, $\Sigma_0$ lies inside a black hole event horizon $\mathcal{H}$.
    \item The final black hole has mass $M_{\mathrm{final}} \ge \sqrt{A(\mathcal{H})/16\pi}$ by the area-mass relation.
    \item By energy conservation, $M_{\mathrm{ADM}} \ge M_{\mathrm{final}}$.
    \item By the area theorem, $A(\mathcal{H}) \ge A(\Sigma_0)$ (area increases along future generators).
\end{enumerate}

Combining: $M_{\mathrm{ADM}} \ge M_{\mathrm{final}} \ge \sqrt{A(\mathcal{H})/16\pi} \ge \sqrt{A(\Sigma_0)/16\pi}$.

\textbf{Problems:}
\begin{itemize}
    \item Cosmic censorship is not proven.
    \item This argument requires spacetime evolution, not just initial data.
    \item The area theorem requires the generators of $\mathcal{H}$ to pass through $\Sigma_0$.
\end{itemize}

\section{New Approach 6: Variational / Min-Max}

\begin{idea}[Area-minimizing surface inside the trapped region]
Inside the trapped region, find an area-minimizing surface with prescribed homology class, and show it has better properties.
\end{idea}

Let $\Sigma_{\min}$ be the area-minimizing surface homologous to $\Sigma_0$ inside the region bounded by $\Sigma^*$.

\textbf{Question:} What are the properties of $\Sigma_{\min}$?

Since $H < 0$ in the trapped region (mean-convex toward interior), a minimizer $\Sigma_{\min}$ satisfies $H = 0$ (minimal).

But $\theta^+ = H + \tr_\Sigma k = \tr_\Sigma k$ on $\Sigma_{\min}$.

If $\tr_{\Sigma_{\min}} k \ge 0$, we're done (apply Jang at $\Sigma_{\min}$ and use $A(\Sigma_{\min}) \le A(\Sigma_0)$).

\textbf{Key question:} Is $\tr_\Sigma k \ge 0$ for an area-minimizing surface inside the trapped region?

This seems unlikely to be true in general...

\section{New Approach 7: Higher Null Expansion}

\begin{idea}[Use $\theta^-$ more fundamentally]
The condition $\theta^- < 0$ is underutilized. Perhaps it provides additional constraints.
\end{idea}

We have:
\[
\theta^+ = H + \tr_\Sigma k, \quad \theta^- = H - \tr_\Sigma k.
\]

For a strictly trapped surface: $\theta^+ < 0$ and $\theta^- < 0$.

This gives:
\[
\theta^+ + \theta^- = 2H < 0, \quad \theta^+ - \theta^- = 2\tr_\Sigma k.
\]

The unfavorable case is $\theta^+ - \theta^- < 0$, i.e., $\theta^+ < \theta^-$.

\textbf{Geometric interpretation:} The outgoing null expansion is MORE negative than the ingoing. This means outgoing light is converging faster than ingoing light.

\textbf{Question:} Does this stronger convergence give any geometric leverage?

\section{Conclusion}

The unfavorable jump case ($\tr_\Sigma k < 0$) remains OPEN.

None of the above approaches provides a complete proof.

\textbf{The most promising directions seem to be:}
\begin{enumerate}
    \item Modified Jang equation with carefully chosen source term
    \item Variational/min-max approach with minimizers inside the trapped region
    \item Spacetime evolution argument (requires cosmic censorship or weaker assumption)
\end{enumerate}

\textbf{This is a genuine gap in mathematical general relativity that requires NEW MATHEMATICS.}

\end{document}
