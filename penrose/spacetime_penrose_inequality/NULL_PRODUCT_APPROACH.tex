% =========================================================================
%     THE PRODUCT θ⁺θ⁻ APPROACH: A POTENTIALLY NEW DIRECTION
%
%     Exploring the symmetric use of both null expansions
%
%     Author: Da Xu
%     Date: December 2025
% =========================================================================

\documentclass[12pt]{article}
\usepackage{amsmath,amsthm,amssymb}
\usepackage{mathrsfs}
\usepackage{tcolorbox}

\theoremstyle{plain}
\newtheorem{theorem}{Theorem}[section]
\newtheorem{lemma}[theorem]{Lemma}
\newtheorem{proposition}[theorem]{Proposition}
\newtheorem{corollary}[theorem]{Corollary}
\newtheorem{conjecture}{Conjecture}

\theoremstyle{definition}
\newtheorem{definition}[theorem]{Definition}
\newtheorem{remark}[theorem]{Remark}

\newcommand{\ADM}{\mathrm{ADM}}
\newcommand{\tr}{\mathrm{tr}}
\newcommand{\Div}{\mathrm{div}}
\newcommand{\Area}{\mathrm{Area}}

\title{\textbf{The Product $\theta^+\theta^-$ Approach}}
\author{Da Xu}
\date{December 2025}

\begin{document}
\maketitle

\section{Motivation}

\subsection{The Sign Problem Review}

All previous approaches fail because they use either:
\begin{itemize}
    \item $H = \frac{\theta^+ + \theta^-}{2}$ (can be negative)
    \item $\tr_\Sigma k = \frac{\theta^+ - \theta^-}{2}$ (can be negative)
\end{itemize}

\subsection{A Sign-Definite Quantity}

For trapped surfaces:
\begin{itemize}
    \item $\theta^+ \leq 0$ (outgoing rays converge)
    \item $\theta^- < 0$ (ingoing rays converge)
\end{itemize}

Therefore: $\theta^+ \theta^- \geq 0$ always!

\textbf{Key Identity:}
\[
    \theta^+ \theta^- = H^2 - (\tr_\Sigma k)^2
\]

\section{The Product Functional}

\subsection{Definition}

Define the \textbf{null product functional}:
\[
    \mathcal{P}(\Sigma) = \int_\Sigma \theta^+ \theta^- \, dA
\]

For trapped surfaces: $\mathcal{P}(\Sigma) \geq 0$.

\subsection{Relation to Hawking Mass}

The Hawking mass:
\[
    m_H(\Sigma) = \sqrt{\frac{\Area(\Sigma)}{16\pi}} \left(1 - \frac{1}{16\pi}\int_\Sigma H^2 \, dA\right)
\]

Using $H^2 = \theta^+\theta^- + (\tr_\Sigma k)^2$:
\[
    m_H(\Sigma) = \sqrt{\frac{A}{16\pi}} \left(1 - \frac{\mathcal{P}(\Sigma)}{16\pi} - \frac{1}{16\pi}\int (\tr_\Sigma k)^2 \, dA\right)
\]

\subsection{A New Mass Proposal}

Define the \textbf{null-product Hawking mass}:
\[
    \tilde{m}_H(\Sigma) = \sqrt{\frac{\Area(\Sigma)}{16\pi}} \left(1 - \frac{\mathcal{P}(\Sigma)}{16\pi \cdot \Area(\Sigma)}\right)
\]

For trapped surfaces: $\mathcal{P} \geq 0$, so:
\[
    \tilde{m}_H(\Sigma) \leq \sqrt{\frac{\Area(\Sigma)}{16\pi}}
\]

This is the \textbf{wrong direction} for Penrose, but let's explore further.

\section{The Inverse Product Functional}

\subsection{Definition}

Define:
\[
    \mathcal{Q}(\Sigma) = \int_\Sigma \frac{1}{\theta^+ \theta^-} \, dA
\]

This is well-defined for strictly trapped surfaces ($\theta^+, \theta^- < 0$).

\subsection{Properties}

For small $|\theta^+|, |\theta^-|$ (near MOTS):
\[
    \frac{1}{\theta^+\theta^-} \to +\infty
\]

For large $|\theta^+\theta^-|$ (deep trapped):
\[
    \frac{1}{\theta^+\theta^-} \to 0
\]

\textbf{Observation:} $\mathcal{Q}$ measures how ``close to MOTS'' the surface is.

\section{The Geometric Mean Expansion}

\subsection{Definition}

The \textbf{geometric mean expansion}:
\[
    \Theta = \sqrt{|\theta^+ \theta^-|} = \sqrt{|H^2 - (\tr_\Sigma k)^2|}
\]

For trapped surfaces with $|H| < |\tr_\Sigma k|$ (possible!):
\[
    \Theta = \sqrt{(\tr_\Sigma k)^2 - H^2}
\]

\subsection{Flow by Geometric Mean}

Consider the flow:
\[
    \partial_t X = \frac{\nu}{\Theta} = \frac{\nu}{\sqrt{|\theta^+\theta^-|}}
\]

\begin{lemma}
Under this flow:
\[
    \frac{dA}{dt} = \int_\Sigma \frac{H}{\sqrt{|\theta^+\theta^-|}} \, dA
\]
\end{lemma}

For trapped surfaces with $H < 0$: $\frac{dA}{dt} < 0$ still!

\textbf{The sign problem persists.}

\section{The Ratio $\theta^+/\theta^-$ Approach}

\subsection{Definition}

For trapped surfaces, define:
\[
    \rho = \frac{\theta^+}{\theta^-} = \frac{H + \tr_\Sigma k}{H - \tr_\Sigma k}
\]

Since both are negative: $\rho > 0$.

\subsection{Properties}

\begin{itemize}
    \item $\rho = 1$: $\tr_\Sigma k = 0$ (time-symmetric)
    \item $\rho < 1$: $\tr_\Sigma k > 0$ (favorable jump)
    \item $\rho > 1$: $\tr_\Sigma k < 0$ (unfavorable jump)
\end{itemize}

\subsection{The $\rho$-Weighted Area}

Define:
\[
    A_\rho(\Sigma) = \int_\Sigma \rho \, dA
\]

For favorable jump ($\rho < 1$): $A_\rho < \Area(\Sigma)$
For unfavorable jump ($\rho > 1$): $A_\rho > \Area(\Sigma)$

\begin{conjecture}[Speculative]
$M_{\ADM} \geq \sqrt{A_\rho(\Sigma)/(16\pi)}$?
\end{conjecture}

\textbf{Problem:} No proof or evidence for this conjecture.

\section{The Stability Operator Product}

\subsection{The Stability Operators}

For null expansions, define stability operators:
\[
    L_+ = -\Delta_\Sigma + \text{(curvature terms from } \theta^+\text{)}
\]
\[
    L_- = -\Delta_\Sigma + \text{(curvature terms from } \theta^-\text{)}
\]

\subsection{Product Operator}

Define $L = L_+ L_-$ (composition of stability operators).

\begin{question}
Does $\lambda_1(L)$ have a relationship to $\Area$ and $M_{\ADM}$?
\end{question}

\textbf{Analysis:} The product $L_+L_-$ is a fourth-order operator. Its spectral
theory is complicated and doesn't obviously connect to mass.

\section{The Hopf Fibration Approach}

\subsection{Idea}

The 3-sphere $S^3$ has a Hopf fibration:
\[
    S^1 \to S^3 \to S^2
\]

The base $S^2$ parametrizes the ratio $\theta^+/\theta^-$ in some sense.

\subsection{Application Attempt}

Map the trapped surface to $S^2$ by:
\[
    \Sigma \ni p \mapsto [\theta^+(p) : \theta^-(p)] \in \mathbb{CP}^1 \cong S^2
\]

\textbf{Problem:} This map doesn't preserve relevant geometric information.

\section{The Complex Null Expansion}

\subsection{Definition}

Define a complex expansion:
\[
    \Theta_\mathbb{C} = \theta^+ + i\theta^-
\]

The modulus: $|\Theta_\mathbb{C}| = \sqrt{(\theta^+)^2 + (\theta^-)^2}$

\subsection{Properties}

For trapped surfaces: Both $\theta^+, \theta^- < 0$, so $\Theta_\mathbb{C}$ is in the
third quadrant of $\mathbb{C}$.

The argument: $\arg(\Theta_\mathbb{C}) = \pi + \arctan(\theta^-/\theta^+)$

\begin{question}
Is there a complex-analytic formulation of the Penrose inequality?
\end{question}

\textbf{Status:} No known formulation.

\section{The Determinant Approach}

\subsection{The Null Second Fundamental Forms}

Define matrices $\chi_+, \chi_-$ as the null second fundamental forms.

Then: $\theta^+ = \tr \chi_+$, $\theta^- = \tr \chi_-$.

\subsection{Determinants}

\[
    \det \chi_+ = \frac{1}{2}((\theta^+)^2 - |\hat{\chi}_+|^2)
\]
where $\hat{\chi}_+$ is the traceless part (shear).

\begin{lemma}
$\theta^+\theta^- = \tr\chi_+ \cdot \tr\chi_- \geq 0$ for trapped surfaces.
\end{lemma}

But: $\det\chi_+ \cdot \det\chi_-$ doesn't have a definite sign.

\section{Critical Assessment}

\begin{tcolorbox}[colback=yellow!10, colframe=orange!75!black, title=\textbf{What $\theta^+\theta^-$ Gives Us}]
\textbf{Positive aspects:}
\begin{itemize}
    \item Sign-definite: $\theta^+\theta^- \geq 0$ for trapped surfaces
    \item Symmetric in null directions
    \item Vanishes on MOTS ($\theta^+ = 0$ or $\theta^- = 0$)
\end{itemize}

\textbf{Negative aspects:}
\begin{itemize}
    \item Flows using $\sqrt{\theta^+\theta^-}$ still have wrong sign for area
    \item No known monotonicity formula involving $\theta^+\theta^-$
    \item Hawking mass with $\theta^+\theta^-$ goes in wrong direction
\end{itemize}
\end{tcolorbox}

\begin{tcolorbox}[colback=red!10, colframe=red!75!black, title=\textbf{Fundamental Barrier}]
\textbf{The core issue:}

The Penrose inequality relates $M_{\ADM}$ (global, at infinity) to $\Area(\Sigma_0)$
(local, at the surface).

Any local quantity on $\Sigma_0$---whether $H$, $\tr_\Sigma k$, $\theta^+$, $\theta^-$,
or $\theta^+\theta^-$---cannot directly control $M_{\ADM}$ without:
\begin{enumerate}
    \item A monotonic flow connecting $\Sigma_0$ to infinity
    \item A comparison principle relating local to global
    \item Global assumptions (cosmic censorship)
\end{enumerate}

For trapped surfaces with $\tr_\Sigma k < 0$, all known monotonic quantities
fail. The product $\theta^+\theta^-$ doesn't resolve this because it still
appears in expressions where the $H < 0$ sign dominates.
\end{tcolorbox}

\section{Conclusion}

The $\theta^+\theta^-$ approach, while providing a sign-definite quantity,
does not resolve the fundamental obstruction. The problem is not finding
positive quantities, but finding \textbf{monotonic} quantities that:
\begin{enumerate}
    \item Equal $\sqrt{\Area/(16\pi)}$ on trapped surfaces
    \item Equal $M_{\ADM}$ at infinity
    \item Are non-decreasing along some path
\end{enumerate}

No such quantity is currently known for the unconditional case.

\end{document}
