%% GLOBAL_MINIMIZER_ATTACK.tex
%%
%% THE REMAINING GAP: Proving Schwarzschild is the GLOBAL minimizer
%% of ADM mass among initial data with trapped surface of given area.
%%
%% We have proven: Schwarzschild is a CRITICAL POINT and LOCAL MINIMUM.
%% We need: Schwarzschild is the GLOBAL MINIMUM.
%%
%% December 2025

\documentclass[11pt]{amsart}
\usepackage{amsmath,amssymb,amsthm}
\usepackage{mathrsfs}
\usepackage{tcolorbox}

\tcbuselibrary{theorems}

\newtcolorbox{innovation}{
    colback=green!5!white,
    colframe=green!50!black,
    title={\textbf{INNOVATION}}
}

\newtcolorbox{keypoint}{
    colback=blue!5!white,
    colframe=blue!75!black,
    title={\textbf{KEY POINT}}
}

\newtcolorbox{warning}{
    colback=red!5!white,
    colframe=red!75!black,
    title={\textbf{CRITICAL GAP}}
}

\newtcolorbox{attack}{
    colback=yellow!10!white,
    colframe=orange!75!black,
    title={\textbf{ATTACK STRATEGY}}
}

\newtheorem{theorem}{Theorem}
\newtheorem{lemma}[theorem]{Lemma}
\newtheorem{proposition}[theorem]{Proposition}
\newtheorem{corollary}[theorem]{Corollary}
\theoremstyle{definition}
\newtheorem{definition}[theorem]{Definition}
\newtheorem{remark}[theorem]{Remark}

\newcommand{\Ric}{\text{Ric}}
\newcommand{\tr}{\text{tr}}
\newcommand{\divv}{\text{div}}
\newcommand{\Vol}{\text{Vol}}
\newcommand{\Area}{\text{Area}}
\newcommand{\Mass}{\mathcal{M}}
\newcommand{\Config}{\mathscr{C}}

\title{The Global Minimizer Problem:\\
Why is Schwarzschild Optimal Among All Data?}
\author{December 2025}

\begin{document}
\maketitle

\begin{abstract}
We attack the key remaining gap in the variational approach to the 
Penrose inequality: proving that Schwarzschild is the \textbf{global} 
minimizer of ADM mass among all initial data containing a trapped 
surface of given area, not just a local minimum.
\end{abstract}

%% ============================================================================
\section{The Current Status}
%% ============================================================================

\begin{keypoint}
\textbf{What we have proven:}
\begin{enumerate}
    \item The Schwarzschild slice is a \textbf{critical point} of ADM mass 
          subject to the constraint of having a trapped surface of area $A$.
    \item The second variation $\delta^2 M_{\text{ADM}} > 0$ for all 
          non-spherically-symmetric perturbations, so Schwarzschild is a 
          \textbf{local minimum}.
    \item Among spherically symmetric data with DEC, Schwarzschild is the 
          unique minimizer (Bray's original result).
\end{enumerate}
\end{keypoint}

\begin{warning}
\textbf{What we need:}

Schwarzschild is the \textbf{global} minimum of $M_{\text{ADM}}$ among 
all initial data $(\Sigma^3, g, k)$ satisfying:
\begin{itemize}
    \item DEC: $\mu \ge |J|$ where $16\pi\mu = R + (\tr k)^2 - |k|^2$
    \item Contains a closed trapped surface $S$ with $\Area(S) = A$
    \item Asymptotically flat
\end{itemize}

\textbf{The gap:} We have local minimum, need global minimum.
\end{warning}

%% ============================================================================
\section{Strategy 1: Topology of the Configuration Space}
%% ============================================================================

\begin{attack}
If the configuration space $\Config_A$ of valid initial data is 
\textbf{connected}, and Schwarzschild is the \textbf{unique} critical 
point that is a local minimum, then it must be the global minimum.
\end{attack}

\begin{definition}[Configuration Space]
Let $\Config_A$ be the space of smooth, asymptotically flat initial data 
$(\Sigma, g, k)$ such that:
\begin{enumerate}
    \item DEC holds: $\mu \ge |J|$
    \item There exists a closed trapped surface $S \subset \Sigma$ with 
          $\Area(S) = A$
    \item The data is complete (or suitable boundary conditions)
\end{enumerate}
\end{definition}

\begin{proposition}[Connectedness]
The space $\Config_A$ is path-connected.
\end{proposition}

\begin{proof}[Proof Sketch]
Given two data sets $(\Sigma, g_0, k_0)$ and $(\Sigma, g_1, k_1)$ in $\Config_A$:
\begin{enumerate}
    \item The constraint equations define a submanifold of the space of 
          $(g, k)$ pairs.
    \item The DEC condition $\mu \ge |J|$ is an \textbf{open} condition 
          in the interior.
    \item The trapped surface condition is \textbf{stable} under small 
          perturbations.
    \item Linear interpolation $(g_t, k_t) = (1-t)(g_0, k_0) + t(g_1, k_1)$ 
          may violate constraints...
\end{enumerate}

\textbf{Issue:} Linear interpolation doesn't preserve the constraint equations!

Need a more sophisticated deformation argument.
\end{proof}

\begin{warning}
The constraint equations are nonlinear, so $\Config_A$ is NOT a linear space. 
Proving connectedness requires careful analysis.
\end{warning}

%% ============================================================================
\section{Strategy 2: No Other Critical Points}
%% ============================================================================

\begin{attack}
Show that Schwarzschild is the \textbf{only} critical point of $M_{\text{ADM}}$ 
in $\Config_A$.

If there are no other critical points, then Schwarzschild must be the 
global minimum (since it's a local minimum and the infimum is achieved).
\end{attack}

\begin{theorem}[Uniqueness of Critical Points - Attempt]
Let $(\Sigma, g, k)$ be a critical point of $M_{\text{ADM}}$ in $\Config_A$.
Then $(\Sigma, g, k)$ is a slice of Schwarzschild spacetime.
\end{theorem}

\begin{proof}[Proof Attempt]
At a critical point, the first variation vanishes:
\[
\delta M_{\text{ADM}} = 0
\]
for all variations preserving the constraints.

The Euler-Lagrange equations give (with Lagrange multiplier $\lambda$):
\begin{align}
    \text{ADM variation} &= \lambda \cdot \text{Area variation}\\
    \frac{1}{16\pi} \int_{\Sigma} \left(R_{ij} - \frac{1}{2}R g_{ij}\right) 
    h^{ij} &= \lambda \int_S H h_{\nu\nu}
\end{align}

This should force $(g, k)$ to be Schwarzschild-like.

\textbf{Issue:} The variation must also preserve DEC and constraint equations.
This adds complexity to the Euler-Lagrange analysis.
\end{proof}

%% ============================================================================
\section{Strategy 3: Concentration-Compactness}
%% ============================================================================

\begin{attack}
Use concentration-compactness analysis for minimizing sequences.

If $\{(\Sigma_n, g_n, k_n)\}$ is a minimizing sequence with 
$M_{\text{ADM}}(\Sigma_n) \to \inf_{\Config_A} M_{\text{ADM}}$, show:
\begin{enumerate}
    \item Compactness: A subsequence converges to a limit
    \item No concentration: Mass doesn't escape to infinity
    \item No vanishing: The limit is non-trivial
\end{enumerate}
\end{attack}

\begin{lemma}[No Mass Escape]
For any $(\Sigma, g, k) \in \Config_A$ containing trapped surface of area $A$:
\[
M_{\text{ADM}} \ge \sqrt{\frac{A}{16\pi}}
\]
(This is the Penrose inequality we're trying to prove!)
\end{lemma}

\begin{remark}
This is circular! We can't use the Penrose inequality to prove itself.

However, we CAN use:
\[
M_{\text{ADM}} \ge 0 \quad \text{(Positive Mass Theorem)}
\]
\end{remark}

\begin{proposition}[Minimizing Sequence Analysis]
Let $\{(\Sigma_n, g_n, k_n)\}$ be a minimizing sequence in $\Config_A$.

Then:
\begin{enumerate}
    \item $\inf_n M_{\text{ADM}}(\Sigma_n) \ge 0$ by positive mass theorem
    \item The infimum is \textbf{achieved} if we can extract a convergent 
          subsequence
    \item The limit must be Schwarzschild (by uniqueness of critical points)
\end{enumerate}
\end{proposition}

\begin{warning}
\textbf{The key technical difficulty:}

Extracting a convergent subsequence requires controlling:
\begin{itemize}
    \item Geometry at infinity (ADM asymptotics)
    \item Geometry near the trapped surface
    \item No ``bubbling off'' of mass
\end{itemize}

This is where the analysis becomes highly technical.
\end{warning}

%% ============================================================================
\section{Strategy 4: The Mountain Pass Obstruction}
%% ============================================================================

\begin{attack}
Show that there is NO mountain pass or saddle point between Schwarzschild 
and any other configuration.

If the ``energy landscape'' is convex-like, then the unique local minimum 
is global.
\end{attack}

\begin{definition}[Mountain Pass]
A mountain pass between critical points $p$ and $q$ in $\Config_A$ is a 
critical point $r$ such that:
\begin{enumerate}
    \item $M_{\text{ADM}}(r) > \max(M_{\text{ADM}}(p), M_{\text{ADM}}(q))$
    \item Every path from $p$ to $q$ passes through a region where 
          $M_{\text{ADM}} \ge M_{\text{ADM}}(r)$
\end{enumerate}
\end{definition}

\begin{theorem}[No Mountain Pass - Conjecture]
There is no mountain pass separating the Schwarzschild data from any 
other configuration in $\Config_A$.

\textbf{Consequence:} Schwarzschild is the global minimum.
\end{theorem}

\begin{proof}[Proof Strategy]
Suppose there exists a mountain pass at some $(\Sigma, g, k) \in \Config_A$.

Then this must be a critical point of index 1 (one unstable direction).

We need to show no such critical point exists...

\textbf{Issue:} This requires classifying ALL critical points of $M_{\text{ADM}}$, 
which is very difficult.
\end{proof}

%% ============================================================================
\section{Strategy 5: Monotonic Deformation}
%% ============================================================================

\begin{innovation}
\textbf{The Key Idea:} Construct a continuous path from ANY initial data 
$(\Sigma, g, k)$ to Schwarzschild such that $M_{\text{ADM}}$ is 
\textbf{monotonically non-increasing} along the path.
\end{innovation}

\begin{theorem}[Monotonic Deformation - The Goal]
For any $(\Sigma, g, k) \in \Config_A$, there exists a continuous path 
$(\Sigma, g_t, k_t)$ for $t \in [0, 1]$ such that:
\begin{enumerate}
    \item $(g_0, k_0) = (g, k)$
    \item $(g_1, k_1) = (\text{Schwarzschild data})$
    \item $\frac{d}{dt} M_{\text{ADM}}(g_t, k_t) \le 0$ for all $t$
    \item The path stays in $\Config_A$ (DEC preserved, trapped surface preserved)
\end{enumerate}

\textbf{Consequence:} $M_{\text{ADM}}(g, k) \ge M_{\text{ADM}}(\text{Sch})$, 
which is the Penrose inequality!
\end{theorem}

\begin{remark}
This is essentially what we tried with Ricci flow, expansion flow, etc.

The challenge: finding a flow that simultaneously:
\begin{enumerate}
    \item Decreases ADM mass
    \item Preserves DEC
    \item Preserves trapped surface
    \item Converges to Schwarzschild
\end{enumerate}
\end{remark}

%% ============================================================================
\section{Strategy 6: The Jang Reduction}
%% ============================================================================

\begin{attack}
Use the Jang equation to reduce to the Riemannian case, where the 
global minimizer property is known.
\end{attack}

Let $f$ solve the Jang equation:
\[
H_{\text{graph}(f)} = \tr_{\text{graph}(f)}(k)
\]

The graph $(\Sigma, \bar{g})$ where $\bar{g} = g + df \otimes df$ has:
\begin{enumerate}
    \item $\bar{R} \ge 0$ if DEC holds (Schoen-Yau)
    \item $M_{\text{ADM}}(\bar{g}) \le M_{\text{ADM}}(g, k)$ (mass comparison)
    \item The boundary behavior at MOTS is controlled
\end{enumerate}

\begin{theorem}[Riemannian Penrose for $\bar{g}$]
For the Jang graph $(\Sigma, \bar{g})$ with $\bar{R} \ge 0$:
\[
M_{\text{ADM}}(\bar{g}) \ge \sqrt{\frac{A(\partial\Sigma)}{16\pi}}
\]
(This is the Riemannian Penrose inequality!)
\end{theorem}

\begin{warning}
\textbf{The Jang approach has issues:}
\begin{enumerate}
    \item The Jang equation blows up at MOTS
    \item The boundary term analysis is subtle
    \item Mass comparison $M_{\text{ADM}}(\bar{g}) \le M_{\text{ADM}}(g, k)$ 
          requires careful justification
\end{enumerate}

These are exactly the issues that have prevented a complete proof!
\end{warning}

%% ============================================================================
\section{Strategy 7: Direct Comparison via Constraint Flow}
%% ============================================================================

\begin{innovation}
\textbf{Constraint Map Flow}

Define a flow on initial data that:
\begin{enumerate}
    \item Preserves the constraint equations exactly
    \item Moves in the direction of decreasing ADM mass
    \item Preserves the trapped surface area
\end{enumerate}
\end{innovation}

\begin{definition}[The Constraint Map]
The constraint map $\Phi: (g, k) \mapsto (\mu - |J|, \divv(k - (\tr k)g))$ 
takes initial data to the constraint ``defects''.

For vacuum data: $\Phi(g, k) = (0, 0)$.
\end{definition}

\begin{definition}[Constraint-Preserving Flow]
A flow $(g_t, k_t)$ is constraint-preserving if:
\[
\frac{d}{dt}\Phi(g_t, k_t) = 0
\]

i.e., if we start satisfying constraints, we continue to satisfy them.
\end{definition}

\begin{lemma}[Mass Gradient]
The $L^2$-gradient of $M_{\text{ADM}}$ with respect to $(g, k)$ is:
\[
\nabla M_{\text{ADM}} = \left(\frac{1}{16\pi}(R_{ij} - \frac{1}{2}R g_{ij}), 
\text{boundary terms}\right)
\]
\end{lemma}

\begin{definition}[Projected Mass Flow]
Define the flow:
\[
\frac{d}{dt}(g, k) = -\Pi_{\text{constraint}}(\nabla M_{\text{ADM}})
\]

where $\Pi_{\text{constraint}}$ projects onto the tangent space of the 
constraint manifold at $(g, k)$.
\end{definition}

\begin{warning}
\textbf{Issues with Constraint Flow:}
\begin{enumerate}
    \item The projection $\Pi_{\text{constraint}}$ is not explicit
    \item DEC preservation is not automatic
    \item Trapped surface preservation requires additional constraints
    \item Long-time existence is unclear
\end{enumerate}
\end{warning}

%% ============================================================================
\section{Strategy 8: Rigidity from Equality Case}
%% ============================================================================

\begin{attack}
Analyze the equality case: when does $M_{\text{ADM}} = \sqrt{A/16\pi}$?

If we can show that equality holds ONLY for Schwarzschild, and that 
Schwarzschild achieves this bound, then we're done.
\end{attack}

\begin{theorem}[Equality Characterization - Known for MOTS Penrose]
For the MOTS Penrose inequality:
\[
M_{\text{ADM}} = \sqrt{\frac{A(\Sigma^*)}{16\pi}}
\]
if and only if the initial data is a slice of Schwarzschild spacetime.
\end{theorem}

\begin{keypoint}
The equality case is characterized for the MOTS version.

But we need the inequality $M_{\text{ADM}} \ge \sqrt{A(\Sigma)/16\pi}$ 
for TRAPPED surfaces $\Sigma$, not just the MOTS $\Sigma^*$.

The Area Dominance gap: $A(\Sigma) \le A(\Sigma^*)$.
\end{keypoint}

%% ============================================================================
\section{The Core Insight}
%% ============================================================================

\begin{innovation}
\textbf{Key Realization:}

All strategies ultimately require one of:
\begin{enumerate}
    \item A monotonic flow that decreases mass while preserving constraints
    \item Classification of all critical points of constrained mass functional
    \item Direct comparison showing non-Schwarzschild data has larger mass
\end{enumerate}

The constraint equations make all of these difficult because they 
\textbf{couple} $g$ and $k$ in a nonlinear way.
\end{innovation}

%% ============================================================================
\section{A New Approach: The Isoperimetric Comparison}
%% ============================================================================

\begin{innovation}
\textbf{Isoperimetric Comparison Principle}

In Schwarzschild with mass $m$:
\begin{itemize}
    \item The horizon has area $A = 16\pi m^2$
    \item This is the MINIMUM area enclosing surface
\end{itemize}

\textbf{Conjecture:} In any $(g, k)$ satisfying DEC with ADM mass $M$:
\[
\min_{\text{enclosing } S} \Area(S) \le 16\pi M^2
\]

with equality iff Schwarzschild.

This is EQUIVALENT to the Penrose inequality!
\end{innovation}

\begin{proof}[Equivalence]
Penrose: $M \ge \sqrt{A/16\pi}$ for trapped $\Sigma$

$\Leftrightarrow M^2 \ge A/16\pi$

$\Leftrightarrow 16\pi M^2 \ge A$

$\Leftrightarrow$ The bound $16\pi M^2$ exceeds the area of any trapped surface.

If $\Sigma$ is the smallest trapped surface, this is exactly the 
isoperimetric comparison.
\end{proof}

%% ============================================================================
\section{Strategy 9: Spinorial Methods}
%% ============================================================================

\begin{attack}
The Positive Mass Theorem uses spinors (Witten's proof).

Can we extend spinorial methods to prove the Penrose inequality?
\end{attack}

\begin{theorem}[Witten's Formula]
For asymptotically flat $(M, g, k)$ with $\mu \ge |J|$:
\[
M_{\text{ADM}} = \frac{1}{4\pi} \int_M \left(|D\psi|^2 + \frac{R}{4}|\psi|^2 
+ \text{(terms with } k\text{)}\right)
\]

where $\psi$ is a spinor satisfying appropriate boundary conditions.
\end{theorem}

\begin{keypoint}
For the Penrose inequality, we need a spinor with BOUNDARY CONDITIONS 
on the trapped surface $\Sigma$.

The natural condition: $\psi$ vanishes or is constrained on $\Sigma$.

This should give a lower bound in terms of $\Area(\Sigma)$.
\end{keypoint}

\begin{warning}
Spinorial proofs of the Penrose inequality exist only in special cases:
\begin{itemize}
    \item Time-symmetric ($k = 0$): Herzlich, etc.
    \item Spherically symmetric: Various authors
    \item General case: OPEN
\end{itemize}
\end{warning}

%% ============================================================================
\section{Synthesis: The Path Forward}
%% ============================================================================

\textbf{Most promising strategies for global minimizer:}

\begin{enumerate}
    \item \textbf{Uniqueness of Critical Points}: If we can show Schwarzschild 
          is the ONLY critical point, global minimum follows.
          
    \item \textbf{Concentration-Compactness}: Show minimizing sequences converge 
          (no mass escape, no bubbling).
          
    \item \textbf{Monotonic Deformation Flow}: Find a flow that decreases mass 
          while preserving all constraints.
          
    \item \textbf{Spinorial Methods}: Extend Witten's proof to include trapped 
          surface boundary.
\end{enumerate}

\begin{keypoint}
\textbf{The fundamental question:}

Why should physical initial data containing a black hole (trapped surface) 
have more mass than a Schwarzschild black hole of the same area?

\textbf{Physical answer:} Extra matter/energy/gravitational waves contribute 
positively to mass.

\textbf{Mathematical answer:} This is what the Penrose inequality asserts, 
and proving it requires controlling how ALL contributions to ADM mass 
relate to the trapped surface area.
\end{keypoint}

%% ============================================================================
\section{Conclusion: The Current State}
%% ============================================================================

\begin{center}
\fbox{\parbox{0.9\textwidth}{
\textbf{Summary of Global Minimizer Problem}

\textbf{Proven:}
\begin{itemize}
    \item Schwarzschild is a critical point of $M_{\text{ADM}}|_{\Config_A}$
    \item Schwarzschild is a local minimum ($\delta^2 M > 0$)
    \item In spherical symmetry, Schwarzschild is global minimum
\end{itemize}

\textbf{Needed for Complete Proof:}
\begin{itemize}
    \item Schwarzschild is the GLOBAL minimum
    \item OR: Equivalent monotonic flow argument
    \item OR: Equivalent spinorial argument
\end{itemize}

\textbf{Key Difficulty:}
The constraint equations $\Phi(g, k) = 0$ couple $g$ and $k$ nonlinearly, 
making direct variational analysis difficult.

\textbf{Most Promising Path:}
Prove uniqueness of critical points + compactness of minimizing sequences.
}}
\end{center}

\end{document}
