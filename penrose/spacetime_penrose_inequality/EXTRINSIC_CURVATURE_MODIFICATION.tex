% =========================================================================
%     EXTRINSIC CURVATURE MODIFICATION: A NEW ATTACK
%
%     The idea: Instead of conformally changing g, modify k to make
%     tr_Sigma k >= 0 while preserving trapping and mass
%
%     Author: Da Xu
%     Date: December 2025
% =========================================================================

\documentclass[12pt]{article}
\usepackage{amsmath,amsthm,amssymb}
\usepackage{mathrsfs}
\usepackage{tcolorbox}
\usepackage{xcolor}

\theoremstyle{plain}
\newtheorem{theorem}{Theorem}[section]
\newtheorem{lemma}[theorem]{Lemma}
\newtheorem{proposition}[theorem]{Proposition}
\newtheorem{corollary}[theorem]{Corollary}
\newtheorem{conjecture}[theorem]{Conjecture}

\theoremstyle{definition}
\newtheorem{definition}[theorem]{Definition}
\newtheorem{remark}[theorem]{Remark}
\newtheorem{gap}[theorem]{\textcolor{red}{GAP}}
\newtheorem{key}[theorem]{\textcolor{blue}{KEY}}

\newcommand{\ADM}{\mathrm{ADM}}
\newcommand{\tr}{\mathrm{tr}}
\newcommand{\Div}{\mathrm{div}}
\newcommand{\Area}{\mathrm{Area}}
\newcommand{\vol}{\mathrm{vol}}
\newcommand{\Ric}{\mathrm{Ric}}
\newcommand{\M}{\mathcal{M}}

\title{\textbf{Extrinsic Curvature Modification:\\
A Novel Strategy for the Unfavorable Case}}
\author{Da Xu\\China Mobile Research Institute}
\date{December 2025}

\begin{document}
\maketitle

\begin{abstract}
We propose modifying the extrinsic curvature $k$ (rather than the metric $g$)
to convert ``unfavorable'' trapped surfaces ($\tr_\Sigma k < 0$) into
``favorable'' ones ($\tr_\Sigma k \geq 0$). If this can be done while
preserving the trapping condition, DEC, and controlling the ADM mass,
it would reduce the general case to the known favorable case.
\end{abstract}

\tableofcontents

%===========================================================================
\section{Motivation: Modifying $k$ Instead of $g$}
%===========================================================================

\subsection{The Standard Approach and Its Failure}

The standard Jang-based approach works as follows:
\begin{enumerate}
    \item Solve the Jang equation to get a ``blowup'' metric $\bar{g}$
    \item Apply conformal change $\tilde{g} = \phi^4 \bar{g}$ to seal the geometry
    \item Use the Robin BVP $\partial_\nu \phi = \frac{\tr_\Sigma k}{4} \phi$ at $\Sigma$
\end{enumerate}

The obstruction: When $\tr_\Sigma k < 0$, the maximum principle forces $\phi \geq 1$,
leading to mass increase.

\textbf{Key observation:} The obstruction involves $\tr_\Sigma k$, which is a property
of the \emph{extrinsic curvature} $k$, not the metric $g$.

\subsection{The New Idea}

\begin{key}[Strategy]
Instead of conformally changing $g$ to handle the boundary condition,
\textbf{change $k$} to make $\tr_\Sigma k \geq 0$ from the start.

Then the standard conformal method works directly!
\end{key}

The questions to answer:
\begin{enumerate}
    \item Can we modify $k \to \tilde{k}$ while preserving DEC?
    \item Can we preserve the trapping condition $\theta^+ \leq 0$?
    \item How does the ADM mass change under $k \to \tilde{k}$?
\end{enumerate}

%===========================================================================
\section{Constraint Equations and ADM Mass}
%===========================================================================

\subsection{The Constraint Equations}

Initial data $(M, g, k)$ satisfies:
\begin{align}
    R_g - |k|_g^2 + (\tr_g k)^2 &= 2\mu \quad \text{(Hamiltonian)} \label{eq:ham}\\
    \Div_g(k - (\tr_g k)g) &= J \quad \text{(Momentum)} \label{eq:mom}
\end{align}

The ADM mass depends only on $g$, not $k$:
\begin{equation}
    M_{\ADM}(g) = \frac{1}{16\pi} \lim_{r\to\infty} \oint_{S_r} (g_{ij,j} - g_{jj,i}) n^i \, dA
\end{equation}

\begin{key}[Observation]
Changing $k$ while keeping $g$ fixed \textbf{preserves the ADM mass exactly}!
\end{key}

\subsection{Constraint on $k$-Modifications}

If we change $k \to \tilde{k}$ with $g$ fixed, the constraint equations become:
\begin{align}
    R_g - |\tilde{k}|_g^2 + (\tr_g \tilde{k})^2 &= 2\tilde{\mu} \\
    \Div_g(\tilde{k} - (\tr_g \tilde{k})g) &= \tilde{J}
\end{align}

The new matter content $(\tilde{\mu}, \tilde{J})$ is determined by $\tilde{k}$.

\textbf{Requirement:} DEC must hold: $\tilde{\mu} \geq |\tilde{J}|_g$.

%===========================================================================
\section{The Modification Construction}
%===========================================================================

\subsection{Goal}

Given $(M, g, k)$ with a trapped surface $\Sigma_0$ satisfying $\tr_{\Sigma_0} k < 0$,
construct $\tilde{k}$ such that:
\begin{enumerate}
    \item[(M1)] $\tr_{\Sigma_0} \tilde{k} \geq 0$ (favorable condition)
    \item[(M2)] $\theta^+_{\tilde{k}}(\Sigma_0) = H + \tr_{\Sigma_0}\tilde{k} \leq 0$ (still trapped)
    \item[(M3)] $(g, \tilde{k})$ satisfies DEC
\end{enumerate}

\subsection{Analysis of the Constraints}

From (M1) and (M2):
\begin{align}
    \tr_{\Sigma_0} \tilde{k} &\geq 0 \\
    H + \tr_{\Sigma_0}\tilde{k} &\leq 0
\end{align}

This gives $\tr_{\Sigma_0} \tilde{k} \in [0, -H]$.

Recall: For trapped surfaces, $H < 0$, so $-H > 0$.

\begin{lemma}[Existence of Target Value]
For any trapped surface $\Sigma_0$ with $H(\Sigma_0) < 0$:
\begin{itemize}
    \item If $\tr_{\Sigma_0} k < 0$: we need $\tr_{\Sigma_0} \tilde{k} \in [0, -H]$ (change required)
    \item If $\tr_{\Sigma_0} k \in [0, -H]$: already favorable, keep $\tilde{k} = k$
    \item If $\tr_{\Sigma_0} k > -H$: then $\theta^+ = H + \tr_\Sigma k > 0$, not trapped (contradiction)
\end{itemize}
\end{lemma}

So any trapped surface has $\tr_\Sigma k \leq -H$, and we want to increase it to $[0, -H]$.

\subsection{Local Modification}

\begin{definition}[Local $k$-Modification]
A \textbf{local $k$-modification} at $\Sigma_0$ is a symmetric 2-tensor $\Delta k$ with:
\begin{itemize}
    \item $\supp(\Delta k) \subset U$, a neighborhood of $\Sigma_0$
    \item $\tilde{k} := k + \Delta k$
\end{itemize}
\end{definition}

\begin{proposition}[Sufficient Conditions for (M1)--(M3)]
Let $\Delta k = \chi(x) \cdot h(x)$ where:
\begin{itemize}
    \item $\chi: M \to [0, 1]$ is a smooth cutoff with $\chi = 1$ on $\Sigma_0$
    \item $h(x)$ is a symmetric 2-tensor with $\tr_\Sigma h > 0$ at $\Sigma_0$
\end{itemize}

If we choose $h = |\tr_\Sigma k| \cdot g|_\Sigma$ (extend somehow to $M$), then:
\begin{equation}
    \tr_{\Sigma_0} \tilde{k} = \tr_{\Sigma_0} k + \chi|_{\Sigma_0} \cdot \tr_{\Sigma_0} h
    = \tr_{\Sigma_0} k + 2|\tr_\Sigma k| \geq 0
\end{equation}
(assuming $\tr_\Sigma k < 0$, so $|\tr_\Sigma k| = -\tr_\Sigma k$).
\end{proposition}

\subsection{The DEC Constraint}

The new matter content is:
\begin{align}
    2\tilde{\mu} &= R_g - |\tilde{k}|^2 + (\tr \tilde{k})^2 \\
    \tilde{J} &= \Div(\tilde{k} - (\tr\tilde{k})g)
\end{align}

For DEC: $\tilde{\mu} \geq |\tilde{J}|$.

\begin{lemma}[DEC Change Under $k$-Modification]
Let $\tilde{k} = k + \Delta k$. Then:
\begin{align}
    \tilde{\mu} - \mu &= \frac{1}{2}\left(-|\tilde{k}|^2 + |k|^2 + (\tr\tilde{k})^2 - (\tr k)^2\right) \\
    &= \frac{1}{2}\left(-2\langle k, \Delta k\rangle - |\Delta k|^2 + 2(\tr k)(\tr\Delta k) + (\tr\Delta k)^2\right)
\end{align}
\end{lemma}

\begin{gap}[DEC Preservation]
The DEC condition $\tilde{\mu} \geq |\tilde{J}|$ is \textbf{not automatically preserved}.

If $\Delta k$ is large, $\tilde{\mu}$ could become negative.
If $\Delta k$ is chosen poorly, $|\tilde{J}|$ could exceed $\tilde{\mu}$.

\textbf{Question:} Can we choose $\Delta k$ small enough to preserve DEC while
still achieving $\tr_\Sigma \tilde{k} \geq 0$?
\end{gap}

%===========================================================================
\section{The Perturbative Regime}
%===========================================================================

\subsection{Small $|\tr_\Sigma k|$ Case}

Suppose $\tr_\Sigma k = -\epsilon$ for small $\epsilon > 0$. We need:
\begin{equation}
    \Delta k \text{ such that } \tr_\Sigma(\Delta k) = \epsilon
\end{equation}

\begin{proposition}[Perturbative DEC]
For $|\Delta k| \lesssim \epsilon$, the DEC change is:
\begin{equation}
    \tilde{\mu} - \mu = O(\epsilon), \quad |\tilde{J} - J| = O(\epsilon)
\end{equation}
If $(g, k)$ satisfies DEC strictly ($\mu > |J| + \delta$ for some $\delta > 0$),
then $(g, \tilde{k})$ also satisfies DEC for $\epsilon$ small enough.
\end{proposition}

\textbf{Conclusion:} For trapped surfaces with $\tr_\Sigma k$ close to 0,
the modification works!

\subsection{Large $|\tr_\Sigma k|$ Case}

When $|\tr_\Sigma k|$ is large, the required $\Delta k$ is large, and
DEC preservation becomes delicate.

\begin{conjecture}[Bounded Modification]
For any trapped surface $\Sigma_0$ in DEC initial data, there exists
$\tilde{k}$ satisfying (M1)--(M3) with:
\begin{equation}
    |\tilde{k} - k|_{L^\infty} \leq C \cdot |\tr_{\Sigma_0} k|
\end{equation}
where $C$ depends only on geometric bounds.
\end{conjecture}

%===========================================================================
\section{Global Modification Strategy}
%===========================================================================

\subsection{The Full Construction}

To avoid the DEC issue, consider a \textbf{non-local modification}:

\begin{definition}[Compensated Modification]
A \textbf{compensated modification} is $\tilde{k} = k + \Delta k^{\text{loc}} + \Delta k^{\text{comp}}$
where:
\begin{itemize}
    \item $\Delta k^{\text{loc}}$ achieves (M1): $\tr_\Sigma \tilde{k} \geq 0$
    \item $\Delta k^{\text{comp}}$ is supported away from $\Sigma_0$ and compensates
    to restore DEC globally
\end{itemize}
\end{definition}

\begin{proposition}[Existence of Compensation]
The momentum constraint allows adding divergence-free tensors to $k$ without
changing $J$. We can use this freedom to adjust $\tilde{\mu}$ in regions
away from $\Sigma_0$.
\end{proposition}

\textbf{Mechanism:}
\begin{enumerate}
    \item Near $\Sigma_0$: add $\Delta k^{\text{loc}}$ to flip $\tr_\Sigma k$ to positive
    \item This may violate DEC locally
    \item Far from $\Sigma_0$: add $\Delta k^{\text{comp}}$ to increase $\tilde{\mu}$
    in the asymptotic region, compensating for the local DEC violation
\end{enumerate}

\begin{gap}[Global DEC]
The compensation strategy requires careful analysis:
\begin{itemize}
    \item Can we always find $\Delta k^{\text{comp}}$ restoring DEC?
    \item Does the compensation affect the ADM mass? (Answer: No, since ADM
    mass depends only on $g$, not $k$.)
    \item Does the compensation affect other trapped surfaces?
\end{itemize}
\end{gap}

%===========================================================================
\section{The Spacetime Perspective}
%===========================================================================

\subsection{Cauchy Data and Einstein Equations}

Changing $k$ while keeping $g$ fixed corresponds to choosing a different
embedding of the initial data slice into spacetime.

\begin{proposition}[Embedding Freedom]
For fixed $(M, g)$, different choices of $k$ satisfying the constraints
correspond to different ways of embedding $M$ as a Cauchy surface in
various vacuum spacetimes.
\end{proposition}

\textbf{Implication:} The modification $k \to \tilde{k}$ can be viewed as
``tilting'' the initial data slice in the ambient spacetime.

\subsection{Does Tilting Preserve Trapping?}

The trapping condition $\theta^+ = H + \tr_\Sigma k \leq 0$ involves both
the intrinsic geometry ($H$) and the extrinsic embedding ($\tr_\Sigma k$).

\begin{lemma}[Trapping Under Tilting]
If we change $k \to \tilde{k}$ with $\tr_\Sigma \tilde{k} = \tr_\Sigma k + \Delta$:
\begin{equation}
    \theta^+_{\tilde{k}}(\Sigma) = H + \tr_\Sigma k + \Delta = \theta^+_k(\Sigma) + \Delta
\end{equation}
So trapping is preserved iff $\Delta \leq -\theta^+_k(\Sigma)$.

Since $\theta^+_k(\Sigma) \leq 0$ for trapped surfaces, $-\theta^+_k(\Sigma) \geq 0$.
We can add $\Delta \in [0, -\theta^+_k(\Sigma)]$ without breaking trapping.
\end{lemma}

\textbf{Conclusion:} There is room to increase $\tr_\Sigma k$ while preserving trapping!

%===========================================================================
\section{Main Theorem (Conditional)}
%===========================================================================

\begin{theorem}[Favorable Reduction---Conditional]\label{thm:reduction}
Let $(M, g, k)$ be asymptotically flat initial data satisfying DEC, with a
trapped surface $\Sigma_0$ having $\tr_{\Sigma_0} k < 0$.

\textbf{Assume:} There exists a modification $\tilde{k}$ satisfying:
\begin{enumerate}
    \item[(M1)] $\tr_{\Sigma_0} \tilde{k} = 0$
    \item[(M2)] $\theta^+_{\tilde{k}}(\Sigma_0) \leq 0$
    \item[(M3)] $(g, \tilde{k})$ satisfies DEC
\end{enumerate}

\textbf{Then:} The Penrose inequality holds for $\Sigma_0$:
\begin{equation}
    M_{\ADM}(g) \geq \sqrt{\frac{\Area(\Sigma_0)}{16\pi}}
\end{equation}
\end{theorem}

\begin{proof}
\begin{enumerate}
    \item By (M1) and (M2), $\Sigma_0$ is a trapped surface in $(g, \tilde{k})$
    with $\tr_{\Sigma_0} \tilde{k} = 0$ (exactly favorable).
    
    \item By (M3), the data $(g, \tilde{k})$ satisfies DEC.
    
    \item Apply the standard Jang method to $(g, \tilde{k})$: the Robin BVP
    has $\alpha = \tr_{\Sigma_0}\tilde{k}/4 = 0$, so there is no obstruction.
    
    \item The Penrose inequality for $(g, \tilde{k})$ gives:
    \begin{equation}
        M_{\ADM}(g) \geq \sqrt{\frac{\Area(\Sigma_0)}{16\pi}}
    \end{equation}
    (using that $M_{\ADM}$ depends only on $g$, not on $\tilde{k}$).
\end{enumerate}
\end{proof}

%===========================================================================
\section{Critical Gap Analysis}
%===========================================================================

\begin{gap}[The Main Gap]
Theorem~\ref{thm:reduction} is conditional on the existence of $\tilde{k}$.
The key open problem is:

\textbf{Does such $\tilde{k}$ always exist?}

Specifically:
\begin{itemize}
    \item Can we solve the constraints with boundary condition $\tr_\Sigma \tilde{k} = 0$?
    \item If so, does the solution satisfy DEC?
    \item Does the modification preserve asymptotic flatness?
\end{itemize}
\end{gap}

\subsection{Conformal Method for $k$}

The York decomposition writes:
\begin{equation}
    k = \frac{1}{3}(\tr k) g + \sigma + LW
\end{equation}
where $\sigma$ is transverse-traceless and $LW = \nabla W + W \otimes - \frac{2}{3}(\Div W)g$
is the conformal Killing derivative of a vector field $W$.

To change $\tr_\Sigma k$, we can:
\begin{itemize}
    \item Change the trace part $\frac{1}{3}(\tr k)g$ globally
    \item This affects the Hamiltonian constraint
    \item Requires solving for new conformal factor to restore DEC
\end{itemize}

This brings us back to a conformal equation! But the structure is different.

\subsection{Connection to Jang Equation}

The Jang equation produces a blowup along MOTS, creating an ``artificial''
boundary. The boundary condition involves $\tr_\Sigma k$.

\textbf{Alternative:} Solve a modified Jang equation with target $\tr_\Sigma \tilde{k} = 0$.

\begin{conjecture}[Modified Jang]
There exists a generalization of the Jang equation that:
\begin{enumerate}
    \item Blows up on a trapped surface $\Sigma_0$
    \item Produces a Jang metric $\bar{g}$ with effective boundary data
    $\tr_\Sigma \bar{k} = 0$ (independent of the original $\tr_\Sigma k$)
\end{enumerate}
\end{conjecture}

%===========================================================================
\section{Conclusion}
%===========================================================================

\textbf{Summary:}
\begin{itemize}
    \item The $k$-modification approach has potential: ADM mass is preserved,
    and there is room in the trapping condition to modify $\tr_\Sigma k$.
    
    \item The main gap is showing that DEC-preserving modifications exist.
    
    \item The perturbative case ($\tr_\Sigma k$ small negative) appears tractable.
    
    \item The general case requires new techniques for solving the constraints
    with specified boundary behavior.
\end{itemize}

\begin{tcolorbox}[colback=yellow!10, colframe=orange!75!black, title=Assessment]
\textbf{Novelty:} High---this approach of modifying $k$ instead of $g$ is 
largely unexplored in the Penrose inequality literature.

\textbf{Potential:} High---if the existence of $\tilde{k}$ can be proven,
the unconditional Penrose inequality would follow immediately.

\textbf{Difficulty:} The constraint equations with prescribed boundary behavior
are well-studied but delicate. The DEC preservation is the crux.
\end{tcolorbox}

\end{document}
