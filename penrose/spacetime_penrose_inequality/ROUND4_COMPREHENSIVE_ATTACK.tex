%% ROUND4_COMPREHENSIVE_ATTACK.tex
%%
%% ROUND 4: Comprehensive Attack Synthesis
%%
%% Summary of all new approaches and their status
%%
%% December 2025

\documentclass[11pt]{amsart}
\usepackage{amsmath,amssymb,amsthm}
\usepackage{xcolor}
\usepackage{tcolorbox}
\usepackage{booktabs}

\tcbuselibrary{theorems}

\newtcolorbox{summary}{
    colback=blue!5!white,
    colframe=blue!75!black,
    title={\textbf{SUMMARY}}
}

\newtcolorbox{breakthrough}{
    colback=green!5!white,
    colframe=green!75!black,
    title={\textbf{BREAKTHROUGH}}
}

\newtcolorbox{deadend}{
    colback=red!5!white,
    colframe=red!75!black,
    title={\textbf{DEAD END}}
}

\newtheorem{theorem}{Theorem}[section]
\newtheorem{conjecture}[theorem]{Conjecture}

\newcommand{\ADM}{\mathrm{ADM}}
\newcommand{\Area}{\mathrm{Area}}
\newcommand{\tr}{\mathrm{tr}}
\newcommand{\Cap}{\mathrm{Cap}}

\title{Round 4: Comprehensive Attack Synthesis\\
\large Status of the 1973 Penrose Conjecture}
\author{}
\date{December 2025}

\begin{document}
\maketitle

\begin{abstract}
We synthesize all attack vectors from Rounds 3 and 4, identifying dead ends, partial results, and the most promising paths forward. The fundamental obstruction — area dominance — remains unsolved, but we identify a potential new approach via the $\theta^+$-weighted Hawking mass.
\end{abstract}

\tableofcontents

%% ============================================================================
\section{Attack Summary Table}
%% ============================================================================

\begin{center}
\begin{tabular}{@{}p{3.5cm}p{4cm}p{3cm}p{2.5cm}@{}}
\toprule
\textbf{Approach} & \textbf{Key Idea} & \textbf{Obstacle} & \textbf{Status} \\
\midrule
\multicolumn{4}{c}{\textit{Round 3}} \\
\midrule
Hull + Maximal & Hull has $H \ge 0$ & No maximal slice & \textcolor{red}{Failed} \\
WCC + Hawking & Dynamic area increase & Spatial comparison & \textcolor{orange}{Incomplete} \\
Perturbative (1st) & Stability from spherical & Higher orders needed & \textcolor{green}{Works} \\
\midrule
\multicolumn{4}{c}{\textit{Round 4}} \\
\midrule
Capacity-Area & $A \le \Cap^2/(4\pi)$ & Wrong direction! & \textcolor{red}{Failed} \\
Perturbative (2nd) & Gap stability & Gap decreases & \textcolor{orange}{Partial} \\
$\theta^+$-Inverse Flow & Flow by $1/\theta^+$ & Goes inward & \textcolor{red}{Failed} \\
Two-Phase Flow & $\Sigma \to \Sigma^* \to \infty$ & No monotonicity & \textcolor{red}{Failed} \\
$\theta^+$-Weighted Mass & $m_\theta = \sqrt{A/16\pi}(1 - \int\theta^2)$ & Monotonicity? & \textcolor{green}{Promising} \\
\bottomrule
\end{tabular}
\end{center}

%% ============================================================================
\section{Dead Ends}
%% ============================================================================

\begin{deadend}
\textbf{1. Capacity-Area Isoperimetric}

We showed:
\begin{itemize}
    \item In Euclidean $\mathbb{R}^3$: $A \ge \Cap^2/(4\pi)$ — OPPOSITE direction
    \item In Schwarzschild: $A = \Cap^2/(4\pi)$ — equality
\end{itemize}

The inequality we need ($A \le \Cap^2/(4\pi)$) is FALSE in flat space. It's an artifact of the special geometry at black hole horizons.

\textbf{Verdict: Cannot use capacity to prove Penrose.}
\end{deadend}

\begin{deadend}
\textbf{2. Hull + Maximal Slice}

Fatal flaw: Not every trapped surface lies on a maximal Cauchy surface.

Even if we could prove the result on maximal slices, it wouldn't apply to general trapped surfaces.

\textbf{Verdict: Fundamental limitation, not fixable.}
\end{deadend}

\begin{deadend}
\textbf{3. Modified Flows ($\theta^+$-inverse, two-phase)}

\begin{itemize}
    \item $\theta^+$-inverse flow goes inward (wrong direction)
    \item Two-phase flow lacks area monotonicity in Phase 1
    \item No flow from trapped surface to infinity with monotone Hawking mass
\end{itemize}

\textbf{Verdict: Flow approaches don't work for H < 0 starting surfaces.}
\end{deadend}

%% ============================================================================
\section{Partial Results}
%% ============================================================================

\subsection{First-Order Perturbative}

\begin{theorem}[First-Order Stability]
For axisymmetric perturbations from Schwarzschild:
\begin{itemize}
    \item MOTS area unchanged to $O(\varepsilon)$
    \item Trapped surface area unchanged to $O(\varepsilon)$  
    \item Area dominance $A(\Sigma) < A(\Sigma^*)$ preserved to first order
\end{itemize}
\end{theorem}

\textbf{Significance:} Area dominance is not immediately violated by small perturbations.

\subsection{Second-Order Perturbative}

\begin{theorem}[Second-Order Behavior]
At second order:
\begin{itemize}
    \item MOTS area tends to decrease (stability effect)
    \item Trapped surface area tends to increase (metric effect)
    \item Area gap $\Delta A = A^* - A$ decreases
\end{itemize}
\end{theorem}

\textbf{Implication:} Area dominance holds for:
\begin{equation}
    \varepsilon < C \cdot \sqrt{\text{dist}(\Sigma, \Sigma^*)}
\end{equation}

For trapped surfaces bounded away from MOTS: area dominance is stable.

For trapped surfaces arbitrarily close to MOTS: perturbative bound degenerates.

\subsection{WCC + Hawking}

What WCC gives:
\begin{enumerate}
    \item Trapped surface is inside black hole ✓
    \item Apparent horizon inside event horizon ✓
    \item Event horizon area non-decreasing in time ✓
\end{enumerate}

What WCC does NOT give:
\begin{enumerate}
    \item $A(\text{trapped}) \le A(\text{apparent})$ — spatial comparison
\end{enumerate}

\textbf{Status:} The Hawking theorem is about time evolution, not spatial relationships.

%% ============================================================================
\section{The Promising Direction: $\theta^+$-Weighted Mass}
%% ============================================================================

\begin{breakthrough}
\textbf{Definition:}
\begin{equation}
    m_\theta(\Sigma) = \sqrt{\frac{A(\Sigma)}{16\pi}}\left(1 - \frac{1}{16\pi}\int_\Sigma (\theta^+)^2 dA\right)
\end{equation}

\textbf{Properties:}
\begin{enumerate}
    \item For MOTS ($\theta^+ = 0$): $m_\theta = \sqrt{A/(16\pi)}$
    \item For trapped ($\theta^+ < 0$): $m_\theta < \sqrt{A/(16\pi)}$
    \item For weakly trapped ($|\theta^+|$ small): $m_\theta \approx \sqrt{A/(16\pi)}$
\end{enumerate}

\textbf{Conjecture:}
\begin{equation}
    M_{\ADM} \ge m_\theta(\Sigma)
\end{equation}
for any surface $\Sigma$ in asymptotically flat initial data with DEC.

\textbf{If true:} This gives:
\begin{equation}
    M_{\ADM} \ge \sqrt{\frac{A}{16\pi}}\left(1 - \frac{1}{16\pi}\int (\theta^+)^2 dA\right)
\end{equation}

For "not too trapped" surfaces where $\int (\theta^+)^2 dA \ll 16\pi$, this approaches Penrose.
\end{breakthrough}

\subsection{Why This Might Work}

The $\theta^+$-weighted mass:
\begin{enumerate}
    \item Interpolates between Hawking mass (for $\theta^+ = H$) and the Penrose bound
    \item Naturally incorporates the spacetime information via $k$
    \item Reduces to the correct value at MOTS
    \item Is bounded below by 0 for physically reasonable surfaces
\end{enumerate}

\subsection{What's Needed}

To prove $M_{\ADM} \ge m_\theta$:
\begin{enumerate}
    \item A monotonicity formula along some flow
    \item OR a direct integral identity relating $m_\theta$ to the constraints
    \item OR a spinor proof with modified boundary conditions
\end{enumerate}

%% ============================================================================
\section{The Current State of Knowledge}
%% ============================================================================

\begin{summary}
\textbf{What IS proven:}

\begin{enumerate}
    \item \textbf{Conditional Penrose:} $M \ge \sqrt{A(\Sigma)/(16\pi)}$ assuming area dominance (OM)
    \item \textbf{Spherically symmetric:} Full Penrose holds
    \item \textbf{Time-symmetric:} Full Riemannian Penrose holds (Bray, Huisken-Ilmanen)
    \item \textbf{First-order perturbative:} Area dominance stable under small perturbations
\end{enumerate}

\textbf{What is NOT proven:}

\begin{enumerate}
    \item Area dominance $A(\Sigma) \le A(\Sigma^*)$ without additional assumptions
    \item Full Penrose for general trapped surfaces
    \item Any direct bound from trapped surface area to ADM mass
\end{enumerate}

\textbf{The fundamental obstruction:}

Trapped surfaces have no variational characterization. MOTS is not area-extremal. Null geometry defeats flows.
\end{summary}

%% ============================================================================
\section{Research Directions}
%% ============================================================================

\subsection{Short-Term (Provable)}

\begin{enumerate}
    \item Complete second-order perturbative analysis for general perturbations
    \item Prove $M_{\ADM} \ge m_\theta$ for surfaces with $\theta^+ = 0$ (MOTS) — should reduce to known result
    \item Investigate monotonicity of $m_\theta$ under modified flows
\end{enumerate}

\subsection{Medium-Term (Challenging)}

\begin{enumerate}
    \item Prove or disprove area dominance for a broader class of spacetimes
    \item Develop spinor methods adapted to the trapped condition
    \item Find the right quasi-local mass that's monotonic from trapped surfaces
\end{enumerate}

\subsection{Long-Term (Open)}

\begin{enumerate}
    \item Prove or disprove Penrose 1973 in full generality
    \item Understand the relationship between area dominance and cosmic censorship
    \item Extend to charged (Reissner-Nordström) and rotating (Kerr) cases
\end{enumerate}

%% ============================================================================
\section{Documents Produced}
%% ============================================================================

Round 4 produced:
\begin{enumerate}
    \item \texttt{CAPACITY\_ISOPERIMETRIC\_PROOF.tex} — Showed wrong direction
    \item \texttt{SECOND\_ORDER\_PERTURBATIVE.tex} — Gap decreases at 2nd order
    \item \texttt{MODIFIED\_GEROCH\_FLOW.tex} — Various flow attempts, $m_\theta$ introduced
    \item \texttt{ROUND4\_COMPREHENSIVE\_ATTACK.tex} — This synthesis
\end{enumerate}

Previous rounds produced:
\begin{enumerate}
    \item Round 1-2: PDE analysis, Jang equation, area dominance attacks
    \item Round 3: Hull approach, WCC+Hawking, first-order perturbative
\end{enumerate}

%% ============================================================================
\section{Conclusion}
%% ============================================================================

\begin{summary}
\textbf{The 1973 Penrose Conjecture remains open.}

After four rounds of adversarial analysis testing 20+ approaches:

\begin{enumerate}
    \item The conditional theorem (assuming OM) is \textbf{proven}
    \item Area dominance remains the \textbf{critical gap}
    \item Capacity and hull approaches are \textbf{dead ends}
    \item Perturbative approach gives \textbf{partial stability}
    \item The $\theta^+$-weighted mass is the \textbf{most promising new direction}
\end{enumerate}

\textbf{Key insight:} The problem is not finding the right technical tool — it's that trapped surfaces fundamentally lack the geometric structure (variational principle, extremality) that makes minimal surface theory work.

\textbf{Possible resolution:} 
\begin{itemize}
    \item Area dominance might require a physical assumption (WCC) to be true
    \item OR there might exist pathological initial data where area dominance fails
    \item OR a completely new geometric insight is needed
\end{itemize}

The conjecture is likely true (it holds in all known examples), but a proof remains one of the major open problems in mathematical general relativity.
\end{summary}

\end{document}
