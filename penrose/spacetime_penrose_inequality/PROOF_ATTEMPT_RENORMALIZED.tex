\documentclass[11pt]{article}
\usepackage{amsmath,amssymb,amsthm}
\usepackage[margin=1in]{geometry}
\usepackage{tcolorbox}
\usepackage{xcolor}

\newtheorem{theorem}{Theorem}[section]
\newtheorem{lemma}[theorem]{Lemma}
\newtheorem{proposition}[theorem]{Proposition}
\newtheorem{corollary}[theorem]{Corollary}
\newtheorem{definition}[theorem]{Definition}
\newtheorem{remark}[theorem]{Remark}
\newtheorem{openproblem}{Open Problem}

\newtcolorbox{gap}{colback=red!5!white, colframe=red!75!black, title=\textbf{GAP}}
\newtcolorbox{success}{colback=green!5!white, colframe=green!75!black, title=\textbf{RIGOROUS}}

\title{\textbf{Proof Attempt: Renormalized Area Method}\\
\large Toward the Unconditional Spacetime Penrose Inequality}
\author{Research Notes}
\date{December 2025}

\begin{document}
\maketitle

\begin{abstract}
We attempt to prove the spacetime Penrose inequality by defining a \textbf{renormalized area} that subtracts the contribution from unfavorable $\mathrm{tr}_\Sigma k < 0$. The goal is to prove a renormalized inequality that implies the original.
\end{abstract}

%==============================================================================
\section{The Renormalized Area}
%==============================================================================

\begin{definition}[Renormalized Area]
For a surface $\Sigma$ in initial data $(M, g, k)$, define:
\begin{equation}
A_{\mathrm{ren}}(\Sigma; \kappa) := A(\Sigma) - \kappa \int_\Sigma (\mathrm{tr}_\Sigma k)^- \, dA
\end{equation}
where $(\mathrm{tr}_\Sigma k)^- := \max(0, -\mathrm{tr}_\Sigma k)$ is the negative part, and $\kappa > 0$ is a renormalization parameter.
\end{definition}

\begin{lemma}[Basic Properties]
\begin{enumerate}
\item $A_{\mathrm{ren}}(\Sigma; \kappa) \leq A(\Sigma)$ with equality iff $\mathrm{tr}_\Sigma k \geq 0$ everywhere
\item $A_{\mathrm{ren}}(\Sigma; \kappa) \geq A(\Sigma) - \kappa \int_\Sigma |\mathrm{tr}_\Sigma k| \, dA$
\item For MOTS ($\theta^+ = 0$): $\mathrm{tr}_\Sigma k = H \geq 0$ by stability, so $A_{\mathrm{ren}} = A$
\end{enumerate}
\end{lemma}

\begin{success}
Properties 1-2 are immediate from the definition. Property 3 uses the known result that outermost stable MOTS have $\mathrm{tr}_\Sigma k = H \geq 0$.
\end{success}

%==============================================================================
\section{The Renormalized Penrose Inequality}
%==============================================================================

\begin{conjecture}[Renormalized Penrose Inequality]
For asymptotically flat $(M^3, g, k)$ satisfying DEC, and any trapped surface $\Sigma_0$:
\begin{equation}
M_{\mathrm{ADM}} \geq \sqrt{\frac{A_{\mathrm{ren}}(\Sigma_0; \kappa_*)}{16\pi}}
\end{equation}
where $\kappa_* = \kappa_*(M, g, k)$ is an optimal renormalization scale.
\end{conjecture}

\begin{remark}[Why This Would Suffice]
If the renormalized inequality holds, then since $A_{\mathrm{ren}} \leq A$:
\begin{equation}
M_{\mathrm{ADM}} \geq \sqrt{\frac{A_{\mathrm{ren}}(\Sigma_0)}{16\pi}} \leq \sqrt{\frac{A(\Sigma_0)}{16\pi}}
\end{equation}
Wait, this goes the wrong way! We need $A_{\mathrm{ren}} \geq$ something, not $\leq$.
\end{remark}

\begin{gap}
\textbf{Fatal flaw:} The renormalized area satisfies $A_{\mathrm{ren}} \leq A$. A lower bound on $M_{\mathrm{ADM}}$ in terms of $A_{\mathrm{ren}}$ is \textbf{weaker} than the original Penrose inequality, not stronger!

To get the original inequality, we would need to prove:
\begin{equation}
M_{\mathrm{ADM}} \geq \sqrt{\frac{A(\Sigma_0)}{16\pi}} \geq \sqrt{\frac{A_{\mathrm{ren}}(\Sigma_0)}{16\pi}}
\end{equation}
But the first inequality is exactly what we're trying to prove!
\end{gap}

%==============================================================================
\section{Alternative: Augmented Area}
%==============================================================================

Let's try the opposite: \textbf{add} a positive term to compensate.

\begin{definition}[Augmented Area]
\begin{equation}
A_{\mathrm{aug}}(\Sigma; \lambda) := A(\Sigma) + \lambda \int_\Sigma (\mathrm{tr}_\Sigma k)^- \, dA
\end{equation}
This satisfies $A_{\mathrm{aug}} \geq A$.
\end{definition}

\begin{conjecture}[Augmented Penrose Inequality]
For trapped surfaces:
\begin{equation}
M_{\mathrm{ADM}} \geq \sqrt{\frac{A(\Sigma_0)}{16\pi}} \quad \Longleftarrow \quad M_{\mathrm{ADM}} \geq \sqrt{\frac{A_{\mathrm{aug}}(\Sigma_0; \lambda_*)}{16\pi}}
\end{equation}
\end{conjecture}

\begin{gap}
Now $A_{\mathrm{aug}} \geq A$, so proving the augmented inequality would imply the original. But this is \textbf{harder} to prove, not easier! The augmented area is larger, so the lower bound on mass is stronger.
\end{gap}

%==============================================================================
\section{The Compensation Approach}
%==============================================================================

\begin{definition}[Compensated Jang Scalar Curvature]
On the Jang manifold, define:
\begin{equation}
R_{\mathrm{comp}} := R_{\bar{g}} + 2|(\mathrm{tr}_\Sigma k)^-| \cdot \delta_\Sigma
\end{equation}
This adds a positive Dirac mass to cancel the negative contribution when $\mathrm{tr}_\Sigma k < 0$.
\end{definition}

\begin{lemma}[Compensated Curvature Sign]
If $[H]_{\bar{g}} = \mathrm{tr}_\Sigma k$, then:
\begin{equation}
R_{\bar{g}} = R_{\bar{g}}^{\mathrm{reg}} + 2\,\mathrm{tr}_\Sigma k \cdot \delta_\Sigma
\end{equation}
With compensation:
\begin{equation}
R_{\mathrm{comp}} = R_{\bar{g}}^{\mathrm{reg}} + 2\,\mathrm{tr}_\Sigma k \cdot \delta_\Sigma + 2|(\mathrm{tr}_\Sigma k)^-| \cdot \delta_\Sigma
\end{equation}
\begin{itemize}
\item If $\mathrm{tr}_\Sigma k \geq 0$: $R_{\mathrm{comp}} = R_{\bar{g}}^{\mathrm{reg}} + 2\,\mathrm{tr}_\Sigma k \cdot \delta_\Sigma \geq 0$ (assuming $R_{\bar{g}}^{\mathrm{reg}} \geq 0$ from DEC)
\item If $\mathrm{tr}_\Sigma k < 0$: $|(\mathrm{tr}_\Sigma k)^-| = -\mathrm{tr}_\Sigma k$, so:
\begin{equation}
R_{\mathrm{comp}} = R_{\bar{g}}^{\mathrm{reg}} + 2\,\mathrm{tr}_\Sigma k \cdot \delta_\Sigma - 2\,\mathrm{tr}_\Sigma k \cdot \delta_\Sigma = R_{\bar{g}}^{\mathrm{reg}} \geq 0
\end{equation}
\end{itemize}
\end{lemma}

\begin{success}
The compensated scalar curvature $R_{\mathrm{comp}} \geq 0$ distributionally!
\end{success}

%==============================================================================
\section{The Mass with Compensation}
%==============================================================================

\begin{theorem}[Compensated Mass]
Define the \textbf{compensated ADM mass}:
\begin{equation}
M_{\mathrm{comp}} := M_{\mathrm{ADM}}(\bar{g}) + \frac{1}{8\pi}\int_\Sigma |(\mathrm{tr}_\Sigma k)^-| \, dA
\end{equation}
Then the positive mass theorem applied to $R_{\mathrm{comp}} \geq 0$ gives:
\begin{equation}
M_{\mathrm{comp}} \geq 0
\end{equation}
\end{theorem}

\begin{gap}
\textbf{Problem:} Adding mass to compensate for negative curvature gives:
\begin{equation}
M_{\mathrm{ADM}}(\bar{g}) \geq -\frac{1}{8\pi}\int_\Sigma |(\mathrm{tr}_\Sigma k)^-| \, dA
\end{equation}
This is a \textbf{lower bound}, but it can be negative! It doesn't give $M_{\mathrm{ADM}} \geq 0$.

Moreover, the AMO monotonicity requires $R \geq 0$ for the \textbf{original} metric, not a modified one. Adding a Dirac mass changes the geometry in a way that invalidates the level set analysis.
\end{gap}

%==============================================================================
\section{Physical Interpretation}
%==============================================================================

\begin{remark}[Why Compensation Fails]
The compensation approach tries to ``add energy'' to make the curvature positive. But:
\begin{enumerate}
\item This increases the mass, not decreases it
\item The Penrose inequality bounds mass from \textbf{below} by area
\item Adding mass moves us in the wrong direction
\end{enumerate}

\textbf{Physical analogy:} It's like trying to prove ``your bank balance $\geq$ \$100'' by depositing money. Yes, you can make the balance positive, but that doesn't prove it was $\geq$ \$100 before!
\end{remark}

%==============================================================================
\section{Assessment}
%==============================================================================

\begin{tcolorbox}[colback=yellow!10!white, colframe=orange!75!black, title=\textbf{STATUS: FAILED}]
The Renormalized/Compensated Area approach fails because:

\textbf{Issue 1:} Renormalized area $A_{\mathrm{ren}} \leq A$ gives a weaker inequality, not stronger

\textbf{Issue 2:} Augmented area $A_{\mathrm{aug}} \geq A$ requires proving a stronger inequality

\textbf{Issue 3:} Compensating scalar curvature by adding mass goes in the wrong direction

\textbf{Fundamental problem:} The Penrose inequality is a \textbf{lower} bound on mass. Compensation techniques naturally give \textbf{upper} bounds or weaker lower bounds.

\textbf{Conclusion:} The renormalization/compensation approach does \textbf{not} work for proving the Penrose inequality.
\end{tcolorbox}

\end{document}
