% NEW MATHEMATICAL APPROACH TO THE AREA COMPARISON PROBLEM
% Draft notes - exploring rigorous alternatives

\documentclass{article}
\usepackage{amsmath,amsthm,amssymb}
\newtheorem{theorem}{Theorem}
\newtheorem{lemma}{Lemma}
\newtheorem{proposition}{Proposition}
\newtheorem{remark}{Remark}

\begin{document}

\section{The Problem}

We want to prove the Spacetime Penrose Inequality for ALL trapped surfaces $\Sigma_0$, not just those satisfying $\tr_{\Sigma_0} k \ge 0$.

\textbf{The obstacle:} When $\tr_{\Sigma_0} k < 0$, the mean curvature jump $[H] = \tr_{\Sigma_0} k < 0$ at the Jang interface, creating negative Dirac mass in the scalar curvature.

\textbf{Failed approach:} Prove $A(\Sigma^*) \ge A(\Sigma_0)$ where $\Sigma^*$ is the outermost MOTS. This is FALSE in general - counterexamples exist.

\section{Key Observation: What ARE the Counterexamples?}

The known counterexamples to $A(\Sigma^*) \ge A(\Sigma_0)$ involve:
\begin{itemize}
    \item Binary black hole mergers where inner MOTS have larger area than outer MOTS
    \item These compare TWO DIFFERENT MOTS, not a trapped surface vs enclosing MOTS
\end{itemize}

\textbf{Critical question:} Is there a counterexample where a \emph{strictly trapped surface} $\Sigma_0$ (not a MOTS) has larger area than the outermost MOTS $\Sigma^*$ enclosing it?

\section{Attempt 1: Hawking Mass Monotonicity}

The Hawking mass is:
\[
m_H(\Sigma) = \sqrt{\frac{A(\Sigma)}{16\pi}} \left(1 - \frac{1}{16\pi} \int_\Sigma H^2 \, dA\right)
\]

For trapped surfaces with $\theta^\pm < 0$:
\[
H = \frac{1}{2}(\theta^+ + \theta^-) < 0
\]

The Hawking mass is NOT monotonic along spacelike foliations without special structure.

\section{Attempt 2: Generalized Geroch Monotonicity}

Consider a foliation $\{\Sigma_t\}$ with $\Sigma_0$ at $t=0$ and $\Sigma^*$ at $t=1$.

Under IMCF ($\partial_t \Sigma = \nu/H$), the Hawking mass is monotonic when $R \ge 0$.

But in the trapped region, $H < 0$, so IMCF flows INWARD, not outward!

\textbf{Key insight:} Run IMCF from $\Sigma^*$ (where $H = \theta^-/2 < 0$) INWARD. This increases Hawking mass as we go inward. But does the flow reach $\Sigma_0$?

\section{Attempt 3: Spacetime Hawking Area Theorem}

In \emph{spacetime}, the Hawking area theorem says: along a future-directed null generator of a horizon, area is non-decreasing if NEC holds.

This involves the SPACETIME null expansion, not the initial data null expansion.

The connection: if $(M, g, k)$ embeds in a spacetime satisfying NEC, then...?

\section{Attempt 4: Direct Penrose Inequality Without Area Comparison}

\textbf{New idea:} Prove the Penrose inequality directly for $\Sigma_0$ without reducing to $\Sigma^*$.

The obstacle is $[H] = \tr_{\Sigma_0} k$ can be negative. But the TOTAL scalar curvature contribution might still be controlled.

Define the \textbf{defect functional}:
\[
\mathcal{D}(\Sigma_0) := \int_{\Sigma_0} (\tr_{\Sigma_0} k)_- \, dA
\]
where $(x)_- = \max(0, -x)$.

\textbf{Conjecture:} The defect is bounded by energy:
\[
\mathcal{D}(\Sigma_0) \le C \cdot M_{\mathrm{ADM}}
\]
for some universal $C$.

If true, a modified inequality might hold:
\[
M_{\mathrm{ADM}} + C' \mathcal{D}(\Sigma_0) \ge \sqrt{\frac{A(\Sigma_0)}{16\pi}}
\]

But this doesn't give the sharp constant.

\section{Attempt 5: The Bray-Khuri Generalized Jang with Modified Warping}

The standard Jang equation is $H_{\bar{g}} = \tr_{\bar{g}} k$.

\textbf{New idea:} Solve a MODIFIED Jang equation that forces $[H] \ge 0$ at ANY trapped surface.

Consider the \textbf{warped Jang equation}:
\[
H_{\bar{g}} = \tr_{\bar{g}} k + \lambda(x) \quad \text{where } \lambda \ge 0 \text{ is chosen to ensure } [H] \ge 0
\]

The question is whether adding $\lambda$ destroys the scalar curvature positivity from DEC.

\section{Attempt 6: Capacity-Weighted Penrose Inequality}

The $p$-capacity of $\Sigma_0$ in the Jang metric might provide the right comparison.

Define:
\[
\text{Cap}_p(\Sigma_0) = \inf \left\{ \int_M |\nabla u|^p \, dV : u|_{\Sigma_0} = 0, u \to 1 \text{ at } \infty \right\}
\]

The AMO monotonicity relates capacity to area and mass. Perhaps:
\[
M_{\mathrm{ADM}} \ge f(\text{Cap}_p(\Sigma_0)) \ge g(A(\Sigma_0))
\]

where the second inequality uses isoperimetric-type bounds.

\section{Attempt 7: Two-Sided Estimate on the Jump}

\textbf{Key observation:} The trapped conditions give:
\begin{align}
\theta^+ &= H + \tr k \le 0 \\
\theta^- &= H - \tr k < 0
\end{align}

Adding: $2H < 0$, so $H < 0$.
Subtracting: $2\tr k = \theta^+ - \theta^-$.

Since $\theta^+ \le 0$ and $\theta^- < 0$:
\[
\tr k = \frac{\theta^+ - \theta^-}{2}
\]

This can have ANY sign depending on which is more negative.

\textbf{But:} The TOTAL contribution to mass involves:
\[
\int_{\Sigma_0} [H] \, dA = \int_{\Sigma_0} \tr_{\Sigma_0} k \, dA = \int_{\Sigma_0} \frac{\theta^+ - \theta^-}{2} \, dA
\]

This integral might be controlled by global constraints from DEC + asymptotic flatness.

\section{Attempt 8: Positive Mass Theorem with Corners}

The key insight from Miao's work: a metric with corners and $[H] < 0$ has distributional scalar curvature with a negative Dirac mass.

But the Positive Mass Theorem says $M_{\mathrm{ADM}} \ge 0$ for $R \ge 0$.

\textbf{Question:} Is there a version of PMT that allows negative Dirac masses if they're ``trapped inside''?

The Schoen-Yau proof uses minimal surfaces. If the negative curvature is behind a minimal surface, it might not affect the mass.

\textbf{Conjecture (Trapped Positive Mass Theorem):}
Let $(M, g)$ have $R^{\mathrm{reg}} \ge 0$ and a Dirac mass $-c \cdot \delta_\Sigma$ where $\Sigma$ is a trapped surface enclosed by a stable minimal surface $\Sigma'$. Then $M_{\mathrm{ADM}} \ge 0$.

This would be NEW MATHEMATICS.

\section{Attempt 9: The Correct Statement}

After reflection, I believe the correct approach is:

\textbf{Theorem (Conditional Spacetime Penrose Inequality):}
The Spacetime Penrose Inequality $M_{\mathrm{ADM}} \ge \sqrt{A(\Sigma_0)/(16\pi)}$ holds for trapped surfaces satisfying EITHER:
\begin{enumerate}
    \item $\tr_{\Sigma_0} k \ge 0$ (favorable jump), OR
    \item $\Sigma_0$ is a stable MOTS
\end{enumerate}

For general trapped surfaces, the inequality
\[
M_{\mathrm{ADM}} \ge \sqrt{\frac{A(\Sigma^*)}{16\pi}}
\]
holds where $\Sigma^*$ is the outermost stable MOTS. The area comparison $A(\Sigma^*) \ge A(\Sigma_0)$ remains OPEN.

\section{Conclusion}

The gap identified by the referee is REAL. The paper should either:
\begin{enumerate}
    \item State the theorem conditionally (with favorable jump hypothesis), or
    \item Prove new mathematics (trapped PMT, area comparison, etc.)
\end{enumerate}

Without new ideas, option 1 is the honest approach.

\end{document}
