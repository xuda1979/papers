%% LEVEL_SET_AREA_THEOREM.tex
%%
%% A COMPLETELY NEW APPROACH: Level Set Method for Area Dominance
%%
%% Key Innovation: Use a carefully constructed function whose level sets
%% interpolate between Σ and Σ*, with controlled area behavior.
%%
%% December 2025

\documentclass[11pt]{amsart}
\usepackage{amsmath,amssymb,amsthm}
\usepackage{tcolorbox}

\tcbuselibrary{theorems}

\newtcolorbox{innovation}{
    colback=green!5!white,
    colframe=green!50!black,
    title={\textbf{INNOVATION}}
}

\newtcolorbox{keypoint}{
    colback=blue!5!white,
    colframe=blue!75!black,
    title={\textbf{KEY POINT}}
}

\newtcolorbox{warning}{
    colback=red!5!white,
    colframe=red!75!black,
    title={\textbf{WARNING}}
}

\newtheorem{theorem}{Theorem}
\newtheorem{lemma}[theorem]{Lemma}
\newtheorem{proposition}[theorem]{Proposition}
\newtheorem{corollary}[theorem]{Corollary}
\theoremstyle{definition}
\newtheorem{definition}[theorem]{Definition}
\newtheorem{remark}[theorem]{Remark}

\newcommand{\Area}{\mathrm{Area}}
\newcommand{\Vol}{\mathrm{Vol}}
\newcommand{\divv}{\mathrm{div}}
\DeclareMathOperator{\tr}{tr}

\title{The Level Set Method for Area Dominance:\\
Controlled Interpolation Between Surfaces}
\author{December 2025}

\begin{document}
\maketitle

\begin{abstract}
We develop a new approach to Area Dominance based on constructing a 
special function $u$ on the initial data $(\mathcal{C}, g, k)$ whose 
level sets interpolate between the trapped surface $\Sigma$ and the 
outermost MOTS $\Sigma^*$, with controlled area behavior.
\end{abstract}

%% ============================================================================
\section{Rigorous Area Dominance: The Variational Formulation}
%% ============================================================================

\begin{keypoint}
\textbf{Key Insight:} The area dominance inequality $A(\Sigma) \le A(\Sigma^*)$ 
is a \textbf{tautology} when $\Sigma^*$ is defined as an \textbf{area maximizer}
in an admissible class containing $\Sigma$. If instead $\Sigma^*$ is the 
\textbf{outermost MOTS}, then area dominance is \textbf{not automatic} and 
requires additional hypotheses or an intermediate max-area surface.
\end{keypoint}

\begin{lemma}[Area dominance for an area maximizer]\label{lem:area-dominance}
Let $\mathcal{A}$ be a nonempty class of closed surfaces in $(M,g)$ and let
$A(\cdot)$ denote the area functional (with respect to $g$).
Assume there exists a surface $\Sigma^* \in \mathcal{A}$ such that
\[
A(\Sigma^*)=\sup_{\Sigma\in\mathcal{A}} A(\Sigma).
\]
Then for every $\Sigma\in\mathcal{A}$ one has
\[
A(\Sigma)\le A(\Sigma^*).
\]
\end{lemma}

\begin{proof}
Fix any $\Sigma\in\mathcal{A}$. By definition of the supremum,
\[
A(\Sigma)\le \sup_{\Gamma\in\mathcal{A}} A(\Gamma) = A(\Sigma^*).
\]
This is exactly the claimed inequality.
\end{proof}

\begin{remark}[Alternative contradiction-style proof]
Suppose for contradiction that there exists $\Sigma\in\mathcal{A}$ with
$A(\Sigma)>A(\Sigma^*)$. Then
\[
\sup_{\Gamma\in\mathcal{A}}A(\Gamma)\ge A(\Sigma)>A(\Sigma^*),
\]
contradicting $A(\Sigma^*)=\sup_{\Gamma\in\mathcal{A}}A(\Gamma)$.
Hence $A(\Sigma)\le A(\Sigma^*)$ for all $\Sigma\in\mathcal{A}$.
\end{remark}

\begin{remark}[Application to Penrose Inequality]
For the spacetime Penrose inequality, take the admissible class to be:
\[
\mathcal{A} := \{\Sigma \subset \overline{\mathcal{T}} : 
\theta^+(\Sigma) \le 0,\ \theta^-(\Sigma) < 0\},
\]
and let $\Sigma^* := \Sigma_{\max}$ be the max-area trapped surface.
Then Lemma~\ref{lem:area-dominance} gives:
\[
A(\Sigma_0) \le A(\Sigma_{\max}).
\]
This is the ``area dominance'' step used in the Penrose inequality proof.
\end{remark}

\begin{warning}
\textbf{If $\Sigma^*$ is the outermost MOTS (not max-area):}

The inequality $A(\Sigma) \le A(\Sigma^*)$ is \textbf{not a purely 
variational tautology} and is \textbf{not true in general} without 
extra assumptions. In that setting, one must either:
\begin{enumerate}
    \item State it as an explicit assumption (``outer-minimizing'' or 
          ``area dominance'' hypothesis), or
    \item Replace $\Sigma^*$ by the max-area surface $\Sigma_{\max}$ and 
          use Lemma~\ref{lem:area-dominance}, which is rigorous once 
          existence of $\Sigma_{\max}$ is established under compactness 
          hypotheses.
\end{enumerate}
\end{warning}

%% ============================================================================
\section{The Strategy}
%% ============================================================================

\begin{innovation}
\textbf{The Level Set Strategy}

\begin{enumerate}
    \item Construct a function $u: \Omega \to [0, 1]$ on the region 
          $\Omega$ between $\Sigma$ and $\Sigma^*$
    \item $u = 0$ on $\Sigma$ and $u = 1$ on $\Sigma^*$
    \item Level sets $\Sigma_t = u^{-1}(t)$ interpolate between them
    \item Derive a differential equation for $\Area(\Sigma_t)$
    \item Show $\frac{d\Area}{dt} \ge 0$ to conclude Area Dominance
\end{enumerate}
\end{innovation}

%% ============================================================================
\section{The Co-Area Formula}
%% ============================================================================

\begin{theorem}[Co-Area Formula]
Let $u: \Omega \to \mathbb{R}$ be a smooth function with $|\nabla u| > 0$.
Let $\Sigma_t = u^{-1}(t)$ be the level sets.

Then for any function $f$ on $\Omega$:
\begin{equation}
    \int_\Omega f |\nabla u| \, dV = \int_{-\infty}^\infty 
    \left(\int_{\Sigma_t} f \, dA\right) dt
\end{equation}

In particular, with $f = 1$:
\begin{equation}
    \int_\Omega |\nabla u| \, dV = \int_0^1 \Area(\Sigma_t) \, dt
\end{equation}
\end{theorem}

%% ============================================================================
\section{The Area Formula via Mean Curvature}
%% ============================================================================

\begin{proposition}[Area Evolution along Level Sets]
Let $u: \Omega \to [0, 1]$ with $\Sigma_t = u^{-1}(t)$.

The area of $\Sigma_t$ evolves as:
\begin{equation}
    \frac{d\Area(\Sigma_t)}{dt} = \int_{\Sigma_t} H_t \cdot 
    \frac{1}{|\nabla u|} \, dA
\end{equation}

where $H_t$ is the mean curvature of $\Sigma_t$ (with respect to the 
outward unit normal $\nu = \frac{\nabla u}{|\nabla u|}$).
\end{proposition}

\begin{proof}
The level set $\Sigma_t$ moves with velocity $\frac{1}{|\nabla u|}$ in the 
normal direction as $t$ increases.

By the first variation of area:
\begin{equation}
    \frac{d\Area}{dt} = \int_{\Sigma_t} H \cdot (\text{normal velocity}) \, dA
    = \int_{\Sigma_t} \frac{H}{|\nabla u|} \, dA
\end{equation}
\end{proof}

%% ============================================================================
\section{The Key Observation}
%% ============================================================================

\begin{keypoint}
To prove $\Area(\Sigma_0) \le \Area(\Sigma_1)$, we need:
\begin{equation}
    \frac{d\Area(\Sigma_t)}{dt} \ge 0 \quad \text{for all } t \in [0, 1]
\end{equation}

This requires:
\begin{equation}
    \int_{\Sigma_t} \frac{H_t}{|\nabla u|} dA \ge 0
\end{equation}

If $|\nabla u| > 0$ is controlled, we need $\int_{\Sigma_t} H_t \, dA \ge 0$.
\end{keypoint}

\begin{warning}
For a TRAPPED surface $\Sigma_0$: $H = \theta^+ - P < \theta^+ < 0$ if $P > 0$.

So $H < 0$ is possible for trapped surfaces!

The level sets near $\Sigma_0$ may have $\int H \, dA < 0$.
\end{warning}

%% ============================================================================
\section{Construction: The Harmonic Function}
%% ============================================================================

\begin{definition}[Harmonic Interpolation]
Let $u$ be the solution to:
\begin{align}
    \Delta_g u &= 0 \quad \text{in } \Omega\\
    u|_\Sigma &= 0\\
    u|_{\Sigma^*} &= 1
\end{align}

This is the harmonic function with prescribed boundary values.
\end{definition}

\begin{proposition}[Properties of Harmonic $u$]
\begin{enumerate}
    \item $u$ exists and is unique (by maximum principle)
    \item $0 < u < 1$ in $\Omega^\circ$
    \item $|\nabla u| > 0$ generically (level sets are smooth)
    \item Level sets $\Sigma_t$ are smooth for almost all $t$
\end{enumerate}
\end{proposition}

%% ============================================================================
\section{The Mean Curvature of Harmonic Level Sets}
%% ============================================================================

\begin{proposition}[Mean Curvature Formula]
For level sets of a harmonic function $u$:
\begin{equation}
    H = -\frac{\Delta u}{|\nabla u|} + \frac{\nabla^2 u(\nabla u, \nabla u)}
    {|\nabla u|^3} = \frac{\nabla^2 u(\nabla u, \nabla u)}{|\nabla u|^3}
\end{equation}

since $\Delta u = 0$.
\end{proposition}

\begin{proof}
General formula: $H = \divv(\nu) = \divv\left(\frac{\nabla u}{|\nabla u|}\right)$

Expanding:
\begin{equation}
    H = \frac{\Delta u}{|\nabla u|} - \frac{\nabla^2 u(\nabla u, \nabla u)}
    {|\nabla u|^3}
\end{equation}

For harmonic $u$: $\Delta u = 0$, so:
\begin{equation}
    H = -\frac{\nabla^2 u(\nabla u, \nabla u)}{|\nabla u|^3}
\end{equation}

Wait, I need to be more careful with signs. Let me redo this.

The unit normal is $\nu = \frac{\nabla u}{|\nabla u|}$.

The mean curvature is $H = \divv(\nu)$.

$\divv(\nu) = \divv\left(\frac{\nabla u}{|\nabla u|}\right)
= \frac{1}{|\nabla u|}\divv(\nabla u) + \nabla u \cdot \nabla\left(
\frac{1}{|\nabla u|}\right)$

$= \frac{\Delta u}{|\nabla u|} - \frac{1}{|\nabla u|^2} 
\frac{\nabla^2 u(\nabla u, \nabla u)}{|\nabla u|}$

$= \frac{\Delta u}{|\nabla u|} - \frac{\nabla^2 u(\nabla u, \nabla u)}
{|\nabla u|^3}$

For harmonic $u$: 
\begin{equation}
    H = -\frac{\nabla^2 u(\nabla u, \nabla u)}{|\nabla u|^3}
\end{equation}
\end{proof}

%% ============================================================================
\section{The Sign of $H$ for Harmonic Level Sets}
%% ============================================================================

\begin{keypoint}
For harmonic level sets:
\begin{equation}
    H = -\frac{\nabla^2 u(\nabla u, \nabla u)}{|\nabla u|^3}
\end{equation}

The sign of $H$ depends on the sign of $\nabla^2 u(\nabla u, \nabla u)$.

This is the second derivative of $u$ in the gradient direction!
\end{keypoint}

\begin{lemma}[Convexity Interpretation]
$\nabla^2 u(\nabla u, \nabla u) = |\nabla u|^2 \cdot \frac{\partial^2 u}
{\partial s^2}$

where $s$ is arc length along gradient flow lines.

So:
\begin{itemize}
    \item $\nabla^2 u(\nabla u, \nabla u) > 0$ $\Leftrightarrow$ $u$ is 
          convex along flow lines $\Leftrightarrow$ $H < 0$
    \item $\nabla^2 u(\nabla u, \nabla u) < 0$ $\Leftrightarrow$ $u$ is 
          concave along flow lines $\Leftrightarrow$ $H > 0$
\end{itemize}
\end{lemma}

%% ============================================================================
\section{The Bochner Formula}
%% ============================================================================

\begin{theorem}[Bochner Formula for Harmonic Functions]
Let $u$ be harmonic ($\Delta u = 0$). Then:
\begin{equation}
    \frac{1}{2}\Delta|\nabla u|^2 = |\nabla^2 u|^2 + \text{Ric}(\nabla u, 
    \nabla u)
\end{equation}
\end{theorem}

\begin{keypoint}
The Bochner formula involves the Ricci curvature of $(\mathcal{C}, g)$.

The constraint equations relate Ric to $(g, k)$ and matter:
\begin{equation}
    R_g = |k|^2 - (\tr k)^2 + 16\pi\mu
\end{equation}

Under DEC: $\mu \ge |J| \ge 0$, so $\mu \ge 0$.

This gives: $R_g \ge |k|^2 - (\tr k)^2$.
\end{equation}
\end{keypoint}

%% ============================================================================
\section{The Integrated Mean Curvature}
%% ============================================================================

\begin{proposition}[Integral Formula]
For harmonic level sets:
\begin{equation}
    \int_{\Sigma_t} H \, dA = -\int_{\Sigma_t} 
    \frac{\nabla^2 u(\nabla u, \nabla u)}{|\nabla u|^3} dA
\end{equation}

By the divergence theorem applied to $\frac{\nabla u}{|\nabla u|}$:
\begin{equation}
    \int_{\Sigma_t} H \, dA = \int_{\Sigma_t} \divv\left(\frac{\nabla u}
    {|\nabla u|}\right) dA = 0
\end{equation}

Wait, that's not right. The divergence theorem relates:
\begin{equation}
    \int_\Omega \Delta u \, dV = \int_{\partial\Omega} 
    \frac{\partial u}{\partial\nu} dA
\end{equation}

For a single level set $\Sigma_t$, there's no direct integral relation.
\end{proposition}

%% ============================================================================
\section{A Different Approach: The Capacity}
%% ============================================================================

\begin{innovation}
\textbf{The Capacity Functional}

Define the capacity of the region $\Omega$ between $\Sigma$ and $\Sigma^*$:
\begin{equation}
    \text{Cap}(\Sigma, \Sigma^*) = \int_\Omega |\nabla u|^2 dV
\end{equation}

where $u$ is the harmonic function with $u|_\Sigma = 0$, $u|_{\Sigma^*} = 1$.
\end{innovation}

\begin{proposition}[Capacity and Boundary Flux]
\begin{equation}
    \text{Cap}(\Sigma, \Sigma^*) = \int_{\Sigma^*} |\nabla u| dA = 
    \int_\Sigma |\nabla u| dA
\end{equation}

(The flux through both boundaries is equal for harmonic functions.)
\end{proposition}

\begin{proof}
By the divergence theorem:
\begin{align}
    \int_\Omega |\nabla u|^2 dV &= \int_\Omega \nabla u \cdot \nabla u \, dV\\
    &= \int_\Omega \divv(u\nabla u) - u\Delta u \, dV\\
    &= \int_{\partial\Omega} u \frac{\partial u}{\partial\nu} dA - 0\\
    &= \int_{\Sigma^*} 1 \cdot |\nabla u| dA - \int_\Sigma 0 \cdot |\nabla u| dA\\
    &= \int_{\Sigma^*} |\nabla u| dA
\end{align}
\end{proof}

%% ============================================================================
\section{Capacity and Area}
%% ============================================================================

\begin{theorem}[Capacity-Area Inequality]
Let $\Sigma$ be a surface bounding region $\Omega_\Sigma$ with capacity 
$\text{Cap}(\Sigma)$ (relative to infinity).

In Euclidean space:
\begin{equation}
    \text{Cap}(\Sigma) \ge \frac{\Area(\Sigma)}{4\pi r}
\end{equation}

where $r$ is the "effective radius" related to $\Area(\Sigma) = 4\pi r^2$.
\end{theorem}

\begin{keypoint}
The capacity provides a bridge between analytical (harmonic functions) 
and geometric (area) quantities.

However, we need to relate capacity to the PHYSICS (DEC, constraints).
\end{keypoint}

%% ============================================================================
\section{The Fundamental Obstruction}
%% ============================================================================

\begin{warning}
\textbf{The harmonic function $u$ is determined by $(g, \Sigma, \Sigma^*)$.}

It does NOT directly involve:
\begin{itemize}
    \item The extrinsic curvature $k$
    \item The expansions $\theta^+, \theta^-$
    \item The DEC or constraint equations
\end{itemize}

So the level set approach using harmonic $u$ doesn't naturally connect 
to the physics of trapped surfaces!
\end{warning}

%% ============================================================================
\section{Modified Level Set: The $\theta^+$-Laplacian}
%% ============================================================================

\begin{innovation}
\textbf{The $\theta^+$-Harmonic Function}

Instead of the standard Laplacian, use a MODIFIED operator that 
incorporates $\theta^+$.

Define $u$ by:
\begin{equation}
    \Delta u - \frac{\theta^+}{H} |\nabla u| \cdot H = 0
\end{equation}

where $H$ is the mean curvature of the level set.

Hmm, this is circular (depends on $u$ through $H$).
\end{innovation}

Let me try a different modification.

\begin{innovation}
\textbf{The Expansion-Weighted Function}

Define $u$ as the solution to:
\begin{equation}
    \Delta u + \alpha \cdot |\nabla u| = 0
\end{equation}

where $\alpha$ is related to the constraint equations.

This is a quasilinear elliptic PDE (related to IMCF).
\end{innovation}

%% ============================================================================
\section{Connection to IMCF}
%% ============================================================================

The Inverse Mean Curvature Flow (IMCF) satisfies:
\begin{equation}
    \frac{\partial\Sigma}{\partial t} = \frac{\nu}{H}
\end{equation}

The level sets $u^{-1}(t)$ of the IMCF arrival time satisfy:
\begin{equation}
    \Delta u = |\nabla u|^2 \cdot H + \text{second order terms}
\end{equation}

\begin{keypoint}
IMCF is the RIGHT level set method for the Riemannian Penrose inequality!

Huisken-Ilmanen used weak IMCF to prove the Riemannian case.

But for the general case, we need a SPACETIME version of IMCF.
\end{keypoint}

%% ============================================================================
\section{The Spacetime IMCF?}
%% ============================================================================

\begin{innovation}
\textbf{Null Mean Curvature Flow}

Instead of flowing by $1/H$, flow by $1/\theta^+$ in a null direction:
\begin{equation}
    \frac{\partial\Sigma}{\partial t} = \frac{\ell}{\theta^+}
\end{equation}

This is singular at MOTS ($\theta^+ = 0$) and flows OUTWARD when 
$\theta^+ < 0$ (trapped).

BUT: This flows in the null direction, not in $\mathcal{C}$!
\end{innovation}

%% ============================================================================
\section{Synthesis: Why Level Sets Are Insufficient}
%% ============================================================================

\textbf{The fundamental issue:}

Level set methods work by controlling mean curvature $H$ along the foliation.

But Area Dominance involves:
\begin{itemize}
    \item Trapped surface: $\theta^+ < 0$, $\theta^- < 0$
    \item MOTS: $\theta^+ = 0$
\end{itemize}

The relationship: $\theta^+ = H + P$ mixes:
\begin{itemize}
    \item $H$ (intrinsic to $(\mathcal{C}, g)$)
    \item $P = \tr_\Sigma k$ (depends on extrinsic curvature $k$)
\end{itemize}

\begin{center}
\fbox{\parbox{0.85\textwidth}{
\textbf{Core Difficulty:}

Any level set approach in $\mathcal{C}$ naturally controls $H$.

But trapped/MOTS conditions involve $\theta^+ = H + P$.

The $P$ term spoils direct area monotonicity arguments.

We need to either:
\begin{enumerate}
    \item Find a level set method that controls $\theta^+$ directly
    \item Show $P$ has the right sign in our setting
    \item Use spacetime methods that incorporate both $H$ and $P$
\end{enumerate}
}}
\end{center}

%% ============================================================================
\section{Conclusion}
%% ============================================================================

The level set approach is a natural idea, but faces the same 
fundamental obstruction:

\textbf{The presence of extrinsic curvature $k$ (through $P = \tr_\Sigma k$) 
prevents direct translation from Riemannian methods.}

Possible paths forward:
\begin{enumerate}
    \item Develop a SPACETIME level set method (surfaces in 4D spacetime)
    \item Use the Jang equation to "quotient out" the $k$ dependence
    \item Accept that Area Dominance may require additional assumptions 
          on the initial data
\end{enumerate}

\end{document}
