% =========================================================================
%     RICCI FLOW APPROACH TO THE PENROSE INEQUALITY
%
%     Using geometric flows to improve the sign condition
%
%     Author: Da Xu
%     Date: December 2025
% =========================================================================

\documentclass[12pt]{article}
\usepackage{amsmath,amsthm,amssymb}
\usepackage{mathrsfs}
\usepackage{tcolorbox}

\theoremstyle{plain}
\newtheorem{theorem}{Theorem}[section]
\newtheorem{lemma}[theorem]{Lemma}
\newtheorem{proposition}[theorem]{Proposition}
\newtheorem{corollary}[theorem]{Corollary}

\theoremstyle{definition}
\newtheorem{definition}[theorem]{Definition}
\newtheorem{remark}[theorem]{Remark}

\newcommand{\ADM}{\mathrm{ADM}}
\newcommand{\tr}{\mathrm{tr}}
\newcommand{\Div}{\mathrm{div}}
\newcommand{\Area}{\mathrm{Area}}
\newcommand{\Ric}{\mathrm{Ric}}
\newcommand{\Rm}{\mathrm{Rm}}

\title{\textbf{Ricci Flow Approach to the Penrose Inequality}}
\author{Da Xu}
\date{December 2025}

\begin{document}
\maketitle

\section{Ricci Flow Basics}

\subsection{Hamilton's Ricci Flow}

The Ricci flow on a Riemannian manifold:
\[
    \partial_t g = -2\Ric_g
\]

Key properties:
\begin{itemize}
    \item Preserves non-negative scalar curvature in dimension 3
    \item Smooths out the metric
    \item Can develop singularities
\end{itemize}

\subsection{Evolution Equations}

Under Ricci flow:
\begin{align}
    \partial_t R &= \Delta R + 2|\Ric|^2 \\
    \partial_t dV &= -R \, dV \\
    \partial_t |\Rm|^2 &= \Delta|\Rm|^2 - 2|\nabla\Rm|^2 + \text{(curvature terms)}
\end{align}

\section{Ricci Flow on Asymptotically Flat Manifolds}

\subsection{ADM Mass Under Ricci Flow}

\begin{theorem}[Dai-Ma, Oliynyk-Woolgar]
Under normalized Ricci flow on asymptotically flat 3-manifolds:
\[
    \frac{dM_{\ADM}}{dt} = -\frac{1}{16\pi} \int_M R \cdot \text{(decay factor)} \, dV
\]
If $R \geq 0$, then $M_{\ADM}$ is non-increasing.
\end{theorem}

\subsection{Application to Penrose}

\textbf{Strategy:}
\begin{enumerate}
    \item Start with $(M, g_0)$ containing minimal surface $\Sigma$ with $R_{g_0} \geq 0$
    \item Run Ricci flow: $g_t$
    \item Track $M_{\ADM}(g_t)$ (decreasing) and $\Area_t(\Sigma)$
    \item At $t = T$, hopefully $g_T$ is Schwarzschild
\end{enumerate}

\textbf{Problem 1:} This is for the Riemannian case ($k = 0$).

\textbf{Problem 2:} Even for minimal surfaces, $\Area_t(\Sigma)$ changes under Ricci flow!

\section{Ricci Flow for Initial Data $(g, k)$}

\subsection{The Challenge}

For spacetime initial data, we need to:
\begin{enumerate}
    \item Evolve both $g$ and $k$
    \item Preserve the constraint equations
    \item Control the sign of $\tr_\Sigma k$
\end{enumerate}

\subsection{Constraint-Preserving Ricci Flow}

The constraints:
\begin{align}
    R_g + (\tr k)^2 - |k|^2 &= 16\pi\mu \\
    \Div(k - (\tr k)g) &= 8\pi J
\end{align}

If we apply pure Ricci flow to $g$, the constraints are violated.

\textbf{Solution:} Coupled flow on $(g, k)$.

\subsection{The List-Ricci Flow}

List proposed a coupled system:
\begin{align}
    \partial_t g &= -2\Ric_g + 2\alpha k \circ k \\
    \partial_t k &= \Delta_g k + \text{(lower order terms)}
\end{align}
where $\alpha$ is chosen to preserve constraints.

\textbf{Problem:} The long-time behavior is poorly understood, and it's unclear
whether $\tr_\Sigma k$ improves.

\section{Strategy: Flow to Favorable Sign}

\subsection{Goal}

Find a flow $(g_t, k_t)$ such that:
\begin{enumerate}
    \item Constraints are preserved
    \item $M_{\ADM}(g_t, k_t)$ is non-increasing
    \item For some $T$: $\tr_{\Sigma_T} k_T \geq 0$
\end{enumerate}

Then we can apply known methods at time $T$.

\subsection{Analysis}

The trace $\tr_\Sigma k$ evolves as:
\[
    \partial_t(\tr_\Sigma k) = \text{(intrinsic Laplacian)} + \text{(geometry terms)} + \text{(constraint terms)}
\]

For this to become positive, we need the RHS to be positive where $\tr_\Sigma k < 0$.

\textbf{Problem:} There's no general maximum principle that guarantees this!

The sign of $\tr_\Sigma k$ depends on the global geometry, not just local data.

\section{Alternative: Ricci Flow with Surgery}

\subsection{Idea}

Use Ricci flow with surgery (Perelman) to:
\begin{enumerate}
    \item Flow the metric until singularity
    \item Perform surgery at singularities
    \item Continue the flow
\end{enumerate}

\subsection{Surgery and Trapped Surfaces}

\begin{question}
Can surgery be performed to ``cut out'' the bad region where $\tr_\Sigma k < 0$?
\end{question}

\textbf{Analysis:} Surgery in Ricci flow replaces high-curvature regions with
standard caps. But the bad sign of $\tr_\Sigma k$ is not a curvature singularity—it's
a sign condition on second fundamental form.

\textbf{Problem:} Surgery doesn't directly address the sign issue.

\section{Approach: Conformal Ricci Flow}

\subsection{Definition}

The conformal Ricci flow:
\[
    \partial_t g = -2(\Ric_g - \lambda g)
\]
where $\lambda$ is chosen to preserve some property (volume, conformal class, etc.).

\subsection{Application}

Choose $\lambda$ to make the flow preserve constraints:
\[
    \lambda = \frac{1}{3}R - \frac{8\pi\mu}{3}
\]

Then the constraint equations are (approximately) preserved.

\textbf{Problem:} This doesn't improve the sign of $\tr_\Sigma k$ because
$\lambda$ depends on $R$ and $\mu$, not on $k$.

\section{Approach: Inverse Mean Curvature Ricci Flow}

\subsection{Coupled System}

Run Ricci flow on $g$ and IMCF on surfaces $\Sigma_t$ simultaneously:
\begin{align}
    \partial_t g &= -2\Ric_g \\
    \partial_t \Sigma &= \frac{\nu}{H}
\end{align}

\subsection{Analysis}

The Hawking mass under this coupled flow:
\[
    \frac{dm_H}{dt} = \frac{\partial m_H}{\partial g}\cdot\partial_t g + \frac{\partial m_H}{\partial\Sigma}\cdot\partial_t\Sigma
\]

The IMCF part contributes positively (if $R \geq 0$).
The Ricci flow part contributes... complicatedly.

\textbf{Problem:} No clean monotonicity for the coupled system.

\section{The Constraint Map Approach}

\subsection{The Constraint Map}

Define $\Phi: (g, k) \mapsto (\mu, J)$ by:
\begin{align}
    \mu &= \frac{1}{16\pi}(R_g + (\tr k)^2 - |k|^2) \\
    J &= \frac{1}{8\pi}\Div(k - (\tr k)g)
\end{align}

\subsection{Linearization}

The linearization $D\Phi$ maps infinitesimal changes in $(g, k)$ to changes in $(\mu, J)$.

\textbf{Key:} The kernel of $D\Phi$ consists of infinitesimal diffeomorphisms
(gauge directions).

\subsection{Gradient Flow}

Consider the gradient flow for the constraint violation:
\[
    \partial_t(g, k) = -D\Phi^*(\mu - \mu_0, J - J_0)
\]

This drives $(g, k)$ toward constraint satisfaction.

\textbf{Problem:} This flow doesn't have good geometric properties and
doesn't help with the sign of $\tr_\Sigma k$.

\begin{tcolorbox}[colback=yellow!10, colframe=orange!75!black, title=\textbf{Partial Insight}]
\textbf{Observation:} The sign of $\tr_\Sigma k$ is not a ``local'' obstruction
that can be fixed by local flows.

It's a \textbf{global} property depending on how $\Sigma$ sits in $(M, g, k)$.

Any flow that improves $\tr_\Sigma k$ must be non-local in nature.

\textbf{Implication:} Ricci flow and similar PDE-based methods, which are
essentially local, cannot directly resolve the sign issue.
\end{tcolorbox}

\begin{tcolorbox}[colback=red!10, colframe=red!75!black, title=\textbf{Conclusion: Ricci Flow}]
\textbf{Summary:} Ricci flow approaches fail because:

\begin{enumerate}
    \item \textbf{Constraint violation:} Pure Ricci flow on $g$ breaks constraints
    \item \textbf{Coupled flows:} No known coupled flow improves $\tr_\Sigma k$ sign
    \item \textbf{Surgery:} Doesn't address sign issues
    \item \textbf{Locality:} Ricci flow is local; sign problem is global
\end{enumerate}

\textbf{Status:} No Ricci flow method resolves the unconditional case.
\end{tcolorbox}

\end{document}
