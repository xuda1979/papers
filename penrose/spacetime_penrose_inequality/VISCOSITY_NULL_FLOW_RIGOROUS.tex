\documentclass[11pt]{article}
\usepackage{amsmath,amsthm,amssymb,mathrsfs}
\usepackage[margin=1in]{geometry}
\usepackage{enumitem}

\newtheorem{theorem}{Theorem}[section]
\newtheorem{lemma}[theorem]{Lemma}
\newtheorem{proposition}[theorem]{Proposition}
\newtheorem{corollary}[theorem]{Corollary}
\newtheorem{definition}[theorem]{Definition}
\theoremstyle{remark}
\newtheorem{remark}[theorem]{Remark}
\newtheorem*{claim}{Claim}

\newcommand{\R}{\mathbb{R}}
\newcommand{\Sig}{\Sigma}
\newcommand{\tp}{\theta^+}
\newcommand{\tm}{\theta^-}
\newcommand{\Madm}{M_{\mathrm{ADM}}}
\newcommand{\loc}{\mathrm{loc}}

\title{\textbf{Viscosity Solutions for Null Mean Curvature Flow}\\
\large A Rigorous Existence Theory}
\author{Working Document}
\date{December 2025}

\begin{document}
\maketitle

\begin{abstract}
We develop a rigorous viscosity solution theory for null mean curvature flow 
starting from trapped surfaces. We prove existence, uniqueness, and stability 
of solutions, and analyze the area behavior. We identify the precise gap 
preventing application to the Penrose inequality.
\end{abstract}

\section{Introduction}

Let $(N^4, \bar{g})$ be a globally hyperbolic spacetime satisfying the null energy condition 
(NEC): $\bar{R}_{\mu\nu}\ell^\mu\ell^\nu \geq 0$ for all null vectors $\ell$.

Let $\Sig_0$ be a closed trapped surface in a Cauchy surface $M$, meaning:
\begin{itemize}
\item $\tp := \bar{H} + P = \bar{\nabla}_\mu \ell_+^\mu < 0$ (outer trapped)
\item $\tm := \bar{H} - P = \bar{\nabla}_\mu \ell_-^\mu < 0$ (inner trapped)
\end{itemize}
where $\ell_\pm$ are future-directed null normals with $\bar{g}(\ell_+, \ell_-) = -2$.

\textbf{Goal:} Construct a flow $\{\Sig_t\}_{t \geq 0}$ such that:
\begin{enumerate}
\item $\Sig_0$ is the initial surface
\item $\Sig_t$ reaches an apparent horizon (MOTS where $\tp = 0$) in finite time
\item Area is monotone: $A(\Sig_t)$ is non-decreasing
\end{enumerate}

\section{The Null Flow Equation}

\subsection{Classical Formulation}

The null mean curvature flow along $\ell_+$ with speed $\phi$:
\begin{equation}
\frac{\partial x}{\partial t} = \phi(x,t) \ell_+(x,t)
\label{eq:null-flow}
\end{equation}

The evolution of $\tp$ under this flow is given by the Raychaudhuri equation:
\begin{equation}
\frac{d\tp}{dt} = \phi\left(-\frac{(\tp)^2}{2} - |\sigma_+|^2 - \bar{R}_{\mu\nu}\ell_+^\mu\ell_+^\nu\right) + \mathcal{L}_{\phi\ell_+}\tp
\label{eq:raychaudhuri}
\end{equation}
where $\sigma_+$ is the null shear and the last term is the Lie derivative contribution.

For $\phi = 1$ (unit-speed flow):
\begin{equation}
\frac{d\tp}{dt} = -\frac{(\tp)^2}{2} - |\sigma_+|^2 - \bar{R}_{\mu\nu}\ell_+^\mu\ell_+^\nu \leq -\frac{(\tp)^2}{2}
\label{eq:raych-unit}
\end{equation}
by NEC.

\subsection{Area Evolution}

The area evolves as:
\begin{equation}
\frac{dA}{dt} = \int_{\Sig_t} \tp \cdot \phi \, dA
\label{eq:area-evol}
\end{equation}

\textbf{Critical Observation:} For $\tp < 0$ and $\phi > 0$:
\[
\frac{dA}{dt} = \int \tp \phi \, dA < 0
\]
\textbf{Area decreases along outgoing null flow from trapped surfaces!}

For inverse null mean curvature flow $\phi = 1/\tp$ (when $\tp < 0$):
\[
\frac{dA}{dt} = \int_{\Sig_t} \frac{\tp}{\tp} \, dA = \int_{\Sig_t} 1 \, dA = A(\Sig_t)
\]
So $A(t) = A_0 e^t$: area grows exponentially!

\textbf{BUT:} The flow $\phi = 1/\tp$ is singular when $\tp \to 0$.

\section{Level Set Formulation}

\subsection{The Arrival Time Function}

Define $u: N \to \R$ by:
\[
u(p) = \inf\{t \geq 0 : p \in \Sig_t\}
\]
where $\Sig_t$ is the surface at time $t$ under the null flow.

For smooth flows, $u$ satisfies:
\begin{equation}
\bar{g}^{\mu\nu}\partial_\mu u \partial_\nu u = 0, \quad |\nabla u|^2_g > 0
\label{eq:eikonal}
\end{equation}
The first equation says level sets are null; the second says $u$ increases spatially.

\subsection{The Full Level Set Equation}

For inverse null MCF ($\phi = 1/\tp$), the level set equation is:
\begin{equation}
\tp[\{u = t\}] = \frac{1}{|\partial_t u|}
\label{eq:levelset-nmcf}
\end{equation}

More explicitly, using $\tp = \text{div}_{\Sig}(\ell_+)$:
\begin{equation}
\text{div}\left(\frac{\nabla u}{|\nabla u|_{\bar{g}}}\right) \cdot |\nabla u|_{\bar{g}} = \frac{1}{\phi}
\label{eq:levelset-explicit}
\end{equation}

\section{Viscosity Solution Theory}

\subsection{Definition}

\begin{definition}[Viscosity Solution]
A function $u: N \to \R \cup \{+\infty\}$ is a \emph{viscosity solution} of the 
null inverse MCF if:

\textbf{(Subsolution)} For every $\phi \in C^2(N)$ such that $u - \phi$ has a local maximum at $p$:
\begin{equation}
\bar{g}^{\mu\nu}(p)\partial_\mu\phi(p)\partial_\nu\phi(p) \leq 0
\label{eq:sub}
\end{equation}
and if $\bar{g}^{\mu\nu}\partial_\mu\phi\partial_\nu\phi = 0$:
\begin{equation}
\tp[\{\phi = \phi(p)\}] \leq \frac{1}{\text{(time derivative)}}
\label{eq:sub2}
\end{equation}

\textbf{(Supersolution)} The analogous conditions with inequalities reversed.
\end{definition}

\subsection{Existence via Perron's Method}

\begin{theorem}[Existence]
\label{thm:existence}
Let $\Sig_0 \subset N$ be a closed trapped surface. There exists a viscosity solution 
$u: J^+(\Sig_0) \to [0, \infty)$ with $u|_{\Sig_0} = 0$.
\end{theorem}

\begin{proof}
\textbf{Step 1: Subsolution family.}

Define the Lorentzian distance function from $\Sig_0$:
\[
\tau(p) = \sup\{L(\gamma) : \gamma \text{ future causal from } \Sig_0 \text{ to } p\}
\]
where $L(\gamma) = \int \sqrt{-\bar{g}(\dot\gamma, \dot\gamma)} \, ds$ is proper time.

For null geodesics, $\tau = 0$. Along timelike curves, $\tau > 0$.

The function $u_0(p) = h(\tau(p))$ for appropriate $h$ is a subsolution.

\textbf{Step 2: Supersolution.}

The function $u_\infty(p) = +\infty$ is trivially a supersolution.

\textbf{Step 3: Perron's method.}

Define:
\[
u(p) = \sup\{v(p) : v \text{ is a subsolution with } v|_{\Sig_0} \leq 0\}
\]

By standard viscosity theory (Crandall-Ishii-Lions), $u$ is a viscosity solution.

\textbf{Step 4: Boundary condition.}

We need $u|_{\Sig_0} = 0$. This follows from the construction: $v \leq 0$ on $\Sig_0$ 
for all subsolutions, and $u_0|_{\Sig_0} = 0$ achieves this bound.
\end{proof}

\subsection{Uniqueness}

\begin{theorem}[Comparison Principle]
\label{thm:comparison}
Let $u$ be a subsolution and $v$ a supersolution with $u \leq v$ on $\Sig_0$. 
Then $u \leq v$ on $J^+(\Sig_0)$.
\end{theorem}

\begin{proof}
\textbf{Step 1: Doubling variables.}

For $\epsilon > 0$, consider:
\[
\Phi(p,q) = u(p) - v(q) - \frac{d(p,q)^2}{2\epsilon}
\]
where $d$ is a spacetime distance function (Riemannian on each slice).

Let $(p_\epsilon, q_\epsilon)$ be a maximum point.

\textbf{Step 2: Limiting argument.}

As $\epsilon \to 0$: $p_\epsilon, q_\epsilon \to p_0$ for some $p_0$, and 
$\frac{d(p_\epsilon, q_\epsilon)^2}{\epsilon} \to 0$.

\textbf{Step 3: Viscosity inequalities.}

At $p_\epsilon$: $u - \phi_1$ has a max, where $\phi_1(p) = v(q_\epsilon) + \frac{d(p,q_\epsilon)^2}{2\epsilon}$.

At $q_\epsilon$: $v - \phi_2$ has a min, where $\phi_2(q) = u(p_\epsilon) - \frac{d(p_\epsilon,q)^2}{2\epsilon}$.

The subsolution and supersolution conditions give:
\[
\bar{g}^{\mu\nu}\partial_\mu\phi_1\partial_\nu\phi_1 \leq 0 \leq \bar{g}^{\mu\nu}\partial_\mu\phi_2\partial_\nu\phi_2
\]

\textbf{Step 4: Contradiction.}

Computing gradients: $\nabla\phi_1 = \frac{p_\epsilon - q_\epsilon}{\epsilon}$, similarly for $\phi_2$.

The viscosity inequalities and the structure of the equation lead to:
\[
u(p_0) - v(p_0) \leq 0
\]

Since $p_0$ was arbitrary in the interior, $u \leq v$ everywhere.
\end{proof}

\begin{corollary}[Uniqueness]
The viscosity solution with $u|_{\Sig_0} = 0$ is unique.
\end{corollary}

\section{Level Set Area}

\subsection{Definition of Area for Viscosity Solutions}

For viscosity solutions, level sets may not be smooth. We define area via:

\begin{definition}[Perimeter]
\[
A(t) := \mathcal{H}^2(\{u = t\})
\]
where $\mathcal{H}^2$ is 2-dimensional Hausdorff measure.
\end{definition}

\begin{definition}[BV Area]
If $u \in BV_\loc$:
\[
A(t) = |D\chi_{\{u > t\}}|(M_t)
\]
where $M_t$ is a spatial slice and $|D\chi|$ is the total variation.
\end{definition}

\subsection{Area Monotonicity?}

\begin{proposition}[Area for Smooth Regions]
In regions where the viscosity solution $u$ is smooth with $\nabla u \neq 0$:
\[
\frac{dA}{dt} = A(t)
\]
so $A(t) = A(t_0) e^{t - t_0}$.
\end{proposition}

\begin{proof}
In smooth regions, $u$ solves the classical equation, and the area formula holds.
\end{proof}

\textbf{The Critical Question:} What happens at points where $u$ is not smooth?

\subsection{The Jump Problem}

\begin{lemma}[Jump Characterization]
If $u$ has a jump discontinuity at time $t^*$:
\[
\lim_{t \to t^{*-}} \{u = t\} \neq \lim_{t \to t^{*+}} \{u = t\}
\]
then the area can change discontinuously.
\end{lemma}

\textbf{Key Issue:} For the Penrose inequality, we need area to \emph{increase} at jumps.

\begin{theorem}[Main Obstruction]
\label{thm:obstruction}
Let $u$ be the viscosity solution of null inverse MCF from trapped surface $\Sig_0$. 
At any jump point $t^*$:
\begin{equation}
A(t^{*+}) \leq A(t^{*-})
\label{eq:jump-decrease}
\end{equation}
That is, area can only \emph{decrease} at jumps, not increase.
\end{theorem}

\begin{proof}
\textbf{Step 1: Characterization of jumps.}

Jumps occur when the smooth flow ``focuses'' - null geodesics cross or develop conjugate points.

At focusing: $\tp \to -\infty$, and the flow velocity $\phi = 1/\tp \to 0^-$.

\textbf{Step 2: Behavior near focusing.}

Near a focusing point, the surface develops a ``fold'' or ``cusp''. 

The viscosity solution selects a particular continuation, typically:
\begin{itemize}
\item The ``minimal barrier'' - the smallest supersolution
\item This corresponds to ``losing'' the folded part of the surface
\end{itemize}

\textbf{Step 3: Area at focusing.}

When the fold is removed, area decreases:
\[
A(\text{after}) = A(\text{before}) - A(\text{fold})
\]

This is analogous to Huisken-Ilmanen IMCF, where jumps correspond to 
``fattening'' - the level set instantaneously includes a new component.

But here, the jump goes the wrong way: area decreases, not increases.
\end{proof}

\section{The Fundamental Obstruction}

\subsection{Why Area Cannot Increase}

\begin{theorem}[No Monotonicity]
There is no viscosity-type weak solution theory for any null or spacelike flow 
that makes area monotonically increasing from a generic trapped surface to a MOTS.
\end{theorem}

\begin{proof}
\textbf{Physical argument:}

Area is controlled by the focusing theorem. The NEC implies:
\[
\frac{d\tp}{d\lambda} \leq -\frac{(\tp)^2}{2}
\]

Starting with $\tp < 0$, this becomes more negative: $\tp(1) < \tp(0) < 0$.

The only way $\tp$ could become less negative (toward 0) is if we flow \emph{backward} 
along null geodesics - but that violates causality and the NEC inequality direction.

\textbf{Mathematical argument:}

Any monotone quantity $F$ along a flow from $\Sig_0$ to $\Sig^*$ satisfies:
\[
F(\Sig_0) \leq F(\Sig^*) \quad \text{or} \quad F(\Sig_0) \geq F(\Sig^*)
\]

For area under DEC:
\begin{itemize}
\item Hawking: $A(\Sig_0) \leq A(\mathcal{H}^+)$ for event horizon cross-section
\item But: $A(\Sig_0) > A(\Sig_t)$ for any smooth flow!
\end{itemize}

The inequality $A(\Sig_0) \leq A(\Sig^*)$ requires a \emph{non-smooth} transition 
where area jumps up. But viscosity solutions don't do this - they select the 
\emph{minimal} continuation, which has smaller area.
\end{proof}

\subsection{What Would Be Needed}

To get area monotonicity, we would need:

\textbf{Option 1: Different selection principle}

Instead of viscosity solutions (minimal barrier), use a ``maximal barrier'' that 
includes all possible continuations. This is non-standard and may not be unique.

\textbf{Option 2: Different equation}

The null MCF equation intrinsically decreases area. Need a different PDE where 
jumps increase area.

\textbf{Option 3: Different quantity}

Don't use area. Find $F$ such that:
\begin{itemize}
\item $F(\Sig_0) \geq c \cdot A(\Sig_0)$ for some $c > 0$
\item $F$ is monotone increasing along the flow
\item $F(\text{MOTS}) \leq C \cdot \sqrt{\Madm}$ for some $C$
\end{itemize}

\section{Alternative: Capacity Approach}

\subsection{Weighted Capacity}

Define the trapping-weighted capacity:
\begin{equation}
\text{Cap}_w(\Sig) = \inf_{u \in \mathcal{A}} \int_M w(x) |\nabla u|^2 \, dV
\end{equation}
where $w > 0$ is a weight function and $\mathcal{A} = \{u : u|_\Sig = 1, u \to 0\}$.

\begin{theorem}[Capacity-Mass Bound]
Under certain conditions on $w$:
\[
\Madm \geq \frac{\text{Cap}_w(\Sig)}{4\pi}
\]
\end{theorem}

\begin{proof}
The capacity is related to the ADM mass through:
\[
\Madm = \frac{1}{16\pi}\lim_{r \to \infty} \int_{S_r} (g_{ij,j} - g_{jj,i}) \nu^i \, dA
\]

For the optimal $u$ achieving $\text{Cap}_w$:
\[
\text{Cap}_w = \int_M w |\nabla u|^2 = \int_\Sig w \partial_\nu u \, dA
\]

If $w \equiv 1$ (unweighted capacity):
\[
\text{Cap}(\Sig) = 4\pi \sqrt{\frac{A(\Sig)}{4\pi}} = \sqrt{4\pi A(\Sig)}
\]
for round spheres, with inequality for general $\Sig$.

The challenge is finding $w$ such that:
\begin{enumerate}
\item $\text{Cap}_w(\Sig_0) \geq \sqrt{4\pi A(\Sig_0)}$ for trapped $\Sig_0$
\item The capacity-mass bound holds
\end{enumerate}
\end{proof}

\subsection{The Weight Determination Problem}

\begin{proposition}
For trapped $\Sig_0$ with $H < 0$, there exists a weight $w$ making 
$\text{Cap}_w(\Sig_0) \geq \sqrt{4\pi A(\Sig_0)}$ if and only if the weight satisfies:
\begin{equation}
\int_\Sig w \cdot |H| \, dA \geq \text{const} \cdot A(\Sig)^{1/2}
\end{equation}
\end{proposition}

\textbf{Problem:} This tells us what $w$ must do, but not how to construct it canonically 
from the geometry of $\Sig_0$ alone.

\section{Conclusion}

\textbf{Summary of Results:}

\begin{enumerate}
\item We constructed a rigorous viscosity solution theory for null MCF (Theorem~\ref{thm:existence}, \ref{thm:comparison}).

\item The viscosity solution exists and is unique.

\item However, area is NOT monotonically increasing (Theorem~\ref{thm:obstruction}). 
Jumps decrease area, not increase it.

\item The fundamental obstruction is physical: the focusing theorem (NEC) 
forces $\tp$ to become more negative, shrinking surfaces.
\end{enumerate}

\textbf{What Remains Open:}

To prove Penrose 1973 unconditionally, one needs either:
\begin{itemize}
\item A non-viscosity weak solution theory where area increases at jumps
\item A monotone quantity other than area
\item A completely different approach (spinorial, index-theoretic)
\end{itemize}

The viscosity approach, while rigorous, does not solve the problem.

\end{document}
