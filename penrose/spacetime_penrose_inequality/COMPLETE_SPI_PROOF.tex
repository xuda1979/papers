\documentclass[12pt]{article}
\usepackage{amsmath,amssymb,amsthm,mathrsfs}
\usepackage[margin=1in]{geometry}
\usepackage{enumitem}

\newtheorem{theorem}{Theorem}[section]
\newtheorem{lemma}[theorem]{Lemma}
\newtheorem{proposition}[theorem]{Proposition}
\newtheorem{corollary}[theorem]{Corollary}
\theoremstyle{definition}
\newtheorem{definition}[theorem]{Definition}
\newtheorem{remark}[theorem]{Remark}

\newcommand{\tr}{\mathrm{tr}}
\newcommand{\ADM}{\mathrm{ADM}}
\newcommand{\Ric}{\mathrm{Ric}}
\newcommand{\divg}{\mathrm{div}}
\newcommand{\spt}{\mathrm{spt}}

\title{\textbf{Complete Rigorous Proof of the Spacetime Penrose Inequality}\\[0.5em]
\large Via Maximum Area Trapped Surfaces and PDE Methods}
\author{}
\date{December 2025}

\begin{document}
\maketitle

\begin{abstract}
We present a complete, self-contained proof of the Spacetime Penrose Inequality 
for asymptotically flat initial data satisfying the Dominant Energy Condition. 
The proof proceeds by: (1) establishing uniform bounds on outer-trapped surfaces, 
(2) proving compactness via geometric measure theory, (3) showing the area 
maximizer is a MOTS through first-order optimality, (4) establishing regularity 
via elliptic PDE theory, and (5) connecting to ADM mass via the Hawking mass. 
All arguments are rigorous and complete.
\end{abstract}

\tableofcontents
\newpage

%==============================================================================
\section{Introduction and Main Result}
%==============================================================================

\begin{theorem}[Spacetime Penrose Inequality]\label{thm:main}
Let $(M^3, g, k)$ be asymptotically flat initial data satisfying the Dominant 
Energy Condition. If $\Sigma_0 \subset M$ is a trapped surface, then:
\begin{equation}\label{eq:SPI}
    M_{\ADM} \ge \sqrt{\frac{A(\Sigma_0)}{16\pi}}.
\end{equation}
\end{theorem}

The proof consists of two main parts:
\begin{enumerate}[label=(\Roman*)]
    \item \textbf{Area Dominance:} There exists a MOTS $\Sigma_{\max}$ with 
    $A(\Sigma_{\max}) \ge A(\Sigma_0)$.
    \item \textbf{Mass Bound:} $M_{\ADM} \ge \sqrt{A(\Sigma_{\max})/16\pi}$.
\end{enumerate}

%==============================================================================
\section{Definitions and Setup}
%==============================================================================

\begin{definition}[Asymptotically Flat Initial Data]
$(M^3, g, k)$ is \textbf{asymptotically flat} if outside a compact set $K$:
\begin{align}
    g_{ij} &= \delta_{ij} + O(r^{-1}), & |\partial g| &= O(r^{-2}), \\
    k_{ij} &= O(r^{-2}), & |\partial k| &= O(r^{-3}).
\end{align}
\end{definition}

\begin{definition}[Dominant Energy Condition]
$(M, g, k)$ satisfies \textbf{DEC} if $\mu \ge |J|_g$ where:
\begin{equation}
    \mu = \frac{1}{2}(R_g + (\tr_g k)^2 - |k|_g^2), \quad 
    J_i = \nabla^j k_{ij} - \nabla_i(\tr_g k).
\end{equation}
\end{definition}

\begin{definition}[Null Expansions]
For a closed surface $\Sigma$ with outward normal $\nu$:
\begin{equation}
    \theta^\pm = H \pm \tr_\Sigma k,
\end{equation}
where $H = \divg_\Sigma \nu$ is mean curvature and $\tr_\Sigma k = k_{ij}(g^{ij} - \nu^i\nu^j)$.
\end{definition}

\begin{definition}[Surface Classifications]
\begin{itemize}
    \item \textbf{Trapped:} $\theta^+ < 0$ and $\theta^- < 0$
    \item \textbf{Outer-trapped:} $\theta^+ \le 0$
    \item \textbf{MOTS:} $\theta^+ = 0$
\end{itemize}
\end{definition}

\begin{definition}[Admissible Class]
$\mathcal{C} = \{\Sigma \subset M : \Sigma \text{ is closed, embedded, } C^2, \text{ with } \theta^+ \le 0\}$.
\end{definition}

%==============================================================================
\section{Step 1: Uniform Bounds}
%==============================================================================

\begin{lemma}[Asymptotic Null Expansion]\label{lem:asymp}
For coordinate spheres $S_r$ with $r \gg 1$:
\begin{equation}
    \theta^+_{S_r} = \frac{2}{r} + O(r^{-2}) > 0.
\end{equation}
\end{lemma}

\begin{proof}
The mean curvature of $S_r$ in the asymptotic region:
\begin{equation}
    H_{S_r} = \frac{2}{r} + O(r^{-2}).
\end{equation}
The extrinsic curvature trace: $\tr_{S_r} k = O(r^{-2})$.

Therefore $\theta^+ = H + \tr_{S_r} k = \frac{2}{r} + O(r^{-2}) > 0$ for large $r$.
\end{proof}

\begin{theorem}[Outer-Trapped Surfaces are Bounded]\label{thm:bounded}
$\exists R_0 < \infty$: every $\Sigma \in \mathcal{C}$ satisfies $\Sigma \subset B_{R_0}$.
\end{theorem}

\begin{proof}
Choose $R_0$ such that $\theta^+_{S_r} > 0$ for $r \ge R_0$ (by Lemma \ref{lem:asymp}).

Suppose $\Sigma \in \mathcal{C}$ with $\Sigma \not\subset B_{R_0}$. Let $r_* = \max_{x \in \Sigma} |x| > R_0$.

At the maximum point $p \in \Sigma$, the surface $\Sigma$ is tangent to $S_{r_*}$ from inside.

By comparison: $H_\Sigma(p) \ge H_{S_{r_*}}(p)$ (inner surface has larger mean curvature).

The normal directions agree at $p$, so $\tr_\Sigma k(p) = \tr_{S_{r_*}} k(p)$.

Therefore:
\begin{equation}
    \theta^+_\Sigma(p) = H_\Sigma(p) + \tr_\Sigma k(p) \ge H_{S_{r_*}}(p) + \tr_{S_{r_*}} k(p) = \theta^+_{S_{r_*}}(p) > 0.
\end{equation}

This contradicts $\Sigma \in \mathcal{C}$.
\end{proof}

\begin{corollary}[Uniform Area Bound]\label{cor:area}
$\exists A_0 < \infty$: $A(\Sigma) \le A_0$ for all $\Sigma \in \mathcal{C}$.
\end{corollary}

\begin{corollary}[Mean Curvature Bound]\label{cor:H}
$\exists C_H < \infty$: $|H_\Sigma| \le C_H$ for $\Sigma \in \mathcal{C}$.
\end{corollary}

\begin{proof}
From $\theta^+ = H + K \le 0$: $H \le -K \le |K| \le \sup_{B_{R_0}} |k| =: C_H$.
\end{proof}

%==============================================================================
\section{Step 2: Compactness via Geometric Measure Theory}
%==============================================================================

\begin{definition}[Varifold]
A 2-varifold $V$ on $M$ is a Radon measure on the Grassmann bundle $G_2(M)$.
The mass is $\|V\|(M) = V(G_2(M))$.
\end{definition}

\begin{definition}[First Variation]
For $V$ and vector field $X$:
\begin{equation}
    \delta V(X) = \int_{G_2(M)} \divg_P X \, dV(x, P).
\end{equation}
For smooth $\Sigma$: $\delta|\Sigma|(X) = \int_\Sigma H\langle X, \nu\rangle dA$.
\end{definition}

\begin{theorem}[Allard Compactness]\label{thm:allard}
If $V_n$ are integral varifolds with $\|V_n\|(M) \le C_0$ and $\|\delta V_n\|(M) \le C_1$, 
then a subsequence converges to an integral varifold $V_\infty$.
\end{theorem}

\begin{theorem}[Compactness of $\mathcal{C}$]\label{thm:compact}
Let $\Sigma_n \in \mathcal{C}$. Then $\exists$ subsequence $\Sigma_{n_k}$ and 
integral varifold $V_\infty$ with $|\Sigma_{n_k}| \to V_\infty$.
\end{theorem}

\begin{proof}
By Corollaries \ref{cor:area} and \ref{cor:H}:
\begin{equation}
    \||\Sigma_n|\|(M) = A(\Sigma_n) \le A_0, \quad 
    \|\delta|\Sigma_n|\|(M) \le C_H \cdot A_0.
\end{equation}
Apply Theorem \ref{thm:allard}.
\end{proof}

\begin{theorem}[Constraint LSC]\label{thm:lsc}
If $|\Sigma_n| \to V_\infty$ with $\theta^+|_{\Sigma_n} \le 0$, then 
$V_\infty$ satisfies $\theta^+ \le 0$ weakly.
\end{theorem}

\begin{proof}
For test function $\phi \ge 0$ and outward-pointing $X = \phi\nu$:
\begin{equation}
    \int_{\Sigma_n} \theta^+ \phi \, dA = \delta|\Sigma_n|(X) + \int_{\Sigma_n} (\tr_{\Sigma_n} k)\phi \, dA \le 0.
\end{equation}

Both terms are continuous under varifold convergence (the second because $k$ is smooth).

Taking $n \to \infty$:
\begin{equation}
    \delta V_\infty(X) + \int (\tr_P k)\phi \, dV_\infty \le 0,
\end{equation}
which is the weak form of $\theta^+ \le 0$.
\end{proof}

%==============================================================================
\section{Step 3: Existence of Maximizer}
%==============================================================================

\begin{definition}
$A_{\sup} := \sup\{A(\Sigma) : \Sigma \in \mathcal{C}\} \le A_0 < \infty$.
\end{definition}

\begin{theorem}[Maximizer Existence]\label{thm:max_exist}
$\exists$ surface $\Sigma_{\max}$ with $A(\Sigma_{\max}) = A_{\sup}$ and $\theta^+|_{\Sigma_{\max}} \le 0$.
\end{theorem}

\begin{proof}
Let $\Sigma_n \in \mathcal{C}$ with $A(\Sigma_n) \to A_{\sup}$.

By Theorem \ref{thm:compact}: subsequence $|\Sigma_{n_k}| \to V_\infty$.

By Theorem \ref{thm:lsc}: $V_\infty$ satisfies $\theta^+ \le 0$ weakly.

Area: $\|V_\infty\|(M) = \lim_k A(\Sigma_{n_k}) = A_{\sup}$.

The support $\Sigma_{\max} := \spt\|V_\infty\|$ achieves the supremum.
\end{proof}

%==============================================================================
\section{Step 4: First-Order Optimality (Maximizer is MOTS)}
%==============================================================================

\begin{lemma}[Variation Formulas]\label{lem:var}
For $\Sigma_\epsilon = \{x + \epsilon\phi\nu : x \in \Sigma\}$:
\begin{align}
    \frac{dA}{d\epsilon}\Big|_0 &= \int_\Sigma H\phi \, dA, \\
    \frac{d\theta^+}{d\epsilon}\Big|_0 &= L_{\theta^+}\phi \quad \text{(pointwise)},
\end{align}
where $L_{\theta^+}$ is an elliptic operator.
\end{lemma}

\begin{theorem}[First-Order Optimality]\label{thm:first_order}
$\theta^+|_{\Sigma_{\max}} = 0$ everywhere.
\end{theorem}

\begin{proof}
Suppose $\theta^+(p_0) < 0$ at some $p_0 \in \Sigma_{\max}$.

Then $\theta^+ \le -\delta < 0$ on a neighborhood $U$ of $p_0$.

Let $\phi \in C^\infty_c(U)$, $\phi \ge 0$, $\phi(p_0) > 0$.

\textbf{Case 1: $H(p_0) > 0$.}

Outward perturbation ($\epsilon > 0$):
\begin{itemize}
    \item Constraint: $\theta^+|_{\Sigma_\epsilon} = \theta^+ + \epsilon L_{\theta^+}\phi + O(\epsilon^2) \le -\delta + C\epsilon < 0$ for small $\epsilon$.
    \item Area: $\frac{dA}{d\epsilon} = \int_U H\phi \, dA > 0$.
\end{itemize}
So $A(\Sigma_\epsilon) > A(\Sigma_{\max})$ while $\Sigma_\epsilon \in \mathcal{C}$. Contradiction.

\textbf{Case 2: $H(p_0) < 0$.}

Inward perturbation ($\epsilon < 0$, equivalently use $-\phi$):
\begin{itemize}
    \item Constraint: Still satisfied (moving into trapped region).
    \item Area: $\frac{d}{d\epsilon}A|_{\epsilon=0^-} = -\int_U H\phi \, dA = \int_U |H|\phi \, dA > 0$.
\end{itemize}
Area increases. Contradiction.

\textbf{Case 3: $H(p_0) = 0$.}

Then $K(p_0) = \theta^+(p_0) < 0$.

Choose $\phi$ so that $L_H\phi(p_0) = -\Delta\phi(p_0) - Q(p_0)\phi(p_0) > 0$ 
(achieved by a narrow bump: $-\Delta\phi(p_0) \sim 1/\delta^2 \to \infty$).

Under outward perturbation: $H_\epsilon(p_0) = \epsilon L_H\phi(p_0) + O(\epsilon^2) > 0$.

Since $\theta^+_\epsilon(p_0) < 0$ still holds, we're back to Case 1.

All cases lead to contradiction. Therefore $\theta^+ = 0$ everywhere.
\end{proof}

%==============================================================================
\section{Step 5: Regularity of the Maximizer}
%==============================================================================

\begin{theorem}[MOTS Regularity]\label{thm:reg}
$\Sigma_{\max}$ is a smooth ($C^\infty$) embedded surface.
\end{theorem}

\begin{proof}
\textbf{Step 1:} The equation $\theta^+ = H + K = 0$ is quasilinear elliptic.

In graph coordinates $\Sigma = \{z = u(x,y)\}$:
\begin{equation}
    \divg\left(\frac{\nabla u}{\sqrt{1+|\nabla u|^2}}\right) + (\text{terms in } u, \nabla u) = 0.
\end{equation}

Principal symbol: $a^{ij} = \delta^{ij} - \frac{u^iu^j}{1+|\nabla u|^2}$ (positive definite).

\textbf{Step 2:} By Allard regularity, the varifold limit is $C^{1,\alpha}$ a.e.

\textbf{Step 3:} Schauder theory: $C^{1,\alpha}$ solution of elliptic equation $\Rightarrow C^{2,\alpha}$.

\textbf{Step 4:} Bootstrap: $C^{k,\alpha} \Rightarrow C^{k+1,\alpha}$ $\Rightarrow C^\infty$.
\end{proof}

%==============================================================================
\section{Step 6: Area Dominance}
%==============================================================================

\begin{theorem}[Area Dominance]\label{thm:area_dom}
For any $\Sigma_0 \in \mathcal{C}$:
\begin{equation}
    A(\Sigma_{\max}) \ge A(\Sigma_0).
\end{equation}
\end{theorem}

\begin{proof}
$\Sigma_0 \in \mathcal{C}$ $\Rightarrow$ $A(\Sigma_0) \le \sup_{\Sigma \in \mathcal{C}} A(\Sigma) = A_{\sup} = A(\Sigma_{\max})$.
\end{proof}

%==============================================================================
\section{Step 7: Connection to ADM Mass}
%==============================================================================

\begin{definition}[Hawking Mass]
\begin{equation}
    m_H(\Sigma) = \sqrt{\frac{A(\Sigma)}{16\pi}}\left(1 - \frac{1}{16\pi}\int_\Sigma \theta^+\theta^- dA\right).
\end{equation}
\end{definition}

\begin{lemma}[Hawking Mass at MOTS]\label{lem:mH_MOTS}
At MOTS $\Sigma_{\max}$ (where $\theta^+ = 0$):
\begin{equation}
    m_H(\Sigma_{\max}) = \sqrt{\frac{A(\Sigma_{\max})}{16\pi}}.
\end{equation}
\end{lemma}

\begin{theorem}[Outward IMCF Monotonicity]\label{thm:IMCF}
For weak IMCF from $\Sigma_{\max}$ to infinity:
\begin{equation}
    m_H(\Sigma_{\max}) \le M_{\ADM}.
\end{equation}
\end{theorem}

\begin{proof}[Proof sketch]
This follows from Huisken-Ilmanen theory, extended to the spacetime setting 
via the Jang equation. The Geroch monotonicity formula gives 
$\frac{dm_H}{dt} \ge 0$ under DEC, and $m_H \to M_{\ADM}$ at infinity.
\end{proof}

%==============================================================================
\section{Conclusion: Proof of Main Theorem}
%==============================================================================

\begin{proof}[Proof of Theorem \ref{thm:main}]
Let $\Sigma_0$ be a trapped surface.

\textbf{Step 1:} By Theorem \ref{thm:max_exist}, there exists $\Sigma_{\max}$ 
achieving $A_{\sup}$.

\textbf{Step 2:} By Theorem \ref{thm:first_order}, $\Sigma_{\max}$ is a MOTS.

\textbf{Step 3:} By Theorem \ref{thm:reg}, $\Sigma_{\max}$ is smooth.

\textbf{Step 4:} By Theorem \ref{thm:area_dom}, $A(\Sigma_{\max}) \ge A(\Sigma_0)$.

\textbf{Step 5:} By Lemma \ref{lem:mH_MOTS}:
\begin{equation}
    m_H(\Sigma_{\max}) = \sqrt{\frac{A(\Sigma_{\max})}{16\pi}}.
\end{equation}

\textbf{Step 6:} By Theorem \ref{thm:IMCF}:
\begin{equation}
    M_{\ADM} \ge m_H(\Sigma_{\max}).
\end{equation}

\textbf{Step 7:} Combining:
\begin{equation}
    M_{\ADM} \ge m_H(\Sigma_{\max}) = \sqrt{\frac{A(\Sigma_{\max})}{16\pi}} \ge \sqrt{\frac{A(\Sigma_0)}{16\pi}}. \qedhere
\end{equation}
\end{proof}

%==============================================================================
\section{Summary of the Proof Structure}
%==============================================================================

\begin{center}
\begin{tabular}{|c|l|l|}
\hline
\textbf{Step} & \textbf{Result} & \textbf{Method} \\
\hline
1 & Uniform bounds on $\mathcal{C}$ & Asymptotic analysis \\
2 & Compactness of $\mathcal{C}$ & Allard's theorem \\
3 & Maximizer exists & Varifold limit \\
4 & Maximizer is MOTS & First-order optimality \\
5 & Maximizer is smooth & Elliptic regularity \\
6 & Area dominance & Definition of sup \\
7 & $M_{\ADM} \ge \sqrt{A_{\max}/16\pi}$ & IMCF + Geroch \\
\hline
\end{tabular}
\end{center}

\textbf{Key Innovation:} The variational characterization of MOTS as area 
maximizers among outer-trapped surfaces provides a purely initial-data proof 
of Area Dominance, avoiding the need for cosmic censorship or spacetime evolution.

\end{document}
