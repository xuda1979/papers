%% SPACETIME_APPROACH.tex
%%
%% THE FULL SPACETIME APPROACH TO PENROSE 1973
%%
%% Instead of working only with initial data, we consider the full
%% spacetime development and use causal structure directly.
%%
%% Key insight: The trapped surface condition is inherently spacetime,
%% so we should work in spacetime rather than reducing to initial data.
%%
%% December 2025

\documentclass[11pt]{amsart}
\usepackage{amsmath,amssymb,amsthm}
\usepackage{tcolorbox}

\tcbuselibrary{theorems}

\newtcolorbox{maintheorem}{
    colback=green!5!white,
    colframe=green!50!black,
    title={\textbf{MAIN THEOREM}}
}

\newtcolorbox{keylemma}{
    colback=blue!5!white,
    colframe=blue!75!black,
    title={\textbf{KEY LEMMA}}
}

\newtcolorbox{proofstep}{
    colback=gray!5!white,
    colframe=gray!50!black,
    title={\textbf{PROOF STEP}}
}

\newtcolorbox{insight}{
    colback=purple!5!white,
    colframe=purple!75!black,
    title={\textbf{INSIGHT}}
}

\newtheorem{theorem}{Theorem}[section]
\newtheorem{lemma}[theorem]{Lemma}
\newtheorem{proposition}[theorem]{Proposition}
\newtheorem{corollary}[theorem]{Corollary}
\theoremstyle{definition}
\newtheorem{definition}[theorem]{Definition}
\newtheorem{remark}[theorem]{Remark}

\newcommand{\Area}{\mathrm{Area}}
\newcommand{\Vol}{\mathrm{Vol}}
\newcommand{\divv}{\mathrm{div}}
\DeclareMathOperator{\tr}{tr}

\title{The Spacetime Approach to Penrose 1973:\\
Using Causal Structure Directly}
\author{December 2025}

\begin{document}
\maketitle

\begin{abstract}
We develop an approach to Penrose 1973 that works directly in spacetime, 
using the causal structure rather than reducing to initial data arguments. 
The key insight is that trapped surfaces are defined by null expansion, 
which is a spacetime concept, and comparing to Schwarzschild should be done 
at the spacetime level.
\end{abstract}

%% ============================================================================
\section{Why Spacetime?}
%% ============================================================================

\begin{insight}
\textbf{The Initial Data Approach Has Fundamental Limitations}

Working with initial data $(M, g, k)$:
\begin{itemize}
    \item Trapped surface = both null expansions negative
    \item But $\theta^\pm$ depend on the embedding in spacetime
    \item Area Dominance fails because $k$ is not constrained by DEC alone
\end{itemize}

Working in full spacetime $(\mathcal{M}, g_{\mu\nu})$:
\begin{itemize}
    \item The DEC constrains $T_{\mu\nu}$ directly
    \item Null geodesics and their focusing are spacetime concepts
    \item Penrose's original argument used spacetime causality
\end{itemize}
\end{insight}

%% ============================================================================
\section{Spacetime Setup}
%% ============================================================================

\begin{definition}[Asymptotically Flat Spacetime]
$(\mathcal{M}^4, g)$ is asymptotically flat if it admits a conformal 
compactification with null infinity $\mathscr{I}^+$.

The Bondi mass at a cut $C$ of $\mathscr{I}^+$:
\begin{equation}
    M_B(C) = \lim_{r \to \infty} \frac{r}{2} \int_C (1 - |\nabla r|^2) \, d\Omega
\end{equation}

satisfies $M_B \le M_{\text{ADM}}$ (Bondi mass loss formula).
\end{definition}

\begin{definition}[Trapped Surface in Spacetime]
A 2-surface $\Sigma \subset \mathcal{M}$ is trapped if both null expansions 
are negative:
\begin{equation}
    \theta^+ < 0, \quad \theta^- < 0
\end{equation}

where $\theta^\pm = \nabla_\mu \ell^\pm_\mu$ for null normals $\ell^\pm$.
\end{definition}

%% ============================================================================
\section{The Spacetime Penrose Inequality}
%% ============================================================================

\begin{maintheorem}
\textbf{Spacetime Penrose Inequality}

Let $(\mathcal{M}, g)$ be an asymptotically flat spacetime satisfying:
\begin{itemize}
    \item Einstein equations with DEC: $G_{\mu\nu} = 8\pi T_{\mu\nu}$, $T_{\mu\nu}V^\mu W^\nu \ge 0$ 
          for future causal $V, W$
    \item Cosmic censorship (no naked singularities)
\end{itemize}

If $\Sigma$ is a trapped surface, then:
\begin{equation}
    M_B \ge \sqrt{\frac{\Area(\Sigma)}{16\pi}}
\end{equation}

where $M_B$ is the Bondi mass at any cut to the future of $\Sigma$.
\end{maintheorem}

\begin{insight}
\textbf{Bondi vs ADM Mass}

For spacetimes with gravitational radiation:
\begin{equation}
    M_{\text{ADM}} \ge M_B(C_1) \ge M_B(C_2) \ge \ldots \ge M_{\text{final}}
\end{equation}

Mass decreases due to radiation.

\textbf{Penrose's original conjecture:} The final state should be Kerr 
(or Schwarzschild if non-rotating), and the area theorem gives:
\begin{equation}
    \Area(\text{final horizon}) \ge \Area(\Sigma)
\end{equation}

Combined with Kerr: $M_{\text{final}} = $ (function of area).
\end{insight}

%% ============================================================================
\section{The Area Theorem}
%% ============================================================================

\begin{theorem}[Hawking Area Theorem]
In a spacetime satisfying DEC and cosmic censorship, the area of the 
event horizon is non-decreasing:
\begin{equation}
    \frac{d\Area(\mathcal{H})}{dv} \ge 0
\end{equation}

where $v$ is an advanced time parameter on the horizon.
\end{theorem}

\begin{proof}[Proof Sketch]
The event horizon is generated by null geodesics with $\theta \ge 0$.

By Raychaudhuri:
\begin{equation}
    \frac{d\theta}{dv} = -\frac{\theta^2}{2} - \sigma^2 - 8\pi T_{\mu\nu}\ell^\mu\ell^\nu \le 0
\end{equation}

If $\theta$ ever becomes negative, it reaches $-\infty$ in finite time 
(focusing theorem), creating a caustic.

But horizon generators can't have caustics to the future (they'd leave 
the horizon).

Therefore $\theta \ge 0$ always, giving $\frac{d\Area}{dv} \ge 0$.
\end{proof}

%% ============================================================================
\section{From Trapped Surface to Event Horizon}
%% ============================================================================

\begin{keylemma}
\textbf{Trapped Surfaces are Inside the Event Horizon}

If $\Sigma$ is a trapped surface in $(\mathcal{M}, g)$ satisfying DEC and 
cosmic censorship, then $\Sigma$ is inside the event horizon $\mathcal{H}$.
\end{keylemma}

\begin{proof}
By contradiction. Suppose $\Sigma$ has points outside $\mathcal{H}$.

The outgoing null normal $\ell^+$ from $\Sigma$ has $\theta^+ < 0$.

By Raychaudhuri with DEC, $\theta^+$ decreases (becomes more negative).

The null geodesics from $\Sigma$ focus to a singularity in finite affine 
parameter.

But if $\Sigma$ is outside $\mathcal{H}$, these geodesics should reach 
$\mathscr{I}^+$, contradiction.
\end{proof}

\begin{corollary}
$\Area(\text{event horizon cross-section}) \ge \Area(\Sigma)$ for some 
cross-section.
\end{corollary}

\begin{insight}
\textbf{The Connection to Area Dominance}

In initial data terms:
\begin{itemize}
    \item Trapped surface $\Sigma$ is inside the black hole region
    \item The event horizon intersected with the initial slice gives the 
          "apparent horizon" or MOTS $\Sigma^*$
    \item $\Sigma$ is inside $\Sigma^*$ spatially
\end{itemize}

But: Being inside spatially does NOT imply smaller area!

This is why Area Dominance fails in general.
\end{insight}

%% ============================================================================
\section{The Final State and Mass}
%% ============================================================================

\begin{proofstep}
\textbf{Penrose's Vision}

Assuming the spacetime settles to a final stationary state:

\textbf{Step 1:} By uniqueness theorems, the final state is Kerr 
(Schwarzschild if $J = 0$).

\textbf{Step 2:} For Kerr with mass $M$ and angular momentum $J$:
\begin{equation}
    \Area = 8\pi\left(M^2 + \sqrt{M^4 - J^2}\right) \ge 16\pi M^2
\end{equation}
with equality for Schwarzschild ($J = 0$).

\textbf{Step 3:} By area theorem:
\begin{equation}
    \Area_{\text{final}} \ge \Area(\Sigma)
\end{equation}

\textbf{Step 4:} By Bondi mass loss:
\begin{equation}
    M_{\text{ADM}} \ge M_{\text{final}}
\end{equation}

\textbf{Step 5:} Combining:
\begin{equation}
    M_{\text{ADM}} \ge M_{\text{final}} \ge \sqrt{\frac{\Area_{\text{final}}}{16\pi}}
    \ge \sqrt{\frac{\Area(\Sigma)}{16\pi}}
\end{equation}
\end{proofstep}

%% ============================================================================
\section{The Gaps in Penrose's Argument}
%% ============================================================================

\begin{enumerate}
    \item \textbf{Final state assumption:} Need the spacetime to settle to 
          a stationary black hole. This is cosmic censorship + dynamical 
          stability.
    
    \item \textbf{Uniqueness:} Need Kerr to be the unique stationary vacuum 
          black hole. Proven under certain conditions.
    
    \item \textbf{Angular momentum:} The Kerr inequality 
          $\Area \ge 16\pi M^2$ requires $J = 0$ for equality. For rotating 
          black holes, the bound is weaker.
    
    \item \textbf{Area theorem:} Requires DEC and no singularities on the 
          horizon. Cosmic censorship is needed.
\end{enumerate}

\begin{insight}
\textbf{Penrose's Original Goal}

Penrose 1973 was meant to be a \textbf{test of cosmic censorship}.

If we could prove the inequality from initial data alone, violations 
would signal naked singularity formation.

The difficulty in proving it from initial data reflects the depth of 
the cosmic censorship conjecture.
\end{insight}

%% ============================================================================
\section{A Direct Spacetime Proof?}
%% ============================================================================

\begin{proofstep}
\textbf{Strategy: Control the Evolution}

Instead of assuming the final state, control the evolution directly.

\textbf{Idea:} Use the focusing of null geodesics to track how the 
trapped surface "grows" to meet the event horizon.
\end{proofstep}

\begin{definition}[Null Hypersurface from Trapped Surface]
Let $\Sigma$ be trapped. The outgoing null hypersurface $\mathcal{N}^+$ 
is generated by outgoing null geodesics from $\Sigma$.

By Raychaudhuri with DEC:
\begin{equation}
    \frac{d\theta^+}{dv} \le -\frac{(\theta^+)^2}{2} \le 0
\end{equation}

So $\theta^+$ stays negative (becomes more negative).
\end{definition}

\begin{proposition}[Area Decrease Along Null]
The cross-sectional area of $\mathcal{N}^+$ decreases:
\begin{equation}
    \frac{d\Area}{dv} = \int_{\Sigma_v} \theta^+ \, dA < 0
\end{equation}

The area DECREASES as we go outward along the null.
\end{proposition}

\begin{insight}
\textbf{This is the Wrong Direction!}

We want to show the trapped surface area is LESS than some reference area.

But the null hypersurface has DECREASING area as we evolve.

The event horizon has area INCREASING by the area theorem.

These are opposite behaviors!
\end{insight}

%% ============================================================================
\section{Resolution: Use the Event Horizon Directly}
%% ============================================================================

\begin{proofstep}
\textbf{The Correct Comparison}

Compare not $\Sigma$ to the evolving null surface from $\Sigma$, but 
$\Sigma$ to the event horizon directly.

\textbf{Key:} $\Sigma$ is inside the event horizon.

The event horizon at the initial time has some area $A_{\mathcal{H}}$.

\textbf{Claim:} $\Area(\Sigma) \le A_{\mathcal{H}}$?

This is exactly Area Dominance again!
\end{proofstep}

\begin{insight}
\textbf{The Fundamental Issue}

No matter how we approach it:
\begin{itemize}
    \item Initial data approach: Need $\Area(\Sigma) \le \Area(\Sigma^*)$
    \item Spacetime approach: Need $\Area(\Sigma) \le \Area(\mathcal{H} \cap \text{initial slice})$
\end{itemize}

Both are forms of Area Dominance!

\textbf{The spacetime approach doesn't avoid the issue - it just reformulates it.}
\end{insight}

%% ============================================================================
\section{The Way Forward: Global Methods}
%% ============================================================================

\begin{proofstep}
\textbf{Avoid Local Comparisons}

Instead of comparing $\Area(\Sigma)$ to $\Area(\Sigma^*)$ locally, use 
global methods:

\textbf{Option A:} Prove Schwarzschild is the unique minimizer of mass 
(variational approach).

\textbf{Option B:} Prove a global monotone quantity exists that connects 
$\Sigma$ to infinity (flow approach).

\textbf{Option C:} Prove the spacetime must be close to Schwarzschild 
if the Penrose bound is nearly saturated (rigidity approach).
\end{proofstep}

\begin{maintheorem}
\textbf{Rigidity Approach}

\textbf{Conjecture:} If $(M, g, k)$ satisfies DEC with trapped surface 
$\Sigma$ and:
\begin{equation}
    M_{\text{ADM}} = \sqrt{\frac{\Area(\Sigma)}{16\pi}}
\end{equation}

then $(M, g, k)$ is a slice of Schwarzschild.

\textbf{Strategy:} Prove the rigidity first, then use it to bootstrap 
the inequality.
\end{maintheorem}

\begin{proof}[Proof Idea for Rigidity]
If equality holds:
\begin{enumerate}
    \item The spacetime development must be vacuum (any matter would add mass)
    \item The trapped surface must be the horizon (any smaller and mass is 
          insufficient)
    \item Spherical symmetry follows from uniqueness of Schwarzschild among 
          vacuum spacetimes with given horizon area
\end{enumerate}
\end{proof}

%% ============================================================================
\section{Conclusion}
%% ============================================================================

The spacetime approach reveals:

\begin{enumerate}
    \item Penrose's original vision uses cosmic censorship and final state 
          assumptions
    \item Avoiding these assumptions leads back to Area Dominance
    \item The fundamental issue is comparing a trapped surface to a reference 
          (MOTS or event horizon)
\end{enumerate}

The most promising paths forward:
\begin{itemize}
    \item \textbf{Variational:} Schwarzschild minimizes mass globally
    \item \textbf{Rigidity:} Prove equality case first, then extend
    \item \textbf{Entropy:} Find a monotone functional that doesn't require 
          local area comparison
\end{itemize}

All these bypass the local Area Dominance problem by working globally.

\end{document}
