%%%%%%%%%%%%%%%%%%%%%%%%%%%%%%%%%%%%%%%%%%%%%%%%%%%%%%%%%%%%%%%%%%%%%%%%%%%%%%%
%                                                                              
%              PUBLISHABLE CONTRIBUTIONS: HONEST INVENTORY                     
%                                                                              
%         What Novel Results Have We Actually Achieved?                        
%                                                                              
%                          December 2025                                       
%                                                                              
%%%%%%%%%%%%%%%%%%%%%%%%%%%%%%%%%%%%%%%%%%%%%%%%%%%%%%%%%%%%%%%%%%%%%%%%%%%%%%%

\documentclass[11pt]{amsart}
\usepackage{amsmath,amssymb,amsthm}
\usepackage[dvipsnames]{xcolor}
\usepackage{tcolorbox}

\theoremstyle{plain}
\newtheorem{theorem}{Theorem}[section]
\newtheorem{proposition}[theorem]{Proposition}

\theoremstyle{definition}
\newtheorem*{contribution}{Contribution}
\newtheorem*{verdict}{Verdict}

\title{Publishable Contributions: An Honest Inventory}
\author{Self-Assessment}
\date{December 2025}

\begin{document}
\maketitle

\section{Executive Summary}

\begin{tcolorbox}[colback=yellow!5,colframe=yellow!75!black,title=Honest Assessment]
After extensive exploration (~200 documents), we have \textbf{not} proven 
Penrose 1973. However, we have produced some results that may be 
publishable as \textbf{partial progress} or \textbf{negative results}.

The question is: what would a referee accept as a genuine contribution?
\end{tcolorbox}

%%%%%%%%%%%%%%%%%%%%%%%%%%%%%%%%%%%%%%%%%%%%%%%%%%%%%%%%%%%%%%%%%%%%%%%%%%%%%%%
\section{Potentially Publishable Results}
%%%%%%%%%%%%%%%%%%%%%%%%%%%%%%%%%%%%%%%%%%%%%%%%%%%%%%%%%%%%%%%%%%%%%%%%%%%%%%%

\subsection{Contribution 1: Obstruction to Area Dominance}

\begin{contribution}[Area Dominance Obstruction]
We have proven that the "Area Dominance" approach to the spacetime 
Penrose inequality is fundamentally blocked by the sign of $P = \tr_\Sigma k$.
\end{contribution}

\textbf{The Result:}

The Area Dominance strategy attempts to show:
\[
\text{Area}(\Sigma_{\text{trapped}}) \leq \text{Area}(\Sigma_{\text{minimal}})
\]
where $\Sigma_{\text{minimal}}$ is the outermost minimal surface.

We proved this approach fails because:
\begin{itemize}
\item The null expansion $\theta^+ = H + P$ can have $P > 0$
\item This means trapped surfaces can have $H > 0$ (mean-convex outward)
\item Such surfaces can be LARGER than any enclosing minimal surface
\item Explicit counterexamples exist in Schwarzschild with non-trivial $k$
\end{itemize}

\begin{verdict}
\textbf{Possibly publishable} as a "negative result" paper explaining why 
this natural approach fails. However:
\begin{itemize}
\item[-] May be considered "folklore" by experts
\item[-] The obstruction might be implicitly known
\item[+] No explicit writeup exists in the literature
\item[+] Could save others from pursuing this dead end
\end{itemize}
\textbf{Estimated novelty: Low to Medium}
\end{verdict}

\subsection{Contribution 2: Variational Framework}

\begin{contribution}[Variational Reformulation]
We developed a complete variational framework for the spacetime Penrose 
inequality via ADM mass minimization.
\end{contribution}

\textbf{The Result:}

The inequality $M_{\ADM} \geq \sqrt{A/16\pi}$ is equivalent to:
\[
\mathcal{P}_A := \inf\{M_{\ADM}[g,k] : (g,k) \in \mathcal{C}_A\} = \sqrt{\frac{A}{16\pi}}
\]

We established:
\begin{itemize}
\item The problem is well-posed
\item Compactness holds for near-minimizing sequences (with some gaps)
\item Connection to Bunting-Masood-ul-Alam uniqueness
\end{itemize}

\begin{verdict}
\textbf{Not publishable as stated.} The variational approach to Penrose 
is well-known. Our specific formulation doesn't add enough novelty.
\begin{itemize}
\item[-] Variational approaches are standard in geometric analysis
\item[-] The key gap (trapped surface preservation) is not resolved
\item[-] Similar frameworks exist in the literature
\end{itemize}
\textbf{Estimated novelty: Low}
\end{verdict}

\subsection{Contribution 3: Time-Symmetry Obstruction}

\begin{contribution}[Time-Symmetry Gap]
We identified the precise obstruction to reducing the spacetime problem 
to the time-symmetric (Riemannian) case.
\end{contribution}

\textbf{The Result:}

The reduction $k \to 0$ at minimizers fails because:
\begin{itemize}
\item Trapped surfaces with $P < 0$ may have $H > 0$
\item Setting $k = 0$ destroys the trapped property
\item No mechanism guarantees existence of a trapped surface with $H \leq 0$
\end{itemize}

\begin{verdict}
\textbf{Not independently publishable.} This is essentially the same 
obstruction as Contribution 1, rephrased.
\textbf{Estimated novelty: Low}
\end{verdict}

\subsection{Contribution 4: MOTS Stability Analysis}

\begin{contribution}[Near-Horizon Compactness]
We developed curvature estimates near MOTS using stability conditions.
\end{contribution}

\textbf{The Result:}

For stable MOTS $\Sigma$ with bounded area and ambient mass:
\[
\|A\|_{L^\infty(\Sigma)} \leq C(A, M_{\ADM})
\]
with bootstrapping to full curvature bounds in tubular neighborhoods.

\begin{verdict}
\textbf{Possibly publishable} as part of a general MOTS compactness paper.
\begin{itemize}
\item[+] MOTS stability and compactness is an active area
\item[+] Estimates are technically solid
\item[-] Similar results may exist in Andersson-Metzger, etc.
\item[-] Needs comparison with existing literature
\end{itemize}
\textbf{Estimated novelty: Medium (needs literature check)}
\end{verdict}

\subsection{Contribution 5: Survey of Approaches}

\begin{contribution}[Comprehensive Survey]
We systematically analyzed and ruled out multiple proof strategies.
\end{contribution}

\textbf{Approaches analyzed:}
\begin{enumerate}
\item Area Dominance — blocked by $P$ sign
\item Spacetime IMCF (IOEF) — blocked by monotonicity signs  
\item Perelman-style entropy — indefinite constraint contributions
\item Optimal transport — constraint preservation issues
\item Variational + BMA — trapped surface preservation gap
\item Symmetrization — doesn't preserve constraints
\item Jang equation — blowup issues not resolved
\end{enumerate}

\begin{verdict}
\textbf{Possibly publishable} as a survey/status paper on the spacetime 
Penrose inequality.
\begin{itemize}
\item[+] Comprehensive analysis of the state of the art
\item[+] Identifies specific obstructions for each approach
\item[+] Could be useful to the community
\item[-] Not a theorem, just exposition
\item[-] Some claims need more rigorous verification
\end{itemize}
\textbf{Estimated novelty: Medium (as exposition)}
\end{verdict}

%%%%%%%%%%%%%%%%%%%%%%%%%%%%%%%%%%%%%%%%%%%%%%%%%%%%%%%%%%%%%%%%%%%%%%%%%%%%%%%
\section{What We Do NOT Have}
%%%%%%%%%%%%%%%%%%%%%%%%%%%%%%%%%%%%%%%%%%%%%%%%%%%%%%%%%%%%%%%%%%%%%%%%%%%%%%%

To be brutally honest, we lack:

\begin{enumerate}
\item \textbf{A new theorem.} We have not proven any new inequality or 
geometric result that wasn't already known.

\item \textbf{A new technique.} Our methods (variational, compactness, 
maximum principles) are all standard. No genuinely new tool was invented.

\item \textbf{A new partial result.} We haven't proven Penrose under any 
new special case (e.g., specific symmetry, curvature bounds, etc.).

\item \textbf{A counterexample.} We haven't constructed initial data that 
violates Penrose (which would be major news).

\item \textbf{A conditional result.} We don't have "Penrose holds IF X" 
for any new condition X.
\end{enumerate}

%%%%%%%%%%%%%%%%%%%%%%%%%%%%%%%%%%%%%%%%%%%%%%%%%%%%%%%%%%%%%%%%%%%%%%%%%%%%%%%
\section{Realistic Publication Options}
%%%%%%%%%%%%%%%%%%%%%%%%%%%%%%%%%%%%%%%%%%%%%%%%%%%%%%%%%%%%%%%%%%%%%%%%%%%%%%%

\subsection{Option A: Survey Paper}

\textbf{Title:} "The Spacetime Penrose Inequality: A Survey of Approaches 
and Obstructions"

\textbf{Content:}
\begin{itemize}
\item History of the conjecture
\item The Riemannian case (Huisken-Ilmanen, Bray)
\item Why spacetime is harder
\item Analysis of failed approaches with specific obstructions
\item Open problems
\end{itemize}

\textbf{Venue:} Bulletin of the AMS, Notices of the AMS, or a survey journal

\textbf{Assessment:} This is publishable if well-written, but requires:
\begin{itemize}
\item Comprehensive literature review
\item Verification that our "obstructions" are actually new observations
\item Proper attribution to existing work
\end{itemize}

\subsection{Option B: Negative Results Paper}

\textbf{Title:} "On the Failure of Area Dominance for the Spacetime 
Penrose Inequality"

\textbf{Content:}
\begin{itemize}
\item The Area Dominance strategy
\item Explicit counterexamples showing it fails
\item Discussion of the role of $\tr_\Sigma k$
\end{itemize}

\textbf{Venue:} Classical and Quantum Gravity, or similar

\textbf{Assessment:} Marginally publishable. Needs:
\begin{itemize}
\item Explicit numerical/analytical counterexamples
\item Check that this isn't already in the literature
\end{itemize}

\subsection{Option C: Wait for a Breakthrough}

Continue working on the problem. If we find:
\begin{itemize}
\item A proof under new conditions
\item A genuinely new technique
\item A way to close the trapped surface gap
\end{itemize}
then we have a strong paper.

%%%%%%%%%%%%%%%%%%%%%%%%%%%%%%%%%%%%%%%%%%%%%%%%%%%%%%%%%%%%%%%%%%%%%%%%%%%%%%%
\section{Final Assessment}
%%%%%%%%%%%%%%%%%%%%%%%%%%%%%%%%%%%%%%%%%%%%%%%%%%%%%%%%%%%%%%%%%%%%%%%%%%%%%%%

\begin{tcolorbox}[colback=red!5,colframe=red!75!black,title=Honest Verdict]
\textbf{Do we have innovations to publish?}

\textbf{Weak yes.} We have:
\begin{itemize}
\item A comprehensive analysis of approaches (survey potential)
\item Identification of specific obstructions (marginally novel)
\item Some technical estimates (need literature comparison)
\end{itemize}

\textbf{But we lack:}
\begin{itemize}
\item A theorem
\item A new technique
\item A partial result
\end{itemize}

\textbf{Recommendation:} 

A survey paper might be publishable in an expository venue. A research 
paper would require either:
\begin{enumerate}
\item Finding a genuine new result (even a small one)
\item OR thoroughly documenting that our obstructions are new to the literature
\end{enumerate}

As of now, we have extensive \textit{exploration} but not a \textit{publication-ready contribution}.
\end{tcolorbox}

%%%%%%%%%%%%%%%%%%%%%%%%%%%%%%%%%%%%%%%%%%%%%%%%%%%%%%%%%%%%%%%%%%%%%%%%%%%%%%%
\section{What Would Make This Publishable}
%%%%%%%%%%%%%%%%%%%%%%%%%%%%%%%%%%%%%%%%%%%%%%%%%%%%%%%%%%%%%%%%%%%%%%%%%%%%%%%

To turn this into a genuine research paper, we would need ONE of:

\begin{enumerate}
\item \textbf{Prove Penrose under a new assumption:}
   \begin{itemize}
   \item Spherical symmetry with general $k$ (might be tractable)
   \item Small $|k|$ perturbation of time-symmetric case
   \item Specific matter content
   \end{itemize}

\item \textbf{Prove a related inequality:}
   \begin{itemize}
   \item Bound involving both area AND $\|k\|$
   \item Inequality for specific quasi-local mass
   \end{itemize}

\item \textbf{Construct an explicit counterexample:}
   \begin{itemize}
   \item Initial data violating Penrose but satisfying WCC
   \item This would be huge but seems unlikely given physical intuition
   \end{itemize}

\item \textbf{Develop a genuinely new technique:}
   \begin{itemize}
   \item New geometric flow adapted to the problem
   \item New quasi-local mass with better properties
   \end{itemize}
\end{enumerate}

\end{document}
