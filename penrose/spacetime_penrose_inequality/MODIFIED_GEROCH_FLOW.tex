%% MODIFIED_GEROCH_FLOW.tex
%%
%% NEW ATTACK: Modified Geroch Flow for Trapped Surfaces
%%
%% Key Idea: Construct a flow that starts from trapped surface (H < 0)
%% and still has monotone Hawking mass
%%
%% December 2025

\documentclass[11pt]{amsart}
\usepackage{amsmath,amssymb,amsthm}
\usepackage{xcolor}
\usepackage{tcolorbox}

\tcbuselibrary{theorems}

\newtcolorbox{keyidea}{
    colback=green!5!white,
    colframe=green!75!black,
    title={\textbf{KEY IDEA}}
}

\newtcolorbox{calculation}{
    colback=yellow!5!white,
    colframe=yellow!75!black,
    title={\textbf{CALCULATION}}
}

\newtcolorbox{breakthrough}{
    colback=purple!5!white,
    colframe=purple!75!black,
    title={\textbf{POTENTIAL BREAKTHROUGH}}
}

\newtheorem{theorem}{Theorem}[section]
\newtheorem{lemma}[theorem]{Lemma}
\newtheorem{proposition}[theorem]{Proposition}
\newtheorem{corollary}[theorem]{Corollary}
\newtheorem{definition}[theorem]{Definition}

\newcommand{\ADM}{\mathrm{ADM}}
\newcommand{\Area}{\mathrm{Area}}
\newcommand{\tr}{\mathrm{tr}}
\newcommand{\mH}{m_H}

\title{Modified Geroch Flow for Trapped Surfaces\\
\large A New Attack on Penrose 1973}
\author{}
\date{December 2025}

\begin{document}
\maketitle

\begin{abstract}
We construct a modified flow that can start from a trapped surface (with $H < 0$) and still yields a monotonic Hawking mass. The key insight is to use a \textit{conformal} or \textit{weighted} version of the mean curvature flow that accounts for the extrinsic curvature $k$.
\end{abstract}

\tableofcontents

%% ============================================================================
\section{The Problem with Standard IMCF}
%% ============================================================================

\subsection{IMCF Review}

The inverse mean curvature flow (IMCF) evolves surfaces by:
\begin{equation}
    \frac{\partial x}{\partial t} = \frac{\nu}{H}
\end{equation}

This requires $H > 0$ (outward mean curvature).

For a trapped surface on a maximal slice:
\begin{itemize}
    \item $\theta^+ = H + \tr_\Sigma k < 0$
    \item $\theta^- = H - \tr_\Sigma k < 0$
    \item Adding: $2H < 0$, so $H < 0$
\end{itemize}

IMCF is not defined for $H < 0$.

\subsection{The Geroch Monotonicity}

Under IMCF with $R_g \ge 0$, the Hawking mass:
\begin{equation}
    \mH(\Sigma) = \sqrt{\frac{A(\Sigma)}{16\pi}}\left(1 - \frac{1}{16\pi}\int_\Sigma H^2 dA\right)
\end{equation}
is non-decreasing.

At infinity: $\mH \to M_{\ADM}$.

For a minimal surface ($H = 0$): $\mH = \sqrt{A/(16\pi)}$, giving the Penrose inequality.

%% ============================================================================
\section{Modified Flow 1: $\theta^+$-Inverse Flow}
%% ============================================================================

\begin{keyidea}
Instead of flowing by $1/H$, flow by $1/\theta^+$ where $\theta^+ = H + \tr_\Sigma k$.

For a trapped surface: $\theta^+ < 0$, so $1/\theta^+ < 0$, giving \textit{inward} motion.

For a MOTS: $\theta^+ = 0$, so the flow stops.
\end{keyidea}

\begin{definition}[$\theta^+$-Inverse Flow]
\begin{equation}
    \frac{\partial x}{\partial t} = \frac{\nu}{\theta^+}
\end{equation}
where $\nu$ is the outward normal.
\end{definition}

\subsection{Analysis}

For $\theta^+ < 0$ (trapped):
\begin{equation}
    \frac{\partial x}{\partial t} = \frac{\nu}{\theta^+} < 0 \quad \text{(inward motion)}
\end{equation}

The surface moves \textit{inward}, toward the singularity.

\textbf{Problem:} This doesn't connect to infinity. The Hawking mass at the singularity is not $M_{\ADM}$.

%% ============================================================================
\section{Modified Flow 2: Outward from MOTS}
%% ============================================================================

\begin{keyidea}
Start from the MOTS $\Sigma^*$ (where $\theta^+ = 0$) and flow outward using IMCF.

This gives:
\begin{equation}
    M_{\ADM} \ge \mH(\Sigma^*) = \sqrt{\frac{A(\Sigma^*)}{16\pi}}\left(1 - \frac{1}{16\pi}\int_{\Sigma^*} H^2 dA\right)
\end{equation}

The question is: what is $H$ on the MOTS?
\end{keyidea}

On the MOTS: $\theta^+ = H + \tr_\Sigma k = 0$, so $H = -\tr_\Sigma k$.

The Hawking mass on the MOTS:
\begin{equation}
    \mH(\Sigma^*) = \sqrt{\frac{A^*}{16\pi}}\left(1 - \frac{1}{16\pi}\int_{\Sigma^*} (\tr_\Sigma k)^2 dA\right)
\end{equation}

\textbf{If $k$ is small:} $\mH(\Sigma^*) \approx \sqrt{A^*/(16\pi)}$.

\textbf{If $k$ is large:} $\mH(\Sigma^*)$ could be much smaller than $\sqrt{A^*/(16\pi)}$ or even negative.

\begin{calculation}
On a maximal slice: $\tr_\Sigma k = k_{ij} h^{ij}$ where $h$ is the induced metric on $\Sigma$.

This is bounded by: $|\tr_\Sigma k| \le |k| \cdot \sqrt{2}$ (norm of trace).

For the Hawking mass to be positive:
\begin{equation}
    \frac{1}{16\pi}\int_{\Sigma^*} (\tr_\Sigma k)^2 dA < 1
\end{equation}

This gives:
\begin{equation}
    \int_{\Sigma^*} (\tr_\Sigma k)^2 dA < 16\pi
\end{equation}

By Cauchy-Schwarz:
\begin{equation}
    \left(\int_{\Sigma^*} |\tr_\Sigma k| dA\right)^2 \le A^* \int_{\Sigma^*} (\tr_\Sigma k)^2 dA < 16\pi A^*
\end{equation}

So $\int |\tr_\Sigma k| dA < 4\sqrt{\pi A^*}$.
\end{calculation}

%% ============================================================================
\section{Modified Flow 3: Jang-Weighted IMCF}
%% ============================================================================

\begin{keyidea}
On the Jang manifold $(\bar{M}, \bar{g})$, the MOTS $\Sigma^*$ becomes a minimal surface ($\bar{H} = 0$).

Run IMCF on the Jang manifold:
\begin{equation}
    \frac{\partial x}{\partial t} = \frac{\bar{\nu}}{\bar{H}}
\end{equation}

Since $R_{\bar{g}} \ge 0$ (distributionally), Geroch monotonicity gives:
\begin{equation}
    M_{\ADM}(\bar{g}) \ge \bar{m}_H(\Sigma^*) = \sqrt{\frac{\bar{A}^*}{16\pi}}
\end{equation}

And $M_{\ADM}(\bar{g}) = M_{\ADM}(g)$, $\bar{A}^* = A^*$.

So $M_{\ADM}(g) \ge \sqrt{A^*/(16\pi)}$.
\end{keyidea}

\textbf{This is the standard Jang approach!} It gives the Penrose inequality for the MOTS, but not for arbitrary trapped surfaces.

%% ============================================================================
\section{Modified Flow 4: Two-Phase Flow}
%% ============================================================================

\begin{keyidea}
\textbf{Phase 1:} Start from trapped surface $\Sigma$ and flow to the MOTS $\Sigma^*$.

\textbf{Phase 2:} From $\Sigma^*$, use standard IMCF to infinity.

The challenge is Phase 1: constructing a flow from $\Sigma$ to $\Sigma^*$ that has good monotonicity properties.
\end{keyidea}

\subsection{Candidate for Phase 1: Level Set Flow}

Define a function $\phi$ on the region between $\Sigma$ and $\Sigma^*$ such that:
\begin{itemize}
    \item $\phi = 0$ on $\Sigma$
    \item $\phi = 1$ on $\Sigma^*$
    \item $\nabla \phi$ points outward
\end{itemize}

The level sets $\Sigma_t = \{\phi = t\}$ for $t \in [0, 1]$ interpolate between $\Sigma$ and $\Sigma^*$.

\textbf{Question:} Can we choose $\phi$ so that $A(\Sigma_t)$ is monotonic?

\subsection{Area Monotonicity Condition}

\begin{proposition}
Let $\phi$ be a function with $\nabla \phi \ne 0$ in a region. The area of level sets is non-decreasing iff:
\begin{equation}
    \frac{\partial}{\partial t} A(\Sigma_t) = \int_{\Sigma_t} \frac{H}{|\nabla \phi|} dA \ge 0
\end{equation}
where $H$ is the mean curvature of $\Sigma_t$.
\end{proposition}

For the level sets between $\Sigma$ (trapped, $H < 0$) and $\Sigma^*$ (MOTS, $H = -\tr k$), the mean curvature changes sign somewhere.

\textbf{Problem:} If $H < 0$ on some level set, the area is decreasing there.

%% ============================================================================
\section{Modified Flow 5: Weighted Area Monotonicity}
%% ============================================================================

\begin{keyidea}
Instead of area, consider a \textbf{weighted area}:
\begin{equation}
    A_w(\Sigma) = \int_\Sigma w \, dA
\end{equation}
where $w > 0$ is a weight function.

Choose $w$ so that:
\begin{enumerate}
    \item $A_w(\Sigma) \ge A(\Sigma)$ (or $A_w = A$ for trapped surfaces)
    \item $A_w$ is monotonic under some flow
    \item At infinity: $A_w \to$ standard area
\end{enumerate}
\end{keyidea}

\subsection{Example: Conformal Weight}

Let $w = e^{2f}$ for some function $f$.

Then $A_w(\Sigma) = \int_\Sigma e^{2f} dA$.

If $f = 0$ on trapped surfaces and $f > 0$ in the exterior:
\begin{equation}
    A_w(\Sigma) = A(\Sigma) \quad \text{(for trapped)}
\end{equation}
\begin{equation}
    A_w(\Sigma') > A(\Sigma') \quad \text{(for exterior surfaces)}
\end{equation}

\textbf{Can we choose $f$ so that $A_w$ is monotonic?}

The evolution of weighted area:
\begin{equation}
    \frac{d A_w}{dt} = \int_\Sigma e^{2f} \left(H \cdot v + 2\nabla f \cdot \mathbf{v} + 2\partial_t f\right) dA
\end{equation}
where $v = \langle \mathbf{v}, \nu \rangle$ is the normal velocity.

For monotonicity: choose $f$ to compensate for negative $H$.

%% ============================================================================
\section{Modified Flow 6: The $\theta^+$-Weighted Mass}
%% ============================================================================

\begin{breakthrough}
Define the \textbf{$\theta^+$-weighted Hawking mass}:
\begin{equation}
    m_{\theta}(\Sigma) = \sqrt{\frac{A(\Sigma)}{16\pi}}\left(1 - \frac{1}{16\pi}\int_\Sigma (\theta^+)^2 dA\right)
\end{equation}

For a trapped surface: $\theta^+ < 0$, so $(\theta^+)^2 > 0$, and:
\begin{equation}
    m_\theta(\Sigma) < \sqrt{\frac{A}{16\pi}}
\end{equation}

For a MOTS: $\theta^+ = 0$, so:
\begin{equation}
    m_\theta(\Sigma^*) = \sqrt{\frac{A^*}{16\pi}}
\end{equation}
\end{breakthrough}

\subsection{Monotonicity Question}

\textbf{Is $m_\theta$ monotonic under some flow?}

Consider the flow:
\begin{equation}
    \frac{\partial x}{\partial t} = f(\theta^+) \nu
\end{equation}
for some function $f$.

The evolution of $\theta^+$:
\begin{equation}
    \frac{\partial \theta^+}{\partial t} = -\Delta_\Sigma (f\theta^+) + \text{(curvature terms)}
\end{equation}

This is complicated. Let's try a different approach.

%% ============================================================================
\section{The Direct Inequality Approach}
%% ============================================================================

\begin{keyidea}
Instead of a flow, directly prove:
\begin{equation}
    M_{\ADM} \ge m_\theta(\Sigma) = \sqrt{\frac{A}{16\pi}}\left(1 - \frac{1}{16\pi}\int_\Sigma (\theta^+)^2 dA\right)
\end{equation}

For this to give Penrose, we need:
\begin{equation}
    \sqrt{\frac{A}{16\pi}} \le m_\theta(\Sigma) + \text{(small)}
\end{equation}
i.e., $\int (\theta^+)^2 dA \le 16\pi$.
\end{keyidea}

\subsection{Bound on $\int (\theta^+)^2$}

For a trapped surface, $\theta^+ < 0$ and $\theta^- < 0$.

Adding: $\theta^+ + \theta^- = 2H < 0$.

The "total trapping":
\begin{equation}
    \int_\Sigma (\theta^+)^2 dA \ge 0
\end{equation}

\textbf{Can we bound this by a constant times $A$?}

If $|\theta^+| \le C$ uniformly, then:
\begin{equation}
    \int (\theta^+)^2 dA \le C^2 A
\end{equation}

For the $m_\theta$ mass to be close to $\sqrt{A/(16\pi)}$:
\begin{equation}
    C^2 A \le 16\pi \quad \Rightarrow \quad A \le \frac{16\pi}{C^2}
\end{equation}

This bounds the area of trapped surfaces in terms of their $\theta^+$ bound.

%% ============================================================================
\section{Synthesis: A Possible Proof Strategy}
%% ============================================================================

\begin{theorem}[Conditional Penrose via $\theta^+$-Mass]\label{thm:theta-penrose}
Let $(M, g, k)$ be asymptotically flat initial data satisfying DEC. Let $\Sigma$ be a trapped surface with:
\begin{equation}
    \int_\Sigma (\theta^+)^2 dA \le 16\pi (1 - \delta)
\end{equation}
for some $\delta > 0$. Then:
\begin{equation}
    M_{\ADM} \ge \delta \sqrt{\frac{A(\Sigma)}{16\pi}}
\end{equation}
\end{theorem}

\begin{proof}[Proof sketch]
\textbf{Step 1:} $m_\theta(\Sigma) = \sqrt{A/(16\pi)}(1 - (1-\delta)) = \delta\sqrt{A/(16\pi)}$.

\textbf{Step 2:} Prove $M_{\ADM} \ge m_\theta(\Sigma)$.

This requires a monotonicity formula or direct bound relating $m_\theta$ to $M_{\ADM}$.

\textbf{Gap:} Step 2 is not proven.
\end{proof}

%% ============================================================================
\section{Conclusion}
%% ============================================================================

\begin{keyidea}
\textbf{Summary of modified flow attempts:}

\begin{enumerate}
    \item $\theta^+$-inverse flow: Goes inward, doesn't reach infinity — \textcolor{red}{FAILED}
    
    \item Outward from MOTS: Standard Jang approach — \textcolor{orange}{GIVES CONDITIONAL}
    
    \item Two-phase flow: Area not monotonic in Phase 1 — \textcolor{red}{FAILED}
    
    \item Weighted area: Complicated, no clear choice of weight — \textcolor{orange}{INCOMPLETE}
    
    \item $\theta^+$-weighted mass: Promising definition, monotonicity unproven — \textcolor{orange}{PROMISING}
\end{enumerate}

\textbf{Most promising direction:} The $\theta^+$-weighted Hawking mass $m_\theta$ gives the right value at MOTS and is bounded below for trapped surfaces. Proving $M_{\ADM} \ge m_\theta$ would give Penrose (with a constant factor for strongly trapped surfaces).
\end{keyidea}

\end{document}
