% =========================================================================
%     ATTACKING MOTS PENROSE: THE FINAL PIECE
%
%     Multiple approaches to the remaining challenge
%
%     Author: Da Xu
%     Date: December 2025
% =========================================================================

\documentclass[12pt]{article}
\usepackage{amsmath,amsthm,amssymb}
\usepackage{mathrsfs}
\usepackage{tcolorbox}

\theoremstyle{plain}
\newtheorem{theorem}{Theorem}[section]
\newtheorem{lemma}[theorem]{Lemma}
\newtheorem{proposition}[theorem]{Proposition}
\newtheorem{corollary}[theorem]{Corollary}
\newtheorem{conjecture}[theorem]{Conjecture}

\theoremstyle{definition}
\newtheorem{definition}[theorem]{Definition}
\newtheorem{remark}[theorem]{Remark}

\newcommand{\ADM}{\mathrm{ADM}}
\newcommand{\tr}{\mathrm{tr}}
\newcommand{\Area}{\mathrm{Area}}

\title{\textbf{The MOTS Penrose Inequality:\\The Final Piece of the Puzzle}}
\author{Da Xu}
\date{December 2025}

\begin{document}
\maketitle

\section{The Statement}

\begin{conjecture}[MOTS Penrose]
Let $(M^3, g, k)$ be asymptotically flat initial data satisfying DEC. Let $\Sigma$ be a stable MOTS ($\theta^+ = 0$). Then:
\[
    M_{\ADM} \geq \sqrt{\frac{\Area(\Sigma)}{16\pi}}
\]
\end{conjecture}

\section{Classification of MOTS}

On a MOTS with $\theta^+ = H + \tr_\Sigma k = 0$:
\[
    H = -\tr_\Sigma k
\]

\textbf{Type I:} $\tr_\Sigma k \leq 0 \Rightarrow H \geq 0$

\textbf{Type II:} $\tr_\Sigma k > 0 \Rightarrow H < 0$

\section{Type I: MOTS with $H \geq 0$}

\subsection{The Strategy}

For $H > 0$, we can run IMCF from the MOTS outward!

\subsection{IMCF from MOTS}

\begin{theorem}[Expected]
Let $\Sigma$ be a MOTS with $H > 0$. The IMCF starting from $\Sigma$ satisfies:
\[
    \frac{dm_H}{d(\log A)} \geq 0
\]
under DEC.
\end{theorem}

At infinity: $m_H \to M_{\ADM}$.

At $\Sigma$: $m_H(\Sigma) = \sqrt{A/(16\pi)}(1 - \int H^2/(16\pi))$.

Since $m_H$ is non-decreasing:
\[
    M_{\ADM} \geq m_H(\Sigma)
\]

\textbf{Issue:} This gives $M_{\ADM} \geq m_H(\Sigma)$, not $M_{\ADM} \geq \sqrt{A/(16\pi)}$.

We need to show $m_H(\Sigma) = \sqrt{A/(16\pi)}$ for MOTS, or use a different argument.

\subsection{Key Observation for MOTS}

For MOTS: $H = -\tr_\Sigma k$, so:
\[
    m_H(\Sigma) = \sqrt{\frac{A}{16\pi}}\left(1 - \frac{\int(\tr_\Sigma k)^2}{16\pi}\right)
\]

This is LESS than $\sqrt{A/(16\pi)}$ unless $\tr_\Sigma k = 0$.

\subsection{The Resolution}

\begin{proposition}
For MOTS with $H > 0$ (equivalently $\tr_\Sigma k < 0$), the IMCF monotonicity gives:
\[
    M_{\ADM} \geq \lim_{t\to\infty} m_H(\Sigma_t)
\]
where $\Sigma_t$ is the IMCF evolution.
\end{proposition}

Now, for large spheres: $m_H \to M_{\ADM}$ and $H \to 2/r \to 0$, so $\int H^2 \to 0$.

The issue is whether $m_H$ is monotonic from $\Sigma$ to infinity.

\textbf{This requires the full IMCF monotonicity formula in the spacetime setting.}

\section{Type II: MOTS with $H < 0$}

This is the hard case. We cannot use IMCF directly.

\subsection{Approach 1: Time Reversal}

Under $k \to -k$:
\begin{itemize}
    \item $\tr_\Sigma k \to -\tr_\Sigma k$
    \item $\theta^+ \to \theta^-$
    \item $M_{\ADM}$ unchanged
\end{itemize}

A MOTS ($\theta^+ = 0$) with $\tr_\Sigma k > 0$ becomes:
\begin{itemize}
    \item A surface with $\theta^- = 0$ (past MOTS)
    \item With $\tr_\Sigma(-k) = -\tr_\Sigma k < 0$
\end{itemize}

\begin{proposition}
If Penrose holds for all past MOTS ($\theta^- = 0$), then it holds for all future MOTS ($\theta^+ = 0$).
\end{proposition}

\textbf{Reduces Type II to proving Penrose for past MOTS with $\tr_\Sigma k < 0$.}

But past MOTS with $\tr_\Sigma k < 0$ has:
\[
    \theta^- = H - \tr_\Sigma k = 0 \Rightarrow H = \tr_\Sigma k < 0
\]

So still $H < 0$!

\textbf{Time reversal does NOT help for the $H < 0$ case.}

\subsection{Approach 2: Conformal Transformation}

\textbf{Idea:} Find a conformal metric in which the MOTS becomes minimal.

Let $\tilde{g} = \phi^4 g$. Then:
\[
    \tilde{H} = \phi^{-2}(H + 4\phi^{-1}\nu(\phi))
\]

For $\tilde{H} = 0$:
\[
    \nu(\phi) = -\frac{H\phi}{4}
\]

\textbf{Problem:} The conformal factor $\phi$ must be positive and satisfy this Neumann condition on $\Sigma$.

\textbf{More serious problem:} The conformal transformation changes both $g$ and effectively $k$, so DEC may not be preserved.

\subsection{Approach 3: The Jang Surface}

The Jang equation blows up at MOTS.

Near the MOTS, the Jang surface becomes cylindrical.

\begin{theorem}[Schoen-Yau, Eichmair-Huang-Lee-Schoen]
The regularized Jang surface has non-negative scalar curvature (under DEC), and the boundary behavior at MOTS contributes to the mass.
\end{theorem}

\textbf{The precise contribution:}

The blow-up at a MOTS $\Sigma$ contributes a term to the mass formula:
\[
    M_{\text{Jang}} = M_{\ADM} + \text{(MOTS contribution)}
\]

The MOTS contribution involves $\int_\Sigma |H|$ and related quantities.

\subsection{Approach 4: Spacetime Positive Mass Theorem}

\begin{theorem}[Witten, Schoen-Yau]
For asymptotically flat initial data satisfying DEC:
\[
    M_{\ADM} \geq 0
\]
with equality iff data embeds in Minkowski.
\end{theorem}

Can we strengthen this to involve $\sqrt{A/(16\pi)}$?

\textbf{Idea:} Cut out the region inside $\Sigma$ and cap it off.

\textbf{Problem:} The cap must satisfy DEC, which constrains its geometry.

\section{A Promising New Approach}

\subsection{The Capacity Method for MOTS}

\begin{definition}
The capacity of $\Sigma$ in $(M, g)$:
\[
    \text{Cap}(\Sigma) = \inf\left\{\int_M |\nabla u|^2 : u|_\Sigma = 1, u \to 0 \text{ at }\infty\right\}
\]
\end{definition}

\begin{theorem}[Bray-Miao type]
For asymptotically flat $(M, g)$ with $R \geq 0$:
\[
    M_{\ADM} \geq \frac{\text{Cap}(\Sigma)}{4\pi}
\]
\end{theorem}

For a round sphere in Euclidean space:
\[
    \text{Cap}(S_r) = 4\pi r, \quad M = 0, \quad A = 4\pi r^2
\]

So $\text{Cap} = 2\sqrt{\pi A}$.

\subsection{MOTS Capacity Conjecture}

\begin{conjecture}
For a MOTS $\Sigma$ in $(M, g, k)$ satisfying DEC:
\[
    \text{Cap}(\Sigma) \geq 2\sqrt{\pi \Area(\Sigma)}
\]
\end{conjecture}

If true:
\[
    M_{\ADM} \geq \frac{\text{Cap}}{4\pi} \geq \frac{2\sqrt{\pi A}}{4\pi} = \frac{\sqrt{A}}{2\sqrt{\pi}} = \sqrt{\frac{A}{4\pi}}
\]

This is 2x stronger than Penrose!

\textbf{So capacity gives more than we need.}

\subsection{Analysis}

For MOTS with $H \neq 0$, the capacity might differ from the minimal surface case.

\textbf{Question:} Does $H \neq 0$ increase or decrease capacity?

\textbf{Physical intuition:} Larger $|H|$ means more "curved" surface, which might trap more capacity.

\section{The Geroch Monotonicity Revisited}

\subsection{The Full Formula}

The Geroch monotonicity for Hawking mass under IMCF:
\[
    \frac{dm_H}{dt} = \frac{\sqrt{A}}{(16\pi)^{3/2}}\int_\Sigma \left[\frac{|\nabla H|^2}{H^2} + G(\nu, \nu) - \frac{1}{2}|k - \frac{\tr k}{3}g|^2 + \cdots\right] dA
\]

The key terms:
\begin{itemize}
    \item $G(\nu, \nu) \geq 0$ by DEC
    \item $|\nabla H|^2/H^2 \geq 0$
    \item The $k$ term has mixed signs!
\end{itemize}

\subsection{For MOTS}

At a MOTS starting point:
\begin{itemize}
    \item $H = -\tr_\Sigma k$
    \item The $k$ contributions may be non-positive
\end{itemize}

\textbf{This is why direct IMCF from MOTS doesn't immediately work.}

\section{A Potential Resolution: The Two-Step Method}

\subsection{Step 1: $\theta^+$-Flow}

From trapped $\Sigma_0$ to MOTS $\Sigma^*$.

Area increases: $A^* \geq A_0$.

\subsection{Step 2: IMCF from Favorable MOTS}

If $\Sigma^*$ has $H \geq 0$ ($\tr_{\Sigma^*} k \leq 0$):
\begin{itemize}
    \item Run IMCF from $\Sigma^*$
    \item $m_H$ is monotonic
    \item Get $M_{\ADM} \geq m_H(\Sigma^*)$
\end{itemize}

\subsection{Step 3: Handle Unfavorable MOTS}

If $\Sigma^*$ has $H < 0$:
\begin{itemize}
    \item Need alternative argument
    \item Or show $\theta^+$-flow reaches favorable MOTS
\end{itemize}

\subsection{Key Question}

\begin{conjecture}[Favorable MOTS Reachability]
The $\theta^+$-flow from any trapped surface eventually reaches a MOTS with $\tr_\Sigma k \leq 0$ (hence $H \geq 0$).
\end{conjecture}

If true, we can always use IMCF from the limiting MOTS!

\section{The Stability Argument}

\subsection{MOTS Stability Operator}

\[
    L\phi = -\Delta\phi - 2\Omega\cdot\nabla\phi + V\phi
\]

A MOTS is stable if $\lambda_1(L) \geq 0$.

\subsection{Stability and Area}

\begin{proposition}
A stable outermost MOTS minimizes area among nearby trapped surfaces.
\end{proposition}

This is consistent with our area monotonicity!

\subsection{Using Stability}

\begin{conjecture}
Stable MOTS satisfy:
\[
    \text{Cap}(\Sigma) \geq f(A, \int H^2)
\]
for some function $f$ with $f \geq 2\sqrt{\pi A}$.
\end{conjecture}

\section{Conclusion}

\begin{tcolorbox}[colback=blue!20, colframe=blue!75!black]
\textbf{STATUS OF MOTS PENROSE}

\textbf{Type I (H ≥ 0):} 
\begin{itemize}
    \item Can apply IMCF from MOTS
    \item Monotonicity gives $M \geq m_H(\Sigma^*)$
    \item Need: Show this implies $M \geq \sqrt{A/(16\pi)}$
\end{itemize}

\textbf{Type II (H < 0):}
\begin{itemize}
    \item Cannot use IMCF directly
    \item Time reversal doesn't help
    \item Capacity method most promising
\end{itemize}

\textbf{BEST PATH FORWARD:}

\begin{enumerate}
    \item Prove MOTS Capacity Conjecture: $\text{Cap}(\Sigma) \geq 2\sqrt{\pi A}$
    \item Combined with Bray-Miao: $M \geq \text{Cap}/(4\pi) \geq \sqrt{A/(4\pi)}$
    \item This gives Penrose with room to spare!
\end{enumerate}

Alternatively: Show $\theta^+$-flow always reaches Type I MOTS.
\end{tcolorbox}

\end{document}
