% GAP 2 ANALYSIS: Alternative Functional Without θ± in Denominator
% Goal: Find a well-defined quasi-local mass at MOTS crossings

\documentclass[11pt]{amsart}
\usepackage{amsmath,amssymb,amsthm}

\newtheorem{proposition}{Proposition}
\newtheorem{lemma}{Lemma}
\newtheorem{theorem}{Theorem}
\newtheorem{remark}{Remark}

\begin{document}

\title{Gap 2 Analysis: Resolving the MOTS Crossing Singularity}
\date{\today}
\maketitle

\section{The Problem}

The current functional is:
\begin{equation}
\mathcal{Q}(\Sigma) = \sqrt{\frac{|\Sigma|}{16\pi}}\left(1 - \frac{1}{16\pi}\int_\Sigma \left[\theta^+\theta^- + |\zeta|^2 + \frac{1}{4}\left|\frac{\sigma^+}{\theta^+} - \frac{\sigma^-}{\theta^-}\right|^2\theta^+\theta^-\right] dA\right)
\end{equation}

\textbf{Problem}: When $\theta^+ \to 0$ with $\sigma^+ \neq 0$, the term $\sigma^+/\theta^+ \to \infty$.

\section{Understanding the Origin}

The bad term in Hawking mass variation is $-\sigma^+:\sigma^-$.

The identity used:
\begin{equation}
-\sigma^+:\sigma^- = -\frac{1}{4}|\sigma^+ + \sigma^-|^2 + \frac{1}{4}|\sigma^+ - \sigma^-|^2
\label{eq:raw-identity}
\end{equation}

This is EXACT and requires NO division by $\theta^\pm$.

The division was introduced to create a "boost-invariant" correction term. But we can ask:

\textbf{Key Question}: Can we absorb $-\sigma^+:\sigma^-$ using \eqref{eq:raw-identity} directly, without normalizing by $\theta^\pm$?

\section{Attempt 1: Direct Shear Correction}

Define:
\begin{equation}
\tilde{\mathcal{Q}}(\Sigma) = \sqrt{\frac{|\Sigma|}{16\pi}}\left(1 - \frac{1}{16\pi}\int_\Sigma \left[\theta^+\theta^- + |\zeta|^2 + \frac{1}{4}|\sigma^+ - \sigma^-|^2\right] dA\right)
\end{equation}

\textbf{Issue}: This is NOT boost-invariant. Under $\ell \to \lambda\ell$, $n \to \lambda^{-1}n$:
\begin{itemize}
\item $\theta^+ \to \lambda\theta^+$, $\theta^- \to \lambda^{-1}\theta^-$, so $\theta^+\theta^-$ is invariant ✓
\item $\sigma^+ \to \lambda\sigma^+$, $\sigma^- \to \lambda^{-1}\sigma^-$, so $|\sigma^+ - \sigma^-|^2$ is NOT invariant ✗
\end{itemize}

\section{Attempt 2: Boost-Invariant Combinations Without Division}

What combinations of $\sigma^\pm$ are boost-invariant?
\begin{itemize}
\item $\sigma^+:\sigma^- = \sigma^+_{ab}\sigma^{-ab}$ — YES (invariant)
\item $|\sigma^+|^2|\sigma^-|^2$ — YES
\item $|\sigma^+||\sigma^-|$ — YES
\item $|\sigma^+|^2/|\theta^+|$ — transforms as $\lambda$
\end{itemize}

The raw cross-term $\sigma^+:\sigma^-$ IS boost-invariant!

So the identity \eqref{eq:raw-identity} cannot directly give us a boost-invariant completion, because $|\sigma^+|^2$ and $|\sigma^-|^2$ are NOT boost-invariant.

\section{Attempt 3: Work in a Fixed Gauge}

Instead of seeking boost invariance, we can:
\begin{enumerate}
\item Fix a canonical gauge (e.g., $\ell \cdot n = -1$ and some normalization)
\item Work entirely in that gauge
\item The final inequality is gauge-independent (it relates physical masses)
\end{enumerate}

\textbf{Canonical gauge choice}: Normalize so that $\theta^+ = \theta^-$ along the flow.

In this gauge: $\sigma^+/\theta^+ - \sigma^-/\theta^- = (\sigma^+ - \sigma^-)/\theta^+$, and we only need $\theta^+ \neq 0$.

But this gauge may not exist globally...

\section{Attempt 4: The "Harmonic Mean" Normalization}

Consider the harmonic mean: $\theta_H := \frac{2\theta^+\theta^-}{\theta^+ + \theta^-}$.

This is boost-invariant and vanishes when either $\theta^+ = 0$ OR $\theta^- = 0$.

Define:
\begin{equation}
\hat{\sigma}^\pm := \frac{\sigma^\pm}{\sqrt{|\theta^\pm|}}
\end{equation}

Then $|\hat{\sigma}^+|^2 = |\sigma^+|^2/|\theta^+|$ transforms as $\lambda^{1}$, not invariant.

Not helpful.

\section{Attempt 5: The Key Insight — Raychaudhuri Coupling}

The Raychaudhuri equation says:
\begin{equation}
\frac{d\theta^+}{ds} = -\frac{1}{2}(\theta^+)^2 - |\sigma^+|^2 - R_{\ell\ell}
\end{equation}

So $|\sigma^+|^2$ and $(\theta^+)^2$ appear TOGETHER in the evolution. 

\textbf{Key observation}: Define
\begin{equation}
\Theta^+ := \theta^+ + \alpha|\sigma^+| \quad \text{for some } \alpha > 0
\end{equation}

This "regularized expansion" includes the shear contribution. When $\theta^+ \to 0$ but $\sigma^+ \neq 0$, we have $\Theta^+ \neq 0$.

\textbf{Problem}: $|\sigma^+|$ is not a scalar — it's the norm of a traceless tensor.

\section{Attempt 6: Re-examine What We Actually Need}

Going back to basics: In the monotonicity formula, we need to show
\begin{equation}
\frac{d\mathcal{Q}}{ds} \geq 0
\end{equation}

The problematic term is $-\sigma^+:\sigma^-$. After completing the square:
\begin{equation}
-\sigma^+:\sigma^- + \frac{1}{4}|\sigma^+ - \sigma^-|^2 = \frac{1}{4}|\sigma^+ + \sigma^-|^2 \geq 0
\end{equation}

So the correction needed is $+\frac{1}{4}|\sigma^+ - \sigma^-|^2$ in $d\mathcal{Q}/ds$.

For this to come from $d(\text{something})/ds$, we need something whose derivative gives $|\sigma^+ - \sigma^-|^2$.

\section{Attempt 7: Integral Formulation}

Instead of defining $\mathcal{Q}$ pointwise and differentiating, define:
\begin{equation}
\mathcal{Q}(s) := \sqrt{\frac{|\Sigma_0|}{16\pi}} + \int_0^s \mathcal{F}(\Sigma_{s'}) \, ds'
\end{equation}
where $\mathcal{F}$ is a non-negative flux term.

Then monotonicity is automatic: $\mathcal{Q}'(s) = \mathcal{F}(s) \geq 0$.

The question becomes: Can we construct $\mathcal{F}$ such that:
\begin{enumerate}
\item $\mathcal{F} \geq 0$ (monotonicity)
\item $\mathcal{Q}(0) = \sqrt{|\Sigma_0|/16\pi}$ (initial value)
\item $\mathcal{Q}(\infty) = M_B$ (asymptotic value)
\end{enumerate}

\textbf{From the Hawking mass variation}:
\begin{equation}
\frac{dm_H}{ds} = \frac{\sqrt{|\Sigma|/16\pi}}{8\pi}\int_\Sigma \left[\frac{1}{2}\theta^+(\mu - |J|) - \sigma^+:\sigma^- - 2|\zeta|^2 + \ldots \right] dA
\end{equation}

The term $\frac{1}{2}\theta^+(\mu - |J|) \geq 0$ by DEC.

So we need a correction $\mathcal{C}(s)$ such that:
\begin{equation}
\mathcal{F}(s) = \frac{dm_H}{ds} + \frac{d\mathcal{C}}{ds} \geq 0
\end{equation}

And $\frac{d\mathcal{C}}{ds}$ must absorb $\sigma^+:\sigma^-$ and $|\zeta|^2$.

\section{The Core Algebraic Fact}

\begin{proposition}
For any traceless symmetric 2-tensors $\sigma^+, \sigma^-$ on a 2-dimensional surface:
\begin{equation}
\sigma^+:\sigma^- \leq \frac{1}{2}(|\sigma^+|^2 + |\sigma^-|^2)
\end{equation}
with equality iff $\sigma^+ = \sigma^-$.
\end{proposition}

\begin{proof}
$0 \leq |\sigma^+ - \sigma^-|^2 = |\sigma^+|^2 - 2\sigma^+:\sigma^- + |\sigma^-|^2$.
\end{proof}

This means:
\begin{equation}
-\sigma^+:\sigma^- \geq -\frac{1}{2}(|\sigma^+|^2 + |\sigma^-|^2)
\end{equation}

So we can BOUND the bad term by shear norms, even without completing the square!

\section{Attempt 8: The Shear-Bounded Functional}

Define:
\begin{equation}
\mathcal{Q}_{\text{bound}}(\Sigma) := m_H(\Sigma) + \frac{\sqrt{|\Sigma|/16\pi}}{16\pi}\int_\Sigma \frac{1}{2}(|\sigma^+|^2 + |\sigma^-|^2) \, dA
\end{equation}

Then:
\begin{align}
\frac{d\mathcal{Q}_{\text{bound}}}{ds} &= \frac{dm_H}{ds} + \frac{d}{ds}\left[\frac{\sqrt{|\Sigma|/16\pi}}{16\pi}\int_\Sigma \frac{1}{2}(|\sigma^+|^2 + |\sigma^-|^2) \, dA\right] \\
&\geq \frac{dm_H}{ds} + \frac{\sqrt{|\Sigma|/16\pi}}{16\pi}\int_\Sigma (-\sigma^+:\sigma^-) \, dA \quad \text{(by Prop.)}
\end{align}

Wait, this goes the wrong way — we're adding MORE not canceling.

\section{Attempt 9: The Evolution Equations}

Let's compute $\frac{d}{ds}|\sigma^+|^2$ directly.

From the Sachs equations (null evolution of shear):
\begin{equation}
\mathcal{L}_\ell \sigma^+_{ab} = -\theta^+ \sigma^+_{ab} - C_{a\ell b\ell}
\end{equation}
where $C_{a\ell b\ell}$ is the Weyl tensor component.

So:
\begin{equation}
\frac{d}{ds}|\sigma^+|^2 = 2\sigma^+:\mathcal{L}_\ell\sigma^+ = -2\theta^+|\sigma^+|^2 - 2\sigma^+:C_{\ell\ell}
\end{equation}

Similarly for $\sigma^-$ (but evolving in $n$ direction is different).

\textbf{Key}: The shear terms couple to Weyl curvature (gravitational radiation), not just to Ricci.

\section{Conclusion of Analysis}

The fundamental obstruction is:
\begin{enumerate}
\item The bad term $-\sigma^+:\sigma^-$ requires a "completing the square" correction
\item Any such correction involving $|\sigma^+ - \sigma^-|^2$ is NOT boost-invariant
\item Making it boost-invariant by dividing by $\theta^\pm$ introduces singularities at MOTS
\item Working in a fixed gauge may avoid this, but gauge-fixing has its own issues
\end{enumerate}

\textbf{Most promising direction}: 
\begin{itemize}
\item Accept that the flow should AVOID MOTS crossings, not pass through them
\item When approaching a MOTS, JUMP to an outer envelope
\item This is exactly the Huisken-Ilmanen strategy, but in null setting
\end{itemize}

This means: Gap 2 should be subsumed into Gap 1. The "correct" weak flow should jump before $\theta^+ = 0$.

\end{document}
