%% FINAL_HONEST_ASSESSMENT.tex
%%
%% FINAL ASSESSMENT: The Area Dominance Problem
%%
%% After extensive exploration, what have we learned?
%%
%% December 2025

\documentclass[11pt]{amsart}
\usepackage{amsmath,amssymb,amsthm}
\usepackage{tcolorbox}
\usepackage{tikz}

\tcbuselibrary{theorems}

\newtcolorbox{proven}{
    colback=green!10!white,
    colframe=green!75!black,
    title={\textbf{PROVEN}}
}

\newtcolorbox{unproven}{
    colback=red!10!white,
    colframe=red!75!black,
    title={\textbf{NOT PROVEN}}
}

\newtcolorbox{insight}{
    colback=blue!10!white,
    colframe=blue!75!black,
    title={\textbf{KEY INSIGHT}}
}

\newtcolorbox{conjecture}{
    colback=yellow!10!white,
    colframe=yellow!75!black,
    title={\textbf{CONJECTURE}}
}

\newtheorem{theorem}{Theorem}
\newtheorem{lemma}[theorem]{Lemma}
\newtheorem{proposition}[theorem]{Proposition}
\newtheorem{corollary}[theorem]{Corollary}
\theoremstyle{definition}
\newtheorem{definition}[theorem]{Definition}
\newtheorem{remark}[theorem]{Remark}

\newcommand{\Area}{\mathrm{Area}}
\newcommand{\Vol}{\mathrm{Vol}}
\newcommand{\divv}{\mathrm{div}}
\DeclareMathOperator{\tr}{tr}

\title{Final Honest Assessment:\\
The State of Area Dominance}
\author{December 2025}

\begin{document}
\maketitle

\begin{abstract}
After extensive exploration using geometric flows, level set methods, 
Jang equation techniques, and causal comparison arguments, we present 
an honest assessment of what is proven, what remains unproven, and 
what we have learned about the Area Dominance problem.
\end{abstract}

%% ============================================================================
\section{Summary: What Is Proven}
%% ============================================================================

\begin{proven}
\textbf{Theorem (MOTS Penrose Inequality).}
Let $(\mathcal{C}, g, k)$ be asymptotically flat initial data satisfying DEC.
Let $\Sigma^*$ be the outermost MOTS.

Then:
\begin{equation}
    M_{\text{ADM}} \ge \sqrt{\frac{\Area(\Sigma^*)}{16\pi}}
\end{equation}
\end{proven}

\textbf{Proof method:} Jang equation + Riemannian Penrose inequality + 
careful analysis of cylindrical ends.

\begin{proven}
\textbf{Theorem (Trapped Inside MOTS).}
Let $\Sigma$ be a trapped surface. Then $\Sigma$ lies inside the outermost 
MOTS $\Sigma^*$:
\begin{equation}
    \Sigma \subset \text{int}(\Sigma^*)
\end{equation}
\end{proven}

\textbf{Proof:} Maximum principle for $\theta^+$.

%% ============================================================================
\section{Summary: What Is NOT Proven}
%% ============================================================================

\begin{unproven}
\textbf{Area Dominance Conjecture.}
Let $\Sigma$ be a trapped surface inside MOTS $\Sigma^*$.

\textbf{Conjecture:}
\begin{equation}
    \Area(\Sigma) \le \Area(\Sigma^*)
\end{equation}
\end{unproven}

\textbf{This remains OPEN despite:}
\begin{itemize}
    \item Ricci-flow inspired techniques
    \item Expansion-normalized flows
    \item Trapped surface entropy functionals
    \item Level set methods
    \item Jang surface techniques
    \item Causal comparison arguments
\end{itemize}

%% ============================================================================
\section{The Fundamental Obstruction}
%% ============================================================================

\begin{insight}
\textbf{All approaches fail at the same point.}

The mean curvature of a surface $\Sigma$ in $(\mathcal{C}, g)$ is:
\begin{equation}
    H = \theta^+ - P
\end{equation}

where $P = \tr_\Sigma(k)$.

For a trapped surface: $\theta^+ < 0$.

For Area Dominance (area increasing outward), we need: $H > 0$.

This requires: $P < \theta^+ < 0$, i.e., $P < 0$.

\textbf{But DEC does NOT constrain the sign of $P$.}
\end{insight}

The DEC constrains:
\begin{equation}
    \mu \ge |J| \ge 0
\end{equation}

The momentum constraint gives:
\begin{equation}
    \divv(k - (\tr k)g) = 8\pi J
\end{equation}

But the TRACE of $k$ on a surface, $P = \tr_\Sigma(k)$, is NOT constrained 
to have a specific sign.

%% ============================================================================
\section{Physical Interpretation}
%% ============================================================================

\begin{insight}
\textbf{$P$ measures the flux of momentum through $\Sigma$.}

\begin{itemize}
    \item $P < 0$: Net ingoing momentum (collapsing matter)
    \item $P > 0$: Net outgoing momentum (expanding matter)
    \item $P = 0$: Time-symmetric (moment of time symmetry)
\end{itemize}
\end{insight}

For gravitational collapse (the physical scenario Penrose had in mind):

\textbf{Physically:} Collapsing matter has ingoing momentum, so $P < 0$.

\textbf{Mathematically:} This is an ADDITIONAL assumption, not implied by DEC.

%% ============================================================================
\section{What Would Prove Area Dominance}
%% ============================================================================

Any of the following would suffice:

\subsection{Approach 1: Prove $P < 0$ for Trapped Surfaces}

\begin{conjecture}
For trapped surfaces in asymptotically flat initial data satisfying DEC:
\begin{equation}
    P = \tr_\Sigma(k) \le 0
\end{equation}
\end{conjecture}

\textbf{Status:} Unknown. May be FALSE for exotic initial data.

\subsection{Approach 2: Find a $P$-Independent Argument}

Find a proof of Area Dominance that doesn't depend on the sign of $P$.

\textbf{All known approaches require $P < 0$ or equivalent.}

\subsection{Approach 3: Spacetime Methods}

Use the EVOLUTION (not just constraint equations) to prove Area Dominance.

Under WCC, the spacetime evolution may provide additional structure 
(event horizon, cosmic censorship) that constrains areas.

\textbf{Status:} The causal comparison approach shows area DECREASES 
along null directions, not increases. Spacetime structure doesn't 
obviously help.

\subsection{Approach 4: Accept Additional Hypothesis}

\begin{definition}[Collapsing Initial Data]
Initial data $(\mathcal{C}, g, k)$ is \emph{collapsing} if:
\begin{equation}
    P = \tr_\Sigma(k) \le 0 \quad \text{on all trapped surfaces } \Sigma
\end{equation}
\end{definition}

\begin{theorem}[Penrose for Collapsing Data]
For collapsing initial data satisfying DEC with trapped surface $\Sigma$:
\begin{equation}
    M_{\text{ADM}} \ge \sqrt{\frac{\Area(\Sigma)}{16\pi}}
\end{equation}
\end{theorem}

\begin{proof}
\begin{enumerate}
    \item $\Sigma$ is inside MOTS $\Sigma^*$ (proven)
    \item For collapsing data: $H = \theta^+ - P > 0$ on $\Sigma$
    \item Area increases from $\Sigma$ to $\Sigma^*$: 
          $\Area(\Sigma) \le \Area(\Sigma^*)$
    \item MOTS Penrose: $M \ge \sqrt{\Area(\Sigma^*)/(16\pi)}$
    \item Combine: $M \ge \sqrt{\Area(\Sigma)/(16\pi)}$
\end{enumerate}
\end{proof}

%% ============================================================================
\section{Counterexample Possibility}
%% ============================================================================

\begin{conjecture}[Possible Counterexample]
There may exist initial data $(\mathcal{C}, g, k)$ satisfying DEC with:
\begin{itemize}
    \item A trapped surface $\Sigma$ with $\Area(\Sigma) = A$
    \item An outermost MOTS $\Sigma^*$ with $\Area(\Sigma^*) < A$
\end{itemize}

In such data, the Penrose inequality would FAIL for $\Sigma$.
\end{conjecture}

\textbf{Construction idea:}
\begin{enumerate}
    \item Start with time-symmetric data (Schwarzschild)
    \item Add outgoing momentum ($P > 0$) near the horizon
    \item The inner trapped surface may have larger area than the MOTS
\end{enumerate}

\textbf{Constraint:} Must satisfy DEC. The momentum constraint:
\begin{equation}
    \divv(k - (\tr k)g) = 8\pi J
\end{equation}
requires $|J| \le \mu$.

It's unclear if such data can be constructed while satisfying DEC.

%% ============================================================================
\section{Novel Mathematics Developed}
%% ============================================================================

Our exploration produced several new constructions:

\subsection{Expansion-Normalized Flow (ENF)}

\begin{equation}
    \frac{\partial g}{\partial t} = -2\theta^+ (h - k)
\end{equation}

\textbf{Properties:}
\begin{itemize}
    \item MOTS are fixed points
    \item Flow rate controlled by $\theta^+$
    \item NOT proven to have good existence/regularity
\end{itemize}

\subsection{Trapped Surface Entropy}

\begin{equation}
    S[\Sigma, g, k] = \Area(\Sigma) - \int_\Sigma \theta^+ \log|\theta^+| dA 
    + \lambda \int_\Omega (\mu - |J|) dV
\end{equation}

\textbf{Properties:}
\begin{itemize}
    \item Combines area with expansion information
    \item Includes DEC contribution
    \item Monotonicity NOT proven
\end{itemize}

\subsection{MOTS Area Hierarchy}

\begin{conjecture}
For nested MOTS $\Sigma_1 \subset \Sigma_2$:
\begin{equation}
    \Area(\Sigma_1) \le \Area(\Sigma_2)
\end{equation}
\end{conjecture}

\textbf{Status:} Related to but distinct from Area Dominance. 
Uses stability theory of MOTS.

%% ============================================================================
\section{Lessons Learned}
%% ============================================================================

\begin{enumerate}
    \item \textbf{The constraint equations are insufficient.}
    
    Pure constraint equation analysis cannot determine the sign of $P$.
    
    \item \textbf{Riemannian techniques don't directly apply.}
    
    The presence of extrinsic curvature $k$ spoils mean curvature comparisons.
    
    \item \textbf{Spacetime methods face opposite direction.}
    
    Null area theorems show area DECREASES from trapped surfaces, 
    opposite to what Area Dominance needs.
    
    \item \textbf{The Jang equation transforms but doesn't solve.}
    
    The Jang surface reformulates the problem in Riemannian terms, 
    but the sign ambiguity persists.
    
    \item \textbf{Physical assumptions may be necessary.}
    
    The Penrose 1973 conjecture may implicitly assume "physical" 
    collapse scenarios where $P < 0$.
\end{enumerate}

%% ============================================================================
\section{Recommendations for Future Work}
%% ============================================================================

\subsection{Direction 1: Characterize When Area Dominance Holds}

Instead of proving Area Dominance in general, characterize the class 
of initial data where it holds.

\textbf{Hypothesis:} Area Dominance holds when the trapped surface 
is "dynamically formed" (arises from gravitational collapse).

\subsection{Direction 2: Construct Counterexamples}

Attempt to construct initial data violating Area Dominance.

Success would show the Penrose conjecture needs additional hypotheses.

Failure might reveal why Area Dominance must hold.

\subsection{Direction 3: Use Full Einstein Evolution}

Analyze the FULL spacetime evolution, not just initial data.

Under WCC, the event horizon provides global structure that constrains 
trapped surfaces.

\textbf{Key question:} Does dynamical evolution force $P < 0$ on 
trapped surfaces?

\subsection{Direction 4: Weak Formulations}

Consider weak versions of Area Dominance that might be provable:

\begin{conjecture}[Weak Area Dominance]
There exists a surface $\Sigma'$ with:
\begin{itemize}
    \item $\theta^+|_{\Sigma'} \le 0$
    \item $\Area(\Sigma) \le \Area(\Sigma') \le \Area(\Sigma^*)$
\end{itemize}
\end{conjecture}

(This allows $\Sigma'$ to be weakly trapped, not necessarily MOTS.)

%% ============================================================================
\section{Final Status}
%% ============================================================================

\begin{center}
\begin{tabular}{|l|c|}
\hline
\textbf{Statement} & \textbf{Status} \\
\hline
MOTS Penrose Inequality & \textcolor{green}{PROVEN} \\
Trapped $\subset$ int(MOTS) & \textcolor{green}{PROVEN} \\
Area Dominance & \textcolor{red}{OPEN} \\
Full Penrose 1973 & \textcolor{red}{OPEN} \\
\hline
\end{tabular}
\end{center}

\vspace{1em}

\begin{center}
\fbox{\parbox{0.9\textwidth}{
\textbf{Honest Conclusion:}

The Penrose 1973 conjecture remains OPEN. The gap is Area Dominance: 
showing $\Area(\Sigma) \le \Area(\Sigma^*)$ for trapped $\Sigma$ 
inside MOTS $\Sigma^*$.

This gap exists because the Dominant Energy Condition does NOT 
constrain the sign of $P = \tr_\Sigma(k)$, which determines whether 
area increases or decreases outward from a trapped surface.

Resolution likely requires either:
\begin{enumerate}
    \item Additional physical assumptions (collapsing data)
    \item Full spacetime analysis (not just initial data)
    \item New mathematical techniques beyond current scope
\end{enumerate}
}}
\end{center}

\end{document}
