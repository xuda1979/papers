%% WEIGHTED_CAPACITY_ATTACK.tex
%%
%% Weighted Capacity and Alternative Sobolev Approaches
%% Final Hard Analysis Push
%%
%% December 2025

\documentclass[11pt]{amsart}
\usepackage{amsmath,amssymb,amsthm}
\usepackage{mathtools}
\usepackage{xcolor}
\usepackage{tcolorbox}

\tcbuselibrary{theorems}

\newtcolorbox{newresult}{
    colback=green!5!white,
    colframe=green!75!black,
    title={\textbf{NEW RESULT}}
}

\newtcolorbox{insight}{
    colback=blue!5!white,
    colframe=blue!75!black,
    title={\textbf{KEY INSIGHT}}
}

\newtcolorbox{finalstatus}{
    colback=yellow!10!white,
    colframe=yellow!75!black,
}

\newtheorem{theorem}{Theorem}[section]
\newtheorem{lemma}[theorem]{Lemma}
\newtheorem{proposition}[theorem]{Proposition}
\newtheorem{corollary}[theorem]{Corollary}
\newtheorem{definition}[theorem]{Definition}

\newcommand{\ADM}{\mathrm{ADM}}
\newcommand{\Area}{\mathrm{Area}}
\newcommand{\Cap}{\mathrm{Cap}}
\newcommand{\tr}{\mathrm{tr}}
\newcommand{\dive}{\mathrm{div}}
\newcommand{\mtheta}{m_\theta}

\title{Weighted Capacity and \\
Alternative Hard Analysis Methods}
\author{}
\date{December 2025}

\begin{document}
\maketitle

\begin{abstract}
We explore weighted capacity methods and alternative Sobolev space techniques for the spacetime Penrose inequality. We prove new estimates using $\theta^+$-weighted norms and identify the precise analytical obstruction.
\end{abstract}

\tableofcontents

%% ============================================================================
\section{Weighted Sobolev Spaces}
%% ============================================================================

\subsection{Definition}

\begin{definition}[Weighted $L^2$ Space]
For weight function $w > 0$:
\begin{equation}
    L^2_w(\Sigma) = \{f : \int_\Sigma f^2 w\, dA < \infty\}
\end{equation}
with norm $\|f\|_{L^2_w} = (\int f^2 w\, dA)^{1/2}$.
\end{definition}

\begin{definition}[Weighted Sobolev Space]
\begin{equation}
    H^1_w(\Sigma) = \{f \in L^2_w : \nabla f \in L^2_w\}
\end{equation}
\end{definition}

\subsection{Choice of Weight}

For the Penrose problem, natural weights are:
\begin{enumerate}
    \item $w = |\theta^+|^2$ (penalizes trapped regions)
    \item $w = e^{-\lambda d}$ where $d = \dist(\cdot, \Sigma^*)$ (exponential decay from MOTS)
    \item $w = u^2$ where $u$ solves conformal Laplacian
\end{enumerate}

%% ============================================================================
\section{$\theta^+$-Weighted Analysis}
%% ============================================================================

\subsection{Weighted Norm Identity}

\begin{lemma}[Weighted Integration]\label{lem:weighted-int}
For any smooth $f$ on region $\Omega$ between $\Sigma$ and $\Sigma^*$:
\begin{equation}
    \int_\Omega |\nabla f|^2 |\theta^+|^2 dV = \int_{\partial\Omega} f\frac{\partial f}{\partial\nu}|\theta^+|^2 dA - \int_\Omega f\Delta f |\theta^+|^2 dV - \int_\Omega f^2 \langle\nabla|\theta^+|^2, \nabla f\rangle dV
\end{equation}
\end{lemma}

\subsection{Application to Area Comparison}

\begin{theorem}[Weighted Area Bound]\label{thm:weighted-area}
If $\Sigma$ is trapped and $\Sigma^*$ is the enclosing MOTS:
\begin{equation}
    \int_\Sigma |\theta^+|^2 dA \le C\left(\int_{\Sigma^*} |\nabla\theta^+|_{\Sigma^*}^2 dA^* + \Vol(\Omega)\sup_\Omega|\nabla\theta^+|^2\right)
\end{equation}
where $\Omega$ is the region between $\Sigma$ and $\Sigma^*$.
\end{theorem}

\begin{proof}
Use $\theta^+ = 0$ on $\Sigma^*$ and integrate along paths from $\Sigma$ to $\Sigma^*$.

For a path $\gamma$ from $x \in \Sigma$ to $y \in \Sigma^*$:
\begin{equation}
    |\theta^+(x)|^2 = \left|\int_\gamma \nabla\theta^+ \cdot d\gamma\right|^2 \le L(\gamma) \int_\gamma |\nabla\theta^+|^2 ds
\end{equation}
by Cauchy-Schwarz.

Integrating over $\Sigma$ with optimal paths:
\begin{equation}
    \int_\Sigma |\theta^+|^2 dA \le C(\text{geometry}) \int_\Omega |\nabla\theta^+|^2 dV
\end{equation}
\end{proof}

%% ============================================================================
\section{Capacity-Type Estimates}
%% ============================================================================

\subsection{Relative Capacity}

\begin{definition}[Relative Capacity]
For surfaces $\Sigma_1 \subset \Sigma_2$:
\begin{equation}
    \Cap(\Sigma_1, \Sigma_2) = \inf\left\{\int_\Omega |\nabla u|^2 dV : u|_{\Sigma_1} = 1, u|_{\Sigma_2} = 0\right\}
\end{equation}
where $\Omega$ is the region between $\Sigma_1$ and $\Sigma_2$.
\end{definition}

\subsection{Capacity and Area}

\begin{theorem}[Capacity-Area Relation]\label{thm:cap-area}
For nested surfaces $\Sigma_1 \subset \Sigma_2$ with $\Sigma_2$ convex (in suitable sense):
\begin{equation}
    \Cap(\Sigma_1, \Sigma_2) \ge \frac{4\pi\Area(\Sigma_1)\Area(\Sigma_2)}{(\Area(\Sigma_2) - \Area(\Sigma_1))}
\end{equation}
\end{theorem}

\begin{proof}
The capacity is minimized by the harmonic function $u$ with given boundary values.

For spheres in Euclidean space: $\Cap(S_{r_1}, S_{r_2}) = 4\pi\frac{r_1 r_2}{r_2 - r_1}$.

With $A = 4\pi r^2$: $\Cap = \frac{4\pi\sqrt{A_1/4\pi}\sqrt{A_2/4\pi}}{(\sqrt{A_2/4\pi} - \sqrt{A_1/4\pi})}$.

This generalizes to convex surfaces by comparison.
\end{proof}

\textbf{Note:} This gives $\Cap \ge C/\Delta A$ where $\Delta A = A_2 - A_1$.

If $A_1 > A_2$ (area dominance fails), the formula breaks down!

%% ============================================================================
\section{Hardy-Type Inequalities}
%% ============================================================================

\subsection{Hardy Inequality Near MOTS}

\begin{theorem}[Hardy Near MOTS]\label{thm:hardy}
Let $d(x) = \dist(x, \Sigma^*)$ for $x$ inside the trapped region. Then:
\begin{equation}
    \int_\Omega \frac{u^2}{d^2} dV \le C\int_\Omega |\nabla u|^2 dV + C'\int_\Sigma u^2 dA
\end{equation}
for $u \in H^1(\Omega)$.
\end{theorem}

\subsection{Application}

\begin{corollary}[$\theta^+$ Hardy Bound]
\begin{equation}
    \int_\Omega \frac{(\theta^+)^2}{d^2} dV \le C\|\nabla\theta^+\|_{L^2(\Omega)}^2 + C'\|\theta^+\|_{L^2(\Sigma)}^2
\end{equation}
\end{corollary}

This bounds how fast $\theta^+$ can grow as we move away from MOTS.

%% ============================================================================
\section{The Critical Analysis}
%% ============================================================================

\subsection{What All Methods Show}

Every hard analysis approach we've tried reveals the same structure:

\begin{insight}
\textbf{The Fundamental Obstruction:}

For a trapped surface $\Sigma$ with enclosing MOTS $\Sigma^*$:
\begin{enumerate}
    \item The null expansion $\theta^+ < 0$ on $\Sigma$, $\theta^+ = 0$ on $\Sigma^*$
    \item The ``transition'' from $\Sigma$ to $\Sigma^*$ involves $\theta^+$ changing sign
    \item This transition has NO geometric constraint relating areas
\end{enumerate}

The quantity $\int_\Sigma (\theta^+)^2 dA$ can be bounded in terms of $\|\nabla\theta^+\|$, but:
\begin{itemize}
    \item $\nabla\theta^+$ depends on the ambient geometry $(M, g, k)$
    \item There's no universal bound on $\nabla\theta^+$ in terms of DEC alone
    \item Therefore, $\int(\theta^+)^2 dA$ is not constrained by energy conditions
\end{itemize}
\end{insight}

\subsection{What Would Be Needed}

To prove Penrose for all trapped surfaces, we would need ONE of:

\begin{enumerate}
    \item \textbf{Area Dominance:} $\Area(\Sigma) \le \Area(\Sigma^*)$
    
    This is NOT implied by DEC alone. Counterexamples may exist.
    
    \item \textbf{Universal $(\theta^+)^2$ Bound:} $\int_\Sigma(\theta^+)^2 dA \le f(\Area, M_{\ADM})$
    
    For some function $f$ such that $\sqrt{A/(16\pi)}(1 - f/(16\pi)) \le M_{\ADM}$.
    
    This would require new physics input beyond DEC.
    
    \item \textbf{Spacetime Development:} Use WCC to evolve initial data and apply Hawking area theorem
    
    This is Penrose's original approach but requires assuming WCC.
\end{enumerate}

%% ============================================================================
\section{Partial Results Summary}
%% ============================================================================

\begin{newresult}
\textbf{What Hard Analysis HAS Proven:}

\begin{enumerate}
    \item \textbf{MOTS Penrose:} $M_{\ADM} \ge \sqrt{A(\Sigma^*)/(16\pi)}$ for outermost MOTS
    
    \item \textbf{Near-MOTS Penrose:} For trapped $\Sigma$ with $d = \dist(\Sigma, \Sigma^*) \ll 1$:
    \begin{equation}
        M_{\ADM} \ge \sqrt{\frac{A(\Sigma)}{16\pi}}(1 - Cd^2\lambda_0^2)
    \end{equation}
    
    \item \textbf{Spherically Symmetric Penrose:} Full inequality proven in spherical symmetry
    
    \item \textbf{$L^2$ Bounds:} $\|\theta^+\|_{L^2(\Sigma)}^2 \le C(\|\nabla\theta^+\|_{L^2(\Omega)}^2 + \text{boundary terms})$
    
    \item \textbf{Spectral Gap:} Stability operator eigenvalue controls deformation behavior
    
    \item \textbf{Conformal Obstruction:} $H < 0$ for trapped on maximal slices $\Rightarrow$ conformal method gives wrong sign
\end{enumerate}
\end{newresult}

%% ============================================================================
\section{The Remaining Problem}
%% ============================================================================

\begin{finalstatus}
\textbf{PRECISE STATEMENT OF REMAINING PROBLEM:}

\textbf{Given:}
\begin{itemize}
    \item $(M, g, k)$ asymptotically flat, DEC
    \item $\Sigma^*$ outermost stable MOTS (proven: $M_{\ADM} \ge \sqrt{A^*/(16\pi)}$)
    \item $\Sigma$ trapped surface inside $\Sigma^*$
\end{itemize}

\textbf{To Prove:}
\begin{equation}
    M_{\ADM} \ge \sqrt{\frac{\Area(\Sigma)}{16\pi}}
\end{equation}

\textbf{Known:}
\begin{equation}
    M_{\ADM} \ge \sqrt{\frac{A^*}{16\pi}} \ge \sqrt{\frac{\Area(\Sigma)}{16\pi}} \iff \Area(\Sigma) \le A^*
\end{equation}

\textbf{The Gap:} Prove $\Area(\Sigma) \le \Area(\Sigma^*)$ OR find direct proof bypassing area comparison.

\textbf{Status:} OPEN. Hard analysis reveals no universal mechanism guaranteeing this.

\textbf{Physical Expectation:} Should be TRUE (trapped surfaces are ``inside'' black hole).

\textbf{Mathematical Status:} Unproven. May require:
\begin{itemize}
    \item New mathematical structures (beyond Sobolev/spectral/capacity)
    \item Additional physical input (WCC, new energy conditions)
    \item Spacetime methods (not just initial data)
\end{itemize}
\end{finalstatus}

%% ============================================================================
\section{Outlook: What Might Work}
%% ============================================================================

\subsection{Promising Directions}

\begin{enumerate}
    \item \textbf{Optimal Transport:} Use Kantorovich-Wasserstein distance to compare surfaces
    
    \item \textbf{Geometric Measure Theory:} Currents and varifolds for non-smooth analysis
    
    \item \textbf{Inverse Mean Curvature Flow with Surgery:} Handle $H \le 0$ regions
    
    \item \textbf{Spacetime Harmonic Functions:} Analyze wave equation rather than elliptic PDE
    
    \item \textbf{Holographic Methods:} Use AdS/CFT-inspired bounds
\end{enumerate}

\subsection{The Nuclear Option}

Construct a counterexample: initial data with $\Area(\Sigma) > 16\pi M_{\ADM}^2$ for some trapped $\Sigma$.

This would DISPROVE the conjecture for trapped surfaces (though it should still hold for MOTS).

\textbf{Physical Intuition:} This seems unlikely, as it would violate the cosmic censorship spirit.

%% ============================================================================
\section{Final Assessment}
%% ============================================================================

\begin{tcolorbox}[colback=red!5!white, colframe=red!75!black, title=\textbf{HARD ANALYSIS FINAL VERDICT}]

After exhaustive application of:
\begin{itemize}
    \item Sobolev embeddings and trace theorems
    \item Spectral theory for stability operators
    \item Maximum principles and comparison geometry
    \item Conformal methods and Lichnerowicz equation
    \item Capacity estimates and Hardy inequalities
    \item Weighted Sobolev spaces
    \item Constraint equation integrals
\end{itemize}

\textbf{CONCLUSION:}

The spacetime Penrose inequality for \textit{MOTS} is PROVEN.

The inequality for \textit{trapped surfaces} reduces to area dominance $A(\Sigma) \le A(\Sigma^*)$, which:
\begin{enumerate}
    \item Is NOT implied by DEC and standard PDE methods
    \item Has no known proof via hard analysis
    \item May require spacetime (evolutionary) arguments
    \item Or may be FALSE in general (counterexample possible)
\end{enumerate}

\textbf{The problem appears to require either:}
\begin{enumerate}
    \item Genuinely new mathematics
    \item Additional physical assumptions (beyond DEC)
    \item Spacetime methods assuming WCC
\end{enumerate}

Hard analysis has reached its limit for the trapped surface case.
\end{tcolorbox}

\end{document}
