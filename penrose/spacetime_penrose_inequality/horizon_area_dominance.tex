% THE HORIZON AREA DOMINANCE CONJECTURE
%
% The key missing piece: Does the event horizon cross-section
% always have area >= any trapped surface inside?

\documentclass[12pt]{article}
\usepackage{amsmath,amsthm,amssymb}
\usepackage{mathrsfs}
\newtheorem{theorem}{Theorem}
\newtheorem{lemma}{Lemma}
\newtheorem{proposition}{Proposition}
\newtheorem{corollary}{Corollary}
\newtheorem{conjecture}{Conjecture}
\newtheorem{remark}{Remark}
\newtheorem{definition}{Definition}
\newtheorem{problem}{Problem}
\newtheorem{claim}{Claim}
\newtheorem{principle}{Principle}
\newtheorem{insight}{Key Insight}
\newtheorem{counterexample}{Potential Counterexample}

\begin{document}

\title{The Horizon Area Dominance Conjecture}
\author{Mathematical Development}
\date{\today}
\maketitle

\section{The Conjecture}

\begin{conjecture}[Horizon Area Dominance - HAD]
Let $(V, g_{\mu\nu})$ be an asymptotically flat spacetime satisfying DEC with 
event horizon $\mathcal{E}$.

For any trapped surface $\Sigma$ on a Cauchy surface $M$, let $\mathcal{E}_M = \mathcal{E} \cap M$ 
be the event horizon cross-section on the same slice.

Then:
\[
A(\mathcal{E}_M) \ge A(\Sigma)
\]
\end{conjecture}

\section{Why This Matters}

If HAD is TRUE:
\[
M_{\mathrm{ADM}} \ge M_{\mathrm{final}} = \sqrt{\frac{A(\mathcal{E}_\infty)}{16\pi}} \ge \sqrt{\frac{A(\mathcal{E}_M)}{16\pi}} \ge \sqrt{\frac{A(\Sigma)}{16\pi}}
\]

This proves the Penrose inequality WITHOUT the favorable jump condition!

If HAD is FALSE:
The Penrose inequality itself might be false, or require modification.

\section{Analysis: When Could HAD Fail?}

\subsection{Geometric Setup}

\begin{center}
\textbf{[Diagram: Spacetime with event horizon and trapped surface]}
\end{center}

The event horizon $\mathcal{E}$ is a null hypersurface.
It intersects the initial slice $M$ in a 2-surface $\mathcal{E}_M$.
The trapped surface $\Sigma$ lies inside: $\Sigma \subset J^-(\mathcal{E}_M)$.

\subsection{Why Shouldn't $A(\mathcal{E}_M) \ge A(\Sigma)$?}

Geometrically, $\mathcal{E}_M$ "surrounds" $\Sigma$ (since $\Sigma$ is inside the horizon).

But area is not monotonic under inclusion! A smaller "container" can have 
less area than a "wrinkled" interior surface.

\subsection{A Heuristic Counterexample}

Consider:
\begin{itemize}
    \item A "smooth" event horizon cross-section $\mathcal{E}_M$ with area $A_\mathcal{E}$
    \item A "highly wrinkled" trapped surface $\Sigma$ inside with area $A_\Sigma >> A_\mathcal{E}$
\end{itemize}

Is this geometrically possible?

\textbf{Constraint}: The wrinkles must be consistent with the trapped condition 
($\theta^+ \le 0$, $\theta^- < 0$).

\subsection{The Trapped Condition Constrains Geometry}

For a surface to be trapped:
\begin{align}
    \theta^+ &= H + \tr_\Sigma k \le 0 \\
    \theta^- &= H - \tr_\Sigma k < 0
\end{align}

A "wrinkled" surface has large $|H|$ (high curvature).

For large positive $H$ (outward-curving wrinkles): $\theta^- = H - \tr k$ is 
positive unless $\tr k$ is even larger. This violates trapping!

For large negative $H$ (inward-curving wrinkles): $\theta^+ = H + \tr k$ is 
very negative. Still trapped, but the wrinkles are inward.

\textbf{Observation}: Inward wrinkles DON'T increase area—they decrease it!

So a trapped surface cannot have arbitrarily large area through "wrinkles."

\section{A More Careful Analysis}

\subsection{Area and Mean Curvature}

The isoperimetric inequality in curved space:
\[
A(\Sigma) \le C(g) \cdot V(\Omega)^{2/3}
\]

where $\Omega$ is the region bounded by $\Sigma$ and $C(g)$ depends on the metric.

In asymptotically flat space, the area is bounded by the "size" of the region.

\subsection{The Trapped Region Size}

The trapped surface $\Sigma$ lies in the trapped region $\mathcal{T}$, which is 
inside the event horizon.

The event horizon $\mathcal{E}_M$ bounds (a portion of) the trapped region.

\textbf{Key}: The trapped region has bounded "size" (contained in the horizon), 
so surfaces inside have bounded area.

\subsection{But Is the Bound $A(\mathcal{E}_M)$?}

Even if $A(\Sigma)$ is bounded, the bound might not be $A(\mathcal{E}_M)$ exactly.

\textbf{Example}: Consider a long thin "tube" horizon with small cross-section 
but large interior volume. A trapped surface inside could be a "blob" with 
area exceeding the tube's cross-section.

\section{Explicit Example: The Boosted Black Hole}

\subsection{Setup}

Take Schwarzschild and apply a Lorentz boost (change to a moving frame).

The metric in the boosted frame describes the same spacetime but with 
different-looking initial data.

\subsection{The Initial Data}

In the boosted frame, the initial data $(M, g, k)$ has:
\begin{itemize}
    \item $g$: Spatial metric (same intrinsic geometry)
    \item $k \ne 0$: Non-trivial extrinsic curvature due to boost
\end{itemize}

The horizon cross-section $\mathcal{E}_M$ is an ellipse (Lorentz-contracted sphere).

\subsection{Area Comparison}

Before boost: Horizon is round sphere, area $A_0 = 16\pi M^2$.

After boost: Horizon is ellipse with same PROPER area $A_0$ (area is Lorentz-invariant).

The trapped surfaces inside: Also have Lorentz-invariant area.

\textbf{Conclusion}: In Schwarzschild (even boosted), HAD holds trivially 
because all surfaces are round spheres!

\section{A More Interesting Example: Collapsing Shell}

\subsection{The Vaidya Spacetime}

The ingoing Vaidya metric:
\[
ds^2 = -\left(1 - \frac{2m(v)}{r}\right)dv^2 + 2\,dv\,dr + r^2 d\Omega^2
\]

describes a collapsing null shell with mass function $m(v)$.

\subsection{The Horizons}

The event horizon is NOT at $r = 2m(v)$ in general—it's teleological.

For an accreting black hole ($m' > 0$): Event horizon is INSIDE the apparent horizon.

For a "done collapsing" case: Event horizon approaches apparent horizon.

\subsection{Initial Data Analysis}

Take a slice at constant $v = v_0$.

The apparent horizon is at $r = 2m(v_0)$.

The event horizon is at some $r = r_{\mathcal{E}}(v_0) < 2m(v_0)$ (for accreting).

\textbf{Question}: Can a trapped surface at $r < r_{\mathcal{E}}$ have area 
$> 4\pi r_{\mathcal{E}}^2$?

In spherical symmetry, the area at radius $r$ is $4\pi r^2$, which is 
monotonically increasing with $r$.

So $A(\Sigma) = 4\pi r_\Sigma^2 < 4\pi r_{\mathcal{E}}^2 = A(\mathcal{E}_M)$ ✓

\textbf{In spherical symmetry, HAD holds!}

\section{Beyond Spherical Symmetry}

\subsection{The Real Challenge}

In non-spherical spacetimes, the horizon and trapped surfaces can have 
complicated shapes.

\textbf{Question}: Can a non-spherical trapped surface inside a non-spherical 
horizon have larger area?

\subsection{A Potential Counterexample Scenario}

Consider:
\begin{enumerate}
    \item A black hole with a "pinched" horizon (like a peanut shape)
    \item The pinch creates a narrow "waist" with small cross-sectional area
    \item Inside, near the center, there's a more spherical trapped surface
    \item The interior surface might have larger area than the waist
\end{enumerate}

\textbf{BUT}: The "waist" is not the entire horizon cross-section!
The full cross-section includes both lobes.

\subsection{The Topology Question}

If the horizon has multiple components (e.g., two merging black holes before 
they touch), each component is separate.

The total area is the SUM of component areas.

A trapped surface inside ONE component should have area $\le$ that component's area.

\section{A Theorem Attempt}

\begin{theorem}[Weak HAD - Conjectured]
Let $\Sigma$ be a trapped surface in spacetime $(V, g)$ satisfying DEC.
Let $\mathcal{E}$ be the event horizon.

If $\mathcal{E}$ is connected and $\Sigma \subset J^-(\mathcal{E})$, then:
\[
A(\mathcal{E}_M) \ge A(\Sigma) - O(\epsilon)
\]
where $\epsilon$ depends on how "non-spherical" the configuration is.
\end{theorem}

\textbf{Proof idea}:
\begin{enumerate}
    \item Use the isoperimetric inequality to bound $A(\Sigma)$ by the "size" 
    of the region.
    \item The region is bounded by $\mathcal{E}_M$, so its size is controlled 
    by $A(\mathcal{E}_M)$.
    \item For nearly spherical regions, the isoperimetric constant is close to 
    the Euclidean value.
\end{enumerate}

\section{The Isoperimetric Approach}

\subsection{The Classical Result}

In $\mathbb{R}^3$: For a region $\Omega$ with boundary $\partial\Omega$:
\[
A(\partial\Omega) \ge (36\pi)^{1/3} V(\Omega)^{2/3}
\]

Equality for spheres.

\subsection{The Curved Space Version}

In a Riemannian 3-manifold $(M, g)$ with non-negative Ricci curvature:
\[
A(\Sigma) \ge c(g) \cdot V(\Omega)^{2/3}
\]

where $c(g)$ is a constant depending on the metric.

\subsection{Application to Trapped Surfaces}

Let $\Omega$ be the region bounded by a trapped surface $\Sigma$ (inside the horizon).

The volume $V(\Omega)$ is bounded by... what?

The horizon $\mathcal{E}_M$ doesn't bound a volume directly (it's a 2D surface 
in 3D space).

\textbf{Idea}: Compare $\Sigma$ and $\mathcal{E}_M$ as boundaries of nested regions.

\subsection{Nested Regions}

Let:
\begin{itemize}
    \item $\Omega_1$ = region bounded by $\Sigma$
    \item $\Omega_2$ = region bounded by $\mathcal{E}_M$
\end{itemize}

Since $\Sigma$ is inside $\mathcal{E}_M$: $\Omega_1 \subset \Omega_2$.

By the isoperimetric inequality:
\begin{align}
    A(\Sigma) &\ge c \cdot V(\Omega_1)^{2/3} \\
    A(\mathcal{E}_M) &\ge c \cdot V(\Omega_2)^{2/3}
\end{align}

Since $V(\Omega_1) \le V(\Omega_2)$:
\[
V(\Omega_1)^{2/3} \le V(\Omega_2)^{2/3}
\]

But this doesn't directly compare $A(\Sigma)$ and $A(\mathcal{E}_M)$!

\section{A Different Approach: The Hawking Mass Bound}

\subsection{The Idea}

Use the positive mass theorem applied to the region between $\Sigma$ and $\mathcal{E}_M$.

\subsection{The Setup}

Consider the "shell" region $\Omega = \mathcal{E}_M - \Sigma$ (between the surfaces).

This region has two boundaries: $\Sigma$ and $\mathcal{E}_M$.

\subsection{The Mass in the Shell}

The total mass in the shell is:
\[
M_{\mathrm{shell}} = \int_\Omega (\mu - |J|) \, dV \ge 0 \quad \text{by DEC}
\]

\subsection{Relating to Area}

By positive mass-type arguments:
\[
M_H[\mathcal{E}_M] - M_H[\Sigma] \ge M_{\mathrm{shell}} \ge 0
\]

This gives: $M_H[\mathcal{E}_M] \ge M_H[\Sigma]$.

\textbf{But}: We want $A(\mathcal{E}_M) \ge A(\Sigma)$, not a Hawking mass comparison!

And $M_H$ is NOT monotonic with area in general.

\section{Conclusion}

The Horizon Area Dominance Conjecture remains OPEN.

\textbf{Evidence FOR}:
\begin{itemize}
    \item Holds in spherical symmetry
    \item Trapped condition restricts "wrinkling"
    \item Isoperimetric considerations suggest bounded area
\end{itemize}

\textbf{Evidence AGAINST}:
\begin{itemize}
    \item No general proof exists
    \item The favorable jump condition failure shows something subtle happens 
    with area comparisons
    \item The modified Penrose inequality might be the correct statement
\end{itemize}

\textbf{Next step}: Either prove HAD or construct an explicit counterexample.

\end{document}
