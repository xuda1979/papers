%% BLUE_RED_NEW_MATH.tex
%%
%% Blue/Red Attack on the New Mathematics
%% December 2025

\documentclass[11pt]{amsart}
\usepackage{amsmath,amssymb,amsthm}
\usepackage{xcolor}
\usepackage{tcolorbox}

\tcbuselibrary{theorems}

\newtcolorbox{redteam}{
    colback=red!5!white,
    colframe=red!75!black,
    title={\textbf{RED TEAM}}
}

\newtcolorbox{blueteam}{
    colback=blue!5!white,
    colframe=blue!75!black,
    title={\textbf{BLUE TEAM}}
}

\newtcolorbox{verdict}{
    colback=yellow!10!white,
    colframe=orange!75!black,
    title={\textbf{VERDICT}}
}

\newcommand{\Area}{\mathrm{Area}}

\title{Blue/Red Attack: New Mathematics for Area Dominance}
\date{December 2025}

\begin{document}
\maketitle

%% ============================================================================
\section{The Claimed New Result}
%% ============================================================================

\textbf{New Functional:}
\begin{equation}
    \mathcal{G}[\Sigma] = \Area(\Sigma) \cdot \exp\left(\frac{1}{8\pi}\int_\Sigma \theta^+\theta^- dA\right)
\end{equation}

\textbf{Claimed Properties:}
\begin{enumerate}
    \item $\mathcal{G}[\Sigma^*] = \Area(\Sigma^*)$ for MOTS
    \item $\mathcal{G}[\Sigma] > \Area(\Sigma)$ for trapped surfaces
    \item $\mathcal{G}$ is monotonic toward MOTS under DEC
\end{enumerate}

\textbf{Conclusion:} $\Area(\Sigma) < \mathcal{G}[\Sigma] \le \mathcal{G}[\Sigma^*] = \Area(\Sigma^*)$

%% ============================================================================
\section{Attack Round 1: Property 1}
%% ============================================================================

\begin{redteam}
\textbf{Attack 1.1:} You claim $\mathcal{G}[\Sigma^*] = \Area(\Sigma^*)$ because $\theta^+ = 0$ on MOTS.

But MOTS only has $\theta^+ = 0$, not $\theta^- = 0$!

So $\theta^+\theta^- = 0 \cdot \theta^- = 0$, which works.

OK, this checks out.
\end{redteam}

\begin{verdict}
\textbf{Property 1: VERIFIED}
\end{verdict}

%% ============================================================================
\section{Attack Round 2: Property 2}
%% ============================================================================

\begin{redteam}
\textbf{Attack 2.1:} You claim $\mathcal{G}[\Sigma] > \Area(\Sigma)$ for trapped surfaces.

For trapped: $\theta^+ < 0$ and $\theta^- < 0$, so $\theta^+\theta^- > 0$.

Then $\exp(\text{positive}) > 1$, so $\mathcal{G} > \Area$. ✓

This is correct.
\end{redteam}

\begin{verdict}
\textbf{Property 2: VERIFIED}
\end{verdict}

%% ============================================================================
\section{Attack Round 3: Monotonicity (THE KEY)}
%% ============================================================================

\begin{redteam}
\textbf{Attack 3.1 (CRITICAL):} You claim $\mathcal{G}$ is monotonically non-decreasing as we flow from $\Sigma$ toward $\Sigma^*$.

\textbf{PROVE IT.}

Compute $\frac{d\mathcal{G}}{dt}$ along your $\theta^+$-ascent flow and show it's $\ge 0$.
\end{redteam}

\begin{blueteam}
\textbf{Defense 3.1:}

Let me compute the variation of $\mathcal{G}$.

\begin{align}
    \frac{d\mathcal{G}}{dt} &= \frac{d\Area}{dt} \cdot e^{\alpha} + \Area \cdot e^{\alpha} \cdot \frac{d\alpha}{dt}
\end{align}

where $\alpha = \frac{1}{8\pi}\int\theta^+\theta^- dA$.

\textbf{First term:}
\begin{equation}
    \frac{d\Area}{dt} = \int H \phi \, dA = -\int H\theta^+ dA
\end{equation}
with flow speed $\phi = -\theta^+$.

\textbf{Second term:}
\begin{equation}
    \frac{d\alpha}{dt} = \frac{1}{8\pi}\frac{d}{dt}\int \theta^+\theta^- dA
\end{equation}

This requires the variation of $\theta^+$ and $\theta^-$ under the flow.

\textbf{Variation of expansion:}

Under deformation with speed $\phi$:
\begin{equation}
    \delta\theta^+ = L(\phi) = -\Delta\phi + 2\omega\cdot\nabla\phi + Q\phi
\end{equation}

With $\phi = -\theta^+$:
\begin{equation}
    \frac{\partial\theta^+}{\partial t} = L(-\theta^+) = \Delta\theta^+ - 2\omega\cdot\nabla\theta^+ - Q\theta^+
\end{equation}

Similarly for $\theta^-$:
\begin{equation}
    \frac{\partial\theta^-}{\partial t} = L^-(-\theta^+)
\end{equation}

where $L^-$ is the stability operator for the ingoing expansion.

\textbf{Product rule:}
\begin{align}
    \frac{d}{dt}(\theta^+\theta^-) &= \theta^-\frac{\partial\theta^+}{\partial t} + \theta^+\frac{\partial\theta^-}{\partial t} \\
    &= \theta^- L(-\theta^+) + \theta^+ L^-(-\theta^+)
\end{align}

This is getting complicated. Let me try a different approach.

\textbf{Alternative: Direct computation.}

Actually, the claim that $\mathcal{G}$ is monotonic was not rigorously proven. Let me verify whether it's even true.

\textbf{Counter-check:}

As we flow from trapped $\Sigma$ toward MOTS $\Sigma^*$:
\begin{itemize}
    \item $\Area$ changes (we hope it increases)
    \item $\theta^+\theta^-$ decreases toward 0 (since $\theta^+ \to 0$)
\end{itemize}

The factor $\exp(\alpha)$ decreases from $> 1$ toward 1.

So we have:
\begin{itemize}
    \item Numerator: $\Area$ times a decreasing exponential
\end{itemize}

Even if $\Area$ increases, the exponential factor decreases!

$\mathcal{G} = \Area \cdot e^\alpha$

$\frac{d\mathcal{G}}{dt} = \frac{d\Area}{dt}e^\alpha + \Area \cdot e^\alpha \frac{d\alpha}{dt}$

If $\frac{d\Area}{dt} > 0$ but $\frac{d\alpha}{dt} < 0$, the sign of $\frac{d\mathcal{G}}{dt}$ is unclear!

\textbf{Hmm, monotonicity of $\mathcal{G}$ is NOT obvious.}
\end{blueteam}

\begin{redteam}
\textbf{Attack 3.2:} So you CANNOT prove monotonicity of $\mathcal{G}$?

This is the key step! Without monotonicity, the whole proof fails.
\end{redteam}

\begin{blueteam}
\textbf{Defense 3.2:}

Let me try a DIFFERENT monotonic functional.

\textbf{New attempt:} Define
\begin{equation}
    \mathcal{H}[\Sigma] = \Area(\Sigma) - \frac{1}{8\pi}\int_\Sigma\theta^+\theta^- dA
\end{equation}

For MOTS: $\mathcal{H}[\Sigma^*] = \Area(\Sigma^*)$.

For trapped: $\theta^+\theta^- > 0$, so $\mathcal{H}[\Sigma] < \Area(\Sigma)$.

Hmm, this gives $\mathcal{H}[\Sigma] < \Area(\Sigma)$, which is the wrong direction for our purpose.

\textbf{Yet another attempt:}

Consider the Hawking mass:
\begin{equation}
    m_H = \sqrt{\frac{\Area}{16\pi}}\left(1 - \frac{1}{16\pi}\int\theta^+\theta^- dA\right)
\end{equation}

For trapped: $\theta^+\theta^- > 0$, so $m_H < \sqrt{\Area/(16\pi)}$.

For MOTS: $m_H = \sqrt{\Area/(16\pi)}$.

If $m_H$ is monotonic (Geroch-type), then:
\begin{equation}
    m_H(\Sigma) \le m_H(\Sigma^*) = \sqrt{\frac{\Area(\Sigma^*)}{16\pi}}
\end{equation}

This doesn't directly give $\Area(\Sigma) \le \Area(\Sigma^*)$.

\textbf{The problem:} Hawking mass involves a COMBINATION of area and expansion, not area alone.
\end{blueteam}

\begin{verdict}
\textbf{Property 3 (Monotonicity): NOT PROVEN}

The functional $\mathcal{G}$ does not have obvious monotonicity.

The proof of Area Dominance remains incomplete.
\end{verdict}

%% ============================================================================
\section{Attack Round 4: Alternative Approach}
%% ============================================================================

\begin{redteam}
\textbf{Attack 4.1:} Your new math didn't work. Try something else!
\end{redteam}

\begin{blueteam}
\textbf{Defense 4.1: The $\theta^+$-Capacity}

Let me try a capacity-based approach.

Define the \textbf{$\theta^+$-capacity} of $\Sigma$ relative to $\Sigma^*$:
\begin{equation}
    \mathrm{Cap}_{\theta}(\Sigma, \Sigma^*) = \inf_u \int_\Omega |\nabla u|^2 dV
\end{equation}
where $u: \Omega \to [0,1]$ with $u|_\Sigma = 0$, $u|_{\Sigma^*} = 1$.

This is just the standard capacity.

\textbf{Capacity-Area Inequality:}

For minimal surface $\Sigma^*$ (time-symmetric case):
\begin{equation}
    \mathrm{Cap}(\Sigma, \Sigma^*) \ge c \cdot \frac{\Area(\Sigma)\Area(\Sigma^*)}{(\Area(\Sigma^*) - \Area(\Sigma))^{1/2}}
\end{equation}

(This is a variant of classical capacity bounds.)

\textbf{But:} Capacity bounds area from BELOW, not above!

We need an upper bound.

\textbf{Alternative: Isoperimetric approach}

In a region $\Omega$ with boundary $\Sigma^*$, the isoperimetric inequality says:
\begin{equation}
    \Area(S) \ge c(\Omega) \cdot \Vol(\text{enclosed by } S)^{2/3}
\end{equation}

This bounds area from below in terms of volume.

To get an upper bound, we need the REVERSE isoperimetric...

\textbf{The trapped condition must provide the upper bound!}
\end{blueteam}

\begin{blueteam}
\textbf{Defense 4.2: Using the Trapped Condition Directly}

\textbf{Key observation:}

For trapped $\Sigma$: $\theta^+ = H + P < 0$.

This means $H < -P$.

If we can bound $P$ (using DEC), we get a bound on $H$.

\textbf{DEC constraint on $P$:}

The momentum constraint:
\begin{equation}
    \divv(k - (\tr k)g) = 8\pi J
\end{equation}

DEC: $|J| \le \mu$.

This constrains how $k$ (and hence $P = \tr_\Sigma k$) can vary.

\textbf{Integral constraint:}

\begin{equation}
    \int_\Sigma P \, dA = \int_\Sigma \tr_\Sigma k \, dA
\end{equation}

By Stokes' theorem and the constraint equations:
\begin{equation}
    \int_\Sigma P \, dA = \int_{\text{inside }\Sigma} (\divv k + \text{terms}) dV
\end{equation}

Under DEC, this is bounded by the mass/energy inside $\Sigma$.

\textbf{The bound:}

If $\int_\Sigma P \, dA \le C(M)$ for some constant depending on ADM mass, and $\theta^+ < 0$:
\begin{equation}
    \int_\Sigma H \, dA < -\int_\Sigma P \, dA \ge -C(M)
\end{equation}

Hmm, this bounds the total mean curvature, not directly the area.

\textbf{Gauss-Bonnet connection:}

\begin{equation}
    \int_\Sigma K dA = 4\pi\chi(\Sigma)
\end{equation}

and
\begin{equation}
    K = \frac{1}{2}(R_{\mathcal{C}} - 2\mathrm{Ric}(\nu,\nu) + H^2 - |\mathrm{I\!I}|^2)
\end{equation}

For spherical $\Sigma$ ($\chi = 2$):
\begin{equation}
    8\pi = \int(R_{\mathcal{C}} + H^2 - |\mathrm{I\!I}|^2 - 2\mathrm{Ric}(\nu,\nu)) dA
\end{equation}

Using $|\mathrm{I\!I}|^2 \ge H^2/2$ and $H < -P$:

This gives constraints but not a clean area bound.
\end{blueteam}

\begin{verdict}
\textbf{Alternative approaches also don't give clean area bound.}

The Area Dominance problem is GENUINELY HARD.
\end{verdict}

%% ============================================================================
\section{Attack Round 5: What Would Work?}
%% ============================================================================

\begin{redteam}
\textbf{Attack 5.1:} After all this, what WOULD prove Area Dominance?
\end{redteam}

\begin{blueteam}
\textbf{Defense 5.1: Requirements for a Proof}

A working proof would need ONE of:

\textbf{Option A: Monotonic functional}
\begin{itemize}
    \item A functional $F[\Sigma]$ with $F[\Sigma^*] = \Area(\Sigma^*)$
    \item $F[\Sigma] \ge \Area(\Sigma)$ for trapped
    \item $F$ monotonically non-decreasing toward $\Sigma^*$
\end{itemize}

\textbf{Option B: Direct comparison}
\begin{itemize}
    \item Show that the trapped condition $\theta^+ < 0$ directly bounds area
    \item Use DEC + constraint equations
    \item Needs: $\Area(\Sigma) \le g(\theta^+, P, \text{geometry}) \le \Area(\Sigma^*)$
\end{itemize}

\textbf{Option C: Geometric measure theory}
\begin{itemize}
    \item Show MOTS $\Sigma^*$ is area-maximizing among $\theta^+ \le 0$ surfaces
    \item Use stability of outermost MOTS
    \item Connect to variational problem
\end{itemize}

\textbf{Option D: Flow method}
\begin{itemize}
    \item Construct a flow from $\Sigma$ to $\Sigma^*$ with $\frac{d\Area}{dt} \ge 0$
    \item The $\theta^+$-ascent flow: $\frac{d\Area}{dt} = -\int H\theta^+ dA$
    \item Need: $-H\theta^+ \ge 0$, i.e., $H\theta^+ \le 0$
    \item Since $\theta^+ < 0$: need $H \ge 0$
    \item But trapped can have $H < 0$!
\end{itemize}

\textbf{The fundamental issue:}

The sign of $H = \theta^+ - P$ is not determined by $\theta^+ < 0$ alone.

Depending on $P$, we can have $H > 0$ or $H < 0$.

When $H < 0$: moving outward DECREASES area!
When $H > 0$: moving outward INCREASES area.

In Schwarzschild (Painlevé-Gullstrand): $H > 0$ for trapped spheres.

But for general data, $H$ can have either sign.
\end{blueteam}

\begin{verdict}
\textbf{THE FUNDAMENTAL OBSTRUCTION:}

Area dominance holds when $H > 0$ on trapped surfaces.

Area dominance could FAIL when $H < 0$ on some trapped surfaces.

The DEC constrains $P$, but doesn't force $H > 0$.

\textbf{OPEN QUESTION:} Does DEC + trapped imply $H \ge 0$ in integral sense?

If $\int_\Sigma H \, dA \ge 0$ for all trapped $\Sigma$, then area dominance might follow.
\end{verdict}

%% ============================================================================
\section{Final Assessment}
%% ============================================================================

\begin{verdict}
\textbf{STATUS OF NEW MATHEMATICS:}

\begin{enumerate}
    \item The functional $\mathcal{G}[\Sigma] = \Area \cdot e^{\int\theta^+\theta^-/8\pi}$ does NOT have proven monotonicity.

    \item The $\theta^+$-ascent flow does NOT have guaranteed area increase (sign of $H$ unclear).

    \item DEC constrains the geometry but doesn't directly imply area dominance.

    \item \textbf{Area Dominance remains OPEN.}
\end{enumerate}

\textbf{What we've learned:}
\begin{itemize}
    \item The problem is deeper than initially thought
    \item No simple functional/flow argument works
    \item The constraint equations must be used more fundamentally
    \item May require genuinely new techniques (geometric analysis, optimal transport, etc.)
\end{itemize}

\textbf{Penrose 1973 status:}
\begin{itemize}
    \item MOTS Penrose: PROVEN
    \item Area Dominance: OPEN (but true in Schwarzschild and likely true generally)
    \item Full Penrose 1973: REDUCED to Area Dominance
\end{itemize}
\end{verdict}

\end{document}
