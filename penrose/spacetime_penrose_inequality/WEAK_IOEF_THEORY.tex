%% WEAK_IOEF_THEORY.tex
%%
%% WEAK SOLUTIONS FOR INVERSE OUTGOING EXPANSION FLOW
%%
%% Developing the Huisken-Ilmanen style weak solution theory
%% for the spacetime generalization of IMCF.
%%
%% December 2025

\documentclass[11pt]{amsart}
\usepackage{amsmath,amssymb,amsthm}
\usepackage{tcolorbox}

\tcbuselibrary{theorems}

\newtcolorbox{maintheorem}{
    colback=green!5!white,
    colframe=green!50!black,
    title={\textbf{MAIN THEOREM}}
}

\newtcolorbox{keylemma}{
    colback=blue!5!white,
    colframe=blue!75!black,
    title={\textbf{KEY LEMMA}}
}

\newtcolorbox{proofstep}{
    colback=gray!5!white,
    colframe=gray!50!black,
    title={\textbf{PROOF STEP}}
}

\newtcolorbox{insight}{
    colback=purple!5!white,
    colframe=purple!75!black,
    title={\textbf{INSIGHT}}
}

\newtheorem{theorem}{Theorem}[section]
\newtheorem{lemma}[theorem]{Lemma}
\newtheorem{proposition}[theorem]{Proposition}
\newtheorem{corollary}[theorem]{Corollary}
\theoremstyle{definition}
\newtheorem{definition}[theorem]{Definition}
\newtheorem{remark}[theorem]{Remark}

\newcommand{\Area}{\mathrm{Area}}
\newcommand{\Vol}{\mathrm{Vol}}
\newcommand{\divv}{\mathrm{div}}
\DeclareMathOperator{\tr}{tr}

\title{Weak Solutions for Spacetime Inverse Mean Curvature Flow}
\author{December 2025}

\begin{document}
\maketitle

\begin{abstract}
We develop a weak solution theory for the inverse outgoing expansion flow 
(IOEF), the spacetime generalization of inverse mean curvature flow. The 
key technical innovation is handling the extrinsic curvature terms that 
appear in $\theta^+ = H + P$.
\end{abstract}

%% ============================================================================
\section{The Level Set Equation}
%% ============================================================================

\begin{definition}[Classical IOEF]
For surfaces $\Sigma_t$ with $\theta^+ > 0$:
\begin{equation}
    \frac{\partial F}{\partial t} = \frac{\nu}{\theta^+}
\end{equation}

In level set form with $\Sigma_t = \{u = t\}$:
\begin{equation}
    \theta^+[\{u = t\}] = |\nabla u|
\end{equation}

Expanding $\theta^+ = H + P$:
\begin{equation}
    \divv\left(\frac{\nabla u}{|\nabla u|}\right) + k_{ij}\frac{u^i u^j}{|\nabla u|^2} 
    - \tr k \cdot \frac{(\nabla u \cdot \nu)}{|\nabla u|} = |\nabla u|
\end{equation}

where $\nu = \nabla u/|\nabla u|$ and $P = k(\nu,\nu) - \tr k$.
\end{definition}

\begin{remark}
For $k = 0$, this reduces to the standard IMCF level set equation:
\begin{equation}
    \divv\left(\frac{\nabla u}{|\nabla u|}\right) = |\nabla u|
\end{equation}
which is the equation studied by Huisken-Ilmanen.
\end{remark}

%% ============================================================================
\section{Elliptic Regularization}
%% ============================================================================

Following Huisken-Ilmanen, we regularize to get existence.

\begin{definition}[Regularized Equation]
For $\epsilon > 0$, consider:
\begin{equation}
    \divv\left(\frac{\nabla u_\epsilon}{\sqrt{|\nabla u_\epsilon|^2 + \epsilon^2}}\right) 
    + P\left[\frac{\nabla u_\epsilon}{|\nabla u_\epsilon|}\right]
    = \sqrt{|\nabla u_\epsilon|^2 + \epsilon^2}
\end{equation}

where $P[\nu] = k(\nu,\nu) - \tr k$ evaluated with $\nu = \nabla u/|\nabla u|$.
\end{definition}

\begin{lemma}[Existence of Regularized Solutions]
For bounded domains with appropriate boundary conditions, the regularized 
equation has smooth solutions $u_\epsilon$.
\end{lemma}

\begin{proof}[Proof Sketch]
The operator is uniformly elliptic for $\epsilon > 0$:
\begin{equation}
    L_\epsilon[u] = \divv(A_\epsilon(\nabla u)) + P[\nabla u] - |\nabla u|_\epsilon
\end{equation}

where $A_\epsilon(\xi) = \xi/\sqrt{|\xi|^2 + \epsilon^2}$ and 
$|\xi|_\epsilon = \sqrt{|\xi|^2 + \epsilon^2}$.

Standard elliptic theory (Schauder estimates, fixed point) gives existence.
\end{proof}

%% ============================================================================
\section{A Priori Estimates}
%% ============================================================================

\begin{keylemma}
\textbf{Gradient Bound}

If $u_\epsilon$ solves the regularized equation with boundary data on 
$\partial\Omega$ and $\Sigma_0 = \{u_\epsilon = 0\}$, then:
\begin{equation}
    |\nabla u_\epsilon| \le C(g, k, \Sigma_0, \partial\Omega)
\end{equation}

uniformly in $\epsilon$.
\end{keylemma}

\begin{proof}[Proof Sketch]
The maximum principle applied to $|\nabla u_\epsilon|^2$ gives:

At a maximum of $|\nabla u_\epsilon|$, we have:
\begin{equation}
    0 \le \Delta|\nabla u_\epsilon|^2 - 2|\nabla^2 u_\epsilon|^2 + \text{curvature terms}
\end{equation}

The key is that the $P$ term (involving $k$) contributes bounded lower-order 
terms. Specifically:
\begin{equation}
    |P[\nu]| \le |k|_{L^\infty}
\end{equation}

This gives an upper bound on $|\nabla u_\epsilon|$ depending on geometry.
\end{proof}

\begin{lemma}[Lower Bound on Gradient]
In regions where $\theta^+ > \delta > 0$:
\begin{equation}
    |\nabla u_\epsilon| \ge c(\delta) > 0
\end{equation}
\end{lemma}

\begin{proof}
From the equation:
\begin{equation}
    \theta^+ = \sqrt{|\nabla u_\epsilon|^2 + \epsilon^2} \ge |\nabla u_\epsilon|
\end{equation}

But also $\theta^+ > \delta$, so the regularized gradient is bounded below.
\end{proof}

%% ============================================================================
\section{Passage to Limit}
%% ============================================================================

\begin{theorem}[Convergence as $\epsilon \to 0$]
The solutions $u_\epsilon$ converge (along a subsequence) to a limit $u$ 
that is a weak solution of IOEF.
\end{theorem}

\begin{proof}[Proof Sketch]
By gradient bounds, $\{u_\epsilon\}$ is equi-Lipschitz.

By Arzelà-Ascoli, a subsequence converges uniformly to some $u$.

The weak formulation passes to the limit.
\end{proof}

%% ============================================================================
\section{The Weak Formulation}
%% ============================================================================

\begin{definition}[Weak Solution of IOEF]
A locally Lipschitz function $u: M \to \mathbb{R}$ is a weak solution if:

\textbf{Subsolution condition:} For any smooth $\phi$ with $u - \phi$ having 
a local maximum at $x$:
\begin{equation}
    |\nabla\phi|(x) \le \theta^+[\nabla\phi](x)
\end{equation}

\textbf{Supersolution condition:} For any smooth $\phi$ with $u - \phi$ having 
a local minimum at $x$ where $\nabla\phi(x) \neq 0$:
\begin{equation}
    |\nabla\phi|(x) \ge \theta^+[\nabla\phi](x)
\end{equation}
\end{definition}

\begin{remark}
This is the viscosity solution concept adapted to IOEF.

Where the gradient vanishes, we have $\theta^+ = 0$, which corresponds to 
a MOTS (marginally outer trapped surface).
\end{remark}

%% ============================================================================
\section{Monotonicity Formula}
%% ============================================================================

The key to Penrose is a monotonicity formula for mass.

\begin{definition}[Generalized Hawking Mass for Level Sets]
For a level set $\Sigma_t = \{u = t\}$:
\begin{equation}
    m_H(t) = \sqrt{\frac{A(t)}{16\pi}}\left(1 - \frac{1}{16\pi}\int_{\Sigma_t} 
    \theta^+\theta^- \, dA\right)
\end{equation}
\end{definition}

\begin{keylemma}
\textbf{Monotonicity of Generalized Hawking Mass}

For weak solutions of IOEF, in a suitable generalized sense:
\begin{equation}
    \frac{d}{dt}m_H(t) \ge 0
\end{equation}

provided DEC holds.
\end{keylemma}

\begin{proofstep}
\textbf{Formal Calculation (Smooth Case)}

Let $A = A(t)$ and $Q = \int_{\Sigma_t} \theta^+\theta^- \, dA$.

Under IOEF with speed $v = 1/\theta^+$:
\begin{equation}
    \dot{A} = \int_{\Sigma_t} Hv \, dA = \int_{\Sigma_t} \frac{H}{\theta^+} \, dA
\end{equation}

For $Q$, we need to compute how $\theta^+\theta^-$ evolves.
\end{proofstep}

\begin{proofstep}
\textbf{Evolution of $\theta^+$}

Under normal motion with speed $v$:
\begin{equation}
    \frac{\partial\theta^+}{\partial t} = -\Delta_\Sigma v - v(|\chi^+|^2 + 8\pi T_{++})
    + v\theta^+\kappa
\end{equation}

where:
\begin{itemize}
    \item $\chi^+$ is the null shear
    \item $T_{++} = T_{\mu\nu}\ell^+_\mu\ell^+_\nu \ge 0$ by DEC
    \item $\kappa$ involves connection terms
\end{itemize}

With $v = 1/\theta^+$:
\begin{equation}
    \frac{\partial\theta^+}{\partial t} = -\frac{1}{\theta^+}\Delta_\Sigma\left(\frac{1}{\theta^+}\right)
    - \frac{|\chi^+|^2 + 8\pi T_{++}}{\theta^+} + \kappa
\end{equation}
\end{proofstep}

\begin{proofstep}
\textbf{The Sign Analysis}

The term $-\frac{|\chi^+|^2 + 8\pi T_{++}}{\theta^+}$ has sign:
\begin{itemize}
    \item Numerator: $|\chi^+|^2 + 8\pi T_{++} \ge 0$ always
    \item Denominator: $\theta^+ > 0$ in the region of flow
\end{itemize}

So this term is $\le 0$, meaning $\theta^+$ tends to DECREASE.

Similarly for $\theta^-$ evolution.

The product $\theta^+\theta^-$ evolution involves these terms and their 
interaction.
\end{proofstep}

%% ============================================================================
\section{The Key Technical Issue}
%% ============================================================================

\begin{insight}
\textbf{Why Direct Monotonicity is Hard}

For IMCF (Huisken-Ilmanen), the Hawking mass:
\begin{equation}
    m_H = \sqrt{\frac{A}{16\pi}}\left(1 - \frac{1}{16\pi}\int H^2 \, dA\right)
\end{equation}

has monotonicity because $H^2$ terms and area growth combine favorably.

For spacetime Hawking mass with $\theta^+\theta^-$:
\begin{itemize}
    \item The evolution of $\theta^+\theta^-$ involves cross-terms
    \item The signs don't obviously combine to give monotonicity
    \item The $k$-dependent terms add complications
\end{itemize}
\end{insight}

\begin{proposition}[What We Need]
To prove Penrose via IOEF, we need ONE of:
\begin{enumerate}
    \item Direct monotonicity of $m_H^{ST}$ under IOEF
    \item A different mass functional that IS monotone
    \item A comparison argument to Schwarzschild that doesn't need monotonicity
\end{enumerate}
\end{proposition}

%% ============================================================================
\section{Alternative: Fokker-Planck Approach}
%% ============================================================================

\begin{insight}
\textbf{Perelman's Entropy via Fokker-Planck}

Perelman introduced the $\mathcal{W}$-entropy:
\begin{equation}
    \mathcal{W}(g, f, \tau) = \int \left[\tau(|\nabla f|^2 + R) + f - n\right]
    (4\pi\tau)^{-n/2}e^{-f} \, dV
\end{equation}

This is monotone under coupled Ricci flow + backwards heat equation.

\textbf{Idea:} Define an analogous entropy for initial data that:
\begin{itemize}
    \item Incorporates both $g$ and $k$
    \item Is monotone under an appropriate flow
    \item Achieves minimum at Schwarzschild
\end{itemize}
\end{insight}

\begin{definition}[Spacetime Entropy (Ansatz)]
Define:
\begin{equation}
    \mathcal{S}[g, k, f] = \int_M \left[|\nabla f|^2 + \alpha R + \beta|k|^2 
    + \gamma(\tr k)^2\right] e^{-f} \, dV_g
\end{equation}

with constants $\alpha, \beta, \gamma$ to be determined.
\end{definition}

\begin{proposition}[DEC Constraint on Entropy]
For DEC: $R - |k|^2 + (\tr k)^2 \ge 0$.

This suggests choosing $\alpha = 1$, $\beta = -1$, $\gamma = 1$ gives:
\begin{equation}
    \mathcal{S} = \int \left[|\nabla f|^2 + (R - |k|^2 + (\tr k)^2)\right] e^{-f} \, dV
    \ge \int |\nabla f|^2 e^{-f} \, dV
\end{equation}

The entropy is bounded below under DEC.
\end{proposition}

%% ============================================================================
\section{The Coupled Flow}
%% ============================================================================

\begin{definition}[Entropic Flow for Initial Data]
Consider the flow:
\begin{align}
    \frac{\partial g}{\partial t} &= -2\text{Ric} + 2\nabla^2 f + \text{constraint correction}\\
    \frac{\partial k}{\partial t} &= \Delta k + \text{lower order} + \text{constraint correction}\\
    \frac{\partial f}{\partial t} &= -\Delta f + |\nabla f|^2 - R + |k|^2 - (\tr k)^2
\end{align}

This is designed so that $\mathcal{S}$ is (formally) monotone.
\end{definition}

\begin{insight}
\textbf{The Challenge}

The constraint equations:
\begin{align}
    R - |k|^2 + (\tr k)^2 &= 16\pi\mu\\
    \divv(k - (\tr k)g) &= 8\pi J
\end{align}

must be preserved by the flow.

This requires adding "constraint corrections" that:
\begin{itemize}
    \item Keep the data on the constraint surface
    \item Don't destroy the monotonicity of $\mathcal{S}$
\end{itemize}

This is technically challenging but may be achievable.
\end{insight}

%% ============================================================================
\section{Conclusion}
%% ============================================================================

The weak solution theory for IOEF faces technical challenges:

\begin{enumerate}
    \item \textbf{Existence:} Achievable via elliptic regularization (following H-I)
    \item \textbf{Monotonicity:} The spacetime Hawking mass doesn't have obvious 
          monotonicity; need a modified functional
    \item \textbf{Limit:} Need to show flow reaches infinity with correct mass limit
\end{enumerate}

The Perelman-style entropy approach is an alternative that may provide:
\begin{itemize}
    \item A monotone quantity under flow
    \item Rigidity at the minimum (Schwarzschild)
    \item A path to proving Penrose via variational characterization
\end{itemize}

Both approaches require significant technical development but represent 
genuine new mathematics tailored to the Penrose problem.

\end{document}
