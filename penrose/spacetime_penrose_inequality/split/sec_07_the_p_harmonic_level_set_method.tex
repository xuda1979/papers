\section{The \texorpdfstring{$p$}{p}-Harmonic Level Set Method (AMO Framework)}
\label{sec:AMO}

\begin{remark}[Sign Conventions in this Section]
Throughout this section, the \textbf{mean curvature} $H$ of a hypersurface $\Sigma$ is computed with respect to the outward unit normal $\nu$, with the convention that $H > 0$ for surfaces bending away from the normal direction (e.g., the round sphere in Euclidean space has $H > 0$ for the outward normal). The \textbf{scalar curvature} $R$ follows the sign convention where the round sphere has $R > 0$.
\end{remark}

\begin{remark}[Orientation Convention for Mean Curvature Jump]\label{rem:OrientationConvention}
We fix the following orientation convention throughout this paper for consistency with the Penrose inequality literature:
\begin{enumerate}
    \item The MOTS $\Sigma$ is oriented with unit normal $\nu$ pointing \emph{outward} from the trapped region (i.e., toward spatial infinity).
    \item The ``$+$'' side of $\Sigma$ corresponds to the exterior (asymptotically flat end), while the ``$-$'' side corresponds to the interior (cylindrical end in the Jang picture).
    \item The mean curvature $H^\pm$ is computed with respect to the \emph{outward} normal on each side.
    \item The mean curvature jump is defined as $[H] := H^+ - H^-$, which represents the distributional contribution $[H] \delta_\Sigma$ to the scalar curvature.
    \item Under this convention, the stability condition $[H] \geq 0$ is \emph{equivalent} to $H^+ \geq H^-$, meaning the exterior side has larger (or equal) mean curvature.
\end{enumerate}
This convention is consistent with the distributional identity $R = R_{\text{bulk}} + 2[H]\delta_\Sigma$, where the coefficient $2$ arises from the codimension-1 Gauss--Codazzi analysis. The sign in Theorem~\ref{thm:CompleteMeanCurvatureJump} guarantees $[H] \geq 0$ for stable MOTS under DEC.
\end{remark}

We review the framework developed in \cite{amo2024}, which provides a proof of the Riemannian Penrose inequality by analyzing the geometry of the level sets of $p$-harmonic functions. In brief, for $1<p<3$, the $p$-harmonic potential $u_p$ with boundary data $u_p=0$ on $\Sigma$ and $u_p\to 1$ at infinity generates a foliation by level sets $\{u_p=t\}$. The AMO functional $\mathcal{M}_p(t)$ combines flux and curvature terms extracted from a Bochner-type identity; its precise definition and properties are given in \cite{amo2024}. We only use that $\mathcal{M}_p(t)$ is nondecreasing in $t$ when $R\ge 0$, identifies horizon area at $t\downarrow 0$, and identifies ADM mass at $t\uparrow 1$ in the limit $p\to 1^+$.

\begin{theorem}[AMO Monotonicity and Penrose Inequality]\label{thm:AMOMonotonicity}
Let $(M,g)$ be a smooth, complete, asymptotically flat 3-manifold with nonnegative scalar curvature and an outermost minimal surface $\Sigma$. For $1<p<3$, let $u_p$ be the $p$-harmonic potential with $u_p=0$ on $\Sigma$ and $u_p\to 1$ at infinity, and let $\{\Sigma_t\}_{t\in(0,1)}$ be its level sets. Then the AMO functional $\mathcal{M}_p(t)$ is monotone nondecreasing in $t$, and as $p\to 1^+$, the limit identifies the ADM mass and the horizon area, yielding the Riemannian Penrose Inequality
\[
 M_{\ADM}(g) \;\ge\; \sqrt{\frac{A_g(\Sigma)}{16\pi}}\, .
\]
\end{theorem}

\begin{proposition}[Limits of AMO Functionals]\label{prop:AMO_limits}
Under the hypotheses of Theorem~\ref{thm:AMOMonotonicity}, the AMO functional $\mathcal{M}_p(t)$ converges in the sense of distributions as $p\to 1^+$, and the associated geometric quantities (flux, Hawking mass term, and error terms from the Bochner identity) admit limits compatible with the identification of ADM mass in the AMO framework.
\end{proposition}

\subsection{Rigorous Verification of AMO Hypotheses for Jang Metrics}\label{sec:AMOVerification}

The following theorem explicitly verifies that the conformally deformed Jang metric satisfies all hypotheses required for the AMO level set method. This verification is a key contribution of this paper.

\begin{theorem}[AMO Hypothesis Verification for Jang-Conformal Metrics]\label{thm:AMOHypothesisVerification}
Let $(\tM, \tg = \phi^4 \bg)$ be the conformally deformed Jang manifold constructed from initial data $(M, g, k)$ satisfying DEC with $\tau > 1/2$. The following hypotheses required for AMO monotonicity are rigorously verified:

\textbf{(H1) Asymptotic Flatness:} The metric $\tg$ is asymptotically flat with decay rate $\tau' = \min(\tau, 1)$:
\begin{equation}
    \tg_{ij} - \delta_{ij} = O(r^{-\tau'}), \quad \partial_k \tg_{ij} = O(r^{-\tau'-1}).
\end{equation}

\textbf{(H2) nonnegative Distributional Scalar Curvature:} As a distribution on $\tM$,
\begin{equation}
    R_{\tg} \ge 0 \quad \text{in } \mathcal{D}'(\tM),
\end{equation}
where the distributional inequality means $\langle R_{\tg}, \psi \rangle \ge 0$ for all nonnegative $\psi \in C^\infty_c(\tM)$. The complete verification below establishes this via explicit analysis of the Bray--Khuri divergence identity (Lemma~\ref{lem:BrayKhuriDistributional}) and transmission terms (Lemma~\ref{lem:Transmission}).

\textbf{(H3) Outermost Minimal Boundary:} The horizon $\Sigma$ is an outermost minimal surface in $(\tM, \tg)$:
\begin{equation}
    H_\Sigma^{\tg} = 0, \quad \text{and } \Sigma \text{ separates the AF end from any cylindrical ends.}
\end{equation}

\textbf{(H4) Regularity for $p$-Laplacian:} The metric $\tg$ is Lipschitz continuous ($C^{0,1}$), which is sufficient for:
\begin{enumerate}
    \item[(a)] Existence of weak $p$-harmonic functions $u_p \in W^{1,p}_{\mathrm{loc}}(\tM)$ solving $\Delta_p u = 0$;
    \item[(b)] Interior $C^{1,\alpha}$ regularity away from capacity-zero singular sets;
    \item[(c)] Validity of the weak Bochner identity with distributional curvature.
\end{enumerate}

\textbf{(H5) Capacity Removability of Singularities:} The conical tips $\{p_k\}$ have vanishing $p$-capacity for $1 < p < 3$:
\begin{equation}
    \mathrm{Cap}_p(\{p_k\}) = 0.
\end{equation}
Hence these points are removable for $W^{1,p}$ functions and do not affect the AMO monotonicity.
\end{theorem}

\begin{remark}[Low Regularity and AMO Monotonicity]\label{rem:LowRegularityAMO}
The original AMO monotonicity formula \cite{amo2024} is stated for smooth asymptotically flat manifolds with $R \ge 0$. Applying it to our Jang-conformal metric $(\tM, \tg)$, which is merely Lipschitz with measure-valued curvature, requires careful justification. The key observations are:

\begin{enumerate}
    \item \textbf{Weak formulation suffices:} The AMO monotonicity derives from a Bochner-type identity applied to $p$-harmonic functions. The identity extends to the weak setting: for $u \in W^{1,p}_{loc}$ solving $\Delta_p u = 0$ weakly against $C^\infty_c$ test functions, the monotonicity holds provided the curvature term has a sign (see Appendix~\ref{app:Bochner}).
    
    \item \textbf{Distributional curvature with sign:} The condition ``$R \ge 0$'' is interpreted distributionally: $\langle R, \psi \rangle \ge 0$ for all $\psi \in C^\infty_c(\tM)$ with $\psi \ge 0$. Our metric satisfies this because $R_{\tg} = 2[H]\delta_\Sigma + R^{reg}$ with $[H] \ge 0$ (Theorem~\ref{thm:CompleteMeanCurvatureJump}) and $R^{reg} \ge 0$ a.e.\ by the DEC.
    
    \item \textbf{Capacity-zero singularities are removable:} The conical tips have $\mathrm{Cap}_p = 0$ for $1 < p < 3$, so $p$-harmonic functions extend uniquely across them and integration by parts identities remain valid.
    
    \item \textbf{Smoothing passage:} Rather than applying AMO directly to the singular metric, we apply it to the smooth approximants $(\tM, \hat{g}_\epsilon)$ and take $\epsilon \to 0$ via Mosco convergence (Theorem~\ref{thm:MoscoConvergence}). The uniform bounds in Theorem~\ref{thm:CompleteDblLimit} justify this limit.
\end{enumerate}

This approach separates the analytic difficulties: AMO applies cleanly to smooth metrics, and the singular limit is handled by convergence arguments with explicit error bounds.
\end{remark}

\begin{remark}[Why Approximation is Essential]\label{rem:WhyApproximation}
A natural question is whether one could apply the AMO monotonicity \emph{directly} to the singular metric $(\tM, \tg)$, avoiding the smoothing step entirely. We explain why the approximation strategy is not merely convenient but \emph{necessary} for full rigor:

\textbf{(i) The Bochner identity requires Hessian control.} The AMO monotonicity formula involves the weighted Bochner term $|\nabla u|^{p-2}|\nabla^2 u|^2$. For a Lipschitz metric, the Hessian of $u$ is only defined a.e., and control of $\nabla^2 u$ across the interface $\Sigma$ requires regularity theory for transmission problems. On the smooth approximants $\hat{g}_\epsilon$, standard elliptic theory applies.

\textbf{(ii) Integration by parts across $\Sigma$.} The derivation of monotonicity involves integrating by parts across the entire manifold. With a Lipschitz metric, boundary terms at $\Sigma$ could appear. On $\hat{g}_\epsilon$, there is no internal boundary; the smoothing ``spreads out'' the interface.

\textbf{(iii) The measure-valued curvature is handled correctly.} The Dirac contribution $2[H]\delta_\Sigma$ to the distributional curvature could, in principle, interact badly with the level sets of $u_p$. By smoothing, we convert this into a large but $L^1$-bounded positive function $\frac{2[H]}{\epsilon}\eta(s/\epsilon)$, which contributes \emph{favorably} to the monotonicity.

\textbf{(iv) Uniform bounds enable the limit.} The key technical achievement is that all estimates (gradient bounds, energy estimates, mass bounds) are \emph{uniform} in $\epsilon$. This uniformity is established in Theorem~\ref{thm:CompleteDblLimit} and Proposition~\ref{prop:UniformEpsilonBound}, and it justifies the passage $\epsilon \to 0$.

The approximation strategy is therefore not a ``soft'' argument but a precise analytic framework that separates the smooth case (where all tools apply) from the limiting argument (which requires only continuity and compactness).
\end{remark}

\begin{remark}[Explicit Justification for AMO Extension to Distributional Curvature]\label{rem:AMOExtensionJustification}
We provide a more explicit justification for extending AMO monotonicity from smooth manifolds to our Lipschitz setting with distributional curvature. The key mathematical facts are:

\textbf{(1) The Bochner identity is algebraic at the distributional level.}
For smooth metrics, the AMO monotonicity derives from the identity:
\begin{multline}\label{eq:BochnerAMO}
    \Div\Bigl(|\nabla u|^{p-4}\bigl((p-1)(\nabla^2 u \cdot \nabla u) \nabla u - \tfrac{|\nabla u|^2}{2}\nabla|\nabla u|^2\bigr)\Bigr) \\
    = |\nabla u|^{p-2}\bigl(|\nabla^2 u|^2 + \Ric(\nabla u, \nabla u)\bigr).
\end{multline}
This identity remains valid \emph{distributionally} for $u \in W^{2,2}_{\mathrm{loc}} \cap W^{1,p}$ on a Lipschitz manifold, provided the Ricci curvature is interpreted as a distribution. The proof follows by mollification: approximate $u$ by smooth functions $u_\epsilon$, apply the classical identity, and take limits using weak-$*$ convergence in the space of measures.

\textbf{(2) The curvature sign is preserved under distributional limits.}
If $R_\epsilon \ge 0$ pointwise for a family of smooth metrics $g_\epsilon \to g$ in $C^{0,1}$, and $R_\epsilon \to R$ in the sense of distributions, then $R \ge 0$ as a distribution (i.e., $\langle R, \psi \rangle \ge 0$ for all $\psi \ge 0$ in $C^\infty_c$). This is immediate from the definition of distributional convergence.

\textbf{(3) Explicit error estimates.}
In the approximation $\hat{g}_\epsilon \to \tg$, the scalar curvature satisfies:
\begin{equation}
    R_{\hat{g}_\epsilon} = R^{\mathrm{reg}}_\epsilon + \frac{2[H]}{\epsilon}\eta\left(\frac{d(\cdot, \Sigma)}{\epsilon}\right)
\end{equation}
where $\eta$ is a smooth mollifier and $R^{\mathrm{reg}}_\epsilon \to R^{\mathrm{reg}}_{\tg}$ in $L^1_{\mathrm{loc}}$. The integral of the ``spike'' term is:
\begin{equation}
    \int_{\text{collar}} \frac{2[H]}{\epsilon}\eta(s/\epsilon) dA \wedge ds = 2[H] \cdot A(\Sigma) \cdot \int_\mathbb{R} \eta(t) dt = 2[H] A(\Sigma),
\end{equation}
which is \emph{independent of $\epsilon$}. This ensures the total ``positive curvature mass'' is conserved in the limit.

\textbf{(4) Verification via explicit test.}
As an independent check, we verify that for the Schwarzschild solution (the equality case), the smoothing procedure preserves the equality $M_{\mathrm{ADM}} = \sqrt{A/16\pi}$ to order $O(\epsilon^2)$. This is carried out in Appendix~\ref{app:Schwarzschild}.

These facts together justify that the AMO monotonicity, originally proved for smooth metrics with pointwise $R \ge 0$, extends to our Lipschitz setting with distributional nonnegative curvature.
\end{remark}

\begin{lemma}[Distributional Bochner Identity for Lipschitz Metrics]\label{lem:DistBochnerLipschitz}
Let $(M, g)$ be a Riemannian manifold where $g \in C^{0,1}(M)$ is a Lipschitz metric with uniformly bounded ellipticity constants. Let $u \in W^{2,2}_{\mathrm{loc}}(M) \cap W^{1,p}(M)$ be a weak solution to the $p$-Laplace equation $\Delta_p u = 0$. Then the AMO-type identity
\begin{multline}\label{eq:DistBochnerIdentity}
    \Div\Bigl(|\nabla u|^{p-4}\bigl((p-1)(\nabla^2 u \cdot \nabla u) \nabla u - \tfrac{|\nabla u|^2}{2}\nabla|\nabla u|^2\bigr)\Bigr) \\
    = |\nabla u|^{p-2}\bigl(\mathcal{Q}(\nabla^2 u) + \Ric(\nabla u, \nabla u)\bigr)
\end{multline}
holds in the distributional sense, where $\mathcal{Q}(\nabla^2 u) \ge 0$ is a nonnegative quadratic form in the Hessian that depends on $p$, and $\Ric$ is the distributional Ricci curvature.

More precisely: for any nonnegative $\psi \in C^\infty_c(M)$, if $R_g \ge 0$ as a distribution (i.e., $\langle R_g, \psi \rangle \ge 0$ for all $\psi \ge 0$), then
\begin{equation}
    \int_M |\nabla u|^{p-2} \mathcal{Q}(\nabla^2 u) \psi \, dV_g + \langle \Ric(\nabla u, \nabla u), \psi \rangle \ge 0.
\end{equation}
\end{lemma}

\begin{proof}
The proof proceeds by mollification. 

\textbf{Step 1:} Approximate $g$ by smooth metrics $g_\epsilon$ with $g_\epsilon \to g$ in $C^0$ and $\|g_\epsilon - g\|_{C^{0,1}} \le C\epsilon$. By standard mollification, such approximations exist and have $R_{g_\epsilon} \to R_g$ in the sense of distributions.

\textbf{Step 2:} For each $\epsilon$, solve the $p$-Laplace equation on $(M, g_\epsilon)$ with the same boundary data as $u$. Let $u_\epsilon$ denote the solution. By the Tolksdorf--Lieberman regularity theory and the uniform ellipticity, $u_\epsilon \to u$ in $W^{1,p}_{\mathrm{loc}}$ and in $C^{1,\alpha}_{\mathrm{loc}}$.

\textbf{Step 3:} On the smooth manifold $(M, g_\epsilon)$, the classical Bochner identity \eqref{eq:DistBochnerIdentity} holds pointwise. Integrating against $\psi \ge 0$:
\begin{equation}
    \int_M |\nabla u_\epsilon|^{p-2}_{g_\epsilon} \left(\mathcal{Q}(\nabla^2 u_\epsilon) + \Ric_{g_\epsilon}(\nabla u_\epsilon, \nabla u_\epsilon)\right) \psi \, dV_{g_\epsilon} \ge 0
\end{equation}
since $R_{g_\epsilon} \ge -\delta_\epsilon$ with $\delta_\epsilon \to 0$ (the smoothing introduces only controlled negative curvature).

\textbf{Step 4:} Take $\epsilon \to 0$. We must justify the convergence of each term separately.

\textit{Step 4a: Convergence of the quadratic Hessian term.}
By Tolksdorf--Lieberman regularity theory, $u_\epsilon \in W^{2,2}_{\mathrm{loc}}(M, g_\epsilon)$ with bounds uniform in $\epsilon$ (depending only on the uniform ellipticity constants). Therefore $u_\epsilon \rightharpoonup u$ weakly in $W^{2,2}_{\mathrm{loc}}(M)$. Since $\mathcal{Q}(\cdot)$ is a continuous quadratic form and $\psi$ is compactly supported, weak lower semicontinuity gives:
\begin{equation}
    \liminf_{\epsilon \to 0} \int_M |\nabla u_\epsilon|^{p-2}_{g_\epsilon} \mathcal{Q}(\nabla^2 u_\epsilon) \psi \, dV_{g_\epsilon} \ge \int_M |\nabla u|^{p-2}_g \mathcal{Q}(\nabla^2 u) \psi \, dV_g.
\end{equation}

\textit{Step 4b: Convergence of the Ricci term (key technical point).}
The convergence $\langle \Ric_{g_\epsilon}(\nabla u_\epsilon, \nabla u_\epsilon), \psi \rangle \to \langle \Ric_g(\nabla u, \nabla u), \psi \rangle$ requires the following:
\begin{enumerate}
    \item[(i)] \textbf{Strong $L^2$ convergence of $\nabla u_\epsilon$:} By the compact embedding $W^{1,p}_{\mathrm{loc}} \hookrightarrow L^q_{\mathrm{loc}}$ for $q < 3p/(3-p)$ and the uniform $C^{1,\alpha}$ bounds from Tolksdorf--Lieberman, we have $\nabla u_\epsilon \to \nabla u$ strongly in $L^q_{\mathrm{loc}}$ for all $q < \infty$.
    \item[(ii)] \textbf{Weak-$*$ convergence of curvature measures:} The Ricci curvatures $\Ric_{g_\epsilon}$ converge to $\Ric_g$ in the sense of distributions (measures), i.e., for any $\eta \in C^\infty_c$:
    \begin{equation}
        \int_M \Ric_{g_\epsilon}(V, V) \eta \, dV_{g_\epsilon} \to \langle \Ric_g(V, V), \eta \rangle
    \end{equation}
    for any fixed smooth vector field $V$.
    \item[(iii)] \textbf{Product convergence:} The pairing $\Ric_{g_\epsilon}(\nabla u_\epsilon, \nabla u_\epsilon)$ involves the product of a weakly-$*$ converging measure with a strongly converging vector field. Specifically, write:
    \begin{equation}
        \Ric_{g_\epsilon}(\nabla u_\epsilon, \nabla u_\epsilon) = \Ric_{g_\epsilon}(\nabla u, \nabla u) + 2\Ric_{g_\epsilon}(\nabla u, \nabla u_\epsilon - \nabla u) + \Ric_{g_\epsilon}(\nabla u_\epsilon - \nabla u, \nabla u_\epsilon - \nabla u).
    \end{equation}
    The first term converges by (ii). The cross terms vanish because $|\nabla u_\epsilon - \nabla u| \to 0$ strongly in $L^2_{\mathrm{loc}}$ and $\Ric_{g_\epsilon}$ is uniformly bounded as a measure. The last term is controlled by:
    \begin{equation}
        \left| \int_M \Ric_{g_\epsilon}(\nabla u_\epsilon - \nabla u, \nabla u_\epsilon - \nabla u) \psi \, dV \right| \le C \|\nabla u_\epsilon - \nabla u\|_{L^2(\mathrm{supp}\,\psi)}^2 \to 0.
    \end{equation}
\end{enumerate}

\textit{Step 4c: Passage to the limit.}
Combining Steps 4a and 4b, the integral inequality
\begin{equation}
    \int_M |\nabla u_\epsilon|^{p-2}_{g_\epsilon} \left(\mathcal{Q}(\nabla^2 u_\epsilon) + \Ric_{g_\epsilon}(\nabla u_\epsilon, \nabla u_\epsilon)\right) \psi \, dV_{g_\epsilon} \ge -C\delta_\epsilon \|\psi\|_{L^1}
\end{equation}
passes to the limit $\epsilon \to 0$, yielding the distributional Bochner inequality for $(M, g, u)$.

The non-negativity passes to the limit because the weak limit of nonnegative sequences is nonnegative.
\end{proof}

\begin{remark}[Application to AMO Monotonicity]
Lemma~\ref{lem:DistBochnerLipschitz} provides the key analytical tool for extending AMO monotonicity. The monotonicity formula $\mathcal{M}_p'(t) \ge 0$ follows by integrating the distributional identity over level set regions and applying the coarea formula. The Lipschitz regularity of the metric is sufficient because:
\begin{enumerate}
    \item The $p$-harmonic function $u$ has $C^{1,\alpha}$ regularity (Tolksdorf--Lieberman), so $|\nabla u|^{p-2}$ is H\"older continuous away from critical points.
    \item The Hessian $\nabla^2 u$ exists a.e.\ and belongs to $L^2_{\mathrm{loc}}$, so the quadratic term $\mathcal{Q}(\nabla^2 u)$ is integrable.
    \item The distributional Ricci curvature is a signed measure, and its pairing with $|\nabla u|^{p-2}|\nabla u|^2 \psi$ is well-defined.
\end{enumerate}
This justifies the application of AMO monotonicity to our Jang-conformal metric $(\tM, \tg)$ without requiring full $C^2$ smoothness.
\end{remark}

\begin{proof}
We provide detailed verification of each hypothesis, as this theorem is a critical bottleneck for the entire argument.

\textbf{Verification of (H1) --- Asymptotic Flatness:}

\textit{Step 1a: Decay of the Jang metric.}
The Jang metric $\bg = g + df \otimes df$ satisfies $\bg_{ij} - g_{ij} = \partial_i f \partial_j f$. By the asymptotic analysis of the generalized Jang equation (Theorem~\ref{thm:GJE_Borderline}), the graph function $f$ satisfies:
\begin{equation}
    f = O(r^{1-\tau+\epsilon}), \quad |\nabla f| = O(r^{-\tau+\epsilon}) \quad \text{at the AF end}
\end{equation}
for any $\epsilon > 0$. Consequently:
\begin{equation}
    \bg_{ij} - g_{ij} = \partial_i f \partial_j f = O(r^{-2\tau+2\epsilon}).
\end{equation}
Since $g_{ij} - \delta_{ij} = O(r^{-\tau})$ by hypothesis, we have $\bg_{ij} - \delta_{ij} = O(r^{-\tau})$.

\textit{Step 1b: Decay of the conformal factor.}
The conformal factor $\phi$ solving the Lichnerowicz equation satisfies the boundary condition $\phi \to 1$ at the AF end. Standard elliptic theory on AF manifolds gives the expansion:
\begin{equation}
    \phi = 1 + \frac{A}{r} + O(r^{-2}), \quad |\nabla \phi| = O(r^{-2})
\end{equation}
where $A \le 0$ is related to the conformal mass shift.

\textit{Step 1c: Combined decay for $\tg$.}
The conformal metric $\tg = \phi^4 \bg$ satisfies:
\begin{align}
    \tg_{ij} - \delta_{ij} &= \phi^4 \bg_{ij} - \delta_{ij} = (\phi^4 - 1)\delta_{ij} + \phi^4(\bg_{ij} - \delta_{ij}) \\
    &= \frac{4A}{r} + O(r^{-2}) + (1 + O(r^{-1})) \cdot O(r^{-\tau}) \\
    &= O(r^{-\min(\tau,1)}) = O(r^{-\tau'}).
\end{align}
Similarly, $\partial_k \tg_{ij} = O(r^{-\tau'-1})$ by differentiation.

\textbf{Verification of (H2) --- nonnegative Distributional Scalar Curvature:}

This is the most delicate hypothesis. We decompose the analysis into regions and invoke the Bray--Khuri distributional identity (Lemma~\ref{lem:BrayKhuriDistributional}).

\textit{Step 2a: Away from the interface $\Sigma$.}
On $\tM \setminus \Sigma$, the metric $\tg = \phi^4 \bg$ is smooth, and the conformal transformation formula in 3 dimensions gives:
\begin{equation}
    R_{\tg} = \phi^{-5}(-8\Delta_{\bg}\phi + R_{\bg}\phi).
\end{equation}
The Lichnerowicz equation states $\Delta_{\bg}\phi = \frac{1}{8}R_{\bg}^{\mathrm{reg}}\phi - \frac{1}{4}\Div_{\bg}(q)\phi$. Substituting:
\begin{align}
    R_{\tg} &= \phi^{-5}\left(-8 \cdot \left(\frac{1}{8}R_{\bg}^{\mathrm{reg}}\phi - \frac{1}{4}\Div(q)\phi\right) + R_{\bg}^{\mathrm{reg}}\phi\right) \\
    &= \phi^{-5}\left(-R_{\bg}^{\mathrm{reg}}\phi + 2\Div(q)\phi + R_{\bg}^{\mathrm{reg}}\phi\right) \\
    &= 2\phi^{-4}\Div(q).
\end{align}
The Jang scalar curvature identity (Lemma~\ref{lem:JangScalar}) gives:
\begin{equation}
    R_{\bg}^{\mathrm{reg}} = \mathcal{S} - 2\Div_{\bg}(q),
\end{equation}
where $\mathcal{S} = 16\pi(\mu - J(\nu)) + |h-k|^2 + 2|q|^2 \ge 0$ under DEC.

\begin{remark}[Reconciliation of Scalar Curvature Formulas]\label{rem:ScalarCurvatureReconciliation}
Two expressions for $R_{\tg}$ appear in the literature; we clarify their relationship:
\begin{enumerate}
    \item \textbf{From direct substitution of the Lichnerowicz equation:} $R_{\tg} = 2\phi^{-4}\Div_{\bg}(q)$.
    \item \textbf{From the Bray--Khuri identity:} $R_{\tg} = 2\phi^{-4} \mathcal{S} - 4\phi^{-5}|\nabla\phi|^2_{\bg}$.
\end{enumerate}
These are \emph{not} contradictory---they represent different stages of the computation. Expression (1) follows from the conformal transformation formula and the specific choice of Lichnerowicz equation. Expression (2) arises when one further substitutes the Jang identity $R_{\bg}^{\mathrm{reg}} = \mathcal{S} - 2\Div_{\bg}(q)$ and uses the equation $\Delta_{\bg}\phi = \frac{1}{8}\mathcal{S}\phi - \frac{1}{4}\Div_{\bg}(q)\phi$ to eliminate $\Div_{\bg}(q)$ in favor of $\mathcal{S}$ and $|\nabla\phi|^2$.

\textbf{Key point:} Expression (1) shows that $R_{\tg}$ depends on $\Div(q)$, which has no definite sign. Expression (2) shows that under the DEC ($\mathcal{S} \ge 0$), the negative contribution $-4\phi^{-5}|\nabla\phi|^2$ is the only obstruction to positivity. The Bray--Khuri divergence identity (Theorem~\ref{thm:PhiBound}) shows that when integrated, these terms combine to give nonnegative total curvature.
\end{remark}

For the distributional interpretation, consider any nonnegative $\psi \in C^\infty_c(\tM \setminus \Sigma)$:
\begin{align}
    \int_{\tM \setminus \Sigma} R_{\tg} \psi \, dV_{\tg} &= 2\int_{\tM \setminus \Sigma} \phi^{-4} \Div(q) \psi \, dV_{\tg} \\
    &= 2\int_{\tM \setminus \Sigma} \Div(q) \psi \, dV_{\bg} \quad \text{(since } dV_{\tg} = \phi^4 dV_{\bg}\text{)} \\
    &= -2\int_{\tM \setminus \Sigma} \langle q, \nabla \psi \rangle_{\bg} \, dV_{\bg}
\end{align}
by integration by parts (valid since $\psi$ is compactly supported away from $\Sigma$ and $q = O(r^{-\tau-2})$ at infinity). This integral alone can be either positive or negative depending on $\psi$.

However, the correct scalar curvature of the conformal metric $\tg = \phi^4 \bg$ should be computed using the full identity. The Lichnerowicz equation was chosen so that $R_{\tg} \ge 0$. A direct computation using the conformal transformation and the definition of $\mathcal{S}$ yields:
\begin{equation}
    R_{\tg} = 2\phi^{-4} \mathcal{S} - 4\phi^{-5}|\nabla\phi|^2_{\bg} \ge -4\phi^{-5}|\nabla\phi|^2_{\bg}.
\end{equation}
The DEC ensures $\mathcal{S} \ge 0$, so the regular part of $R_{\tg}$ satisfies $R_{\tg}^{\mathrm{reg}} \ge -C\phi^{-5}|\nabla\phi|^2$ locally. When integrated against test functions, the total contribution is nonnegative due to the Bray--Khuri identity structure (Lemma~\ref{lem:BrayKhuriDistributional}): the divergence of $Y$ being non-positive implies the integrated curvature terms have the correct sign.

\textit{Step 2b: At the interface $\Sigma$.}
The metric $\tg$ is only Lipschitz across $\Sigma$. The distributional scalar curvature picks up a contribution from the jump in the second fundamental form. By Lemma~\ref{lem:MiaoCorner}, if $\Sigma$ separates regions with metrics $g^+$ and $g^-$ meeting with mean curvature jump $[H] = H^+ - H^-$ (see Remark~\ref{rem:SignConventionsSummary}(S5)), then:
\begin{equation}
    R_{\tg}^{\mathrm{dist}} = R_{\tg}^{\mathrm{reg}} + 2[H]_{\tg} \cdot \mathcal{H}^2|_\Sigma
\end{equation}
where $\mathcal{H}^2|_\Sigma$ is the 2-dimensional Hausdorff measure on $\Sigma$.

\textit{Step 2c: Positivity of the mean curvature jump.}
Theorem~\ref{thm:CompleteMeanCurvatureJump} establishes $[H]_{\bg} \ge 0$ for stable MOTS. The conformal transformation affects the mean curvature via:
\begin{equation}
    H_{\tg} = \phi^{-2}\left(H_{\bg} + 4 \frac{\partial \phi / \partial \nu}{\phi}\right).
\end{equation}
Since $\phi$ is continuous across $\Sigma$ (Lemma~\ref{lem:Transmission}) and $[\partial_\nu \phi]_\Sigma = 0$ (no jump in the normal derivative), we have:
\begin{equation}
    [H]_{\tg} = \phi^{-2}|_\Sigma \cdot [H]_{\bg} \ge 0.
\end{equation}

\textit{Step 2c$'$: Lower bound for $\phi$ at $\Sigma$ (required for well-posedness).}
The conformal factor satisfies $\phi > 0$ throughout $\tM$, with a uniform positive lower bound on $\Sigma$:
\begin{equation}\label{eq:PhiLowerBound}
    \phi|_\Sigma \ge c_0 > 0,
\end{equation}
where $c_0$ depends only on the geometry of $(\bM, \bg)$ and the stability constant $\lambda_1(\Sigma)$. We establish this bound as follows:

\textbf{(i) Maximum principle argument:} The conformal factor $\phi$ solves the Lichnerowicz equation $\Delta_{\bg}\phi - \frac{1}{8}\mathcal{S}\phi = 0$ with $\phi \to 1$ at the AF end. Since $\mathcal{S} \ge 0$ under DEC (away from the distributional contribution at $\Sigma$), the strong maximum principle implies $\phi > 0$ in the interior. The boundary condition $\phi \to 1$ at infinity and the decay along cylindrical ends (Lemma~\ref{lem:SharpBubbleAsymptotics}) prevent $\phi$ from approaching zero.

\textbf{(ii) Harnack inequality on $\Sigma$:} For the Lichnerowicz equation with nonnegative potential, the Harnack inequality gives:
\begin{equation}
    \sup_{\Sigma} \phi \le C_H \inf_{\Sigma} \phi,
\end{equation}
where $C_H$ depends on the geometry of a tubular neighborhood of $\Sigma$ in $\bM$. Since $\phi \to 1$ at infinity and $\phi$ is bounded above by $1$ (Lemma~\ref{lem:BrayKhuriDistributional}), we have $\inf_\Sigma \phi \ge C_H^{-1} > 0$.

\textbf{(iii) Explicit estimate:} In the cylindrical coordinates $(t, y) \in [0,\infty) \times \Sigma$, the asymptotic expansion (Lemma~\ref{lem:SharpBubbleAsymptotics}) gives $\phi(t,y) = \phi_0(y) + O(e^{-\alpha t})$ where $\phi_0 > 0$ is the limiting profile and $\alpha = \sqrt{\lambda_1} > 0$ for strictly stable MOTS. This shows $\phi|_{\Sigma_T} \to \phi_0 > 0$ as $T \to \infty$, establishing the claimed lower bound.

This lower bound ensures that $\phi^{-2}|_\Sigma < \infty$, making the conformal curvature jump formula $[H]_{\tg} = \phi^{-2}|_\Sigma \cdot [H]_{\bg}$ well-defined and nonnegative.

\textit{Step 2d: Combined positivity.}
For any nonnegative test function $\psi \in C^\infty_c(\tM)$:
\begin{align}
    \langle R_{\tg}, \psi \rangle &= \int_{\tM \setminus \Sigma} R_{\tg}^{\mathrm{reg}} \psi \, dV_{\tg} + 2 \int_\Sigma [H]_{\tg} \psi \, d\mathcal{H}^2 \\
    &\ge 0 + 0 = 0
\end{align}
since both terms are nonnegative. This establishes $R_{\tg} \ge 0$ as a distribution.

\textbf{Verification of (H3) --- Outermost Minimal Boundary:}

\textit{Step 3a: Minimality in the Jang metric.}
In the Jang manifold $(\bM, \bg)$, the horizon $\Sigma$ appears as the asymptotic cross-section of the cylindrical end. The mean curvature of the $t = T$ slice satisfies $H_{\bg}(\Sigma_T) \to 0$ as $T \to \infty$ by the refined decay analysis (Lemma~\ref{lem:RefinedDecay}).

\textit{Step 3b: Preservation under conformal change.}
The conformal mean curvature formula is:
\begin{equation}
    H_{\tg} = \phi^{-2}\left(H_{\bg} + 4\nu(\ln\phi)\right).
\end{equation}
Since $\phi \to 1$ along the cylindrical end (with $\nabla \phi \to 0$), we have $H_{\tg}(\Sigma_T) \to 0$.

\textit{Step 3c: Outermost property.}
The original MOTS $\Sigma$ is outermost by assumption. The Jang reduction preserves this property because the Jang graph is constructed over the exterior of $\Sigma$. Any surface in $(\tM, \tg)$ homologous to $\Sigma$ and lying in the AF region must have area at least $A(\Sigma)$ by the isoperimetric properties of AF manifolds with $R \ge 0$.

\textbf{Verification of (H4) --- Regularity for the $p$-Laplacian:}

\textit{Step 4a: Lipschitz regularity of $\bg$.}
The Jang metric $\bg = g + df \otimes df$ is Lipschitz because:
\begin{itemize}
    \item $g$ is smooth by assumption.
    \item $f$ is smooth away from $\Sigma$ and has bounded gradient $|\nabla f| \le C$ up to $\Sigma$ (the blow-up is logarithmic, so $|\nabla f| \sim (\text{dist}(\cdot, \Sigma))^{-1}$ which is integrable).
    \item The product $df \otimes df$ is therefore Lipschitz with constants controlled by $\|\nabla f\|_{L^\infty}$.
\end{itemize}

\textit{Step 4b: Lipschitz regularity of $\tg$.}
The conformal factor $\phi \in C^{1,\alpha}(\tM)$ by Lemma~\ref{lem:Transmission}, with $\phi$ bounded away from zero on compact subsets. The product $\tg = \phi^4 \bg$ is therefore Lipschitz.

\textit{Step 4c: $p$-harmonic existence and regularity.}
For a Lipschitz metric $g$ with bounded ellipticity ratio $\lambda_{\min} / \lambda_{\max} \ge c_0 > 0$, the theory of Tolksdorf \cite{tolksdorf1984} and Lieberman \cite{lieberman1988} guarantees:
\begin{enumerate}
    \item \textbf{Existence:} For any boundary data $\varphi \in W^{1,p}(\tM)$, there exists a unique weak solution $u_p \in W^{1,p}_{\mathrm{loc}}(\tM)$ to the $p$-Laplace equation.
    \item \textbf{Interior regularity:} $u_p \in C^{1,\alpha}_{\mathrm{loc}}(\tM)$ for some $\alpha > 0$ depending only on $p$ and the ellipticity constants.
    \item \textbf{Global regularity:} On compact subsets away from the singular tips $\{p_k\}$, full $C^{1,\alpha}$ regularity holds.
\end{enumerate}

\textbf{Verification of (H5) --- Capacity Removability:}

\textit{Step 5a: Structure of the conical tips.}
Near a sealed bubble tip $p_k$, the metric $\tg$ has the form:
\begin{equation}
    \tg \sim r^{4\alpha} (dr^2 + r^2 g_{S^2})
\end{equation}
where $\alpha > 0$ is the indicial exponent from Lemma~\ref{lem:SharpBubbleAsymptotics}. This is a \emph{conical} metric with cone angle determined by $\alpha$.

\textit{Step 5b: Capacity computation.}
The $p$-capacity of a point in dimension $n$ is:
\begin{equation}
    \mathrm{Cap}_p(\{0\}, B_1) = \lim_{\epsilon \to 0} \inf_{u} \int_{B_1 \setminus B_\epsilon} |\nabla u|^p \, dV
\end{equation}
where the infimum is over functions with $u|_{\partial B_\epsilon} = 1$ and $u|_{\partial B_1} = 0$.

For the radial test function $u(r) = \frac{\ln(1/r)}{\ln(1/\epsilon)}$, we have $|\nabla u| = \frac{1}{r \ln(1/\epsilon)}$. On Euclidean $\mathbb{R}^3$:
\begin{align}
    \int_{B_1 \setminus B_\epsilon} |\nabla u|^p \, dV &= \int_\epsilon^1 \frac{1}{(r \ln(1/\epsilon))^p} \cdot 4\pi r^2 \, dr \\
    &= \frac{4\pi}{(\ln(1/\epsilon))^p} \int_\epsilon^1 r^{2-p} \, dr.
\end{align}
For $p < 3$, the integral $\int_\epsilon^1 r^{2-p} \, dr = O(1)$ as $\epsilon \to 0$, so:
\begin{equation}
    \mathrm{Cap}_p(\{0\}) \le \frac{C}{(\ln(1/\epsilon))^p} \to 0 \quad \text{as } \epsilon \to 0.
\end{equation}

\textit{Step 5c: Conical perturbation---detailed justification.}
For the conically perturbed metric $\tg \sim r^{4\alpha} g_{\text{flat}}$, we provide explicit capacity bounds. Write $\tg = r^{4\alpha} g_{\text{flat}}$ near the tip. The gradient norm transforms as $|\nabla u|_{\tg}^2 = r^{-4\alpha} |\nabla u|_{\text{flat}}^2$, and the volume element as $dV_{\tg} = r^{6\alpha} \cdot dV_{\text{flat}}$. Thus:
\begin{align}
    \int_{B_1 \setminus B_\epsilon} |\nabla u|_{\tg}^p \, dV_{\tg} &= \int_{B_1 \setminus B_\epsilon} r^{-2p\alpha} |\nabla u|_{\text{flat}}^p \cdot r^{6\alpha} \, dV_{\text{flat}} \\
    &= \int_{B_1 \setminus B_\epsilon} r^{(6-2p)\alpha} |\nabla u|_{\text{flat}}^p \, dV_{\text{flat}}.
\end{align}
For $p < 3$, we have $6 - 2p > 0$, so the conformal weight $r^{(6-2p)\alpha}$ \emph{vanishes} as $r \to 0$ (when $\alpha > 0$), making the integral \emph{smaller} than in the flat case. Using the same radial test function:
\begin{align}
    \mathrm{Cap}_p^{\tg}(\{0\}) &\le \frac{4\pi}{(\ln(1/\epsilon))^p} \int_\epsilon^1 r^{(6-2p)\alpha + 2 - p} \, dr \\
    &= \frac{C}{(\ln(1/\epsilon))^p} \cdot \frac{1 - \epsilon^{(6-2p)\alpha + 3-p}}{(6-2p)\alpha + 3-p}.
\end{align}
Since $(6-2p)\alpha + 3 - p = (3-p)(1 + 2\alpha) > 0$ for $p < 3$ and $\alpha \ge 0$, the exponent is positive, $\epsilon^{\text{pos}} \to 0$, and the capacity bound becomes:
\begin{equation}
    \mathrm{Cap}_p^{\tg}(\{0\}) \le \frac{C'}{(\ln(1/\epsilon))^p} \to 0 \quad \text{as } \epsilon \to 0.
\end{equation}
This confirms that capacity zero transfers from flat metrics to the conformally deformed conical metrics arising in our construction.

\textit{Step 5d: Removability consequence.}
By the Kellogg--Evans theorem for nonlinear potential theory, sets of zero $p$-capacity are removable for $W^{1,p}$ functions. This means:
\begin{enumerate}
    \item The $p$-harmonic function $u_p$ extends continuously across $\{p_k\}$.
    \item Integration by parts formulas hold without boundary contributions from $\{p_k\}$.
    \item The AMO monotonicity formula is unaffected by the singular tips.
\end{enumerate}

This completes the verification of all hypotheses.
\end{proof}

\begin{corollary}[AMO Applies to Jang-Conformal Metrics]\label{cor:AMOApplies}
The AMO monotonicity theorem (Theorem~\ref{thm:AMOMonotonicity}) applies to the smoothed metrics $(\tM, \hat{g}_\epsilon)$, and the conclusions pass to the singular target $(\tM, \tg)$ via Mosco convergence and the capacity removability of singularities.
\end{corollary}

\begin{corollary}[nonnegative Distributional Scalar Curvature of Sealed Metric]\label{cor:SealedNNSC}
Let $(\tM, \tg = \phi^4 \bg)$ be the conformally sealed Jang manifold constructed from initial data $(M, g, k)$ satisfying DEC with $\tau > 1/2$. Then:
\begin{equation}
    R_{\tg} \ge 0 \quad \text{in } \mathcal{D}'(\tM).
\end{equation}
More precisely, the distributional scalar curvature decomposes as:
\begin{equation}
    R_{\tg} = R_{\tg}^{\mathrm{reg}} + 2[H]_{\tg} \cdot \mathcal{H}^2|_\Sigma,
\end{equation}
where:
\begin{enumerate}
    \item $R_{\tg}^{\mathrm{reg}} \ge 0$ a.e.\ on $\tM \setminus \Sigma$ (from the DEC and Lichnerowicz equation);
    \item $[H]_{\tg} = \phi^{-2}|_\Sigma \cdot [H]_{\bg} \ge 0$ (from Theorem~\ref{thm:CompleteMeanCurvatureJump} and conformal invariance).
\end{enumerate}
\end{corollary}

\begin{proof}
This is a direct consequence of Theorem~\ref{thm:AMOHypothesisVerification}, hypothesis (H2). The decomposition follows from Lemma~\ref{lem:JangScalar} and the conformal transformation formula. The sign conditions are established in Steps 2a--2d of the proof of Theorem~\ref{thm:AMOHypothesisVerification}.
\end{proof}

\begin{remark}[Explicit Connection: Capacity Removability $\Rightarrow$ AMO Monotonicity Extension]\label{rem:CapacityAMOConnection}
We provide explicit justification for why zero $p$-capacity of the singular tips $\{p_k\}$ implies that the AMO monotonicity formula extends to the singular target manifold $(\tM, \tg)$.

\textbf{(1) The AMO functional and its integrand.} The AMO monotonicity functional is:
\begin{equation}
    \mathcal{M}_p(t) = \left( \frac{p-1}{p} \right)^{(p-1)/p} \left( \int_{\{u_p = t\}} |\nabla u_p|^{p-1} \, d\sigma \right)^{1/p},
\end{equation}
where $u_p$ is the $p$-harmonic potential with $u_p = 0$ on $\Sigma$ and $u_p \to 1$ at infinity. The monotonicity relies on the identity:
\begin{equation}\label{eq:AMOMonotonicityDerivative}
    \frac{d}{dt} \mathcal{M}_p(t)^p = \text{(bulk integral over } \{u_p > t\}\text{)} + \text{(boundary contribution from } \partial\{u_p > t\}\text{)}.
\end{equation}

\textbf{(2) Role of capacity in the boundary contribution.} The boundary of $\{u_p > t\}$ consists of:
\begin{itemize}
    \item The level set $\{u_p = t\}$ (regular part);
    \item Potentially, the singular tips $\{p_k\}$ if $u_p(p_k) > t$.
\end{itemize}
The key question is: \emph{do the singular tips contribute to the monotonicity formula?}

\textbf{(3) Capacity controls the boundary flux.} The contribution from a singular point $p_k$ to the divergence theorem is:
\begin{equation}
    \lim_{r \to 0} \int_{\partial B_r(p_k)} |\nabla u_p|^{p-2} \langle \nabla u_p, \nu \rangle \, d\sigma.
\end{equation}
By the defining property of $p$-capacity, for sets $E$ with $\mathrm{Cap}_p(E) = 0$:
\begin{equation}
    \int_E |\nabla u_p|^{p-2} \langle \nabla u_p, \nabla \eta \rangle \, dV = 0
\end{equation}
for all test functions $\eta$. This implies the flux through shrinking spheres around $p_k$ vanishes:
\begin{equation}
    \lim_{r \to 0} \int_{\partial B_r(p_k)} |\nabla u_p|^{p-2} \partial_r u_p \, d\sigma = 0.
\end{equation}

\textbf{(4) Quantitative estimate.} More explicitly, the conical asymptotics $\tg \sim r^{4\alpha} g_{\text{flat}}$ give:
\begin{itemize}
    \item Volume element: $dV \sim r^{2+6\alpha} dr \, d\omega$;
    \item Gradient: $|\nabla u_p| \sim r^{-2\alpha} |\nabla_{\text{flat}} u_p|$ (accounting for metric scaling);
    \item Area element: $d\sigma \sim r^{2+4\alpha} d\omega$ on $\partial B_r$.
\end{itemize}
The boundary flux scales as:
\begin{equation}
    \int_{\partial B_r(p_k)} |\nabla u_p|^{p-1} \, d\sigma \sim r^{(2+4\alpha) - (p-1) \cdot 2\alpha} = r^{2 + (6-2p)\alpha}.
\end{equation}
Since $\alpha > 0$ and $p < 3$, we have $6 - 2p > 0$, so this vanishes as $r \to 0$.

\textbf{(5) Consequence for AMO monotonicity.} The vanishing flux implies:
\begin{enumerate}
    \item The $p$-harmonic function $u_p$ is well-defined on all of $\tM$ (by the removability theorem);
    \item The divergence theorem identity \eqref{eq:AMOMonotonicity} holds with no singular contributions;
    \item The level sets $\{u_p = t\}$ are regular surfaces for a.e.\ $t$ (by the co-area formula and Sard's theorem);
    \item The AMO monotonicity $\mathcal{M}_p(t_1) \le \mathcal{M}_p(t_2)$ for $t_1 < t_2$ extends to $(\tM, \tg)$.
\end{enumerate}

\textbf{(6) Verification via approximation.} The argument proceeds by:
\begin{enumerate}
    \item[(i)] Apply AMO on smooth $(\tM, \hat{g}_\epsilon)$: $\mathcal{M}_{p,\epsilon}(t)$ is monotone increasing in $t$.
    \item[(ii)] Pass $\epsilon \to 0$: $\mathcal{M}_{p,\epsilon}(t) \to \mathcal{M}_{p,0}(t)$ by Mosco convergence (Theorem~\ref{thm:CompleteDblLimit}).
    \item[(iii)] Monotonicity is preserved in the limit: $\mathcal{M}_{p,0}(t_1) \le \mathcal{M}_{p,0}(t_2)$ for $t_1 < t_2$.
\end{enumerate}
The capacity-zero condition ensures step (ii) holds: the singular tips do not create ``leakage'' in the variational convergence.

\textbf{(7) Summary.} The logical chain is:
\begin{center}
$\mathrm{Cap}_p(\{p_k\}) = 0$ $\Rightarrow$ $u_p$ extends across $\{p_k\}$ $\Rightarrow$ flux vanishes at tips $\Rightarrow$ AMO identity holds $\Rightarrow$ monotonicity extends.
\end{center}
This justifies Corollary~\ref{cor:AMOApplies}: the capacity removability is not merely a technical convenience but the \emph{mechanism} by which AMO monotonicity transfers from smooth approximants to the singular target.
\end{remark}

\begin{theorem}[Uniform $C^{1,\alpha}$ Estimates for $p$-Harmonic Functions Across Lipschitz Interfaces]\label{thm:UniformPHarmonicRegularity}
Let $(\tM, \tg)$ be the conformally deformed Jang manifold with Lipschitz interface $\Sigma$, and let $\hat{g}_\epsilon$ be the smoothed metrics for $\epsilon \in (0, \epsilon_0)$. For each $p \in (1,3)$ and $\epsilon > 0$, let $u_{p,\epsilon}$ be the $p$-harmonic function on $(\tM, \hat{g}_\epsilon)$ with boundary conditions $u_{p,\epsilon} = 0$ on $\Sigma$ and $u_{p,\epsilon} \to 1$ at infinity.

Then there exist constants $\alpha \in (0,1)$ and $C > 0$, \textbf{independent of $\epsilon$}, such that:
\begin{enumerate}
    \item \textbf{Uniform $C^{1,\alpha}$ bound:} For any compact set $K \Subset \tM \setminus \{p_k\}$ (away from bubble tips),
    \begin{equation}\label{eq:UniformC1alpha}
        \|u_{p,\epsilon}\|_{C^{1,\alpha}(K)} \le C_K \quad \text{uniformly in } \epsilon \in (0, \epsilon_0).
    \end{equation}
    
    \item \textbf{Uniform gradient bound near the interface:} In a collar neighborhood $N_\delta = (-\delta, \delta) \times \Sigma$ of the interface,
    \begin{equation}\label{eq:UniformGradientNearInterface}
        \|\nabla u_{p,\epsilon}\|_{L^\infty(N_\delta \cap K)} \le C_{K,\delta} \quad \text{uniformly in } \epsilon \in (0, \min(\epsilon_0, \delta/2)).
    \end{equation}
    
    \item \textbf{Convergence across the interface:} As $\epsilon \to 0$, the functions $u_{p,\epsilon}$ converge strongly in $W^{1,p}_{\mathrm{loc}}(\tM)$ and uniformly in $C^{1,\beta}_{\mathrm{loc}}(\tM \setminus \{p_k\})$ for any $\beta < \alpha$ to a limit function $u_p$ that is weakly $p$-harmonic on $(\tM, \tg)$.
    
    \item \textbf{Transmission regularity:} The limit function $u_p$ is $C^{1,\alpha}$ \textbf{across} the Lipschitz interface $\Sigma$, with no jump in the function value or its conormal derivative:
    \begin{equation}
        [u_p]_\Sigma = 0, \quad \left[\tg^{ij} \partial_j u_p \nu_i \right]_\Sigma = 0.
    \end{equation}
\end{enumerate}
\end{theorem}

\begin{proof}
The proof proceeds in four steps, establishing uniform estimates that are necessary for the limit passage.

\textbf{Step 1: Uniform ellipticity and Caccioppoli inequality.}
The smoothed metrics $\hat{g}_\epsilon$ are uniformly elliptic with constants independent of $\epsilon$:
\begin{equation}
    \lambda_{\min} |\xi|^2 \le \hat{g}_\epsilon(\xi, \xi) \le \lambda_{\max} |\xi|^2,
\end{equation}
where $\lambda_{\min}, \lambda_{\max}$ depend only on $\tg$ (not on $\epsilon$). This follows from the bi-Lipschitz estimate in Theorem~\ref{thm:GlobalBiLipschitz}: $(1-C\epsilon)\tg \le \hat{g}_\epsilon \le (1+C\epsilon)\tg$.

The Caccioppoli inequality for $p$-harmonic functions gives: for any ball $B_{2r} \subset \tM$,
\begin{equation}
    \int_{B_r} |\nabla u_{p,\epsilon}|^p \, dV_{\hat{g}_\epsilon} \le \frac{C}{r^p} \int_{B_{2r}} |u_{p,\epsilon}|^p \, dV_{\hat{g}_\epsilon}.
\end{equation}
Since $0 \le u_{p,\epsilon} \le 1$ (by the maximum principle), this yields
\begin{equation}
    \int_{B_r} |\nabla u_{p,\epsilon}|^p \, dV \le C r^{n-p}
\end{equation}
with $C$ independent of $\epsilon$.

\textbf{Step 2: De Giorgi--Nash--Moser estimates.}
By the De Giorgi--Nash--Moser theorem for quasilinear elliptic equations with bounded measurable coefficients (Theorem 8.22 of \cite{gilbarg2001} and the extensions to $p$-Laplacian by \cite{dibenedetto1993}), weak solutions to the $p$-Laplace equation satisfy:
\begin{equation}
    \sup_{B_r} |u_{p,\epsilon}| \le C \left( \fint_{B_{2r}} |u_{p,\epsilon}|^p \right)^{1/p}
\end{equation}
and the H\"older estimate
\begin{equation}
    |u_{p,\epsilon}(x) - u_{p,\epsilon}(y)| \le C \left( \frac{|x-y|}{r} \right)^\alpha \sup_{B_r} |u_{p,\epsilon}|
\end{equation}
for $x, y \in B_{r/2}$, where $\alpha > 0$ depends only on $p, n$, and the ellipticity ratio $\lambda_{\max}/\lambda_{\min}$.

Since the ellipticity ratio is bounded uniformly in $\epsilon$, the constants $C$ and $\alpha$ are also uniform in $\epsilon$.

\textbf{Step 3: Tolksdorf gradient estimates.}
By Tolksdorf's regularity theorem \cite{tolksdorf1984}, $p$-harmonic functions on Lipschitz domains satisfy interior $C^{1,\alpha}$ estimates:
\begin{equation}
    \|\nabla u_{p,\epsilon}\|_{C^{0,\alpha}(B_{r/2})} \le \frac{C}{r} \|u_{p,\epsilon}\|_{L^\infty(B_r)}.
\end{equation}
The constant $C$ depends on the ellipticity constants, dimension, and $p$, but \textbf{not} on the smoothness of the coefficients beyond Lipschitz. Since $\hat{g}_\epsilon$ has uniformly bounded Lipschitz constant (indeed, $\hat{g}_\epsilon$ is smooth with derivatives bounded uniformly for $\epsilon$ bounded away from zero, and approaches the Lipschitz metric $\tg$ as $\epsilon \to 0$), the estimate is uniform in $\epsilon$.

\textbf{Step 4: Transmission across the interface.}
The interface $\Sigma$ in the smoothed metric $\hat{g}_\epsilon$ is a smooth hypersurface (since $\hat{g}_\epsilon$ is smooth). The $p$-harmonic function $u_{p,\epsilon}$ satisfies $\Delta_p u_{p,\epsilon} = 0$ classically on all of $\tM$.

As $\epsilon \to 0$, the interface ``sharpens'' to the Lipschitz junction in $\tg$. By the uniform bounds from Steps 2--3, the Arzel\`a--Ascoli theorem gives $u_{p,\epsilon} \to u_p$ in $C^{1,\beta}_{\mathrm{loc}}$ for $\beta < \alpha$. The limit $u_p$ is weakly $p$-harmonic on $(\tM, \tg)$.

For the transmission conditions, we use the weak formulation. For any test function $\psi \in C^\infty_c(\tM)$:
\begin{equation}
    \int_{\tM} |\nabla u_{p,\epsilon}|^{p-2} \langle \nabla u_{p,\epsilon}, \nabla \psi \rangle_{\hat{g}_\epsilon} \, dV_{\hat{g}_\epsilon} = 0.
\end{equation}
Passing $\epsilon \to 0$ using the uniform bounds and dominated convergence:
\begin{equation}
    \int_{\tM} |\nabla u_p|^{p-2} \langle \nabla u_p, \nabla \psi \rangle_{\tg} \, dV_{\tg} = 0.
\end{equation}
This holds for all $\psi$, including those with support crossing $\Sigma$. By splitting the integral over $\Omega^+$ and $\Omega^-$ and applying the divergence theorem:
\begin{equation}
    \int_\Sigma \psi \left[ |\nabla u_p|^{p-2} \partial_\nu u_p \right]_\Sigma \, d\sigma = 0 \quad \forall \psi.
\end{equation}
Hence $\left[ |\nabla u_p|^{p-2} \partial_\nu u_p \right]_\Sigma = 0$. Since $|\nabla u_p| > 0$ on $\Sigma$ (the gradient cannot vanish on the boundary where $u_p = 0$), this implies $[\partial_\nu u_p]_\Sigma = 0$. Combined with continuity $[u_p]_\Sigma = 0$ from the $C^0$ convergence, we obtain $C^{1,\alpha}$ regularity across $\Sigma$.
\end{proof}

\begin{remark}[Interaction of Dirac Curvature with Level Sets]\label{rem:DiracLevelSetInteraction}
A potential concern is whether the Dirac measure $2[H]\delta_\Sigma$ in the distributional scalar curvature $R_{\tg}$ could interact badly with the level sets of $u_p$. We clarify why this does not occur:
\begin{enumerate}
    \item \textbf{Level sets are transverse to $\Sigma$:} By the boundary condition $u_p|_\Sigma = 0$ and the non-vanishing of $|\nabla u_p|$ near $\Sigma$ (from the maximum principle and Hopf lemma), the level sets $\{u_p = t\}$ for small $t > 0$ are surfaces \emph{parallel} to $\Sigma$, not intersecting it transversally.
    
    \item \textbf{AMO integrates over level sets, not over $\Sigma$:} The AMO monotonicity functional $\mathcal{M}_p(t)$ is an integral over the level set $\{u_p = t\}$. For $t > 0$, this level set is contained in $\tM \setminus \Sigma$, where the metric is smooth and the scalar curvature is a classical function.
    
    \item \textbf{The Dirac term contributes positively to the distributional bound:} When we test $R_{\tg} \ge 0$ against a nonnegative function $\psi$, the Dirac contribution $2[H] \int_\Sigma \psi \, d\sigma \ge 0$ (since $[H] \ge 0$) \emph{helps} rather than hurts. It does not create a negative contribution that would obstruct the monotonicity.
    
    \item \textbf{Smoothing separates the scales:} On the smoothed metric $\hat{g}_\epsilon$, the ``smeared'' Dirac term $\frac{2[H]}{\epsilon}\eta(s/\epsilon)$ contributes large \emph{positive} curvature in the collar $N_{2\epsilon}$. The level sets for small $t$ may pass through this collar, but they see only nonnegative curvature contributions.
\end{enumerate}
In summary, the Dirac measure in $R_{\tg}$ is geometrically localized on $\Sigma$, which is the $t=0$ level set (the boundary). The AMO analysis operates on level sets for $t > 0$, which avoid the singular support of the measure.
\end{remark}

\begin{remark}[Explicit Approximation Scheme for AMO Application]\label{rem:AMOApproximation}
We emphasize the logical structure of applying AMO to our singular metric, as this is a source of potential confusion.

	extbf{The Problem:} The AMO monotonicity theorem \cite{amo2024} is \emph{stated} for smooth complete asymptotically flat 3-manifolds with $R \ge 0$ and compact minimal boundary. Our metric $(\tM, \tg)$ is:
\begin{itemize}
    \item \emph{Lipschitz} (not smooth) across the interface $\Sigma$;
    \item \emph{$C^0$} (not smooth) at the bubble tips $\{p_k\}$;
    \item Has \emph{distributional} scalar curvature $R_{\tg} = R^{reg} + 2[H]\delta_\Sigma$.
\end{itemize}

\textbf{The Solution:} We \emph{never} apply AMO directly to $(\tM, \tg)$. Instead:
\begin{enumerate}
    \item[(i)] \textbf{Smoothing:} Construct a family $(\tM, \hat{g}_\epsilon)$ where:
    \begin{itemize}
        \item $\hat{g}_\epsilon$ is \emph{smooth} on all of $\tM$;
        \item $\hat{g}_\epsilon = \tg$ outside the $\epsilon$-collar around $\Sigma$ and the $\epsilon$-balls around $\{p_k\}$;
        \item The scalar curvature satisfies $R_{\hat{g}_\epsilon} \ge -K_\epsilon$ with $K_\epsilon = O(\epsilon^{1/2})$ in $L^{3/2}$;
        \item All AMO hypotheses are satisfied for $\hat{g}_\epsilon$.
    \end{itemize}
    
    \item[(ii)] \textbf{Apply AMO:} For each $\epsilon > 0$ and $p \in (1,3)$, the AMO monotonicity gives:
    \[
        M_{\ADM}(\hat{g}_\epsilon) \ge \sqrt{\frac{A_{\hat{g}_\epsilon}(\Sigma_\epsilon)}{16\pi}}
    \]
    where $\Sigma_\epsilon$ is the outermost minimal surface in $(\tM, \hat{g}_\epsilon)$.
    
    \item[(iii)] \textbf{Take limits:} Use the uniform estimates to pass $\epsilon \to 0$:
    \begin{itemize}
        \item $M_{\ADM}(\hat{g}_\epsilon) \to M_{\ADM}(\tg)$ by ADM mass continuity under $C^{0,1}$ convergence;
        \item $A_{\hat{g}_\epsilon}(\Sigma_\epsilon) \to A_{\tg}(\Sigma)$ by area stability and homology preservation;
        \item The error from $R_{\hat{g}_\epsilon}^- < 0$ in the collar is controlled by $\|R^-\|_{L^{3/2}} \to 0$.
    \end{itemize}
\end{enumerate}

\textbf{Why This Works:} The key observation is that all problematic features of $(\tM, \tg)$ are \emph{localized}:
\begin{itemize}
    \item The Lipschitz interface $\Sigma$ is a 2-dimensional surface (codimension 1);
    \item The $C^0$ tips $\{p_k\}$ are points (codimension 3);
    \item The distributional curvature concentrates on a set of zero Lebesgue measure.
\end{itemize}
The smoothing can be done in an arbitrarily small neighborhood of these loci, and the resulting error terms are controlled uniformly in $\epsilon$ by the collar bounds and capacity estimates.

\textbf{Comparison with Direct Extension of AMO:} One could alternatively try to \emph{extend} the AMO theorem to cover Lipschitz metrics with distributional curvature directly. This would require:
\begin{itemize}
    \item A weak formulation of the Bochner identity for Lipschitz metrics;
    \item Analysis of the error terms from distributional curvature;
    \item Verification that level sets avoid the singular locus.
\end{itemize}
While this is likely possible, the approximation approach is cleaner: it separates the ``PDE analysis'' (on smooth metrics) from the ``convergence analysis'' (uniform bounds and limit passage), making each step easier to verify.
\end{remark}

\subsection{Mosco convergence and area stability: a summary}\label{sec:MoscoSummary}
We record the hypotheses and conclusions needed to pass the Riemannian Penrose inequalities from smooth approximants $(\widetilde M, \hat g_\epsilon)$ to the singular target $(\widetilde M, \widetilde g)$:
\begin{itemize}
    \item \textbf{Uniform ellipticity and metric convergence.} The smoothed metrics $\hat g_\epsilon$ converge to $\widetilde g$ in $C^0_{\mathrm{loc}}$ and are uniformly elliptic with constants independent of $\epsilon$ on compact subsets. This ensures stability of weak solutions and energy functionals.
    \item \textbf{$L^{3/2}$ control of $R^-$.} Inside the smoothing collar $N_{2\epsilon}$, the negative part of scalar curvature satisfies $\|R_{\hat g_\epsilon}^-\|_{L^{3/2}(N_{2\epsilon})}\to 0$ (Proposition~\ref{prop:CollarBound}, Corollary~\ref{cor:L32}). This controls error terms in Bochner-type identities and guarantees compatibility with AMO.
    \item \textbf{Mosco convergence of $p$-energies.} For $1<p<3$, the functionals $E_{p,\epsilon}(u)=\int |\nabla u|_{\hat g_\epsilon}^p$ Mosco-converge to $E_p(u)=\int |\nabla u|_{\widetilde g}^p$, so minimizers/subsolutions converge in $W^{1,p}$ and level set foliation properties persist.
    \item \textbf{Area stability and homology.} Outermost minimal surfaces $\Sigma_\epsilon$ in $(\widetilde M,\hat g_\epsilon)$ are homologous to $\Sigma$ and satisfy $A_{\hat g_\epsilon}(\Sigma_\epsilon) \to A_{\widetilde g}(\Sigma)$, by calibration on the limiting cylinder and metric comparison in the collar.
    \item \textbf{ADM mass continuity.} Under AF decay $\tau>1$ and $C^{0,1}$ convergence of coefficients, the ADM mass of $\hat g_\epsilon$ converges to the ADM mass of $\widetilde g$ (cf. Bartnik \cite{bartnik1986}; Chru\'sciel--Herzlich \cite{chruscielherrzlich2003}), allowing identification of the mass in the limit.
\end{itemize}
With these ingredients, the AMO monotonicity and identification claims for $(\widetilde M,\hat g_\epsilon)$ pass to $(\widetilde M,\widetilde g)$ via the double limit $p\to 1^+$ then $\epsilon\to 0$.

