\section{Synthesis: Limit of Inequalities}
\label{sec:Synthesis}

\subsection{Limit of Inequalities via Mosco Convergence}
To upgrade the classical Riemannian Penrose Inequality from the smooth setting to the Lipschitz geometry of $(\tM,\tg)$, we approximate the singular metric by the refined smoothing family $(\tM,\geps)$ introduced in \S\ref{sec:Construction}. The parameters of the corner smoothing are tuned so that $\geps$ agrees with $\tg$ away from the collar $N_{2\epsilon}$, the average scalar curvature inside the collar is improved, and the Miao-style uniform isoperimetric inequality persists with constants independent of $\epsilon$. In particular, every outermost minimal surface $\Sigma_{\min,\epsilon}$ on $(\tM,\geps)$ remains homologous to the original horizon and satisfies the quantitative area lower bound of Theorem~\ref{thm:AreaStability}. Running the AMO monotonicity formula on each smooth approximant yields
\[
    M_{\ADM}(\geps) \ge \sqrt{\frac{A_{\geps}(\Sigma_{\min,\epsilon})}{16\pi}}.
\]
The Mosco convergence of the $p$-energy functionals (Theorem~\ref{thm:MoscoConvergence}) guarantees that the $p$-capacitary potentials and their Hawking mass profiles converge strongly as $\epsilon \to 0$, so no energy is lost across the Lipschitz interface. Combined with the convergence of the ADM mass (Lemma~\ref{lem:MassContinuity}) and the area stability estimate, we can safely pass to the limit in the inequality and recover $M_{\ADM}(\tg) \ge \sqrt{A(\Sigma)/16\pi}$ without interchanging the geometric and variational limits.

\begin{theorem}[The Spacetime Penrose Inequality --- Conditional (Restatement)]
Let $(M, g, k)$ be an asymptotically flat initial data set satisfying the Dominant Energy Condition. Let $\Sigma_0$ be any closed trapped surface (satisfying $\theta^+ \le 0$). 
\textbf{Assume one of the following conditions holds:}
\begin{enumerate}
    \item \textbf{Favorable Jump:} $\tr_{\Sigma_0} k \ge 0$ pointwise;
    \item \textbf{Compactness:} Conditions (C1)--(C3) of Theorem~\ref{thm:MaxAreaTrapped} (conditional on Conjecture C for $k \neq 0$);
    \item \textbf{Cosmic Censorship:} The data embeds in a spacetime satisfying WCC.
\end{enumerate}
Then $M_{\ADM}(g) \ge \sqrt{A(\Sigma_0)/16\pi}$.
\end{theorem}

\begin{proof}[Proof of Theorem \ref{thm:SPI} (The Spacetime Penrose Inequality)]
We assume the initial data $(M,g,k)$ satisfies the DEC.

\begin{enumerate}
\item \textbf{Fixed $\epsilon$ Step:} For fixed $\epsilon > 0$, $(\tM, \geps)$ is smooth with nonnegative scalar curvature. The standard AMO result applies:
   \[ M_{\ADM}(\geps) \ge \sqrt{\frac{A(\Sigma_{\min, \epsilon})}{16\pi}}. \]
\item \textbf{Limit Step:} We take $\epsilon \to 0$.
   \begin{itemize}
       \item LHS: By Lemma \ref{lem:MassContinuity}, $M_{\ADM}(\geps) \to M_{\ADM}(\tg)$.
    \item RHS: By Theorem \ref{thm:AreaStability}, $\liminf A(\Sigma_{\min, \epsilon}) \ge A(\Sigma)$.
       \item \textbf{Justification via Mosco Convergence:} The validity of passing the inequality to the limit relies on the convergence of the energies. As established in Theorem \ref{thm:MoscoConvergence}, the functional $\mathcal{E}_{p,\geps}$ Mosco-converges to $\mathcal{E}_{p,\tg}$. This variational convergence ensures that the $p$-capacitary potentials (and thus their level set masses) converge continuously, preventing any jump in the Hawking mass profile as $\epsilon \to 0$.
   \end{itemize}
   Combining these yields $M_{\ADM}(\tg) \ge \sqrt{A(\Sigma)/16\pi}$.
\end{enumerate}

\begin{remark}[Order of Limits and Moore--Osgood Verification]
We emphasize that we do \emph{not} naively interchange the limits $p \to 1$ and $\epsilon \to 0$. We derive the Riemannian Penrose Inequality for the smooth manifold $(\tM, \geps)$ for fixed $\epsilon$ (taking $p \to 1$ first), and only \emph{then} take the geometric limit $\epsilon \to 0$ of the resulting inequality. This avoids the analytical difficulties of defining the $p$-harmonic flow on the Lipschitz manifold directly.

\textbf{Rigorous justification via Moore--Osgood:} The interchange of limits is \emph{rigorously justified} by the Moore--Osgood theorem (Lemma~\ref{lem:MooreOsgood} in Section~\ref{sec:Analysis}). The three hypotheses are:
\begin{enumerate}
    \item[(MO1)] For each fixed $\epsilon$, $\lim_{p \to 1^+} f(p,\epsilon)$ exists (standard AMO theory on smooth manifolds).
    \item[(MO2)] The convergence is \textbf{uniform} in $\epsilon$: the rate $|\mathcal{M}_{p,\epsilon}(\Sigma) - \sqrt{A/(16\pi)}| \le C_A(p-1)^{1/2}$ is independent of $\epsilon$ by Tolksdorf--Lieberman gradient estimates with uniform ellipticity (Lemma~\ref{lem:UniformEllipticity}).
    \item[(MO3)] For each fixed $p$, $\lim_{\epsilon \to 0^+} f(p,\epsilon)$ exists (Mosco convergence, Theorem~\ref{thm:MoscoConvergence}).
\end{enumerate}
See Theorem~\ref{thm:CompleteDblLimit} for the complete statement and quantitative bounds.
\end{remark}

The rigorous proof of the bound $\phi \le 1$ (\Cref{thm:PhiBound}) guarantees the mass reduction during the conformal deformation (\Cref{thm:MassReduction}), so $M_{\ADM}(\bg) \ge M_{\ADM}(\tg)$.
Combining this with the mass reduction property of the Jang map ($M_{\ADM}(g) \ge M_{\ADM}(\bg)$) and the area preservation, we obtain:
\[ M_{\ADM}(g) \ge \sqrt{\frac{A(\Sigma)}{16\pi}}. \]
This completes the proof.
\end{proof}

\begin{remark}[Verification Points]\label{rem:VerificationSummary}
The proof hinges on four points:

\begin{enumerate}
    \item \textbf{Verification of AMO Hypotheses (Theorem~\ref{thm:AMOHypothesisVerification}):}
    The singular Jang-conformal metric $(\tM, \tg)$, despite being only Lipschitz with measure-valued curvature, satisfies all requirements for the AMO monotonicity formula. We apply AMO to the smoothed metrics $\hat{g}_\epsilon$ (which are smooth with $R \ge 0$) and take $\epsilon \to 0$ via Mosco convergence. The uniform bounds in Theorem~\ref{thm:CompleteDblLimit} justify this passage.
    
    \item \textbf{The Double Limit Interchange (Theorem~\ref{thm:CompleteDblLimit}):}
    Justifying $(p, \epsilon) \to (1^+, 0)$ requires establishing uniform bounds independent of both $p$ and $\epsilon$. The estimate $|E_{p,\epsilon} - E_p| \le C\epsilon^{1/2}$ (uniform in $p$) derives from: (i) $\Vol(N_{2\epsilon}) = O(\epsilon)$; (ii) Tolksdorf gradient bounds for $p$-harmonic functions; (iii) Lieberman regularity theory extending these bounds to Lipschitz interfaces. The stability of mass and area under smoothing (Miao adaptation) is verified in Appendix~\ref{app:InternalSmoothing}.
    
    \item \textbf{Mean Curvature Jump Positivity (Lemma~\ref{lem:TrappedMeanCurvatureJump} and Theorem~\ref{thm:CompleteMeanCurvatureJump}):}
    The Miao corner formula gives $[H]_{\bar{g}} = \tr_{\Sigma_0} k$ at any trapped surface. Thus, the favorable jump condition $\tr_{\Sigma_0} k \ge 0$ is required to ensure $[H]_{\bar{g}} \ge 0$. This condition is an \emph{additional hypothesis} (not implied by $\theta^\pm$ conditions alone).
    
    \item \textbf{Regularity Across the Interface (Lemma~\ref{lem:InterfaceRegularity}):}
    The conformal factor $\phi$ is $C^{1,\alpha}$ across the Lipschitz interface $\Sigma_0$ because the PDE potential $V = \frac{1}{8}R^{reg} - \frac{1}{4}\Div(q)$ does not contain the Dirac mass $2[H]\delta_{\Sigma_0}$. The singular term contributes to the geometric scalar curvature $R_{\tg}$ (favorably, since $[H] \ge 0$), not to the elliptic PDE.
\end{enumerate}
\end{remark}

\begin{remark}[Potential Failure Modes]\label{rem:FailureModes}
To aid verification, we explicitly describe what would fail in hypothetical scenarios where the proof contains an error. This ``counter-example thinking'' helps identify the most critical steps for independent verification.

\textbf{Failure Mode 1: Incorrect Mean Curvature Jump Sign.}
If the favorable jump condition were not satisfied ($[H]_{\bg} < 0$), the distributional scalar curvature could be negative as a measure: $R_{\tg} = R_{\tg}^{reg} + 2[H]_{\tg}\delta_{\Sigma_0}$ with $[H]_{\tg} < 0$. In this case:
\begin{itemize}
    \item The AMO monotonicity would fail, since $\mathcal{M}_p'(t) \ge 0$ requires $R \ge 0$ distributionally.
    \item For general trapped surfaces (not stable MOTS), the favorable jump condition $\tr_{\Sigma_0} k \ge 0$ is an \emph{additional hypothesis}---it is \textbf{not} implied by $\theta^\pm \le 0$ alone. Counterexample: $H = -3$, $\tr k = -1$ gives $\theta^+ = -4$, $\theta^- = -2$ (both trapped), but $\tr k = -1 < 0$.
\end{itemize}

\textbf{Failure Mode 2: Conformal Factor Exceeding 1.}
If the proof of $\phi \le 1$ (Theorem~\ref{thm:PhiBound}) were incorrect, then $\phi > 1$ somewhere would imply $M_{\ADM}(\tg) > M_{\ADM}(\bg)$, reversing the mass reduction. This would happen if:
\begin{itemize}
    \item The Bray-Khuri divergence identity failed due to singularities in the vector field $Y$.
    \item The boundary terms at infinity or bubble tips had incorrect signs.
\end{itemize}
A counter-example would require constructing initial data where the conformal solution exceeds 1, which would violate the Lichnerowicz maximum principle in regions where $R_{\bg} \ge 0$. Since we allow $R_{\bg} < 0$ (from DEC deficit), the integral method is essential.

\textbf{Failure Mode 3: Loss of Area in Smoothing.}
If the area stability (Theorem~\ref{thm:AreaStability}) failed, the minimal area could shrink as $\epsilon \to 0$: $\liminf A(\Sigma_{\min,\epsilon}) < A(\Sigma)$. This would happen if:
\begin{itemize}
    \item The smoothing introduced new minimal surfaces with smaller area.
    \item The isoperimetric profile degraded as $\epsilon \to 0$.
\end{itemize}
The Miao smoothing technique explicitly prevents this by controlling the scalar curvature in $L^{3/2}$.

\textbf{Failure Mode 4: Non-Convergence of Mosco Limit.}
If Mosco convergence (Theorem~\ref{thm:MoscoConvergence}) failed, the $p$-harmonic potentials $u_{p,\epsilon}$ might not converge as $\epsilon \to 0$, causing the Hawking mass profile to jump. This would require:
\begin{itemize}
    \item The Sobolev embedding $W^{1,p} \hookrightarrow L^{p^*}$ to fail for the limiting metric.
    \item The variational problem to have non-unique minimizers in the limit.
\end{itemize}
The uniform ellipticity of $\tg$ away from measure-zero sets prevents this.

\textbf{Critical Steps:}
The following results represent the essential components of the proof:
\begin{enumerate}
    \item \textbf{Theorem~\ref{thm:DirectTrappedJang}} (Jang Reduction for MOTS) -- Jang blow-up at MOTS with favorable jump $\tr_\Sigma k \ge 0$.
    \item \textbf{Lemma~\ref{lem:TrappedMeanCurvatureJump}} (mean curvature jump formula) -- $[H] = \tr_\Sigma k$; favorable jump condition $\tr_\Sigma k \ge 0$ required for $[H] \ge 0$.
    \item \textbf{Theorem~\ref{thm:PhiBound}} (conformal factor bound) -- the key analytic bottleneck.
    \item \textbf{Lemma~\ref{lem:InterfaceRegularity}} (transmission regularity) -- ensures PDEs are well-posed.
    \item \textbf{Theorem~\ref{thm:MoscoConvergence}} (variational convergence) -- validates the limit passage.
    \item \textbf{Proposition~\ref{prop:DegeneratePI}} (degenerate trapped surfaces) -- perturbation for $\theta^- = 0$ case.
\end{enumerate}
\end{remark}

\begin{lemma}[Uniform Ellipticity Bounds for Double Limit]\label{lem:UniformEllipticityBounds}
Let $\{(\hat{g}_\epsilon, \tM_\epsilon)\}_{\epsilon \in (0,\epsilon_0]}$ be the family of smoothed manifolds from Theorem~\ref{thm:MiaoSmoothing}. The following uniform bounds hold independently of $\epsilon$:
\begin{enumerate}
    \item \textbf{Metric regularity:} $\|\hat{g}_\epsilon\|_{C^{0,1}(\tM_\epsilon)} \le C_0$ for a constant $C_0$ depending only on $(M,g,k)$.
    
    \item \textbf{Uniform ellipticity:} There exist $0 < \lambda_{\min} \le \lambda_{\max} < \infty$ independent of $\epsilon$ such that
    \[
    \lambda_{\min}|\xi|^2 \le \hat{g}_\epsilon^{ij}\xi_i\xi_j \le \lambda_{\max}|\xi|^2, \quad \forall \xi \in T\tM_\epsilon.
    \]
    
    \item \textbf{$p$-Laplacian coefficient bounds:} For the $p$-Laplacian operator $\Delta_p u = \div(|\nabla u|^{p-2}\nabla u)$ on $(\tM_\epsilon, \hat{g}_\epsilon)$, the structure constants in the Tolksdorf--Lieberman estimates~\cite{tolksdorf1984} are uniformly bounded:
    \[
    C_{\mathrm{TL}}(p, \hat{g}_\epsilon) \le C_1(1 + (p-1)^{-1}), \quad \forall p \in (1,2], \; \epsilon \in (0,\epsilon_0].
    \]
    
    \item \textbf{Boundary regularity:} The smoothed minimal surface $\partial\tM_\epsilon$ has principal curvatures $|\kappa_i| \le C_2\epsilon^{-1}$ near the smoothing region, but the $p$-harmonic functions satisfy Neumann conditions with uniform flux estimates $|\nabla u_p \cdot \nu| \le C_3$.
\end{enumerate}
\end{lemma}

\begin{proof}
(1)--(2): The Miao smoothing replaces the corner region $\{|s| < \epsilon\}$ with a smooth interpolation preserving $R \ge 0$. By explicit construction (Proposition~\ref{prop:UniformIsoperimetry}), the interpolated metric components satisfy $\hat{g}_\epsilon^{ij} = g^{ij} + O(\epsilon)$ outside the corner, and the corner region itself has controlled geometry from the prescribed interpolation profile.

(3): The Tolksdorf--Lieberman gradient estimates~\cite[Theorem~1.7]{tolksdorf1984} for $p$-harmonic functions depend on:
\begin{itemize}
    \item The ellipticity ratio $\lambda_{\max}/\lambda_{\min}$, which is uniformly bounded by (2);
    \item The $C^{0,1}$ norm of the metric, bounded by (1);
    \item The domain geometry, which is controlled by the uniform smoothing construction.
\end{itemize}
The explicit dependence on $p$ involves $(p-1)^{-1}$ terms that degenerate as $p \to 1^+$, but this is absorbed into the capacity functional convergence analysis.

(4): The principal curvature bound $O(\epsilon^{-1})$ is localized to a region of measure $O(\epsilon^2)$, so the integrated boundary contributions remain $O(\epsilon)$ and vanish in the limit.
\end{proof}

\begin{corollary}[Moore--Osgood Applicability]\label{cor:MooreOsgood}
The uniform bounds of Lemma~\ref{lem:UniformEllipticityBounds} verify the hypotheses of the Moore--Osgood theorem (iterated limits interchange) for the double limit
\[
\lim_{p \to 1^+} \lim_{\epsilon \to 0^+} \mathscr{A}^{\text{out}}(p, \epsilon) = \lim_{\epsilon \to 0^+} \lim_{p \to 1^+} \mathscr{A}^{\text{out}}(p, \epsilon),
\]
where $\mathscr{A}^{\text{out}}(p, \epsilon)$ denotes the outer area functional for the $p$-harmonic level set on $(\tM_\epsilon, \hat{g}_\epsilon)$.

Specifically, the uniform bound $\mathscr{A}^{\text{out}}(p, \epsilon) \le C_4 \Cap_p^{p-1}(\Sigma, \hat{g}_\epsilon)$ holds for all $(p, \epsilon) \in (1,2] \times (0,\epsilon_0]$, which provides the equicontinuity required by Moore--Osgood.
\end{corollary}

\begin{remark}[Marginal Stability Consistency Verification: $\lambda_1 = 0$ Throughout Pipeline]\label{rem:MarginalStabilityConsistency}
The marginal stability case $\lambda_1(L_\Sigma) = 0$ (corresponding to extremal black holes) requires careful tracking throughout the proof pipeline. We verify consistency at each stage:

\textbf{(I) Jang Equation Stage:}
\begin{itemize}
    \item \textbf{Decay rate:} When $\lambda_1 = 0$, the indicial roots are $\alpha_\pm = \{0, -2\}$ (double root at 0). The Jang solution has polynomial decay $|\bg - g_{\text{cyl}}| = O(t^{-2})$ rather than exponential.
    \item \textbf{Mean curvature jump:} By the favorable jump condition, $[H]_{\bar{g}} \ge 0$. The interface is $C^1$ if $[H]_{\bar{g}} = 0$.
    \item \textbf{Distributional curvature:} With $[H] = 0$, the singular term $2[H]\delta_\Sigma$ vanishes, so $R_{\bar{g}} = \mathcal{S}_{\text{bulk}} \ge 0$ without Dirac contributions.
\end{itemize}

\textbf{(II) Lichnerowicz Equation Stage:}
\begin{itemize}
    \item \textbf{Fredholm theory:} Per Theorem~\ref{thm:MarginalSpectralComplete}(3), we choose $\beta \in (-\sqrt{\lambda_2}, 0)$ where $\lambda_2 > 0$ is the \emph{second} eigenvalue (using 1-indexing; equivalently $\lambda_1 > 0$ in 0-indexing). The operator remains Fredholm of index zero.
    \item \textbf{Conformal factor bound:} The proof of $\phi \le 1$ (Theorem~\ref{thm:PhiBound}) uses the Bray--Khuri identity, which remains valid since the boundary flux terms decay as $O(T^{-4})$ (polynomial decay from $O(T^{-2})$ gradient and $O(T^2)$ area).
\end{itemize}

\textbf{(III) Corner Smoothing Stage:}
\begin{itemize}
    \item \textbf{No smoothing needed:} When $[H] = 0$, the metric $\tg$ is already $C^1$ across $\Sigma$, so Miao's corner smoothing is not required at this interface. Any bubble tip singularities are still handled by smoothing.
    \item \textbf{Scalar curvature:} Since $R_{\tg} = R_{\tg}^{\text{smooth}} \ge 0$ (no Dirac mass), the smoothed metric $\hat{g}_\epsilon$ also has $R_{\hat{g}_\epsilon} \ge 0$ by standard interior smoothing estimates.
\end{itemize}

\textbf{(IV) AMO Level Set Stage:}
\begin{itemize}
    \item \textbf{Tolksdorf bounds:} The uniform gradient estimate (Lemma~\ref{lem:TolksdorfUniformity}) holds with $C$ independent of $p \in (1,2]$, regardless of $\lambda_1$.
    \item \textbf{Capacity removability:} Bubble tips retain $\Cap_p(\{p_k\}) = 0$ for $1 < p < 3$, unaffected by marginal stability.
    \item \textbf{Double limit:} The interchange $(p, \epsilon) \to (1^+, 0)$ (Theorem~\ref{thm:CompleteDblLimit}) uses uniform bounds that hold for both $\lambda_1 > 0$ and $\lambda_1 = 0$.
\end{itemize}

\textbf{Conclusion:} The marginal stability case flows through all stages consistently, with the main simplification being the absence of the mean curvature jump (improving regularity) and the change from exponential to polynomial decay (handled by adjusted weight choices).
\end{remark}

