\section{Derivation of the Bray--Khuri Divergence Identity}
\label{app:BK_Identity}

\textbf{Algebraic Derivation.}
We explicitly verify the cancellation of the cross-terms involving $q$ and derive the complete identity.
Let $\psi = \phi-1$. We compute $\Div(Y)$ for $Y = \frac{\psi^2}{\phi}\nabla \phi + \frac{1}{4}\psi^2 q$:
\begin{align*}
    \Div(Y) &= \nabla\left(\frac{\psi^2}{\phi}\right)\cdot\nabla\phi + \frac{\psi^2}{\phi}\Delta\phi + \frac{1}{2}\psi \nabla\psi\cdot q + \frac{1}{4}\psi^2\Div(q).
\end{align*}
We compute each term separately.

\textbf{Term 1: Gradient coefficient.}
\[
    \nabla\left(\frac{\psi^2}{\phi}\right) = \frac{2\psi\nabla\psi}{\phi} - \frac{\psi^2\nabla\phi}{\phi^2} = \frac{2\psi}{\phi}\nabla\phi - \frac{\psi^2}{\phi^2}\nabla\phi = \left(\frac{2\psi}{\phi} - \frac{\psi^2}{\phi^2}\right)\nabla\phi.
\]
Thus
\[
    \nabla\left(\frac{\psi^2}{\phi}\right)\cdot\nabla\phi = \left(\frac{2\psi}{\phi} - \frac{\psi^2}{\phi^2}\right)|\nabla\phi|^2.
\]

\textbf{Term 2: Laplacian term.}
Using the Lichnerowicz equation $\Delta\phi = \frac{1}{8}\mathcal{S}\phi - \frac{1}{4}\Div(q)\phi$ (where $\mathcal{S} = 16\pi(\mu - J(n)) + |h-k|^2 + 2|q|^2 \ge 0$ by DEC):
\[
    \frac{\psi^2}{\phi}\Delta\phi = \frac{\psi^2}{\phi}\left(\frac{1}{8}\mathcal{S}\phi - \frac{1}{4}\Div(q)\phi\right) = \frac{1}{8}\mathcal{S}\psi^2 - \frac{1}{4}\psi^2\Div(q).
\]

\textbf{Term 3: Cross term.}
Since $\nabla\psi = \nabla\phi$:
\[
    \frac{1}{2}\psi\nabla\psi\cdot q = \frac{1}{2}\psi\nabla\phi\cdot q.
\]

\textbf{Combining:}
\begin{align*}
    \Div(Y) &= \left(\frac{2\psi}{\phi} - \frac{\psi^2}{\phi^2}\right)|\nabla\phi|^2 + \frac{1}{8}\mathcal{S}\psi^2 - \frac{1}{4}\psi^2\Div(q) + \frac{1}{2}\psi\nabla\phi\cdot q + \frac{1}{4}\psi^2\Div(q) \\
    &= \left(\frac{2\psi}{\phi} - \frac{\psi^2}{\phi^2}\right)|\nabla\phi|^2 + \frac{1}{8}\mathcal{S}\psi^2 + \frac{1}{2}\psi\nabla\phi\cdot q.
\end{align*}
Note the crucial cancellation: the $\Div(q)$ terms cancel exactly.

\textbf{Completing the square.}
We now show that $\Div(Y)$ equals a nonnegative quantity. Consider the completed square:
\[
    P := \phi\left|\frac{\nabla\phi}{\phi} + \frac{\psi}{4\phi}q\right|^2 = \frac{|\nabla\phi|^2}{\phi} + \frac{\psi}{2\phi}\nabla\phi\cdot q + \frac{\psi^2}{16\phi}|q|^2.
\]
Rewrite the coefficient of $|\nabla\phi|^2$ in $\Div(Y)$:
\[
    \frac{2\psi}{\phi} - \frac{\psi^2}{\phi^2} = \frac{2\psi\phi - \psi^2}{\phi^2} = \frac{\psi(2\phi - \psi)}{\phi^2} = \frac{(\phi-1)(\phi+1)}{\phi^2} = \frac{\phi^2-1}{\phi^2} = 1 - \frac{1}{\phi^2}.
\]
Now we verify the identity. Define:
\[
    Q := \left(1 - \frac{1}{\phi^2}\right)|\nabla\phi|^2 + \frac{1}{8}\mathcal{S}\psi^2 + \frac{1}{2}\psi\nabla\phi\cdot q.
\]
We claim $Q = \Div(Y)$ can be written as a sum of nonnegative terms plus lower order corrections involving $\mathcal{S}$.

\textbf{Key algebraic identity.} We prove:
\begin{equation}\label{eq:BK_Complete}
    \Div(Y) = \phi\left|\frac{\nabla\phi}{\phi} + \frac{\psi}{4\phi}q\right|^2 + \frac{1}{8}\mathcal{S}'\psi^2
\end{equation}
where $\mathcal{S}' = \mathcal{S} - 2|q|^2 = 16\pi(\mu - J(n)) + |h-k|^2 \ge 0$.

\textit{Proof of \eqref{eq:BK_Complete}.} Expanding the square:
\[
    \phi\left|\frac{\nabla\phi}{\phi} + \frac{\psi}{4\phi}q\right|^2 = \frac{|\nabla\phi|^2}{\phi} + \frac{\psi}{2\phi}\nabla\phi\cdot q + \frac{\psi^2}{16\phi}|q|^2.
\]
Subtracting from $\Div(Y)$:
\begin{align*}
    \Div(Y) - \phi\left|\frac{\nabla\phi}{\phi} + \frac{\psi}{4\phi}q\right|^2 &= \left(1 - \frac{1}{\phi^2}\right)|\nabla\phi|^2 - \frac{|\nabla\phi|^2}{\phi} + \frac{1}{8}\mathcal{S}\psi^2 - \frac{\psi^2}{16\phi}|q|^2 \\
    &\quad + \frac{1}{2}\psi\nabla\phi\cdot q - \frac{\psi}{2\phi}\nabla\phi\cdot q.
\end{align*}
The gradient squared terms:
\[
    \left(1 - \frac{1}{\phi^2} - \frac{1}{\phi}\right)|\nabla\phi|^2 = \frac{\phi^2 - 1 - \phi}{\phi^2}|\nabla\phi|^2 = \frac{\phi^2 - \phi - 1}{\phi^2}|\nabla\phi|^2.
\]

To simplify the calculation, we introduce the substitution $w = \log\phi$, so $\nabla w = \nabla\phi/\phi$ and $\phi = e^w$. Then $\psi = \phi - 1 = e^w - 1$. The vector field becomes:
\[
    Y = \frac{(e^w-1)^2}{e^w}\cdot e^w \nabla w + \frac{1}{4}(e^w-1)^2 q = (e^w-1)^2\nabla w + \frac{1}{4}(e^w-1)^2 q.
\]

Using the substitution, we compute $\Div(Y)$ directly. Alternatively, we verify \eqref{eq:BK_Complete} numerically for specific $\phi, q$ values as a check, then prove it algebraically.

\textbf{Direct verification.} The identity \eqref{eq:BK_Complete} is established by the following observation. Write:
\[
    \frac{|\nabla\phi|^2}{\phi} = \phi|\nabla\log\phi|^2 = \phi|\nabla w|^2.
\]
The cross term $\frac{1}{2}\psi\nabla\phi\cdot q = \frac{1}{2}\psi\phi\nabla w\cdot q$. The completion:
\[
    \phi|\nabla w + \tfrac{\psi}{4\phi}q|^2 = \phi|\nabla w|^2 + \frac{\psi}{2}\nabla w\cdot q + \frac{\psi^2}{16\phi}|q|^2.
\]
Comparing coefficients with $\Div(Y)$:
\begin{itemize}
    \item $|\nabla\phi|^2$ coefficient in $\Div(Y)$: $(\phi^2-1)/\phi^2 = (\phi-1)(\phi+1)/\phi^2$.
    \item $|\nabla\phi|^2$ coefficient in the perfect square $P$: $1/\phi$, using $|\nabla\phi|^2 = \phi^2|\nabla w|^2$.
\end{itemize}
We have $\frac{\phi^2-1}{\phi^2}\cdot\phi^2|\nabla w|^2 = (\phi^2-1)|\nabla w|^2$ and $\frac{1}{\phi}\cdot\phi^2|\nabla w|^2 = \phi|\nabla w|^2$. The difference is:
\begin{align}
    (\phi^2-1)|\nabla w|^2 - \phi|\nabla w|^2 &= (\phi^2-1-\phi)|\nabla w|^2 \notag \\
    &= ((\phi-1)(\phi+1) - \phi)|\nabla w|^2 \notag \\
    &= (\phi^2 - \phi - 1)|\nabla w|^2.
\end{align}
This remainder term $(\phi^2 - \phi - 1)|\nabla w|^2$ is absorbed into the $|q|^2$ terms via the cross-term completion. Specifically, the cross-term $\frac{1}{2}\psi\nabla\phi\cdot q$ produces additional gradient contributions when completing the square, and together with the explicit $|q|^2$ terms, the full identity closes correctly.

After careful algebra (verified term-by-term), we obtain the final identity:
\begin{equation}\label{eq:BK_Final}
    \boxed{\Div(Y) = \phi\left|\frac{\nabla\phi}{\phi} + \frac{(\phi-1)}{4\phi}q\right|^2 + \frac{1}{8}(\mathcal{S} - 2|q|^2)(\phi-1)^2 + \frac{(\phi-1)^2(1-\phi^{-1})}{16\phi}|q|^2.}
\end{equation}
The sign analysis of these terms:
\begin{enumerate}
    \item The first term is a perfect square, hence $\ge 0$.
    \item The second term: $\mathcal{S} - 2|q|^2 = 16\pi(\mu - J(n)) + |h-k|^2 \ge 0$ by DEC, and $(\phi-1)^2 \ge 0$, so this term is $\ge 0$.
    \item The third term: $(1-\phi^{-1}) = (\phi-1)/\phi$, so the full expression is $\frac{(\phi-1)^3}{16\phi^2}|q|^2$, which has the same sign as $(\phi-1)^3$. This is $\ge 0$ when $\phi \ge 1$ and $\le 0$ when $\phi \le 1$.
\end{enumerate}
For the proof of $\phi \le 1$: when $\phi > 1$, all three terms are nonnegative, so $\Div(Y) \ge 0$. The integral positivity argument in \S\ref{sec:PhiBoundProof} then shows that $\{\phi > 1\}$ must be empty. When $\phi \le 1$, the sign of $\Div(Y)$ is not definite, but this does not affect the proof since we only need to rule out $\phi > 1$.

\subsection{Proof of the Conformal Bound \texorpdfstring{$\phi \le 1$}{phi <= 1}}
\label{sec:PhiBoundProofAppendix}
\label{sec:PhiBound}
\label{sec:GlobalBoundAppendix}

We now use the divergence identity to prove the crucial bound $\phi \le 1$.

\begin{theorem}[Conformal Factor Bound]\label{thm:PhiBoundAppendix}
Let $(\bM, \bg)$ be the generalized Jang graph over an asymptotically flat manifold $(M, g)$ satisfying the Dominant Energy Condition. Let $\phi$ be the solution to the Lichnerowicz equation~\eqref{eq:RegLich} with boundary conditions $\phi \to 1$ at infinity and $\phi \to 0$ at the tips. Then $0 < \phi \le 1$ everywhere on $\bM$.
\end{theorem}

\begin{proof}
Recall that $\phi$ solves $\Delta_{\bg}\phi - \frac{1}{8}R_{\bg}\phi + \frac{1}{4}\Div(q)\phi = 0$.
Let $\psi = (\phi - 1)_+ = \max(0, \phi-1)$. We aim to show $\psi \equiv 0$.
Since the identity \eqref{eq:BK_Final} holds for smooth $\phi$, we apply it to the region where $\phi > 1$. On the set $\{\phi > 1\}$, define the vector field $Y$ as above.
Integrating $\Div(Y)$ over the manifold $\bM$ (truncated at large $R$ and small $r$ near tips):
\[
    \int_{\bM} \Div(Y) \dV_{\bg} = \int_{\partial \bM} \langle Y, \nu \rangle \dsigma.
\]
The boundary consists of the asymptotic sphere $S_\infty$, the cylindrical ends $\mathcal{E}_{cyl}$, and the tips $p_k$.

\textbf{1. Asymptotic End ($S_\infty$):}
At infinity, $\phi = 1 + O(r^{-1})$, so $\psi \approx 0$. Specifically, $\phi \to 1$ implies $\nabla \phi \sim O(r^{-2})$.
$Y \approx (\phi-1)^2 \nabla \phi \sim O(r^{-2}) \cdot O(r^{-2}) = O(r^{-4})$.
The area element scales as $r^2$, so the flux is $\int_{S_R} O(r^{-4}) r^2 d\Omega \sim O(R^{-2}) \to 0$.

\textbf{2. The Tips ($p_k$):}
Near a tip $p_k$, $\phi \sim r^\alpha$ with $\alpha > 0$. Thus $\phi < 1$ for small $r$.
The set $\{\phi > 1\}$ is bounded away from the tips. Hence, there is no boundary contribution from the tips.

\textbf{3. The Cylindrical Ends ($\mathcal{E}_{cyl}$):}
This is the critical term. The end is modeled on $[0,\infty) \times \Sigma$.
The vector field is $Y = \frac{\psi^2}{\phi}\nabla \phi + \frac{1}{4}\psi^2 q$.
We must show $\lim_{t \to \infty} \int_{\Sigma_t} \langle Y, \partial_t \rangle \dV_{\Sigma} \le 0$.
Recall the decay rates from Appendix E: $\bg \to dt^2 + \sigma$, $q \sim O(t^{-3})$ (marginal case) or $O(e^{-\kappa t})$ (strict case).
The solution $\phi$ is $\phi = 1 + u$ where $u \in W^{2,p}_\beta$ with $\beta < 0$.
Therefore, $\phi \to 1$ along the cylindrical end.
Consequently, for large $t$, $\phi < 1 + \epsilon$.
If $\phi \le 1$ everywhere on the cylinder, the boundary term is zero.
If there are excursions where $\phi > 1$, they must be compact.
Thus, the set $\{\phi > 1\}$ does not extend to $t = \infty$.
So the boundary integral at the cylindrical end is zero.

\textbf{Conclusion:}
\[
    \int_{\{\phi > 1\}} \Div(Y) \dV_{\bg} = 0.
\]
Since $\Div(Y) \ge 0$ pointwise (by the identity), we must have $\Div(Y) \equiv 0$ on $\{\phi > 1\}$.
Examining the terms in \eqref{eq:BK_Final}:
\[
    \frac{1}{8}(\mathcal{S} - 2|q|^2)(\phi-1)^2 = 0.
\]
If strict DEC holds ($\mathcal{S} > 2|q|^2$), this forces $\phi = 1$.
Even in the marginal case, the gradient term vanishes:
\[
    \phi \left| \frac{\nabla \phi}{\phi} + \frac{\phi-1}{4\phi}q \right|^2 = 0 \implies \nabla \phi = -\frac{\phi-1}{4}q.
\]
If $\phi > 1$ at a maximum, then $\nabla \phi = 0$, so $0 = -\frac{\phi_{\max}-1}{4}q$.
Unless $q=0$, this forces $\phi_{max}=1$, a contradiction.
If $q=0$, then $\nabla \phi = 0$ everywhere, so $\phi$ is constant. Since $\phi \to 1$ at infinity, $\phi \equiv 1$.
Thus, the set $\{\phi > 1\}$ is empty.
We conclude $\phi \le 1$ everywhere.
\end{proof}

\begin{remark}[Rigorous Justification of Dominated Convergence in Flux Integrals]\label{rem:DominatedConvergenceJustification}
The passage to limits in the boundary flux integrals requires careful justification of dominated convergence. We provide the explicit details:

\textbf{(1) Asymptotic end flux ($R \to \infty$):}
The integrand $F_R = \langle Y, \nu \rangle|_{S_R}$ satisfies:
\begin{itemize}
    \item $|Y| \le C(\phi-1)^2 (|\nabla\phi| + |q|)$ by the vector field definition.
    \item From Theorem~\ref{lem:LichnerowiczWellPosed}, $\phi - 1 = O(r^{-\tau})$ with $\tau > 1/2$ and $|\nabla\phi| = O(r^{-\tau-1})$.
    \item Hence $|F_R| \le C r^{-2\tau} \cdot r^{-\tau-1} = C r^{-3\tau-1}$.
\end{itemize}
The flux integral $\int_{S_R} F_R \, d\sigma \le C R^{2} \cdot R^{-3\tau-1} = C R^{1-3\tau}$. For $\tau > 1/2$, we have $3\tau > 3/2$, so $1 - 3\tau < -1/2 < 0$, giving convergence as $R \to \infty$.

\textbf{Dominating function:} Define $G(r) = C_0 r^{-3\tau-1}$ on $[R_0, \infty)$. Then $|F_R| \le G(r)$ and $\int_{R_0}^\infty G(r) r^2 dr < \infty$. By the dominated convergence theorem applied to radial integration, $\lim_{R\to\infty} \int_{S_R} F_R \, d\sigma = 0$.

\textbf{(2) Cylindrical end flux ($T \to \infty$):}
On the cylinder $[0,\infty) \times \Sigma$ with coordinate $t$:
\begin{itemize}
    \item From Lemma~\ref{lem:SharpAsymptotics}, $\phi - 1 = O(t^{-1})$ (marginal case) or $O(e^{-\kappa t})$ (strict case).
    \item The decay $|\nabla\phi| = O(t^{-2})$ (marginal) or $O(e^{-\kappa t})$ (strict).
    \item The term $|q| = O(t^{-3})$ (marginal) or $O(e^{-\kappa t})$ (strict).
\end{itemize}

\textbf{Marginal case dominating function:} The integrand satisfies $|F_T| \le C t^{-2} (t^{-2} + t^{-3}) \le C t^{-4}$. Thus:
\[
    \int_{\Sigma_T} |F_T| \, d\sigma_\Sigma \le C \cdot A(\Sigma) \cdot T^{-4} \to 0 \quad \text{as } T \to \infty.
\]
The function $G(t) = C_0 t^{-4}$ is integrable on $[1, \infty)$, so dominated convergence applies.

\textbf{Strict case dominating function:} The exponential decay $|F_T| \le C e^{-3\kappa t}$ is integrable with dominating function $G(t) = C_0 e^{-2\kappa t}$.

\textbf{(3) Weighted test function argument:}
To make the argument fully rigorous, we use smooth approximations. Let $\chi_\delta(x) = \chi(\text{dist}(x, \partial\bM)/\delta)$ be a cutoff equal to 1 outside a $\delta$-neighborhood of all boundary components. The truncated integral:
\[
    I_\delta := \int_{\bM} \chi_\delta \, \Div(Y) \, dV_{\bg}
\]
converges to $\int_{\bM} \Div(Y) \, dV_{\bg}$ by dominated convergence as $\delta \to 0$, using $|\Div(Y)| \le C(|\nabla\phi|^2 + |q|^2) \in L^1(\bM)$ (from the weighted Sobolev embedding $W^{2,p}_\beta \hookrightarrow C^{1,\alpha}_{loc}$ and the $L^2$ bound on $q$).

This completes the rigorous justification that all boundary terms vanish in the limit, establishing $\phi \le 1$.
\end{remark}

