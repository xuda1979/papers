\section{Spectral Positivity and Removability of Singularities}
\label{app:Singularities}

We verify that the compactified bubble tips $p_k$ do not obstruct the analysis. The argument combines the positivity of the Yamabe operator on the bubble cross-sections with the vanishing $p$-capacity of the tips.

\subsection{Positivity of the Decay Rate}
Near a bubble end the conformal factor behaves like $\phi \sim e^{-\alpha t}$. The exponent $\alpha$ is determined by the indicial equation for the conformal Laplacian $L = -\Delta_{\Sigma} + \tfrac18 R_{\Sigma}$ on the cross-section:
\[
    \alpha^2 - \lambda_1(L) = 0 \qquad \Longrightarrow \qquad \alpha = \sqrt{\lambda_1(L)}.
\]
The surface $\Sigma$ is Yamabe positive because a stable MOTS in a DEC-satisfying $3$-manifold is a union of two-spheres \cite{gallowayschoen2006}. Hence $\lambda_1(L) > 0$ and $\alpha > 0$. Two consequences follow:
\begin{enumerate}
    \item The flux $\int_{\partial B_r} \phi \, \partial_\nu \phi$ decays as $r^{2\alpha+1}$ and vanishes at the tip, so no boundary term survives.
    \item The cone angle is controlled and the volume of $B_r(p_k)$ is $O(r^3)$, preventing volume defects.
\end{enumerate}

\subsection{Capacity Zero}
Using $r = e^{-\alpha t}$ as the radial coordinate, the metric is asymptotic to $dr^2 + c^2 r^2 g_{S^2}$. For a cutoff $\psi$ supported in $B_{2\epsilon}$ and equal to $1$ on $B_\epsilon$ we have
\[
    \int_{B_{2\epsilon}} |\nabla \psi|^p \, dV \lesssim \epsilon^{3-p}.
\]
Thus $\text{Cap}_p(\{p_k\}) = 0$ for every $1 < p < 3$. Since the $p$-harmonic potentials we use satisfy $p \in (1,3)$, the tips are removable for $W^{1,p}$ functions.

\subsection{Absence of Ghost Curvature}
The cone angle for our bubble tips satisfies $\Theta = 2\pi(2\alpha + 1) > 2\pi$ (angle excess), which corresponds to negative distributional curvature at the singularities. However, the capacity zero result ensures the Bochner identity is unaffected. Test functions can be chosen to vanish on $\{p_k\}$, so the term $\int \phi \, \mathcal{K}_p(u)$ remains well-defined. Moreover, $u$ cannot take a constant value on a zero-capacity set unless it is constant globally, so the level sets $\{u=t\}$ generically avoid $\{p_k\}$. Consequently, no ghost curvature or mass accumulates at the bubble tips.

\subsection{Vanishing Flux at Tips}
\label{subsec:FluxVanishing}

We clarify the precise condition for flux vanishing at the conical tips.

\begin{lemma}[Flux Vanishing Condition]\label{lem:FluxVanishing}
Let $\phi \sim r^\alpha$ near a conical tip with $\alpha > 0$. Then the boundary flux integral
\[
    \mathcal{F}_r := \int_{\partial B_r(p_k)} \phi \, \partial_\nu \phi \, d\sigma
\]
vanishes as $r \to 0$.
\end{lemma}

\begin{proof}
Near the tip, $\phi \sim r^\alpha$ implies $\partial_r \phi \sim \alpha r^{\alpha - 1}$. The product satisfies:
\[
    \phi \, \partial_r \phi \sim r^\alpha \cdot \alpha r^{\alpha - 1} = \alpha r^{2\alpha - 1}.
\]
The surface area of $\partial B_r(p_k)$ in the cone metric is $4\pi c^2 r^2$. Thus:
\[
    \mathcal{F}_r \sim 4\pi c^2 \alpha r^{2\alpha - 1} \cdot r^2 = 4\pi c^2 \alpha r^{2\alpha + 1}.
\]
The condition for vanishing as $r \to 0$ is $2\alpha + 1 > 0$, i.e., $\alpha > -1/2$. Since the positivity of the bubble scalar curvature guarantees $\alpha > 0$ (from the indicial equation $\alpha = \sqrt{\lambda_1(L)} > 0$), this condition is satisfied with ample margin.
\end{proof}

\begin{remark}[Sufficient vs.\ Necessary Conditions]
The flux vanishing requires only $\alpha > -1/2$, but our construction guarantees $\alpha > 0$. We do \textbf{not} require the stronger condition $\alpha > 1/2$ that would arise from certain alternative arguments. The spectral positivity of the bubble cross-section (Yamabe positive two-spheres) ensures $\alpha = \sqrt{\lambda_1(L)} > 0$, which is sufficient.
\end{remark}

