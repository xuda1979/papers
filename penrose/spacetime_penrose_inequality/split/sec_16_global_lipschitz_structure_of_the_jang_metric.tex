\section{Global Lipschitz Structure of the Jang Metric}
\label{app:JangRegularity}

A crucial prerequisite for the smoothing estimates in Appendices~\ref{app:GMT} and~\ref{app:InternalSmoothing} is that the Jang metric $\bg$ is Lipschitz continuous with a uniform constant $K$. In the standard coordinates of the initial data $(M,g)$, the graph function $f$ blows up as $f \sim \ln s$, so the component $\bg_{ss} = 1 + (\partial_s f)^2$ diverges like $s^{-2}$. We therefore construct a coordinate atlas in which all components remain bounded and manifestly Lipschitz.

\subsection{The Cylindrical Transformation}
Let $s$ denote the geodesic distance to the horizon $\Sigma$ in $(M,g)$. Near $\Sigma$ the Jang solution satisfies
\[
    f(s,y) = \frac{1}{\kappa} \ln s + \psi(s,y),
\]
where $\psi$ stays bounded (and decays in the marginal case with $\kappa = 1$). The induced metric on the graph is
\[
    \bg = g_M + df \otimes df = (1 + (\partial_s f)^2) ds^2 + 2 (\partial_s f)(\partial_y f) ds \: dy + (g_{ab} + \partial_a f \partial_b f) dy^a dy^b,
\]
which clearly diverges as $s \to 0$.

Introduce the cylindrical coordinate $t = -\ln s$, so $ds = -e^{-t} dt$ and $\partial_s = -e^{t} \partial_t$. The dominant term then behaves as
\[
    (\partial_s f)^2 ds^2 \approx \left(-e^t \frac{1}{\kappa}\right)^2 (-e^{-t} dt)^2 = \frac{1}{\kappa^2} dt^2,
\]
revealing that the apparent blow-up is a coordinate artifact.

\subsection{The Regularized Atlas}
We define a chart transition near the interface $\Sigma$ (conceptually at $s \approx \epsilon$ or $t \approx T$) using $(t,y)$ coordinates on the cylindrical end $\mathcal{E}_{cyl}$.

\begin{lemma}[Boundedness in Cylindrical Coordinates]
In the $(t,y)$ chart on $\mathcal{E}_{cyl}$ the components of the Jang metric satisfy
\[
    \|\bg_{ij}\|_{L^\infty} \le C, \qquad \|\nabla \bg_{ij}\|_{L^\infty} \le C.
\]
\end{lemma}

\begin{proof}
In $(t,y)$ coordinates the base metric reads $g_M = e^{-2t} dt^2 + g_\Sigma(e^{-t})$. The differential of the Jang graph is $df = -\tfrac{1}{\kappa} dt + d\psi$, so
\[
    \bg = g_M + df \otimes df.
\]
The $dt^2$ component tends to $1/\kappa^2$, the cross terms decay because $\partial_t \psi$ decays, and the tangential components are controlled by $g_\Sigma + \partial_y \psi \otimes \partial_y \psi$. Since $\psi$ is smooth in the bulk and decays asymptotically, all derivatives are bounded. Thus $\bg$ is $C^1$ (hence Lipschitz) in the $(t,y)$ chart.
\end{proof}

\subsection{Implication for Smoothing}
The smoothing $\hat{g}_\epsilon = \rho_\epsilon * \bg$ defined in Section~\ref{sec:Construction} and Appendix~\ref{app:InternalSmoothing} is performed \textbf{explicitly in this $(t,y)$ coordinate chart} over the collar region $[-\epsilon, \epsilon] \times \Sigma$ (identifying the interface $s=0$ with a finite value $t=T$ in the glued manifold, or by using reflection coordinates). Because the components $\bg_{ij}$ are Lipschitz in this chart (derivative bounded by $C$), the standard convolution estimates apply:
\begin{enumerate}
    \item $\|\hat{g}_\epsilon - \bg\|_{C^0} \le (\sup |\partial_t \bg|) \cdot \epsilon \le C \epsilon$.
    \item The isoperimetric constant is stable, since the distortion of the volume form is bounded: $\tfrac{\det \hat{g}_\epsilon}{\det \bg} = 1 + O(\epsilon)$.
\end{enumerate}
This validates the use of a uniform bi-Lipschitz constant $K$ in the stability theory, ensuring that the collapse analysis in Appendix~\ref{app:GMT} is carried out in a non-degenerate coordinate system.

\subsection{Complete Coordinate Transition Analysis}
We now provide the complete analysis of the coordinate transition between the bulk and cylindrical regions, establishing the global Lipschitz structure with explicit estimates.

\begin{theorem}[Global Bi-Lipschitz Structure]\label{thm:GlobalBiLipschitz}
The Jang metric $\bg$ on the manifold $\bM$ admits a global atlas $\mathcal{A} = \{(U_\alpha, \varphi_\alpha)\}$ such that:
\begin{enumerate}
    \item In each chart, the metric components $\bg_{ij}$ are uniformly Lipschitz: $|\bg_{ij}|_{C^{0,1}(U_\alpha)} \le K$ for a constant $K$ independent of $\alpha$.
    \item The transition functions between overlapping charts are bi-Lipschitz with explicit bounds depending only on the geometry of $(\Sigma, g_\Sigma)$.
    \item The metric converges to the product cylinder: $\|\bg - g_{cyl}\|_{C^{0,1}(K)} = O(t^{-2})$ for any compact $K \subset \mathcal{C}_{[T,\infty)}$ in the cylindrical end.
\end{enumerate}
\end{theorem}

\begin{proof}
\textbf{Step 1: Construction of the Atlas.}
We construct a finite atlas covering $\bM$ consisting of:
\begin{itemize}
    \item \textbf{Bulk charts} $\{(U_\alpha^{bulk}, \varphi_\alpha^{bulk})\}$: Standard coordinate charts on the compact region $\bM_0 := \bM \cap \{t \le T_0\}$ for some fixed $T_0 > 0$.
    \item \textbf{Cylindrical charts} $\{(U_\beta^{cyl}, \varphi_\beta^{cyl})\}$: Charts of the form $(t, y) \in [T_0 - 1, \infty) \times V_\beta$ where $\{V_\beta\}$ is a finite cover of $\Sigma$.
    \item \textbf{Transition charts} $\{(U_\gamma^{trans}, \varphi_\gamma^{trans})\}$: Charts covering the overlap region $t \in [T_0 - 1, T_0 + 1]$ where the bulk and cylindrical coordinates must be matched.
\end{itemize}

\textbf{Step 2: Lipschitz Estimates in Bulk Charts.}
In the bulk region $\bM_0$, the Jang solution $f$ is smooth (by elliptic regularity for the GJE away from the blow-up surface). The induced metric $\bg = g_M + df \otimes df$ inherits smoothness:
\[
    \|\bg_{ij}\|_{C^k(U_\alpha^{bulk})} \le C_k(\|g_M\|_{C^k}, \|f\|_{C^{k+1}}) < \infty.
\]
In particular, $|\bg_{ij}|_{C^{0,1}} \le K_{bulk}$ on each bulk chart.

\textbf{Step 3: Lipschitz Estimates in Cylindrical Charts.}
In the cylindrical coordinates $(t, y)$ where $t = -\ln s$ and $y \in \Sigma$, the Jang solution has the expansion (from Lemma~\ref{lem:SharpAsymptotics}):
\[
    f(t, y) = \frac{t}{\kappa} + A(y) + v(t, y),
\]
where $\kappa > 0$ is determined by the principal eigenvalue of the stability operator, and $v$ satisfies:
\begin{itemize}
    \item Strictly stable case: $|v|_{C^2} \le C e^{-\beta t}$ for some $\beta > 0$.
    \item Marginally stable case: $|v|_{C^2} \le C t^{-2}$.
\end{itemize}

The induced metric in cylindrical coordinates is computed as follows. Let $s = e^{-t}$, so:
\[
    ds = -e^{-t} dt, \quad \partial_s = -e^t \partial_t.
\]
The base metric in $(s, y)$ coordinates is $g_M = ds^2 + g_\Sigma(s, y)$ where $g_\Sigma(s, y) = g_\Sigma(y) + O(s)$ as $s \to 0$. In $(t, y)$ coordinates:
\[
    g_M = e^{-2t} dt^2 + g_\Sigma(e^{-t}, y) = e^{-2t} dt^2 + g_\Sigma(y) + O(e^{-t}).
\]

The differential of $f$ is:
\[
    df = \partial_t f \, dt + \partial_y f \, dy = \left(\frac{1}{\kappa} + \partial_t v\right) dt + (\partial_y A + \partial_y v) \, dy.
\]

The induced metric $\bg = g_M + df \otimes df$ has components:
\begin{align}
    \bg_{tt} &= e^{-2t} + \left(\frac{1}{\kappa} + \partial_t v\right)^2 = \frac{1}{\kappa^2} + \frac{2}{\kappa} \partial_t v + O(t^{-4}) + O(e^{-2t}), \label{eq:gtt}\\
    \bg_{ta} &= (\partial_t f)(\partial_a f) = \left(\frac{1}{\kappa} + O(t^{-3})\right)(\partial_a A + O(t^{-2})) = \frac{1}{\kappa} \partial_a A + O(t^{-2}), \label{eq:gta}\\
    \bg_{ab} &= g_{\Sigma,ab}(y) + \partial_a f \partial_b f + O(e^{-t}) = g_{\Sigma,ab} + \partial_a A \partial_b A + O(t^{-2}). \label{eq:gab}
\end{align}

The limiting cylindrical metric is:
\[
    g_{cyl} = \frac{1}{\kappa^2} dt^2 + g_\Sigma + \partial_y A \otimes \partial_y A = \frac{1}{\kappa^2} dt^2 + \tilde{g}_\Sigma,
\]
where $\tilde{g}_\Sigma = g_\Sigma + dA \otimes dA$ is the induced metric on the MOTS viewed as the graph of $A$ over a reference surface.

\textbf{Explicit Decay Estimates:}
From equations \eqref{eq:gtt}--\eqref{eq:gab} and the decay of $v$:
\begin{align}
    |\bg_{tt} - \kappa^{-2}| &\le C t^{-2}, \quad |\partial_t(\bg_{tt} - \kappa^{-2})| \le C t^{-3}, \label{eq:decay1}\\
    |\bg_{ta} - \kappa^{-1} \partial_a A| &\le C t^{-2}, \quad |\partial_t \bg_{ta}| \le C t^{-3}, \label{eq:decay2}\\
    |\bg_{ab} - (g_{\Sigma,ab} + \partial_a A \partial_b A)| &\le C t^{-2}, \quad |\partial_t \bg_{ab}| \le C t^{-3}. \label{eq:decay3}
\end{align}

These estimates establish:
\[
    \|\bg - g_{cyl}\|_{C^0} = O(t^{-2}), \quad \|\partial_t(\bg - g_{cyl})\|_{C^0} = O(t^{-3}).
\]

For the tangential Lipschitz bound, the covariant derivatives with respect to $y$ involve the Christoffel symbols of $g_\Sigma$ and derivatives of $A$, both of which are bounded since $\Sigma$ is compact:
\[
    \|\nabla_y(\bg - g_{cyl})\|_{C^0} = O(t^{-2}).
\]

Combining, $\|\bg - g_{cyl}\|_{C^{0,1}} = O(t^{-2})$ in the cylindrical end.

\textbf{Step 4: Transition Chart Matching.}
In the overlap region $t \in [T_0 - 1, T_0 + 1]$, we must verify that the bulk and cylindrical chart descriptions are compatible. The transition map $\Phi: (s, y) \mapsto (t, y) = (-\ln s, y)$ is smooth for $s > 0$. The Jacobian is:
\[
    D\Phi = \begin{pmatrix} -1/s & 0 \\ 0 & I \end{pmatrix}.
\]
At $t = T_0$, i.e., $s = e^{-T_0}$, the Jacobian is bounded: $|D\Phi| \le e^{T_0}$. The inverse Jacobian $(D\Phi)^{-1}$ has norm bounded by $e^{-T_0}$.

For $T_0$ fixed, the metric transformation gives:
\[
    \bg_{(t,y)} = (D\Phi)^T \bg_{(s,y)} D\Phi.
\]
Since $\bg_{(s,y)}$ is smooth (hence Lipschitz) for $s \in [e^{-T_0-1}, e^{-T_0+1}]$ and $D\Phi$ is smooth and bounded on this region, $\bg_{(t,y)}$ is also Lipschitz with constant:
\[
    K_{trans} \le e^{2T_0} \cdot K_{bulk} \cdot C(g_M).
\]

\textbf{Step 5: Global Lipschitz Constant.}
The global Lipschitz constant is:
\[
    K = \max\{K_{bulk}, K_{cyl}, K_{trans}\} < \infty.
\]
The existence of this finite upper bound follows from:
\begin{enumerate}
    \item Compactness of $\bM_0$ and smoothness of $\bg$ in the bulk.
    \item The explicit decay estimates \eqref{eq:decay1}--\eqref{eq:decay3} showing $\bg$ approaches a smooth limit $g_{cyl}$ in the cylindrical end.
    \item Smoothness of the transition map $\Phi$ on the bounded overlap region.
\end{enumerate}
\end{proof}

\begin{corollary}[Uniform Ellipticity Constants]\label{cor:UniformEllipticity}
There exist constants $0 < \lambda \le \Lambda < \infty$ such that for all $\xi \in T_x \bM$:
\[
    \lambda |\xi|^2 \le \bg(\xi, \xi) \le \Lambda |\xi|^2,
\]
where the bounds are uniform over $\bM$ when measured in the global atlas $\mathcal{A}$.
\end{corollary}

\begin{proof}
In the bulk, $\bg$ is a smooth positive-definite metric, hence uniformly elliptic on compact sets.

In the cylindrical end, $\bg \to g_{cyl}$ with $\|bg - g_{cyl}\|_{C^0} \le C t^{-2}$. The cylindrical metric $g_{cyl} = \kappa^{-2} dt^2 + \tilde{g}_\Sigma$ is uniformly elliptic:
\[
    \min(\kappa^{-2}, \lambda_{\min}(\tilde{g}_\Sigma)) |\xi|^2 \le g_{cyl}(\xi, \xi) \le \max(\kappa^{-2}, \lambda_{\max}(\tilde{g}_\Sigma)) |\xi|^2.
\]
For $t \ge T_0$ with $C T_0^{-2} < \frac{1}{2} \min(\kappa^{-2}, \lambda_{\min})$:
\[
    \frac{1}{2} \lambda_{\min}(g_{cyl}) |\xi|^2 \le \bg(\xi, \xi) \le 2 \lambda_{\max}(g_{cyl}) |\xi|^2.
\]
The global bounds are obtained by taking the minimum and maximum over the compact overlap region.
\end{proof}

\begin{remark}[Metric Completion and Boundary Regularity]
The analysis above establishes that $(\bM, \bg)$ is a \emph{metrically complete} Riemannian manifold with Lipschitz metric tensor. The boundary behavior at $\Sigma$ (the MOTS) and at spatial infinity requires separate discussion:
\begin{enumerate}
    \item \textbf{At $\Sigma$:} The cylindrical end $\mathcal{C} \cong [0, \infty) \times \Sigma$ is \emph{incomplete} in the direction $t \to -\infty$ (i.e., approaching $\Sigma$). However, the blow-up of the Jang solution means the proper distance $\int_0^T |\nabla f| \, dt$ diverges as $T \to \infty$, so the end is metrically complete. The horizon $\Sigma$ lies ``at infinity'' along the cylinder.
    \item \textbf{At spatial infinity:} On the asymptotically flat end, $\bg \to g_M + O(r^{-\tau})$ for some $\tau > 0$, ensuring completeness and providing the decay needed for the ADM mass.
\end{enumerate}
\end{remark}

