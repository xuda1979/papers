\section{Rigidity and the Uniqueness of Schwarzschild}
\label{sec:Rigidity}

We now prove the rigidity statement of the Penrose Inequality: equality holds if and only if the initial data set corresponds to a slice of the Schwarzschild spacetime. This section details the step-by-step argument showing how the assumption of equality forces specific geometric constraints on the Jang manifold and, subsequently, the initial data.

\begin{theorem}[Rigidity of the Equality Case]
Suppose an initial data set $(M,g,k)$ satisfies the assumptions of Theorem \ref{thm:SPI} and that equality holds in the Spacetime Penrose Inequality:
\begin{equation}
    M_{\ADM}(g) = \sqrt{\frac{A(\Sigma)}{16\pi}}.
\end{equation}
\textbf{Assumption:} We assume the outermost apparent horizon $\Sigma$ is connected.
Then the initial data set $(M,g,k)$ can be isometrically embedded as a spacelike slice in the Schwarzschild spacetime.
\begin{remark}
If $\Sigma$ is disconnected, the inequality $M \ge \sqrt{A/16\pi}$ still holds (where $A$ is total area), but the rigidity analysis must account for the possibility of multi-black hole configurations. Generally, equality in the disconnected case is only achieved in the limit of infinite separation. Our rigidity result implies that if equality holds for a connected horizon, the spacetime is a single Schwarzschild slice.
\end{remark}
\end{theorem}
\begin{proof}
The proof relies on forcing the saturation of every inequality in the construction.

\textbf{Step 1: Saturation of Inequalities.}
The equality $M_{\ADM}(g) = \sqrt{A(\Sigma)/16\pi}$ implies:
\begin{enumerate}
    \item $M_{\ADM}(g) = M_{\ADM}(\bg)$. The mass difference formula vanishes:
    \[ \int_{\bM} (16\pi(\mu-J(n)) + |h-k|_{\bg}^2 + 2|q|_{\bg}^2) dV = 0. \]
    This implies $\mu=J(n)$, $h=k$, and $q=0$.
    \item $M_{\ADM}(\bg) = M_{\ADM}(\tg)$. The mass change is given by the integral of the scalar curvature source. The condition $M_{\ADM}(\bg) = M_{\ADM}(\tg)$ forces $\int_{\bM} R_{\bg} \phi \, dV = 0$.
    Recall that $R_{\bg}$ is a measure: $R_{\bg} = \mathcal{S}_{bulk} + 2[H]\delta_\Sigma$.
    Since $\mathcal{S}_{bulk} \ge 0$ (DEC) and $[H] \ge 0$ (Stability), and $\phi > 0$, the vanishing of the integral forces both terms to vanish individually:
    \[ \mathcal{S}_{bulk} \equiv 0 \quad \text{and} \quad [H] \equiv 0. \]
    The vanishing of the bulk term implies $R_{\bg}^{reg} = 0$.
    The vanishing of the jump term implies the mean curvature is continuous across $\Sigma$.
    Consequently, the Lichnerowicz equation becomes $\Delta_{\bg} \phi = 0$. With $\phi \to 1$ at infinity, the unique solution is $\phi \equiv 1$.
    \item \textbf{Vanishing of Internal Bubbles:} In the conformal construction (\Cref{thm:Deformation}), any internal Jang bubble $\mathcal{B}$ is sealed by enforcing the Dirichlet boundary condition $\phi \to 0$ on $\partial \mathcal{B}$. The conclusion $\phi \equiv 1$ is therefore compatible only if the set of bubbles is empty. Hence the equality case forces $\mathcal{B} = \emptyset$ and the only boundary component is the outermost horizon $\Sigma$.
    \item **Interface Regularity:** The condition $[H]=0$ is the geometric key. It upgrades the regularity of the Jang metric across $\Sigma$. Since $\bg$ is Lipschitz and the mean curvature (first derivative) matches, $\bg \in C^{1,1}_{loc}$. This allows the static vacuum bootstrap to proceed.
\end{enumerate}

\textbf{Step 2: Static Vacuum Equations.}
We establish that the Jang graph is a slice of a static vacuum spacetime.
Let $N = (1+|\nabla f|^2)^{-1/2}$ be the lapse function. The vanishing of the rigidity term implies the metric pair $(\bg, N)$ satisfies the **Static Vacuum Equations**:
\begin{equation}\label{eq:StaticVacuum}
    \Delta_{\bg} N = 0, \quad N \Ric_{\bg} - \nabla^2 N = 0.
\end{equation}
The condition $q=0$ is equivalent to $h_{ij} = k_{ij}$.
Since $h_{ij} = \frac{\nabla^2_{ij} f}{\sqrt{1+|\nabla f|^2}}$ is the second fundamental form of the graph in the product spacetime, the condition $q=0$ implies the normal to the graph is a Killing vector field direction.
Substituting $q=0, h=k, \mu=J(n)=0$ into the Jang identity yields $R_{\bg} = 0$.
These equations hold in the distributional sense across the interface $\Sigma$.

\textbf{Step 3: $C^{1,1}$ Regularity across the Interface.}
We now upgrade the regularity of the solution $(\bg, N)$. Initially $\bg$ is only Lipschitz across $\Sigma$, but the equality case forces $[H]=0$.
1. **Vanishing Shear via a Killing Horizon:** In a static vacuum spacetime the horizon $\Sigma$ is a Killing horizon, so the shear of its null generator vanishes and $k_{ab}|_{\Sigma}=0$.
2. **Matching Second Fundamental Forms:** Along the cylindrical side the second fundamental form is zero, whereas on the bulk side $h=k$. Together with $k|_{\Sigma}=0$ this yields $A_{bulk}=0$ at the interface.
3. **$C^{1,1}$ Regularity:** Continuity of the metric and its matched normal derivatives imply $\partial_s g_{ij}$ is continuous in Gaussian coordinates, so $\bg \in C^{1,1}_{loc}$.

Specifically, the regularity lift proceeds as follows:
\begin{enumerate}
    \item Since $g \in C^{1,1}$, the Christoffel symbols are Lipschitz, so the Laplacian has $C^{0,1}$ coefficients.
    \item Solving $\Delta_g N = 0$ with Lipschitz coefficients yields $N \in C^{2,\alpha}$ for every $\alpha \in (0,1)$.
    \item The static equation $N\Ric = \nabla^2 N$ then forces $\Ric$ to lie in $C^{0,\alpha}$.
    \item In harmonic coordinates the Ricci tensor becomes an elliptic operator applied to $g$, so the $C^{0,\alpha}$ source promotes $g$ to $C^{2,\alpha}$.
    \item Iterating the previous steps improves $(g,N)$ to $C^{k,\alpha}$ for all $k$, ultimately yielding smoothness and (via Anderson \cite{anderson2000}) analyticity.
\end{enumerate}

\textbf{Step 4: Harmonic Bootstrap to Real Analyticity.}

\begin{proof}[Proof of Rigidity Regularity Bootstrap]
We provide the explicit bootstrap for the equality case $(\bg, N)$.
\begin{enumerate}
    \item \textbf{Initial state.} Equality implies $\mathcal{S}=0$, $\phi=1$, and $q=0$. The Jang metric $\bg$ is Lipschitz across $\Sigma$, while the lapse $N=(1+|\nabla f|^2)^{-1/2}$ vanishes linearly on $\Sigma$, so the horizon is non-degenerate.
    \item \textbf{Static vacuum system.} The pair $(\bg,N)$ solves $\Delta_{\bg} N = 0$ and $N\Ric_{\bg} = \nabla^2 N$ distributionally. In harmonic coordinates the Ricci tensor takes the form $-\tfrac12 \Delta \bg_{ij} + Q_{ij}(\bg,\partial \bg)$ with uniformly elliptic principal part because $\bg \in C^{0,1}$.
    \item \textbf{Elliptic bootstrap in adapted gauge.} Since $\partial \bg \in L^\infty$, elliptic regularity yields $N \in W^{2,p}_{loc}$ for all $p<\infty$, hence $N \in C^{1,\alpha}$. Rewriting the static equations as a system for $\bg$ shows the apparent singularity at $\Sigma$ is a gauge artifact: use $N$ as a coordinate near the non-degenerate horizon and apply the regularity theory of Chru\'sciel \cite{chrusciel1990} and Anderson \cite{anderson2000} to propagate smoothness across $\Sigma$.
    \item \textbf{Analyticity.} Iterating Schauder estimates gives $(\bg,N) \in C^{k,\alpha}$ for all $k$. Anderson's theorem then promotes harmonic-coordinate solutions of the static vacuum equations to real-analyticity.
\end{enumerate}
Thus the apparent "kink" at $\Sigma$ is a coordinate artifact and $(\bM \times \mathbb{R}, -N^2 dt^2 + \bg)$ is a smooth analytic static vacuum spacetime.
\end{proof}

\textbf{Step 5: Characterization of the Horizon (Lapse Vanishing).}
To conclude the spacetime is Schwarzschild we must ensure the horizon is a \emph{non-degenerate} Killing horizon. Lemma~\ref{lem:SharpAsymptotics} shows that the Jang graph satisfies $f(s,y) = \ln s + O(1)$ near the blow-up surface, so $|\nabla f| \sim s^{-1}$ when $s$ measures signed distance to $\Sigma$. The lapse of the associated static spacetime is $N = (1+|\nabla f|^2)^{-1/2}$; therefore, as $s \to 0$,
\[
    N \approx \left(1 + \frac{1}{s^2}\right)^{-1/2} \approx s.
\]
The linear vanishing shows that the surface gravity $\kappa = |\nabla N|_{\Sigma}$ is strictly positive, so the Killing horizon is non-degenerate.
\textbf{Exclusion of Disconnected Horizons:} This linear rate $N \sim s$ rules out the Majumdar--Papapetrou multi-black-hole geometries, which are extremal and satisfy $N \sim s^2$ near each component. Hence, in the equality case constructed here, the outermost horizon must be connected.
Once non-degeneracy is known, the rigidity results of Chru\'sciel, Isenberg, and Moncrief \cite{chrusciel1990} guarantee that the static vacuum solution extends analytically across $\Sigma$. Combining this with the uniqueness theorem of Bunting and Masood-ul-Alam establishes that the only asymptotically flat, analytic, static vacuum extension with a connected non-degenerate horizon is the Schwarzschild metric.
\end{proof}

\begin{remark}[Area Preservation at the Horizon]
A potential concern is whether the conformal factor $\phi$ significantly shrinks the area of the horizon (i.e., the integral of $\phi^4$ over $\Sigma$). Unlike a product cylinder where $R>0$ forces $\phi \to 0$, the Jang metric near the horizon asymptotically matches a static vacuum slice with $R_{\bg}=0$ in the regular sense. Consequently the potential $V = \tfrac{1}{8}\Rg$ is small near $\Sigma$, allowing the solution $\phi \approx 1$ to persist. Imposing the Neumann condition $\partial_\nu \phi = 0$ (which preserves minimality) shows that the first variation of $A_{\tg}(\Sigma)$ vanishes and the second variation is controlled by $\|\phi-1\|_{C^0}^2$. Hence $A_{\tg}(\Sigma) = A_{\bg}(\Sigma) + O((\phi-1)^2)$, so the conformal deformation leaves the horizon area unchanged to second order, consistent with the rigidity argument.
\end{remark}

\subsection{Summary: The Chain of Rigidity Implications}

We summarize the logical structure of the rigidity argument:

\begin{center}
\begin{tabular}{|c|c|c|}
\hline
\textbf{Step} & \textbf{Implication} & \textbf{Key Tool} \\
\hline
1 & $M_{\ADM}(g) = M_{\ADM}(\bg)$ & Jang mass formula \\
  & $\Rightarrow \mu = J(n), h = k, q = 0$ & \\
\hline
2 & $M_{\ADM}(\bg) = M_{\ADM}(\tg)$ & Conformal mass change \\
  & $\Rightarrow R_{\bg} = 0, [H] = 0, \phi \equiv 1$ & \\
\hline
3 & $[H] = 0$ & Transmission conditions \\
  & $\Rightarrow \bg \in C^{1,1}$ across $\Sigma$ & \\
\hline
4 & $R_{\bg} = 0, q = 0$ & Static vacuum equations \\
  & $\Rightarrow (\bg, N)$ is static vacuum & \\
\hline
5 & $N \sim s$ near $\Sigma$ & Lapse asymptotics \\
  & $\Rightarrow$ horizon is non-degenerate & \\
\hline
6 & Non-degenerate static vacuum & Bunting--Masood-ul-Alam \\
  & $\Rightarrow$ Schwarzschild & uniqueness theorem \\
\hline
\end{tabular}
\end{center}

\begin{remark}[Why Extremal Black Holes Are Excluded]
The rigidity argument specifically excludes extremal (Majumdar--Papapetrou) multi-black-hole configurations because:
\begin{enumerate}
    \item \textbf{Lapse behavior:} For extremal black holes, $N \sim s^2$ near the horizon (quadratic vanishing), corresponding to zero surface gravity $\kappa = 0$.
    \item \textbf{Jang asymptotics:} The Jang solution satisfies $f \sim \kappa^{-1} \ln s$. For non-zero $\kappa$, this gives $|\nabla f| \sim s^{-1}$, hence $N \sim s$ (linear).
    \item \textbf{Consequence:} The equality case forces $\kappa > 0$, which is incompatible with extremal horizons.
\end{enumerate}
This is consistent with the physical expectation: extremal black holes saturate the mass-area bound differently (via angular momentum in Kerr), not through the Penrose inequality.
\end{remark}

\begin{remark}[Extension to Multiple Components]
For disconnected horizons $\Sigma = \Sigma_1 \cup \cdots \cup \Sigma_N$, the Penrose inequality becomes:
\begin{equation}
    M_{\ADM} \ge \sqrt{\frac{A(\Sigma_1) + \cdots + A(\Sigma_N)}{16\pi}}.
\end{equation}
Equality would require \emph{each} component to achieve equality individually, which by the single-component rigidity theorem forces the data near each $\Sigma_i$ to be Schwarzschild. The gluing of multiple Schwarzschild regions is obstructed by the asymptotically flat constraint unless the components are infinitely separated. Thus:
\begin{itemize}
    \item \textbf{Strict inequality} holds for finitely separated multi-black-hole configurations.
    \item \textbf{Equality is achieved} only in the limiting sense as separation $\to \infty$.
\end{itemize}
This reflects the binding energy of gravitational systems.
\end{remark}

