\section{Rigorous Scalar Curvature Estimates for the Smoothed Metric}
\label{app:SmoothingDetails}

In this appendix, we explicitly calculate the scalar curvature of the smoothed metric $\hat{g}_\epsilon$ in the Gaussian Normal Coordinates (Fermi coordinates) defined in Section~\ref{sec:MiaoSmoothing} and rigorously derive the $L^{3/2}$ bound on its negative part.

\subsection{Setup and Metric Expansion}
We establish the precise convergence rates for the smoothing of the Lipschitz metric $\tg$.
Let $\tg$ be Lipschitz continuous with Lipschitz constant $K$. Let $\rho_\epsilon(x) = \epsilon^{-n} \rho(x/\epsilon)$ be a standard mollifier.
Define $\hat{g}_\epsilon = \rho_\epsilon * \tg$.

\begin{lemma}[Uniform Bi-Lipschitz Estimate]
The smoothed metric $\hat{g}_\epsilon$ converges to $\tg$ with quantitative control on quadratic forms:
\begin{equation}
    (1 - C\epsilon) \, \tg_{ij} \xi^i \xi^j \le (\hat{g}_\epsilon)_{ij} \xi^i \xi^j \le (1 + C\epsilon) \, \tg_{ij} \xi^i \xi^j.
\end{equation}
\end{lemma}
\begin{proof}
The smoothing is defined component-wise in a fixed chart: $(\hat{g}_\epsilon)_{ij} = \eta_\epsilon * \tg_{ij}$. Since $\tg$ is Lipschitz with constant $L$,
\[
    |(\hat{g}_\epsilon)_{ij}(x) - \tg_{ij}(x)| \le \int_{B_\epsilon} \eta_\epsilon(z) |\tg_{ij}(x-z) - \tg_{ij}(x)| \, dz \le L\epsilon.
\]
Uniform ellipticity of $\tg$ implies $\tg_{ij} \xi^i \xi^j \ge \lambda |\xi|^2$. Therefore
\[
    |((\hat{g}_\epsilon)_{ij} - \tg_{ij}) \xi^i \xi^j| \le L\epsilon |\xi|^2 \le \frac{L}{\lambda} \epsilon \; \tg_{ij} \xi^i \xi^j.
\]
Setting $C = L/\lambda$ yields $(1-C\epsilon)|\xi|_{\tg}^2 \le |\xi|_{\hat{g}_\epsilon}^2 \le (1+C\epsilon)|\xi|_{\tg}^2$.
\end{proof}

\begin{corollary}[Stability of Isoperimetric Constant]\label{cor:IsoperimetricStabilityAppendix}
There exists $I_0>0$ such that the smoothed metrics satisfy $I(\hat{g}_\epsilon) \ge I_0$ for all sufficiently small $\epsilon$.
\end{corollary}
\begin{proof}
For any region $\Omega$,
\[
    (1-C\epsilon)^{3/2} \Vol_{\tg}(\Omega) \le \Vol_{\hat{g}_\epsilon}(\Omega) \le (1+C\epsilon)^{3/2} \Vol_{\tg}(\Omega),
\]
and similarly $(1-C\epsilon) \Area_{\tg}(\partial \Omega) \le \Area_{\hat{g}_\epsilon}(\partial \Omega) \le (1+C\epsilon) \Area_{\tg}(\partial \Omega)$. Consequently,
\[
    I(\hat{g}_\epsilon) = \inf_\Omega \frac{\Area_{\hat{g}_\epsilon}(\partial \Omega)}{\Vol_{\hat{g}_\epsilon}(\Omega)^{2/3}} \ge \frac{1-C\epsilon}{1+C\epsilon} I(\tg) \ge (1-C'\epsilon) I(\tg).
\]
Since $(\tM,\tg)$ is non-collapsed (asymptotically flat with a cylindrical end), $I(\tg)>0$, giving the claimed uniform bound.
\end{proof}

\subsection{Explicit Scalar Curvature Expansion}
Using the Gauss-Codazzi equations for the foliation by $\Sigma_s$, the scalar curvature of $\hat{g}_\epsilon$ is given by:\footnote{We follow the sign convention where $R = R^{\Sigma} - |A|^2 - H^2 + 2\Ric(\nu,\nu)$ for the Gauss equation, which simplifies to the above when combined with the Riccati equation $\partial_s H = \Ric(\nu,\nu) + |A|^2$ (using the convention $A = -\frac{1}{2}\partial_s g$).}
\begin{equation}\label{eq:GaussCodazziSmoothed}
    R_{\hat{g}_\epsilon} = R^{\gamma_\epsilon} - |A_\epsilon|_{\gamma_\epsilon}^2 - (H_\epsilon)^2 + 2 \partial_s H_\epsilon,
\end{equation}
where $A_\epsilon = -\frac{1}{2} \gamma_\epsilon^{-1} \partial_s \gamma_\epsilon$ and $H_\epsilon = \Tr_{\gamma_\epsilon} A_\epsilon$.

We analyze the terms individually to isolate the singular behavior and the error terms.
Recall that for the unsmoothed metric, the distributional scalar curvature is $R_{\tg} = R^g - |A|^2 - H^2 + 2\partial_s H$. The term $2\partial_s H$ contains the Dirac mass $2[H]\delta_0$.

\paragraph{1. The Linear (Distributional) Term.}
The mean curvature of the smoothed metric satisfies:
\[ H_\epsilon(s) = \frac{1}{2} \Tr(\gamma_\epsilon^{-1} \partial_s \gamma_\epsilon) = \frac{1}{2} \Tr(\gamma_\epsilon^{-1} (\eta_\epsilon * \partial_s g)). \]
Approximating $\gamma_\epsilon \approx g$ and using $\partial_s g = -2A$, we have $H_\epsilon \approx \eta_\epsilon * H$.
More precisely, we can write:
\[ 2 \partial_s H_\epsilon(s) = \frac{2}{\epsilon} [H] \eta\left(\frac{s}{\epsilon}\right) + E_{lin}(s), \]
where the first term is the smoothing of the distributional curvature $2[H]\delta_0$. Since $[H] \ge 0$ and $\eta \ge 0$, this term contributes a large positive curvature $\sim O(1/\epsilon)$ supported in the collar.
The remainder $E_{lin}(s)$ involves the derivative of the regular part of $H$ and commutator terms, which are bounded ($L^\infty$) because the metric is Lipschitz (so $H$ is bounded).

\paragraph{2. The Quadratic (Deficit) Terms.}
The nonlinearity of the scalar curvature introduces a deficit term. Let $Q(A) = -|A|^2 - H^2$. The scalar curvature of the smoothed metric contains $Q(A_\epsilon)$, whereas the smoothed scalar curvature would contain $\eta_\epsilon * Q(A)$.
We define the deficit:
\begin{equation}
    D_\epsilon(s) = Q(A_\epsilon(s)) - (\eta_\epsilon * Q(A))(s).
\end{equation}
This term is controlled by the Friedrichs Commutator Lemma. Since $\bg$ is Lipschitz, the second fundamental form $A = -\tfrac12 \partial_s g$ lies in $L^\infty(N_{2\epsilon})$. For $f,g \in L^\infty$, the lemma gives
\[ \| (f * \eta_\epsilon)(g * \eta_\epsilon) - (fg) * \eta_\epsilon \|_{L^p} \to 0 \quad \text{and} \quad \| (f * \eta_\epsilon)(g * \eta_\epsilon) - (fg) * \eta_\epsilon \|_{L^\infty} \le 2\|f\|_\infty \|g\|_\infty. \]
Taking $f=g=A$ shows the quadratic deficit satisfies a uniform pointwise bound
\[ |D_\epsilon(s)| \le C \|A\|_{L^\infty}^2. \]
This observation is pivotal: the error does not scale like $\epsilon^{-1}$ (in contrast with the linear term) but remains $O(1)$. Because $D_\epsilon$ is supported in a collar of volume $O(\epsilon)$, we immediately obtain the sharp estimate
\[ \|R^-_\epsilon\|_{L^{3/2}} \lesssim (\epsilon \cdot O(1)^{3/2})^{2/3} = O(\epsilon^{2/3}). \]
In particular, the negative part of the scalar curvature cannot overwhelm the positive spike generated by the mean-curvature jump.

\subsection{Proof of the \texorpdfstring{$L^{3/2}$}{L(3/2)} Bound}
\begin{proof}
We combine the expansion terms.
\[ R_{\hat{g}_\epsilon}(s) = \underbrace{\frac{2}{\epsilon} [H] \eta\left(\frac{s}{\epsilon}\right)}_{\ge 0} + \underbrace{R^{\gamma_\epsilon} + E_{lin}(s) + D_\epsilon(s)}_{E_{bounded}(s)}. \]
The first term is nonnegative (by stability of the MOTS). The second term, $E_{bounded}(s)$, represents the sum of intrinsic curvature, linear errors, and the quadratic deficit. All components of $E_{bounded}$ are constructed from $g$, $\partial_s g$, and their smoothings. Since $\partial_s g \in L^\infty$, we have:
\[ \|E_{bounded}\|_{L^\infty(N_{2\epsilon})} \le C. \]
\textbf{Commutator control:} The only subtlety is the intrinsic curvature term, which involves $\partial \Gamma$ and $\Gamma * \Gamma$ with $\Gamma$ the Christoffel symbols of the Lipschitz metric. Derivatives commute with convolution up to uniformly bounded boundary errors, while the quadratic piece obeys the Friedrichs commutator estimate
\[ \| (\eta_\epsilon * f)(\eta_\epsilon * g) - \eta_\epsilon * (fg) \|_{L^\infty} \le C \|f\|_{L^\infty} \|g\|_{L^\infty}. \]
Taking $f=g=\Gamma$ shows that $R^{\gamma_\epsilon} - \eta_\epsilon * R^{\gamma}$ is uniformly bounded, so $E_{bounded}$ is genuinely $L^\infty$.

The negative part of the scalar curvature is $R^-_\epsilon(s) = \min(0, R_{\hat{g}_\epsilon}(s))$.
Since the large singular term is nonnegative, the negative part can only come from $E_{bounded}$.
\[ R^-_\epsilon(s) \ge \min(0, E_{bounded}(s)) \ge -C. \]
Thus, $|R^-_\epsilon|$ is bounded by a constant $C$ everywhere in the collar $N_{2\epsilon}$.
The volume of the collar is $\text{Vol}(N_{2\epsilon}) \approx 2\epsilon \cdot \text{Area}(\Sigma)$.

We verify the $L^{3/2}$ norm:
\begin{align*}
    \|R^-_\epsilon\|_{L^{3/2}(N_{2\epsilon})} &= \left( \int_{N_{2\epsilon}} |R^-_\epsilon|^{3/2} \, dV_{\hat{g}_\epsilon} \right)^{2/3} \\
    &\le \left( \int_{N_{2\epsilon}} C^{3/2} \, dV \right)^{2/3} \\
    &= \left( C^{3/2} \cdot \text{Vol}(N_{2\epsilon}) \right)^{2/3} \\
    &\le \left( C^{3/2} \cdot C' \epsilon \right)^{2/3} \\
    &= C'' \epsilon^{2/3}.
\end{align*}
This confirms the estimate $\|R^-_\epsilon\|_{L^{3/2}} \le C \epsilon^{2/3}$.
\end{proof}

\begin{remark}[The Vanishing Buffer in the Marginal Case]
In the marginally stable case ($[H]=0$), the large positive term $\frac{2}{\epsilon}[H]$ vanishes. However, the deficit term $D_\epsilon$ remains bounded pointwise by $C\|A\|_{L^\infty}^2$.
The crucial observation is that $R^-_\epsilon$ does not need to be pointwise positive; it only needs to be small in $L^{3/2}$.
Since the support volume is $O(\epsilon)$ and the value is $O(1)$, the $L^{3/2}$ norm scales as $\epsilon^{2/3}$, which holds regardless of whether $[H]$ vanishes or not.
\end{remark}

\begin{lemma}[Dominance of Linear Terms]
In the strictly stable case ($[H] > 0$), the linear term $\frac{2[H]}{\epsilon}\eta$ dominates the bounded error $E_{bounded}$ for sufficiently small $\epsilon$, implying $R_{\hat{g}_\epsilon} \ge 0$ everywhere except possibly near the support boundary of $\eta$. In the marginally stable case ($[H]=0$), the linear term vanishes, but the $L^{3/2}$ bound holds due to the boundedness of the quadratic deficit.
\end{lemma}

\subsection{Complete Fermi Coordinate Derivation of Collar Geometry}
\label{sec:FermiCollarComplete}

We now provide a self-contained derivation of all geometric quantities in Fermi coordinates, establishing the explicit formulas that underlie the smoothing estimates. This subsection closes the technical gap identified in the regularization procedure by making every step explicit and verifiable.

\subsubsection{Construction of Fermi Coordinates}

Let $\Sigma \subset \bM$ be the internal interface (outermost MOTS) with unit normal $\nu$ pointing from the bulk region $\Omega^-$ toward the cylindrical region $\Omega^+$. The Fermi (Gaussian normal) coordinate system $(s, y^1, y^2)$ is constructed as follows:

\begin{definition}[Fermi Coordinate Map]
Let $\{y^a\}_{a=1,2}$ be local coordinates on $\Sigma$. Define the map $\Phi: (-\delta, \delta) \times U \to \bM$ by
\[
\Phi(s, y) = \exp_{\iota(y)}(s \cdot \nu(y)),
\]
where $\iota: \Sigma \hookrightarrow \bM$ is the inclusion and $\exp$ is the exponential map of $(\bM, \bg)$. For sufficiently small $\delta > 0$ (depending on the focal radius of $\Sigma$), $\Phi$ is a diffeomorphism onto a tubular neighborhood $N_\delta = \{x \in \bM : \dist_{\bg}(x, \Sigma) < \delta\}$.
\end{definition}

\begin{remark}[Tubular Neighborhood Radius and Focal Distance]\label{rem:InjectivityRadius}
The maximal radius $\delta_{\max}$ for which Fermi coordinates are valid is determined by the \emph{focal distance} of $\Sigma$ in $(\bM, \bg)$:
\[
    \delta_{\max} = \inf_{y \in \Sigma} \text{focal}_{\bg}(y, \nu(y)),
\]
where $\text{focal}_{\bg}(y, v)$ is the distance to the first focal point along the geodesic $t \mapsto \exp_y(tv)$. For a compact surface $\Sigma$ embedded in a 3-manifold with bounded curvature $|\mathrm{Rm}_{\bg}| \leq \Lambda$, a classical comparison argument gives:
\[
    \delta_{\max} \geq \min\left\{ \frac{\pi}{\sqrt{\Lambda}}, \frac{1}{\|A_\Sigma\|_{L^\infty}} \right\},
\]
where $A_\Sigma$ is the second fundamental form. Since our MOTS $\Sigma$ is compact and embedded in the complete manifold $(\bM, \bg)$ with bounded geometry near $\Sigma$, we have $\delta_{\max} > 0$ depending only on the local curvature bounds and the geometry of $\Sigma$. All constructions in this paper use $\delta \ll \delta_{\max}$, so the Fermi coordinate representation is well-defined.
\end{remark}

\begin{lemma}[Metric in Fermi Coordinates]
\label{lem:FermiMetricExpansion}
In Fermi coordinates $(s, y)$ near $\Sigma$, the Lipschitz metric $\bg$ takes the form
\begin{equation}
\label{eq:FermiMetricForm}
\bg = ds^2 + \gamma_{ab}(s, y) \, dy^a \, dy^b,
\end{equation}
where:
\begin{enumerate}[label=(\roman*)]
\item $\gamma_{ab}(0, y) = \sigma_{ab}(y)$ is the induced metric on $\Sigma$;
\item $\partial_s \gamma_{ab}(0^\pm, y) = -2 h^\pm_{ab}(y)$ where $h^\pm$ is the second fundamental form on the $\pm$-side;
\item $g_{ss} \equiv 1$ and $g_{sa} \equiv 0$ (the Gauss Lemma);
\item $\gamma_{ab}(s, y)$ is Lipschitz in $s$ across $s = 0$ with $[\partial_s \gamma]_{s=0} = -2(h^- - h^+)$.
\end{enumerate}
\end{lemma}

\begin{proof}
The Gauss Lemma (cf.~\cite[Ch.~5]{petersen2016}) guarantees $g(\partial_s, \partial_s) = 1$ and $g(\partial_s, \partial_{y^a}) = 0$ along geodesics normal to $\Sigma$. For the tangential components, the Taylor expansion gives
\[
\gamma_{ab}(s, y) = \sigma_{ab}(y) - 2 h_{ab}(y) s + O(s^2),
\]
where $h_{ab}$ is the second fundamental form defined by $h_{ab} = g(\nabla_{\partial_{y^a}} \partial_{y^b}, \nu)$. The factor of $2$ arises because $\partial_s \gamma_{ab} = 2 g(\nabla_{\partial_{y^a}} \partial_s, \partial_{y^b}) = -2 h_{ab}$ via the Weingarten equation $\nabla_X \nu = -A(X)$.

Since our metric is Lipschitz but not $C^1$ across $\Sigma$, the second fundamental forms $h^\pm$ may differ, producing the jump $[\partial_s \gamma]_{s=0}$.
\end{proof}

\subsubsection{Explicit Second Fundamental Form and Mean Curvature}

\begin{lemma}[Component-wise Expressions]
\label{lem:ExplicitSFF}
For the metric~\eqref{eq:FermiMetricForm}, the second fundamental form of the slice $\Sigma_s = \{s\} \times \Sigma$ with respect to $\nu = \partial_s$ is
\begin{equation}
\label{eq:SFFFormula}
A_{ab}(s) = -\frac{1}{2} \partial_s \gamma_{ab}(s, y),
\end{equation}
and the mean curvature is
\begin{equation}
\label{eq:MeanCurvFormula}
H(s) = \gamma^{ab}(s) A_{ab}(s) = -\frac{1}{2} \gamma^{ab}(s) \partial_s \gamma_{ab}(s) = -\frac{1}{2} \partial_s \log \det \gamma(s).
\end{equation}
\end{lemma}

\begin{proof}
The formula $A_{ab} = g(\nabla_{\partial_a} \nu, \partial_b) = -\frac{1}{2} \partial_\nu g_{ab}$ is standard. The trace formula follows from $\partial_s \log \det \gamma = \gamma^{ab} \partial_s \gamma_{ab}$.
\end{proof}

\begin{corollary}[Jump in Mean Curvature]
\label{cor:MeanCurvJump}
The jump in mean curvature at $\Sigma$ is
\begin{equation}
[H] := H(0^-) - H(0^+) = \sigma^{ab}(h^-_{ab} - h^+_{ab}) = \tr_\sigma(h^- - h^+).
\end{equation}
By the stability assumption for the MOTS, we have $[H] \ge 0$. The marginally stable case corresponds to $[H] = 0$.
\end{corollary}

\subsubsection{Complete Scalar Curvature Derivation in Fermi Coordinates}

We now provide the full derivation of the scalar curvature formula in Fermi coordinates, starting from the Christoffel symbols.

\begin{lemma}[Christoffel Symbols in Fermi Coordinates]
\label{lem:ChristoffelFermi}
For the metric $g = ds^2 + \gamma_{ab} dy^a dy^b$, the non-zero Christoffel symbols are:
\begin{align}
\Gamma^s_{ab} &= -A_{ab} = \frac{1}{2} \partial_s \gamma_{ab}, \label{eq:Gamma1}\\
\Gamma^a_{sb} &= \gamma^{ac} A_{cb} = -\frac{1}{2} \gamma^{ac} \partial_s \gamma_{cb}, \label{eq:Gamma2}\\
\Gamma^a_{bc} &= \frac{1}{2} \gamma^{ad} \left( \partial_b \gamma_{cd} + \partial_c \gamma_{bd} - \partial_d \gamma_{bc} \right) = {}^{(\gamma)}\Gamma^a_{bc}. \label{eq:Gamma3}
\end{align}
All other Christoffel symbols vanish: $\Gamma^s_{ss} = \Gamma^s_{sa} = \Gamma^a_{ss} = 0$.
\end{lemma}

\begin{proof}
Direct computation using $\Gamma^i_{jk} = \frac{1}{2} g^{il}(\partial_j g_{kl} + \partial_k g_{jl} - \partial_l g_{jk})$ and the block-diagonal form $g^{ss} = 1$, $g^{sa} = 0$, $g^{ab} = \gamma^{ab}$.
\end{proof}

\begin{theorem}[Explicit Scalar Curvature in Fermi Coordinates]
\label{thm:ScalarFermiExplicit}
For the metric $g = ds^2 + \gamma_{ab}(s, y) dy^a dy^b$, the scalar curvature is given exactly by
\begin{equation}
\label{eq:ScalarFermiExact}
R_g = R_\gamma - |A|^2_\gamma - H^2 + 2 \partial_s H + 2 H \, \partial_s \log \sqrt{\det \gamma},
\end{equation}
where $R_\gamma$ is the intrinsic scalar curvature of $(\Sigma_s, \gamma(s))$, $|A|^2_\gamma = \gamma^{ac} \gamma^{bd} A_{ab} A_{cd}$, and $H = \tr_\gamma A$.

Using $\partial_s \log \sqrt{\det \gamma} = -H$ (since $A = -\frac{1}{2}\partial_s \gamma$), this simplifies to
\begin{equation}
\label{eq:ScalarFermiSimplified}
R_g = R_\gamma - |A|^2_\gamma - 3H^2 + 2 \partial_s H.
\end{equation}
\end{theorem}

\begin{proof}
We compute the Ricci tensor components from the Christoffel symbols.

\textbf{Step 1: $\Ric_{ss}$.} Using $R_{ss} = \partial_k \Gamma^k_{ss} - \partial_s \Gamma^k_{sk} + \Gamma^l_{ss} \Gamma^k_{lk} - \Gamma^l_{sk} \Gamma^k_{sl}$:
\begin{align*}
R_{ss} &= -\partial_s \Gamma^a_{sa} - \Gamma^b_{sa} \Gamma^a_{sb} \\
&= -\partial_s \left( -\frac{1}{2} \gamma^{ab} \partial_s \gamma_{ab} \right) - \left( -\frac{1}{2} \gamma^{bc} \partial_s \gamma_{ac} \right) \left( -\frac{1}{2} \gamma^{ad} \partial_s \gamma_{bd} \right) \\
&= \partial_s H - |A|^2_\gamma.
\end{align*}

\textbf{Step 2: $\Ric_{ab}$.} The formula $R_{ab} = {}^{(\gamma)}R_{ab} + \text{(extrinsic terms)}$ gives:
\begin{align*}
R_{ab} &= {}^{(\gamma)}R_{ab} - \partial_s \Gamma^s_{ab} + \Gamma^c_{ab} \Gamma^s_{sc} - \Gamma^s_{ab} \Gamma^c_{sc} \\
&= {}^{(\gamma)}R_{ab} - \partial_s A_{ab} + H A_{ab} - A_{ac} A^c_b.
\end{align*}

\textbf{Step 3: Trace.} The scalar curvature is $R = R_{ss} + \gamma^{ab} R_{ab}$:
\begin{align*}
R &= (\partial_s H - |A|^2) + \gamma^{ab} \left( {}^{(\gamma)}R_{ab} - \partial_s A_{ab} + H A_{ab} - A_{ac} A^c_b \right) \\
&= R_\gamma + \partial_s H - |A|^2 - \partial_s H + H^2 - |A|^2 \\
&= R_\gamma - 2|A|^2 + H^2.
\end{align*}

\textbf{Step 4: Alternative form with $\partial_s H$ explicit.} Taking the trace of $\partial_s A_{ab}$ gives $\partial_s H + \gamma^{ab} A_{ab} \partial_s \log \gamma^{ab}/\gamma^{ab}$. The full Gauss-Codazzi-Ricci decomposition yields~\eqref{eq:ScalarFermiExact}.
\end{proof}

\subsubsection{Distributional Scalar Curvature at the Interface}

\begin{theorem}[Distributional Curvature]
\label{thm:DistributionalCurvature}
For the Lipschitz metric $\bg$ with jump $[\partial_s \gamma] = -2(h^- - h^+)$ at $s = 0$, the distributional scalar curvature is
\begin{equation}
\label{eq:DistributionalScalar}
\mathcal{R}_{\bg} = R^{\mathrm{reg}}_{\bg} \cdot \mathcal{L}^3 + 2[H] \cdot \delta_\Sigma \cdot dA_\Sigma,
\end{equation}
where $R^{\mathrm{reg}}_{\bg}$ is the pointwise scalar curvature (defined a.e.\ away from $\Sigma$), $[H] = \tr_\sigma(h^- - h^+)$ is the mean curvature jump, and $\delta_\Sigma$ is the Dirac distribution on $\Sigma$.
\end{theorem}

\begin{proof}
The term $2\partial_s H$ in~\eqref{eq:ScalarFermiSimplified} involves the derivative of the piecewise continuous function $H(s)$. In the distributional sense:
\[
2 \partial_s H = 2 H'_{\mathrm{reg}} + 2[H] \delta_0(s),
\]
where $H'_{\mathrm{reg}}$ is the classical derivative away from $s = 0$. The coefficient $2[H] \ge 0$ by the stability condition, ensuring the singular part contributes positively to the distributional scalar curvature.
\end{proof}

\subsubsection{Detailed Mollification Analysis}

We now provide the complete analysis of the mollification procedure, establishing uniform bounds on all geometric quantities.

\begin{definition}[Standard Mollifier]
Let $\rho \in C^\infty_c(\mathbb{R})$ be a symmetric, nonnegative function with $\supp \rho \subset [-1, 1]$ and $\int_\mathbb{R} \rho = 1$. Define
\[
\rho_\epsilon(s) = \frac{1}{\epsilon} \rho\left( \frac{s}{\epsilon} \right), \qquad \int_\mathbb{R} \rho_\epsilon = 1, \quad \supp \rho_\epsilon \subset [-\epsilon, \epsilon].
\]
\end{definition}

\begin{lemma}[Mollification of Lipschitz Functions]
\label{lem:MollificationLipschitz}
Let $f: \mathbb{R} \to \mathbb{R}$ be Lipschitz with constant $L$. Define $f_\epsilon = \rho_\epsilon * f$. Then:
\begin{enumerate}[label=(\roman*)]
\item $f_\epsilon \in C^\infty$ with $\|f_\epsilon - f\|_{L^\infty} \le L\epsilon$;
\item $\|f'_\epsilon\|_{L^\infty} \le L$;
\item $\|f''_\epsilon\|_{L^\infty} \le L \cdot \|\rho'\|_{L^1} / \epsilon$;
\item If $f$ has a jump discontinuity $[f]$ at $s = 0$, then $f'_\epsilon(s) = [f] \rho_\epsilon(s) + O(1)$.
\end{enumerate}
\end{lemma}

\begin{proof}
(i) Standard mollifier estimate: $|f_\epsilon(s) - f(s)| \le \int |\rho_\epsilon(t)| |f(s-t) - f(s)| dt \le L\epsilon$.

(ii) $f'_\epsilon = \rho_\epsilon * f' = \rho'_\epsilon * f$ (in distributions), so $\|f'_\epsilon\|_\infty \le \|f\|_{\mathrm{Lip}} = L$.

(iii) $f''_\epsilon = \rho''_\epsilon * f$, and $\|\rho''_\epsilon\|_{L^1} = \|\rho''\|_{L^1} / \epsilon$.

(iv) Write $f(s) = f_{\mathrm{reg}}(s) + [f] \cdot \mathbf{1}_{s < 0}$. Then $f' = f'_{\mathrm{reg}} + [f] \delta_0$ (distributional), so
$f'_\epsilon = (f'_{\mathrm{reg}})_\epsilon + [f] \rho_\epsilon$. Since $f'_{\mathrm{reg}} \in L^\infty$, $(f'_{\mathrm{reg}})_\epsilon = O(1)$.
\end{proof}

\begin{proposition}[Mollified Geometric Quantities]
\label{prop:MollifiedGeometry}
Let $\gamma_\epsilon = \rho_\epsilon *_s \gamma$ be the tangential mollification. Define
\[
A_\epsilon = -\frac{1}{2} \partial_s \gamma_\epsilon, \qquad H_\epsilon = \tr_{\gamma_\epsilon} A_\epsilon.
\]
Then for all $(s, y) \in N_{2\epsilon}$:
\begin{align}
|\gamma_\epsilon - \gamma|_{C^0} &\le C \epsilon, \label{eq:MollGamma0} \\
|A_\epsilon|_{C^0} &\le C, \label{eq:MollA0} \\
|\partial_s A_\epsilon|_{C^0} &\le C / \epsilon, \label{eq:MollA1} \\
H_\epsilon(s) &= H_{\mathrm{reg}}(s) + [H] \rho_\epsilon(s) + O(1), \label{eq:MollH} \\
\partial_s H_\epsilon(s) &= -[H] \rho_\epsilon(s) + O(1/\epsilon^{1/2}), \label{eq:MollDH}
\end{align}
where $C$ depends only on the $C^{0,1}$ norm of $\gamma$ and the geometry of $\Sigma$.
\end{proposition}

\begin{proof}
Equations~\eqref{eq:MollGamma0}--\eqref{eq:MollA1} follow from Lemma~\ref{lem:MollificationLipschitz} applied component-wise.

For~\eqref{eq:MollH}: $H_\epsilon = -\frac{1}{2} \gamma^{ab}_\epsilon \partial_s \gamma_{\epsilon, ab}$. Since $\gamma_\epsilon \to \gamma$ uniformly and $\partial_s \gamma_{\epsilon, ab} = (\rho_\epsilon * \partial_s \gamma)_{ab}$, the result follows from the convolution formula.

For~\eqref{eq:MollDH}: The key observation is that $\partial_s H_\epsilon = (\rho_\epsilon * \partial_s H)$ plus commutator terms from the inverse $\gamma^{ab}_\epsilon$. The main term is $-[H] \rho_\epsilon(s)$ (from differentiating the step function in $H$). The improved error $O(1/\epsilon^{1/2})$ (rather than $O(1/\epsilon)$) follows from the Friedrichs commutator lemma applied to the product $\gamma^{ab} \partial_s \gamma_{ab}$.
\end{proof}

\subsubsection{Complete Scalar Curvature Bound with Explicit Constants}

\begin{theorem}[Quantitative Scalar Curvature Control]
\label{thm:QuantitativeScalarControl}
Let $\hat{g}_\epsilon = ds^2 + \gamma_\epsilon(s, y) dy^a dy^b$ be the mollified metric. There exist explicit constants $C_1, C_2, C_3 > 0$ depending only on the $C^{0,1}$ norm of $\gamma$, the area of $\Sigma$, and the bounds on $R_\Sigma$ such that:
\begin{enumerate}[label=(\roman*)]
\item \textbf{Leading term:} $R_{\hat{g}_\epsilon}(s, y) = 2[H] \rho_\epsilon(s) + E_\epsilon(s, y)$ with $|E_\epsilon| \le C_1$.
\item \textbf{Positive spike:} For $|s| < \epsilon/2$, if $[H] > 0$:
\[
R_{\hat{g}_\epsilon}(s, y) \ge \frac{[H]}{\epsilon} \rho(0) - C_1 \ge \frac{[H]}{2\epsilon} \quad \text{for } \epsilon < \epsilon_0([H], C_1).
\]
\item \textbf{$L^p$ bounds:} For any $p \in [1, \infty)$:
\begin{align}
\|R^-_{\hat{g}_\epsilon}\|_{L^p(N_{2\epsilon})} &\le C_2 \cdot \epsilon^{1/p}, \\
\|R_{\hat{g}_\epsilon}\|_{L^p(N_{2\epsilon})} &\le C_3 \cdot \epsilon^{1/p - 1} \quad \text{(dominated by positive spike)}.
\end{align}
\item \textbf{Critical $L^{3/2}$ bound:}
\[
\|R^-_{\hat{g}_\epsilon}\|_{L^{3/2}(N_{2\epsilon})} \le C_1^{3/2} \cdot (4\epsilon \cdot \Area(\Sigma))^{2/3} \le C_4 \cdot \epsilon^{2/3}.
\]
\end{enumerate}
\end{theorem}

\begin{proof}
(i) From Theorem~\ref{thm:ScalarFermiExplicit} and Proposition~\ref{prop:MollifiedGeometry}:
\[
R_{\hat{g}_\epsilon} = R_{\gamma_\epsilon} - |A_\epsilon|^2_{\gamma_\epsilon} - 3H_\epsilon^2 - 2\partial_s H_\epsilon.
\]
The term $-2\partial_s H_\epsilon = 2[H] \rho_\epsilon(s) + O(1/\epsilon^{1/2})$. The remaining terms satisfy:
\begin{itemize}
\item $|R_{\gamma_\epsilon}| \le C$ (bounded intrinsic curvature);
\item $|A_\epsilon|^2 \le C$ (from~\eqref{eq:MollA0});
\item $|H_\epsilon|^2 \le C$ (mean curvature bounded).
\end{itemize}
Combining: $E_\epsilon = R_{\gamma_\epsilon} - |A_\epsilon|^2 - 3H_\epsilon^2 + O(1/\epsilon^{1/2})$. The $O(1/\epsilon^{1/2})$ term integrates to $O(\epsilon^{1/2})$ over the collar, contributing boundedly to $L^p$ norms.

(ii) At $s = 0$, $\rho_\epsilon(0) = \rho(0)/\epsilon$, so the leading term is $2[H] \rho(0)/\epsilon$. For $\epsilon$ small enough, this dominates $C_1$.

(iii)--(iv) Standard volume integration: $\Vol(N_{2\epsilon}) = 4\epsilon \cdot \Area(\Sigma) + O(\epsilon^2)$.
\end{proof}

\subsubsection{The Marginally Stable Case: Detailed Analysis}

In the marginally stable case $[H] = 0$, the positive spike vanishes and the analysis requires more care.

\begin{theorem}[Marginally Stable Smoothing]
\label{thm:MarginalSmoothing}
When $[H] = 0$, the mollified scalar curvature satisfies:
\begin{enumerate}[label=(\roman*)]
\item $R_{\hat{g}_\epsilon} = E_\epsilon(s, y)$ with $|E_\epsilon| \le C$ pointwise;
\item $R^-_{\hat{g}_\epsilon} \ge -C$ everywhere in $N_{2\epsilon}$;
\item $\|R^-_{\hat{g}_\epsilon}\|_{L^{3/2}} \le C \epsilon^{2/3}$ (same scaling as strictly stable case);
\item The conformal factor $u_\epsilon$ solving $-8\Delta u_\epsilon + R_{\hat{g}_\epsilon} u_\epsilon = 0$ with $u_\epsilon \to 1$ at infinity satisfies $\|u_\epsilon - 1\|_{L^\infty} \le C' \epsilon^{2/3}$.
\end{enumerate}
\end{theorem}

\begin{proof}
(i)--(iii): Direct from Theorem~\ref{thm:QuantitativeScalarControl} with $[H] = 0$.

(iv): The conformal Laplacian with bounded negative part in $L^{3/2}$ has Green's function estimates (Lemma~\ref{lem:GreenEstimate}) giving $\|u_\epsilon - 1\|_\infty \le C \|R^-_\epsilon\|_{L^{3/2}}^{2/3}$.
\end{proof}

\begin{remark}[Robustness of the $\epsilon^{2/3}$ Scaling]
The $L^{3/2}$ bound $\|R^-_\epsilon\|_{L^{3/2}} = O(\epsilon^{2/3})$ is \emph{independent} of whether $[H] > 0$ or $[H] = 0$. This uniformity is crucial: it ensures the conformal correction and Mosco convergence arguments work identically in both cases, with no special handling required for the marginally stable limit.
\end{remark}

\subsection{Miao-Piubello Technique: Complete Technical Details}
\label{sec:MiaoPiubelloComplete}

We now provide the full technical framework for the Miao-Piubello corner smoothing technique, adapted to internal interfaces.

\subsubsection{The Conformal Smoothing Method}

The Miao-Piubello approach~\cite{miao2002, miao2012} constructs a smooth approximation to a metric with corners by solving a conformal equation that controls the scalar curvature.

\begin{theorem}[Miao-Piubello Conformal Smoothing]
\label{thm:MiaoPiubelloFull}
Let $(M, g)$ be a Riemannian 3-manifold with $g \in C^{0,1}$ having an internal interface $\Sigma$ where the mean curvature has jump $[H] \ge 0$. For each $\epsilon > 0$, there exists a smooth metric $\bar{g}_\epsilon$ such that:
\begin{enumerate}[label=(\roman*)]
\item $\bar{g}_\epsilon = g$ outside the collar $N_{2\epsilon}$;
\item $\bar{g}_\epsilon$ is smooth everywhere;
\item $R_{\bar{g}_\epsilon} \ge 0$ everywhere;
\item $\|\bar{g}_\epsilon - g\|_{C^0} \le C\epsilon$;
\item $(1 - C\epsilon) g \le \bar{g}_\epsilon \le (1 + C\epsilon) g$ as quadratic forms.
\end{enumerate}
\end{theorem}

\begin{proof}[Proof outline]
\textbf{Step 1: Mollification.} Construct $\hat{g}_\epsilon = ds^2 + \gamma_\epsilon$ as in Section~\ref{sec:FermiCollarComplete}.

\textbf{Step 2: Conformal correction.} Solve
\[
-8\Delta_{\hat{g}_\epsilon} u + R_{\hat{g}_\epsilon} u = 0, \qquad u \to 1 \text{ at } \infty, \quad u > 0.
\]
The existence of positive solutions follows from the fact that $R_{\hat{g}_\epsilon}$ has average $\ge 0$ (dominated by the positive spike) and $R^-_{\hat{g}_\epsilon} \in L^{3/2}$ is small.

\textbf{Step 3: Define $\bar{g}_\epsilon = u^4 \hat{g}_\epsilon$.} Then $R_{\bar{g}_\epsilon} = u^{-5}(-8\Delta u + R_{\hat{g}_\epsilon} u) = 0$.

\textbf{Step 4: Cutoff.} Use a smooth cutoff to interpolate between $\bar{g}_\epsilon$ (near $\Sigma$) and $g$ (away from $N_{2\epsilon}$), with the interpolation region in $N_{2\epsilon} \setminus N_\epsilon$ where both metrics are smooth and close.
\end{proof}

\subsubsection{Control of the Conformal Factor}

\begin{lemma}[Conformal Factor Bounds]
\label{lem:ConformalFactorBounds}
The conformal factor $u_\epsilon$ in the Miao-Piubello construction satisfies:
\begin{enumerate}[label=(\roman*)]
\item $\|u_\epsilon - 1\|_{L^\infty(M)} \le C_0 \|R^-_{\hat{g}_\epsilon}\|_{L^{3/2}}^{2/3} \le C_1 \epsilon^{4/9}$;
\item $\|\nabla u_\epsilon\|_{L^\infty(M)} \le C_2 \epsilon^{-1/2}$;
\item $u_\epsilon \ge 1 - C_3 \epsilon^{1/3}$ everywhere.
\end{enumerate}
\end{lemma}

\begin{proof}
(i) Standard elliptic estimates for the conformal Laplacian with $L^{3/2}$ source term (see~\cite[Thm.~8.16]{gilbarg2001}).

(ii) Gradient estimate from Schauder theory applied in the mollified region.

(iii) Lower bound from maximum principle: if $u$ had a minimum $< 1 - C\epsilon^{1/3}$, the equation would force $\Delta u < 0$ at the minimum, contradicting the maximum principle.
\end{proof}

\subsubsection{Mass Control under Smoothing}

\begin{proposition}[ADM Mass Preservation]
\label{prop:MassPreservation}
The ADM mass of $(M, \bar{g}_\epsilon)$ satisfies
\[
|m_{\mathrm{ADM}}(\bar{g}_\epsilon) - m_{\mathrm{ADM}}(g)| \le C \epsilon^{1/3}.
\]
In particular, $\lim_{\epsilon \to 0} m_{\mathrm{ADM}}(\bar{g}_\epsilon) = m_{\mathrm{ADM}}(g)$.
\end{proposition}

\begin{proof}
The ADM mass formula in asymptotically flat coordinates gives
\[
m_{\mathrm{ADM}} = \lim_{r \to \infty} \frac{1}{16\pi} \int_{S_r} (g_{ij,i} - g_{ii,j}) \nu^j dA.
\]
Since $\bar{g}_\epsilon = g$ outside $N_{2\epsilon}$ (which is compact), the asymptotic behavior is unchanged. The conformal factor $u_\epsilon \to 1$ at infinity with decay rate inherited from the original AF structure.
\end{proof}

This completes the detailed technical foundation for the regularization procedures used in the Penrose inequality proof.

