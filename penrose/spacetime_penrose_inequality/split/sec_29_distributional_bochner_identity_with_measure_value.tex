\section{Distributional Bochner Identity with Measure-Valued Curvature}
\label{app:WeakBochner}

This appendix provides the detailed technical foundations for the distributional Bochner inequality presented in Theorem~\ref{thm:DistrBochner}. The key innovation is extending the classical Bochner identity to settings where the scalar curvature is a signed measure rather than a function.

\subsection{Setup and Preliminaries}
Let $(M, g)$ be a complete Riemannian manifold of dimension $n = 3$ with $g \in C^{0,1}(M)$. The Christoffel symbols $\Gamma^k_{ij}$ are bounded measurable functions, and the curvature tensor is defined in the distributional sense.

\begin{definition}[Distributional Curvature Tensor]
The Riemann curvature tensor $\mathcal{R}_{ijkl} \in \mathcal{D}'(M)$ is defined by
\begin{equation}
    \langle \mathcal{R}_{ijkl}, \varphi \rangle := -\int_M \varphi \left( \partial_i \Gamma^l_{jk} - \partial_j \Gamma^l_{ik} + \Gamma^l_{im} \Gamma^m_{jk} - \Gamma^l_{jm} \Gamma^m_{ik} \right) dV_g
\end{equation}
for $\varphi \in C^\infty_c(M)$. The scalar curvature distribution is $\mathcal{R} = g^{ik} g^{jl} \mathcal{R}_{ijkl}$.
\end{definition}

\begin{lemma}[Decomposition of Distributional Scalar Curvature]\label{lem:ScalarDecomp}
If $g \in C^{0,1}$ and the pointwise scalar curvature $R_g^{\text{reg}}$ is well-defined a.e., then
\begin{equation}
    \mathcal{R} = R_g^{\text{reg}} \cdot \mathcal{L}^3 + \mathcal{R}^{\text{sing}},
\end{equation}
where $\mathcal{L}^3$ is the Lebesgue measure and $\mathcal{R}^{\text{sing}}$ is a signed measure supported on the singular set $\Sigma_g = \{x : g \text{ is not } C^{1,1} \text{ near } x\}$.
\end{lemma}

\begin{proof}
The Christoffel symbols satisfy $\Gamma^k_{ij} \in L^\infty(M)$. Away from $\Sigma_g$, the metric is $C^{1,1}$, so $\partial_i \Gamma^k_{jl}$ exists classically and equals $R^{\text{reg}}$. Near $\Sigma_g$, the distributional derivative may concentrate, producing the singular part.
\end{proof}

\subsection{The Weighted Bochner Identity}
For a smooth $p$-harmonic function $u$ on a smooth manifold, the classical Bochner identity reads:
\begin{equation}
    \frac{1}{2} \Delta |\nabla u|^2 = |\nabla^2 u|^2 + \langle \nabla \Delta u, \nabla u \rangle + \Ric(\nabla u, \nabla u).
\end{equation}

For $p$-harmonic functions satisfying $\Div(|\nabla u|^{p-2} \nabla u) = 0$, we derive a weighted version.

\begin{proposition}[Weighted Bochner Identity]
Let $u \in C^3(M \setminus \Sigma_g) \cap W^{1,p}(M)$ be weakly $p$-harmonic. Then on $M \setminus \Sigma_g$:
\begin{multline}\label{eq:WeightedBochner}
    \Div\left( |\nabla u|^{p-2} \nabla \frac{|\nabla u|^2}{2} \right) = |\nabla u|^{p-2} |\nabla^2 u|^2 + \frac{p-2}{2} |\nabla u|^{p-4} |\nabla |\nabla u|^2|^2 \\
    + |\nabla u|^{p-2} \Ric(\nabla u, \nabla u) + |\nabla u|^{p-2} \langle \nabla \Delta u, \nabla u \rangle.
\end{multline}
\end{proposition}

\begin{proof}
Start with the identity $\Div(f \nabla v) = f \Delta v + \langle \nabla f, \nabla v \rangle$ applied to $f = |\nabla u|^{p-2}$ and $v = \frac{|\nabla u|^2}{2}$:
\begin{align*}
    \Div\left( |\nabla u|^{p-2} \nabla \frac{|\nabla u|^2}{2} \right) &= |\nabla u|^{p-2} \Delta \frac{|\nabla u|^2}{2} + \left\langle \nabla |\nabla u|^{p-2}, \nabla \frac{|\nabla u|^2}{2} \right\rangle.
\end{align*}
Using the classical Bochner formula for $\Delta \frac{|\nabla u|^2}{2}$ and computing $\nabla |\nabla u|^{p-2} = (p-2) |\nabla u|^{p-4} \nabla \frac{|\nabla u|^2}{2}$ yields~\eqref{eq:WeightedBochner}.
\end{proof}

\subsection{Extension to Lipschitz Metrics}
The main technical challenge is passing to the limit when the metric has only Lipschitz regularity.

\begin{theorem}[Distributional Weighted Bochner]\label{thm:DistrWeightedBochner}
Let $(M, g)$ satisfy $g \in C^{0,1}$ with $\mathcal{R} \ge -\Lambda$ as distributions for some $\Lambda \ge 0$. Let $u \in W^{1,p}(M)$ be weakly $p$-harmonic. Then for any nonnegative $\varphi \in C^\infty_c(M)$:
\begin{multline}\label{eq:DistrBochnerFull}
    \int_M \varphi \, |\nabla u|^{p-2} |\nabla^2 u|^2 \, dV_g \le \int_M |\nabla u|^{p-2} \left\langle \nabla \frac{|\nabla u|^2}{2}, \nabla \varphi \right\rangle dV_g \\
    + \frac{\Lambda}{3} \int_M \varphi \, |\nabla u|^p \, dV_g + \int_M \varphi \, |\nabla u|^p \, d\mathcal{R}^-,
\end{multline}
where the last integral is against the negative part of the singular measure $\mathcal{R}^{\text{sing}}$.
\end{theorem}

\begin{proof}
\textbf{Step 1: Mollification.}
Let $g_\epsilon = \rho_\epsilon * g$ be a standard mollification. The smoothed metric satisfies $g_\epsilon \in C^\infty$ and $\|g_\epsilon - g\|_{C^0} \le C\epsilon$.

Let $u_\epsilon$ be the $p$-harmonic function on $(M, g_\epsilon)$ with the same boundary data as $u$. By stability of $p$-harmonic functions, $u_\epsilon \to u$ in $W^{1,p}_{\mathrm{loc}}$.

\textbf{Step 2: Classical Bochner on smooth approximation.}
On $(M, g_\epsilon)$, the classical weighted Bochner identity~\eqref{eq:WeightedBochner} holds pointwise. Integrating against $\varphi \ge 0$ and using integration by parts:
\begin{multline}
    -\int_M |\nabla u_\epsilon|_{g_\epsilon}^{p-2} \left\langle \nabla \frac{|\nabla u_\epsilon|^2}{2}, \nabla \varphi \right\rangle_{g_\epsilon} dV_{g_\epsilon} = \int_M \varphi \, |\nabla u_\epsilon|^{p-2} |\nabla^2 u_\epsilon|^2 \, dV_{g_\epsilon} \\
    + \frac{p-2}{2} \int_M \varphi \, |\nabla u_\epsilon|^{p-4} |\nabla |\nabla u_\epsilon|^2|^2 \, dV_{g_\epsilon} + \int_M \varphi \, |\nabla u_\epsilon|^{p-2} \Ric_{g_\epsilon}(\nabla u_\epsilon, \nabla u_\epsilon) \, dV_{g_\epsilon}.
\end{multline}

\textbf{Step 3: Curvature term.}
The Ricci term satisfies:
\begin{equation}
    \Ric_{g_\epsilon}(\nabla u_\epsilon, \nabla u_\epsilon) \ge \frac{R_{g_\epsilon}}{n-1} |\nabla u_\epsilon|^2 \ge -\frac{\Lambda + C\epsilon}{2} |\nabla u_\epsilon|^2
\end{equation}
using the lower bound $R_{g_\epsilon} \ge R_g - C\epsilon \ge -\Lambda - C\epsilon$ (by stability of distributional curvature under mollification).

\textbf{Step 4: Passage to the limit.}
Taking $\epsilon \to 0$:
\begin{itemize}
    \item The left-hand side converges by weak convergence of $\nabla u_\epsilon$ in $L^p$.
    \item The Hessian term on the right satisfies $\liminf \int \varphi |\nabla^2 u_\epsilon|^2 |\nabla u_\epsilon|^{p-2} \ge \int \varphi |\nabla^2 u|^2 |\nabla u|^{p-2}$ by weak lower semicontinuity.
    \item The Ricci term converges, with the singular part contributing via the measure $\mathcal{R}^-$.
\end{itemize}
Rearranging yields~\eqref{eq:DistrBochnerFull}.
\end{proof}

\begin{remark}[Detailed Justification for Lipschitz Metrics]\label{rem:LipschitzBochnerJustification}
The extension of the Bochner identity to Lipschitz metrics requires careful justification of three technical points:

\textbf{(i) Existence of $\nabla^2 u$ in $L^2$:} For a $p$-harmonic function $u$ on a $C^{0,1}$ metric, Tolksdorf's regularity theorem \cite{tolksdorf1984} yields $u \in C^{1,\alpha}_{\mathrm{loc}}$ for some $\alpha > 0$. The second derivatives $\nabla^2 u$ exist in $L^2_{\mathrm{loc}}$ by Calderon--Zygmund theory applied to the linearized equation
\[
\Div(A(x,\nabla u)\nabla v) = f,
\]
where $A(x,\xi) = |\xi|^{p-2}(I + (p-2)\hat\xi \otimes \hat\xi)$ is the coefficient matrix. For $p \in (1,2]$, the ellipticity degenerates only at $\{|\nabla u|=0\}$, which has measure zero by unique continuation. On $\{|\nabla u| > 0\}$, the equation is uniformly elliptic with $L^\infty$ coefficients (since $\nabla u \in C^{0,\alpha}$), so standard $W^{2,2}$ theory applies.

\textbf{(ii) Integration by parts across the singular set:} The singular set $\Sigma_g$ (where $g$ fails to be $C^{1,1}$) has codimension $\ge 1$. For the Jang-conformal metric, $\Sigma_g = \Sigma$ (the MOTS interface), which is a smooth 2-dimensional surface. The integration-by-parts identity
\[
\int_\Omega \Div(X) \, dV = \int_{\partial\Omega} \langle X, \nu \rangle \, d\sigma + \int_{\Sigma \cap \Omega} [X \cdot \nu_\Sigma] \, d\mathcal{H}^2
\]
holds for vector fields $X \in L^\infty$ with $\Div(X) \in L^1$, where $[X\cdot\nu_\Sigma]$ denotes the jump across $\Sigma$. The singular curvature contribution arises from this jump term.

\textbf{(iii) Stability of mollification:} The mollified metric $g_\epsilon = \rho_\epsilon * g$ satisfies: (a) $\|g_\epsilon - g\|_{C^0} = O(\epsilon)$ by standard approximation theory; (b) $R_{g_\epsilon} \to \mathcal{R}_g$ in the sense of distributions, with the singular part concentrating as $\epsilon \to 0$ (this uses the specific structure of Lipschitz corners---see \cite{lee2019}); (c) the $p$-harmonic functions $u_\epsilon$ on $(M,g_\epsilon)$ converge to $u$ in $W^{1,p}$ by the stability theorem for quasilinear elliptic equations \cite{lindqvist2017}.
\end{remark}

\begin{remark}[Interaction of Singular Curvature with Vanishing Gradient]\label{rem:SingCurvGradInteraction}
A subtle point in the distributional Bochner inequality~\eqref{eq:DistrBochnerFull} concerns the integral $\int_M \varphi |\nabla u|^p \, d\mathcal{R}^-$ when the gradient $\nabla u$ might vanish on (part of) the support of the singular measure $\mathcal{R}^{\text{sing}}$. We address this in detail.

\textbf{(I) Structure of the Singular Set:}
In our application, the singular measure $\mathcal{R}^{\text{sing}}$ is supported on the horizon $\Sigma$, where the metric $\tg$ has a Lipschitz corner (the mean curvature jump). Specifically:
\begin{equation}
    \mathcal{R}^{\text{sing}} = [H]_{\tg} \cdot \mathcal{H}^2|_\Sigma,
\end{equation}
where $[H]_{\tg} \ge 0$ by the stability condition (Theorem~\ref{thm:CompleteMeanCurvatureJump}).

\textbf{(II) Gradient Behavior Near the Horizon:}
The $p$-harmonic function $u$ satisfies $u|_\Sigma = 0$ with $u \to 1$ at infinity. By the strong maximum principle and Hopf boundary lemma for $p$-harmonic functions \cite{tolksdorf1984, lieberman1988}, the gradient is bounded away from zero near $\Sigma$:
\begin{equation}
    |\nabla u|(x) \ge c \cdot d(x, \Sigma)^{(p-2)/(p-1)} \quad \text{for } x \in N_\delta(\Sigma) \setminus \Sigma,
\end{equation}
where $c > 0$ depends on the geometry of $\Sigma$ and the ellipticity of the $p$-Laplacian. For $p > 1$, this gives $|\nabla u| > 0$ on $N_\delta(\Sigma) \setminus \Sigma$.

However, the \emph{trace} of $|\nabla u|^p$ on $\Sigma$ itself requires care. We analyze this via the one-sided limits:
\begin{equation}
    |\nabla u|^p|_{\Sigma^\pm} = \lim_{s \to 0^\pm} |\nabla u(x_0 + s\nu)|^p,
\end{equation}
where $\nu$ is the unit normal to $\Sigma$ and $x_0 \in \Sigma$.

\textbf{(III) Trace Lemma for $p$-Harmonic Functions:}
\begin{lemma}\label{lem:GradientTrace}
Let $u$ be the $p$-harmonic function on $(\tM, \tg)$ with $u|_\Sigma = 0$. Then:
\begin{enumerate}
    \item[(a)] The one-sided traces $|\nabla u|^p|_{\Sigma^\pm}$ exist in $L^1(\Sigma, \mathcal{H}^2)$.
    \item[(b)] For almost every $x_0 \in \Sigma$:
    \begin{equation}
        |\nabla u|^p|_{\Sigma^+} = |\nabla u|^p|_{\Sigma^-} =: |\nabla u|^p|_\Sigma.
    \end{equation}
    \item[(c)] The trace satisfies $|\nabla u|^p|_\Sigma \ge c_0 > 0$ on a set of positive measure in $\Sigma$.
\end{enumerate}
\end{lemma}

\begin{proof}
Part (a) follows from the $W^{1,p}$ regularity of $u$ and the trace theorem for Sobolev functions on Lipschitz domains.

For part (b), the continuity of the trace follows from the Lipschitz regularity of $u$ across $\Sigma$. Since $u \in C^{0,1}(\tM)$ by Tolksdorf's theorem, the gradient has well-defined one-sided limits that agree $\mathcal{H}^2$-a.e.\ on $\Sigma$.

Part (c) follows from the fact that $u$ is not constant (since $u|_\Sigma = 0$ and $u \to 1$ at infinity), combined with the unique continuation property for $p$-harmonic functions: if $|\nabla u|$ vanished on all of $\Sigma$, then $u \equiv 0$ by the strong unique continuation theorem of Garofalo--Lin \cite{garofalolin1987}.
\end{proof}

\textbf{(IV) Well-Definedness of the Singular Integral:}
With Lemma~\ref{lem:GradientTrace}, the integral
\begin{equation}
    \int_M \varphi |\nabla u|^p \, d\mathcal{R}^- = \int_\Sigma \varphi|_\Sigma \cdot |\nabla u|^p|_\Sigma \cdot [H]_{\tg}^- \, d\mathcal{H}^2
\end{equation}
is well-defined in $[0, \infty]$. Moreover, since $[H]_{\tg} \ge 0$ by our stability analysis, we have $\mathcal{R}^- = 0$ at $\Sigma$, and the integral is identically zero in our setting.

\textbf{(V) The Critical Set of $u$:}
Away from the horizon, the critical set
\begin{equation}
    \Crit(u) = \{x \in \tM : \nabla u(x) = 0\}
\end{equation}
has measure zero by Sard's theorem (since $u \in C^{1,\alpha}$ away from $\Sigma$). Moreover, by the unique continuation property, $\Crit(u)$ has Hausdorff dimension at most $n-2 = 1$. The singular measure $\mathcal{R}^{\text{sing}}$ is supported on $\Sigma$, which has dimension $2$, so $\Crit(u) \cap \supp(\mathcal{R}^{\text{sing}})$ has $\mathcal{H}^2$-measure zero.

\textbf{(VI) Conclusion:}
The integral $\int \varphi |\nabla u|^p \, d\mathcal{R}^-$ is well-defined and finite because:
\begin{enumerate}
    \item The trace $|\nabla u|^p|_\Sigma$ exists and is non-zero $\mathcal{H}^2$-a.e.
    \item The set where the trace vanishes has $\mathcal{H}^2$-measure zero.
    \item In our setting, $\mathcal{R}^- \equiv 0$ (the mean curvature jump is nonnegative).
\end{enumerate}
This completes the justification of the distributional Bochner inequality in the presence of singular curvature.
\end{remark}

\subsection{Application to AMO Monotonicity}
The distributional Bochner inequality directly implies the monotonicity of the AMO functional.

\begin{corollary}
Under the hypotheses of Theorem~\ref{thm:DistrWeightedBochner}, if additionally $\mathcal{R} \ge 0$ (i.e., $\mathcal{R}^- = 0$), then the AMO functional $\mathcal{M}_p(t)$ is nondecreasing in $t$.
\end{corollary}

\begin{proof}
The derivative $\mathcal{M}_p'(t)$ is expressed as an integral over the level set $\{u = t\}$ involving the Bochner term and the curvature term. When $\mathcal{R} \ge 0$, the inequality~\eqref{eq:DistrBochnerFull} with $\varphi$ a test function localizing near $\{u = t\}$ shows each term is nonnegative, hence $\mathcal{M}_p'(t) \ge 0$.
\end{proof}

