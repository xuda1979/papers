\section{Weak Inverse Mean Curvature Flow and Hawking Mass Monotonicity}
\label{app:WeakIMCF}

This appendix develops the weak formulation of inverse mean curvature flow (IMCF) in the context of initial data sets with the dominant energy condition, proving the monotonicity of the generalized Hawking mass without requiring nonnegative scalar curvature.

\subsection{Level Set Formulation}
Following Huisken--Ilmanen, we formulate IMCF as the level sets of a function $u: M \setminus \Sigma \to [0, \infty)$ satisfying:
\begin{equation}\label{eq:LevelSetIMCF}
    \Div\left( \frac{\nabla u}{|\nabla u|} \right) = |\nabla u|.
\end{equation}
This is equivalent to the evolution $\partial_t \Sigma_t = H^{-1} \nu$, where $H$ is the mean curvature.

\begin{definition}[Weak Solution to IMCF]\label{def:WeakIMCF}
A function $u \in BV_{\mathrm{loc}}(M \setminus \Sigma) \cap C^0(M \setminus \Sigma)$ is a \emph{weak solution} to IMCF starting from $\Sigma$ if:
\begin{enumerate}
    \item $u(x) \to 0$ as $x \to \Sigma$ and $u(x) \to \infty$ as $x \to \infty$,
    \item The level sets $\Sigma_t = \partial^* \{u > t\}$ are sets of locally finite perimeter,
    \item For a.e. $t > 0$, the variational inequality holds:
    \begin{equation}\label{eq:WeakIMCFVar}
        \frac{d}{dt} \int_{\{u > t\}} \varphi \, dV_g \ge \int_{\Sigma_t} \varphi \, H^{-1} \, d\sigma
    \end{equation}
    for all nonnegative $\varphi \in C^\infty_c(M)$, where $H$ is the generalized mean curvature of $\Sigma_t$.
\end{enumerate}
\end{definition}

\subsection{Existence via \texorpdfstring{$p$}{p}-Regularization}
\begin{theorem}[Existence of Weak IMCF]\label{thm:ExistenceWeakIMCF}
Let $(M, g, k)$ be a 3-dimensional AF initial data set with a MOTS $\Sigma$ (satisfying $\theta^+ = H + \tr_\Sigma k = 0$). There exists a weak solution $u$ to IMCF in the sense of Definition~\ref{def:WeakIMCF}.
\end{theorem}

\begin{proof}
\textbf{Step 1: $p$-regularized equation.}
For $p > 1$, consider the regularized problem:
\begin{equation}
    \Div\left( \frac{\nabla u_p}{|\nabla u_p|^{2-p}} \right) = |\nabla u_p|^{p-1}, \quad u_p|_\Sigma = 0, \quad u_p \to \infty \text{ at } \infty.
\end{equation}
This is a quasilinear elliptic equation with a unique weak solution $u_p \in W^{1,p}_{\mathrm{loc}}(M \setminus \Sigma)$ by standard theory (comparison principle and Perron's method).

\textbf{Step 2: A priori estimates.}
The key estimate is the bound on the $p$-energy:
\begin{equation}
    \int_{M \setminus \Sigma} |\nabla u_p|^p \, dV_g \le C(A(\Sigma), M_{\ADM}),
\end{equation}
independent of $p$. This follows from integrating the equation against $u_p$ and using the decay of $u_p$ at infinity (which is controlled by the ADM mass through Green's function estimates).

\textbf{Step 3: Compactness and limit.}
As $p \to 1^+$, the bound on $\int |\nabla u_p|^p$ implies (after passing to a subsequence):
\begin{itemize}
    \item $u_p \to u$ in $L^1_{\mathrm{loc}}$,
    \item $|\nabla u_p| \cdot \mathcal{L}^3 \rightharpoonup |Du|$ weakly as measures,
\end{itemize}
where $|Du|$ is the total variation measure of the BV function $u$.

\textbf{Step 4: Verification of weak formulation.}
The variational inequality~\eqref{eq:WeakIMCFVar} follows from testing the regularized equation against $\varphi \cdot \mathbf{1}_{\{u_p > t\}}$ and passing to the limit. The right-hand side converges to the integral of $\varphi / H$ by the definition of generalized mean curvature for sets of finite perimeter.
\end{proof}

\subsection{Hawking Mass Monotonicity under DEC}
The generalized Hawking mass for a surface $\Sigma$ in initial data $(M, g, k)$ is:
\begin{equation}
    m_H(\Sigma) := \sqrt{\frac{A(\Sigma)}{16\pi}} \left( 1 - \frac{1}{16\pi} \int_\Sigma \theta^+ \theta^- \, d\sigma \right),
\end{equation}
where $\theta^\pm = H \pm \tr_\Sigma k$ are the null expansions.

\begin{theorem}[Hawking Mass Monotonicity]\label{thm:HawkingMonotone}
Let $u$ be a weak solution to IMCF starting from a MOTS $\Sigma$ in an initial data set $(M, g, k)$ satisfying the DEC. Then for a.e. $0 < s < t$:
\begin{equation}
    m_H(\Sigma_t) \ge m_H(\Sigma_s).
\end{equation}
\end{theorem}

\begin{proof}
\textbf{Step 1: First variation of area.}
The area of the level set evolves as:
\begin{equation}
    \frac{d}{dt} A(\Sigma_t) = \int_{\Sigma_t} H \cdot \frac{1}{H} \, d\sigma = A(\Sigma_t).
\end{equation}
Thus $A(\Sigma_t) = A(\Sigma) e^t$.

\textbf{Step 2: Evolution of the null expansion integral.}
The key computation is the evolution of $\int_{\Sigma_t} \theta^+ \theta^- \, d\sigma$ under IMCF. Using the constraint equations and the Gauss--Codazzi relations, one derives:
\begin{multline}
    \frac{d}{dt} \int_{\Sigma_t} \theta^+ \theta^- \, d\sigma = \int_{\Sigma_t} \theta^+ \theta^- \, d\sigma \\
    + \int_{\Sigma_t} \frac{1}{H} \left[ 2(\mu - J(\nu)) + |\overset{\circ}{A}|^2 + (\theta^+ - \theta^-)^2 / 4 \right] d\sigma,
\end{multline}
where $\overset{\circ}{A}$ is the traceless second fundamental form.

\textbf{Step 3: DEC and positivity.}
Under the DEC, $\mu \ge |J|$, so $\mu - J(\nu) \ge 0$. The other terms are manifestly nonnegative. Therefore:
\begin{equation}
    \frac{d}{dt} \left( A(\Sigma_t) - \frac{1}{16\pi} \int_{\Sigma_t} \theta^+ \theta^- \, d\sigma \cdot A(\Sigma_t) \right) \ge 0.
\end{equation}

\textbf{Step 4: Monotonicity of $m_H$.}
Combining the area evolution and the null expansion evolution:
\begin{equation}
    \frac{d}{dt} m_H(\Sigma_t) = \frac{d}{dt} \sqrt{\frac{A(\Sigma_t)}{16\pi}} \left( 1 - \frac{\int \theta^+ \theta^-}{16\pi} \right) \ge 0.
\end{equation}
The inequality is strict unless the integrand vanishes, which occurs if and only if $\mu = |J|$ (saturating DEC), $\overset{\circ}{A} = 0$ (umbilical), and $\theta^+ = \theta^-$ (zero expansion in both null directions).
\end{proof}

\subsection{Limit at Infinity}
\begin{proposition}
For a weak IMCF in AF initial data, the Hawking mass converges to the ADM mass:
\begin{equation}
    \lim_{t \to \infty} m_H(\Sigma_t) = M_{\ADM}(g, k).
\end{equation}
\end{proposition}

\begin{proof}
At large $t$, the level set $\Sigma_t$ is approximately a large coordinate sphere $S_r$ with $r \sim e^{t/2}$. The mean curvature satisfies $H = 2/r + O(r^{-1-\tau})$, the area is $A = 4\pi r^2 + O(r^{2-\tau})$, and the null expansions satisfy $\theta^\pm = 2/r \pm \tr_\Sigma k = 2/r + O(r^{-1-\tau})$.

The Hawking mass expands as:
\begin{align}
    m_H(\Sigma_t) &= \sqrt{\frac{4\pi r^2}{16\pi}} \left( 1 - \frac{1}{16\pi} \cdot 4\pi r^2 \cdot \frac{4}{r^2} + O(r^{-\tau}) \right) \\
    &= \frac{r}{2} \left( 1 - 1 + O(r^{-\tau}) \right) + \text{mass correction} \\
    &= M_{\ADM} + O(r^{-\tau}).
\end{align}
The precise mass correction arises from the deviation of the metric from Euclidean, matching the ADM flux formula.
\end{proof}

