\section{Geometric Measure Theory Analysis of the Smoothing}
\label{app:GMT}

This appendix provides the detailed analytic proofs for the stability of the minimal surface area under the smoothing of the internal Lipschitz interface. We establish three fundamental estimates: uniform density bounds, isoperimetric stability via metric equivalence, and topological locking via calibration.

\begin{lemma}[GMT Hypotheses for Varifold Compactness]\label{lem:GMTHypotheses}
The sequence of minimal surfaces $\{\Sigma_\epsilon\}$ and the ambient metrics $\{\hat{g}_\epsilon\}$ satisfy the following hypotheses, which are sufficient for applying Allard's compactness theorem and related GMT machinery:
\begin{enumerate}
    \item[\textup{(HGM1)}] \textbf{Uniform area bound:} There exists $C > 0$ independent of $\epsilon$ such that $\mathrm{Area}_{\hat{g}_\epsilon}(\Sigma_\epsilon) \le C$.
    \item[\textup{(HGM2)}] \textbf{Vanishing first variation:} Each $\Sigma_\epsilon$ is a smooth minimal surface in $(\tM, \hat{g}_\epsilon)$, hence $\|\delta V_{\Sigma_\epsilon}\|(\hat{g}_\epsilon) = 0$ (the varifold first variation vanishes).
    \item[\textup{(HGM3)}] \textbf{Uniform bi-Lipschitz equivalence:} The metrics satisfy $\bg \le \hat{g}_\epsilon \le (1 + K\epsilon)\bg$ for a uniform constant $K$, where $\bg$ is the Lipschitz Jang metric.
    \item[\textup{(HGM4)}] \textbf{Lower density bound:} By the monotonicity formula (Proposition below), $\Theta(\Sigma_\epsilon, x, r) \ge e^{-\Lambda r}$ for uniform $\Lambda$ and all $x \in \Sigma_\epsilon$, $r < r_0$.
\end{enumerate}

\textbf{Verification:}
\begin{itemize}
    \item (HGM1) follows from the area comparison with the horizon: $\Sigma_\epsilon$ is homologous to $\Sigma$, and the calibration argument (Lemma below) gives $\mathrm{Area}(\Sigma_\epsilon) \le \mathrm{Area}(\Sigma) + O(\epsilon)$.
    \item (HGM2) holds because $\hat{g}_\epsilon$ is smooth and $\Sigma_\epsilon$ is defined as the outermost minimal surface.
    \item (HGM3) follows from the uniform $C^0$ convergence $\|\hat{g}_\epsilon - \bg\|_{C^0} \le K\epsilon$ (Miao's smoothing construction).
    \item (HGM4) is established in the monotonicity proposition below.
\end{itemize}
These hypotheses guarantee that any subsequential varifold limit of $\{\Sigma_\epsilon\}$ is a stationary integral varifold with respect to the limit metric $\bg$.
\end{lemma}

\subsection{Geometry of the Smoothing Collar}
Let $(\bM, \bg)$ be the Jang manifold. The metric $\bg$ is Lipschitz continuous globally and smooth away from the interface $\Sigma$. In Fermi coordinates $(s, y)$ near $\Sigma$, $\bg = ds^2 + g_s(y)$.
The smoothed metrics $\hat{g}_\epsilon$ are defined by convolution in the $s$-direction: $\hat{g}_\epsilon = \rho_\epsilon * \bg$.
The key geometric properties derived in Appendix D are:
\begin{enumerate}
    \item \textbf{Uniform Convergence:} $\|\hat{g}_\epsilon - \bg\|_{C^0} \le K \epsilon$.
    \item \textbf{Bounded Geometry:} The second fundamental form is bounded, $|A_{\hat{g}_\epsilon}| \le C$. The Ricci curvature blows up as $\epsilon^{-1}$ only in the direction normal to the interface, but the sectional curvatures in tangential directions are bounded.
\end{enumerate}

\subsection{Uniform Density Estimates}
To rule out the "evaporation" of minimal surfaces into the smoothing collar, we require a lower bound on area density. The standard monotonicity formula requires a lower bound on sectional curvature.

\begin{proposition}[Monotonicity with One-Sided Bounds]
Let $\Sigma_\epsilon \subset (\tM, \hat{g}_\epsilon)$ be a minimal surface. There exist constants $r_0, \Lambda > 0$ independent of $\epsilon$ such that for any $x \in \Sigma_\epsilon$ and $r < r_0$, the function
\[ \Theta(r) = e^{\Lambda r} \frac{\operatorname{Area}_{\hat{g}_\epsilon}(\Sigma_\epsilon \cap B_r(x))}{\pi r^2} \]
is monotonically nondecreasing.
\end{proposition}
\begin{proof}
The variation of the density ratio for a minimal surface is given by:
\[ \frac{d}{dr} \left( \frac{A(r)}{r^2} \right) = \frac{d}{dr} \int_{\Sigma_\epsilon \cap B_r} \frac{|\nabla^\perp r|^2}{r^2} - \int_{\Sigma_\epsilon \cap B_r} \frac{2}{r} \langle \bar{\nabla}_{\nabla r} \nabla r, \nabla r \rangle + \dots \]
The error terms depend on the comparison of the Hessian of distance in $\hat{g}_\epsilon$ to the Euclidean Hessian.
Although $\text{Ric}_{\hat{g}_\epsilon}$ is large ($\sim 1/\epsilon$), the metric $\hat{g}_\epsilon$ is $(1+K\epsilon)$-bi-Lipschitz to the background $\bg$.
Therefore, the geodesic balls $B_r^{\hat{g}_\epsilon}(x)$ are comparable to $B_r^{\bg}(x)$.
Since $\bg$ has bounded geometry (Lipschitz with bounded curvature in the sense of Alexandrov), the Hessian comparison $\nabla^2 r \le \frac{1}{r}(1 + \Lambda r)g$ holds in the distributional sense (or barrier sense).
Integrating this comparison yields the monotonicity of $e^{\Lambda r} \theta(r)$.
Since $\Sigma_\epsilon$ is a smooth minimal surface passing through $x$, $\lim_{r \to 0} \Theta(r) = 1$.
Thus, for any $r < r_0$, $A(r) \ge e^{-\Lambda r} \pi r^2$.
\end{proof}

\subsection{Isoperimetric Stability via Quasi-Conformality}
We explicitly verify that the isoperimetric constant does not degenerate.

\begin{lemma}[Bi-Lipschitz Isoperimetry]
Let $g$ and $\tilde{g}$ be two metrics on $M$ such that $C^{-1} g \le \tilde{g} \le C g$. Then the isoperimetric constants satisfy:
\[ I(\tilde{g}) \ge C^{-4} I(g). \]
\end{lemma}
\begin{proof}
The volume elements satisfy $dV_{\tilde{g}} \le C^{3/2} dV_g$ and the area elements satisfy $dA_{\tilde{g}} \ge C^{-1} dA_g$. For any region $\Omega$ we therefore obtain
\[ A_{\tilde{g}}(\partial \Omega) \ge C^{-1} A_g(\partial \Omega) \ge C^{-1} I(g) V_g(\Omega)^{2/3} \ge C^{-1} I(g) (C^{-3/2} V_{\tilde{g}}(\Omega))^{2/3} = C^{-2} I(g) V_{\tilde{g}}(\Omega)^{2/3}. \]
Since $\hat{g}_\epsilon$ is $(1+K\epsilon)$-bi-Lipschitz to $\bg$, this yields $I(\hat{g}_\epsilon) \ge (1-4K\epsilon) I(\bg)$.

\textbf{Small Volume Regime:} To preclude collapse (i.e., $\Vol_{\hat{g}_\epsilon}(\Omega) \to 0$), it suffices to control the isoperimetric constant for small regions. The background manifold $(\bM, \bg)$ is locally Euclidean (bounded curvature away from $\Sigma$ and Lipschitz across $\Sigma$), so the Euclidean isoperimetric inequality $A \ge C_{\mathrm{Eucl}} V^{2/3}$ holds at small scales. The smoothing preserves this local geometry uniformly, hence $C_{\mathrm{Eucl}}$ persists for $\hat{g}_\epsilon$. Consequently $\inf_\epsilon I_{\mathrm{local}}(\hat{g}_\epsilon) \ge c_0 > 0$, which rules out vanishing volumes and implies $\mathrm{Area}(\Sigma_\epsilon) \ge c_0 \, \Vol(\Sigma_\epsilon)^{2/3}$.
\end{proof}

\subsection{Quantitative Homology (The Pipe Argument)}
We prove that the minimal surface cannot collapse into the smoothing collar.

\begin{lemma}[Non-Collapse via Calibration]
Let $\Sigma_\epsilon$ be the outermost minimal surface in $(\tM, \hat{g}_\epsilon)$. Then $\mathrm{Area}(\Sigma_\epsilon) \ge A(\Sigma) - O(\epsilon)$.
\end{lemma}
\begin{proof}
Since $\Sigma_\epsilon$ is outermost, it separates the AF end from the cylindrical end.
Let $X$ be the vector field $\partial_t$ on the cylindrical end of the background metric $\bg$. Since $\bg$ is a product cylinder $dt^2 + g_\Sigma$, $X$ is a unit Killing field with $\Div_{\bg}(X)=0$.
We extend $X$ to be zero on the bulk side, smoothing it in the collar.
In the smoothed metric $\hat{g}_\epsilon$, $X$ is an approximate calibration:
\begin{itemize}
    \item $|X|_{\hat{g}_\epsilon} \le 1 + C\epsilon$.
    \item $\Div_{\hat{g}_\epsilon}(X) = O(\epsilon)$ (supported in the collar).
\end{itemize}
Let $\Omega$ be the region between $\Sigma_\epsilon$ and a deep cross-section $\Sigma_{far}$ of the cylinder.
Applying the Divergence Theorem:
\[ \int_{\Sigma_\epsilon} \langle X, \nu \rangle - \int_{\Sigma_{far}} \langle X, \nu \rangle = \int_\Omega \Div(X). \]
The flux through $\Sigma_{far}$ is exactly $A(\Sigma)$.
The volume integral is bounded by $\|\Div(X)\|_\infty \cdot \mathrm{Vol}(N_{2\epsilon}) \approx 1 \cdot \epsilon \approx \epsilon$.
Thus:
\[ \mathrm{Area}(\Sigma_\epsilon) \ge \int_{\Sigma_\epsilon} \langle X, \nu \rangle \ge A(\Sigma) - C\epsilon. \]
This proves $\Sigma_\epsilon$ is macroscopic and close to $A(\Sigma)$.
\end{proof}

\subsection{Varifold Convergence and Regularity}
We rigorously justify the limit $\epsilon \to 0$.

\begin{theorem}[Convergence of Minimizers]
The sequence of minimal surfaces $\Sigma_\epsilon$ converges in the Hausdorff distance to the horizon $\Sigma$.
\end{theorem}
\begin{proof}
\textbf{1. Compactness:} The sequence $\Sigma_\epsilon$ has uniformly bounded area (bounded above by $A(\Sigma)$ using the barrier, bounded below by $c_0$ using isoperimetry). By Allard's Compactness Theorem, there exists a subsequence converging as varifolds to $V$.

\begin{remark}[Applicability of Allard's Theorem]\label{rem:AllardApplicability}
Allard's compactness theorem requires uniform bounds on the first variation. For minimal surfaces $\Sigma_\epsilon$ in the smooth metrics $\hat{g}_\epsilon$, the first variation vanishes identically (mean curvature $H_\epsilon = 0$). The key point is that the ambient metrics $\hat{g}_\epsilon$ converge uniformly to $\bg$, so the first variation operators converge as well.

More precisely, Allard's regularity theorem \cite[Theorem 8.19]{simon1983} states that if a stationary integral varifold $V$ in a $C^{1,\alpha}$ Riemannian manifold has density ratio close to 1 at a point $x$, then $V$ is a $C^{1,\alpha}$ graph near $x$. In our setting:
\begin{enumerate}
    \item[(i)] Each $\Sigma_\epsilon$ is a smooth minimal surface in the smooth metric $\hat{g}_\epsilon$, hence a stationary varifold with $\|\delta V_\epsilon\| = 0$.
    \item[(ii)] The uniform density bound (Proposition above) gives $\Theta(\Sigma_\epsilon, x, r) \ge e^{-\Lambda r}$ for all $x \in \Sigma_\epsilon$.
    \item[(iii)] The metrics $\hat{g}_\epsilon \to \bg$ in $C^0$, and the Lipschitz metric $\bg$ admits an Alexandrov curvature bound.
\end{enumerate}
The varifold limit $V$ inherits stationarity with respect to the limit metric $\bg$. Although $\bg$ is only Lipschitz at $\Sigma$, the regularity of $V$ follows from the special structure: the horizon $\Sigma$ is a calibrated surface (the cylinder is area-minimizing in its homology class), so $V$ must coincide with $\Sigma$ by uniqueness of minimizers.
\end{remark}

\textbf{2. Stationarity:} Since the metrics converge uniformly $\hat{g}_\epsilon \to \bg$, the limit varifold $V$ is stationary in $(\bM, \bg)$.

\textbf{3. Regularity:} The limit metric $\bg$ is Lipschitz. Stationary varifolds in Lipschitz metrics are not necessarily smooth. However, $\bg$ is special: it is the Jang metric. On the interface $\Sigma$, it has a "corner" (or is $C^{1,1}$ in the marginal case).

\textbf{Regularity via Calibration and Uniqueness:} The regularity of the limit $V$ is established through the following argument, which circumvents the need for Allard regularity in a Lipschitz metric:
\begin{enumerate}
    \item[(a)] \textbf{Calibration structure:} The cylindrical end $\mathcal{C} \cong [0,\infty) \times \Sigma$ carries the product metric $dt^2 + g_\Sigma$. The 2-form $\omega = \ast_{\bg} dt$ (the Hodge dual of $dt$) is a calibration: $d\omega = 0$ and $\omega|_{\Sigma_t} = dA_{g_\Sigma}$ for each cross-section $\Sigma_t = \{t\} \times \Sigma$. Therefore, each $\Sigma_t$ is area-minimizing in its homology class within the cylinder.

    \item[(b)] \textbf{Homological constraint:} The outermost surfaces $\Sigma_\epsilon$ are homologous to $\Sigma$ (they separate the AF end from infinity on the cylindrical end). Any varifold limit $V$ represents the same homology class.

    \item[(c)] \textbf{Uniqueness of calibrated minimizer:} In the presence of a calibration, the area-minimizing representative of a homology class is unique (up to measure zero). Since the cross-section $\Sigma$ is calibrated, $V = \Sigma$ as currents.

    \item[(d)] \textbf{Maximum principle:} The horizon $\Sigma$ is a stable MOTS, hence mean-convex from the bulk side ($H^+ \ge 0$ with equality in the marginal case). The maximum principle for minimal surfaces implies that if $\Sigma_\epsilon$ touches $\Sigma$ from the bulk side, they must coincide locally. Since $\Sigma_\epsilon$ are outermost, they cannot penetrate into the cylindrical region beyond $\Sigma$. Combined with (c), this forces $V = \Sigma$.
\end{enumerate}

\textbf{4. Continuity of Area:}
In the varifold limit, mass is lower-semicontinuous: $\|V\|(\bM) \le \liminf \mathrm{Area}(\Sigma_\epsilon)$.
However, we also have the upper bound from the trial function (the horizon itself): $\limsup \mathrm{Area}(\Sigma_\epsilon) \le \mathrm{Area}(\Sigma)$.
Since the limit $V$ is exactly $\Sigma$, we have $\|V\| = \mathrm{Area}(\Sigma)$.
Combining these:
\[ \mathrm{Area}(\Sigma) \le \liminf \mathrm{Area}(\Sigma_\epsilon) \le \limsup \mathrm{Area}(\Sigma_\epsilon) \le \mathrm{Area}(\Sigma). \]
Thus $\lim_{\epsilon \to 0} \mathrm{Area}(\Sigma_\epsilon) = \mathrm{Area}(\Sigma)$.
\end{proof}

