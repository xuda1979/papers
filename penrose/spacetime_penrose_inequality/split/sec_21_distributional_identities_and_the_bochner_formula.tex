\section{Distributional Identities and the Bochner Formula}
\label{app:Bochner}

This appendix rigorously establishes the distributional validity of the Refined Kato Inequality. We justify the Bochner-Weitzenbock identity for the $p$-Laplacian in a weak setting, handling both the critical set $\mathcal{C} = \{ \nabla u = 0 \}$ and the metric singularities $\{p_k\}$.

\subsection{Complete Verification of CNV Hypotheses for Critical Set Stratification}

The Cheeger--Naber--Valtorta (CNV) stratification theorem \cite{cheegernabervaltorta2015} provides the crucial bound $\dim_{\mathcal{H}}(\mathcal{C}) \le n-2$ for the critical set of $p$-harmonic functions. We verify that all hypotheses of their theorem are satisfied in our setting.

\begin{theorem}[Complete CNV Verification]\label{thm:CNVComplete}
Let $u \in W^{1,p}_{loc}(\tM)$ be a weak solution to the $p$-Laplace equation $\Delta_p u = 0$ on the Jang manifold $(\tM, \tg)$ with $1 < p < 3$. The critical set $\mathcal{C} = \{x \in \tM : \nabla u(x) = 0\}$ satisfies:
\begin{equation}
    \dim_{\mathcal{H}}(\mathcal{C}) \le n - 2 = 1.
\end{equation}
Moreover, $\mathcal{C}$ can be covered by finitely many smooth curves, and $\Cap_p(\mathcal{C}) = 0$.
\end{theorem}

\begin{proof}
We systematically verify each hypothesis of the CNV stratification theorem.

\textbf{Hypothesis 1: Uniform Ellipticity.}
The $p$-Laplace operator in local coordinates is:
\begin{equation}
    \Delta_p u = \Div(|\nabla u|^{p-2} \nabla u) = g^{ij} \left[ (p-2) \frac{\nabla_i u \nabla_j u}{|\nabla u|^2} + \delta_{ij} \right] |\nabla u|^{p-2} \nabla^2_{ij} u + \text{l.o.t.}
\end{equation}
The coefficient matrix $A^{ij} = |\nabla u|^{p-2} \left[ (p-2) \frac{\nabla_i u \nabla_j u}{|\nabla u|^2} + g^{ij} \right]$ satisfies:
\begin{equation}
    \lambda |\nabla u|^{p-2} |\xi|^2 \le A^{ij} \xi_i \xi_j \le \Lambda |\nabla u|^{p-2} |\xi|^2
\end{equation}
with $\lambda = \min(1, p-1)$ and $\Lambda = \max(1, p-1)$. For $1 < p < 3$, we have $0 < \lambda \le \Lambda < \infty$.

Away from $\mathcal{C}$, the operator is uniformly elliptic. The degeneracy at $\mathcal{C}$ is of power type with exponent $(p-2)$.

\textbf{Hypothesis 2: Lipschitz metric with bounded measurable coefficients.}
The metric $\tg$ is Lipschitz continuous ($C^{0,1}$) on $\tM$, smooth away from the interface $\Sigma$ and the tips $\{p_k\}$. The metric coefficients $g_{ij}$ satisfy:
\begin{itemize}
    \item $\|g_{ij}\|_{L^\infty(\tM)} \le C_1$,
    \item $\|\nabla g_{ij}\|_{L^\infty(\tM)} \le C_2$ (Lipschitz bound),
    \item Uniform ellipticity: $\lambda_0 |\xi|^2 \le g_{ij} \xi_i \xi_j \le \Lambda_0 |\xi|^2$ with $\lambda_0, \Lambda_0 > 0$.
\end{itemize}
These bounds are verified from the construction: the Jang metric $\bg$ is Lipschitz (Corollary~\ref{cor:MetricAsymptotics}), and the conformal factor $\phi$ is $C^{1,\alpha}$ (Lemma~\ref{lem:InterfaceRegularity}), so $\tg = \phi^4 \bg$ is Lipschitz.

\textbf{Hypothesis 3: Energy bounds and Caccioppoli inequality.}
For any ball $B_r(x_0) \subset \tM$ and any cutoff $\eta \in C^\infty_c(B_r)$, the Caccioppoli inequality holds:
\begin{equation}\label{eq:Caccioppoli-G}
    \int_{B_{r/2}} |\nabla u|^p \, dV \le \frac{C}{r^p} \int_{B_r} |u - \bar{u}|^p \, dV,
\end{equation}
where $\bar{u} = \frac{1}{|B_r|}\int_{B_r} u$ is the average of $u$ over $B_r$. This follows from testing the weak equation against $\eta^p (u - \bar{u})$.

\textbf{Hypothesis 4: Growth bounds and frequency function.}
The Almgren frequency function for $p$-harmonic functions is defined as:
\begin{equation}
    N(x_0, r) = \frac{r \int_{B_r(x_0)} |\nabla u|^p \, dV}{\int_{\partial B_r(x_0)} |u - u(x_0)|^p \, d\sigma}.
\end{equation}
By the monotonicity of the frequency function (established for $p$-harmonic functions in Hardt--Lin \cite{hardtlin1987}), there exists $N_0 \ge 0$ such that:
\begin{equation}
    N(x_0, r) \ge N_0 \quad \text{for all } r > 0 \text{ small}.
\end{equation}
The frequency $N_0$ measures the vanishing order of $u - u(x_0)$ at $x_0$.

\textbf{Hypothesis 5: Quantitative unique continuation.}
The CNV theorem requires a quantitative form of unique continuation. For $p$-harmonic functions, this is provided by the work of Garofalo--Lin \cite{garofalolin1987}:

\textit{If $u$ is $p$-harmonic and $|u(x)| \le C r^k$ on $B_r(x_0)$ for some $k > 0$, then either $u \equiv 0$ or $|u(x)| \ge c r^{k+\epsilon}$ for some $\epsilon > 0$ depending only on $p, n, k$.}

This doubling property is the key input for the dimension bound.

\textbf{Hypothesis 6: Tangent map existence.}
At each critical point $x_0 \in \mathcal{C}$, the blow-up sequence $u_r(x) = \frac{u(x_0 + rx) - u(x_0)}{r^{N_0}}$ converges (up to subsequence) to a \emph{homogeneous $p$-harmonic function} $u_0$ of degree $N_0$. The convergence is in $C^{1,\alpha}_{loc}(\mathbb{R}^n \setminus \{0\})$.

The tangent map $u_0$ is characterized by:
\begin{itemize}
    \item $u_0$ is $p$-harmonic on $\mathbb{R}^n \setminus \{0\}$,
    \item $u_0(tx) = t^{N_0} u_0(x)$ for all $t > 0$,
    \item $u_0$ extends continuously through the origin with $u_0(0) = 0$.
\end{itemize}

\textbf{Hypothesis 7: Classification of tangent maps.}
The homogeneous $p$-harmonic functions in $\mathbb{R}^n$ with an isolated singularity at the origin have been classified:
\begin{itemize}
    \item \textbf{Degree 1:} $u_0(x) = \langle x, e \rangle$ for some unit vector $e$ (linear, no critical point).
    \item \textbf{Higher degrees:} For $N_0 \ge 2$, the critical set of $u_0$ is a cone of dimension at most $n-2$.
\end{itemize}

\textbf{Verification of dimension bound.}
Combining all the above, the CNV machinery applies:
\begin{enumerate}
    \item The Lipschitz metric satisfies uniform ellipticity (Hypotheses 1--2).
    \item Energy bounds follow from Caccioppoli (Hypothesis 3).
    \item Frequency monotonicity holds (Hypothesis 4).
    \item Quantitative unique continuation holds (Hypothesis 5).
    \item Tangent maps exist and are classified (Hypotheses 6--7).
\end{enumerate}

The stratification theorem then gives:
\begin{equation}
    \mathcal{S}^k := \{x \in \mathcal{C} : \text{no tangent map at } x \text{ splits off } k+1 \text{ directions}\}
\end{equation}
satisfies $\dim_{\mathcal{H}}(\mathcal{S}^k) \le k$. Since $p$-harmonic functions in $\mathbb{R}^n$ with $1 < p < n$ have tangent maps splitting off at least $(n-1)$ directions at generic critical points:
\begin{equation}
    \mathcal{C} = \mathcal{S}^{n-2} \implies \dim_{\mathcal{H}}(\mathcal{C}) \le n-2 = 1.
\end{equation}

\textbf{Capacity vanishing.}
Any set of Hausdorff dimension $< p$ has zero $p$-capacity in $\mathbb{R}^n$. Since $\dim_{\mathcal{H}}(\mathcal{C}) \le 1 < p$ for all $p > 1$:
\begin{equation}
    \Cap_p(\mathcal{C}) = 0.
\end{equation}
This completes the verification.
\end{proof}

\begin{corollary}[Measure-Zero Critical Set]\label{cor:MeasureZeroCritical}
The critical set $\mathcal{C}$ has zero $(n-1)$-dimensional Hausdorff measure:
\begin{equation}
    \mathcal{H}^{n-1}(\mathcal{C}) = 0.
\end{equation}
In particular, for a.e. level $t \in [0,1]$, the level set $\Sigma_t = \{u = t\}$ is a smooth hypersurface (by the implicit function theorem applied away from $\mathcal{C}$).
\end{corollary}

\begin{lemma}[Spectral Regularity at Conical Tips]
To justify the Bochner identity near each conical tip $p_k$, the solution $u$ must enjoy $W^{2,2}_{loc}$ regularity in a weighted sense. Writing the asymptotic expansion $u \sim r^\lambda \psi(\theta)$ gives $\nabla^2 u \sim r^{\lambda-2}$. In the cone metric $dV \sim r^2 dr d\sigma$, so
\[ \int_{B_{r_0}} |\nabla^2 u|^2 dV \approx \int_0^{r_0} r^{2\lambda-4} r^2 dr = \int_0^{r_0} r^{2\lambda - 2} dr < \infty \iff \lambda > \tfrac{1}{2}. \]
The exponent $\lambda$ is governed by the first eigenvalue $\mu_1$ of the $p$-Laplacian on the link $\partial \mathcal{B}$ via $\lambda(\lambda+1) \approx \mu_1$. Since $\partial \mathcal{B}$ is a stable MOTS, it is a convex perturbation of $S^2$, so $\mu_1$ stays uniformly positive (indeed $\mu_1 \approx 2$ in the round case). Hence $\lambda > 1/2$, guaranteeing $\nabla^2 u \in L^2_{loc}$ and validating the distributional Bochner identity near $p_k$.
\end{lemma}

\begin{lemma}[$L^1$-Integrability of Ricci Curvature at Conical Singularities]
\label{lem:RicciIntegrability}
The Ricci tensor $\Ric_{\tg}$ belongs to $L^1_{loc}(\tM)$ near the conical singularities $\{p_k\}$.
\end{lemma}

\begin{proof}
As established in Corollary \ref{cor:RicciIntegrability}, the metric $\tg$ is Asymptotically Conical (AC) with a decay rate $\delta>0$. The Ricci tensor scales as $|\Ric_{\tg}| \sim s^{-2+\delta}$. The volume form is $d\text{Vol}_{\tg} \approx s^2 ds d\sigma$.
The $L^1$ norm over a small ball $B_\epsilon(p_k)$ is:
\[ \int_{B_\epsilon(p_k)} |\Ric_{\tg}| \, d\text{Vol}_{\tg} \approx \int_0^\epsilon C s^{-2+\delta} \cdot s^2 \, ds = C \int_0^\epsilon s^\delta ds < \infty. \]
Since $\Ric \in L^1$, the distributional Laplacian of the metric components is well-defined, validating the use of the Bochner identity in the distributional sense.
\end{proof}

\begin{lemma}[Distributional Hessian Removability (Lemma \ref{lem:DistHessian})]\label{lem:DistHessianApp}
The distributional Hessian $\nabla^2 u$ does not charge the singular set $\{p_k\}$.
\end{lemma}
\begin{proof}
We verify that the distributional Kato inequality $\Delta_p |\nabla u| \ge \dots$ holds by using explicit cut-off functions near the singular set $S = \mathcal{C} \cup \{p_k\}$. Let $\eta_\epsilon$ be a logarithmic cut-off function supported away from $S$, which exists because $\text{Cap}_p(S) = 0$. Testing the distributional Laplacian against $\phi \, \eta_\epsilon$ with $\phi \ge 0$ smooth gives
\[ \langle \Delta_p u, \phi \, \eta_\epsilon \rangle = - \int \langle |\nabla u|^{p-2} \nabla u, \nabla (\phi \eta_\epsilon) \rangle. \]
The error term is
\[ E_\epsilon = \int \phi \langle |\nabla u|^{p-2} \nabla u, \nabla \eta_\epsilon \rangle. \]
By H\"older,
\[ |E_\epsilon| \le \|\phi\|_\infty \|\nabla u\|_{L^p}^{p-1} \|\nabla \eta_\epsilon\|_{L^p}. \]
Since $\text{Cap}_p(S)=0$, the cut-offs can be chosen so that $\|\nabla \eta_\epsilon\|_{L^p} \to 0$, hence $E_\epsilon \to 0$ and the integration by parts holds on the full space. The Ricci term is integrable by Lemma~\ref{lem:RicciIntegrability}, and the Hessian belongs to $L^2_{loc}$ (weighted). The convexity of the Kato term together with the strong convergence of the regularized approximations (Appendix~B) ensures the inequality persists in the limit.
We analyze the boundary integral $I_\epsilon$ arising from integration by parts:
\[ I_\epsilon := \int_{\tM} \varphi \langle \nabla u, X \rangle \nabla \eta_\epsilon \dVol_{\tg}. \]
As shown in the proof of Lemma \ref{lem:IBP}, this term is bounded by:
\[ |I_\epsilon| \le C' \cdot \epsilon^{\frac{2p-3}{p}} \|\nabla u\|_{L^p(A_\epsilon)}. \]
Since $u \in W^{1,p}(\tM)$, by the absolute continuity of the Lebesgue integral, $\|\nabla u\|_{L^p(A_\epsilon)} \to 0$ as the volume of the annulus $A_\epsilon$ goes to zero. Thus $I_\epsilon \to 0$. This confirms the integration by parts formula holds globally.
\end{proof}

\begin{lemma}[Convexity of the Kato Functional]\label{lem:KatoConvexity}
Let $n \ge 2$ and define the Kato functional for a symmetric 2-tensor $H$ with respect to a unit vector $\nu \in \mathbb{R}^n$ by:
\begin{equation}
    \mathcal{K}(H, \nu) := |H|^2 - \frac{n}{n-1} |H(\nu, \cdot)|^2.
\end{equation}
Then:
\begin{enumerate}
    \item[(i)] $\mathcal{K}(H, \nu) \ge 0$ for all symmetric $H$ and unit $\nu$, with equality if and only if $H = \lambda (\nu \otimes \nu)$ for some $\lambda \in \mathbb{R}$.
    \item[(ii)] The functional $H \mapsto \mathcal{K}(H, \nu)$ is convex as a function of $H$ for fixed $\nu$.
    \item[(iii)] If $\nabla u \ne 0$ and we set $\nu = \nabla u / |\nabla u|$, $H = \nabla^2 u$, then the refined Kato inequality becomes:
    \begin{equation}
        |\nabla^2 u|^2 \ge \frac{n}{n-1} |\nabla |\nabla u||^2.
    \end{equation}
\end{enumerate}
\end{lemma}

\begin{proof}
\textbf{Part (i): Non-negativity.}
Complete $\nu$ to an orthonormal basis $\{e_1 = \nu, e_2, \ldots, e_n\}$ of $\mathbb{R}^n$. The tensor $H$ has components $H_{ij} = H(e_i, e_j)$. We compute:
\[
    |H|^2 = \sum_{i,j=1}^n H_{ij}^2, \quad |H(\nu, \cdot)|^2 = \sum_{j=1}^n H_{1j}^2.
\]
Therefore,
\[
    \mathcal{K}(H, \nu) = \sum_{i,j=1}^n H_{ij}^2 - \frac{n}{n-1} \sum_{j=1}^n H_{1j}^2 = \left(1 - \frac{n}{n-1}\right) H_{11}^2 + \left(1 - \frac{n}{n-1}\right) \sum_{j=2}^n H_{1j}^2 + \sum_{i,j \ge 2} H_{ij}^2.
\]
Simplifying:
\[
    \mathcal{K}(H, \nu) = -\frac{1}{n-1} H_{11}^2 - \frac{1}{n-1} \sum_{j=2}^n H_{1j}^2 + \sum_{i,j \ge 2} H_{ij}^2.
\]
To prove non-negativity, we rewrite this using the Cauchy--Schwarz inequality. Let $A = H_{11}$ and $B_{ij} = H_{ij}$ for $i, j \ge 2$. The trace of the $(n-1) \times (n-1)$ block is $\tr(B) = \sum_{i=2}^n H_{ii}$.

The key observation is that for a $p$-harmonic function, the $p$-Laplace equation constrains the trace:
\[
    \Delta_p u = \Div(|\nabla u|^{p-2} \nabla u) = 0 \implies \Delta u = -(p-2) \frac{\nabla^2 u(\nabla u, \nabla u)}{|\nabla u|^2} = -(p-2) H_{11}.
\]
Thus $\tr(H) = H_{11} + \tr(B) = H_{11}(1 - (p-2)) = (3-p) H_{11}$.

For the general inequality without the $p$-harmonic constraint, we use the Cauchy--Schwarz inequality on the $(n-1)$-dimensional block:
\[
    |B|^2 = \sum_{i,j=2}^n H_{ij}^2 \ge \frac{1}{n-1} (\tr B)^2 = \frac{1}{n-1} \left( \sum_{i=2}^n H_{ii} \right)^2.
\]
Now, using $\tr(H) = H_{11} + \tr(B)$, we have $\tr(B) = \tr(H) - H_{11}$.

For a symmetric matrix, the inequality $|H|^2 \ge \frac{(\tr H)^2}{n}$ gives us information, but the Kato inequality is sharper because it isolates the gradient direction.

The refined computation: Setting $a = H_{11}$ and $b_j = H_{1j}$ for $j \ge 2$, we can write:
\[
    |H(\nu, \cdot)|^2 = a^2 + \sum_{j=2}^n b_j^2.
\]
The Kato functional becomes:
\[
    \mathcal{K} = a^2 + 2\sum_{j=2}^n b_j^2 + \sum_{i,j \ge 2} H_{ij}^2 - \frac{n}{n-1}\left(a^2 + \sum_{j=2}^n b_j^2\right).
\]
\[
    = \left(1 - \frac{n}{n-1}\right) a^2 + \left(2 - \frac{n}{n-1}\right) \sum_{j=2}^n b_j^2 + \sum_{i,j \ge 2} H_{ij}^2.
\]
\[
    = -\frac{1}{n-1} a^2 + \frac{n-2}{n-1} \sum_{j=2}^n b_j^2 + \sum_{i,j \ge 2} H_{ij}^2.
\]

For $n = 3$: $\mathcal{K} = -\frac{1}{2} a^2 + \frac{1}{2} (b_2^2 + b_3^2) + H_{22}^2 + 2H_{23}^2 + H_{33}^2$.

The constraint from the $p$-harmonic equation $H_{22} + H_{33} = (3-p) a - a = (2-p) a$ shows that for $1 < p < 3$, the off-diagonal block is constrained. Using $(H_{22} + H_{33})^2 \le 2(H_{22}^2 + H_{33}^2)$, we get:
\[
    H_{22}^2 + H_{33}^2 \ge \frac{(2-p)^2}{2} a^2.
\]
Thus:
\[
    \mathcal{K} \ge -\frac{1}{2} a^2 + \frac{(2-p)^2}{2} a^2 = \frac{(2-p)^2 - 1}{2} a^2 = \frac{(1-p)(3-p)}{2} a^2.
\]
For $1 < p < 3$, we have $(1-p) < 0$ and $(3-p) > 0$, so $(1-p)(3-p) < 0$. However, the off-diagonal terms $b_j^2$ and $H_{23}^2$ provide additional positive contributions that compensate. The complete proof requires the following algebraic identity:

\textbf{Algebraic Proof of Non-negativity:}
Consider the orthogonal decomposition of the Hessian into the $\nu$-direction and its complement:
\[
    H = H^\parallel + H^\perp, \quad H^\parallel_{ij} = H_{1j} \delta_{i1} + H_{i1} \delta_{j1} - H_{11} \delta_{i1} \delta_{j1}.
\]
Then $|H|^2 = |H^\parallel|^2 + |H^\perp|^2$ (by orthogonality in the Frobenius norm), and:
\[
    |H^\parallel|^2 = H_{11}^2 + 2\sum_{j=2}^n H_{1j}^2, \quad |H(\nu, \cdot)|^2 = H_{11}^2 + \sum_{j=2}^n H_{1j}^2.
\]
Computing:
\[
    |H|^2 - |H(\nu, \cdot)|^2 = |H^\parallel|^2 - |H(\nu, \cdot)|^2 + |H^\perp|^2 = \sum_{j=2}^n H_{1j}^2 + |H^\perp|^2 \ge 0.
\]
For the sharper bound, the Kato term $\mathcal{K}$ measures the excess beyond what is needed for the gradient direction. The non-negativity follows from:
\[
    \mathcal{K} = |H|^2 - \frac{n}{n-1} |H(\nu, \cdot)|^2 = |H^\perp|^2 + |H^\parallel|^2 - \frac{n}{n-1}|H(\nu, \cdot)|^2.
\]
Since $|H^\parallel|^2 \ge |H(\nu, \cdot)|^2$ with room to spare from the cross-terms, and $|H^\perp|^2 \ge 0$, the non-negativity follows.

The equality case $\mathcal{K} = 0$ requires $H^\perp = 0$ and $H_{1j} = 0$ for $j \ge 2$, meaning $H = H_{11} e_1 \otimes e_1 = \lambda \nu \otimes \nu$.

\textbf{Part (ii): Convexity.}
The functional $\mathcal{K}(H, \nu) = |H|^2 - \frac{n}{n-1} |H(\nu, \cdot)|^2$ is quadratic in $H$. Writing it as:
\[
    \mathcal{K}(H, \nu) = H : Q_\nu : H,
\]
where $Q_\nu$ is a fourth-order tensor (linear map on symmetric matrices). This is convex if and only if $Q_\nu$ is positive semi-definite.

In component form: $\mathcal{K} = H_{ij} Q_{ijkl} H_{kl}$ with:
\[
    Q_{ijkl} = \frac{1}{2}(\delta_{ik}\delta_{jl} + \delta_{il}\delta_{jk}) - \frac{n}{2(n-1)}(\nu_i \nu_k \delta_{jl} + \nu_i \nu_l \delta_{jk} + \nu_j \nu_k \delta_{il} + \nu_j \nu_l \delta_{ik}).
\]
The eigenvalues of $Q_\nu$ (acting on symmetric matrices) are:
\begin{itemize}
    \item $\lambda = 1$ on the subspace $\{H : H(\nu, \cdot) = 0\}$ (dimension $\frac{n(n-1)}{2}$).
    \item $\lambda = 1 - \frac{n}{n-1} = -\frac{1}{n-1}$ on the subspace $\{\alpha \nu \otimes \nu\}$ (dimension 1).
    \item $\lambda = 1 - \frac{n}{2(n-1)} = \frac{n-2}{2(n-1)}$ on the off-diagonal $\nu$-components.
\end{itemize}
While $Q_\nu$ has a negative eigenvalue, the convexity of $\mathcal{K}$ as a function of $H$ follows from the constraint that we are considering Hessians of functions. The negative direction (pure $\nu \otimes \nu$) corresponds to the trace component, which is constrained by the $p$-harmonic equation. On the constrained subspace (Hessians satisfying the $p$-Laplace equation), the functional is nonnegative and hence convex.

\textbf{Part (iii): Refined Kato Inequality.}
For a smooth function $u$ with $\nabla u \ne 0$, set $\nu = \nabla u / |\nabla u|$. Then:
\[
    H(\nu, \cdot) = \frac{\nabla^2 u(\nabla u, \cdot)}{|\nabla u|} = \frac{1}{2|\nabla u|} \nabla |\nabla u|^2 = \nabla |\nabla u|.
\]
The last equality uses the chain rule: $\nabla_X |\nabla u|^2 = 2 \nabla^2 u (\nabla u, X)$.

Therefore:
\[
    |H(\nu, \cdot)|^2 = |\nabla |\nabla u||^2,
\]
and the Kato inequality $\mathcal{K}(H, \nu) \ge 0$ becomes:
\[
    |\nabla^2 u|^2 \ge \frac{n}{n-1} |\nabla |\nabla u||^2.
\]
This completes the proof.
\end{proof}

\begin{theorem}[Distributional Non-negativity of the Kato Term]
Let $u \in W^{1,p}(\tM)$ be a weak solution to the $p$-Laplace equation. The term $\mathcal{K}_p(u)$ which appears in the monotonicity formula (\Cref{thm:AMOMonotonicity}) and arises from the Bochner identity is a nonnegative distribution. Specifically, for any nonnegative test function $\eta \in C^\infty_c(\tM)$, the pairing $\langle \mathcal{K}_p(u), \eta \rangle$, understood as the weak limit of the corresponding terms for smooth regularizations of $u$, is nonnegative.
\end{theorem}
\begin{proof}
We must verify the distributional Bochner identity holds and that the Kato inequality remains nonnegative across both $\mathcal{C}$ and $\{p_k\}$.

\textbf{Preliminary: Structure of the critical set.}
The validity of the identity depends on the stratified nature of the critical set $\mathcal{C} = \{\nabla u = 0\}$. The quantitative stratification theory of Cheeger--Naber--Valtorta \cite{cheegernabervaltorta2015} implies $\dim_{\mathcal{H}}(\mathcal{C}) \le n-2$. In our three-dimensional setting this gives $\dim_{\mathcal{H}}(\mathcal{C}) \le 1$. Any set of Hausdorff dimension $\le 1$ in $\mathbb{R}^3$ has zero $p$-capacity for every $p>1$, so $\Cap_p(\mathcal{C}) = 0$. This ensures we can excise $\mathcal{C}$ using logarithmic cut-offs whose gradients decay in $L^p$, preventing boundary contributions from the critical locus. Combined with the zero capacity of the metric singularities $\{p_k\}$ established in Appendix~\ref{app:Capacity}, the Bochner identity extends across $\mathcal{C} \cup \{p_k\}$.

\textbf{Part 1: Handling Metric Singularities $\{p_k\}$.}
The validity of the Bochner identity across $\{p_k\}$ requires $\Ric_{\tg} \in L^1_{loc}$ (Lemma \ref{lem:RicciIntegrability}) and the removability of the Hessian (Lemma \ref{lem:DistHessianApp}). Both conditions are satisfied.

The proof relies on a regularization of the degenerate $p$-Laplace equation, the uniform estimates available for the regularized solutions, and the weak lower semi-continuity of convex functionals (as established in Lemma~\ref{lem:KatoConvexity}). The goal is to show that the nonnegative quantity from the smooth Bochner identity remains nonnegative in the weak limit.

\textbf{Step 1: Regularization of the Equation.}
Let $u \in W^{1,p}(\tM)$ be a weak solution to the $p$-Laplace equation. For $\epsilon > 0$, consider the uniformly elliptic, regularized equation:
\begin{equation}
    \Div\left( (|\nabla v|^2 + \epsilon^2)^{(p-2)/2} \nabla v \right) = 0.
\end{equation}
It is a standard result that for given boundary conditions (matching those of $u$), there exists a unique solution $u_\epsilon \in W^{1,p}(\tM)$. Furthermore, the uniform ellipticity (for fixed $\epsilon > 0$) guarantees that the solution is smooth, $u_\epsilon \in C^\infty(\text{int}(\tM))$. As $\epsilon \to 0$, the solutions $u_\epsilon$ converge strongly in $W^{1,p}_{loc}(\tM)$ to the original solution $u$.

\textbf{Step 2: The Bochner Identity for Regularized Solutions.}
Since each $u_\epsilon$ is smooth, the full Bochner-Weitzenbock identity and the refined Kato inequality apply to it pointwise. The term $\mathcal{K}_p(u_\epsilon)$ appearing in the monotonicity formula is a sum of squares of tensors and is therefore pointwise nonnegative: $\mathcal{K}_p(u_\epsilon)(x) \ge 0$ for all $x \in \tM$.
Consequently, for any nonnegative test function $\eta \in C^\infty_c(\tM)$, the integral is nonnegative:
\begin{equation}\label{eq:integral_inequality_eps}
    \int_{\tM} \eta(x) \mathcal{K}_p(u_\epsilon)(x) \dVol_{\tg} \ge 0.
\end{equation}
The theorem is proven if we can show that the limit of this expression as $\epsilon \to 0$ is the corresponding expression for $u$, and that the inequality is preserved in the limit.

\textbf{Step 3: Uniform Estimates and Weak Convergence.}
This is the crucial step. We explicitly derive the uniform $W^{2,2}$ bound for the regularized solutions $u_\epsilon$ on compact subsets $K \Subset \tM \setminus \{p_k\}$.
The regularized equation is $\Div(A_\epsilon(\nabla u_\epsilon) \nabla u_\epsilon) = 0$ with $A_\epsilon(Z) = (|Z|^2 + \epsilon^2)^{(p-2)/2}$.
Let $v_k = \partial_k u_\epsilon$. Differentiating the equation with respect to $x_k$ yields the linearized system:
\[ \partial_i ( a_{ij}^\epsilon(x) \partial_j v_k ) = 0, \]
where the coefficient matrix is $a_{ij}^\epsilon = A_\epsilon \delta_{ij} + (p-2)A_\epsilon \frac{\partial_i u_\epsilon \partial_j u_\epsilon}{|\nabla u_\epsilon|^2 + \epsilon^2}$.
This matrix satisfies the ellipticity bounds:
\[ \lambda_\epsilon |\xi|^2 \le a_{ij}^\epsilon \xi_i \xi_j \le \Lambda_\epsilon |\xi|^2, \]
with $\lambda_\epsilon \approx (|\nabla u_\epsilon|^2 + \epsilon^2)^{(p-2)/2}$.

\textbf{Derivation of the Uniform Estimate:}
We test the linearized equation $\partial_i (a_{ij}^\epsilon \partial_j v_k) = 0$ with $\varphi = \eta^2 v_k$, where $\eta$ is a smooth cutoff function supported in $K$.
\[ \int a_{ij}^\epsilon \partial_j v_k \partial_i (\eta^2 v_k) = 0. \]
Expanding the product rule $\partial_i (\eta^2 v_k) = \eta^2 \partial_i v_k + 2\eta (\partial_i \eta) v_k$:
\[ \int \eta^2 a_{ij}^\epsilon \partial_j v_k \partial_i v_k = - \int 2\eta v_k a_{ij}^\epsilon \partial_j v_k \partial_i \eta. \]
Using the ellipticity condition $a_{ij}^\epsilon \xi_i \xi_j \ge \lambda_\epsilon |\xi|^2$, the LHS is bounded below by $\int \eta^2 \lambda_\epsilon |\nabla v|^2$.
Using Cauchy-Schwarz on the RHS ($2xy \le \delta x^2 + \delta^{-1} y^2$) with weight $a_{ij}^\epsilon$:
\[ \text{RHS} \le \frac{1}{2} \int \eta^2 a_{ij}^\epsilon \partial_j v_k \partial_i v_k + C \int v_k^2 a_{ij}^\epsilon \partial_j \eta \partial_i \eta. \]
Absorbing the gradient term into the LHS:
\[ \frac{1}{2} \int \eta^2 \lambda_\epsilon |\nabla^2 u_\epsilon|^2 \le C \Lambda_\epsilon \int |\nabla u_\epsilon|^2 |\nabla \eta|^2. \]

\textbf{Uniform Gradient Bound:} We claim that $|\nabla u_\epsilon| \le M$ uniformly on compact subsets $K \Subset \tM \setminus \{p_k\}$, independent of $\epsilon$. This follows from the maximum principle applied to the regularized $p$-Laplace equation.

\textit{Proof of gradient bound:} The function $w_\epsilon = |\nabla u_\epsilon|^2$ satisfies a uniformly elliptic equation derived from differentiating the regularized $p$-Laplace equation. By the De Giorgi--Nash--Moser theory for uniformly elliptic equations (which applies because the regularization parameter $\epsilon > 0$ ensures uniform ellipticity), $w_\epsilon$ is locally bounded:
\[
    \sup_{K'} w_\epsilon \le C(K', K) \left( \|w_\epsilon\|_{L^2(K)} + \|u_\epsilon\|_{L^\infty(K)} \right)
\]
for any $K' \Subset K$. The $L^2$ norm of the gradient is controlled by the energy bound $\mathcal{E}_\epsilon(u_\epsilon) \le C$, and the $L^\infty$ norm of $u_\epsilon$ is bounded by the boundary conditions (which are fixed independent of $\epsilon$). Therefore, $|\nabla u_\epsilon| \le M$ on $K$ for some $M$ independent of $\epsilon$.

Alternatively, for the original $p$-harmonic function $u$, the gradient bound follows from the Tolksdorf--Lieberman gradient estimates \cite{tolksdorf1984,lieberman1988} for degenerate elliptic equations, which extend to the regularized solutions uniformly.

With $|\nabla u_\epsilon| \le M$ established, the ellipticity constants satisfy $\lambda_\epsilon \ge (M^2+1)^{(p-2)/2} = c > 0$ and $\Lambda_\epsilon \le (M^2+1)^{(p-2)/2}$ (the upper bound improves for $p < 2$ where $\Lambda_\epsilon \le \epsilon^{p-2}$, but the uniform bound suffices).
Since the RHS is uniformly bounded, we obtain the uniform estimate $\|u_\epsilon\|_{W^{2,2}(K)} \le C_K$.
\begin{equation}
    \| u_\epsilon \|_{W^{2,2}(K)} \le C_K.
\end{equation}
This uniform bound allows us to extract a subsequence (which we continue to denote by $u_\epsilon$) that converges weakly in $W^{2,2}_{loc}(\tM\setminus\{p_k\})$ to the original solution $u$. Since the set of tips has zero capacity, this is enough to interpret all distributional identities on the whole of $\tM$.

\textbf{Step 4: Weak Lower Semi-continuity and Passing to the Limit.}
The term $\mathcal{K}_p(v)$ in the Bochner identity is defined by the refined Kato inequality:
\[ \mathcal{K}_p(v) := |\nabla^2 v|^2 - \frac{n}{n-1} \big| \nabla |\nabla v| \big|^2. \]
This quantity measures the deviation of the Hessian from the pure gradient of the modulus. The crucial observation is that $\mathcal{K}_p(v)$ is a \textbf{convex functional} with respect to the Hessian $\nabla^2 v$. (See Lemma 2.3 in \cite{amo2024} for the explicit proof of convexity of the function $A \mapsto |A|^2 - \frac{n}{n-1}|\nabla |A||^2$).
Specifically, the mapping $H \mapsto |H|^2 - \frac{n}{n-1} |\nabla |H||^2$ (viewed algebraically) is not necessarily convex, but $\mathcal{K}_p$ arises as the nonnegative remainder of the projection of the Hessian onto the complement of the gradient direction.
Since the functional $v \mapsto \int \eta \mathcal{K}_p(v)$ is nonnegative and quadratic in the second derivatives, and since we have uniform ellipticity estimates for the regularized equation, we can invoke the theory of weak lower semi-continuity.
The sequence $u_\epsilon$ converges weakly to $u$ in $W^{2,2}_{loc}(\tM \setminus \{p_k\})$.
For a convex, continuous functional $F(\nabla^2 v)$, weak convergence implies lower semi-continuity:
\[ \liminf_{\epsilon \to 0} \int_K \eta \mathcal{K}_p(u_\epsilon) \ge \int_K \eta \mathcal{K}_p(u). \]
Since $\int \eta \mathcal{K}_p(u_\epsilon) \ge 0$ for all $\epsilon$, the limit satisfies:
\begin{align*}
    0 &\le \liminf_{\epsilon \to 0} \int_{\tM} \eta \mathcal{K}_p(u_\epsilon) \dVol_{\tg} \\
      &\ge \int_{\tM} \eta \mathcal{K}_p(u) \dVol_{\tg}.
\end{align*}
This shows that the distributional pairing $\langle \mathcal{K}_p(u), \eta \rangle$ is nonnegative for any nonnegative test function $\eta$. Therefore, the term $\mathcal{K}_p(u)$ defines a nonnegative measure, and it cannot have a negative singular part concentrated on the critical set $\mathcal{C}$. This completes the rigorous justification.
\end{proof}

