\section{Capacity of Singularities and Flux Estimates}
\label{app:Capacity}
\label{sec:Capacity}

In this appendix we compute the $p$-capacity of the conical tips explicitly and show it vanishes, thereby justifying the removability statements used in the main text.

\begin{definition}[$p$-Capacity]
For a compact set $K \subset (\tM, \tg)$ and $1 < p < n$, the $p$-capacity is defined as:
\[
    \Cap_p(K) = \inf \left\{ \int_{\tM} |\nabla \psi|^p \, dV_{\tg} : \psi \in C^\infty_c(\tM), \, \psi \ge 1 \text{ on } K \right\}.
\]
A set $K$ is said to be \emph{removable} for $W^{1,p}$ functions if $\Cap_p(K) = 0$, meaning that $W^{1,p}(\tM) = W^{1,p}(\tM \setminus K)$ with equal norms.
\end{definition}

\begin{theorem}[Zero Capacity of Conical Tips]\label{thm:CapacityZero}
Let $(\tM, \tg)$ be the 3-dimensional manifold with isolated conical singularities $\{p_k\}$. Near $p_k$ the metric is asymptotic to $dr^2 + c^2 r^2 g_{S^2}$ with cone constant $c > 0$. For $1<p<3$, $\Cap_p(\{p_k\})=0$.
\end{theorem}

\begin{remark}[Cone Angle Specification]\label{rem:ConeAngleSpec}
The cone constant $c > 0$ is related to the \emph{deficit angle} $\delta$ by $c = 1 - \delta/(2\pi)$, or equivalently, the \emph{cone angle} $\theta$ by $c = \theta/(2\pi)$. The standard cone has $c = 1$ (no deficit); a deficit angle $\delta > 0$ corresponds to $c < 1$.

In our application, the conical tips arise from the \emph{bubble sealing} procedure in the Jang conformal deformation. The cone constant is determined by the asymptotic behavior of the Jang graph near the bubble MOTS. Specifically:

\textbf{(1) Origin of the cone constant:}
Near a Jang bubble $B_k$, the conformal factor $\phi$ satisfies $\phi(x) \sim d(x, B_k)^\alpha$ for some $\alpha > 0$. The conformal metric $\tg = \phi^4 \bg$ then has:
\begin{equation}
    \tg \sim d^{4\alpha} (dr^2 + r^2 g_{S^2}) = d\rho^2 + c^2 \rho^2 g_{S^2},
\end{equation}
where $\rho = r^{1+2\alpha}/(1+2\alpha)$ is the conformally rescaled radial coordinate and:
\begin{equation}
    c = \frac{1}{1 + 2\alpha}.
\end{equation}

\textbf{(2) Range of $c$ in our setting:}
The exponent $\alpha$ is determined by the stability/mean curvature properties of the bubble MOTS. For the conformal Laplacian equation, the natural range is $\alpha \in (0, 1/2)$, giving:
\begin{equation}
    c \in (1/2, 1).
\end{equation}
This corresponds to cone angles $\theta \in (\pi, 2\pi)$, i.e., deficit angles $\delta \in (0, \pi)$.

\textbf{(3) Why $c > 0$ suffices for capacity zero:}
The capacity computation in Theorem~\ref{thm:CapacityZero} shows:
\begin{equation}
    \Cap_p(\{p_k\}) \lesssim c^2 \epsilon^{3-p} \to 0 \quad \text{as } \epsilon \to 0,
\end{equation}
for any $c > 0$ and $1 < p < 3$. The factor $c^2$ enters through the volume element $dV_{\tg} = c^2 r^2 dr d\sigma_{S^2}$, but does not affect the vanishing of capacity.

\textbf{(4) Geometric interpretation:}
The capacity vanishes because the conical tip is \emph{sharp enough} that test functions can be cut off with arbitrarily small $W^{1,p}$ energy. The critical dimension is $p = n = 3$; for $p < 3$, even a point in $\R^3$ has zero $p$-capacity. The conical structure (with any $c > 0$) is quasi-isometric to a neighborhood of a point in $\R^3$, preserving this property.

\textbf{(5) Excluded case $c = 0$:}
If $c = 0$, the tip would be \emph{cusp-like} rather than conical, and the metric would degenerate. This case does not arise in our construction because the conformal factor $\phi$ remains uniformly positive away from the bubble (by the maximum principle for the Lichnerowicz equation).
\end{remark}

\begin{proof}
Fix a tip $p_k$ and work inside a geodesic ball $B_R(p_k)$ where the metric is comparable to the model cone $dr^2 + c^2 r^2 g_{S^2}$.

\textbf{Step 1: Volume element on the cone.}
The volume form in the cone metric is:
\[
    dV_{\tg} = \sqrt{\det(\tg)} \, dr \, d\sigma = c^2 r^2 \, dr \, d\sigma_{S^2},
\]
where $d\sigma_{S^2}$ is the standard area element on the unit sphere with total area $4\pi$. Integrating over the sphere:
\[
    \text{Vol}(B_r(p_k)) = \int_0^r \int_{S^2} c^2 s^2 \, d\sigma \, ds = 4\pi c^2 \int_0^r s^2 \, ds = \frac{4\pi c^2}{3} r^3.
\]

\textbf{Step 2: Construction of test functions.}
For $0 < \epsilon < R/2$, we construct a radial test function $\psi_\epsilon : \tM \to [0,1]$ as follows:
\[
\psi_\epsilon(r) = \begin{cases}
1 & \text{if } 0 \le r \le \epsilon,\\[4pt]
\displaystyle\frac{\log(R/r)}{\log(R/\epsilon)} & \text{if } \epsilon < r < R,\\[6pt]
0 & \text{if } r \ge R.
\end{cases}
\]
This logarithmic cutoff is adapted to the critical dimension $p = 3$ in dimension $n = 3$. Alternatively, for explicit calculations we use:
\[
\psi_\epsilon(r) = \begin{cases}
1 & \text{if } 0 \le r \le \epsilon,\\[4pt]
\displaystyle\left(\frac{R^{(p-3)/(p-1)} - r^{(p-3)/(p-1)}}{R^{(p-3)/(p-1)} - \epsilon^{(p-3)/(p-1)}}\right) & \text{if } \epsilon < r < R,\\[6pt]
0 & \text{if } r \ge R.
\end{cases}
\]
This is the $(p,n)$-capacitary test function in the cone geometry.

\textbf{Step 3: Gradient computation.}
For the power-law cutoff, the radial derivative in the annulus $\epsilon < r < R$ is:
\[
    \partial_r \psi_\epsilon = \frac{-(p-3)/(p-1) \cdot r^{(p-3)/(p-1)-1}}{R^{(p-3)/(p-1)} - \epsilon^{(p-3)/(p-1)}} = \frac{(3-p)/(p-1) \cdot r^{-2/(p-1)}}{R^{(p-3)/(p-1)} - \epsilon^{(p-3)/(p-1)}}.
\]
Since $\psi_\epsilon$ is radial, $|\nabla \psi_\epsilon|^2 = |\partial_r \psi_\epsilon|^2$ in the cone metric. Thus:
\[
    |\nabla \psi_\epsilon|^p = \left| \frac{(3-p)/(p-1)}{R^{(p-3)/(p-1)} - \epsilon^{(p-3)/(p-1)}} \right|^p r^{-2p/(p-1)}.
\]

\textbf{Step 4: Energy integral computation.}
The $p$-energy of $\psi_\epsilon$ is:
\begin{align*}
    \int_{B_R} |\nabla \psi_\epsilon|^p \, dV_{\tg} &= \int_\epsilon^R |\nabla \psi_\epsilon|^p \cdot 4\pi c^2 r^2 \, dr \\
    &= 4\pi c^2 \left| \frac{(3-p)/(p-1)}{R^{(p-3)/(p-1)} - \epsilon^{(p-3)/(p-1)}} \right|^p \int_\epsilon^R r^{-2p/(p-1)} \cdot r^2 \, dr.
\end{align*}
The exponent in the integrand is:
\[
    -\frac{2p}{p-1} + 2 = \frac{-2p + 2(p-1)}{p-1} = \frac{-2}{p-1}.
\]
Let $\beta = \frac{2}{p-1}$. Since $1 < p < 3$, we have $\beta > 1$. The integral is:
\[
    \int_\epsilon^R r^{-\beta} \, dr = \left[ \frac{r^{1-\beta}}{1-\beta} \right]_\epsilon^R = \frac{R^{1-\beta} - \epsilon^{1-\beta}}{1-\beta}.
\]
Note that $1-\beta = 1 - \frac{2}{p-1} = \frac{p-3}{p-1}$. Since $p < 3$, this exponent is negative. Let $-\gamma = \frac{p-3}{p-1}$ with $\gamma > 0$. Then:
\[
    \int_\epsilon^R r^{-\beta} \, dr = \frac{1}{\gamma} (\epsilon^{-\gamma} - R^{-\gamma}).
\]
Substituting this back into the energy expression:
\begin{align*}
    \int_{B_R} |\nabla \psi_\epsilon|^p \, dV_{\tg} &= C_p \frac{1}{|R^{-\gamma} - \epsilon^{-\gamma}|^p} \cdot (\epsilon^{-\gamma} - R^{-\gamma}) \\
    &= C_p \frac{1}{(\epsilon^{-\gamma} - R^{-\gamma})^{p-1}}.
\end{align*}
As $\epsilon \to 0$, $\epsilon^{-\gamma} \to \infty$. The expression behaves as:
\[
    \epsilon^{\gamma(p-1)} = \epsilon^{\frac{3-p}{p-1} \cdot (p-1)} = \epsilon^{3-p}.
\]

\textbf{Step 5: Conclusion.}
Since $p < 3$, we have $3 - p > 0$, so:
\[
    \Cap_p(\{p_k\}) \le \int_{B_R} |\nabla \psi_\epsilon|^p \, dV_{\tg} \asymp C \epsilon^{3-p} \xrightarrow{\epsilon \to 0} 0.
\]
This proves $\Cap_p(\{p_k\}) = 0$ for all $1 < p < 3$.

\textbf{Step 6: Finite union of singularities.}
The singular set consists of finitely many points $\{p_1, \ldots, p_N\}$ (one for each bubble). The $p$-capacity is subadditive:
\[
    \Cap_p(\{p_1, \ldots, p_N\}) \le \sum_{k=1}^N \Cap_p(\{p_k\}) = 0.
\]
Thus the entire singular set has zero $p$-capacity, and the removability results apply globally.

\textbf{Step 7: Extension to general asymptotically conical metrics.}
The above computation used the exact cone metric. For the metric $\tg$ which is only \emph{asymptotically} conical with $\tg = dr^2 + c^2 r^2 g_{S^2}(1 + O(r^\delta))$ for some $\delta > 0$, the volume element satisfies $dV_{\tg} = c^2 r^2 (1 + O(r^\delta)) dr \, d\sigma$. The correction factor $1 + O(r^\delta)$ is bounded as $r \to 0$, so the leading-order asymptotics are unchanged. The capacity estimate $\Cap_p(\{p_k\}) \lesssim \epsilon^{3-p} \to 0$ remains valid.
\end{proof}

\begin{remark}[Logarithmic Divergence at $p = 3$ and Higher Dimensions]\label{rmk:CriticalCapacity}
The capacity computation reveals why $p < 3$ is essential in three dimensions:

\textbf{(i) Critical exponent $p = n$:} In dimension $n$, a point has zero $p$-capacity if and only if $p < n$. At the critical value $p = n$, the capacitary test function becomes logarithmic rather than power-law:
\[
    \psi_\epsilon(r) = \frac{\log(R/r)}{\log(R/\epsilon)}, \quad |\nabla \psi_\epsilon| = \frac{1}{r \log(R/\epsilon)}.
\]
The $n$-energy integral then involves $\int_\epsilon^R r^{-n} \cdot r^{n-1} \, dr = \int_\epsilon^R r^{-1} \, dr = \log(R/\epsilon)$, yielding
\[
    \Cap_n(\{p\}) = \frac{c_n}{\log(R/\epsilon)^{n-1}} \not\to 0 \quad \text{as } \epsilon \to 0.
\]
Thus $\Cap_3(\{\text{point}\}) > 0$ in dimension 3, and the removability argument fails at $p = 3$.

\textbf{(ii) Implications for higher dimensions $n \ge 4$:} In dimensions $n \ge 4$, the Jang bubble tips would still be isolated points with Hausdorff dimension 0. The capacity vanishing requires $p < n$, which is satisfied for $p \in (1, n)$. However, several complications arise:
\begin{itemize}
    \item \textbf{Topology of stable MOTS:} In dimensions $n \ge 4$, stable MOTS need not be spherical (e.g., toroidal black rings in 5D). The spectral analysis of the bubble link $(\partial \mathcal{B}, g_{\mathcal{B}})$ becomes more complex, and the Yamabe positivity required for the indicial root analysis may fail for non-spherical links.
    
    \item \textbf{$p$-harmonic framework:} The AMO method requires $p \in (1, n)$ with $p$ close to 1 for the connection to IMCF. In higher dimensions, the range of admissible $p$ expands ($p \in (1, n)$ instead of $(1, 3)$), but the identification of the limiting mass functional requires careful extension of the renormalization procedures.
    
    \item \textbf{Bochner identity:} The distributional Bochner identity (Theorem~\ref{thm:DistrBochner}) generalizes to dimension $n$, but the exact form of the error terms and the required integrability conditions depend on $n$.
\end{itemize}

\textbf{(iii) No logarithmic obstruction in our setting:} In this paper, we work strictly with $1 < p < 3$ in dimension $n = 3$. The exponent $3 - p > 0$ ensures polynomial decay of the capacity: $\Cap_p(\{p_k\}) = O(\epsilon^{3-p})$. This decay is faster for $p$ closer to 1, which is precisely where the AMO method needs the strongest removability. There are no logarithmic corrections or borderline phenomena in the $p$-range relevant to our proof.

\textbf{(iv) Extension to $n \ge 4$:} A complete extension of this proof to dimensions $n \ge 4$ would require:
\begin{enumerate}
    \item A generalized topology theorem for stable MOTS (beyond the 3D Galloway--Schoen result);
    \item Extension of the AMO $p$-harmonic method to dimensions $n \ge 4$ with $p \in (1, n)$;
    \item Verification that the Jang bubble links have positive Yamabe invariant in higher dimensions.
\end{enumerate}
These questions are beyond the scope of the present work but represent natural directions for future research.
\end{remark}

Consequently we may choose logarithmic (or power-law) cutoffs $\eta_\epsilon$ supported away from $p_k$ with $\|\nabla \eta_\epsilon\|_{L^p} \to 0$. Testing the weak equation against $\phi \eta_\epsilon$ and letting $\epsilon \to 0$ yields global integration-by-parts identities: for any test function $\phi$,
\[
\int_{\tM} \langle |\nabla u|^{p-2} \nabla u, \nabla \phi \rangle dV = \lim_{\epsilon\to 0} \int_{\tM} \langle |\nabla u|^{p-2} \nabla u, \nabla (\phi \eta_\epsilon) \rangle dV.
\]
The error term $E_\epsilon = \int \phi \langle |\nabla u|^{p-2} \nabla u, \nabla \eta_\epsilon \rangle$ obeys
\[
|E_\epsilon| \le \|\phi\|_\infty \|\nabla u\|_{L^p}^{p-1} \|\nabla \eta_\epsilon\|_{L^p} \longrightarrow 0,
\]
establishing the global weak formulation invoked in Appendix~\ref{app:Bochner}.

\begin{theorem}[Complete Capacity Removability for Jang Bubbles]\label{thm:JangBubbleRemovability}
Let $(\tM, \tg)$ be the conformally deformed Jang manifold with isolated bubble singularities $\{p_k\}_{k=1}^N$. The following removability properties hold:
\begin{enumerate}
    \item \textbf{Hausdorff Dimension:} $\dim_{\mathcal{H}}(\{p_k\}) = 0 < 3 - p$ for all $1 < p < 3$.
    \item \textbf{$p$-Capacity Zero:} $\Cap_p(\{p_k\}) = 0$ for all $1 < p < 3$.
    \item \textbf{$W^{1,p}$ Removability:} $W^{1,p}(\tM) = W^{1,p}(\tM \setminus \{p_k\})$ isometrically.
    \item \textbf{AMO Compatibility:} The $p$-harmonic potentials $u_p$ extend continuously across $\{p_k\}$ and the level set flow $\{\Sigma_t\}$ does not accumulate area at the tips.
    \item \textbf{Monotonicity Preservation:} The AMO functional $\mathcal{M}_p(t)$ is well-defined and monotone on $(\tM, \tg)$ despite the singularities.
\end{enumerate}
\end{theorem}

\begin{proof}
\textbf{(1) Hausdorff Dimension:} The singular set $\{p_k\}$ is a finite set of isolated points, hence has Hausdorff dimension 0. For any $1 < p < 3$, we have $0 < 3 - p$, satisfying the dimension bound required for removability.

\textbf{(2) Capacity Zero:} This is Theorem~\ref{thm:CapacityZero}. The explicit computation shows $\Cap_p(\{p_k\}) \lesssim \epsilon^{3-p} \to 0$.

\textbf{(3) $W^{1,p}$ Removability:} By definition, $\Cap_p(K) = 0$ implies that for any $u \in W^{1,p}(\tM \setminus K)$, there exists a unique extension $\tilde{u} \in W^{1,p}(\tM)$ with $\|\tilde{u}\|_{W^{1,p}(\tM)} = \|u\|_{W^{1,p}(\tM \setminus K)}$. Conversely, restriction from $W^{1,p}(\tM)$ to $W^{1,p}(\tM \setminus K)$ is isometric. This follows from the density of $C^\infty_c(\tM \setminus K)$ in $W^{1,p}(\tM)$ (Theorem~\ref{thm:MoscoConvergence}, Step 2a).

\textbf{(4) AMO Compatibility:} The $p$-harmonic potential $u_p$ minimizes the $p$-energy $E_p(u) = \int |\nabla u|^p$ subject to boundary conditions. Since:
\begin{itemize}
    \item The boundary condition $u = 0$ on the horizon $\Sigma$ is well-defined (the horizon is a smooth surface).
    \item The asymptotic condition $u \to 1$ at the AF end is controlled by weighted decay.
    \item The singular set $\{p_k\}$ has zero $p$-capacity, hence does not affect the energy minimization problem.
\end{itemize}
The existence and uniqueness of $u_p$ follows from the direct method. By Lemma~\ref{lem:GradientNearTip}, $\nabla u_p \neq 0$ in a punctured neighborhood of each $p_k$, so the level sets $\Sigma_t = \{u_p = t\}$ are smooth hypersurfaces that do not pass through the tips. Lemma~\ref{lem:NoGhostArea} ensures no area concentration at $\{p_k\}$.

\textbf{(5) Monotonicity Preservation:} The AMO monotonicity formula relies on the Bochner identity integrated over level sets:
\[
\frac{d}{dt} \mathcal{M}_p(t) = \int_{\Sigma_t} \text{(nonnegative terms from } R_{\tg} \ge 0 \text{ and geometric quantities)}.
\]
The integration is over the regular level sets $\Sigma_t \subset \tM \setminus \{p_k\}$. By (4), these level sets are smooth and do not intersect the singular set for generic $t$. The distributional scalar curvature $R_{\tg}$ does not have a singular measure component at $\{p_k\}$ (Lemma~\ref{lem:DistHessian}), so the integrated Bochner identity holds. The monotonicity $\mathcal{M}_p(t_1) \le \mathcal{M}_p(t_2)$ for $t_1 < t_2$ follows by integration.
\end{proof}

\begin{corollary}[Bubble Singularities are Analytically Invisible]\label{cor:BubbleInvisible}
The Jang bubble singularities $\{p_k\}$ do not affect the validity of the Penrose inequality. Specifically:
\begin{enumerate}
    \item The ADM mass computation does not depend on the bubble tips (they are at finite distance in the Jang metric).
    \item The horizon area $A(\Sigma)$ is computed on the cylindrical end, away from the bubbles.
    \item The AMO monotonicity holds on the full manifold $(\tM, \tg)$.
    \item The double limit $(p, \epsilon) \to (1^+, 0)$ commutes, yielding the Penrose inequality.
\end{enumerate}
\end{corollary}

