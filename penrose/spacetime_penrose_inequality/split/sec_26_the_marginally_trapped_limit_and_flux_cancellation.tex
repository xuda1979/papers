\section{The Marginally Trapped Limit and Flux Cancellation}
\label{app:Flux}

\begin{lemma}[Vanishing of the Jang Flux]
\label{lem:FluxVanishingJang}
Let $(\overline M,\overline g)$ be the Jang deformation of an initial
data set satisfying the hypotheses of Theorem~\ref{thm:SPI}.
Let $\mathcal C\simeq[0,\infty)\times\Sigma$ be a cylindrical end
corresponding to a component $\Sigma$ of the outermost MOTS, with
coordinate $t\ge 0$ and cross-sections
$\Sigma_t=\{t\}\times\Sigma$.
Let $q$ be the Jang vector field appearing in
identity~\eqref{eq:JangScalar}, and let $\nu$ be the unit
normal to $\Sigma_t$ in $\overline g$ pointing towards increasing $t$.
Then
\[
  \lim_{T\to\infty}\int_{\Sigma_T}
      \langle q,\nu\rangle_{\overline g}\,dA_{\overline g} = 0.
\]
\end{lemma}

\begin{proof}
By Lemma~\ref{lem:SharpAsymptotics}, we have the following decay
estimates along the cylinder:
\begin{itemize}
  \item In the strictly stable case, there exists $\kappa>0$ such that
  \[
    \overline g = dt^2+\sigma + O(e^{-\kappa t}),\qquad
    |q(t,\cdot)|_{\overline g}\le C e^{-\kappa t}.
  \]

  \item In the marginally stable case,
  \[
    \overline g = dt^2+\sigma + O(t^{-2}),\qquad
    |q(t,\cdot)|_{\overline g}\le C t^{-3}.
  \]
\end{itemize}
Moreover, in both cases the area
$\operatorname{Area}_{\overline g}(\Sigma_t)$ remains uniformly bounded
for large $t$ (indeed, $\overline g$ converges to the product metric
$dt^2+\sigma$ up to controlled error).

Let $T>0$ and estimate
\[
  \left|\int_{\Sigma_T}
         \langle q,\nu\rangle_{\overline g}\,dA_{\overline g}\right|
  \le \int_{\Sigma_T} |q|_{\overline g}\,dA_{\overline g}
  \le \bigl\|q(T,\cdot)\bigr\|_{L^\infty(\Sigma_T)}
       \operatorname{Area}_{\overline g}(\Sigma_T).
\]
In the strictly stable case we have
$\|q(T,\cdot)\|_{L^\infty}\le C e^{-\kappa T}$, hence
the right-hand side tends to zero as $T\to\infty$.
In the marginally stable case the refined decay gives
$\|q(T,\cdot)\|_{L^\infty}\le C T^{-3}$, and the same conclusion
follows.
\end{proof}




