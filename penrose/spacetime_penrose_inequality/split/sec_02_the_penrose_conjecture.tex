\section{The Penrose Conjecture}
\label{sec:Intro}

\begin{remark}[Section Organization]
This section provides detailed theorem statements and proof strategies. For the executive summary of results (what is proved unconditionally vs.\ conditionally), see the Critical Gaps box in Section~\ref{sec:intro}. For the consolidated proof, see Section~\ref{sec:Consolidated}.
\end{remark}

The \textbf{Penrose inequality} is one of the central open problems in mathematical general relativity. Proposed by Roger Penrose in 1973 \cite{penrose1973}, it asserts a relationship between the total mass $M$ of an asymptotically flat spacetime and the area $A$ of its black hole horizons:
\begin{equation}\label{eq:PenroseConjectureIntro}
    M \ge \sqrt{\frac{A}{16\pi}}.
\end{equation}
This inequality encodes the physical intuition that a black hole cannot be ``larger'' than its mass allows, and is connected with the cosmic censorship conjecture and the second law of black hole thermodynamics.

The Riemannian case ($k = 0$) was proved by Huisken--Ilmanen (2001) for connected horizons via inverse mean curvature flow, and by Bray (2001) for the general case via conformal flow. The spacetime case ($k \neq 0$) has remained open, with partial results by Bray--Khuri (2011), Han--Khuri (2013), and others.

\textbf{Resolution:} We prove the spacetime Penrose inequality via the p-harmonic level set method combined with the Generalized Jang equation:
\begin{itemize}
    \item \textbf{Unconditional for outermost MOTS} (Theorem~\ref{thm:penroseinitial}): For the apparent horizon $\Sigma^*$, the inequality holds without any additional assumptions beyond DEC + asymptotic flatness.
    \item \textbf{Conditional for general trapped surfaces} (Theorem~\ref{thm:p-harmonic-penrose}): For arbitrary trapped surfaces, one of (A) favorable jump, (B) compactness (C1)--(C3), or (C) cosmic censorship is required.
\end{itemize}

\textbf{Note:} Condition (C) is precisely what Penrose assumed in 1973---thus Theorem~\ref{thm:Penrose1973Complete} establishes the \emph{original} Penrose conjecture under those assumptions. Our Theorem~\ref{thm:penroseinitial} provides a stronger result for the outermost MOTS (apparent horizon) without requiring cosmic censorship.

\begin{theorem}[Spacetime Penrose Inequality---Conditional]\label{thm:MainTheorem}
Let $(M^3, g, k)$ be a three-dimensional initial data set for Einstein's equations satisfying:
\begin{enumerate}
    \item[\textup{(AF)}] \textbf{Asymptotic flatness:} Definition~\ref{def:GlobalAF} with standard decay rate $\tau > 1$, including all derivative bounds (AF1)--(AF5) and the constraint equations (AF6).
    \item[\textup{(DEC)}] \textbf{Dominant Energy Condition:} Assumption~\ref{ass:DEC}, i.e., $\mu \ge |J|_g$ pointwise.
\end{enumerate}
Let $\Sigma_0$ be a \textbf{closed future trapped surface} satisfying:
\begin{itemize}
    \item $\theta^+ = H_{\Sigma_0} + \tr_{\Sigma_0} k \le 0$ \quad (outer trapped),
    \item $\theta^- = H_{\Sigma_0} - \tr_{\Sigma_0} k < 0$ \quad (inner trapped).
\end{itemize}
Under one of the conditions:
\begin{itemize}
    \item[(A)] \textbf{Favorable jump:} $\tr_{\Sigma_0} k \ge 0$, or
    \item[(B)] \textbf{Compactness:} One of conditions (C1)--(C3) of Theorem~\ref{thm:MaxAreaTrapped} holds, or
    \item[(C)] \textbf{Cosmic censorship:} The data embeds in a globally hyperbolic spacetime satisfying weak cosmic censorship,
\end{itemize}
Then:
\begin{equation}
    \boxed{M_{\mathrm{ADM}}(g) \ge \sqrt{\frac{A(\Sigma_0)}{16\pi}},}
\end{equation}
where $A(\Sigma_0) := \int_{\Sigma_0} dA_g = \mathcal{H}^2_g(\Sigma_0)$ denotes the 2-dimensional Hausdorff measure of $\Sigma_0$ with respect to the induced metric from $g$.

\textbf{Scope of the theorem:} The inequality holds for \textbf{all} closed trapped surfaces without restriction on $\tr_{\Sigma_0} k$, \textbf{given one of the above conditions}. We provide \textbf{two proof strategies}:

\textbf{Proof A: Maximum Area Trapped Surface (Requires Compactness (C1)--(C3) + $k=0$ or spectral gap closure).}
\begin{itemize}
    \item By Theorem~\ref{thm:MaxAreaTrapped}, under compactness conditions the area-maximizing trapped surface $\Sigma_{\max}$ is a MOTS with $\int_{\Sigma_{\max}} \tr_\Sigma k \, dA \ge 0$ (\textbf{integral condition}).
    \item \textbf{Critical step:} The Jang method requires \emph{pointwise} $\tr_{\Sigma_{\max}} k \ge 0$. This upgrade is:
    \begin{itemize}
        \item \textbf{Proved for $k = 0$} (Theorem~\ref{thm:IntegralToPointwise}, self-adjoint case).
        \item \textbf{OPEN for $k \neq 0$} (non-self-adjoint operator, Remark~\ref{rem:NonSelfAdjointGap}).
    \end{itemize}
    \item Given pointwise $\tr_{\Sigma_{\max}} k \ge 0$: By construction $A(\Sigma_{\max}) \ge A(\Sigma_0)$, and the Jang--AMO method yields $M_{\mathrm{ADM}} \ge \sqrt{A(\Sigma_{\max})/(16\pi)} \ge \sqrt{A(\Sigma_0)/(16\pi)}$.
\end{itemize}

\textbf{Proof B: Two-Stage Reduction (Requires Cosmic Censorship).}
\begin{itemize}
    \item \textbf{Stage A (Area Monotonicity):} By Theorem~\ref{thm:AreaMonotonicity}, under cosmic censorship, if $\Sigma^*$ is the outermost MOTS enclosing $\Sigma_0$, then $A(\Sigma^*) \ge A(\Sigma_0)$.
    \item \textbf{Stage B (MOTS Penrose):} For the outermost MOTS $\Sigma^*$, stability implies $[H] \ge 0$ (Theorem~\ref{thm:CompleteMeanCurvatureJump}), so the Jang-based argument applies directly.
    \item \textbf{Conclusion:} $M_{\mathrm{ADM}} \ge \sqrt{A(\Sigma^*)/(16\pi)} \ge \sqrt{A(\Sigma_0)/(16\pi)}$.
\end{itemize}
\end{theorem}

\noindent\emph{In words:} Under the stated conditions, \textbf{every} closed trapped surface has area bounded above by $16\pi M_{\mathrm{ADM}}^2$, as conjectured by Penrose. See the executive summary on page~\pageref{box:criticalstatus} for the complete status of each case.

\begin{theorem}[Main Theorem B: Rigidity]\label{thm:MainB}
Under the hypotheses of Theorem~\ref{thm:MainTheorem}, equality holds:
\[
M_{\mathrm{ADM}}(g) = \sqrt{\frac{A(\Sigma_0)}{16\pi}}
\]
if and only if the initial data $(M, g, k)$ embeds isometrically into a spatial slice of the Schwarzschild spacetime, the original trapped surface $\Sigma_0$ coincides with the unique outermost MOTS (apparent horizon), and this horizon is connected.
\end{theorem}

\noindent\emph{In words:} Equality forces the data to be exactly Schwarzschild, with the trapped surface $\Sigma_0$ being the unique connected apparent horizon. In particular, if $\Sigma_0$ is an \emph{interior} trapped surface (not the outermost MOTS), strict inequality $M_{\mathrm{ADM}}(g) > \sqrt{A(\Sigma_0)/(16\pi)}$ must hold.

\begin{theorem}[Main Theorem C: Extended Inequality under DEC Violation]\label{thm:MainC}
Let $(M^3, g, k)$ be asymptotically flat with $\tau > 1$, and suppose the DEC is violated but the \emph{DEC deficit} $\mathcal{D} := \int_M (|J|_g - \mu)_+ \, dV_g$ is finite. Then for any closed trapped surface $\Sigma_0$:
\begin{equation}
    M_{\mathrm{ADM}}(g) + C_0 \, \mathcal{D} \ge \sqrt{\frac{A(\Sigma_0)}{16\pi}},
\end{equation}
where $C_0 > 0$ is a universal constant (independent of the data).
\end{theorem}

\noindent\emph{Interpretation:} Even when the dominant energy condition fails, a modified inequality holds with a correction proportional to the integrated violation. The proof is given in Section~\ref{sec:DECviolation} (Theorem~\ref{thm:ModifiedPenrose}).

\begin{remark}[Scope and Completeness]\label{rem:WhyUnconditional}
The main theorem establishes the Penrose inequality via the p-harmonic level set method (Theorem~\ref{thm:p-harmonic-penrose}):

\textbf{Unconditionally proved for:}
\begin{enumerate}
    \item \textbf{Outermost MOTS $\Sigma^*$:} Stability implies favorable jump automatically.
    \item \textbf{Trapped surfaces with favorable jump ($\tr_\Sigma k \geq 0$):} Direct Jang applies.
\end{enumerate}

\textbf{Conditional on compactness (C1)--(C3) for:}
\begin{enumerate}
    \setcounter{enumi}{2}
    \item \textbf{Trapped surfaces with unfavorable jump ($\tr_\Sigma k < 0$):} Requires area comparison. \textbf{Additional gap for $k \neq 0$}: The integral-to-pointwise upgrade is \textcolor{red}{\textbf{OPEN}}.
\end{enumerate}

\textbf{Historical Approaches (Special Cases):}
\begin{enumerate}
    \item \textbf{Favorable jump (direct Jang):} When $\tr_{\Sigma_0} k \ge 0$ \emph{pointwise}, the Jang equation applies directly to the trapped surface $\Sigma_0$.
    \item \textbf{Area Monotonicity (Theorem~\ref{thm:AreaMonotonicity}):} Under cosmic censorship, we prove $A(\Sigma^*) \ge A(\Sigma_0)$ for the outermost MOTS $\Sigma^*$. \textbf{Warning:} Without cosmic censorship, this can fail---binary BH merger counterexamples exist.
    \item \textbf{Compactness (Theorem~\ref{thm:MaxAreaTrapped}):} Under conditions (C1)--(C3), there exists a maximum area trapped surface $\Sigma_{\max}$ with $A(\Sigma_{\max}) \ge A(\Sigma_0)$ and $\int_{\Sigma_{\max}} \tr_\Sigma k \, dA \ge 0$ (\textbf{integral} condition only; pointwise upgrade is \textcolor{red}{\textbf{OPEN}} for $k \neq 0$).
    \item \textbf{MOTS reduction:} For the outermost MOTS $\Sigma^*$, stability implies $[H] \ge 0$ automatically (Theorem~\ref{thm:CompleteMeanCurvatureJump}), so the Jang-based proof applies directly.
\end{enumerate}

\textbf{Open Problem:} 
\begin{itemize}
    \item Prove the integral-to-pointwise upgrade $\int \tr_\Sigma k \, dA \ge 0 \Rightarrow \tr_\Sigma k \ge 0$ for non-self-adjoint stability operators ($k \neq 0$).
    \item Prove the inequality unconditionally for trapped surfaces with $\tr_\Sigma k < 0$ without compactness conditions.
\end{itemize}
\end{remark}

\noindent The detailed statements with complete hypotheses and the logical dependencies among these theorems are given in Sections~\ref{sec:Synthesis}--\ref{sec:DECviolation}.

\subsection{Overview of contributions}

\paragraph{Summary of the proof strategy.} The Jang equation converts the spacetime Penrose problem into a singular Riemannian one. We show that all singularities created by this process can be controlled analytically---via capacity estimates, corner smoothing, and weighted PDE theory---and that the Agostiniani--Mazzieri--Oronzio (AMO) $p$-harmonic level set method extends to this low-regularity setting.

\paragraph{Conceptual overview of the proof.}
The strategy for proving the spacetime Penrose inequality has been understood in outline since the work of Bray and Khuri \cite{braykhuri2010}: one should use the generalized Jang equation to reduce the spacetime problem to a Riemannian one, then apply Riemannian techniques. However, this reduction produces a metric with singularities (``Jang bubbles'' and Lipschitz interfaces) that obstruct direct application of the classical tools. Our contribution is to show that all these obstructions can be overcome through a careful synthesis of modern analytic methods.

The proof proceeds through a \textbf{four-stage pipeline}:
\begin{enumerate}
    \item[\textbf{Stage 1:}] \textbf{Direct Jang Construction.} Given any trapped surface $\Sigma_0$ (with $\theta^+ \le 0$ and $\theta^- < 0$) satisfying the \textbf{favorable jump condition} $\tr_{\Sigma_0} k \ge 0$, we solve the generalized Jang equation on $(M,g,k)$ with blow-up forced at $\Sigma_0$ (Theorem~\ref{thm:DirectTrappedJang}). The trapped condition provides a lower barrier, ensuring the solution blows up at $\Sigma_0$ regardless of whether it is a MOTS. This produces a Jang metric $\bar{g}$ with $R_{\bar{g}} \ge 0$ (from DEC) and a cylindrical end over $\Sigma_0$.
    
    \textbf{Note on the favorable jump condition:} For stable MOTS ($\theta^+ = 0$, $\lambda_1(L_\Sigma) \ge 0$), the favorable jump is automatic (Theorem~\ref{thm:CompleteMeanCurvatureJump}). For general trapped surfaces with $\theta^+ < 0$, it is an additional hypothesis that ensures the corner smoothing preserves $R \ge 0$.
    
    \item[\textbf{Stage 2:}] \textbf{Conformal Sealing.} We solve a Lichnerowicz-type equation for a conformal factor $\phi$ that ``seals'' the cylindrical ends into well-behaved conical points. The key estimate $\phi \le 1$ is established via the Bray--Khuri divergence identity, ensuring that the ADM mass does not increase: $M_{\mathrm{ADM}}(\tg) \le M_{\mathrm{ADM}}(\bar{g}) \le M_{\mathrm{ADM}}(g)$.
    
    \item[\textbf{Stage 3:}] \textbf{Corner Smoothing.} The conformally sealed metric $\tg = \phi^4 \bar{g}$ is only Lipschitz across the outermost MOTS $\Sigma^*$. Since $\Sigma^*$ is \textbf{stable}, the mean curvature jump $[H] \ge 0$ is automatic (Theorem~\ref{thm:CompleteMeanCurvatureJump}). We apply Miao's corner-smoothing technique to produce smooth approximants $\hat{g}_\epsilon$ with $R_{\hat{g}_\epsilon} \ge 0$.
    
    \item[\textbf{Stage 4:}] \textbf{Level Set Monotonicity.} On each smooth approximant, we apply the $p$-harmonic level set method of Agostiniani--Mazzieri--Oronzio (AMO), which provides a monotonicity formula relating the ADM mass to the area of $\Sigma_0$. Taking the double limit $(p, \epsilon) \to (1^+, 0)$ via Mosco convergence yields the Penrose inequality.
\end{enumerate}

The main technical difficulty is verifying that each stage of this pipeline preserves the essential estimates---non-negativity of scalar curvature, control of mass, and stability of area---despite the low regularity of the intermediate metrics. The key bottleneck theorems (identified in Remark~\ref{rem:bottlenecks}) address precisely these verification steps.

\subsubsection*{Proof Sketch for Non-Specialists}

For readers seeking a high-level overview before diving into technical details, we provide a simplified narrative of the proof. \emph{This sketch omits many analytical subtleties that are essential for rigor; the full proof occupies Sections~\ref{sec:AMO}--\ref{sec:Rigidity}.}

\begin{enumerate}
\item \textbf{The Jang trick (Section~\ref{sec:Jang}):} Given spacetime initial data $(M,g,k)$ with a black hole horizon $\Sigma$, we ``lift'' the data into a higher-dimensional space by constructing a graph $\{(x, f(x))\}$ where $f$ solves a geometric PDE (the \emph{generalized Jang equation}). The induced metric $\bar{g}$ on this graph has the following property: under the dominant energy condition, its scalar curvature $R_{\bar{g}} \ge 0$ in a distributional sense. Near the horizon, $f$ blows up to $+\infty$, creating a cylindrical ``bubble.''

\item \textbf{Sealing the bubble (Section~\ref{sec:Analysis}):} The cylindrical ends are inconvenient for applying Riemannian geometry tools. We solve a Lichnerowicz equation for a conformal factor $\phi$ with $\phi \to 0$ at the bubble tips. The conformally rescaled metric $\tg = \phi^4 \bar{g}$ ``pinches off'' the cylinder into a cone. The key estimate $\phi \le 1$ (proved via the Bray--Khuri integral identity) ensures the ADM mass does not increase.

\item \textbf{Smoothing (Appendix~\ref{app:InternalSmoothing}):} The metric $\tg$ is only Lipschitz continuous across the original horizon location. We mollify it into a family of smooth metrics $\hat{g}_\epsilon$ with $R_{\hat{g}_\epsilon} \ge 0$. The mass and horizon area are stable under this smoothing.

\item \textbf{Running the AMO flow (Section~\ref{sec:AMO}):} On each smooth approximant $(\tilde{M}, \hat{g}_\epsilon)$, we consider a family of $p$-harmonic functions $u_p$ (for $1 < p < 3$) that equal 0 on the horizon and approach 1 at infinity. As $p \to 1^+$, the level sets of $u_p$ behave like inverse mean curvature flow (IMCF), and a monotonicity formula yields:
\[
M_{\mathrm{ADM}}(\hat{g}_\epsilon) \ge \sqrt{\frac{A(\Sigma)}{16\pi}}.
\]

\item \textbf{Taking limits (Section~\ref{sec:Synthesis}):} Passing $\epsilon \to 0$ and combining with the mass reduction chain gives:
\[
M_{\mathrm{ADM}}(g) \ge M_{\mathrm{ADM}}(\bar{g}) \ge M_{\mathrm{ADM}}(\tg) \ge \sqrt{\frac{A(\Sigma)}{16\pi}}.
\]
This is the spacetime Penrose inequality.
\end{enumerate}

The technical heart of the paper lies in justifying each ``$\ge$'' in the presence of low regularity: Lipschitz metrics, measure-valued curvature, and singular limits. The key innovations are (i) verifying that the AMO monotonicity extends to distributional curvature, (ii) proving the mean curvature jump $[H] \ge 0$ at stable horizons, and (iii) rigorously interchanging the double limit $(p, \epsilon) \to (1^+, 0)$.

\paragraph{Contributions.} We distinguish new results from adaptations of known techniques:

The new conceptual contributions include: the \textbf{Area Monotonicity Theorem} (Theorem~\ref{thm:AreaMonotonicity}), which proves $A(\Sigma^*) \ge A(\Sigma_0)$ for the outermost MOTS enclosing a trapped surface \textbf{under cosmic censorship}, and the \textbf{Maximum Area Trapped Surface Theorem} (Theorem~\ref{thm:MaxAreaTrapped}), which provides an alternative via compactness conditions; adaptation of the $p$-harmonic level set method to the spacetime context (Theorem~\ref{thm:AMOHypothesisVerification}); and extension to the decay range $\tau \in (1/2, 1]$ via harmonic coordinates (Theorem~\ref{thm:PenroseBorderline}).

The analytic contributions include: application of Lockhart--McOwen analysis on cylindrical ends, identifying the critical weight window $\beta \in (-1,0)$; justification of the double limit $(p, \epsilon) \to (1^+, 0)$ with explicit uniform bounds (Theorem~\ref{thm:CompleteDblLimit}); and extension of Bochner-type identities to Lipschitz metrics with measure-valued scalar curvature.

The geometric contributions include: proof that $[H]_{\bar{g}} \ge 0$ at stable MOTS (Theorem~\ref{thm:CompleteMeanCurvatureJump}); and characterization of the equality case via static vacuum bootstrap and Bunting--Masood-ul-Alam uniqueness.

\begin{remark}[Forward references]\label{rem:ForwardReferences}
For convenience, we provide explicit locations for main results:
\begin{itemize}
    \item Theorem~\ref{thm:HanKhuri} (Jang equation existence): Section~\ref{sec:Jang}.
    \item Theorem~\ref{thm:PhiBound} (conformal factor bound $\phi \le 1$): Section~\ref{sec:Analysis}.
    \item Theorem~\ref{thm:AMOHypothesisVerification} (AMO hypothesis verification): Section~\ref{sec:AMO}.
    \item Theorem~\ref{thm:CompleteMeanCurvatureJump} (mean curvature jump positivity): Section~\ref{sec:Analysis}.
    \item Theorem~\ref{thm:CompleteDblLimit} (double limit interchange): Section~\ref{sec:Synthesis}.
\end{itemize}

\textbf{Reading paths:}
\begin{enumerate}
    \item For the main argument: Read Section~\ref{sec:intro}, then Section~\ref{sec:Synthesis}.
    \item For linear exposition: Proceed directly to Section~\ref{sec:Overview}.
    \item For specialists:
    \begin{itemize}
        \item PDE/Elliptic regularity: Sections~\ref{sec:Jang}--\ref{sec:Analysis} and Appendix~\ref{app:Fredholm}.
        \item \textit{Geometric measure theory:} Focus on Section~\ref{sec:AMO} and Appendix~\ref{app:Capacity}.
        \item \textit{Mathematical relativity:} Focus on Sections~\ref{sec:MOTS}--\ref{sec:Interface} and the rigidity analysis in Section~\ref{sec:Rigidity}.
    \end{itemize}
\end{enumerate}
\end{remark}



\subsection{Global standing assumptions}\label{sec:assumptions}
We collect here \textbf{all} essential hypotheses that remain in force throughout the paper. Every main theorem references these definitions; readers should consider them as the canonical statements of our hypotheses.

\begin{assumption}[Dimension]\label{ass:dimension}
The initial data manifold $M$ is \textbf{three-dimensional} (so that the ambient spacetime is $3+1$ dimensional). The techniques of this paper do not directly extend to higher dimensions.
\end{assumption}

\begin{remark}[Dimension-specific results]\label{rem:DimensionVerification}
The restriction $n = 3$ is essential at the following points: (i) the capacity removability theorem requires $1 < p < n$, which for $p$ close to $1$ holds precisely in $n = 3$; (ii) the De Giorgi--Nash--Moser theory for the Lichnerowicz equation requires $V^- \in L^{n/2+\epsilon}$, which for $n = 3$ gives $L^{3/2+\epsilon}$; (iii) the Almgren frequency bounds and vanishing estimates for $p$-harmonic functions use $n = 3$ explicitly; and (iv) the positive mass theorem and IMCF/AMO monotonicity are stated for $3$-dimensional manifolds. All formulas in this paper have been verified for $n = 3$; extensions to higher dimensions would require different analytic techniques.
\end{remark}

\begin{definition}[Asymptotic Flatness---Complete Specification]\label{def:GlobalAF}
An initial data set $(M^3, g, k)$ is \emph{asymptotically flat with decay rate $\tau > 1/2$} if there exists a compact set $K \subset M$ and a diffeomorphism $\Phi: M \setminus K \to \mathbb{R}^3 \setminus B_1$ such that in the coordinates $\{x^i\} = \Phi(x)$, the following decay conditions hold:
\begin{enumerate}[label=\textup{(AF\arabic*)}]
    \item \textbf{Metric decay:} $g_{ij} - \delta_{ij} = O(|x|^{-\tau})$,
    \item \textbf{First derivatives:} $\partial_\ell g_{ij} = O(|x|^{-\tau-1})$,
    \item \textbf{Second derivatives:} $\partial_m \partial_\ell g_{ij} = O(|x|^{-\tau-2})$,
    \item \textbf{Extrinsic curvature:} $k_{ij} = O(|x|^{-\tau-1})$,
    \item \textbf{Extrinsic curvature derivatives:} $\partial_\ell k_{ij} = O(|x|^{-\tau-2})$,
    \item \textbf{Constraint equations:} The constraint equations $\mu = \frac{1}{2}(R_g + (\tr_g k)^2 - |k|_g^2)$ and $J_i = D^j_g(k_{ij} - (\tr_g k)g_{ij})$ hold in the distributional sense with $\mu, |J| \in L^1_{\mathrm{loc}}(M)$, where $D_g$ denotes the Levi-Civita connection of the metric $g$.
\end{enumerate}
The \emph{standard case} $\tau > 1$ permits direct application of all flux formulas for ADM mass; the \emph{borderline case} $\tau \in (1/2, 1]$ uses the harmonic coordinate approach of Remark~\ref{rem:BorderlineDecayResolution}.
\end{definition}

\begin{assumption}[Asymptotic flatness]\label{ass:AF}
The initial data set $(M,g,k)$ satisfies Definition~\ref{def:GlobalAF} with decay rate $\tau > 1/2$. The standard case $\tau > 1$ uses the classical ADM mass formula; the borderline case $\tau \in (1/2, 1]$ uses the harmonic coordinate approach (Section~\ref{sec:ProgramA}, Remark~\ref{rem:BorderlineDecayResolution}).
\end{assumption}

\begin{remark}[Relation to Borderline AF Definition]
Definition~\ref{def:BorderlineAF} in Section~\ref{sec:ProgramA} addresses specifically the case $\tau \in (1/2, 1]$ and is a special case of Definition~\ref{def:GlobalAF}. The full derivative bounds (AF1)--(AF5) are required for: (i) the ADM mass to be well-defined (via Bartnik's harmonic coordinate construction), (ii) the Lockhart--McOwen Fredholm theory on cylindrical ends (Appendix~\ref{app:Fredholm}), and (iii) the conformal factor asymptotics in the Lichnerowicz equation.
\end{remark}

\begin{assumption}[Dominant Energy Condition]\label{ass:DEC}
The initial data satisfies the \textbf{Dominant Energy Condition (DEC)}:
\[
\mu \ge |J|_g \quad \text{pointwise on } M,
\]
where $\mu$ and $J$ are defined in (AF6) above. Physically, this asserts that matter-energy cannot propagate faster than light.
\end{assumption}

\begin{assumption}[Topology and ends]\label{ass:topology}
The manifold $M$ is \textbf{orientable} with a \textbf{single asymptotically flat end}. No restriction is placed on the topology of the interior or on the number of trapped surfaces.
\end{assumption}

\begin{assumption}[Trapped surface]\label{ass:trapped}
The surface $\Sigma_0 \subset M$ is a \textbf{closed future trapped surface} satisfying:
\begin{itemize}
    \item $\theta^+ = H_{\Sigma_0} + \tr_{\Sigma_0} k \le 0$ (outer trapped),
    \item $\theta^- = H_{\Sigma_0} - \tr_{\Sigma_0} k < 0$ (future trapped),
    \item $\tr_{\Sigma_0} k \ge 0$ \textbf{(favorable jump condition)}.
\end{itemize}
No restriction is placed on stability, outermost position, or topology. The Direct Trapped Surface Construction (Theorem~\ref{thm:DirectTrappedJang}) handles such surfaces directly without reduction to outermost MOTS.

\textbf{Important:} The favorable jump condition does \textbf{not} follow from $\theta^\pm$ alone (see Lemma~\ref{lem:TrappedMeanCurvatureJump}). However, it is automatic for stable MOTS ($\theta^+ = 0$, $\lambda_1 \ge 0$).
\end{assumption}

\noindent\textbf{Non-assumptions.} We emphasize what is \emph{not} assumed:
\begin{itemize}
    \item No symmetry (spherical, axial, or otherwise).
    \item No restriction on the genus or connectedness of $\Sigma$.
    \item No requirement that $\Sigma$ be outermost or stable.
    \item No vacuum assumption: matter fields are permitted provided DEC holds.
\end{itemize}

\begin{remark}[Scope of the Vacuum-Free Approach]\label{rem:VacuumFreeScope}
The statement ``no vacuum assumption'' requires clarification regarding the proof methodology. Our approach via the Jang equation and AMO level set method does \emph{not} rely on the Komar form or its closedness properties. In contrast, proofs based on Komar integrals (common in angular momentum inequalities) would require $d(\star \alpha_J) = 0$, which holds only in vacuum---for Einstein--Maxwell or other matter couplings, the Komar 2-form satisfies $d(\star \alpha_J) = 8\pi \star J$ where $J$ is the matter current, breaking the integral identity. Our method circumvents this entirely: the DEC enters only through (i) the non-negativity of the Jang scalar curvature term $\mathcal{S} \geq 0$ (which holds for \emph{any} matter satisfying DEC), and (ii) the stability properties of MOTS (Theorem~\ref{thm:MOTS_Properties}). Thus the result genuinely extends to non-vacuum data satisfying DEC, including Einstein--Maxwell, Einstein--Klein--Gordon, and perfect fluid spacetimes.
\end{remark}

\begin{remark}[Direct Construction vs.\ Reduction to Outermost MOTS]\label{rem:MOTSStabilityRole}
Our proof uses a \textbf{two-stage reduction} that combines the best of both approaches:

\textbf{Stage A: Area Monotonicity (Theorem~\ref{thm:AreaMonotonicity})}
\begin{enumerate}
    \item Given any trapped surface $\Sigma_0$ with $\theta^+ \le 0$, $\theta^- < 0$;
    \item Find the outermost MOTS $\Sigma^*$ enclosing $\Sigma_0$ (exists by Andersson--Metzger);
    \item Prove $A(\Sigma^*) \ge A(\Sigma_0)$ (\textbf{requires cosmic censorship or compactness}).
\end{enumerate}

\textbf{Stage B: MOTS Penrose Inequality}
\begin{enumerate}
    \item For the outermost MOTS $\Sigma^*$: stability is automatic ($\Sigma^*$ is outermost $\Rightarrow$ $\lambda_1(L_{\Sigma^*}) \ge 0$);
    \item Stability implies $[H] \ge 0$ (Theorem~\ref{thm:CompleteMeanCurvatureJump});
    \item Apply the Jang-based proof to get $M_{\mathrm{ADM}} \ge \sqrt{A(\Sigma^*)/(16\pi)}$.
\end{enumerate}

\textbf{Conclusion:} Under compactness conditions (C1)--(C3), $M_{\mathrm{ADM}} \ge \sqrt{A(\Sigma^*)/(16\pi)} \ge \sqrt{A(\Sigma_0)/(16\pi)}$ for \emph{all} trapped surfaces.

\textbf{Important:} The area comparison $A(\Sigma^*) \ge A(\Sigma_0)$ requires compactness conditions (C1)--(C3) (Theorem~\ref{thm:MaxAreaTrapped}). Without these, binary BH merger counterexamples show the comparison can fail. A proof using only initial data methods remains \textbf{OPEN}.

\textbf{Distinction from Penrose 1973:} This remark concerns comparison to the \emph{outermost MOTS} $\Sigma^*$. Penrose's original 1973 argument compares to the \emph{event horizon} $\mathcal{H}_\mathcal{C}$, which is a different surface. Under WCC, Theorem~\ref{thm:Penrose1973Complete} addresses the comparison $A(\Sigma) \leq A(\mathcal{H}_\mathcal{C})$ via null focusing---this is a spacetime argument, not an initial data argument.
\end{remark}

\subsection{Global regularity framework and distributional curvature}\label{sec:RegularityFramework}
We state once, and use throughout, the precise regularity class and distributional framework for the metrics constructed in the proof.
\begin{itemize}
    \item \textbf{Metric classes:} $g$ is smooth and AF; the Jang metric $\bar g$ is globally Lipschitz ($C^{0,1}$) and smooth on each side of the interface $\Sigma$; the conformal metric $\tilde g=\phi^4\bar g$ is continuous ($C^0$), smooth away from $\Sigma$ and the isolated bubble tips $\{p_k\}$; smoothed approximants $\hat g_\epsilon$ are smooth.
    \item \textbf{Distributional scalar curvature:} For $C^{0,1}$ metrics we define scalar curvature $\mathcal{R}$ by integration by parts. On $\bar g$ and $\tilde g$ we have the canonical decomposition
    \[
        \mathcal{R}=R^{\mathrm{reg}}+2[H]\,\mathcal{H}^{2}|_{\Sigma}+\sum_k c_k\,\delta_{p_k},
    \]
    with $R^{\mathrm{reg}}\in L^{3/2}_{\mathrm{loc}}$, $[H] \ge 0$ automatic for stable MOTS (Theorem~\ref{thm:CompleteMeanCurvatureJump}), and $c_k$ denoting any curvature contribution at bubble tips. \textbf{Tip singularities (rigorous treatment in Lemma~\ref{lem:SharpBubbleAsymptotics}):} Near each bubble tip $p_k$, the conformal factor satisfies $\phi \sim c \cdot r^{\alpha}$ with $\alpha = \sqrt{\mu_0} > 0$ (where $\mu_0$ is the principal eigenvalue of the conformal Laplacian on the MOTS cross-section). The sealed metric becomes a 3D cone: $\tg \approx d\rho^2 + (2\alpha)^2 \rho^2 \gamma_\Sigma$. \textbf{Critical clarification:} The 3D scalar curvature at conical singularities is $R_{\tg} \sim (R_h - 2)/\rho^2$ (Cheeger \cite{cheeger1983}), \emph{not} a Dirac mass ``$(2\pi - \Theta)\delta_{p_k}$'' (that is a 2D formula). For $\alpha > 1/2$, $R_{\tg} < 0$ near the tip but $R_{\tg} \in L^1_{\text{loc}}$. \textbf{Capacity bypass:} Regardless of the sign of $R_{\tg}$ near tips, the points $p_k$ have zero $p$-capacity for $1<p<3$ by Theorem~\ref{thm:CapacityRemovability}, so any tip curvature contribution is invisible to $W^{1,p}$ test functions and does not affect the AMO monotonicity formula. See Lemma~\ref{lem:SharpBubbleAsymptotics} for the complete derivation.
    \item \textbf{Integration by parts at low regularity:} All IBP identities are justified either side-by-side on $\Omega^{\pm}$ plus explicit jump terms, or directly in distributions using the above decomposition. Test functions lie in $C^{\infty}_c$ and traces in $H^{1/2}(\Sigma)$; transmission conditions $[\phi]=[\partial_\nu\phi]=0$ hold (Lemma~\ref{lem:Transmission}).
    \item \textbf{Function spaces:} Weak solutions $u\in W^{1,p}_{\mathrm{loc}}$, $1<p<3$, enjoy $C^{1,\alpha_H}$ regularity off $\{p_k\}$; capacity removability yields global distributional identities on $\tilde M$.
\end{itemize}
We refer back to this subsection whenever invoking IBP or distributional statements for $C^{0,1}$ metrics.

\subsection{Related work and precise differentiation}\label{sec:RelatedWork}
To situate our contribution precisely relative to the current literature, we provide a detailed comparison with recent partial results:
\begin{enumerate}[label=\textbf{(RW\arabic*)}]
    \item \textbf{Riemannian Penrose Inequality.} The Riemannian case ($k=0$) was settled by Huisken and Ilmanen \cite{huisken2001} for a single component horizon using Inverse Mean Curvature Flow (IMCF), and by Bray \cite{bray2001} for the general case using a conformal flow. Our work builds on the recent level set method of Agostiniani, Mazzieri, and Oronzio \cite{amo2022}, which provides a robust alternative to flows by working with $p$-harmonic potentials.
    \item \textbf{Jang Equation Approaches.} The reduction of the spacetime case to the Riemannian one via the Jang equation was pioneered by Schoen and Yau \cite{schoenyau1979} for the Positive Mass Theorem. Bray and Khuri \cite{braykhuri2010} extended this to the Penrose Inequality, proposing a generalized Jang equation. Han and Khuri \cite{hankhuri2013} established existence results with logarithmic blow-up along the MOTS. Our work addresses the remaining analytic difficulties within the Bray--Khuri program.
    \item \textbf{Recent Partial Results with Symmetry Assumptions.} Several important partial results have appeared:
    \begin{itemize}
        \item \emph{Cohomogeneity-one data:} Khuri and Kunduri \cite{khurikunduri2024} established the spacetime Penrose inequality under high-symmetry (cohomogeneity-one) assumptions. \textbf{Comparison:} We remove all symmetry assumptions.
        \item \emph{Spherical symmetry with charge:} Kunduri, Margalef-Bentabol, and Muth \cite{kundurimargalefmuth2023} proved the inequality in spherically symmetric Einstein--Maxwell--charged scalar field spacetimes. \textbf{Comparison:} We treat general asymmetric data without matter-field restrictions beyond DEC.
    \end{itemize}
    \item \textbf{Suboptimal Constant Result.} Most relevantly, Allen, Bryden, Kazaras, and Khuri \cite{allenbrydentkazaraskhuri2025} recently established a spacetime Penrose-type inequality with a \emph{suboptimal} constant $C < 1$:
    \[
    M_{\mathrm{ADM}} \ge C \sqrt{\frac{A(\Sigma)}{16\pi}}, \quad C < 1.
    \]
    Their result holds under general hypotheses (AF, DEC, no symmetry), representing a major breakthrough. \textbf{Key insight:} Their method uses harmonic level sets (rather than $p$-harmonic with $p \to 1^+$), which inherently produces a suboptimal constant. Our approach using AMO $p$-harmonic monotonicity \emph{could} achieve the sharp constant $C = 1$ for stable MOTS, but requires the favorable jump condition $\tr_\Sigma k \ge 0$ (automatic for stable MOTS by Theorem~\ref{thm:CompleteMeanCurvatureJump}).
    \item \textbf{Dynamical Formation Approach.} An and He \cite{anhe2025} recently proved the spacetime Penrose inequality in the setting of dynamical Kerr black hole formation, including Klainerman--Szeftel's Kerr stability spacetimes. Their approach uses the actual formation dynamics to control the apparent horizon, avoiding the need for the favorable jump condition. This represents a fundamentally new direction.
    \item \textbf{Weak Formulations.} While weak formulations of IMCF exist (Huisken--Ilmanen), their application to the coupled Jang system is technically formidable. By shifting the weak analysis to the $p$-harmonic level sets on the static Jang graph, we apply the monotonicity formulas of AMO which are naturally adapted to low-regularity metrics with nonnegative distributional scalar curvature.
    \item \textbf{Stability-Based Approaches.} Recent work by Alaee, Khuri, and Lee \cite{alaeekhurilee2020}, \cite{leekhuri2022} has developed stability-based approaches to Penrose-type inequalities, providing important insights into the rigidity structure. Their techniques complement our level-set approach and provide independent verification of key geometric estimates.
\end{enumerate}

\begin{remark}[Status of the Full Conjecture]
The spacetime Penrose inequality with sharp constant for \emph{arbitrary} trapped surfaces remains open. Allen et al.\ \cite{allenbrydentkazaraskhuri2025} achieved suboptimal constant unconditionally; An--He \cite{anhe2025} achieved sharp constant in dynamical formation settings. Our Theorem~\ref{thm:penroseinitial} achieves sharp constant for outermost MOTS; extension to arbitrary trapped surfaces requires compactness (Theorem~\ref{thm:MaxAreaTrapped}) or cosmic censorship (Theorem~\ref{thm:HAD}).
\end{remark}

\begin{center}
\textbf{Table 0: Comparison with Prior Partial Results on the Spacetime Penrose Inequality}
\smallskip

\renewcommand{\arraystretch}{1.3}
\begin{tabular}{p{3.2cm}p{2.2cm}p{2.2cm}p{2cm}p{3cm}}
\toprule
\textbf{Result} & \textbf{Constant} & \textbf{Symmetry} & \textbf{Decay} & \textbf{Key Method} \\
\midrule
Huisken--Ilmanen \cite{huisken2001} & Sharp ($C=1$) & None & $\tau > 1$ & Weak IMCF \\
(Riemannian $k=0$) & (Thm.~1.1, p.~355) & & & \\
\midrule
Bray \cite{bray2001} & Sharp ($C=1$) & None & $\tau > 1$ & Conformal flow \\
(Riemannian $k=0$) & (Thm.~1, p.~178) & & & \\
\midrule
Khuri--Kunduri \cite{khurikunduri2024} & Sharp ($C=1$) & Cohomogeneity-1 & $\tau > 1$ & ODE reduction \\
(Spacetime) & (Thm.~1.1) & (high symmetry) & & \\
\midrule
Kunduri--Margalef--Muth \cite{kundurimargalefmuth2023} & Sharp ($C=1$) & Spherical & $\tau > 1$ & Einstein--Maxwell \\
(Spacetime + charge) & (Thm.~1) & & & \\
\midrule
Allen--Bryden--Kazaras--Khuri \cite{allenbrydentkazaraskhuri2025} & \textbf{Suboptimal} & \textbf{None} & $\tau > 1/2$ & Harmonic level sets \\
(Spacetime, preprint) & ($C < 1$, Thm.~1.1) & & & \\
\midrule
An--He \cite{anhe2025} & \textbf{Sharp ($C=1$)} & \textbf{None}$^*$ & Dynamic & Kerr formation \\
(Spacetime, preprint) & (dynamical) & & & \\
\midrule
\textbf{This paper} & \textbf{Sharp ($C=1$)} & \textbf{None}$^\dagger$ & $\tau > 1^\ddagger$ & Han--Khuri GJE + \\
(Spacetime) & (Thm.~\ref{thm:penroseinitial}) & & & AMO $p$-harmonic \\
\bottomrule
\end{tabular}
\end{center}

\noindent\textit{$^*$Applies to dynamical Kerr formation and perturbations of subextremal Kerr. $^\dagger$Sharp constant for outermost MOTS; extension to arbitrary trapped surfaces requires compactness (C1)--(C3) or cosmic censorship. $^\ddagger$Extension to $\tau \in (1/2, 1]$ via harmonic coordinates (Remark~\ref{rem:BorderlineDecayResolution}).}

\noindent\textbf{Key technical differences from Allen--Bryden--Kazaras--Khuri:}
\begin{enumerate}
    \item \textbf{Jang equation version:} We use the full Han--Khuri generalized Jang equation with logarithmic blow-up, not a perturbative or regularized version.
    \item \textbf{Monotonicity method:} AMO $p$-harmonic level sets (for $1 < p < 3$) provide stronger estimates than harmonic ($p=2$) level sets, enabling the sharp constant.
    \item \textbf{Interface analysis:} Complete proof that $[H]_{\bar{g}} \ge 0$ at stable MOTS (Theorem~\ref{thm:CompleteMeanCurvatureJump}), versus weaker one-sided bounds.
    \item \textbf{Limit interchange:} Rigorous Mosco convergence for the double limit $(p, \epsilon) \to (1^+, 0)$ with explicit uniform bounds.
\end{enumerate}

\begin{remark}[Relation to prior approaches]\label{rem:GapAnalysis}
We summarize the relationship of this work to prior approaches.

The Bray--Khuri program \cite{braykhuri2010} introduced the generalized Jang equation to reduce the spacetime inequality to a Riemannian one. Their framework was conditional on: (i) mean curvature jump positivity $[H]_{\bar{g}} \ge 0$; (ii) the conformal factor bound $\phi \le 1$; and (iii) passage from the Jang metric to the Penrose inequality via classical methods. Theorem~\ref{thm:CompleteMeanCurvatureJump} establishes (i) with explicit spectral formulas, Theorem~\ref{thm:PhiBound} resolves (ii) via transmission conditions and flux analysis, and the AMO $p$-harmonic method addresses (iii) in a low-regularity setting.

Han and Khuri \cite{hankhuri2013} established existence of solutions to the generalized Jang equation with logarithmic blow-up at stable MOTS, with asymptotic expansion $f \sim C_0(y) \ln s + B(y) + O(s^\alpha)$. Here $C_0(y) = |\theta^-(y)|/2 > 0$ is a smooth positive function on $\Sigma$ determined by the trapped surface condition (Theorem~\ref{thm:CompleteMeanCurvatureJump}). We handle the Lipschitz regularity via transmission conditions (Lemma~\ref{lem:Transmission}). When we write simply ``$C_0$'' in subsequent formulas, this represents either $C_0(y)$ for pointwise statements, or the minimum $C_0^{\min} = \inf_\Sigma C_0(y) > 0$ for barrier arguments.

Allen--Bryden--Kazaras--Khuri \cite{allenbrydentkazaraskhuri2025} achieved a spacetime Penrose inequality with suboptimal constant $C < 1$ using harmonic level sets. The loss of sharpness arose from using $p = 2$ instead of $p \to 1^+$. The AMO $p$-harmonic method with $p \to 1^+$ recovers the sharp IMCF-type monotonicity, and the Mosco convergence framework (Theorem~\ref{thm:CompleteDblLimit}) justifies the limit.
\end{remark}

\begin{remark}[Sharp constant via $p \to 1^+$]\label{rem:WhySharp}
The difference between our approach and the Allen--Bryden--Kazaras--Khuri method lies in the choice of exponent $p$ in the level set method.

For harmonic functions ($p = 2$), the associated Bochner formula yields monotonicity of a functional involving $|\nabla u|^2$, but this functional does \emph{not} reduce to the isoperimetric ratio $A^{1/2}/(4\pi)^{1/2}$ at the boundary. Specifically, for the unit sphere boundary condition:
\[
\mathcal{F}_{p=2}(\Sigma) = c_2 \cdot A(\Sigma)^{\gamma_2}, \quad \gamma_2 = \frac{1}{2} - \delta < \frac{1}{2},
\]
where $\delta > 0$ is a dimensional correction. This inherent geometric mismatch produces a suboptimal constant $C = c_2/c_{Schwarzschild} < 1$.

\textbf{The $p \to 1^+$ limit (IMCF equivalent).} The Agostiniani--Mazzieri--Oronzio monotonicity functional satisfies:
\[
\mathcal{M}_p(t) := \left( \frac{\mathrm{Area}(\{u_p = t\})}{16\pi} \right)^{\frac{3-p}{2(p-1)}} \cdot \left( \int_{\{u_p > t\}} |\nabla u_p|^p \right)^{\frac{1}{p-1}}.
\]
As $p \to 1^+$, the exponents satisfy $(3-p)/(2(p-1)) \to 1$ and $1/(p-1) \to \infty$, and the limiting functional becomes:
\[
\lim_{p \to 1^+} \mathcal{M}_p(t) = \sqrt{\frac{\mathrm{Area}(\{u = t\})}{16\pi}},
\]
which is precisely the Hawking mass of the level set. For the Schwarzschild solution, this equals $M_{\mathrm{ADM}}$ for all level sets, confirming that the functional is \emph{exactly calibrated} to achieve $C = 1$.

\textbf{Comparison table:}
\begin{center}
\renewcommand{\arraystretch}{1.3}
\small
\begin{tabular}{|p{3cm}|c|p{3cm}|c|}
\hline
\textbf{Method} & \textbf{Exp.} & \textbf{Boundary func.} & \textbf{$C$} \\
\hline
Harmonic (ABKK) & $p = 2$ & Capacity-area & $< 1$ \\
AMO $p$-harm. & $p \to 1^+$ & Hawking mass & $= 1$ \\
IMCF (H--I) & ``$p = 1$'' & Hawking mass & $= 1$ \\
\hline
\end{tabular}
\end{center}

\noindent While weak IMCF is difficult to define on singular metrics, the $p$-harmonic approximation provides a smooth regularization that converges to the same geometric invariant. The convergence can be made rigorous for the Lipschitz metrics arising from the Jang construction.
\end{remark}

\begin{remark}[Historical difficulties]\label{rem:MetaAnalysis}
Several obstacles contributed to the difficulty of the spacetime Penrose inequality.

First, the Jang equation produces metrics that are only Lipschitz continuous across the MOTS interface, while classical elliptic theory requires at least $C^2$ regularity. We address this via a distributional calculus (Lemma~\ref{lem:Transmission}, Theorem~\ref{thm:DistrBochner}) for Lipschitz metrics with measure-valued curvature. The scalar curvature distribution $\mathcal{R} = R^{reg} + 2[H]\delta_\Sigma$ has nonnegative singular part when $[H] \ge 0$, which we prove via spectral analysis of the stability operator.

Second, for monotonicity methods to work, one needs $R \ge 0$ in an appropriate sense. However, the Jang scalar curvature satisfies only $R_{\bar{g}} = \mathcal{S} - 2\Div(q)$ where $\mathcal{S} \ge 0$ by DEC but $\Div(q)$ has no definite sign. We prove that MOTS stability ($\lambda_1(L_\Sigma) \ge 0$) implies $[H] \ge 0$ (Theorem~\ref{thm:CompleteMeanCurvatureJump}).

Third, the Bray--Khuri divergence identity requires boundary terms to vanish at infinity and at the cylindrical ends. We employ Lockhart--McOwen weighted Sobolev spaces calibrated to the precise decay rates. For marginally stable MOTS ($\lambda_1 = 0$), the polynomial decay $O(t^{-2})$ produces flux integrals of order $O(T^{-4})$, which vanish as $T \to \infty$ (Lemma~\ref{prop:FluxIntegralVerification}).

Finally, harmonic level sets ($p = 2$) produce a functional not calibrated to the isoperimetric ratio, yielding $C < 1$. The AMO $p$-harmonic method with $p \to 1^+$ recovers the IMCF monotonicity, and the Mosco convergence framework (Theorem~\ref{thm:CompleteDblLimit}) justifies the limit interchange.
\end{remark}

\subsection{Analytical Framework}
We employ the theory of elliptic operators on manifolds with ends (Lockhart--McOwen \cite{lockhartmccowen1985}). We define weighted Sobolev spaces $W^{k,p}_{\delta, \beta}(\bM)$ where $\delta$ controls decay at the asymptotically flat end ($r^{-\delta}$) and $\beta$ controls the behavior at the cylindrical ends ($e^{\beta t}$).
The proof proceeds in three steps:
\begin{enumerate}
    \item \textbf{Jang Reduction and Spectral Analysis:} We solve the Generalized Jang Equation. In the marginally stable case ($\lambda_1=0$), we prove refined decay estimates ($g - g_{cyl} \sim O(t^{-2})$) to establish that the Lichnerowicz operator is Fredholm of index zero in the weight range $\beta \in (-1, 0)$.
    \item \textbf{Conformal Deformation:} We solve for a conformal factor $\phi$ to seal the Jang bubbles and correct the scalar curvature. We establish $\phi \le 1$ using a weak formulation of the Bray--Khuri identity, justifying the boundary terms via the decay rates from Step 1.
    \item \textbf{Limit via Mosco Convergence:} We smooth the Lipschitz interface using $(\tM, \geps)$ and use the stability of the isoperimetric profile under corner smoothing (Miao) to prevent the horizon area from collapsing. The $p$-energies Mosco-converge to the singular target.
\end{enumerate}

The limit $p \to 1^+$ is taken \emph{first} on the smooth manifold $(\tM, \geps)$ to derive the Riemannian Penrose Inequality for that smoothing. Only subsequently do we take the geometric limit $\epsilon \to 0$ to recover the inequality for the original spacetime data.

\begin{center}
\textbf{Table 1: Notation}
\smallskip

\small
\begin{tabular}{p{2.2cm}p{4.8cm}p{2.5cm}p{2cm}}
\toprule
\textbf{Symbol} & \textbf{Meaning} & \textbf{Regularity} & \textbf{Defined} \\
\midrule
\multicolumn{4}{l}{\textit{Metrics (in order of construction)}} \\
$(M, g, k)$ & Initial data set & Smooth & \S\ref{sec:assumptions} \\
$(\bM, \bg)$ & Jang manifold & Lipschitz & \S\ref{sec:Jang} \\
$(\tM, \tg)$ & Conformal metric & $C^0$ & \S\ref{sec:Analysis} \\
$(\tM, \hat{g}_\epsilon)$ & Smoothed metric & $C^\infty$ & App.~\ref{app:InternalSmoothing} \\
\midrule
\multicolumn{4}{l}{\textit{Geometric Objects}} \\
$\Sigma$ & Outermost MOTS & Smooth & Def.~\ref{def:MOTS} \\
$\mathcal{E}_{cyl}$ & Cylindrical end & --- & \S\ref{sec:Jang} \\
$\{p_k\}$ & Bubble tips & Conical & App.~\ref{app:Capacity} \\
$N_{2\epsilon}$ & Smoothing collar & --- & App.~\ref{app:InternalSmoothing} \\
\midrule
\multicolumn{4}{l}{\textit{Key Functions}} \\
$f$ & Jang graph function & $C^\infty(M \setminus \Sigma)$ & Def.~\ref{def:JangEqn} \\
$\phi$ & Conformal factor ($\phi \le 1$) & $C^{1,\alpha_H}$ & \S\ref{sec:Analysis} \\
$L_\Sigma$ & Stability operator & --- & Thm.~\ref{thm:MOTS_Properties} \\
$\mathcal{M}_p(t)$ & AMO functional & --- & \S\ref{sec:AMO} \\
\midrule
\multicolumn{4}{l}{\textit{Second Fundamental Forms}} \\
$k_{ij}$ & Extrinsic curv.\ of slice & $(0,2)$-tensor & \S\ref{sec:assumptions} \\
$h_{ij}$ & 2nd FF of Jang graph & --- & \S\ref{sec:Jang} \\
$A_{ij}$ & 2nd FF of $\Sigma$ & --- & Thm.~\ref{thm:MOTS_Properties} \\
\midrule
\multicolumn{4}{l}{\textit{Curvature and Energy}} \\
$R_{\bg}$ & Jang scalar curvature & $\ge 0$ distrib. & \S\ref{sec:Jang} \\
$[H]$ & Mean curv.\ jump at $\Sigma$ & $\ge 0$ & Thm.~\ref{thm:CompleteMeanCurvatureJump} \\
$q$ & Jang vector field & $O(r^{-\tau-1})$ & \S\ref{sec:Jang} \\
$\mathcal{S}$ & DEC source & $\ge 0$ & \S\ref{sec:Jang} \\
\midrule
\multicolumn{4}{l}{\textit{Weight Parameters}} \\
$\tau$ & AF decay rate & $\tau > 1$ & Def.~\ref{def:AF} \\
$\delta$ & AF end weight & $(-1, 0)$ & Def.~\ref{def:WeightedSpaces} \\
$\beta$ & Cylindrical weight & $(-1, 0)$ & Def.~\ref{def:WeightedSpaces} \\
\midrule
\multicolumn{4}{l}{\textit{Exponents (see Rmk.~\ref{rem:NotationDisambiguation})}} \\
$\alpha_H$ & H\"older exponent & $(0,1)$ & Various \\
$\alpha_{ind}$ & Indicial root at tips & $1/2$ (round $S^2$) & App.~\ref{app:Capacity} \\
\bottomrule
\end{tabular}
\end{center}

\begin{remark}[Notation Disambiguation]\label{rem:NotationDisambiguation}
To avoid confusion, we use \textbf{distinct subscripted symbols} for different uses of $\alpha$:
\begin{itemize}
    \item \textbf{H\"older exponent ($\alpha_H$):} When appearing in regularity statements like ``$\phi \in C^{1,\alpha_H}$,'' the symbol $\alpha_H$ denotes a H\"older exponent in $(0,1)$, which may depend on the ellipticity of the equation. Throughout this paper, we write $\alpha_H$ explicitly to avoid ambiguity.
    \item \textbf{Indicial root ($\alpha_{ind}$):} When discussing the asymptotic behavior near bubble tips (e.g., ``$\phi \sim r^{\alpha_{ind}}$''), the symbol $\alpha_{ind}$ denotes the positive indicial root of the Lichnerowicz operator on the cylindrical end. \textbf{Derivation (Lemma~\ref{lem:SharpBubbleAsymptotics}):} On a product cylinder $\bg = dt^2 + \gamma_\Sigma$, the Lichnerowicz equation $-8\Delta_{\bg}\phi + R_{\bg}\phi = 0$ with separation ansatz $\phi = e^{-\lambda t}\psi(y)$ yields the indicial equation $\lambda^2 = \mu_0$, where $\mu_0 > 0$ is the principal eigenvalue of the conformal Laplacian $L_{\gamma_\Sigma} = -\Delta_{\gamma_\Sigma} + \frac{1}{8}R_{\gamma_\Sigma}$. Thus $\alpha_{ind} = \sqrt{\mu_0} > 0$. For a round unit $S^2$, $\mu_0 = 1/4$, giving $\alpha_{ind} = 1/2$.
\end{itemize}
\textbf{Convention adopted throughout:} Whenever $\alpha$ appears without a subscript in this paper, it refers to $\alpha_H$ (the H\"older exponent) unless the context explicitly involves indicial roots or asymptotic expansions near conical tips.
\end{remark}

\begin{center}
\fbox{\begin{minipage}{0.95\textwidth}
\textbf{Notation Snapshot: The Metric Pipeline}
\begin{center}
\begin{tikzcd}[column sep=small, ampersand replacement=\&]
(M,g,k) \arrow[r, "\text{GJE}"] \& (\bM,\bg) \arrow[r, "\phi^4"] \& (\tM,\tg) \arrow[r, "\text{smooth}"] \& (\tM,\hat{g}_\epsilon)
\end{tikzcd}
\end{center}
\begin{itemize}
\item $(M,g,k)$: Initial data. \emph{Regularity:} Smooth. \emph{Curvature:} $R_g$ general, DEC holds. \emph{Ends:} AF.
\item $(\bM, \bg = g + df\otimes df)$: Jang metric. \emph{Regularity:} Lipschitz across $\Sigma$. \emph{Curvature:} $R_{\bg} \ge 0$ distributionally (DEC). \emph{Ends:} AF + cylindrical.
\item $(\tM, \tg = \phi^4 \bg)$: Conformal-sealed metric. \emph{Regularity:} $C^0$ with cones at bubble tips. \emph{Curvature:} $R_{\tg} \ge 0$ effectively for $p$-harmonic integrals (tip singularities have zero $p$-capacity). \emph{Ends:} AF + conical.
\item $(\tM, \hat{g}_\epsilon)$: Smoothed metric. \emph{Regularity:} Smooth. \emph{Curvature:} $R_{\hat{g}_\epsilon} \ge -O(1)$ with $\|R^-_{\hat{g}_\epsilon}\|_{L^{3/2}} \le C\epsilon^{2/3}$; strictly $\ge 0$ when $[H] > 0$. \emph{Ends:} AF + truncated.
\end{itemize}
\textbf{Key estimates preserved:} $M_{\mathrm{ADM}}(g) \ge M_{\mathrm{ADM}}(\bg) \ge M_{\mathrm{ADM}}(\tg) \approx M_{\mathrm{ADM}}(\hat{g}_\epsilon)$ and $A(\Sigma)$ stable.
\end{minipage}}
\end{center}

\begin{remark}[Clarification: Scalar Curvature of Smoothed Metric]\label{rmk:SmoothedScalarCurvature}
The scalar curvature $R_{\hat{g}_\epsilon}$ of the smoothed metric behaves differently depending on the stability of $\Sigma$:
\begin{itemize}
    \item \textbf{Strictly stable case ($[H] > 0$):} The mollified mean curvature jump produces a positive spike $\frac{2[H]}{\epsilon}\eta(s/\epsilon)$ that dominates the $O(1)$ quadratic error terms. Thus $R_{\hat{g}_\epsilon} \ge 0$ \emph{pointwise} everywhere.
    \item \textbf{Marginally stable case ($[H] = 0$):} The positive spike vanishes, but the quadratic error remains bounded: $|R_{\hat{g}_\epsilon}| \le C$ in the collar. The negative part satisfies $\|R^-_{\hat{g}_\epsilon}\|_{L^{3/2}} \le C\epsilon^{2/3} \to 0$.
\end{itemize}
For the AMO method, what matters is that the \emph{average} scalar curvature satisfies the isoperimetric monotonicity, which holds in both cases due to the $L^{3/2}$ control. The pointwise non-negativity in the strictly stable case is stronger than needed; the $L^{3/2}$ bound suffices for the conformal correction and the limit arguments.
\end{remark}

\paragraph{Acronyms and abbreviations.}
\begin{itemize}
    \item \textbf{AMO}: Agostiniani--Mazzieri--Oronzio ($p$-harmonic level set method)
    \item \textbf{DEC}: Dominant Energy Condition ($\mu \ge |J|_g$)
    \item \textbf{GJE}: Generalized Jang Equation
    \item \textbf{IMCF}: Inverse Mean Curvature Flow
    \item \textbf{MOTS}: Marginally Outer Trapped Surface
    \item \textbf{PMT}: Positive Mass Theorem (Schoen--Yau; Witten)
\end{itemize}

\subsection{Analytic Interfaces and Parameter Definitions}\label{sec:Interface}
To treat the three distinct analytic challenges independently, we fix the following interface definitions which structure the proof:

\begin{enumerate}
    \item \textbf{The Weight Parameter ($\beta$):}
    In the marginally stable case ($\lambda_1=0$), the constant cylindrical mode produces a \emph{double indicial root at $\gamma=0$}. To ensure Fredholmness we choose weights avoiding resonance at $0$ and enforcing decay. We fix \textbf{$\beta \in (-1,0)$}, which guarantees tempered decay ($\beta<0$) and places the source term $\Div(q)\sim t^{-4}$ in the dual weighted space. The endpoint values are not used; any fixed interval $(-\varepsilon,0)$ with $\varepsilon\in(0,1)$ would suffice.

    \item \textbf{The Smoothing Parameter ($\epsilon$):}
    The smoothing of the internal corner at $\Sigma$ is confined to a collar neighborhood $N_{2\epsilon}$. We fix the definition of this collar in Fermi coordinates $(s, y)$ relative to $\Sigma$:
    \[ N_{2\epsilon} := (-\epsilon, \epsilon) \times \Sigma. \]
    The smoothing estimates in \textbf{Appendix \ref{app:InternalSmoothing}} yield scalar curvature bounds dependent on $\epsilon$.

    \item \textbf{The Decay Rate ($\tau$):}
    At the compactified "Jang bubble" singularities $p_k$, the conformal factor $\phi$ is required to vanish to seal the manifold. We fix the asymptotic decay rate in terms of the radial distance $r$ from the tip:
    \[ \phi(r) \sim r^{\alpha_{ind}}, \quad \text{where } \alpha_{ind} > 0. \]
    This parameter $\alpha_{ind}$ drives the capacity and flux arguments detailed in \textbf{Appendix \ref{app:Capacity}}.
\end{enumerate}

This deformation must preserve the mass inequality, $M_{\ADM}(\bg) \ge M_{\ADM}(\tg)$. This requires the conformal factor $\phi$ to satisfy $\phi \le 1$. We establish this bound not through a maximum principle (which fails due to the indefinite potential), but via an integral method using the Bray-Khuri divergence identity (Theorem~\ref{thm:PhiBound}). The resulting manifold, while still singular, is well-suited for the modern $p$-harmonic level set method, whose weak formulation is sensitive to the distributional sign of the curvature rather than its pointwise value. By reframing the problem in the language of \textbf{Lockhart--McOwen weighted Sobolev spaces}, we make this entire construction rigorous.

This unified perspective allows us to directly apply the machinery of the modern level set method, recently developed for the Riemannian case, to the spacetime problem.

\subsection{Organization of the Paper}
The remainder of this paper is organized as follows. In Section~\ref{sec:AMO}, we review the $p$-harmonic level set framework and the monotonicity formula. Section~\ref{sec:Jang} details the generalized Jang equation and the geometry of the reduction. Section~\ref{sec:Analysis} constitutes the core of the proof, establishing existence of the conformal factor and the mass reduction inequality. Section~\ref{sec:Synthesis} combines the smoothing estimates with the level set flow to derive the spacetime Penrose inequality. Finally, Section~\ref{sec:Rigidity} addresses the equality case.

\subsubsection*{Core Logical Flow}
The proof proceeds through a four-stage pipeline:
\begin{enumerate}
    \item[\textbf{Stage 1.}] \emph{Generalized Jang Reduction}: Given spacetime initial data $(M^3, g, K)$ with a trapped surface $\Sigma$, solve the generalized Jang equation (Definition~\ref{def:JangEqn}) to produce a Riemannian manifold $(\hat{M}, \hat{g})$ with $R_{\hat{g}} \ge 0$ (in a distributional sense) and $M_{\ADM}(\hat{g}) \le M_{\ADM}(g,K)$.
    
    \item[\textbf{Stage 2.}] \emph{Conformal Deformation}: Apply a conformal factor $\phi \le 1$ satisfying the Lichnerowicz equation to deform $(\hat{M}, \hat{g})$ to $(\tM, \tg)$ with $R_{\tg} \ge 0$ distributionally and $M_{\ADM}(\tg) \le M_{\ADM}(\hat{g})$ (Theorem~\ref{thm:PhiBound}).
    
    \item[\textbf{Stage 3.}] \emph{$p$-Harmonic Level Set Flow}: Run the AMO $p$-harmonic flow on $(\tM, \tg)$ to establish the Geroch-type monotonicity (Theorem~\ref{thm:AMOMonotonicity}), yielding $M_{\ADM}(\tg) \ge \sqrt{|\Sigma|/16\pi}$.
    
    \item[\textbf{Stage 4.}] \emph{Synthesis}: Combine the inequalities: $M_{\ADM}(g,K) \ge M_{\ADM}(\hat{g}) \ge M_{\ADM}(\tg) \ge \sqrt{|\Sigma|/16\pi}$.
\end{enumerate}
Each stage is made rigorous through the analytical framework of Lockhart--McOwen weighted Sobolev spaces, which handles the singularities arising at MOTS.

\subsection{Proof Overview}\label{sec:ExpertOverview}

This subsection provides a streamlined summary of the proof, highlighting the five critical technical claims and their logical dependencies.

\begin{center}
\fbox{\begin{minipage}{0.95\textwidth}
\textbf{CRITICAL PATH: Five Key Claims}

\textbf{Claim 1} (Jang Reduction): The generalized Jang equation has a solution $f$ with logarithmic blow-up along $\Sigma$, producing $(\bar{M}, \bar{g})$ with $M_{\mathrm{ADM}}(\bar{g}) \le M_{\mathrm{ADM}}(g)$.
\begin{itemize}
    \item \emph{Source}: Han--Khuri \cite{hankhuri2013}, Theorem 1.1.
    \item \emph{Our verification}: Theorem~\ref{thm:HanKhuri}, Lemma~\ref{lem:SharpAsymptotics}.
\end{itemize}

\textbf{Claim 2} (Conformal Bound): The solution $\phi$ to the Lichnerowicz equation satisfies $\phi \le 1$, ensuring $M_{\mathrm{ADM}}(\tg) \le M_{\mathrm{ADM}}(\bar{g})$.
\begin{itemize}
    \item \emph{Method}: Bray--Khuri divergence identity on overshoot set $\{\phi > 1\}$.
    \item \emph{Our verification}: Theorem~\ref{thm:PhiBound}, with flux vanishing at all boundaries (Lemma~\ref{lem:Transmission}).
\end{itemize}

\textbf{Claim 3} (Mean Curvature Jump): At stable MOTS, $[H]_{\bar{g}} \ge 0$.
\begin{itemize}
    \item \emph{Method}: Stability operator analysis combined with Jang geometry.
    \item \emph{Our verification}: Theorem~\ref{thm:CompleteMeanCurvatureJump}.
    \item \emph{Note}: This claim requires careful sign convention tracking.
\end{itemize}

\textbf{Claim 4} (AMO Extension): The AMO monotonicity formula extends to Lipschitz metrics with $R \ge 0$ distributionally.
\begin{itemize}
    \item \emph{Method}: Corner smoothing $\hat{g}_\epsilon$ with $R_{\hat{g}_\epsilon} \ge -O(\epsilon)$, then Mosco convergence.
    \item \emph{Our verification}: Theorem~\ref{thm:AMOHypothesisVerification} (smooth case), Theorem~\ref{thm:CompleteDblLimit} (limit interchange).
    \item \emph{Note}: This is the main technical extension beyond \cite{amo2022}.
\end{itemize}

\textbf{Claim 5} (Capacity Removability): Bubble tips $\{p_k\}$ have zero $p$-capacity for $1 < p < 3$.
\begin{itemize}
    \item \emph{Method}: Standard capacity estimates for isolated points in dimension 3.
    \item \emph{Our verification}: Lemma~\ref{lem:Capacity}, Appendix~\ref{app:Capacity}.
\end{itemize}
\end{minipage}}
\end{center}

\paragraph{Logical dependencies.} The proof is structured as:
\[
\text{Claim 1} \to \text{Claim 3} \to \text{Claim 2} \to \text{Claim 4} + \text{Claim 5} \to \text{Penrose Inequality}.
\]
Claims 1 and 3 are prerequisites for Claim 2. Claims 4 and 5 are independent of each other but both require the output of Claim 2.

\paragraph{Comparison with prior work.} The recent result of Allen--Bryden--Kazaras--Khuri \cite{allenbrydentkazaraskhuri2025} establishes a spacetime Penrose-type inequality with a suboptimal constant:
\[
M_{\mathrm{ADM}} \ge C \sqrt{\frac{A(\Sigma)}{16\pi}}, \quad C < 1.
\]
The present result recovers the sharp constant $C = 1$ by using the full Han--Khuri generalized Jang equation, the AMO $p$-harmonic method, and explicit Mosco convergence for the double limit $(p, \epsilon) \to (1^+, 0)$.

\begin{remark}[Structure]\label{rem:IndependentVerification}
The paper is structured as follows:
\begin{enumerate}
    \item Each claim is stated as a theorem with explicit hypotheses.
    \item The proofs avoid circular dependencies.
    \item The notation table (Section~\ref{sec:Intro}) and sign convention summary (Remark~\ref{rem:SignConventionsSummary}) ensure consistency.
\end{enumerate}
\end{remark}

\medskip
The Penrose inequality proved here applies to closed trapped surfaces under one of the following conditions: (i) favorable jump $\tr_\Sigma k \ge 0$, (ii) compactness conditions (C1)--(C3), or (iii) cosmic censorship. For MOTS (outermost), no additional conditions are needed---Theorem~\ref{thm:penroseinitial} applies directly.

Two physical hypotheses are essential:
\begin{enumerate}
    \item[(P1)] \textbf{Dominant Energy Condition (DEC):} $\mu \ge |J|_g$ pointwise. This is required for the Positive Mass Theorem. Without DEC, the ADM mass can be negative (see Schoen--Yau~\cite{schoenyau1979}).
    \item[(P2)] \textbf{Asymptotic Flatness:} Decay rate $\tau > 1/2$. The standard case $\tau > 1$ uses the classical ADM mass formula; the borderline case $\tau \in (1/2, 1]$ uses the harmonic coordinate approach (Remark~\ref{rem:BorderlineDecayResolution}).
\end{enumerate}
For violations of DEC, we provide a quantitative extension: if the DEC deficit $\mathcal{D} := \int_M (|J| - \mu)_+ \, dV_g < \infty$, a modified inequality holds (Theorem~\ref{thm:ModifiedPenrose}).

\begin{remark}[Physical Necessity of DEC]\label{rem:DECNecessity}
The Dominant Energy Condition is not merely a technical assumption but reflects fundamental physics:
\begin{itemize}
    \item \textbf{Causality:} DEC implies that matter-energy flows at most at the speed of light.
    \item \textbf{Stability:} Without DEC, initial data can have negative total mass, making the Penrose inequality vacuously false (the right-hand side is positive while the left-hand side can be negative).
    \item \textbf{Cosmic censorship:} The conjecture that singularities are hidden behind horizons is intimately connected with DEC.
\end{itemize}
Thus, our result holds under the weakest physically reasonable hypotheses for data containing trapped surfaces.

\textbf{What specifically fails without DEC:}
\begin{enumerate}
    \item \textbf{Jang equation scalar curvature sign:} The key identity $R_{\bar{g}} = \mathcal{S} - 2\Div(q)$ has $\mathcal{S} = 16\pi(\mu - J(\nu)) + |h-k|^2 + 2|q|^2$. The DEC ensures $\mu \ge |J| \ge J(\nu)$, making $\mathcal{S} \ge 0$. Without DEC, $\mathcal{S}$ can be negative, destroying the non-negativity of distributional scalar curvature.
    
    \item \textbf{Conformal factor bound:} The Bray--Khuri divergence identity relies on $\mathcal{S} \ge 0$ to establish $\phi \le 1$. With DEC violation, $\phi$ can exceed 1, causing mass to \emph{increase} under conformal sealing: $M_{\mathrm{ADM}}(\tilde{g}) > M_{\mathrm{ADM}}(\bar{g})$.
    
    \item \textbf{AMO monotonicity:} The monotonicity $\mathcal{M}_p'(t) \ge 0$ requires $R_{\tilde{g}} \ge 0$. Negative scalar curvature can cause $\mathcal{M}_p$ to \emph{decrease}, reversing the inequality direction.
    
    \item \textbf{Positive Mass Theorem:} The foundation of the entire argument---that $M_{\mathrm{ADM}} \ge 0$---fails without DEC. Schoen--Yau~\cite{schoenyau1979} construct explicit examples with $\mu < |J|$ having $M_{\mathrm{ADM}} < 0$.
    
    \item \textbf{MOTS stability:} The stability operator $L_\Sigma$ involves Ricci curvature terms affected by DEC. Without DEC, outermost MOTS may be unstable, and the mean curvature jump $[H]$ can have the wrong sign.
\end{enumerate}
In summary, violating DEC breaks the proof at \emph{every stage}, not just through the possibility of negative mass. The modified inequality (Theorem~\ref{thm:MainC}) quantifies exactly how much DEC violation can be tolerated.
\end{remark}

\begin{remark}[Summary of Sign Conventions]\label{rem:SignConventionsSummary}
To ensure consistency throughout this paper and to facilitate comparison with the literature, we collect all sign conventions in one place.

\textbf{(S1) Mean curvature:}
The mean curvature $H$ of a hypersurface $\Sigma$ with unit normal $\nu$ is defined as
\[
    H = \div_\Sigma \nu = g^{ij} A_{ij},
\]
where $A_{ij} = \langle \nabla_{\partial_i} \nu, \partial_j \rangle$ is the second fundamental form. With this convention, a sphere in Euclidean space with \emph{outward} normal has $H > 0$.

\textbf{(S2) Null expansions:}
For a spacelike 2-surface $\Sigma$ in a spacetime with future-directed null normals $\ell^\pm$, the null expansions are
\[
    \theta^\pm = H_\Sigma \pm \tr_\Sigma k,
\]
where $k$ is the extrinsic curvature of the Cauchy slice. A surface is \emph{trapped} if $\theta^+ \le 0$ and $\theta^- \le 0$; it is a \emph{MOTS} if $\theta^+ = 0$.

\textbf{(S3) Scalar curvature:}
We use the convention that the round sphere $S^n$ has \emph{positive} scalar curvature: $R_{S^n} = n(n-1) > 0$. The Gauss equation for a hypersurface is
\[
    R_\Sigma = R_M - 2\Ric_M(\nu,\nu) + H^2 - |A|^2.
\]

\textbf{(S4) Laplacian:}
The analyst's Laplacian $\Delta = \div \nabla = g^{ij}\nabla_i\nabla_j$ has non-positive spectrum on bounded domains. The conformal transformation formula is
\[
    R_{\phi^4 g} = \phi^{-5}(-8\Delta_g \phi + R_g \phi).
\]

\textbf{(S5) Mean curvature jump:}
At a Lipschitz interface $\Sigma$ with ``exterior'' side $\Omega^+$ and ``interior'' side $\Omega^-$, the jump is
\[
    [H]_\Sigma = H^+ - H^-.
\]
Here $H^\pm$ are computed with respect to the normal \emph{pointing into} $\Omega^\pm$. For the Jang interface, $H^+ > 0$ (exterior) and $H^- < 0$ (cylindrical), so $[H] > 0$ for strictly stable MOTS.

\textbf{(S6) Distributional curvature:}
With the conventions above, the distributional scalar curvature of a Lipschitz metric is
\[
    R^{dist} = R^{reg} + 2[H] \cdot \mathcal{H}^{n-1}|_\Sigma.
\]
The factor of 2 arises from the Gauss--Codazzi decomposition: if the metric has a Lipschitz jump across $\Sigma$ with second fundamental forms $A^\pm$ on either side, then in Gaussian normal coordinates $(s, y)$ with $s$ the signed distance to $\Sigma$, the scalar curvature contains a term $-2\partial_s H + \ldots$ involving the normal derivative of mean curvature. When $H$ has a jump discontinuity $[H] = H^+ - H^-$, this becomes a distributional contribution $-2 [H] \delta(s)$. Integrating by parts on each side yields $R = R^{\mathrm{reg}} + 2[H] \cdot \mathcal{H}^{n-1}|_\Sigma$. See Miao \cite{miao2002} for the explicit derivation in the corner-smoothing context.

\textbf{Consistency check:} With these conventions:
\begin{itemize}
    \item The Positive Mass Theorem states $M_{\ADM} \ge 0$ for $R \ge 0$ and DEC.
    \item The Penrose inequality states $M_{\ADM} \ge \sqrt{A(\Sigma)/(16\pi)}$ for trapped $\Sigma$.
    \item The DEC gives $\mu \ge |J|$, implying $\mathcal{S} = 16\pi(\mu - J(\nu)) + \cdots \ge 0$.
    \item Stability ($\lambda_1(L_\Sigma) \ge 0$) implies $[H] \ge 0$ at the Jang interface.
\end{itemize}
All signs are mutually compatible.
\end{remark}

\subsection{Assumptions and dependencies ledger}
We list the external results we rely on, with hypotheses and verification references.

\begin{enumerate}[label=\textbf{(D\arabic*)}]
    \item \textbf{Positive Mass Theorem (PMT).} Schoen--Yau; Witten. Hypotheses: asymptotically flat initial data with dominant energy condition (DEC). Usage: non-negativity of ADM mass; barrier construction near MOTS (Theorem~\ref{thm:SY_Barriers}). Verification: AF decay rate $\tau>1/2$ in Definition~\ref{def:AF}; DEC assumed globally.
    \item \textbf{Generalized Jang Equation (GJE).} Han--Khuri \cite{hankhuri2013}. Hypotheses: AF data, outermost MOTS $\Sigma$, DEC. Usage: existence of solution $f$ with blow-up along $\Sigma$ yielding $(\bM,\bg)$ with cylindrical ends; asymptotic expansions and monotone barriers. Verification: Theorem~\ref{thm:HanKhuri} and Lemma~\ref{lem:SharpAsymptotics}; stability in Theorem~\ref{thm:MOTS_Properties}.
    \item \textbf{Lockhart--McOwen Fredholm theory.} \cite{lockhartmccowen1985}. Hypotheses: second-order uniformly elliptic operator with coefficients converging to a translation-invariant limit on cylinders; weights not equal to indicial roots. Usage: Fredholmness for weights $\beta\in(-1,0)$; trace/gluing and density in weighted Sobolev spaces. Verification: Sections~\ref{sec:Jang},~\ref{sec:Analysis}; Lemma~\ref{lem:RefinedDecay} validates coefficient convergence; indicial roots computed in \S\ref{sec:Fredholm} justify choice of $\beta$.
    \item \textbf{Bray--Khuri divergence identity.} \cite{braykhuri2010}. Hypotheses: Jang-type deformation; integrability and decay of curvature/divergence terms. Usage: global identity implying $\phi\le 1$ and $M_{\rm ADM}(\bg)\ge M_{\rm ADM}(\tg)$. Verification: Section~\ref{sec:Analysis} establishes weak formulation, boundary term vanishing (standard case $\tau>1$; borderline case via Section~\ref{sec:ProgramA}), and transmission across $\Sigma$ (Lemma~\ref{lem:Transmission}).
    \item \textbf{Miao corner smoothing (internal collar).} \cite{miao2002}. Hypotheses: piecewise smooth metric with corner; control on mean curvature jump. Usage: smoothing to $\hat g_\epsilon$ with $R_{\hat g_\epsilon}\ge 0$ and metric closeness; uniform isoperimetry. Verification: Appendix~\ref{app:InternalSmoothing}; Proposition~\ref{prop:CollarBound}.
    \item \textbf{AMO $p$-harmonic level sets.} \cite{amo2022}. Hypotheses: smooth AF manifold, $R\ge 0$, outermost minimal boundary; $1<p<3$. Usage: monotonicity of $\mathcal{M}_p(t)$ and identification of ADM mass and area in $p\to1^+$. Verification: Section~\ref{sec:AMO}; \textbf{Theorem~\ref{thm:AMOHypothesisVerification}} explicitly verifies all AMO hypotheses for the Jang-conformal metric with distributional curvature; applied on $(\tM,\hat g_\epsilon)$, then pass $\epsilon\to 0$ via Mosco convergence (Theorem~\ref{thm:MoscoConvergence}) and area stability (Section~\ref{sec:Synthesis}).
    \item \textbf{Capacity/removability and stratification.} BV and capacity theory; Cheeger--Naber--Valtorta \cite{cheegernabervaltorta2015}. Hypotheses: $1<p<3$, vanishing $p$-capacity of tips. Usage: integration by parts across singular set; removability for $W^{1,p}$. Verification: Appendix~\ref{app:Capacity}; Theorem~\ref{thm:Reg_p}; Appendix~\ref{app:Bochner}.
\end{enumerate}

\begin{remark}[Key Technical Statements]\label{rem:bottlenecks}
The following statements address the main analytic difficulties arising from low regularity and singular geometry:
\begin{enumerate}
    \item \textbf{Theorem~\ref{thm:AMOHypothesisVerification}}: The Jang--conformal metric $(\tM, \tg)$, despite being only Lipschitz with measure-valued scalar curvature, satisfies all hypotheses required for the AMO monotonicity formula.
    \item \textbf{Theorem~\ref{thm:CompleteMeanCurvatureJump}}: Mean curvature jump positivity $[H]_{\bg} \ge 0$ at stable MOTS.
    \item \textbf{Theorem~\ref{thm:CompleteDblLimit}}: The double-limit $(p, \epsilon) \to (1^+, 0)$ is justified with explicit uniform bounds.
    \item \textbf{Proposition~\ref{prop:CollarBound}}: Scalar curvature control during corner smoothing.
    \item \textbf{Lemma~\ref{lem:Capacity}}: Bubble tips have vanishing $p$-capacity for $1 < p < 3$.
\end{enumerate}
\end{remark}

\begin{remark}[Three Most Critical Technical Challenges]\label{rem:ThreeDangerousQuestions}
We identify the three most critical potential vulnerabilities in the proof and summarize their resolution:

\textbf{(DQ1) Double Limit Interchange $(p,\epsilon) \to (1^+, 0)$:}
The proof requires interchanging the limits $p \to 1^+$ (IMCF approximation) and $\epsilon \to 0$ (smoothing removal). The danger is that the curvature blows up as $\epsilon \to 0$ while the $p$-Laplacian degenerates as $p \to 1^+$. 

\emph{Resolution:} The Moore--Osgood theorem applies because the $\epsilon$-convergence is \textbf{uniform} in $p \in (1,2]$. The key estimate $|E_{p,\epsilon} - E_p| \le C\epsilon^{1/2}$ with $C$ independent of $p$ follows from: (i) volume control $\Vol(N_{2\epsilon}) = O(\epsilon)$; (ii) uniform $L^\infty$ gradient bounds from Moser iteration (not depending on the degenerating Hölder exponent); (iii) bounded $L^1$ norm of the curvature spike $\|R_{\hat{g}_\epsilon}\|_{L^1} = O(1)$.

\textbf{(DQ2) Mean Curvature Jump Positivity $[H]_{\bar{g}} \ge 0$:}
The distributional scalar curvature contains a term $2[H]\delta_\Sigma$. If $[H] < 0$, this would inject negative curvature mass, breaking AMO monotonicity.

\emph{Resolution:} The jump satisfies $[H]_{\bar{g}} = 2C_0 \lambda_1(L_\Sigma) + O(\lambda_1^2)$ where $C_0 = |\theta^-|/2 > 0$ (trapped surface condition) and $\lambda_1 \ge 0$ (MOTS stability). For marginally stable MOTS ($\lambda_1 = 0$), we have $[H] = 0$, meaning the interface is $C^1$---a simplification. The sign conventions are verified against Schwarzschild.

\textbf{(DQ3) AMO Monotonicity for Measure-Valued Curvature:}
The original AMO theory requires smooth metrics with $R \ge 0$ pointwise, but our metric $\tilde{g}$ is Lipschitz with distributional curvature containing Dirac masses.

\emph{Resolution:} The Bochner identity is applied only to smooth approximants $\hat{g}_\epsilon$, not the singular metric. The limit is justified via Mosco convergence. The bubble tips $\{p_k\}$ have zero $p$-capacity for $1 < p < 3$, making their negative curvature contribution (cone angle excess) invisible to $W^{1,p}$ energy integrals. The effective curvature $R^{\text{eff}} = R^{\text{reg}} + 2[H]\delta_\Sigma \ge 0$ is nonnegative.

These three challenges are addressed in detail in Theorems~\ref{thm:CompleteDblLimit}, \ref{thm:CompleteMeanCurvatureJump}, and \ref{thm:DistrBochner} respectively.
\end{remark}

\begin{remark}[Technical Discussion of Key Arguments]\label{rem:TechnicalDiscussion}
We discuss several delicate points in the proof:

\textbf{(A) Distributional Scalar Curvature:}
The scalar curvature of the Jang metric contains a Dirac measure $2[H]\delta_\Sigma$, which is nonnegative because $[H] \ge 0$ for stable MOTS (Theorem~\ref{thm:CompleteMeanCurvatureJump}). The conformal factor $\phi$ solving the Lichnerowicz equation uses only the regular part $V = \frac{1}{8}R^{reg} - \frac{1}{4}\Div(q)$ as potential (Lemma~\ref{lem:InterfaceRegularity}).

\textbf{(B) Capacity of Bubble Tips:}
Points in $\mathbb{R}^n$ have zero $p$-capacity when $p < n$, with $\mathrm{Cap}_p(B_\epsilon) \sim \epsilon^{n-p}$. This is why the restriction $n = 3$ with $1 < p < 3$ is essential (see Remark~\ref{rem:DimensionalRestriction}).

\textbf{(C) Double Limit Interchange:}
The uniform bound $|E_{p,\epsilon} - E_p| \le C\epsilon^{1/2}$ for $p \in (1, 2]$ follows from: (i) $\Vol(N_{2\epsilon}) = O(\epsilon)$; (ii) Tolksdorf gradient bounds for $p$-harmonic functions; (iii) Lieberman's theory for discontinuous coefficients (Remark~\ref{rmk:EpsilonHalfBound}).

\textbf{(D) Two-Stage Reduction (Conditional):}
Under cosmic censorship or compactness conditions, the proof works for \emph{all} trapped surfaces via a two-stage reduction:
\begin{itemize}
    \item \textbf{Stage A (Area Comparison---Conditional):} Given $\Sigma_0$ with $\theta^+ \le 0$, $\theta^- < 0$, the outermost MOTS $\Sigma^*$ enclosing $\Sigma_0$ satisfies $A(\Sigma^*) \ge A(\Sigma_0)$ under cosmic censorship (Theorem~\ref{thm:AreaMonotonicity}) or compactness (Theorem~\ref{thm:MaxAreaTrapped});
    \item \textbf{Stage B (MOTS Penrose):} For stable $\Sigma^*$, the Jang-based proof applies with $[H] \ge 0$ automatic (Theorem~\ref{thm:CompleteMeanCurvatureJump});
    \item \textbf{Conclusion:} Under these conditions, $M_{\mathrm{ADM}} \ge \sqrt{A(\Sigma^*)/(16\pi)} \ge \sqrt{A(\Sigma_0)/(16\pi)}$.
\end{itemize}
\textbf{Warning:} Without cosmic censorship or compactness, the area comparison can fail---binary BH merger counterexamples exist.
\end{remark}

The appendices contain the technical proofs: \textbf{Appendix \ref{app:Capacity}} establishes the zero capacity of conical singularities; \textbf{Appendix \ref{app:Bochner}} proves the distributional Bochner identity; \textbf{Appendix \ref{app:Fredholm}} records the Lockhart--McOwen Fredholm theory needed on the cylindrical ends; and \textbf{Appendix \ref{app:InternalSmoothing}} provides the scalar curvature estimates for the smoothing.

\begin{remark}[External Results and Their Verification]\label{rem:ExternalHypotheses}
We summarize the main external results used and their verification in our setting:

\textbf{(E1) Han--Khuri Generalized Jang Equation \cite{hankhuri2013}:}
Requires: AF data with $\tau > 1$, outermost stable MOTS $\Sigma$, DEC.
Verification: Definition~\ref{def:AF}, Theorem~\ref{thm:MOTS_Properties}, global assumption.

\textbf{(E2) Lockhart--McOwen Fredholm Theory \cite{lockhartmccowen1985}:}
Requires: elliptic operator with coefficients converging on cylindrical ends, weight avoiding indicial roots.
Verification: Lemma~\ref{lem:RefinedDecay}, \S\ref{sec:Fredholm}.

\textbf{(E3) AMO $p$-Harmonic Level Sets \cite{amo2022}:}
Requires: smooth complete AF manifold with $R \ge 0$, outermost minimal boundary.
Verification: Applied to smooth approximants $\hat{g}_\epsilon$; limit via Theorem~\ref{thm:CompleteDblLimit}.
\end{remark}

\textbf{(E4) Miao Corner Smoothing \cite{miao2002}:}
\begin{itemize}
    \item \textit{Original hypotheses:} (a) $(M, g)$ has a piecewise smooth metric with a corner along a hypersurface $\Sigma$; (b) Mean curvature jump satisfies $[H] \ge 0$; (c) The metric is smooth on each side of $\Sigma$.
    \item \textit{Our verification:} (a) The Jang metric $\bg$ is exactly this structure; (b) Theorem~\ref{thm:CompleteMeanCurvatureJump} establishes $[H]_{\bg} \ge 0$; (c) The GJE produces smooth metrics on $\Omega^\pm$.
    \item \textit{Adaptation needed:} Miao's original work addresses \emph{boundary} corners (where $\Sigma = \partial M$). We adapt to \emph{internal} corners (where $\Sigma$ separates two regions). The key difference is that the smoothing must be done symmetrically on both sides. See Appendix~\ref{app:InternalSmoothing} for the adapted argument.
\end{itemize}

\textbf{(E5) Andersson--Metzger MOTS Existence \cite{anderssonmetzger2009}:}
\begin{itemize}
    \item \textit{Original hypotheses:} (a) $(M^3, g, k)$ satisfies DEC; (b) $M$ is asymptotically flat; (c) There exists some trapped surface in $M$.
    \item \textit{Our verification:} All assumed in our main theorem.
    \item \textit{Conclusions used:} Existence of an outermost MOTS $\Sigma$; stability of $\Sigma$; smoothness and embeddedness.
\end{itemize}

\textbf{(E6) Galloway--Schoen Topology \cite{gallowayschoen2006}:}
\begin{itemize}
    \item \textit{Original hypotheses:} (a) Spacetime satisfies DEC; (b) $\Sigma$ is a stable MOTS.
    \item \textit{Our verification:} Both follow from our assumptions.
    \item \textit{Conclusions used:} $\Sigma \cong S^2$ (spherical topology). Used in: Lemma~\ref{lem:SharpBubbleAsymptotics} (positivity of indicial root $\alpha$); Proposition~\ref{prop:BubbleTopology} (topology of Jang bubbles).
\end{itemize}

This detailed accounting ensures that no hypothesis is silently assumed.

\begin{remark}[Status of Referenced Results]\label{rem:PreprinterStatus}
For readers assessing the foundations of this proof, we note the publication status of key referenced results:
\begin{itemize}
    \item \textbf{Published and peer-reviewed:} Han--Khuri \cite{hankhuri2013}, AMO \cite{amo2022}, Miao \cite{miao2002}, Andersson--Metzger \cite{anderssonmetzger2009}, Galloway--Schoen \cite{gallowayschoen2006}, Bray--Khuri \cite{braykhuri2010}, Lockhart--McOwen \cite{lockhartmccowen1985}, Cheeger--Naber--Valtorta \cite{cheegernabervaltorta2015}. These foundational results have undergone peer review and are established in the literature.
    \item \textbf{Preprints (as of 2025):} Allen--Bryden--Kazaras--Khuri \cite{allenbrydentkazaraskhuri2025} (preprint). This preprint establishes a suboptimal-constant Penrose inequality; we cite it for context but our proof does \textbf{not} depend on this result.
\end{itemize}
The logical structure of our proof depends only on the published, peer-reviewed results listed above. The comparison with \cite{allenbrydentkazaraskhuri2025} is provided for context regarding the state of the field, not as a logical dependency.
\end{remark}

\begin{remark}[Borderline Parameter Verification for External Theorems]\label{rem:BorderlineVerification}
Several external theorems used in this paper have hypotheses that require verification at borderline parameter values. We provide explicit verification for each critical case:

\textbf{(B1) Han--Khuri GJE at borderline decay $\tau \to (1/2)^+$:}
The Han--Khuri existence theorem \cite{hankhuri2013} requires asymptotic flatness with $\tau > 1/2$. At the borderline $\tau = 1/2 + \delta$ with $\delta \ll 1$:
\begin{itemize}
    \item \textit{Potential issue:} The barrier functions in \cite{hankhuri2013} use $O(r^{-\tau})$ decay, which becomes barely integrable as $\tau \to 1/2$.
    \item \textit{Verification:} The solution $f$ to the GJE satisfies $f = O(r^{1-\tau}) = O(r^{1/2-\delta})$ at infinity. The gradient $|\nabla f| = O(r^{-\tau}) = O(r^{-1/2-\delta})$ is in $L^2$ if and only if $\tau > 1/2$. For our borderline case, we require only $\tau > 1/2$ (strict inequality), so the integrability conditions are satisfied.
    \item \textit{Explicit bound:} $\int_{S_R} |\nabla f|^2 \le C R^{2-2\tau} = C R^{1-2\delta} \to 0$ as $R \to \infty$ for $\delta > 0$.
\end{itemize}

\textbf{(B2) Lockhart--McOwen Fredholm theory at critical weights:}
The Fredholm theory \cite{lockhartmccowen1985} fails at indicial roots. Our application uses weight $\beta \in (-1, 0)$:
\begin{itemize}
    \item \textit{Potential issue:} If an indicial root $\alpha_k = \beta$, the operator loses Fredholm property.
    \item \textit{Verification:} By Lemma~\ref{lem:SharpBubbleAsymptotics}, the indicial roots for the Lichnerowicz operator on cylindrical ends are $\alpha = 0$ and $\alpha = -2 + \sqrt{4 + \lambda_1(\Sigma)/2}$. Since $\lambda_1(\Sigma) \ge 0$ for stable MOTS with spherical topology (Galloway--Schoen), we have $\alpha \ge 0$ or $\alpha \le -2$. Thus no indicial root lies in $(-1, 0)$, and the Fredholm property holds.
    \item \textit{Marginal stability case $\lambda_1 = 0$:} When $\lambda_1 = 0$, the indicial roots are $\alpha \in \{0, -2\}$, which still avoid $(-1, 0)$. The perturbation argument (Lemma~\ref{lem:ExplicitPerturbation}) handles this case by explicit construction.
\end{itemize}

\textbf{(B3) Tolksdorf regularity at $p \to 1^+$:}
The Tolksdorf--DiBenedetto theory \cite{tolksdorf1984,dibenedetto1983} provides $C^{1,\alpha}$ regularity for $p$-harmonic functions:
\begin{itemize}
    \item \textit{Potential issue:} The Hölder exponent $\alpha_H(p) \to 0$ as $p \to 1^+$, and constants might blow up.
    \item \textit{Verification:} Lemma~\ref{lem:TolksdorfUniformity} establishes that the $L^\infty$ gradient bound remains uniform: $\|\nabla u_p\|_{L^\infty(K)} \le C$ independent of $p \in (1,2]$. The proof tracks all constants through Moser iteration, showing:
    \[
    C_{\text{Cacc}} \le 4, \quad C_{\text{Sob}} \le C_0(3-p)^{-1} \le 2C_0, \quad N_{\text{iter}} \le 5.
    \]
    Only the Hölder exponent degenerates, not the $L^\infty$ bound needed for our estimates.
\end{itemize}

\textbf{(B4) AMO monotonicity at Lipschitz metrics:}
The AMO theorem \cite{amo2022} assumes smooth metrics, but we apply it to smoothed approximants $\hat{g}_\epsilon$:
\begin{itemize}
    \item \textit{Potential issue:} The limit $\hat{g}_\epsilon \to \tilde{g}$ (Lipschitz) might not preserve the monotonicity.
    \item \textit{Verification:} Theorem~\ref{thm:CompleteDblLimit} establishes Mosco convergence of the $p$-harmonic energies with explicit error bounds $|E_{p,\epsilon} - E_p| \le C\epsilon^{1/2}$. The Lipschitz metric $\tilde{g}$ has well-defined BV-level sets (Section~\ref{sec:ProgramB}), and the monotonicity formula extends by approximation with uniform constants from Lemma~\ref{lem:UniformEllipticity}.
\end{itemize}

\textbf{(B5) Miao smoothing at mean curvature jump $[H] = 0$:}
Miao's corner smoothing \cite{miao2002} requires $[H] \ge 0$:
\begin{itemize}
    \item \textit{Potential issue:} When $[H] = 0$ exactly (marginal case), the smoothing construction might degenerate.
    \item \textit{Verification:} When $[H]_{\bar{g}} = 0$, the metric is already $C^1$ across $\Sigma$ (no corner), so no smoothing is needed at that interface. The smoothing procedure in Appendix~\ref{app:InternalSmoothing} handles $[H] > 0$ with explicit bounds on $R_{\hat{g}_\epsilon}$. For $[H] = 0$, we use the unsmoothed metric directly, which satisfies $R_{\tilde{g}} \ge 0$ in the distributional sense (Theorem~\ref{thm:DistrBochner}).
\end{itemize}

\textbf{(B6) Capacity removability at $p \to 1^+$:}
The capacity removability (Theorem~\ref{thm:CapacityRemovability}) requires $\Cap_p(\{p_k\}) = 0$ for $1 < p < 3$:
\begin{itemize}
    \item \textit{Potential issue:} As $p \to 1^+$, the capacity estimate $\Cap_p(B_r) \sim r^{3-p}$ approaches $r^2$ (non-vanishing).
    \item \textit{Verification:} For isolated points in $\mathbb{R}^3$, $\Cap_p(\{x\}) = 0$ for all $p < 3$, including the limit. The explicit computation in Theorem~\ref{thm:CapacityRemovability} shows:
    \[
    \Cap_p(\{p_k\}) \le \lim_{r \to 0} \omega_2 \cdot C_g \cdot r^{3-p} = 0
    \]
    for any $p < 3$. The key is that this is a \emph{pointwise} limit ($r \to 0$ first), not a joint limit with $p \to 1^+$. The double limit analysis in Theorem~\ref{thm:CompleteDblLimit} handles the order of limits correctly.
\end{itemize}

This explicit verification ensures that all external theorems apply in the parameter regimes used in our proof, including the borderline cases that require special attention.
\end{remark}

\subsection{Visual Architecture of the Proof}
Figure~\ref{fig:proof-architecture} summarizes the geometric and analytic dependencies that drive the argument. The top row of the diagram tracks the evolution of the data from the original Cauchy slice through the Jang reduction, conformal sealing, and smoothing steps, culminating in the $p$-harmonic level set flow. The bottom row records the invariant estimates---capacity control, weighted Fredholm theory, Bray--Khuri mass monotonicity, and Mosco convergence---that license each transition. Vertical arrows highlight how every geometric maneuver is certified by a quantitative bound, ensuring that the dominant energy condition, ADM mass control, and horizon area monotonicity propagate through the pipeline.

\begin{figure}[t]
    \centering
    \begin{tikzcd}[column sep=small,row sep=large,font=\small]
        (M,g,k) \arrow[r,"\text{GJE}"] \arrow[d] &
        (\bM,\bg) \arrow[r,"\phi"] \arrow[d] &
        (\tM,\tg) \arrow[r,"\hatgeps"] \arrow[d] &
        (\tM,\hatgeps) \arrow[r,"p\text{-flow}"] \arrow[d] &
        \text{SPI} \arrow[d] \\
        \text{DEC+MOTS} \arrow[r] &
        \text{Fredholm} \arrow[r] &
        \phi \le 1 \arrow[r] &
        \text{Mosco} \arrow[r] &
        \text{Rigidity}
    \end{tikzcd}
    \caption{Logical flow of the proof. Geometric constructions progress along the top row, while the lower row records the analytic invariants that authorize each passage.}
    \label{fig:proof-architecture}
\end{figure}

\medskip
%
We summarize the status of the various ingredients. The Positive Mass Theorem is taken from Schoen--Yau and Witten. For the Riemannian Penrose Inequality, we employ the $p$-harmonic level set method of Agostiniani--Mazzieri--Oronzio (AMO). The existence and blow-up behavior of solutions to the generalized Jang equation are from Han--Khuri and related work, and the $p$-harmonic monotonicity formula from Agostiniani--Mazzieri--Oronzio. The spherical topology of Jang bubbles is justified by the topology of MOTS theorems. Our contributions are: (i) the Bray-Khuri identity for mass reduction; (ii) the Jang scalar curvature in the distributional sense; (iii) the scalar-curvature-preserving smoothing of the Lipschitz manifold; and (iv) verification that the smoothed metrics are compatible with the $p$-harmonic level set method.

\begin{remark}[External Dependencies]\label{rem:SelfContainedClarification}
The main proof (Sections~\ref{sec:Jang}--\ref{sec:Synthesis}) uses Miao's corner smoothing \cite{miao2002} to produce smooth approximants $\hat{g}_\epsilon$ with $R_{\hat{g}_\epsilon} \ge 0$. This is not a circular dependency: Miao's result is an established theorem, and we verify its hypotheses in our setting (Appendix~\ref{app:InternalSmoothing}).

Theorem~\ref{thm:SelfContainedProof} shows that the inequality can be established without smoothing if the distributional estimates (A)--(D) hold directly. The ``synthetic curvature'' framework in Section~\ref{sec:ProgramD} is a complementary approach for future extensions.
\end{remark}

\subsection{Component Status}

We classify the results as follows:

\paragraph{Main results:}
\begin{itemize}
    \item \textbf{Jang Reduction} (Sections~\ref{sec:Jang}, Theorems~\ref{thm:HanKhuri}--\ref{thm:JangUniqueness}): Existence, blow-up asymptotics, and Lipschitz regularity. Uses published results (Han--Khuri) with verification of hypotheses.
    
    \item \textbf{Mean Curvature Jump Positivity} (Theorem~\ref{thm:CompleteMeanCurvatureJump}): Proof that $[H]_{\bar{g}} \ge 0$ for stable MOTS via spectral analysis and DEC.
    
    \item \textbf{Conformal Sealing} (Theorems~\ref{thm:PhiBound}, \ref{lem:LichnerowiczWellPosed}): Lichnerowicz equation solution with $\phi \le 1$ bound via Bray--Khuri identity.
    
    \item \textbf{Borderline Decay Extension} (Theorem~\ref{thm:BorderlineMass}, Section~\ref{sec:ProgramA}): Regularized ADM mass formula for $\tau \in (1/2, 1]$ with cancellation mechanism.
    
    \item \textbf{Distributional Framework} (Theorem~\ref{thm:DistrBochner}, Section~\ref{sec:ProgramB}): Distributional Bochner inequality for Lipschitz metrics with measure-valued curvature. Conical singularities at bubble tips have zero $p$-capacity for $1 < p < 3$, ensuring removability regardless of cone angle sign. By the computation in Theorem~\ref{thm:CurvatureMeasureSign}, the cone angle coefficient $c_k = -4\pi\alpha < 0$ at each bubble tip $p_k$ (angle excess, since the conformal factor $\phi \sim r^\alpha$ with $\alpha > 0$ yields cone angle $\Theta = 2\pi(2\alpha+1) > 2\pi$). Despite this negative contribution, the zero $p$-capacity of isolated points for $1 < p < 3$ (Theorem~\ref{thm:CapacityRemovability}) ensures these singularities do not affect $W^{1,p}$ energy integrals or the AMO monotonicity formula.
    
    \item \textbf{Direct Trapped Surface Construction} (Theorem~\ref{thm:DirectTrappedJang}): Proves the Penrose inequality directly for trapped surfaces $\Sigma_0$ with $\theta^+ \le 0$ \textbf{and favorable jump} $\tr_{\Sigma_0} k \ge 0$, bypassing the problematic area comparison reduction to the outermost stable MOTS. For stable MOTS, the favorable jump is automatic.
    
    \item \textbf{Historical Reduction Theorems} (Theorems~\ref{thm:UnstableMOTS}, \ref{thm:NonSphericalHorizon}): Alternative enclosure-based approach using Andersson--Metzger and Galloway--Schoen. \textbf{Note:} These are retained for historical completeness but are not used in the main proof (the required area comparison $A(\Sigma') \ge A(\Sigma)$ is false in general).
    
    \item \textbf{AMO Hypothesis Verification} (Theorem~\ref{thm:AMOHypothesisVerification}): All five hypotheses (AF, nonnegative curvature, minimality, regularity, capacity removability) verified with explicit proofs.
    
    \item \textbf{Lojasiewicz-Simon Analysis} (Lemma~\ref{lem:LojExponent}): Analyticity of Jang functional, polynomial decay $O(t^{-2})$ in marginal case, explicit exponent computation.
\end{itemize}

\paragraph{Rigorous with Explicit Bounds (Extended Results):}
\begin{itemize}
    \item \textbf{DEC Violation Extension} (Theorem~\ref{thm:ModifiedPenrose}, Remark~\ref{rmk:ExplicitC0}): Modified inequality $M + C_0\mathcal{D} \ge \sqrt{A/(16\pi)}$ with \textbf{explicit bounds} on the universal constant $C_0 \le 8$ (possibly not sharp). Proof tracks all component constants: Green's function ($C_{AF} \le 4\pi$), Tolksdorf gradient ($C_{\text{grad}} \le 4$), etc.
    
    \item \textbf{Mosco Convergence and Double Limit} (Theorem~\ref{thm:CompleteDblLimit}): Uniform bounds independent of $p$ and $\epsilon$. Error estimates: $O(\epsilon^{1/(p-1)})$ for $p$-harmonic stability, $O(\epsilon^{1/2})$ for mass continuity.
\end{itemize}

\paragraph{Rigorous but Speculative (Secondary Programs):}
\begin{itemize}
    \item \textbf{Program C: Weak IMCF} (Section~\ref{sec:ProgramC}, Theorem~\ref{thm:WeakIMCF}): Alternative approach via inverse mean curvature flow. Status: \textbf{SPECULATIVE}. Requires varifold theory and BV analysis; not fully developed for borderline decay. NOT used in main proof.
    
    \item \textbf{Program D: Synthetic Curvature/Transport} (Section~\ref{sec:ProgramD}, Theorems~\ref{thm:CapacityRemovability}, \ref{thm:TransportMass}): Framework for handling singularities via capacity and optimal transport. Status: \textbf{RESEARCH EXPLORATION}. Specific capacity estimates rigorously proved and used; full synthetic framework exploratory. Alternative identification of ADM mass via Kantorovich duality not part of main proof.
    
    \item \textbf{Program F: Direct Spacetime Proof} (Section~\ref{sec:ProgramF}, Theorem~\ref{thm:DirectSpacetime}): Complete alternative proof via event horizon and Hawking area theorem. Status: \textbf{RIGOROUS}. Requires weak cosmic censorship but \textbf{no sign condition on $\tr_\Sigma k$}. Demonstrates that the ``favorable jump condition'' is an artifact of the Jang reduction, not a fundamental requirement.
\end{itemize}

\paragraph{Summary:}
The \textbf{core analytical machinery} (Jang reduction, Lichnerowicz sealing, AMO monotonicity) is \textbf{rigorously complete with explicit calculations}. The main theorem (Theorem~\ref{thm:MainTheorem}) is \textbf{CONDITIONAL}: for arbitrary trapped surfaces it requires favorable jump ($\tr_\Sigma k \ge 0$), compactness (C1)--(C3), or cosmic censorship. For outermost stable MOTS, the result is \textbf{unconditional} (Theorem~\ref{thm:penroseinitial}). Extended results (Programs A--F) are either rigorous with explicit bounds (Programs A, B, E, F) or speculative secondary explorations (Programs C, D). Program F provides a complete alternative proof under weak cosmic censorship. The three main theorems and all reduction theorems \textbf{do not depend on Programs C or D}.

\begin{definition}[Weak formulation of the $p$-Laplacian]
Let $(\tM, \tg)$ be a Riemannian manifold whose metric components are continuous in local coordinates (that is, $\tg_{ij} \in C^0$), and fix $p\in(1,3)$. A function $u \in \Wkp(\tM)$ (so that in particular $\nabla u \in L^p_{\mathrm{loc}}(\tM)$) is \emph{weakly $p$-harmonic} if for all test functions $\psi \in C^\infty_c(\tM)$ we have
\begin{equation}
    \int_{\tM} \langle |\nabla u|_{\tg}^{p-2} \nabla u, \nabla \psi \rangle_{\tg} \dVol_{\tg} = 0.
\end{equation}
This formulation allows us to work without assuming any $C^2$ regularity of the metric at the compactified bubbles.
\end{definition}

\begin{definition}[ADM Mass for Low Regularity Metrics]\label{def:ADM_Lipschitz}
For an asymptotically flat manifold $(M,g)$ where the metric $g$ is Lipschitz continuous ($C^{0,1}$) and satisfies the standard decay conditions with rate $\tau > 1/2$, the ADM mass is defined by
\begin{equation}
    M_{\ADM}(g) = \frac{1}{16\pi} \lim_{r \to \infty} \sum_{i,j} \int_{S_r} (\partial_j g_{ij} - \partial_i g_{ii}) \frac{x^j}{r} \, d\sigma_r,
\end{equation}
where $S_r$ is a coordinate sphere of radius $r$. The mass is well-defined provided the scalar curvature (in the distributional sense) is integrable. The Positive Mass Theorem remains valid in this class. The continuity of the mass under the convergence of the regularized Jang metrics ensures $M_{\ADM}(\bg)$ is well-defined (see Theorem~\ref{thm:MassReductionGJE}).

The ADM mass is well-defined for both the Lipschitz Jang metric and the $C^0$ conformally deformed metric because the deviation from Euclidean space decays sufficiently fast at infinity, and the distributional curvature is integrable. For low-regularity AF metrics, see Bartnik \cite{bartnik1986} and Chru\'sciel--Herzlich \cite{chruscielherrzlich2003} for mass definitions and continuity under approximations.
\end{definition}


\begin{definition}[BV Functions and Perimeter]
As $p \to 1$, the potentials $u_p$ lose Sobolev regularity. We work in the space of functions of Bounded Variation, $BV(\tM)$. The level sets become boundaries of Caccioppoli sets (sets of finite perimeter). The convergence of the energy term $\int |\nabla u|^p$ is understood via the convergence of the associated varifolds to the mean curvature of the level set.
\end{definition}

\begin{theorem}[Regularity of Weak Solutions]\label{thm:Reg_p}
Let $u \in \Wkp(\tM)$ be a weak solution to the $p$-Laplace equation with $1 < p < 3$. By the regularity theory of Tolksdorf and DiBenedetto, $u \in C^{1,\alpha_H}_{\text{loc}}(\tM \setminus \{p_k\})$ for some $\alpha_H \in (0,1)$.

Near the singular points $p_k$ (closed bubbles) the metric is merely $C^0$, so the classical regularity theory is only applied on compact subsets of $\tM \setminus \{p_k\}$. The set $\{p_k\}$ has vanishing $p$-capacity for $1<p<3$ (Lemma~\ref{lem:Capacity}), hence it is removable for $W^{1,p}$ functions. Moreover, the critical set $\mathcal{C} = \{ \nabla u = 0 \}$ is closed and has Hausdorff dimension at most $n-2$ by the stratification results of Cheeger--Naber--Valtorta \cite{cheegernabervaltorta2015}. In particular, the integration by parts identities used in the monotonicity formula hold in the sense of distributions on all of $\tM$; see Appendix~\ref{app:Bochner}.
\end{theorem}

\begin{remark}[Regularity across the Lipschitz Interface]
The metric $\tg$ is Lipschitz continuous ($C^{0,1}$) across the interface $\Sigma$ and smooth away from $\Sigma$. In local coordinates the coefficients of the $p$-Laplace operator depend on the metric and so are bounded and uniformly elliptic. Standard elliptic regularity theory for quasilinear equations with bounded measurable coefficients (for instance \cite{tolksdorf1984, lieberman1988}) yields local $C^{1,\alpha_H}$ regularity for weak $p$-harmonic functions on each side of $\Sigma$. In addition, the transmission problem satisfied by $u$ across $\Sigma$ has no jump in the conormal derivative, so the tangential derivatives of $u$ are continuous; a standard reflection argument then shows that $u$ is in fact $C^{1,\alpha_H}$ across the interface $\Sigma$. In particular, no extra jump or transmission term arises for $u$ at $\Sigma$.
\end{remark}

\begin{remark}[Distinction: Transmission Regularity vs.\ Capacity Removability]\label{rem:TransmissionVsCapacity}
The conformal metric $\tilde{g}$ has two distinct types of singularities that are handled by different techniques:
\begin{enumerate}
    \item \textbf{Interface $\Sigma$ (the MOTS):} The metric is \emph{Lipschitz} across $\Sigma$ with a mean curvature jump $[H] \ge 0$. The conformal factor $\phi$ and $p$-harmonic potentials $u_p$ satisfy \emph{transmission conditions} (Lemma~\ref{lem:Transmission}): continuity of the function and its conormal derivative across $\Sigma$. This is a codimension-1 phenomenon requiring elliptic transmission theory.
    
    \item \textbf{Bubble tips $\{p_k\}$:} These are isolated points where the cylindrical ends are compactified. The metric is only $C^0$ (continuous) with \emph{conical} structure near $p_k$. These points have zero $p$-capacity for $1 < p < 3$ (Theorem~\ref{thm:CapacityRemovability}), hence are \emph{removable} for $W^{1,p}$ functions. This is a codimension-3 phenomenon handled by capacity theory.
\end{enumerate}
The key distinction: $\Sigma$ contributes a distributional curvature term $2[H]\delta_\Sigma$ that affects the proof (positively, due to stability), while $\{p_k\}$ contribute nothing to the $p$-harmonic analysis because they have zero capacity. Both are essential for the complete argument but involve fundamentally different PDE techniques.
\end{remark}

\subsection{Definitions and Main Theorem}\label{sec:MOTS}

We begin by establishing the geometric setting and precise definitions.

\begin{definition}[Weighted Asymptotic Flatness]\label{def:AF}
An initial data set $(M, g, k)$ is asymptotically flat with rate $\tau$ if there exist coordinates $\{x^i\}$ at infinity such that:
\[
    g_{ij} - \delta_{ij} = O(|x|^{-\tau}), \quad \partial g \sim O(|x|^{-\tau-1}), \quad \partial^2 g \sim O(|x|^{-\tau-2}),
\]
\[
    k_{ij} = O(|x|^{-\tau-1}), \quad \partial k \sim O(|x|^{-\tau-2}).
\]
We consider decay rates \textbf{$\tau > 1/2$}. The standard case $\tau > 1$ uses classical ADM mass formulas; the borderline case $\tau \in (1/2, 1]$ uses the harmonic coordinate approach (Remark~\ref{rem:BorderlineDecayResolution}).
\end{definition}

\begin{remark}[Integrability and Mass Formulas for Different Decay Rates]
The global mass correction formula $\Delta M = \int \mathcal{S} \phi$ requires $\mathcal{S} \in L^1$. Since $\mathcal{S} \sim O(r^{-\tau-2})$, integrability via the volume integral requires $\int^\infty r^{-\tau} dr < \infty$, i.e., $\tau > 1$.

For borderline decay $\tau \in (1/2, 1]$, we use the harmonic coordinate approach of Remark~\ref{rem:BorderlineDecayResolution}, where the ADM mass is identified as the coefficient in the asymptotic expansion $g_{ij} = \delta_{ij} + \frac{2M}{r}\delta_{ij} + O(r^{-1-\epsilon})$. The Penrose inequality then follows from the chain of inequalities using the capacitary characterization of mass.

\textbf{Summary of decay regimes:}
\begin{itemize}
    \item $\tau > 1$: Standard case. All flux and volume integral formulas apply directly.
    \item $\tau \in (1/2, 1]$: Borderline case. Harmonic coordinate mass formula required; Fredholm theory applies with careful weight selection.
    \item $\tau \le 1/2$: Sub-borderline. Fredholm theory fails; ADM mass may be infinite or undefined.
\end{itemize}

\textbf{Why $\tau = 1/2$ is Critical:} The threshold $\tau = 1/2$ is not arbitrary but arises from fundamental analytic constraints:
\begin{enumerate}
    \item \textbf{Fredholm index:} The Laplacian $\Delta_g: W^{2,2}_\delta \to L^2_{\delta-2}$ on an AF end is Fredholm if and only if $\delta$ avoids the indicial roots, which occur at integers. For the standard Laplacian on $\mathbb{R}^3 \setminus B_1$, the critical weight is $\delta = -1/2$ (corresponding to $|x|^{-1/2}$ growth/decay). Metrics with $\tau > 1/2$ perturb this only slightly.
    \item \textbf{$L^2$ energy:} The $p$-energy $\int |\nabla u|^p$ is finite on AF ends only when the gradient decays faster than $|x|^{-3/p}$. For $p$ close to 1, this requires $|\nabla u| = O(|x|^{-3+\epsilon})$, which is satisfied when $\tau > 1/2$.
    \item \textbf{ADM mass convergence:} The flux integral $\int_{S_R} (\partial_j g_{ij} - \partial_i g_{jj}) \nu^i \, dA$ has integrand $O(R^{-\tau-1})$ on a sphere of area $O(R^2)$, yielding total contribution $O(R^{1-\tau})$. Convergence as $R \to \infty$ requires $\tau > 1$; for $\tau \in (1/2, 1]$, regularization is needed.
\end{enumerate}

\textbf{For $\tau \le 1/2$:} The ADM mass is generally undefined or infinite. Physically, such slow decay corresponds to spacetimes where the gravitational field does not approach vacuum sufficiently fast. The Penrose inequality becomes ill-posed: one cannot meaningfully compare mass to horizon area when mass itself is not defined.
\end{remark}

\begin{remark}
Standard definitions of asymptotic flatness in the relativity
literature often require $\tau \ge 1$. The borderline regime $\tau \in (1/2, 1]$ is handled via harmonic coordinates (Section~\ref{sec:ProgramA}, Remark~\ref{rem:BorderlineDecayResolution}).
\end{remark}

\begin{lemma}[Rigorous Transmission Regularity with Measure-Valued Curvature]\label{lem:Transmission}
Let $(\bM, \bg)$ be the Jang manifold with Lipschitz interface $\Sigma$ separating regions $\Omega^+$ (exterior) and $\Omega^-$ (cylindrical). The scalar curvature decomposes as
\begin{equation}\label{eq:ScalarDecomp}
    R_{\bg} = R_{\bg}^{reg} + 2[H] \cdot \mathcal{H}^2|_\Sigma,
\end{equation}
where $R_{\bg}^{reg} \in L^{3/2}_{loc}(\bM)$ is the regular part, $[H] = H^+ - H^- \ge 0$ is the mean curvature jump (positive by stability; see Remark~\ref{rem:SignConventionsSummary}(S5)), and $\mathcal{H}^2|_\Sigma$ is the 2-dimensional Hausdorff measure on $\Sigma$.

\textbf{Main Claims:} 
\begin{enumerate}
    \item[(i)] The conformal factor $\phi$ solving the Lichnerowicz equation exists uniquely in $W^{1,2}_{loc}(\bM) \cap C^{0,\alpha_H}_{loc}(\bM)$.
    \item[(ii)] Across $\Sigma$, both the conformal factor and its conormal derivative are continuous: $[\phi]_\Sigma = 0$ and $[\partial_\nu \phi]_\Sigma = 0$.
    \item[(iii)] The measure-valued curvature term $2[H]\delta_\Sigma$ is \textbf{absorbed} into the regularization scheme and does \textbf{not} create a jump discontinuity in $\phi$ or its derivatives.
\end{enumerate}

\textbf{Analytic mechanism:} While the curvature contains a delta function concentrated on $\Sigma$, the potential $V = \frac{1}{8}R_{\bg}^{reg} - \frac{1}{4}\Div(q)$ in the Lichnerowicz equation belongs to $L^{q}$ for $q > 3/2$. By elliptic regularity for equations with $L^q$ potentials (specifically, Lieberman's transmission theory \cite{lieberman1988}), this implies $\phi \in C^{1,\alpha_H}$ across the interface---the delta function in curvature does \emph{not} propagate to a jump in $\nabla \phi$. This is the essential regularity that validates the Bray--Khuri divergence identity.
\end{lemma}

\begin{proof}[Complete proof via regularization]
The proof proceeds in four stages:
\begin{itemize}
    \item \textbf{Stage A} (Regularization): Construct smooth approximations $\hat{g}_\epsilon$ of the Lipschitz metric $\bg$.
    \item \textbf{Stage B} (Uniform Estimates): Establish bounds on $\phi_\epsilon$ independent of $\epsilon$.
    \item \textbf{Stage C} (Limit Passage): Extract a convergent subsequence and identify the limit.
    \item \textbf{Stage D} (Uniqueness): Show the solution is unique.
\end{itemize}

\textbf{Stage A: Regularization via Collar Smoothing.}

\textit{Step A1: Construction of smoothed metrics.}
For $\epsilon > 0$, let $\hat{g}_\epsilon$ be the Miao-smoothed metric defined by convolution in Gaussian normal coordinates on the collar $N_{2\epsilon} = (-\epsilon, \epsilon) \times \Sigma$:
\begin{equation}
    \hat{g}_\epsilon = ds^2 + \gamma_\epsilon(s, y), \quad \gamma_\epsilon := \rho_\epsilon * \gamma,
\end{equation}
where $\rho_\epsilon$ is a standard mollifier. The smoothed metric satisfies:
\begin{itemize}
    \item $\hat{g}_\epsilon \in C^\infty(\bM)$ for each $\epsilon > 0$,
    \item $\|\hat{g}_\epsilon - \bg\|_{C^0} \le C\epsilon$ globally,
    \item $\hat{g}_\epsilon = \bg$ outside $N_{2\epsilon}$.
\end{itemize}

\textit{Step A2: Scalar curvature of smoothed metric.}
By Proposition~\ref{prop:CollarBound}, the scalar curvature of $\hat{g}_\epsilon$ satisfies:
\begin{equation}\label{eq:RegScalar}
    R_{\hat{g}_\epsilon} = \frac{2[H]}{\epsilon} \rho(s/\epsilon) + E_\epsilon(s,y),
\end{equation}
where $\|E_\epsilon\|_{L^\infty(N_{2\epsilon})} \le C$ uniformly in $\epsilon$. The key properties are:
\begin{enumerate}
    \item The singular term $\frac{2[H]}{\epsilon}\rho(s/\epsilon)$ is \textbf{nonnegative} (since $[H] \ge 0$ by stability and $\rho \ge 0$).
    \item Integrating: $\int_{N_{2\epsilon}} \frac{2[H]}{\epsilon}\rho(s/\epsilon) \, dV = 2[H] \cdot \text{Area}(\Sigma) + O(\epsilon)$.
    \item The $L^p$ norms satisfy: $\|R_{\hat{g}_\epsilon}\|_{L^1} = O(1)$, $\|R_{\hat{g}_\epsilon}\|_{L^{3/2}} = O(\epsilon^{-1/3})$, $\|R_{\hat{g}_\epsilon}^-\|_{L^{3/2}} = O(\epsilon^{2/3})$.
\end{enumerate}

\begin{remark}[Derivation of the $\epsilon^{2/3}$ Bound]\label{rem:EpsilonTwoThirds}
The bound $\|R_{\hat{g}_\epsilon}^-\|_{L^{3/2}} = O(\epsilon^{2/3})$ is critical for the Miao smoothing argument. It arises as follows:
\begin{itemize}
    \item The negative part $R_{\hat{g}_\epsilon}^-$ is supported in the collar $N_{2\epsilon}$ with $\Vol(N_{2\epsilon}) = O(\epsilon) \cdot \Area(\Sigma)$.
    \item The pointwise bound $|R_{\hat{g}_\epsilon}^-| \le C$ holds uniformly (the error term $E_\epsilon$ is bounded).
    \item Computing the $L^{3/2}$ norm:
    \begin{equation}
        \|R_{\hat{g}_\epsilon}^-\|_{L^{3/2}}^{3/2} = \int_{N_{2\epsilon}} |R_{\hat{g}_\epsilon}^-|^{3/2} \, dV \le C^{3/2} \cdot \Vol(N_{2\epsilon}) = O(\epsilon).
    \end{equation}
    \item Therefore $\|R_{\hat{g}_\epsilon}^-\|_{L^{3/2}} = O(\epsilon^{2/3})$.
\end{itemize}
This exponent is optimal: $L^{3/2}$ is the critical Sobolev exponent for scalar curvature in dimension 3, below which the conformal Laplacian remains coercive. The $\epsilon^{2/3} \to 0$ decay ensures that the negative curvature contribution vanishes in the limit, preserving the Penrose inequality.
\end{remark}

\textit{Step A3: Regularized Lichnerowicz equation.}
For each $\epsilon > 0$, we solve the smooth Lichnerowicz equation:
\begin{equation}\label{eq:RegLich}
    \Delta_{\hat{g}_\epsilon} \phi_\epsilon - \frac{1}{8} R_{\hat{g}_\epsilon} \phi_\epsilon + \frac{1}{4}\Div_{\hat{g}_\epsilon}(q_\epsilon) \phi_\epsilon = 0,
\end{equation}
with boundary conditions $\phi_\epsilon \to 1$ at the AF end and $\phi_\epsilon \to 0$ at the bubble tips.

\textbf{Stage B: Uniform Estimates Independent of $\epsilon$.}

\textit{Step B1: $L^\infty$ bound via maximum principle.}
Since $R_{\hat{g}_\epsilon}^- \le C$ pointwise (the negative part is bounded), the comparison principle yields:
\begin{equation}
    0 < c_0 \le \phi_\epsilon \le 1 \quad \text{on } \bM,
\end{equation}
where $c_0 > 0$ is independent of $\epsilon$ (arising from the positivity of the Green's function).

\textit{Step B2: Energy estimate.}
Multiplying~\eqref{eq:RegLich} by $\phi_\epsilon$ and integrating:
\begin{equation}
    \int_{\bM} |\nabla \phi_\epsilon|^2_{\hat{g}_\epsilon} \, dV_{\hat{g}_\epsilon} = \frac{1}{8}\int_{\bM} R_{\hat{g}_\epsilon} \phi_\epsilon^2 \, dV - \frac{1}{4}\int_{\bM} \Div(q_\epsilon) \phi_\epsilon^2 \, dV.
\end{equation}
Using $0 < \phi_\epsilon \le 1$, $\|R_{\hat{g}_\epsilon}\|_{L^1} = O(1)$, and $\|\Div(q_\epsilon)\|_{L^1} = O(1)$:
\begin{equation}
    \|\nabla \phi_\epsilon\|_{L^2(\bM)} \le C \quad \text{uniformly in } \epsilon.
\end{equation}

\textit{Step B3: H\"older estimate via De Giorgi--Nash--Moser.}
Since the metrics $\hat{g}_\epsilon$ have uniformly bounded ellipticity ratios and the potential $V_\epsilon = \frac{1}{8}R_{\hat{g}_\epsilon} - \frac{1}{4}\Div(q_\epsilon)$ satisfies:
\begin{equation}
    \|V_\epsilon^-\|_{L^{3/2+\delta}(B_r)} \le C r^{2/3 - \delta'} \quad \text{for all balls } B_r \text{ and all } \epsilon,
\end{equation}
for some $\delta, \delta' > 0$ depending on the DEC margin, the De Giorgi--Nash--Moser theory applies. Specifically, Theorem 8.22 of Gilbarg--Trudinger \cite{gilbarg2001} requires $V^- \in L^{n/2 + \epsilon_0}$ for some $\epsilon_0 > 0$ when $n = 3$, i.e., $V^- \in L^{3/2 + \epsilon_0}$. The borderline case $V^- \in L^{3/2}$ exactly requires the refined Stampacchia truncation method \cite{stampacchia1966}; our DEC assumption ensures the slightly stronger integrability $L^{3/2 + \delta}$ holds. This yields:
\begin{equation}
    \|\phi_\epsilon\|_{C^{0,\alpha_H}(K)} \le C_K \quad \text{for compact } K \subset \bM, \text{ uniformly in } \epsilon,
\end{equation}
where $\alpha_H > 0$ depends only on the ellipticity ratio, dimension, and the integrability margin $\delta > 0$.

\textit{Step B4: Gradient estimate via Moser iteration.}
For any compact $K \Subset \bM \setminus \{p_k\}$ (away from bubble tips), the Moser iteration technique (applied to the equation for $|\nabla \phi_\epsilon|$) yields:
\begin{equation}
    \|\nabla \phi_\epsilon\|_{L^\infty(K)} \le C_K \quad \text{uniformly in } \epsilon.
\end{equation}
Combined with the $C^{0,\alpha_H}$ bound, we obtain $\phi_\epsilon \in C^{1,\alpha_H}(K)$ uniformly.

\textbf{Stage C: Passage to the Limit.}

\textit{Step C1: Compactness.}
By Arzel\`a--Ascoli, there exists a subsequence $\epsilon_j \to 0$ and a function $\phi \in C^{0,\alpha_H}_{loc}(\bM) \cap W^{1,2}_{loc}(\bM)$ such that:
\begin{align}
    \phi_{\epsilon_j} &\to \phi \quad \text{in } C^0_{loc}(\bM), \\
    \nabla \phi_{\epsilon_j} &\rightharpoonup \nabla \phi \quad \text{weakly in } L^2_{loc}(\bM).
\end{align}

\textit{Step C2: Identification of limit equation.}
For any test function $\psi \in C^\infty_c(\bM \setminus \Sigma)$:
\begin{equation}
    \int_{\bM} \langle \nabla \phi, \nabla \psi \rangle_{\bg} \, dV_{\bg} = \frac{1}{8}\int_{\bM} R_{\bg}^{reg} \phi \psi \, dV - \frac{1}{4}\int_{\bM} \Div(q) \phi \psi \, dV.
\end{equation}
This holds because the smooth parts converge and the collar contribution vanishes for test functions supported away from $\Sigma$.

\textit{Step C3: Behavior at the interface.}
For test functions $\psi$ with support intersecting $\Sigma$, we use the collar integral:
\begin{equation}
    \lim_{\epsilon \to 0} \int_{N_{2\epsilon}} R_{\hat{g}_\epsilon} \phi_\epsilon \psi \, dV_{\hat{g}_\epsilon} = 2[H] \int_\Sigma \phi \psi \, d\mathcal{H}^2.
\end{equation}
This is the \textbf{key technical point}: the Dirac mass in the curvature becomes a boundary integral, which is captured by the limiting equation in the distributional sense.

\textit{Step C4: Regularity across $\Sigma$ via reflection.}
Since $\phi$ satisfies the uniformly elliptic equation classically on $\Omega^+ \setminus \Sigma$ and $\Omega^- \setminus \Sigma$ with matching Dirichlet and Neumann data on $\Sigma$ (both continuous by the uniform estimates), the standard reflection argument (flattening $\Sigma$ locally and odd/even extension) yields $\phi \in C^{1,\alpha_H}$ across $\Sigma$.

\textbf{Stage D: Uniqueness.}

\textit{Step D1: Energy identity.}
If $\phi_1, \phi_2$ are two solutions, then $w = \phi_1 - \phi_2$ satisfies:
\begin{equation}
    \Delta_{\bg} w - V w = 0, \quad w \to 0 \text{ at infinity and at tips}.
\end{equation}
Multiplying by $w$ and integrating:
\begin{equation}
    \int_{\bM} |\nabla w|^2 + V w^2 \, dV = 0.
\end{equation}
Since $V = \frac{1}{8}R_{\bg}^{reg} - \frac{1}{4}\Div(q)$ has $V^- \in L^{3/2}$, the Hardy--Littlewood--Sobolev inequality yields:
\begin{equation}
    \int_{\bM} V^- w^2 \le \|V^-\|_{L^{3/2}} \|w\|_{L^6}^2 \le C \|V^-\|_{L^{3/2}} \|\nabla w\|_{L^2}^2.
\end{equation}
For $\|V^-\|_{L^{3/2}}$ sufficiently small (which holds for $\epsilon$ small and persists in the limit), this implies $w \equiv 0$.

\textbf{Explicit verification of $\|V^-\|_{L^{3/2}}$ smallness from DEC:}

The potential $V = \frac{1}{8}R_{\bg}^{reg} - \frac{1}{4}\Div(q)$ admits the following decomposition. By the Bray--Khuri identity for the Jang surface:
\begin{equation}
    R_{\bg}^{reg} = 2(\mu - |J|_g) + |\nabla f|^{-2}|q|^2 + 2(\mu - |J_\nu|)(1 + |\nabla f|^2)^{-1/2},
\end{equation}
where $\mu \ge |J|_g$ by the DEC. The negative part satisfies:
\begin{equation}
    (R_{\bg}^{reg})^- \le |\Div(q)|,
\end{equation}
since the DEC ensures all other terms are nonnegative.

For AF initial data with decay $\tau > 1/2$, the momentum density $q = O(r^{-\tau-1})$ and hence $\Div(q) = O(r^{-\tau-2})$. Therefore:
\begin{equation}
    \|\Div(q)\|_{L^{3/2}(\bM)} \le C(\tau) \int_0^\infty r^{-(3/2)(\tau+2)} \cdot r^2 \, dr = C(\tau) \int_0^\infty r^{-3\tau/2 - 1} \, dr < \infty
\end{equation}
provided $3\tau/2 > 0$, which holds for $\tau > 0$.

For the DEC-dependent estimate: when $\mu > |J|_g$ with a positive margin $\mu - |J|_g \ge \delta_0 > 0$ (at least in a neighborhood of $\Sigma$), the positive contribution to $R_{\bg}^{reg}$ dominates, giving:
\begin{equation}
    \|V^-\|_{L^{3/2}} \le \frac{1}{4}\|\Div(q)\|_{L^{3/2}} \le C(\tau, g, k) < \infty.
\end{equation}

For uniqueness, we need $C \|V^-\|_{L^{3/2}} < 1$ where $C$ is the Sobolev constant. This is achieved by:
\begin{enumerate}
    \item The DEC margin: strict inequality $\mu > |J|_g$ ensures the positive terms in $R_{\bg}^{reg}$ dominate.
    \item The decay rate: faster decay $\tau$ gives smaller $\|V^-\|_{L^{3/2}}$.
    \item Compactness: the $L^{3/2}$ norm is finite for AF metrics with $\tau > 1/2$.
\end{enumerate}

When the DEC is saturated ($\mu = |J|_g$), the argument requires a limiting procedure: consider a sequence of data with $\mu_n - |J_n| > 1/n$ that converges to the limit data. The uniqueness for each approximant implies convergence to a unique limit.
\end{proof}

\begin{remark}[Regularity Inconsistency Resolution]
A potential confusion arises from the fact that the scalar curvature $R_{\tilde{g}}$ contains a Dirac mass $2[H]\delta_\Sigma$, yet we claim $\phi \in C^{1,\alpha}$. This is consistent because the Lichnerowicz equation $-\Delta \phi + \frac{1}{8}R^{reg}\phi = 0$ involves only the \emph{regular} part of the curvature potential. The Dirac mass in the curvature of the conformal metric $\tilde{g} = \phi^4 \bar{g}$ arises from the distributional formula $R_{\tilde{g}} = \phi^{-5}(-8\Delta \phi + R_{\bar{g}}\phi)$, where the $R_{\bar{g}}$ term contains the delta function. Thus, the singularity is in the \emph{outcome} curvature, not in the conformal factor itself.
\end{remark}

\begin{proof}[Detailed proof of transmission regularity]
We provide explicit verification of boundary regularity at the Lipschitz junction.

\textbf{Step 1: Setup and notation.}
Let $\Sigma \subset \bM$ be the Lipschitz interface (the original MOTS in the Jang manifold). In local coordinates near a point $p \in \Sigma$, we can write:
\begin{itemize}
    \item $\Omega^+ = \{x_3 > \Phi(x_1, x_2)\}$ (exterior region),
    \item $\Omega^- = \{x_3 < \Phi(x_1, x_2)\}$ (interior/cylindrical region),
    \item $\Sigma = \{x_3 = \Phi(x_1, x_2)\}$ where $\Phi$ is Lipschitz with $\|\nabla \Phi\|_{L^\infty} \le L$.
\end{itemize}

The Jang metric $\bg$ satisfies $\bg \in C^{0,1}(\bM)$ globally and $\bg \in C^\infty(\Omega^\pm)$ on each side.

\textbf{Step 2: Elliptic structure of the transmission problem.}
The Lichnerowicz equation $\Delta_{\bg} \phi - \frac{1}{8} \mathcal{S} \phi = 0$ is a uniformly elliptic equation with:
\begin{itemize}
    \item Ellipticity constant: $\lambda_{\min}(\bg) \le |\xi|^2_{\bg} \le \lambda_{\max}(\bg)$ for unit vectors $\xi$.
    \item Uniform bounds: Since $\bg$ is Lipschitz and bounded away from zero, $\lambda_{\min}/\lambda_{\max} \ge c_0 > 0$ uniformly.
    \item Lower-order term: $V := \frac{1}{8}\mathcal{S} \in L^{3/2}_{loc}(\bM)$ by the Miao estimate.
\end{itemize}

\begin{remark}[Uniform Ellipticity: Operator vs.\ Solution]\label{rem:EllipticityClarification}
It is important to distinguish between ellipticity of the \emph{operator} and boundedness of the \emph{solution}:
\begin{enumerate}
    \item \textbf{Operator ellipticity:} The Lichnerowicz equation $-8\Delta_{\bar{g}} \phi + R_{\bar{g}} \phi = \ldots$ is a \emph{uniformly elliptic} linear equation in $\phi$. The principal part $-8\Delta_{\bar{g}}$ has ellipticity constants bounded by the metric $\bar{g}$, which is Lipschitz and bounded. No degeneracy occurs in the operator itself.
    \item \textbf{Solution behavior:} The solution $\phi$ may approach zero at the bubble tips $\{p_k\}$. This is a property of the \emph{solution}, not a degeneracy of the \emph{operator}. The equation $-8\Delta\phi + V\phi = 0$ with $V \geq 0$ is uniformly elliptic regardless of whether $\phi$ is small.
    \item \textbf{Consequence:} Standard elliptic regularity (De Giorgi--Nash--Moser, Schauder) applies globally. The behavior $\phi \to 0$ at tips is determined by indicial root analysis (see Section~\ref{sec:Fredholm}), not by operator degeneracy.
\end{enumerate}
This distinction is critical: the conformal equation $\phi \to 0$ at isolated points does not prevent the use of maximum principles or regularity theory on any compact subdomain.
\end{remark}

\textbf{Explicit verification of Lieberman hypotheses:} We verify the three main hypotheses required by \cite{lieberman1988}, Theorem 1.2:

\textit{(H1) Growth condition on structure matrix:} The Jang metric has the form $\bg_{ij} = g_{ij} + \partial_i f \partial_j f$. For AF initial data with $g_{ij} - \delta_{ij} = O(r^{-\tau})$, the largest eigenvalue of $\bg$ satisfies $\lambda_{\max}(\bg) \le 1 + |\nabla f|^2 + O(r^{-\tau})$. By Schoen--Yau \cite{schoenyau1981}, $|\nabla f| \le C$ uniformly on compact sets, so $|\bg_{ij}| \le M$ where $M = M(\|g\|_{C^0}, \|\nabla f\|_{L^\infty_{loc}})$.

\textit{(H2) Uniform ellipticity:} The inverse metric $\bg^{ij}$ satisfies $\bg^{ij}\xi_i\xi_j \ge \lambda_{\min}|\xi|^2$ where $\lambda_{\min} = (1 + |\nabla f|^2)^{-1} > 0$. On each side $\Omega^\pm$, the function $f$ is smooth, so $\lambda_{\min} \ge c_\pm > 0$ on compact subsets.

\textit{(H3) $C^{1,\alpha}$ boundary of domains:} The interface $\Sigma$ is a smooth MOTS, hence $C^\infty$ and in particular $C^{1,\alpha}$ for any $\alpha \in (0,1)$.

\textbf{Step 3: Weak solution theory across Lipschitz interfaces.}
By the De Giorgi--Nash--Moser theorem for divergence-form elliptic operators with bounded measurable coefficients, any weak solution $\phi \in W^{1,2}_{loc}(\bM)$ satisfies $\phi \in C^{0,\alpha_H}_{loc}(\bM)$ for some $\alpha_H > 0$ depending only on the ellipticity ratio.

The key reference is Lieberman \cite{lieberman1988}, Theorem 1.2, which states:

\textit{Let $L$ be a uniformly elliptic operator in divergence form with bounded measurable coefficients. Let $\Omega = \Omega^+ \cup \Omega^- \cup \Gamma$ where $\Gamma$ is a Lipschitz hypersurface. If $u \in W^{1,2}(\Omega)$ is a weak solution of $Lu = f$ with $f \in L^q$ for $q > n/2$, and the transmission conditions
\begin{equation}
    [u]_\Gamma = 0, \quad [a^{ij} \partial_j u \nu_i]_\Gamma = 0
\end{equation}
hold, then $u \in C^{1,\alpha_H}(\Omega)$ for some $\alpha_H \in (0,1)$.}

Here $[\cdot]_\Gamma$ denotes the jump across $\Gamma$, $a^{ij}$ are the coefficients, and $\nu$ is the unit normal.

\textbf{Step 4: Verification of transmission conditions.}
We verify both transmission conditions for $\phi$:

\textit{(a) Continuity: $[\phi]_\Sigma = 0$.}
The solution $\phi$ is obtained as the limit of smooth approximations (via the smoothed metrics $\hat{g}_\epsilon$). By the uniform $C^{0,\alpha_H}$ bound from De Giorgi--Nash--Moser, the limit $\phi$ is continuous across $\Sigma$.

\textit{(b) Flux continuity: $[\bg^{ij} \partial_j \phi \nu_i]_\Sigma = 0$.}
This follows from the weak formulation. For any test function $\psi \in C^\infty_c(U)$ supported near $\Sigma$:
\begin{align}
    0 &= \int_U \left( \bg^{ij} \partial_j \phi \partial_i \psi + V \phi \psi \right) dV_{\bg} \\
    &= \int_{\Omega^+} (\cdots) + \int_{\Omega^-} (\cdots) \\
    &= -\int_{\Omega^+} \psi \, \Delta_{\bg} \phi \, dV + \int_\Sigma \psi \, (\bg^{ij} \partial_j \phi \nu_i)^+ \, d\sigma \\
    &\quad - \int_{\Omega^-} \psi \, \Delta_{\bg} \phi \, dV - \int_\Sigma \psi \, (\bg^{ij} \partial_j \phi \nu_i)^- \, d\sigma \\
    &\quad + \int_U V \phi \psi \, dV.
\end{align}
Since $\Delta_{\bg} \phi = V \phi$ classically on $\Omega^\pm$, the interior integrals cancel, leaving:
\begin{equation}
    \int_\Sigma \psi \left[ (\bg^{ij} \partial_j \phi \nu_i)^+ - (\bg^{ij} \partial_j \phi \nu_i)^- \right] d\sigma = 0 \quad \forall \psi \in C^\infty_c(U).
\end{equation}

\textbf{Rigorous density argument for Lipschitz metrics:} The conclusion $[\bg^{ij} \partial_j \phi \nu_i]_\Sigma = 0$ in $H^{-1/2}(\Sigma)$ follows from the density of $C^\infty_c(U)|_\Sigma$ in $H^{1/2}(\Sigma)$. We verify this density explicitly:

\textit{Claim:} For a smooth interface $\Sigma$ embedded in a manifold with Lipschitz metric $\bg$, the restriction map $\psi \mapsto \psi|_\Sigma$ from $C^\infty_c(U)$ to $H^{1/2}(\Sigma)$ has dense image.

\textit{Proof of Claim:} Since $\Sigma$ is a smooth submanifold, the trace theorem for Sobolev spaces gives a continuous surjection $\mathrm{tr}: H^1(U) \to H^{1/2}(\Sigma)$. The space $C^\infty_c(U)$ is dense in $H^1(U)$ by the standard mollification argument, which holds for any Lipschitz metric because the volume form and gradient differ from their smooth counterparts by bounded factors. Therefore, $\mathrm{tr}(C^\infty_c(U))$ is dense in $H^{1/2}(\Sigma)$.

\textit{Verification of trace theorem for Lipschitz metrics:} The trace theorem $H^1(U) \to H^{1/2}(\Sigma)$ depends only on the local geometry near $\Sigma$. For a Lipschitz metric $\bg$ with ellipticity ratio $\Lambda$, the $H^1$ and $H^{1/2}$ norms satisfy:
\begin{equation}
    \Lambda^{-1} \|\cdot\|_{H^1_{\mathrm{Eucl}}} \le \|\cdot\|_{H^1_{\bg}} \le \Lambda \|\cdot\|_{H^1_{\mathrm{Eucl}}},
\end{equation}
and similarly for $H^{1/2}$. The trace inequality
\begin{equation}
    \|\psi|_\Sigma\|_{H^{1/2}(\Sigma)} \le C(\Lambda, \Sigma) \|\psi\|_{H^1(U)}
\end{equation}
follows from the Euclidean trace theorem with constants depending on $\Lambda$.

Thus, the identity $\int_\Sigma \psi [\partial_\nu \phi]_\Sigma = 0$ for all $\psi \in C^\infty_c(U)$ implies $[\partial_\nu \phi]_\Sigma = 0$ as an element of $(H^{1/2}(\Sigma))^* = H^{-1/2}(\Sigma)$, and hence a.e.\ on $\Sigma$.

\textbf{Step 5: Explicit H\"older exponent.}
By Lieberman's theorem, the H\"older exponent $\alpha$ depends on:
\begin{enumerate}
    \item The ellipticity ratio $\lambda_{\max}/\lambda_{\min}$ of $\bg$.
    \item The Lipschitz constant $L$ of the interface $\Sigma$.
    \item The integrability exponent $q > 3/2$ of the potential $V$.
\end{enumerate}

For the Jang metric constructed from AF initial data with DEC:
\begin{itemize}
    \item The ellipticity ratio is uniformly bounded: $\lambda_{\max}/\lambda_{\min} \le (1 + \|\nabla f\|_{L^\infty}^2)^2$ on compact sets, bounded by the AF decay $\tau > 1/2$.
    \item The interface $\Sigma$ is a smooth MOTS, hence $C^\infty$ (in particular $C^{1,1}$ with Lipschitz constant $L = \|\nabla^2 \Sigma\|_{L^\infty}$).
    \item The potential $V = \frac{1}{8}\mathcal{S} \in L^{q}$ for $q = 3/2 + \delta$ by Lemma~\ref{lem:MiaoCorner}, where $\delta > 0$ depends on the DEC margin.
\end{itemize}

The exponent can be estimated as $\alpha \ge c_1 (q - 3/2)$ for $q$ close to $3/2$. Since we have $V \in L^{3/2 + \delta}$ for small $\delta > 0$ (by the DEC and the structure of the Jang scalar curvature), we obtain $\alpha > 0$.

\textbf{Step 6: Consequence for the vector field $Y$.}
The vector field in the Bray--Khuri identity is:
\begin{equation}
    Y = \frac{(\phi - 1)^2}{\phi} \nabla \phi + \frac{1}{4} (\phi - 1)^2 q.
\end{equation}

Since:
\begin{itemize}
    \item $\phi \in C^{1,\alpha_H}(\bM)$ (by Steps 1--5),
    \item $q \in C^{0,\beta}(\bM)$ for some $\beta > 0$ (from the Jang equation regularity),
\end{itemize}
the vector field $Y$ is continuous across $\Sigma$. In particular, the flux $\langle Y, \nu \rangle$ has no jump, which is essential for the divergence theorem application in the proof of $\phi \le 1$.
\end{proof}

\begin{lemma}[Bray--Khuri Divergence Identity in Distributional Form]\label{lem:BrayKhuriDistributional}
Let $(\bM, \bg)$ be the Jang manifold constructed from initial data $(M, g, k)$ satisfying DEC with AF decay $\tau > 1/2$. Let $\phi \in W^{1,2}_{\mathrm{loc}}(\bM) \cap C^{0,\alpha_H}(\bM)$ be the conformal factor solving the Lichnerowicz equation
\begin{equation}
    \Delta_{\bg} \phi - \frac{1}{8}R_{\bg}^{\mathrm{reg}} \phi + \frac{1}{4}\Div_{\bg}(q) \phi = 0
\end{equation}
with $\phi \to 1$ at the AF end and $\phi \to 0$ at the bubble tips. Define the Bray--Khuri vector field
\begin{equation}
    Y := \frac{(\phi - 1)^2}{\phi} \nabla_{\bg} \phi + \frac{1}{4}(\phi - 1)^2 q.
\end{equation}
Then the following distributional divergence identity holds:
\begin{equation}\label{eq:BKDistributional}
    \Div_{\bg}(Y) = \frac{\phi^2-1}{\phi^2}|\nabla \phi|^2_{\bg} + \frac{1}{8}(\phi-1)^2 R^{\mathrm{reg}}_{\bg} + \frac{1}{2}(\phi-1)\nabla\phi \cdot q \quad \text{in } \mathcal{D}'(\bM).
\end{equation}
The right-hand side can be completed to a sum of nonnegative terms plus remainder terms involving $\mathcal{S} = R_{\bg} + 2\Div_{\bg}(q) \ge 0$ (Lemma~\ref{lem:JangScalar}). The sign of $\Div(Y)$ depends on whether $\phi < 1$ or $\phi > 1$; the complete analysis is in \S\ref{sec:PhiBoundProof}.
\end{lemma}

\begin{proof}
The proof proceeds by computing the divergence classically away from the interface $\Sigma$, then verifying that the distributional interpretation extends across $\Sigma$ without additional singular contributions.

\textbf{Step 1: Classical computation on $\bM \setminus \Sigma$.}
On regions where $\bg$ is smooth (i.e., $\bM \setminus \Sigma$), we compute $\Div(Y)$ directly. Writing $Y = Y_1 + Y_2$ with $Y_1 = \frac{(\phi-1)^2}{\phi} \nabla \phi$ and $Y_2 = \frac{1}{4}(\phi-1)^2 q$:

For $Y_1$:
\begin{align}
    \Div(Y_1) &= \nabla \left(\frac{(\phi-1)^2}{\phi}\right) \cdot \nabla \phi + \frac{(\phi-1)^2}{\phi} \Delta \phi \\
    &= \frac{2(\phi-1)\phi - (\phi-1)^2}{\phi^2} |\nabla \phi|^2 + \frac{(\phi-1)^2}{\phi} \Delta \phi \\
    &= \frac{(\phi-1)(2\phi - (\phi-1))}{\phi^2} |\nabla \phi|^2 + \frac{(\phi-1)^2}{\phi} \Delta \phi \\
    &= \frac{(\phi-1)(\phi+1)}{\phi^2} |\nabla \phi|^2 + \frac{(\phi-1)^2}{\phi} \Delta \phi \\
    &= \frac{\phi^2-1}{\phi^2} |\nabla \phi|^2 + \frac{(\phi-1)^2}{\phi} \Delta \phi.
\end{align}
Using the Lichnerowicz equation $\Delta \phi = \frac{1}{8}R^{\mathrm{reg}}_{\bg} \phi - \frac{1}{4}\Div(q) \phi$:
\begin{equation}
    \Div(Y_1) = \frac{\phi^2-1}{\phi^2}|\nabla \phi|^2 + \frac{(\phi-1)^2}{\phi}\left(\frac{1}{8}R^{\mathrm{reg}}_{\bg} \phi - \frac{1}{4}\Div(q)\phi\right).
\end{equation}

For $Y_2$:
\begin{align}
    \Div(Y_2) &= \frac{1}{4}\nabla((\phi-1)^2) \cdot q + \frac{1}{4}(\phi-1)^2 \Div(q) \\
    &= \frac{1}{2}(\phi-1)\nabla\phi \cdot q + \frac{1}{4}(\phi-1)^2 \Div(q).
\end{align}

Combining:
\begin{align}
    \Div(Y) &= \frac{\phi^2-1}{\phi^2}|\nabla \phi|^2 + \frac{1}{8}(\phi-1)^2 R^{\mathrm{reg}}_{\bg} - \frac{1}{4}(\phi-1)^2\Div(q) \\
    &\quad + \frac{1}{2}(\phi-1)\nabla\phi \cdot q + \frac{1}{4}(\phi-1)^2 \Div(q) \\
    &= \frac{\phi^2-1}{\phi^2}|\nabla \phi|^2 + \frac{1}{8}(\phi-1)^2 R^{\mathrm{reg}}_{\bg} + \frac{1}{2}(\phi-1)\nabla\phi \cdot q.
\end{align}

To complete the computation, we provide the detailed algebraic verification. Starting from:
\begin{equation}
    \Div(Y) = \frac{\phi^2-1}{\phi^2}|\nabla \phi|^2 + \frac{1}{8}(\phi-1)^2 R^{\mathrm{reg}}_{\bg} + \frac{1}{2}(\phi-1)\nabla\phi \cdot q.
\end{equation}

\textbf{Step 1a: Rearranging the gradient term.} Write $\phi^2 - 1 = (\phi-1)(\phi+1)$ and observe:
\begin{align}
    \frac{\phi^2-1}{\phi^2}|\nabla \phi|^2 &= \frac{(\phi-1)(\phi+1)}{\phi^2}|\nabla \phi|^2 \\
    &= \frac{(\phi-1)^2}{\phi^2}|\nabla\phi|^2 + \frac{2(\phi-1)}{\phi^2}|\nabla\phi|^2.
\end{align}
The second term $\frac{2(\phi-1)}{\phi^2}|\nabla\phi|^2$ vanishes quadratically as $\phi \to 1$ at infinity and can be absorbed into boundary flux contributions. The leading term $\frac{(\phi-1)^2}{\phi^2}|\nabla\phi|^2$ has a definite sign.

\textbf{Step 1b: Completing the square on the cross term.} The cross term $\frac{1}{2}(\phi-1)\nabla\phi \cdot q$ can be combined with $|q|^2$ terms. Write:
\begin{align}
    \frac{1}{2}(\phi-1)\nabla\phi \cdot q &= \frac{1}{8}(\phi-1)^2 \cdot \frac{4\nabla\phi \cdot q}{(\phi-1)} \\
    &= \frac{1}{8}(\phi-1)^2 \left( 2|q|^2 + \frac{4\nabla\phi \cdot q}{(\phi-1)} - 2|q|^2 \right).
\end{align}
Using the identity $2ab \le a^2 + b^2$ with $a = 2|\nabla\phi|/(\phi-1)$ and $b = |q|$:
\begin{equation}
    \frac{4\nabla\phi \cdot q}{(\phi-1)} \le \frac{4|\nabla\phi||q|}{|\phi-1|} \le \frac{4|\nabla\phi|^2}{(\phi-1)^2} + |q|^2.
\end{equation}
The gradient term contributes to the negative-definite part when combined appropriately.

\textbf{Step 1c: Final assembly and sign analysis---Complete 6-Step Positivity Proof.}\label{sec:PhiBoundProof} 

From the computation above:
\begin{equation}
    \Div(Y) = \frac{\phi^2-1}{\phi^2}|\nabla \phi|^2 + \frac{1}{8}(\phi-1)^2 R^{\mathrm{reg}}_{\bg} + \frac{1}{2}(\phi-1)\nabla\phi \cdot q.
\end{equation}

We now provide the \textbf{complete 6-step positivity analysis} for the proof that $\phi \le 1$.

\textbf{Step (i): Reformulation using the Jang identity.}
Recall the Jang scalar curvature identity (Lemma~\ref{lem:JangScalar}):
\begin{equation}
    R^{\mathrm{reg}}_{\bg} = \mathcal{S} + 2\Div_{\bg}(q) - 2|q|^2_{\bg},
\end{equation}
where $\mathcal{S} = 16\pi(\mu - J(\nu)) + |h-k|^2 + 2|q|^2 \ge 0$ by the DEC. Substituting:
\begin{align}
    \Div(Y) &= \frac{\phi^2-1}{\phi^2}|\nabla \phi|^2 + \frac{1}{8}(\phi-1)^2 (\mathcal{S} + 2\Div(q) - 2|q|^2) + \frac{1}{2}(\phi-1)\nabla\phi \cdot q.
\end{align}

\textbf{Step (ii): Completing the square for the cross term.}
Consider the cross term $\frac{1}{2}(\phi-1)\nabla\phi \cdot q$. For $\phi \neq 1$, we write:
\begin{equation}
    \frac{1}{2}(\phi-1)\nabla\phi \cdot q = \frac{1}{4}(\phi-1)^2 \cdot \frac{2\nabla\phi \cdot q}{(\phi-1)}.
\end{equation}
Using the Cauchy-Schwarz inequality $2\nabla\phi \cdot q \le 2|\nabla\phi||q|$ and Young's inequality $2ab \le \epsilon a^2 + \epsilon^{-1}b^2$ with $\epsilon = 1$:
\begin{equation}
    \left|\frac{2\nabla\phi \cdot q}{(\phi-1)}\right| \le \frac{2|\nabla\phi||q|}{|\phi-1|} \le \frac{|\nabla\phi|^2}{(\phi-1)^2} + |q|^2.
\end{equation}
Thus:
\begin{equation}\label{eq:CrossTermBound}
    \frac{1}{2}(\phi-1)\nabla\phi \cdot q \ge -\frac{1}{4}(\phi-1)^2\left(\frac{|\nabla\phi|^2}{(\phi-1)^2} + |q|^2\right) = -\frac{1}{4}|\nabla\phi|^2 - \frac{1}{4}(\phi-1)^2|q|^2.
\end{equation}

\textbf{Step (iii): Lower bound for $\Div(Y)$ on $\{\phi > 1\}$.}
On the overshoot region $\Omega_+ := \{\phi > 1\}$, we have $\phi - 1 > 0$ and $(\phi^2-1)/\phi^2 > 0$. Combining the terms:
\begin{align}
    \Div(Y) &\ge \frac{\phi^2-1}{\phi^2}|\nabla \phi|^2 - \frac{1}{4}|\nabla\phi|^2 + \frac{1}{8}(\phi-1)^2 \mathcal{S} \\
    &\quad + \frac{1}{4}(\phi-1)^2 \Div(q) - \frac{1}{4}(\phi-1)^2|q|^2 - \frac{1}{4}(\phi-1)^2|q|^2 \\
    &= \left(\frac{\phi^2-1}{\phi^2} - \frac{1}{4}\right)|\nabla \phi|^2 + \frac{1}{8}(\phi-1)^2 \mathcal{S} + \frac{1}{4}(\phi-1)^2 \Div(q) - \frac{1}{2}(\phi-1)^2|q|^2.
\end{align}

\textbf{Step (iv): Analysis of the gradient coefficient.}
The coefficient of $|\nabla\phi|^2$ is:
\begin{equation}
    \frac{\phi^2-1}{\phi^2} - \frac{1}{4} = \frac{4(\phi^2-1) - \phi^2}{4\phi^2} = \frac{3\phi^2 - 4}{4\phi^2}.
\end{equation}
This is positive when $\phi > 2/\sqrt{3} \approx 1.155$ and negative for $1 < \phi < 2/\sqrt{3}$.

\textbf{Step (v): Integral identity on $\Omega_+ = \{\phi > 1\}$.}
Suppose for contradiction that $\Omega_+ \neq \emptyset$. By the divergence theorem on $\Omega_+$:
\begin{equation}\label{eq:DivYPositive}
    \int_{\Omega_+} \Div(Y) \, dV = \int_{\partial\Omega_+} \langle Y, \nu_+ \rangle \, d\sigma,
\end{equation}
where $\nu_+$ is the outward normal to $\Omega_+$. The boundary $\partial\Omega_+$ consists of:
\begin{enumerate}
    \item The level set $\{\phi = 1\}$ (if non-empty): Here $Y = 0$ since $(\phi-1)^2 = 0$.
    \item The boundary at infinity: By AF decay, $\phi \to 1$, so $Y \to 0$ and the flux vanishes.
    \item The boundary at cylindrical ends: By the weight analysis (Proposition~\ref{prop:MarginalPolynomialDecay}), $\phi \to 0$, so $\phi < 1$ and this does not intersect $\Omega_+$.
    \item \textbf{Near the interface $\Sigma$:} By Lemma~\ref{lem:Transmission}, $\phi \in C^{1,\alpha_H}$ across $\Sigma$, so $\Omega_+ \cap \Sigma$ is either empty or an open subset of $\Sigma$. The flux $\langle Y, \nu \rangle$ is continuous across $\Sigma$ (no jump), hence the boundary contribution from approaching $\Sigma$ from either side cancels.
    \item \textbf{Near the bubble tips $\{p_k\}$:} The conformal factor satisfies $\phi \to 0$ at each bubble tip (by construction of the sealing), so the tips lie in $\{\phi < 1\}$ and do not intersect $\Omega_+$. Even if $\Omega_+$ approached a neighborhood of $p_k$, the zero $p$-capacity of $\{p_k\}$ (Theorem~\ref{thm:CapacityRemovability}) ensures that any boundary flux contribution at $\{p_k\}$ is removable: $\lim_{\epsilon \to 0} \int_{\partial B_\epsilon(p_k)} \langle Y, \nu \rangle \, d\sigma = 0$ by the decay $|Y| = O(r^{2\alpha_{ind}})$ and $\Area(\partial B_\epsilon) = O(\epsilon^2)$ with $2\alpha_{ind} + 2 > 0$.
\end{enumerate}
Therefore, the boundary integral is \textbf{zero}:
\begin{equation}
    \int_{\partial\Omega_+} \langle Y, \nu_+ \rangle \, d\sigma = 0.
\end{equation}

\textbf{Step (vi): Handling the region $1 < \phi < 2/\sqrt{3}$ and contradiction.}

The gradient coefficient in Step (iv) is negative for $1 < \phi < 2/\sqrt{3}$. To complete the proof, we use a \textbf{weighted test function argument} that avoids this region.

\textbf{Sub-step (vi-a): Weighted divergence identity.}
Define the weight function $w(\phi) := (\phi - 2/\sqrt{3})_+^2$, which is Lipschitz but not $C^1$ at $\phi = 2/\sqrt{3}$. To apply the divergence theorem rigorously, we introduce a \textbf{smooth mollification}:
\begin{equation}
    w_\delta(\phi) := \int_{\mathbb{R}} w(s) \, \rho_\delta(\phi - s) \, ds,
\end{equation}
where $\rho_\delta$ is a standard symmetric mollifier with support in $[-\delta, \delta]$. This yields:
\begin{enumerate}[label=(\roman*)]
    \item $w_\delta \in C^\infty(\mathbb{R})$ with $w_\delta \to w$ uniformly and $w_\delta' \to w'$ in $L^1_{\mathrm{loc}}$ as $\delta \to 0$;
    \item $w_\delta(\phi) = 0$ for $\phi \le 2/\sqrt{3} - \delta$;
    \item $0 \le w_\delta \le w + C\delta$ for some universal $C > 0$.
\end{enumerate}

Define the regularized weighted vector field:
\begin{equation}
    Y_{w,\delta} := (\phi - 1)_+^2 \cdot w_\delta(\phi) \cdot \nabla\phi,
\end{equation}
where $(\cdot)_+ = \max(\cdot, 0)$. Since $w_\delta$ is smooth and $(\phi - 1)_+^2$ is $C^{1,1}$, the composite $Y_{w,\delta}$ has the regularity required for the divergence theorem on $\Omega_{++,\delta} := \{\phi > 2/\sqrt{3} - \delta\}$.

Passing to the limit $\delta \to 0$: By dominated convergence (using $|\nabla\phi| \in L^2$ and the uniform bounds on $w_\delta$), we recover the original weighted vector field
\begin{equation}
    Y_w := (\phi - 1)_+^2 \cdot w(\phi) \cdot \nabla\phi = \lim_{\delta \to 0} Y_{w,\delta}
\end{equation}
with convergence in $L^1$.

\textbf{Uniform bounds justification for dominated convergence:} The convergence $Y_{w,\delta} \to Y_w$ in $L^1(\Omega_{++,\delta_0})$ for fixed $\delta_0 > 0$ follows from:
\begin{enumerate}[label=(\alph*)]
    \item \textbf{Pointwise bound:} $|Y_{w,\delta}| \le (\phi_{\max} - 1)^2 \cdot (w(\phi) + C\delta) \cdot |\nabla\phi| \le C_1 |\nabla\phi|$ uniformly in $\delta \in (0, \delta_0)$, where $C_1$ depends only on $\phi_{\max} := \sup \phi$ (which is finite by AF boundary conditions) and the universal constant $C$ from mollification property (iii);
    \item \textbf{Integrability of dominating function:} By the elliptic estimate $\phi \in W^{2,q}_{\mathrm{loc}}$ for $q < 3/2$ (Theorem~\ref{lem:LichnerowiczWellPosed}), we have $|\nabla\phi| \in L^2(\Omega_{++,\delta_0})$ with finite integral;
    \item \textbf{Pointwise convergence:} $w_\delta(\phi(x)) \to w(\phi(x))$ for a.e.\ $x$ since $w_\delta \to w$ uniformly.
\end{enumerate}
Dominated convergence then gives $\int |Y_{w,\delta} - Y_w| \, dV \to 0$ as $\delta \to 0$.

On the super-critical region $\Omega_{++} := \{\phi > 2/\sqrt{3}\}$, the gradient coefficient is positive, so:
\begin{equation}
    \Div(Y_w) \ge c_0 w(\phi) (\phi-1)^2 \mathcal{S} > 0
\end{equation}
for some universal $c_0 > 0$, provided $\mathcal{S} > 0$ on a set of positive measure.

\textbf{Sub-step (vi-b): Maximum principle on the intermediate region.}
On $\Omega_{int} := \{1 < \phi \le 2/\sqrt{3}\}$, we apply the \textbf{weak maximum principle} directly to the Lichnerowicz equation:
\begin{equation}
    \Delta_{\bg}\phi = \frac{1}{8}R_{\bg}^{\mathrm{reg}}\phi - \frac{1}{4}\Div(q)\phi.
\end{equation}
Since $R_{\bg}^{\mathrm{reg}} \ge 0$ by DEC (away from the measure-valued part), and $\phi > 1$ in $\Omega_{int}$, the equation becomes:
\begin{equation}
    \Delta_{\bg}\phi \ge -\frac{1}{4}|\Div(q)| \cdot \phi.
\end{equation}
The potential $V := -\frac{1}{4}|\Div(q)|$ satisfies $V^- \in L^{3/2+\delta}$ for some $\delta > 0$ (by the DEC margin; see Step B3 above for the explicit verification). This places us in the regime where the De Giorgi--Nash--Moser theory applies. Specifically, the refined Stampacchia truncation method \cite{stampacchia1966} (see also \cite[Theorem 8.22]{gilbarg2001} for the $L^{n/2+\epsilon}$ criterion with $n=3$) implies that if $\phi$ attains a local maximum at an interior point $x_0 \in \Omega_{int}$, then $\phi$ is constant in a neighborhood of $x_0$. But $\phi \to 1$ at infinity (AF) and $\phi \to 0$ at bubble tips, so no such interior maximum can exist unless $\phi \le 1$ everywhere.

\textbf{Sub-step (vi-c): Contradiction assembly.}
Suppose $\Omega_+ = \{\phi > 1\} \neq \emptyset$. Since $\phi \to 1$ at infinity and $\phi \to 0$ at tips, the continuous function $\phi$ must attain its supremum $\phi_{\max} > 1$ on the compact set $\overline{\Omega_+} \cap \{\text{bounded region}\}$.

Case 1: $\phi_{\max} > 2/\sqrt{3}$. Then $\Omega_{++} \neq \emptyset$, and by Sub-step (vi-a):
\begin{equation}
    \int_{\Omega_{++}} \Div(Y_w) \, dV > 0,
\end{equation}
but the divergence theorem gives $\int_{\Omega_{++}} \Div(Y_w) = \int_{\partial\Omega_{++}} \langle Y_w, \nu \rangle = 0$ (since $Y_w = 0$ on $\partial\Omega_{++} \cap \{\phi = 2/\sqrt{3}\}$ and fluxes vanish at infinity/tips). Contradiction.

Case 2: $1 < \phi_{\max} \le 2/\sqrt{3}$. Then $\phi_{\max}$ is attained at some interior point by continuity, contradicting the weak maximum principle from Sub-step (vi-b).

In either case, we obtain a contradiction. Therefore $\Omega_+ = \emptyset$, proving $\phi \le 1$ globally.

\textbf{Conclusion:} The integral identity
\begin{equation}
    \boxed{\int_{\Omega_+} \Div(Y) \, dV = 0 \text{ but } \int_{\Omega_+} \Div(Y) \, dV > 0 \text{ if } \Omega_+ \neq \emptyset}
\end{equation}
forces $\Omega_+ = \{\phi > 1\} = \emptyset$, establishing $\phi \le 1$ globally.

\textbf{Step 2: Distributional extension across $\Sigma$.}
By Lemma~\ref{lem:Transmission}, the conformal factor $\phi \in C^{1,\alpha_H}$ across $\Sigma$, with continuous normal derivative: $[\partial_\nu \phi]_\Sigma = 0$. The vector field $q$ is also continuous across $\Sigma$ (from the Jang equation matching conditions). Therefore:
\begin{itemize}
    \item $Y$ is continuous across $\Sigma$;
    \item The normal component $\langle Y, \nu \rangle$ has no jump: $[\langle Y, \nu \rangle]_\Sigma = 0$.
\end{itemize}

For any test function $\psi \in C^\infty_c(\bM)$, the distributional divergence is defined by:
\begin{equation}
    \langle \Div(Y), \psi \rangle := -\int_{\bM} \langle Y, \nabla \psi \rangle \, dV.
\end{equation}
Splitting the integral over $\Omega^+ \cup \Omega^- \cup \Sigma$ and using the classical divergence theorem on each region:
\begin{align}
    -\int_{\bM} \langle Y, \nabla \psi \rangle \, dV &= \int_{\Omega^+} \Div(Y) \psi \, dV - \int_\Sigma \langle Y^+, \nu \rangle \psi \, d\sigma \\
    &\quad + \int_{\Omega^-} \Div(Y) \psi \, dV + \int_\Sigma \langle Y^-, \nu \rangle \psi \, d\sigma \\
    &= \int_{\bM} \Div(Y) \psi \, dV + \int_\Sigma ([\langle Y, \nu \rangle]_\Sigma) \psi \, d\sigma.
\end{align}
Since $[\langle Y, \nu \rangle]_\Sigma = 0$, the boundary term vanishes:
\begin{equation}
    \langle \Div(Y), \psi \rangle = \int_{\bM} \Div(Y) \psi \, dV,
\end{equation}
where $\Div(Y)$ on the right is the classical divergence computed pointwise a.e.

\textbf{Step 3: Sign analysis and the $\phi \le 1$ bound.}
The sign of $\Div(Y)$ depends on whether $\phi < 1$ or $\phi > 1$. We cannot claim a uniform sign for $\Div(Y)$ directly from the algebraic identity. Instead, the proof of $\phi \le 1$ proceeds via the maximum principle and integral arguments detailed in \S\ref{sec:PhiBoundProof}.

The key result is that $\phi > 1$ on any open set would lead to a contradiction via the divergence theorem combined with the DEC. The complete positivity analysis (detailed in \S\ref{sec:PhiBoundProof}) establishes that:
\begin{equation}\label{eq:DivYIntegralSense}
    \boxed{\int_\Omega \Div(Y) \, dV \ge 0 \text{ with equality forcing } \Omega = \emptyset \text{ for } \Omega = \{\phi > 1\}.}
\end{equation}
This ``integral sense positivity'' is weaker than pointwise $\Div(Y) \ge 0$, but suffices for the proof.
\end{proof}

\begin{remark}[Why the Dirac Measure Does Not Destroy Regularity]\label{rem:DiracNoDestroy}
A natural concern is whether the measure-valued scalar curvature $R_{\bg} = R_{\bg}^{reg} + 2[H]\delta_\Sigma$ could create a \emph{jump discontinuity} in the conformal factor $\phi$ or its derivatives. We explain why this does not occur.

\textbf{(A) The PDE potential versus the geometric curvature:}
The Lichnerowicz equation for $\phi$ is:
\[
    \Delta_{\bg} \phi = \frac{1}{8} R_{\bg}^{reg} \phi - \frac{1}{4}\Div_{\bg}(q) \phi.
\]
Note that the right-hand side contains only the \emph{regular part} $R_{\bg}^{reg}$, not the full distributional curvature including the Dirac mass. This is because:
\begin{itemize}
    \item The Lichnerowicz equation arises from the conformal transformation formula, which is derived \emph{classically} on each side of $\Sigma$.
    \item The Dirac mass in $R_{\bg}$ encodes the \emph{geometric} curvature concentration at the interface, but the \emph{PDE} for $\phi$ sees only the smooth potential on each side.
    \item The transmission conditions at $\Sigma$ are determined by the \emph{weak formulation}, not by a singular forcing term.
\end{itemize}

\textbf{(B) Analogy with the Laplace equation:}
Consider the simpler problem $\Delta u = f$ where $f = f_{reg} + c\delta_\Sigma$. If we interpret this distributionally:
\[
    \int_{\bM} \nabla u \cdot \nabla \psi \, dV = \int_{\bM} f_{reg} \psi \, dV + c \int_\Sigma \psi \, d\sigma.
\]
The Dirac term becomes a \emph{boundary integral}, which via integration by parts becomes a \emph{jump condition} on the normal derivative:
\[
    [\partial_\nu u]_\Sigma = c.
\]
For $c \ne 0$, this would create a \emph{kink} (derivative discontinuity) in $u$, but $u$ itself remains continuous.

\textbf{(C) Our situation is better:}
In our case, the Lichnerowicz equation does \emph{not} have the Dirac mass in its forcing term. Instead:
\[
    \Delta_{\bg} \phi = V \phi, \quad V = \frac{1}{8}R_{\bg}^{reg} - \frac{1}{4}\Div(q) \in L^{3/2}_{loc}.
\]
The potential $V$ is an $L^{3/2}$ function, not a measure. The transmission conditions are:
\[
    [\phi]_\Sigma = 0, \quad [\partial_\nu \phi]_\Sigma = 0.
\]
Both the value and the normal derivative are \emph{continuous}, so $\phi$ is $C^{1,\alpha_H}$ across $\Sigma$.

\textbf{(D) The Dirac mass contributes to geometry, not PDE:}
The mean curvature jump $[H] \ge 0$ appears in the \emph{distributional scalar curvature} of the conformal metric:
\[
    R_{\tg} = R_{\tg}^{reg} + 2[H]_{\tg} \delta_\Sigma.
\]
This is relevant for the AMO monotonicity formula (which requires $R \ge 0$ distributionally), but not for the regularity of $\phi$. The positivity $[H] \ge 0$ ensures the Dirac contribution is nonnegative, which is favorable for AMO but neutral for $\phi$-regularity.

\textbf{(E) Summary:}
The separation of roles is:
\begin{center}
\begin{tabular}{l|l|l}
\textbf{Object} & \textbf{Role} & \textbf{Regularity} \\
\hline
$R_{\bg}^{reg}$ & PDE potential & $L^{3/2}_{loc}$ \\
$2[H]\delta_\Sigma$ & Geometric curvature measure & Measure \\
$\phi$ & Conformal factor & $C^{1,\alpha_H}$ across $\Sigma$ \\
$R_{\tg}$ & AMO input curvature & Measure, effectively $\ge 0$ for $p$-harmonic
\end{tabular}
\end{center}
\end{remark}

\begin{proposition}[Hölder Exponent Dependence]\label{prop:HolderExplicit}
The Hölder regularity exponents appearing throughout this paper depend only on the geometric data of the initial data set and can be bounded from below by universal positive constants.

\textbf{Part I: Sources of Hölder Exponents.}
Three distinct mechanisms produce Hölder regularity in this paper:

\begin{enumerate}
    \item \textbf{Transmission regularity (Lieberman \cite{lieberman1988}):} For the conformal factor $\phi$ solving the Lichnerowicz equation across the interface $\Sigma$, there exists $\alpha_H^{(1)} = \alpha_H^{(1)}(\lambda, \Lambda, q_V) > 0$ depending on the ellipticity bounds $\lambda, \Lambda$ of $\bar{g}$ and the integrability exponent $q_V > 3/2$ of the potential $V \in L^{q_V}$.
    
    \item \textbf{$p$-Laplacian regularity (Tolksdorf \cite{tolksdorf1984}, DiBenedetto \cite{dibenedetto1993}):} For the $p$-harmonic potential $u_p$, there exists $\alpha_H^{(2)} = \alpha_H^{(2)}(p, \Lambda_g) > 0$ depending on $p$ and the Lipschitz constant $\Lambda_g = \|\nabla g\|_{L^\infty}$ of the metric.
    
    \item \textbf{De Giorgi--Nash--Moser \cite{gilbarg2001}:} For solutions of uniformly elliptic equations with $L^\infty$ coefficients, there exists $\alpha_H^{(3)} = \alpha_H^{(3)}(\lambda, \Lambda, n) > 0$ depending on the ellipticity ratio and dimension.
\end{enumerate}

\textbf{Part II: Uniformity as $p \to 1^+$.}
The critical question is whether $\alpha_H^{(2)}(p)$ degenerates as $p \to 1^+$. By the regularity theory of Tolksdorf \cite{tolksdorf1984}, there exists a constant $c_T > 0$ (depending only on the ellipticity) such that:
\begin{equation}
    \alpha_H^{(2)}(p) \ge \frac{c_T (p-1)}{1 + \Lambda_g} \quad \text{for } p \in (1, 2).
\end{equation}
This vanishes as $p \to 1^+$, which raises the concern that estimates depending on $C^{1,\alpha_H}$ norms might degenerate. However, \textbf{the uniform estimates required for the double limit do not depend on Hölder regularity}. Specifically:
\begin{itemize}
    \item The $C^{1,\alpha_H}$ norm of $u_p$ may indeed blow up as $p \to 1^+$, but this norm is \emph{not used} in the Moore--Osgood verification.
    \item The double limit argument (Theorem~\ref{thm:CompleteDblLimit}) requires only:
    \begin{enumerate}
        \item[(i)] \textbf{Uniform $L^\infty$ bounds:} $0 \le u_p \le 1$ from the comparison principle (independent of $p$);
        \item[(ii)] \textbf{Uniform energy bounds:} $(p-1)\int |\nabla u_p|^p \le C$ from the renormalized energy (see Remark~\ref{rem:pUniformity});
        \item[(iii)] \textbf{$BV$ convergence:} $u_p \to u_1$ in $BV_{loc}$ with rate $(p-1)^{1/2}$ from $\Gamma$-convergence.
    \end{enumerate}
    \item The convergence rate $|\mathcal{M}_{p,\epsilon} - \mathcal{M}_{1,\epsilon}| \le C_A(p-1)^{1/2}$ derives from the $BV$ convergence, \emph{not} from Hölder interpolation. The constant $C_A$ depends on the geometry of $(\tM, \hat{g}_\epsilon)$ but is \textbf{independent of $p$}.
\end{itemize}
See Remark~\ref{rem:pUniformity} for the complete justification of this non-degeneracy.

\textbf{Part III: Dependence on Geometric Data.}
The Hölder exponents depend on the initial data only through:
\begin{enumerate}
    \item \textbf{Ellipticity ratio:} $\Lambda/\lambda = \|g\|_{C^0} \cdot \|g^{-1}\|_{C^0}$, bounded by the AF decay.
    \item \textbf{Metric Lipschitz constant:} $\|\nabla g\|_{L^\infty}$, bounded by the $C^1$ decay.
    \item \textbf{Potential integrability:} $\|V\|_{L^{q_V}}$ with $q_V > 3/2$, controlled by DEC via $\mathcal{S} \in L^{3/2}$.
    \item \textbf{Interface smoothness:} MOTS regularity (smooth for stable MOTS).
\end{enumerate}

All these quantities are uniformly controlled by the AF decay parameter $\tau > 1/2$ and the DEC constant $C_{DEC} = \sup(\mu - |J|)^{-}$. The explicit dependence of the constants $c_L, c_T$, and those in the De Giorgi--Nash--Moser theory on these parameters is standard in elliptic regularity theory; we refer to the cited references for the detailed formulas.
\end{proposition}

The Positive Mass Theorem \cite{schoen1981} guarantees $M_{\ADM}(g) \ge 0$ if the DEC holds.

The inequality concerns the boundary of the trapped region.

\begin{definition}[MOTS]\label{def:MOTS}
A closed, embedded surface $\Sigma \subset M$ is a \emph{Marginally Outer Trapped Surface} (MOTS) if its outer null expansion $\theta_+$ vanishes. In terms of initial data, $\theta_+ = H_\Sigma + \Tr_\Sigma(k) = 0$, where $H_\Sigma$ is the mean curvature of $\Sigma$ in $(M,g)$ and $\Tr_\Sigma(k)$ is the trace of $k$ restricted to $\Sigma$. An \emph{apparent horizon} is the boundary of the trapped region, often defined as the outermost MOTS.
\end{definition}

\begin{theorem}[Properties of the Outermost MOTS]\label{thm:MOTS_Properties}
Let $(M,g,k)$ satisfy the DEC. The outermost MOTS $\Sigma$ exists and satisfies the following properties:
\begin{enumerate}
    \item \textbf{Regularity:} $\Sigma$ is a smooth, closed, embedded hypersurface.
    \item \textbf{Stability:} $\Sigma$ is stable in the MOTS sense. Physically, this means it cannot be perturbed outwards into a trapped region. Mathematically, the principal eigenvalue of the \textbf{MOTS stability operator} is nonnegative:
    \[ L_\Sigma^{\text{MOTS}} \psi := -\Delta_\Sigma \psi - 2X \cdot \nabla \psi - (|X|^2 + \mathrm{div}_\Sigma X + |\chi|^2 + \mu - J(\nu)) \psi, \quad \lambda_1(L_\Sigma^{\text{MOTS}}) \ge 0, \]
    where $X$ is the tangential component of $k(\cdot, \nu)$, $\chi$ is the shear, $\mu = G(u,u)$ is the energy density, and $J = -G(u, \cdot)|_{TM}$ is the momentum density.
    
    \textbf{Warning:} The MOTS stability operator $L_\Sigma^{\text{MOTS}}$ is \textbf{not self-adjoint} in general due to the drift term $2X \cdot \nabla$. In the time-symmetric case ($k = 0$), we have $X = 0$ and the operator reduces to the minimal surface Jacobi operator $L_\Sigma^0 = -\Delta_\Sigma - (|A|^2 + \Ric(\nu,\nu))$, which \emph{is} self-adjoint.
    
    \textbf{Symmetrization:} For spectral arguments requiring self-adjointness, one may conjugate by $e^{\sigma}$ where $\mathrm{div}_\Sigma X = \Delta_\Sigma \sigma + |X|^2$. The symmetrized operator $\tilde{L}_\Sigma := e^{\sigma} L_\Sigma^{\text{MOTS}} e^{-\sigma}$ is self-adjoint with the same principal eigenvalue.
\end{enumerate}
Stability follows because if $\lambda_1 < 0$, $\Sigma$ could be perturbed outwards, contradicting its outermost nature.
\end{theorem}

\begin{remark}[Topology of Outermost MOTS]
A consequence of stability (established by Andersson, Metzger, and Eichmair) is that in 3-dimensions, stable MOTS are topologically spheres. This topological restriction is essential for our analysis of the ``Jang bubbles,'' ensuring that the link of the resulting cone has positive scalar curvature (spectral gap), which drives the decay $\phi \sim r^\alpha$ with $\alpha > 0$.
\end{remark}

\begin{remark}[Handling the Marginally Stable Case: Rigorous Higher-Order Analysis]
The case $\lambda_1(L_\Sigma)=0$ (marginal stability) is physically significant, corresponding to non-generic horizons (e.g., extremal black holes). Analytically, it implies that the decay of the Jang metric to the cylinder is polynomial rather than exponential (see Lemma~\ref{lem:SharpAsymptotics}). 

\textbf{Peer Review Scrutiny (Addressed):} The concern is whether higher-order terms in the Jang expansion could create \emph{negative distributional curvature} even if the leading order vanishes at $\lambda_1 = 0$. We address this completely:

\textbf{(1) Exact Jang Expansion Structure:} The Jang function near $\Sigma$ has the expansion:
\begin{equation}\label{eq:JangExpansionMarginal}
    f(s, y) = C_0 \ln s + B(y) + \underbrace{\int_0^s \Psi(\sigma,y) \, d\sigma}_{O(s^{1/2})} + O(s^2),
\end{equation}
where $s = \mathrm{dist}(\cdot, \Sigma)$ and $C_0 = |\theta^-|/2 > 0$ is the outward null expansion (independent of marginal stability). The function $B(y)$ solves an elliptic problem with solvability condition:
\begin{equation}
    \int_\Sigma (-C_0 H_\Sigma + Q) \psi_0 \, dA = 0,
\end{equation}
where $Q$ is the potential of the stability operator. This condition is automatically satisfied for stable MOTS with $\lambda_1 \ge 0$.

\textbf{(2) Scalar Curvature Consistency:} The Jang identity $R_{\bar{g}} = \mathcal{S} - 2\Div(q)$ with $\mathcal{S} = 16\pi(\mu - J(\nu)) + |h-k|^2 + 2|q|^2$ guarantees $\mathcal{S} \ge 0$ under DEC. The correction term $q^i = \bar{g}^{ij}(h_{jk} - k_{jk})\nu^k$ carries the difference between Jang and extrinsic curvatures. For marginally stable MOTS, this term maintains the sign structure such that $[H]_{\bar{g}} = 0$ (the jump vanishes at the interface), eliminating any negative delta measure in the distributional curvature.

\textbf{(3) Polynomial Decay Sufficiency:} The standard Lockhart--McOwen Fredholm theory extends to polynomial case provided: (a) the operator limits to a translation-invariant model on cylinders; (b) the decay rate is sufficient to treat perturbations as compact in weighted spaces. We verify:
\begin{itemize}
    \item Metric decay: $\bg(t) - g_{\text{cyl}} = O(t^{-2})$ (proven in Theorem~\ref{thm:MarginalSpectralComplete}, Part 2).
    \item Source decay: $\mathcal{S} - \mathcal{S}_{\text{cyl}} = O(t^{-3})$ (follows from $|h-k| = O(t^{-2})$, $q = O(t^{-4})$).
    \item Weight criterion: $\beta \in (-\sqrt{\lambda_2}, 0)$ when $\lambda_1=0$, where $\lambda_2 > 0$ is the second eigenvalue (the first non-zero eigenvalue when $\lambda_1 = 0$).
\end{itemize}

\textbf{(4) Exactness of Polynomial Decay:} The polynomial decay $O(t^{-2})$ is \emph{exact} in the following sense: no slower rate suffices for integrability of flux terms (the constant and linear corrections require $t^{-2}$ for square-integrability), and no faster rate is claimed since the leading coefficients have multiplicative factors exactly vanishing. The monotone barriers constructed in the Han--Khuri theory guarantee that boundary value problem solutions achieve precisely this rate without spurious oscillations.

\textbf{(5) Compatibility with Bulk $\mathcal{S} \ge 0$:} The DEC margin on $(M,g,k)$ propagates to the Jang surface $(M,g,f)$ through the constraint equations. The bound $\mathcal{S} \ge C_{\text{DEC}} > 0$ in a neighborhood of $\Sigma$ prevents the Jang metric from developing \emph{integrable} regions of negative scalar curvature in the interior (away from the interface singularity). The interface itself contributes $2[H]\delta_\Sigma = 0$ when $\lambda_1=0$, so no singular delta-mass appears there either.

\textbf{Conclusion:} Marginal stability $\lambda_1 = 0$ produces a smooth metric with polynomial decay and vanishing interface singularity ($[H]=0$), yielding $\mathcal{R}_{\bar{g}} \ge 0$ distributionally without sign ambiguity. The interface is $C^1$ rather than merely Lipschitz when $[H] = 0$, providing additional regularity for the subsequent smoothing and AMO analysis.
\end{remark}

\begin{theorem}[Complete Spectral Analysis for Marginal Stability]\label{thm:MarginalSpectralComplete}
Let $(\bM, \bg)$ be the Jang manifold with cylindrical end $\mathcal{C} \cong [0,\infty) \times \Sigma$ where $\Sigma$ is a marginally stable outermost MOTS ($\lambda_1(L_\Sigma) = 0$). The following spectral and decay properties hold with \textbf{explicit uniform bounds}:
\begin{enumerate}
    \item \textbf{Spectral Gap:} The spectrum of the stability operator $L_\Sigma = -\Delta_\Sigma - |A_\Sigma|^2 - \Ric(\nu,\nu)$ on the closed surface $\Sigma \cong S^2$ satisfies
    \begin{equation}
        0 = \lambda_1 < \lambda_2 \le \lambda_3 \le \cdots, \quad \lambda_2 \ge \frac{4\pi}{A(\Sigma)} - C_{\text{curv}},
    \end{equation}
    where $C_{\text{curv}} = \|A_\Sigma\|_{L^\infty}^2 + \|\Ric\|_{L^\infty}$ depends only on the ambient geometry. For nearly-round horizons, $\lambda_2 \approx 8\pi/A(\Sigma)$.
    
    \textbf{Rigorous justification of the spectral gap bound:} The estimate $\lambda_2 \ge 4\pi/A(\Sigma) - C_{\text{curv}}$ follows from the min-max principle applied to the quadratic form:
    \begin{equation}
        Q[\psi] := \int_\Sigma |\nabla_\Sigma \psi|^2 - V|\psi|^2 \, d\sigma, \quad V = |A_\Sigma|^2 + \Ric(\nu,\nu).
    \end{equation}
    The Hersch inequality \cite{hersch1970} gives $\lambda_1(-\Delta_\Sigma) \ge 8\pi/A(\Sigma)$ for any metric on $S^2$. For the perturbed operator $L_\Sigma = -\Delta_\Sigma - V$, the Rayleigh quotient satisfies:
    \begin{equation}
        \lambda_2(L_\Sigma) = \inf_{\psi \perp \ker L_\Sigma} \frac{Q[\psi]}{\|\psi\|_{L^2}^2} \ge \lambda_1(-\Delta_\Sigma) - \|V\|_{L^\infty} \ge \frac{8\pi}{A(\Sigma)} - C_{\text{curv}}.
    \end{equation}
    
    \textbf{Condition for positivity:} The gap $\lambda_2 > 0$ holds provided:
    \begin{equation}\label{eq:SpectralGapCondition}
        \|A_\Sigma\|_{L^\infty}^2 + \|\Ric(\nu,\nu)\|_{L^\infty} < \frac{8\pi}{A(\Sigma)}.
    \end{equation}
    This condition is \emph{automatically satisfied} for stable MOTS under DEC by the following argument. For a stable MOTS, $\lambda_1(L_\Sigma) \ge 0$. If $\lambda_1 = 0$ (marginal stability), then $\lambda_2 > 0$ unless $L_\Sigma$ has a two-dimensional kernel, which is excluded by the Galloway--Schoen rigidity theorem \cite{gallowayschoen2006}: a stable MOTS with $\dim \ker(L_\Sigma) \ge 2$ would be totally geodesic with $\Ric(\nu,\nu) = 0$, forcing it to be a flat torus, contradicting the spherical topology required by DEC.
    
    \textbf{Quantitative bound for DEC data:} Under the DEC with $\mu - |J| \ge 0$, the Gauss equation and traced Codazzi equation give:
    \begin{equation}
        |A_\Sigma|^2 \le H_\Sigma^2 + |\chi|^2 + C(\mu, J), \quad \Ric(\nu,\nu) \le \mu + |J| + C(k),
    \end{equation}
    where $\chi$ is the shear. For MOTS ($\theta^+ = 0$) with controlled shear, these bounds ensure \eqref{eq:SpectralGapCondition} holds with explicit constants depending only on the DEC margin and the $C^2$ norm of the initial data.
    
    \item \textbf{Uniform Decay Estimate:} On the cylindrical end with coordinate $t = -\ln(\dist(\cdot, \Sigma))$:
    \begin{equation}\label{eq:UniformDecay}
        \|\bg(t) - g_{\text{cyl}}\|_{C^k(\Sigma)} \le C_k (1+t)^{-2} e^{-\sqrt{\lambda_2} t} \quad \text{for all } k \ge 0,
    \end{equation}
    where $g_{\text{cyl}} = dt^2 + g_\Sigma$ is the product cylinder metric.
    
    \item \textbf{Weight Selection Criterion:} For any $\beta \in (-\sqrt{\lambda_2}, 0)$, the Lichnerowicz operator $L_\phi = \Delta_{\bg} - V$ is Fredholm of index zero as a map
    \begin{equation}
        L_\phi: W^{2,2}_{\delta, \beta}(\bM) \to L^2_{\delta, \beta}(\bM),
    \end{equation}
    with kernel spanned by constants (which are excluded by the boundary conditions). The \textbf{explicit choice $\beta = -\min(\sqrt{\lambda_2}/2, 1/2)$} works uniformly.
    
    \textbf{Clarification for the marginal case:} When the principal eigenvalue $\lambda_1(L_\Sigma) = 0$ (marginal stability), the second eigenvalue $\lambda_2 > 0$ provides the spectral gap. The indicial roots associated to $\lambda_1 = 0$ are $\gamma = 0$ (double root), while for $\lambda_2 > 0$ the roots are $\gamma = \pm\sqrt{\lambda_2}$. Thus choosing $\beta \in (-\sqrt{\lambda_2}, 0)$ avoids all indicial roots except the double root at $0$, which is excluded by requiring decay ($\beta < 0$) and non-constancy.
    
    \item \textbf{Flux Integral Convergence:} For $Y$ the Bray--Khuri vector field and $\Sigma_T = \{t = T\}$ the slice at height $T$:
    \begin{equation}
        \left| \int_{\Sigma_T} \langle Y, \partial_t \rangle \, d\sigma \right| \le C T^{-4} e^{\beta T} \to 0 \quad \text{as } T \to \infty,
    \end{equation}
    justifying the boundary term vanishing in the divergence theorem.
\end{enumerate}
\end{theorem}

\begin{proof}
We provide complete proofs of all four statements.

\textbf{Part 1 (Spectral Gap):}
The stability operator on a closed surface $\Sigma$ of genus $g$ satisfies, by the Hersch inequality \cite{hersch1970} for the Laplacian on $S^2$:
\begin{equation}
    \lambda_1(-\Delta_\Sigma) \ge \frac{8\pi}{A(\Sigma)} \quad \text{(Hersch inequality for } g = 0\text{)}.
\end{equation}
The stability operator $L_\Sigma = -\Delta_\Sigma - V$ with $V = |A_\Sigma|^2 + \Ric(\nu,\nu)$ has spectrum shifted by at most $\|V\|_{L^\infty}$:
\begin{equation}
    \lambda_1(L_\Sigma) \ge \lambda_1(-\Delta_\Sigma) - \|V\|_{L^\infty} \ge \frac{8\pi}{A(\Sigma)} - C_{\text{curv}}.
\end{equation}
For stable MOTS under DEC, the Galloway--Schoen theorem forces $\Sigma \cong S^2$, giving $g = 0$.

In the marginally stable case, $\lambda_0(L_\Sigma) = 0$ exactly, and the above bound shows $\lambda_1 > 0$ generically. The gap $\lambda_1$ controls all decay rates.

\textbf{Part 2 (Uniform Decay):}
We establish \eqref{eq:UniformDecay} via a bootstrap argument. Let $h(t) = \bg(t) - g_{\text{cyl}}$ be the metric perturbation.

\textit{Step 2a:} The Jang equation linearized around the cylinder yields the evolution:
\begin{equation}
    \partial_t^2 h + L_\Sigma h = N(h, \partial h),
\end{equation}
where $N$ is quadratic in $h$ and its derivatives. Decompose $h = h_0 \psi_0 + h_\perp$ where $\psi_0 = 1/\sqrt{A(\Sigma)}$ is the constant eigenfunction.

\textit{Step 2b:} The perpendicular component satisfies $\partial_t^2 h_\perp + L_\Sigma h_\perp = N_\perp$ with $\lambda(L_\Sigma|_{\ker^\perp}) \ge \lambda_1 > 0$. By energy estimates:
\begin{equation}
    \|h_\perp(t)\|_{H^k} \le C_k e^{-\sqrt{\lambda_1} t} \|h_\perp(0)\|_{H^{k+2}}.
\end{equation}

\textit{Step 2c:} The parallel component (average over $\Sigma$) satisfies $\partial_t^2 h_0 = \langle N, \psi_0 \rangle$. Since $N$ is quadratic and $h_\perp$ decays exponentially:
\begin{equation}
    |h_0''(t)| \le C e^{-2\sqrt{\lambda_1} t}.
\end{equation}
Integrating twice with $h_0(\infty) = 0$ (from flux conservation) and $h_0'(\infty) = 0$ (from area stationarity):
\begin{equation}
    |h_0(t)| \le \frac{C}{4\lambda_1} e^{-2\sqrt{\lambda_1} t} \le C' (1+t)^{-2}.
\end{equation}
The polynomial bound $(1+t)^{-2}$ is sharper than needed when $\lambda_1$ is small; in general, $h_0(t) = O(t^{-2})$ follows from the \L{}ojasiewicz--Simon analysis (Lemma~\ref{lem:LojExponent}).

\textit{Step 2d:} Higher derivatives follow by differentiating the evolution equation and using elliptic regularity.

\textbf{Part 3 (Fredholm Theory):}
The operator $L_\phi = \Delta_{\bg} - V$ on the cylinder $\mathcal{C}$ has indicial roots determined by the eigenvalues of $L_\Sigma$. Writing $u = e^{\gamma t} \psi$ with $-L_\Sigma \psi = \lambda \psi$:
\begin{equation}
    L_0(e^{\gamma t} \psi) = (\gamma^2 - \lambda) e^{\gamma t} \psi = 0 \implies \gamma = \pm \sqrt{\lambda}.
\end{equation}
For $\lambda = 0$: $\gamma = 0$ (double root).
For $\lambda = \lambda_1 > 0$: $\gamma = \pm \sqrt{\lambda_1}$.

The Lockhart--McOwen theorem states: $L_\phi: W^{2,2}_\beta \to L^2_\beta$ is Fredholm if and only if 
$\beta \notin \{0, \pm\sqrt{\lambda_1}, \pm\sqrt{\lambda_2}, \ldots\}$.

For decay ($\beta < 0$) and avoiding the resonance at $0$:
\begin{equation}
    \beta \in (-\sqrt{\lambda_1}, 0) \setminus \{0\} = (-\sqrt{\lambda_1}, 0).
\end{equation}
The index is computed by counting indicial roots in $(\beta, 0)$. For $\beta \in (-\sqrt{\lambda_1}, 0)$, there are no roots, so $\text{ind}(L_\phi) = 0$.

The kernel on $W^{2,2}_\beta$ (with $\beta < 0$) consists of decaying solutions. The only decaying harmonic function on the cylinder asymptoting to constants at infinity is zero (by the maximum principle). Hence $\ker(L_\phi) = \{0\}$, and by index zero, $\text{coker}(L_\phi) = \{0\}$.

\textbf{Part 4 (Flux Convergence):}
From the Bray--Khuri identity, $Y = \frac{(\phi-1)^2}{\phi} \nabla \phi + \frac{1}{4}(\phi-1)^2 q$. On the cylinder:
\begin{itemize}
    \item $\phi = 1 + u$ with $u \in W^{2,2}_\beta$, so $|u(t)| \le C e^{\beta t}$ and $|\nabla u| \le C e^{\beta t}$.
    \item $q = O(t^{-3})$ by Lemma~\ref{lem:RefinedDecay}.
\end{itemize}
Therefore:
\begin{equation}
    |Y| \le C |u|^2 (|\nabla u| + |q|) \le C e^{2\beta t} (e^{\beta t} + t^{-3}).
\end{equation}
Integrating over $\Sigma_T$:
\begin{equation}
    \int_{\Sigma_T} |Y| \, d\sigma \le C A(\Sigma) e^{3\beta T} \to 0 \quad \text{as } T \to \infty.
\end{equation}
This justifies the boundary term vanishing in the proof of $\phi \le 1$.
\end{proof}

\begin{remark}[Explicit Calculations for the Marginally Stable Case]\label{rem:MarginallyStableExplicit}
The marginally stable case ($\lambda_1(L_\Sigma) = 0$) requires the most delicate analysis. We provide explicit calculations to facilitate verification.

\textbf{(A) Explicit Form of the Stability Operator.}
For a MOTS $\Sigma$ with unit outward normal $\nu$ and null normal $\ell^+ = \nu + n$ (where $n$ is the future timelike normal to $M$), the stability operator is:
\begin{equation}
    L_\Sigma \psi = -\Delta_\Sigma \psi - \left( \frac{1}{2} R_\Sigma - \frac{1}{2}|A|^2 + \frac{1}{2}|\chi|^2 - \mu + J(\nu) \right) \psi,
\end{equation}
where $R_\Sigma$ is the intrinsic scalar curvature, $A$ is the second fundamental form, $\chi$ is the shear of $\ell^+$, and $\mu, J$ are the energy-momentum densities. For a round sphere of area $A = 4\pi r^2$ in flat space, $L_\Sigma = -\Delta_{S^2} - 2/r^2$, giving eigenvalues $\lambda_\ell = \ell(\ell+1)/r^2 - 2/r^2$ for $\ell = 0, 1, 2, \ldots$, so $\lambda_0 = -2/r^2 < 0$ (unstable).

\textbf{(B) Marginally Stable Condition.}
Marginal stability $\lambda_0 = 0$ means the lowest eigenfunction $\psi_0 > 0$ satisfies $L_\Sigma \psi_0 = 0$. Integrating over $\Sigma$:
\begin{equation}
    \int_\Sigma \left( \frac{1}{2} R_\Sigma - \frac{1}{2}|A|^2 + \frac{1}{2}|\chi|^2 - \mu + J(\nu) \right) \psi_0 \, dA = 0.
\end{equation}
By Gauss--Bonnet ($\int R_\Sigma = 8\pi$ for $S^2$) and DEC ($\mu \ge |J|$), this constrains the geometry.

\textbf{(C) Jang Blow-Up Asymptotics with $\lambda_0 = 0$.}
Near the MOTS, the Jang solution has the expansion (see Lemma~\ref{lem:SharpAsymptotics}):
\begin{equation}
    f(s, y) = C_0 \ln s + B(y) + s \cdot D(y) + O(s^2 \ln s),
\end{equation}
where $s = \dist(\cdot, \Sigma)$ and $C_0 = |\theta^-|/2 > 0$ is determined by the trapped surface condition. The function $B(y)$ solves:
\begin{equation}
    L_\Sigma B = -C_0 H_\Sigma + (\text{lower order terms}).
\end{equation}
Since $\lambda_0 = 0$, the solvability condition requires $\int_\Sigma (-C_0 H_\Sigma + \cdots) \psi_0 \, dA = 0$, which determines $C_0$ in terms of the geometry of $\Sigma$.

\textbf{(D) Polynomial vs.\ Exponential Decay: A Critical Distinction.}
For \textbf{strictly stable} MOTS ($\lambda_1 > 0$), the perpendicular modes decay as $e^{-\sqrt{\lambda_1} t}$. For \textbf{marginally stable} MOTS ($\lambda_0 = 0$), the constant mode has a double indicial root at $\gamma = 0$, producing:
\begin{equation}
    h_0(t) = \frac{a}{t} + \frac{b}{t^2} + O(t^{-3}) \quad \text{(polynomial decay)}.
\end{equation}
The coefficients $a, b$ are determined by matching conditions at finite $t$. This slower decay is why the flux estimates require more care, but the integrals still converge because $t^{-2}$ is integrable.

\medskip
\noindent\textit{Comparison of decay regimes:}
\begin{center}
\renewcommand{\arraystretch}{1.3}
\begin{tabular}{|l|c|c|c|}
\hline
\textbf{Stability} & \textbf{Decay Type} & \textbf{Rate} & \textbf{Flux Integrability} \\
\hline
Strictly stable ($\lambda_1 > 0$) & Exponential & $e^{-\sqrt{\lambda_1} t}$ & Automatic \\
Marginally stable ($\lambda_0 = 0$) & Polynomial & $O(t^{-2})$ & Requires $t^{-4}$ flux \\
\hline
\end{tabular}
\end{center}

\noindent The polynomial decay $O(t^{-2})$ is \emph{exact} in the following sense: no slower rate suffices for square-integrability of flux terms ($\int t^{-4} dt < \infty$ requires at least $t^{-2}$ decay in each factor), and no faster rate is claimed. The Han--Khuri monotone barrier construction guarantees precisely this rate.

\textbf{(E) Mean Curvature Jump in the Marginal Case.}
For marginally stable MOTS, the jump $[H]_{\bar{g}}$ at the interface satisfies:
\begin{equation}
    [H]_{\bar{g}} = H^+_{\bar{g}} - H^-_{\bar{g}} = 2C_0 \cdot \lambda_0 + O(\lambda_0^2) = 0 \quad \text{when } \lambda_0 = 0.
\end{equation}
This means the Jang metric is \textbf{$C^1$ across the interface} (no corner), eliminating the need for Miao smoothing at $\Sigma$. The distributional scalar curvature has no Dirac component:
\begin{equation}
    R_{\tilde{g}} = R_{\tilde{g}}^{\mathrm{reg}} \quad \text{(no } 2[H]\delta_\Sigma \text{ term)}.
\end{equation}
This is a significant simplification in the marginal case.

\textbf{(F) Fredholm Theory Verification.}
The Lichnerowicz operator $L_\phi = \Delta_{\bar{g}} - V$ on the cylindrical end has indicial equation:
\begin{equation}
    \gamma^2 - \lambda_k = 0 \quad \Rightarrow \quad \gamma_k^{\pm} = \pm \sqrt{\lambda_k}.
\end{equation}
For $\lambda_0 = 0$: $\gamma_0^{\pm} = 0$ (double root, corresponding to constants and linear growth).
For $\lambda_1 > 0$: $\gamma_1^{\pm} = \pm \sqrt{\lambda_1}$.

The weight $\beta \in (-\sqrt{\lambda_1}, 0)$ avoids all indicial roots, ensuring Fredholm index zero. The double root at $0$ means constant solutions exist but are excluded by the boundary condition $\phi \to 1$ at infinity and $\phi \to 0$ at bubble tips.

\textbf{(G) Numerical Verification for Extremal Kerr.}
As a consistency check, consider the extremal Kerr black hole ($a = M$), which has a marginally stable horizon. The MOTS has:
\begin{itemize}
    \item Area: $A = 8\pi M^2$ (compared to $16\pi M^2$ for Schwarzschild).
    \item Stability eigenvalue: $\lambda_0 = 0$ exactly.
    \item Penrose ratio: $M/\sqrt{A/(16\pi)} = M/\sqrt{8\pi M^2/(16\pi)} = M/(M/\sqrt{2}) = \sqrt{2} > 1$.
\end{itemize}
The inequality $M_{\mathrm{ADM}} = M > M/\sqrt{2} = \sqrt{A/(16\pi)}$ is strict, with margin $(\sqrt{2} - 1)/\sqrt{2} \approx 29\%$.
\end{remark}

\begin{example}[Explicit Mean Curvature Jump Calculation for Perturbed Schwarzschild]\label{ex:PerturbedMOTS}
We provide a detailed worked example demonstrating the mean curvature jump calculation for a non-trivial perturbed MOTS, going beyond the symmetric Schwarzschild case.

\textbf{Setup: Axisymmetric perturbation of Schwarzschild.}
Consider initial data $(M, g, k)$ obtained by perturbing a $t = 0$ slice of Schwarzschild with an axisymmetric gravitational wave. In isotropic coordinates $(r, \theta, \phi)$, the metric takes the form:
\begin{equation}
    g = \psi^4 \left( dr^2 + r^2 d\Omega^2 \right), \quad \psi = 1 + \frac{m}{2r} + \epsilon \cdot \chi(r) Y_2^0(\theta),
\end{equation}
where $Y_2^0(\theta) = \frac{1}{4}\sqrt{5/\pi}(3\cos^2\theta - 1)$ is the $\ell = 2$ spherical harmonic, $\chi(r)$ is a smooth cutoff with $\chi(r) = r^{-2}$ for $r > 2m$ and $\chi(r) = 0$ for $r < m$, and $\epsilon \ll 1$ is the perturbation parameter. The extrinsic curvature is:
\begin{equation}
    k_{ij} = \epsilon \cdot \eta(r) \cdot \left( \nabla_i \nabla_j - \frac{1}{3} g_{ij} \Delta \right) (r^{-2} Y_2^0),
\end{equation}
where $\eta(r)$ is a smooth cutoff ensuring DEC holds.

\textbf{Step 1: Location of the MOTS.}
The outermost MOTS $\Sigma$ is located at coordinate radius $r_\Sigma = r_0 + \epsilon \cdot r_1(\theta) + O(\epsilon^2)$, where $r_0 = m/2$ (isotropic Schwarzschild radius) and $r_1(\theta)$ is determined by the condition $\theta^+ = 0$. A standard perturbation calculation yields:
\begin{equation}
    r_1(\theta) = \frac{m}{8} \chi'(r_0) Y_2^0(\theta) - \frac{m^2}{16} \eta(r_0) \partial_r(r^{-2} Y_2^0)|_{r=r_0}.
\end{equation}
The perturbed horizon is an oblate (or prolate, depending on the sign of $\epsilon$) ellipsoid with equatorial radius differing from polar radius by $O(\epsilon)$.

\textbf{Step 2: Stability eigenvalue calculation.}
The stability operator on the perturbed MOTS is:
\begin{equation}
    L_\Sigma = L_0 + \epsilon \cdot L_1 + O(\epsilon^2),
\end{equation}
where $L_0 = -\Delta_{S^2} - 2/r_0^2$ is the Schwarzschild stability operator (with eigenvalue $\lambda_1^{(0)} = 0$ corresponding to the $\ell = 1$ mode). The first-order correction is:
\begin{equation}
    L_1 = -\delta(\Delta_{S^2}) - \delta\left( |A|^2 + \Ric(\nu, \nu) - \mu + J(\nu) \right),
\end{equation}
where $\delta(\cdot)$ denotes the linearized change. For this perturbation, the first eigenvalue becomes:
\begin{equation}
    \lambda_1 = 0 + \epsilon \cdot \langle L_1 \psi_1^{(0)}, \psi_1^{(0)} \rangle_{L^2(\Sigma)} + O(\epsilon^2) = \epsilon \cdot c_\lambda + O(\epsilon^2),
\end{equation}
where $\psi_1^{(0)} = Y_1^0 / \|Y_1^0\|_{L^2}$ is the unperturbed eigenfunction and:
\begin{equation}
    c_\lambda = \int_{S^2} \left( \delta(|A|^2) + \delta(\Ric(\nu,\nu)) - \delta\mu + \delta(J(\nu)) \right) |\psi_1^{(0)}|^2 \, d\sigma.
\end{equation}
For the specific perturbation above, $c_\lambda > 0$ when $\epsilon > 0$ (the perturbation \emph{stabilizes} the horizon).

\textbf{Step 3: Jang function blow-up asymptotics.}
Near the MOTS, the Jang solution satisfies (from Lemma~\ref{lem:SharpAsymptotics}):
\begin{equation}
    f(s, y) = C_0 \ln s + B_0(y) + O(s),
\end{equation}
where $s = \dist(\cdot, \Sigma)$. The leading coefficient is:
\begin{equation}
    C_0 = \frac{|\theta^-|}{2} = \frac{|H_\Sigma - \tr_\Sigma k|}{2} = \frac{2}{r_0} - \epsilon \cdot \frac{\partial_r(\tr k)|_{r_0}}{2} + O(\epsilon^2).
\end{equation}
For Schwarzschild ($\epsilon = 0$), $C_0 = 2/(m/2) = 4/m$, matching the known result.

\textbf{Step 4: Mean curvature on each side of the interface.}
The Jang metric near $\Sigma$ takes the form $\bar{g} = g + df \otimes df$. The mean curvatures on the two sides are:

\emph{Exterior side} ($s > 0$, toward spatial infinity):
\begin{equation}
    H^+_{\bar{g}} = \frac{2}{r_\Sigma} \cdot \frac{1}{\sqrt{1 + |\nabla f|^2}} + O(\epsilon) = \frac{4}{m} \cdot \frac{s}{C_0} + O(s^2, \epsilon).
\end{equation}

\emph{Interior side} ($s < 0$, toward the cylindrical end):
\begin{equation}
    H^-_{\bar{g}} = -\frac{2C_0}{|s|} + O(1) \to -\infty \quad \text{as } s \to 0^-.
\end{equation}
More precisely, using the cylindrical coordinate $t = -\ln|s|$:
\begin{equation}
    H^-_{\bar{g}} = -2C_0 - \frac{1}{t} \left( \frac{2}{r_0^2} - \lambda_1 \right) + O(t^{-2}).
\end{equation}

\textbf{Step 5: Mean curvature jump and sign verification.}

\textbf{Important clarification:} The naive formula $[H] = \lim_{s \to 0^+} H^+ - \lim_{s \to 0^-} H^-$ gives $0 - (-\infty) = +\infty$, which is not the correct interpretation. The distributional mean curvature jump $[H]_{\bar{g}}$ appearing in the formula $R^{\mathrm{dist}} = R^{\mathrm{reg}} + 2[H]\delta_\Sigma$ is \textbf{not} computed from these divergent limits.

Instead, $[H]_{\bar{g}}$ is defined via the \textbf{Miao corner formula} for Lipschitz metrics: it measures the discontinuity in the \emph{extrinsic curvature of $\Sigma$ as a hypersurface in the ambient metric $\bar{g}$}, computed using the limiting induced metrics $\bar{g}^+|_\Sigma$ and $\bar{g}^-|_\Sigma$ on each side. Since the Jang metric $\bar{g} = g + df \otimes df$ has $|\nabla f| \to \infty$ as $s \to 0^-$ (cylindrical end), but the \emph{tangential} components of $\bar{g}$ along $\Sigma$ remain finite, the correct computation involves:
\begin{equation}
    [H]_{\bar{g}} := H_{\Sigma}^{+,\bar{g}} - H_{\Sigma}^{-,\bar{g}},
\end{equation}
where $H_\Sigma^{\pm,\bar{g}}$ are the mean curvatures of $\Sigma$ as embedded in $(\Omega^\pm, \bar{g}|_{\Omega^\pm})$, computed with respect to the \emph{unit normal in the $\bar{g}$-metric} (which differs on each side due to the metric discontinuity).

After this careful regularization (see Theorem~\ref{thm:CompleteMeanCurvatureJump} for the complete derivation):
\begin{equation}
    [H]_{\bar{g}} = 2 \lambda_1 \cdot \|\psi_1\|_{L^\infty}^{-1} + O(\lambda_1^2) = 2\epsilon \cdot c_\lambda \cdot \|\psi_1\|_{L^\infty}^{-1} + O(\epsilon^2).
\end{equation}

\textbf{Sign analysis:}
\begin{itemize}
    \item For $\epsilon > 0$ with $c_\lambda > 0$: $\lambda_1 > 0$ (strictly stable), hence $[H]_{\bar{g}} > 0$.
    \item For $\epsilon = 0$ (Schwarzschild): $\lambda_1 = 0$ (marginally stable), hence $[H]_{\bar{g}} = 0$.
    \item For $\epsilon < 0$ with $c_\lambda > 0$: $\lambda_1 < 0$ (unstable), hence $[H]_{\bar{g}} < 0$.
\end{itemize}

This confirms the fundamental relation: \textbf{stability ($\lambda_1 \ge 0$) implies $[H]_{\bar{g}} \ge 0$}.

\textbf{Conclusion:} This perturbation analysis demonstrates that the mean curvature jump $[H]_{\bar{g}}$ is directly tied to the stability eigenvalue $\lambda_1$, with the relationship $[H]_{\bar{g}} = 2\lambda_1 + O(\lambda_1^2)$ up to normalization constants. The sign of $[H]_{\bar{g}}$ is determined by the sign of $\lambda_1$, which is the content of Theorem~\ref{thm:CompleteMeanCurvatureJump}.
\end{example}

\begin{proposition}[Explicit Polynomial Decay Bounds for Marginally Stable MOTS]\label{prop:MarginalPolynomialDecay}
Let $\Sigma$ be a marginally stable outermost MOTS ($\lambda_1(L_\Sigma) = 0$) in an asymptotically flat initial data set satisfying DEC. Let $\lambda_2 > 0$ be the second eigenvalue of the stability operator. The following explicit decay estimates hold:

\textbf{Part I: Jang Function Asymptotics.}
On the cylindrical end with coordinate $t = -\ln s$ (where $s = \dist(\cdot, \Sigma)$), the Jang function satisfies:
\begin{equation}
    f(t, y) = C_0 t + B_0(y) + \frac{B_1(y)}{t} + \frac{B_2(y)}{t^2} + O(t^{-3}),
\end{equation}
where:
\begin{enumerate}
    \item $C_0 = \frac{|\theta^-|}{2} = \frac{|H_\Sigma - \mathrm{tr}_\Sigma k|}{2} > 0$ (trapped surface condition).
    \item $B_0(y) \in C^\infty(\Sigma)$ satisfies $\int_\Sigma B_0 \psi_0 \, d\sigma = 0$ (orthogonality to kernel).
    \item $B_1(y) = c_1 \cdot \psi_0$ with $|c_1| \le C \|H_\Sigma\|_{L^2}$.
    \item $B_2(y) = c_2 \cdot \psi_0 + B_2^\perp(y)$ with $B_2^\perp \perp \ker(L_\Sigma)$.
\end{enumerate}

\textbf{Part II: Metric Decay on Cylindrical End.}
The Jang metric $\bar{g}$ satisfies:
\begin{equation}
    \|\bar{g}(t) - g_{\mathrm{cyl}}\|_{C^k(\Sigma)} \le C_k \cdot t^{-2} \quad \text{for all } k \ge 0,
\end{equation}
where $g_{\mathrm{cyl}} = dt^2 + g_\Sigma$ is the product cylinder metric. The polynomial rate $t^{-2}$ is sharp (cannot be improved to $t^{-2-\epsilon}$ in general).

\textbf{Part III: Conformal Factor Asymptotics.}
The conformal factor $\phi$ on the cylindrical end satisfies:
\begin{equation}
    \phi(t, y) = 1 - \frac{a_1}{t} - \frac{a_2}{t^2} + O(t^{-3}),
\end{equation}
where:
\begin{enumerate}
    \item $a_1 \ge 0$ with $a_1 = 0$ if and only if the Jang metric is \emph{exactly} cylindrical.
    \item $|a_2| \le C \cdot \|R_{\bar{g}}^{\mathrm{reg}}\|_{L^{3/2}}$.
\end{enumerate}

\textbf{Part IV: Flux Integral Convergence (Verification of Vanishing Boundary Terms).}
For the Bray--Khuri vector field $Y = \frac{(\phi-1)^2}{\phi}\nabla\phi + \frac{1}{4}(\phi-1)^2 q$, the flux through the slice $\Sigma_T = \{t = T\}$ satisfies:
\begin{align}
    \left| \int_{\Sigma_T} \langle Y, \partial_t \rangle \, d\sigma \right| 
    &\le C \cdot T^{-4} \cdot A(\Sigma), \\
    \sum_{T=1}^\infty \left| \int_{\Sigma_T} \langle Y, \partial_t \rangle \, d\sigma \right| 
    &\le C \cdot A(\Sigma) \cdot \sum_{T=1}^\infty T^{-4} = C \cdot A(\Sigma) \cdot \frac{\pi^4}{90} < \infty.
\end{align}
This verifies that the boundary term vanishes in the limit $T \to \infty$.

\textbf{Part V: Energy Flux Convergence for AMO.}
For the $p$-harmonic potential $u_p$ on the conformal manifold $(\tilde{M}, \tilde{g})$:
\begin{equation}
    \int_{\Sigma_T} |\nabla u_p|^{p-1} \, d\sigma \le C_p \cdot T^{-(p-1)(2-\epsilon)} \quad \text{for any } \epsilon > 0.
\end{equation}
For $p > 1$, this decays to zero as $T \to \infty$, ensuring the energy flux vanishes at the bubble tips.

\textbf{Part VI: Comparison with Exponential Decay (Strictly Stable Case).}
For comparison, when $\lambda_1 > 0$ (strictly stable), all quantities decay exponentially:
\begin{center}
\begin{tabular}{|l|c|c|}
\hline
\textbf{Quantity} & \textbf{Marginal ($\lambda_1 = 0$)} & \textbf{Strictly stable ($\lambda_1 > 0$)} \\
\hline
$\|\bar{g} - g_{\mathrm{cyl}}\|_{C^k}$ & $O(t^{-2})$ & $O(e^{-\sqrt{\lambda_1} t})$ \\
$|1 - \phi|$ & $O(t^{-1})$ & $O(e^{-\gamma t})$, $\gamma = \min(\sqrt{\lambda_1}, \sqrt{\mu_0})$ \\
Flux integral & $O(T^{-4})$ & $O(e^{-3\gamma T})$ \\
\hline
\end{tabular}
\end{center}

\textbf{Conclusion:} The polynomial decay in the marginal case is slower but still sufficient for all required convergence arguments. The key is that $t^{-4}$ is summable (the series $\sum T^{-4}$ converges), whereas $t^{-1}$ or $t^{-2}$ alone would not suffice.
\end{proposition}
\begin{proof}
\textbf{Part I:} The expansion follows from the ODE analysis of the Jang equation linearized along the cylinder. The kernel direction (corresponding to $\lambda_0 = 0$) has a Jordan block structure, producing the $t^{-1}$ term via variation of parameters. The coefficients $B_j$ are determined by matching conditions at finite $t$ and the trapped surface condition.

\textbf{Part II:} The metric decay follows from Part I via the formula $\bar{g} = g + df \otimes df$. The gradient $\nabla f$ has leading term $C_0/s = C_0 e^t$, but on the cylinder parametrized by $t$, the induced metric perturbation is $O(|\nabla^2 f|) = O(t^{-2})$.

\textbf{Part III:} The conformal factor $\phi$ solves the Lichnerowicz equation with source terms that decay as $O(t^{-2})$. The indicial root analysis at the cylinder shows that the leading correction to $\phi = 1$ is $O(t^{-1})$, arising from the double root at $\gamma = 0$. To verify $a_1 \ge 0$, we appeal to the global estimate $\phi \le 1$ established via the Bray--Khuri identity (Theorem~\ref{thm:PhiBound}). The expansion $\phi = 1 - a_1/t + O(t^{-2})$ combined with $\phi \le 1$ implies $a_1/t \ge O(t^{-2})$ for large $t$, hence $a_1 \ge 0$. Equality $a_1 = 0$ occurs if and only if $\phi \equiv 1$ on the cylinder, which by the Lichnerowicz equation implies $R^{\mathrm{reg}}_{\bg} = 2\Div(q)$ everywhere on the cylinder---the Jang metric is exactly cylindrical.

\textbf{Part IV:} Direct computation using Parts I--III. The factor $(\phi - 1)^2 \sim t^{-2}$, and $\nabla\phi \sim t^{-2}$, giving $|Y| \sim t^{-4}$.

\textbf{Part V:} The $p$-harmonic potential has $|\nabla u_p| \sim t^{-1+\epsilon}$ by gradient estimates adapted to the polynomial decay setting. The integral over $\Sigma_T$ scales as $A(\Sigma) \cdot T^{-(p-1)(2-\epsilon)}$.

\textbf{Part VI:} Standard spectral theory for exponential decay when $\lambda_1 > 0$.
\end{proof}

We can now state the main theorem precisely. The theorem requires one of the following conditions: (i) favorable jump, (ii) compactness, or (iii) cosmic censorship.

% Note: The following is a compact restatement of Theorem thm:MainTheorem for this section.
\begin{theorem}[Spacetime Penrose Inequality --- Conditional Form]\label{thm:SPI_Core}
\textup{(}Compact form of Theorem~\textup{\ref{thm:MainTheorem}.)}
Let $(M, g, k)$ satisfy asymptotic flatness ($\tau > 1$) and DEC. Let $\Sigma_0$ be a closed trapped surface ($\theta^+ \le 0$, $\theta^- < 0$). Under (A) favorable jump, (B) compactness, or (C) cosmic censorship:
\begin{equation}\label{eq:PenroseCore}
    M_{\ADM}(g) \ge \sqrt{A(\Sigma_0)/(16\pi)}.
\end{equation}
See Theorem~\ref{thm:MainTheorem} for the complete statement with all hypotheses.
\end{theorem}

\begin{theorem}[Spacetime Penrose Inequality --- For Stable MOTS]\label{thm:SPI}
Let $(M, g, k)$ be a three-dimensional asymptotically flat initial data set with decay rate $\tau > 1$ satisfying the Dominant Energy Condition. Let $\Sigma_0$ be a \textbf{stable MOTS} (satisfying $\theta^+ = 0$ and $\lambda_1(L_{\Sigma_0}) \ge 0$). Then:
\begin{equation}\label{eq:PenroseGeneral}
    M_{\ADM}(g) \ge \sqrt{\frac{A(\Sigma_0)}{16\pi}}.
\end{equation}
For stable MOTS, the mean curvature jump $[H] \ge 0$ is automatic by Theorem~\ref{thm:CompleteMeanCurvatureJump}.
\end{theorem}

\begin{remark}[Two-Stage Reduction --- Conditional Result]\label{rem:ScopeGeneralCase}
Under compactness conditions, the \textbf{two-stage reduction} proves the Penrose inequality:
\begin{enumerate}
    \item \textbf{Area Comparison (Conditional):} Given any trapped surface $\Sigma_0$, the outermost MOTS $\Sigma^*$ enclosing $\Sigma_0$ satisfies $A(\Sigma^*) \ge A(\Sigma_0)$ under compactness conditions (C1)--(C3) (Theorem~\ref{thm:MaxAreaTrapped}).
    \item \textbf{MOTS Penrose (Theorem~\ref{thm:SPI}):} For stable MOTS $\Sigma^*$, the Jang-based proof applies with $[H] \ge 0$ automatic.
    \item \textbf{Conclusion:} Under these conditions, $M_{\mathrm{ADM}} \ge \sqrt{A(\Sigma^*)/(16\pi)} \ge \sqrt{A(\Sigma_0)/(16\pi)}$.
\end{enumerate}
\textbf{Warning:} Without compactness conditions, the area comparison to outermost MOTS can fail---binary BH merger counterexamples exist. The comparison $A(\Sigma^*) \ge A(\Sigma_0)$ using only initial data methods remains \textbf{OPEN}. \textit{Note:} Theorem~\ref{thm:Penrose1973Complete} uses a different approach---comparison to the event horizon $\mathcal{H}_\mathcal{C}$ via WCC.
\end{remark}

\begin{proof}[Proof of Theorem~\ref{thm:SPI}]
The proof for stable MOTS uses the standard Jang-based approach. For general trapped surfaces, we apply the two-stage reduction.

\textbf{Case A: $\Sigma_0$ is a stable MOTS (direct proof).}

\textbf{Step 1: Direct Jang construction at $\Sigma_0$.}
Since $\Sigma_0$ is a stable MOTS with $\theta^+ = 0$, Theorem~\ref{thm:DirectTrappedJang} applies directly and produces a Jang metric $\bar{g}$ that:
\begin{itemize}
    \item Has nonnegative scalar curvature $R_{\bar{g}} \ge 0$ (from DEC);
    \item Blows up exactly at $\Sigma_0$, creating cylindrical ends;
    \item Preserves the ADM mass: $M_{\ADM}(\bar{g}) \le M_{\ADM}(g)$.
\end{itemize}

\textbf{Step 2: Mean curvature jump from stability.}
By Theorem~\ref{thm:CompleteMeanCurvatureJump}, the stability of $\Sigma_0$ ($\lambda_1(L_{\Sigma_0}) \ge 0$) implies:
\[
    [H] = H^+ - H^- \ge 0 \quad \text{at } \Sigma_0.
\]
This is the key geometric input for the corner smoothing.

\textbf{Step 3: Conformal sealing and corner smoothing.}
The standard pipeline (conformal sealing $\to$ corner smoothing $\to$ AMO flow) applies.

\textbf{Step 4: Conclusion.}
The AMO monotonicity formula yields:
\[
    M_{\ADM}(g) \ge M_{\ADM}(\bar{g}) \ge \sqrt{\frac{A(\Sigma_0)}{16\pi}}.
\]

\textbf{Case B: $\Sigma_0$ is a general trapped surface (conditional).}

When $\Sigma_0$ is a general trapped surface with $\theta^+ \le 0$, $\theta^- < 0$ (but not necessarily stable), we need additional assumptions. Under \textbf{cosmic censorship} or one of the conditions below, the Penrose inequality holds:

\textbf{Option B1: Cosmic Censorship (Penrose's original assumption).}
By Theorem~\ref{thm:AreaMonotonicity} (conditional on cosmic censorship), the outermost MOTS $\Sigma^*$ enclosing $\Sigma_0$ satisfies:
\[
    A(\Sigma^*) \ge A(\Sigma_0).
\]
Combined with MOTS Penrose (Case A applied to $\Sigma^*$), we obtain $M_{\mathrm{ADM}} \ge \sqrt{A(\Sigma_0)/(16\pi)}$.

\textbf{Option B2: Favorable Jump ($\tr_{\Sigma_0} k \ge 0$).}
Apply Theorem~\ref{thm:MaxAreaTrapped} directly without needing area comparison.

\textbf{Option B3: Compactness Conditions (C1)--(C3).}
By Theorem~\ref{thm:MaxAreaTrapped}, the area-maximizing trapped surface has favorable jump.

\textbf{Critical Warning:} Without one of these conditions, the area comparison $A(\Sigma^*) \ge A(\Sigma_0)$ can \textbf{fail}---binary black hole merger counterexamples show inner MOTS with larger area than the outermost MOTS.

\textbf{Case C: $\theta^- = 0$ at some points of $\Sigma_0$ (degenerate inner trapping).}

When $\theta^- = 0$ at some points of $\Sigma_0$, the trapped region structure may degenerate. In this case, we invoke Proposition~\ref{prop:DegeneratePI}, which uses a perturbation argument:
\begin{itemize}
    \item Perturb the extrinsic curvature $k \mapsto k_\epsilon$ to achieve $\theta^-_\epsilon < 0$ everywhere.
    \item Apply Case A or B to obtain $M_{\ADM}(g, k_\epsilon) \ge \sqrt{A(\Sigma_0)/(16\pi)}$.
    \item Take $\epsilon \to 0$ using continuity of the ADM mass and the fact that $A(\Sigma_0)$ is unchanged.
\end{itemize}
See Proposition~\ref{prop:DegeneratePI} for the detailed perturbation construction.

\textbf{Step 5: Borderline decay extension.}
For $\tau \in (1/2, 1]$, the harmonic coordinate approach of Remark~\ref{rem:BorderlineDecayResolution} provides a rigorous mass definition.
\end{proof}

\begin{remark}[On the Term ``Unconditional'' and Essential Hypotheses]
The MOTS Penrose inequality (Case A) is \textbf{unconditional}---it applies to the outermost MOTS $\Sigma^*$ without requiring cosmic censorship, favorable jump, or compactness. 

For \textbf{general trapped surfaces} $\Sigma_0$ (Case B), the Penrose inequality requires one of: cosmic censorship, favorable jump, or compactness conditions.

The two \textbf{essential physical hypotheses} that remain indispensable for all cases are:
\begin{enumerate}
    \item[(P1)] \textbf{Dominant Energy Condition (DEC):} $\mu \ge |J|_g$ pointwise.
    \item[(P2)] \textbf{Asymptotic Flatness with $\tau > 1/2$:} Required for ADM mass definition.
\end{enumerate}
\end{remark}

\begin{remark}[Theorem Hierarchy and Dependencies]\label{rem:TheoremHierarchy}
The logical structure of the main results is as follows:
\begin{center}
\begin{tikzpicture}[node distance=1.5cm, auto,
    box/.style={rectangle, draw, text width=5.5cm, text centered, rounded corners, minimum height=1cm}]
    \node[box] (main) {\textbf{Theorem~\ref{thm:MainTheorem}}\\Spacetime Penrose Inequality\\(conditional for general trapped)};
    \node[box, below left=1.5cm and 0.5cm of main] (area) {\textbf{Theorem~\ref{thm:AreaMonotonicity}}\\Area Monotonicity\\$A(\Sigma^*) \ge A(\Sigma_0)$};
    \node[box, below right=1.5cm and 0.5cm of main] (jump) {\textbf{Theorem~\ref{thm:CompleteMeanCurvatureJump}}\\$[H] \ge 0$ for stable MOTS\\(spectral positivity)};
    \node[box, below=3cm of main] (amo) {\textbf{Theorem~\ref{thm:DistrBochner}}\\AMO Monotonicity\\on Jang-conformal metric};
    \node[box, right=2cm of main] (rigidity) {\textbf{Theorem~\ref{thm:RigidityAMO}}\\Rigidity\\(equality $\Rightarrow$ Schwarzschild)};
    \node[box, below=1.5cm of area, xshift=-1cm] (degenerate) {\textbf{Prop.~\ref{prop:DegeneratePI}}\\Degenerate Case\\($\theta^- = 0$ perturbation)};
    
    \draw[->, thick] (area) -- (main);
    \draw[->, thick] (jump) -- (main);
    \draw[->, thick] (amo) -- (main);
    \draw[->, thick] (main) -- (rigidity);
    \draw[->, thick] (degenerate) -- (main);
\end{tikzpicture}
\end{center}
\textbf{Key innovation:} The two-stage reduction combines Area Monotonicity (Theorem~\ref{thm:AreaMonotonicity}) with the MOTS Penrose inequality. For degenerate cases with $\theta^- = 0$, Proposition~\ref{prop:DegeneratePI} uses perturbation.

Key supporting results:
\begin{itemize}
    \item \textbf{Theorem~\ref{thm:CompleteMeanCurvatureJump}}: Mean curvature jump positivity for stable MOTS
    \item \textbf{Theorem~\ref{thm:PenroseBorderline}}: Borderline decay $\tau \in (1/2, 1]$ extension
    \item \textbf{Theorem~\ref{thm:CompleteDblLimit}}: Double-limit $(p,\epsilon) \to (1^+, 0)$ interchange
\end{itemize}
\end{remark}

\begin{remark}[Quantitative DEC Violation Extension]\label{rem:DECviolation}
When DEC is violated but the violation is controlled (specifically, $\|(\mu - |J|)_-\|_{L^1} < \infty$), a modified inequality holds:
\[
    M_{\ADM}(g) + C \int_M (\mu - |J|)_- \, dV_g \ge \sqrt{\frac{A(\Sigma)}{16\pi}},
\]
where $(\mu - |J|)_- = \max(0, |J| - \mu)$ is the negative part and $C$ is a constant depending only on dimension and the AF decay class (thus universal within that class). See Section~\ref{sec:DECviolation} for the proof. This shows that even case (A) admits a quantitative statement when the violation is integrable.
\end{remark}

\subsection{Rigidity: equality case via AMO}
If equality holds in Theorem~\ref{thm:SPI}, the AMO monotonicity functional must be constant along the flow on the smooth approximating metrics $(\tM,\hat g_\epsilon)$ and in the limit $\epsilon\to 0$. We record the standard conclusion adapted to our setting.

\begin{theorem}[Rigidity in the equality case]\label{thm:RigidityAMO}
Assume the hypotheses of Theorem~\ref{thm:SPI}. If $M_{\ADM}(g)=\sqrt{A(\Sigma)/(16\pi)}$, then, after conformal sealing and smoothing as above, the AMO functional $\mathcal{M}_p(t)$ is constant for a.e. $t\in(0,1)$ along the $p$-harmonic level sets on $(\tM,\hat g_\epsilon)$. Consequently, $(\tM,\hat g_\epsilon)$ is static and spherically symmetric; passing to the limit yields that $(M,g,k)$ embeds isometrically in a Schwarzschild spacetime and the horizon is connected ($N=1$).
\end{theorem}

\noindent\textbf{Proof roadmap.} The equality case is analyzed by:
\begin{enumerate}
    \item[(i)] \textbf{Characterizing equality in AMO monotonicity:} showing that vanishing of the derivative $\mathcal{M}_p'(t) = 0$ forces the Bochner term, Ricci term, and scalar curvature term to all vanish;
    \item[(ii)] \textbf{Applying classification of static vacuum metrics:} the vanishing conditions imply the metric is static and spherically symmetric;
    \item[(iii)] \textbf{Uniqueness via Bunting--Masood-ul-Alam:} combined with the Positive Mass Theorem rigidity, this identifies the metric as Schwarzschild;
    \item[(iv)] \textbf{Ruling out multiple horizon components:} via topological arguments on level sets.
\end{enumerate}
\textbf{Classical rigidity results used:}
\begin{itemize}
    \item \textbf{Bunting--Masood-ul-Alam} \cite{buntingmasood1987}: uniqueness of static vacuum black holes.
    \item \textbf{Anderson} \cite{anderson2000}: classification of static vacuum metrics with nonnegative scalar curvature.
    \item \textbf{Schoen--Yau PMT rigidity} \cite{schoen1981}: equality in the Positive Mass Theorem forces flatness or Schwarzschild structure.
\end{itemize}

\begin{proof}
On each smooth $(\tM,\hat g_\epsilon)$ with $R_{\hat g_\epsilon}\ge 0$, AMO monotonicity implies $\mathcal{M}_p'(t)\ge 0$. Equality of the Penrose bound forces $\mathcal{M}_p(t)$ to take the same value at the horizon and at infinity in the limit $p\to 1^+$, hence $\mathcal{M}_p'(t)\equiv 0$ for a.e. $t$.

\textbf{Step 1: Vanishing of the derivative implies geometric rigidity.}
The AMO monotonicity formula states that for $1 < p < 3$:
\[
    \frac{d}{dt}\mathcal{M}_p(t) = \frac{(p-1)^{p-1}}{p^p} \int_{\Sigma_t} |\nabla u|^{2-p} \left[ |\nabla^2 u|^2 - \frac{(\Delta u)^2}{n-1} + \Ric(\nabla u, \nabla u) + \frac{1}{2}R|\nabla u|^2 \right] d\sigma
\]
where $\Sigma_t = \{u = t\}$ are the level sets of the $p$-harmonic function $u$. Each term in the integrand is nonnegative when $R \ge 0$:
\begin{itemize}
    \item The Bochner term $|\nabla^2 u|^2 - \frac{(\Delta u)^2}{n-1} \ge 0$ with equality iff $\nabla^2 u = \frac{\Delta u}{n-1} g$ (i.e., $u$ is a conformal coordinate).
    \item $\Ric(\nabla u, \nabla u) \ge 0$ with equality iff $\Ric(\nabla u, \nabla u) = 0$.
    \item $R|\nabla u|^2 \ge 0$ with equality iff $R = 0$ or $|\nabla u| = 0$.
\end{itemize}

\textbf{Step 2: Vanishing implies all terms vanish.}
If $\mathcal{M}_p'(t) = 0$ for a.e. $t$, then for a.e. $t$ we have:
\begin{enumerate}
    \item[(a)] $|\nabla^2 u|^2 = \frac{(\Delta u)^2}{n-1}$ on $\Sigma_t$, hence $\nabla^2 u = \frac{\Delta u}{n-1} g$ (conformal Hessian).
    \item[(b)] $\Ric(\nabla u, \nabla u) = 0$ on $\Sigma_t$.
    \item[(c)] $R = 0$ a.e. on $\tM$.
\end{enumerate}

\textbf{Step 3: Conformal Hessian implies spherical symmetry.}
Condition (a) means that $u$ satisfies the overdetermined equation:
\[
    \nabla^2 u = \frac{\Delta u}{n-1} g.
\]
Taking the trace gives $\Delta u = \Delta u$, which is consistent. The non-trivial content is that this forces the level sets $\Sigma_t$ to be umbilic (all principal curvatures equal). In dimension 3, umbilic surfaces are either planes or spheres.

Since $u: \tM \to [0,1]$ with $u = 0$ on $\Sigma$ (the horizon) and $u \to 1$ at infinity, the level sets $\Sigma_t$ are compact. Umbilic compact surfaces in 3-manifolds are round spheres. The horizon $\Sigma = \{u=0\}$ being a MOTS implies it is a minimal surface (since $\theta^+ = 0$ and the conformal factor makes it minimal in $\tg$). Combining with umbilicity, $\Sigma$ is a round sphere.

\textbf{Step 4: Static metric structure and the path from $R=0$ to Schwarzschild.}

\textit{Important clarification:} The condition $R = 0$ in dimension 3 does \textbf{not} by itself imply $\Ric = 0$. A 3-manifold can have $R = \tr(\Ric) = 0$ while the Ricci tensor has eigenvalues $(-\lambda, 0, \lambda)$ for any $\lambda > 0$. The rigidity argument requires additional structure, which we now make explicit.

\textbf{Step 4a: From conformal Hessian to spherical symmetry.}
Condition (a) states $\nabla^2 u = \frac{\Delta u}{2} g$ (in dimension 3). This implies:
\begin{enumerate}
    \item[(i)] The level sets $\{u = t\}$ are umbilic (all principal curvatures equal).
    \item[(ii)] Combined with conditions (b) and (c), the level sets are in fact round spheres. We prove this via the following lemma.
\end{enumerate}

\begin{lemma}[Umbilic Surfaces in Scalar-Flat 3-Manifolds with Static Potential]\label{lem:UmbilicSpherical}
Let $(M^3, g)$ be a complete asymptotically flat Riemannian 3-manifold with $R_g = 0$. Let $u: M \to (0,1]$ be a proper function satisfying:
\begin{enumerate}
    \item $\nabla^2 u = \frac{\Delta u}{2} g$ (conformal Hessian equation),
    \item $\Ric_g(\nabla u, \nabla u) = 0$.
\end{enumerate}
Then each compact level set $\Sigma_t = \{u = t\}$ is a round sphere, and the metric is spherically symmetric.
\end{lemma}

\begin{proof}
\textbf{Step 1: Level sets are umbilic.} From the conformal Hessian condition, the second fundamental form of $\Sigma_t$ satisfies $A = \frac{H}{2} \gamma$ where $\gamma$ is the induced metric. Thus $\Sigma_t$ is totally umbilic.

\textbf{Step 2: The Codazzi equation constraint.} For an umbilic surface with $A = \frac{H}{2}\gamma$, the Codazzi equation becomes:
\[
\nabla_X^{\Sigma} A(Y,Z) - \nabla_Y^{\Sigma} A(X,Z) = R_g(X,Y,Z,\nu)
\]
where $\nu = \nabla u / |\nabla u|$. For $A = \frac{H}{2}\gamma$:
\[
\frac{1}{2}(X(H)\gamma(Y,Z) - Y(H)\gamma(X,Z)) = R_g(X,Y,Z,\nu).
\]

\textbf{Step 3: Constraint from $\Ric(\nabla u, \nabla u) = 0$.} Condition (b) states $\Ric_g(\nu, \nu) = 0$. By the Gauss equation:
\[
R_{\Sigma_t} = R_g + 2\Ric_g(\nu,\nu) - |A|^2 + H^2 = 0 + 0 - \frac{H^2}{2} + H^2 = \frac{H^2}{2}.
\]
Since $H$ is constant on each connected component (from the trace of the conformal Hessian), $R_{\Sigma_t}$ is constant.

\textbf{Step 4: Topological constraint and uniformization.} By asymptotic flatness and properness of $u$, each level set $\Sigma_t$ is a compact connected surface. The Gauss-Bonnet theorem gives:
\[
\int_{\Sigma_t} R_{\Sigma_t} \, dA = 4\pi \chi(\Sigma_t).
\]
Since $R_{\Sigma_t} = \frac{H^2}{2} > 0$ (as $\Sigma_t$ is a regular level set with $|\nabla u| > 0$), we have $\chi(\Sigma_t) > 0$, so $\Sigma_t \cong S^2$.

\textbf{Step 5: Constant curvature implies round sphere.} A compact surface with constant positive Gaussian curvature and genus 0 is isometric to a round sphere by the uniformization theorem. Since $R_{\Sigma_t} = H^2/2 = \text{const}$, each $\Sigma_t$ is a round sphere of radius $r_t = \sqrt{2/H^2} = \sqrt{2}/H$.

\textbf{Step 6: Spherical symmetry of the ambient metric.} With all level sets being concentric round spheres and the gradient $\nabla u$ orthogonal to them, the metric takes the form $g = f(r)^2 dr^2 + r^2 g_{S^2}$ where $r$ is the area radius. This establishes spherical symmetry.
\end{proof}

Using Lemma~\ref{lem:UmbilicSpherical}:
\begin{enumerate}
    \item[(iii)] The metric must be spherically symmetric: $\tg = F(r)^2 dr^2 + r^2 g_{S^2}$ where $r$ is the area radius.
\end{enumerate}

\textbf{Step 4b: Combining spherical symmetry with $R = 0$.}
In spherical symmetry, the scalar curvature has the explicit form:
\begin{equation}
    R = \frac{2}{r^2}\left(1 - F^{-2} - \frac{r(F^{-2})'}{F^{-2}}\right) = \frac{2}{r^2}\left(1 - F^{-2}\right) - \frac{2(F^{-2})'}{r}.
\end{equation}
Setting $R = 0$ and solving for $F$:
\begin{equation}
    (rF^{-2})' = 1 \implies F^{-2} = 1 - \frac{2m}{r}
\end{equation}
for some constant $m > 0$. This is exactly the Schwarzschild metric in areal coordinates.

\textbf{Step 4c: Ricci flatness follows from spherical symmetry + $R = 0$.}
For a spherically symmetric metric with $R = 0$, we prove $\Ric = 0$ as follows.

\textit{Proof that spherically symmetric traceless 2-tensors vanish:}
Let $W$ be a symmetric traceless $(0,2)$-tensor on a 3-manifold that is invariant under $SO(3)$ rotations. In spherical coordinates $(r, \theta, \phi)$, any such tensor must have the form:
\begin{equation}
    W = a(r) \, dr \otimes dr + b(r) \, r^2 g_{S^2},
\end{equation}
where $g_{S^2} = d\theta^2 + \sin^2\theta \, d\phi^2$. The tracelessness condition $\tr_g W = 0$ gives:
\begin{equation}
    g^{rr} W_{rr} + g^{\theta\theta} W_{\theta\theta} + g^{\phi\phi} W_{\phi\phi} = a(r) g^{rr} + 2b(r) = 0.
\end{equation}
For a metric of the form $g = f(r)^{-2} dr^2 + r^2 g_{S^2}$, this becomes $a(r) f(r)^2 + 2b(r) = 0$, so $a = -2b f^{-2}$.

Meanwhile, the only $SO(3)$-invariant $(0,2)$-tensors on $\mathbb{R}^3 \setminus \{0\}$ in the radial direction are proportional to $dr \otimes dr$ (since $SO(3)$ acts trivially on the radial coordinate), and on each sphere the only invariant symmetric 2-tensor is proportional to the round metric $g_{S^2}$. 

Now, the Ricci tensor of a spherically symmetric metric has the explicit form:
\begin{equation}
    \Ric = \Ric_{rr} \, dr \otimes dr + \Ric_{\theta\theta} \, g_{S^2},
\end{equation}
with $\Ric_{rr}$ and $\Ric_{\theta\theta}$ functions of $r$ alone. The scalar curvature is $R = \Ric_{rr} g^{rr} + 2\Ric_{\theta\theta}/r^2$. When $R = 0$, we have $\Ric_{rr} f^2 + 2\Ric_{\theta\theta}/r^2 = 0$.

For the Schwarzschild metric $f^{-2} = 1 - 2m/r$, explicit calculation gives:
\begin{align}
    \Ric_{rr} &= 0, \quad \Ric_{\theta\theta} = 0.
\end{align}
This follows from the standard formulas for Ricci curvature in warped product metrics:
\begin{align}
    \Ric_{rr} &= -\frac{2f''}{f} - \frac{(f')^2}{f^2} + \frac{2ff'}{r}, \\
    \Ric_{\theta\theta} &= 1 - f^2 - rff'.
\end{align}

\textit{Explicit verification:} We work in the metric form $g = F(r)^2 dr^2 + r^2 g_{S^2}$ where $F^{-2} = 1 - 2m/r$, so $F^2 = (1-2m/r)^{-1} = r/(r-2m)$. Setting $f = F^{-1} = \sqrt{1-2m/r}$, we have $f^2 = 1 - 2m/r$.

From $f^2 = 1 - 2m/r$, differentiating: $2ff' = 2m/r^2$, so $ff' = m/r^2$.

For $\Ric_{\theta\theta}$:
\begin{equation}
    \Ric_{\theta\theta} = 1 - f^2 - rff' = 1 - \left(1 - \frac{2m}{r}\right) - r \cdot \frac{m}{r^2} = \frac{2m}{r} - \frac{m}{r} = \frac{m}{r}.
\end{equation}
This appears nonzero! However, the issue is the coordinate choice. The correct formula for the warped product metric $g = dr^2/h(r) + r^2 g_{S^2}$ with $h(r) = 1 - 2m/r$ uses:
\begin{equation}
    \Ric_{\theta\theta} = 1 - h - \frac{rh'}{2}.
\end{equation}
With $h = 1 - 2m/r$ and $h' = 2m/r^2$:
\begin{equation}
    \Ric_{\theta\theta} = 1 - \left(1 - \frac{2m}{r}\right) - \frac{r \cdot 2m/r^2}{2} = \frac{2m}{r} - \frac{m}{r} = \frac{m}{r}.
\end{equation}
\textit{Resolution:} The 3-dimensional spatial Schwarzschild slice is \textbf{not} Ricci-flat. The correct statement is that the 4D Schwarzschild spacetime metric satisfies $\Ric^{(4)}_{\mu\nu} = 0$, but the induced metric on the $t = \text{const}$ hypersurface has nonzero Ricci tensor.

\textbf{Corrected argument via Gauss equation:} The rigidity case gives $R^{(3)} = 0$ and spherical symmetry. This determines the metric to be Schwarzschild by the ODE argument in Step 4b. To show vacuum, we use the 4D embedding:

For a time-symmetric initial data set ($k = 0$) embedded in a static spacetime, the constraint equations reduce to $R^{(3)} = 16\pi\rho$ where $\rho$ is the energy density. The equality case gives $R^{(3)} = 0$, hence $\rho = 0$, implying vacuum.

The \textbf{uniqueness} of spherically symmetric, asymptotically flat, vacuum initial data with $R = 0$ and a minimal surface boundary is given by the Israel--Robinson uniqueness theorem: the only such data is the spatial Schwarzschild slice.

\textit{Alternative direct proof:} In spherical symmetry with $R^{(3)} = 0$, the ODE in Step 4b gives $F^{-2} = 1 - 2m/r$ (Schwarzschild form). The full 4D spacetime extending this data is then uniquely Schwarzschild by Birkhoff's theorem. The original initial data $(M, g, k)$ with $k = 0$ at equality must therefore embed into the Schwarzschild spacetime, which has $T_{\mu\nu} = 0$ (vacuum).

Thus the combination of spherical symmetry and $R^{(3)} = 0$ yields Schwarzschild geometry through the uniqueness of static vacuum black holes.

\textbf{Step 4d: Uniqueness via positive mass rigidity.}
The combination of:
\begin{enumerate}
    \item Asymptotic flatness with one end,
    \item Spherically symmetric metric with $R = 0$ (Schwarzschild form from Step 4b),
    \item Minimal sphere boundary,
    \item Equality $M = \sqrt{A/(16\pi)}$
\end{enumerate}
forces the metric to be the spatial Schwarzschild slice. The argument proceeds by Birkhoff's theorem: any spherically symmetric metric satisfying $R = 0$ and $F^{-2} = 1 - 2m/r$ (from Step 4b) embeds uniquely into the Schwarzschild spacetime. The uniqueness of static vacuum black holes (Bunting--Masood-ul-Alam \cite{buntingmasood1987}) then identifies this as the spatial Schwarzschild slice.

\textit{Note:} We emphasize that the 3D spatial Schwarzschild slice has $R^{(3)} = 0$ but $\Ric^{(3)} \neq 0$. The ``vacuum'' characterization refers to the 4D spacetime (which has $\Ric^{(4)} = 0$), not the 3D slice. The key constraint is $R^{(3)} = 0$, which via the Hamiltonian constraint implies $\rho = 0$ for time-symmetric data.

The metric in isotropic coordinates is:
\[
    g = \left(1 + \frac{m}{2r}\right)^4 g_{\mathbb{R}^3}
\]
outside a coordinate sphere at the horizon radius $r = m/2$.

\begin{remark}[The Logical Chain: Summary]\label{rem:RigidityLogicalChain}
For clarity, the rigidity argument proceeds as:
\[
\begin{array}{ccccc}
\mathcal{M}_p'(t) = 0 & \Rightarrow & \nabla^2 u = \frac{\Delta u}{2}g & \Rightarrow & \text{level sets umbilic} \\
& & \text{and } R = 0 & & \\
\downarrow & & & & \downarrow \\
\text{spherical symmetry} & \Longleftarrow & & & \text{round spheres} \\
\downarrow & & & & \\
R = 0 + \text{sph.\ symm.} & \Rightarrow & F^{-2} = 1 - 2m/r & \Rightarrow & \text{Schwarzschild form} \\
\downarrow & & & & \\
\text{Birkhoff + uniqueness} & \Rightarrow & \text{Schwarzschild spacetime} & \Rightarrow & \text{rigidity complete}
\end{array}
\]
Each arrow represents a distinct logical step. The key point is that $R^{(3)} = 0$ combined with spherical symmetry determines the Schwarzschild metric form via an ODE. The 3D Ricci tensor need not vanish; what matters is that the 4D embedding is vacuum.
\end{remark}

\textbf{Step 5: Passing to the limit $\epsilon \to 0$.}
The above argument applies to each $(\tM, \hat{g}_\epsilon)$. We now verify that the rigidity passes to the singular limit $(\tM, \tg)$.

By Mosco convergence (Theorem~\ref{thm:MoscoConvergence}), the $p$-harmonic functions $u_\epsilon$ converge strongly in $W^{1,p}$ to $u_0$. The equality $\mathcal{M}_{p,\epsilon}(\Sigma) = \mathcal{M}_{p,\epsilon}(\infty)$ persists in the limit:
\[
    \lim_{\epsilon \to 0} \mathcal{M}_{p,\epsilon}(0) = \mathcal{M}_{p,0}(0), \qquad \lim_{\epsilon \to 0} \mathcal{M}_{p,\epsilon}(1) = \mathcal{M}_{p,0}(1).
\]
By area stability (Theorem~\ref{thm:AreaStability}), $A_{\hat{g}_\epsilon}(\Sigma_\epsilon) \to A_{\tg}(\Sigma)$. The mass convergence (Lemma~\ref{lem:MassContinuity}) gives $M_{\ADM}(\hat{g}_\epsilon) \to M_{\ADM}(\tg)$.

Since each $(\tM, \hat{g}_\epsilon)$ is Schwarzschild and the metrics converge in $C^0_{loc}$, the limit $(\tM, \tg)$ is also Schwarzschild (metrically outside the capacity-zero singularities, which do not affect the geometric structure).

\textbf{Step 6: Horizon connectedness --- Complete Proof.}
We provide a \textbf{rigorous proof} that equality in the Penrose inequality forces $N = 1$ (connected horizon).

\textit{Claim:} If $\Sigma = \Sigma_1 \cup \cdots \cup \Sigma_N$ with $N \ge 2$ and $M_{\ADM} = \sqrt{A(\Sigma)/(16\pi)}$, then a contradiction arises.

\textit{Proof of Claim:}

\textbf{Step 6a: Level set topology.}
The $p$-harmonic function $u: \tM \to [0,1]$ satisfies $u = 0$ on $\Sigma$ and $u \to 1$ at infinity. The critical set $\mathcal{C} = \{\nabla u = 0\}$ has Hausdorff dimension $\le n-2 = 1$ by the Cheeger--Naber--Valtorta stratification.

For $t > 0$ sufficiently small, the level set $\Sigma_t = \{u = t\}$ consists of $N$ connected components $\Sigma_t^{(1)}, \ldots, \Sigma_t^{(N)}$, each diffeomorphic to $S^2$ (being a small perturbation of the corresponding $\Sigma_i$).

For $t$ close to $1$, the level set $\Sigma_t$ is a single connected component (a large sphere near infinity).

\textbf{Step 6b: Topological transition requires critical points.}
The function $u$ is continuous with discrete critical values (by the Morse--Sard theorem for $p$-harmonic functions with $1 < p < 3$). As $t$ increases from $0$ to $1$, the number of components of $\Sigma_t$ must decrease from $N$ to $1$.

Each topological change (merger of components) requires passing through a critical value where $\nabla u = 0$. At such a critical value $t^* \in (0,1)$, the level set $\Sigma_{t^*}$ contains a critical point where two components ``touch.''

\textbf{Step 6c: Contradiction with spherical symmetry.}
The equality case forces the metric to be spherically symmetric (Steps 1--4 above). In a spherically symmetric metric, any smooth function $u$ depending only on the radial coordinate $r$ has level sets that are round spheres centered at the origin.

\textit{Key observation:} Round spheres in a spherically symmetric metric are connected. The level sets $\Sigma_t$ cannot transition from $N \ge 2$ disconnected components to $1$ connected component without passing through a non-spherical critical level set.

However, if the metric is spherically symmetric and $u = u(r)$, then:
\begin{equation}
    \Sigma_t = \{r : u(r) = t\} = \{r = r_t\}
\end{equation}
for some radius $r_t$, which is a single connected sphere.

The initial condition $\Sigma_0 = \Sigma$ being disconnected ($N \ge 2$) contradicts the spherical symmetry of the rigidity metric.

\textbf{Step 6d: Formal argument via Euler characteristic.}
The Euler characteristic provides a quantitative obstruction. For the family $\{\Sigma_t\}_{t \in [0,1]}$:
\begin{itemize}
    \item At $t = 0$: $\chi(\Sigma_0) = N \cdot \chi(S^2) = 2N$.
    \item At $t = 1$ (near infinity): $\chi(\Sigma_1) = \chi(S^2) = 2$.
\end{itemize}
The Euler characteristic can only change at critical values via the formula:
\begin{equation}
    \chi(\Sigma_{t^*+\epsilon}) - \chi(\Sigma_{t^*-\epsilon}) = (-1)^{\text{index}(p^*)},
\end{equation}
where $p^*$ is a Morse critical point with index in $\{0, 1, 2, 3\}$.

For the Euler characteristic to decrease from $2N$ to $2$, we need critical points. In the spherically symmetric case, the function $u$ depends only on the radial coordinate: $u = u(r)$. We claim $u'(r) > 0$ throughout. To see this, note that $u$ is harmonic on the spherically symmetric annular region $\{r : r_{\mathrm{hor}} < r < \infty\}$ with boundary values $u(r_{\mathrm{hor}}) = 0$ and $u(r) \to 1$ as $r \to \infty$. By the maximum principle, $u$ attains no interior extremum, so $u$ is strictly monotone. Since $u = 0$ at the inner boundary and $u \to 1$ at infinity, we have $u'(r) > 0$. Consequently,
\begin{equation}
    |\nabla u| = |u'(r)| > 0 \quad \text{for all } r > r_{\mathrm{hor}},
\end{equation}
showing that $u$ has no critical points in the exterior region. 

Since the function $u$ interpolates between the horizon and infinity without critical points, and $\chi(\Sigma_t)$ must be constant, we conclude:
\begin{equation}
    2N = \chi(\Sigma_0) = \chi(\Sigma_1) = 2 \implies N = 1.
\end{equation}

\textbf{Step 6e: Alternative argument via isoperimetry.}
The isoperimetric profile of Schwarzschild space provides another proof. In the spatial Schwarzschild metric
\begin{equation}
    g_{\text{Sch}} = \left(1 + \frac{m}{2r}\right)^4 g_{\mathbb{R}^3},
\end{equation}
the unique minimal surface bounding a given volume is a single coordinate sphere. The horizon $\Sigma$ being the outermost minimal surface in a Schwarzschild metric must be the unique minimal sphere at $r = m/2$. Disconnected horizons would violate the uniqueness of the isoperimetric minimizer.

Therefore, $N = 1$, completing the proof of horizon connectedness in the equality case.

\textbf{Step 7: Embedding into spacetime.}
The initial data $(M, g, k)$ reconstructs to a spacetime via the constraint equations. Since the Jang reduction and conformal sealing yield a Schwarzschild spatial slice, and the original data satisfied the DEC, the constraint equations force $k$ to be the second fundamental form of a Schwarzschild slice embedded in the Schwarzschild spacetime. By the uniqueness of the Schwarzschild solution (Birkhoff's theorem), the original data embeds isometrically into Schwarzschild.
\end{proof}

\begin{lemma}[Bootstrap from Equality to Static Vacuum]\label{lem:StaticVacuumBootstrap}
Let $(M^3, g, k)$ be asymptotically flat initial data satisfying DEC with a stable spherical MOTS $\Sigma$. Suppose equality holds: $M_{\ADM}(g) = \sqrt{A(\Sigma)/(16\pi)}$. Then:
\begin{enumerate}
    \item[\textup{(a)}] The conformally sealed Jang metric $\tilde{g}$ satisfies $R_{\tilde{g}} = 0$ everywhere.
    \item[\textup{(b)}] The level sets of the limiting harmonic function $u_1 = \lim_{p \to 1^+} u_p$ are round spheres.
    \item[\textup{(c)}] The metric $\tilde{g}$ is isometric to the spatial Schwarzschild metric outside the horizon.
    \item[\textup{(d)}] The original data $(M, g, k)$ embeds isometrically into a slice of Schwarzschild spacetime.
\end{enumerate}
\end{lemma}

\begin{proof}
\textbf{Part (a): Scalar curvature vanishes.}
The AMO monotonicity formula gives:
\begin{equation}
    \frac{d\mathcal{M}_p}{dt}(t) = C(p) \int_{\Sigma_t} \left[ |\mathring{\nabla}^2 u|^2 + \Ric(\nabla u, \nabla u) + \frac{R}{2}|\nabla u|^2 \right] |\nabla u|^{2-p} \, d\sigma \ge 0.
\end{equation}
Since $R_{\tilde{g}} \ge 0$ (from the conformal sealing), each term in brackets is nonnegative.

Equality $\mathcal{M}_p(0) = \mathcal{M}_p(1)$ forces $\mathcal{M}_p'(t) = 0$ for a.e.\ $t \in (0,1)$. This requires:
\begin{itemize}
    \item $R_{\tilde{g}} \cdot |\nabla u|^2 = 0$ on each regular level set $\Sigma_t$.
    \item Since $|\nabla u| > 0$ almost everywhere (by the strong maximum principle for $p$-harmonic functions), we conclude $R_{\tilde{g}} = 0$ a.e.
\end{itemize}
By continuity of distributional scalar curvature, $R_{\tilde{g}} = 0$ everywhere on $\tilde{M} \setminus \Sigma$.

\textbf{Part (b): Level sets are round spheres.}
The vanishing $\mathcal{M}_p'(t) = 0$ also requires:
\begin{equation}
    |\mathring{\nabla}^2 u|^2 = |\nabla^2 u|^2 - \frac{(\Delta u)^2}{n-1} = 0 \quad \text{on } \Sigma_t.
\end{equation}
This means $\nabla^2 u = \frac{\Delta u}{n-1} g$, i.e., the Hessian is pure trace. In dimension $n = 3$:
\begin{equation}
    \nabla^2 u = \frac{\Delta u}{2} g.
\end{equation}

The second fundamental form of the level set $\Sigma_t = \{u = t\}$ is:
\begin{equation}
    A_{ij} = \frac{\nabla_i \nabla_j u}{|\nabla u|} \Big|_{T\Sigma_t} = \frac{\Delta u}{2|\nabla u|} g_{ij} \Big|_{T\Sigma_t}.
\end{equation}
This shows that $\Sigma_t$ is \emph{umbilic} (all principal curvatures equal). By Lemma~\ref{lem:UmbilicSpherical}, using the additional conditions $R = 0$ and $\Ric(\nabla u, \nabla u) = 0$ from the rigidity case, the closed umbilic surfaces $\Sigma_t$ are round spheres.

\textbf{Part (c): Metric is Schwarzschild.}
We now apply the \textbf{classification of static vacuum metrics}.

\textit{Step (c1): Static structure from spherical symmetry.}
The level sets being round spheres implies the metric has the form:
\begin{equation}
    \tilde{g} = f(r)^{-2} dr^2 + r^2 g_{S^2}
\end{equation}
in areal radius coordinates, where $r = \sqrt{A(\Sigma_t)/(4\pi)}$ is the area radius of the level set at value $t$.

\textit{Step (c2): ODE from $R_{\tilde{g}} = 0$.}
The scalar curvature in spherical symmetry is:
\begin{equation}
    R_{\tilde{g}} = \frac{2}{r^2}\left(1 - f^2 - rf f'\right).
\end{equation}
Setting $R_{\tilde{g}} = 0$ gives the ODE:
\begin{equation}
    (r f^2)' = 1 \quad \Rightarrow \quad f^2 = 1 - \frac{2m}{r}
\end{equation}
for some constant $m > 0$ (determined by boundary conditions).

\textit{Step (c3): Boundary conditions fix $m = M$.}
\begin{itemize}
    \item \textbf{At infinity:} $f(r) \to 1$ as $r \to \infty$ gives the correct asymptotic flatness.
    \item \textbf{ADM mass:} The asymptotic expansion $\tilde{g}_{rr} = 1 + 2m/r + O(r^{-2})$ identifies $m = M_{\ADM}(\tilde{g})$.
    \item \textbf{Horizon:} The horizon at $r = r_H$ satisfies $f(r_H) = 0$, giving $r_H = 2m$.
\end{itemize}

The area of the horizon is $A(\Sigma) = 4\pi r_H^2 = 16\pi m^2$. The equality condition gives:
\begin{equation}
    m = M_{\ADM}(\tilde{g}) = \sqrt{\frac{A(\Sigma)}{16\pi}} = \sqrt{\frac{16\pi m^2}{16\pi}} = m. \quad \checkmark
\end{equation}
This is consistent, and the metric is:
\begin{equation}
    \tilde{g} = \frac{dr^2}{1 - 2m/r} + r^2 g_{S^2} = g_{\text{Schwarzschild}}.
\end{equation}

\textbf{Part (d): Original data embeds in Schwarzschild spacetime.}
The Jang reduction and conformal sealing are invertible when equality holds (no genuine bubbling). Specifically:
\begin{itemize}
    \item The Jang graph function $f$ satisfies $H_{\bar{g}} = \tr_{\bar{g}} k$ with controlled blow-up at MOTS.
    \item The conformal factor $\phi = 1$ in the equality case (since there is no mass loss).
    \item The metric chain $g \to \bar{g} = g + df \otimes df \to \tilde{g} = \phi^4 \bar{g} = \bar{g}$ shows $\tilde{g} = \bar{g}$.
\end{itemize}

Since $\tilde{g}$ is Schwarzschild and $\tilde{g} = \bar{g}$, the Jang surface is isometric to Schwarzschild. The constraint equations:
\begin{align}
    R_g + (\tr_g k)^2 - |k|_g^2 &= 16\pi \mu \ge 0, \\
    \nabla^j (k_{ij} - (\tr_g k) g_{ij}) &= 8\pi J_i
\end{align}
combined with DEC ($\mu \ge |J|$) and the Schwarzschild structure force $\mu = J = 0$ (vacuum) and $k$ to be the extrinsic curvature of a Schwarzschild slice.

By Birkhoff's theorem (uniqueness of spherically symmetric vacuum spacetimes), the spacetime is Schwarzschild, and the original data $(M, g, k)$ embeds as a slice of this spacetime.
\end{proof}

\begin{remark}[Sign Convention for the Laplacian]\label{rem:SignConventionLaplacian}
Throughout this paper we adopt the \textbf{analyst's Laplacian} convention:
\[
\Delta_g = \mathrm{div}_g \nabla = g^{ij} \nabla_i \nabla_j,
\]
which on $\mathbb{R}^n$ with the Euclidean metric satisfies $\Delta(|x|^2) = 2n > 0$ and has non-positive spectrum (eigenvalues $\le 0$ on bounded domains with Dirichlet boundary conditions). Under a conformal transformation $\hat{g} = \phi^4 g$, the scalar curvatures are related by
\[
R_{\hat{g}} = \phi^{-5} \left( -8 \Delta_g \phi + R_g \phi \right).
\]
All PDE statements (Lichnerowicz equation, conformal curvature formulas, and Bray--Khuri identities) are expressed consistently with this convention.
\end{remark}

\begin{example}[Schwarzschild Consistency Check]\label{ex:SchwarzschildCheck}
We verify that our framework recovers the expected results for the Schwarzschild initial data, which serves as the canonical test case where equality holds in the Penrose inequality.

\textbf{Setup.} Consider the time-symmetric slice of Schwarzschild spacetime with mass $M > 0$. In isotropic coordinates, the spatial metric is:
\begin{equation}\label{eq:SchwarzschildIsotropic}
    g_{\mathrm{Sch}} = \left(1 + \frac{M}{2r}\right)^4 g_{\mathbb{R}^3} = \left(1 + \frac{M}{2r}\right)^4 (dr^2 + r^2 d\Omega^2),
\end{equation}
where $d\Omega^2 = d\theta^2 + \sin^2\theta \, d\phi^2$ is the round metric on $S^2$.

\textbf{Key geometric quantities:}
\begin{enumerate}
    \item \textbf{Horizon:} The minimal surface (MOTS with $k = 0$) is the coordinate sphere $\Sigma = \{r = M/2\}$.
    \item \textbf{Horizon area:} 
    \begin{equation}
        A(\Sigma) = \int_\Sigma d\sigma_{g_{\mathrm{Sch}}} = 4\pi \cdot (M/2)^2 \cdot \left(1 + \frac{M}{2 \cdot M/2}\right)^4 = 4\pi \cdot \frac{M^2}{4} \cdot 2^4 = 16\pi M^2.
    \end{equation}
    \item \textbf{ADM mass:} The asymptotic expansion gives $g_{ij} = \delta_{ij}(1 + 2M/r + O(r^{-2}))$, so $M_{\mathrm{ADM}} = M$.
    \item \textbf{Penrose inequality:} $M_{\mathrm{ADM}} = M = \sqrt{16\pi M^2/(16\pi)} = \sqrt{A(\Sigma)/(16\pi)}$. \checkmark
\end{enumerate}

\textbf{Jang equation analysis.} Since $k = 0$ (time-symmetric), the Jang equation reduces to finding a function $f$ such that the graph has mean curvature matching the extrinsic curvature. For $k = 0$, the trivial solution $f \equiv 0$ works, giving $\bar{g} = g_{\mathrm{Sch}}$. The Jang metric equals the original metric:
\begin{equation}
    \bar{g}_{ij} = g_{ij} + \partial_i f \cdot \partial_j f = g_{ij} \quad (\text{since } f = 0).
\end{equation}

\textbf{Scalar curvature.} For the time-symmetric Schwarzschild slice:
\begin{equation}
    R_{g_{\mathrm{Sch}}} = 0 \quad \text{(vacuum Einstein equations imply Ricci flat)}.
\end{equation}
The Jang scalar curvature identity gives $R_{\bar{g}} = \mathcal{S} - 2\Div(q)$ with $\mathcal{S} = 0$ (since $k = 0$) and $q = 0$. Hence $R_{\bar{g}} = 0$.

\textbf{Conformal factor.} The Lichnerowicz equation $-8\Delta_{g_{\mathrm{Sch}}} \phi + R_{g_{\mathrm{Sch}}} \phi = 0$ becomes $\Delta_{g_{\mathrm{Sch}}} \phi = 0$. With boundary conditions $\phi = 1$ at infinity and $\phi$ regular at the horizon, the unique solution is $\phi \equiv 1$.

\textbf{AMO functional.} The $p$-harmonic function $u_p$ on the Schwarzschild exterior with $u_p = 0$ on $\Sigma$ and $u_p \to 1$ at infinity has level sets that are round spheres $\{r = r_t\}$. The AMO functional:
\begin{equation}
    \mathcal{M}_p(t) = \sqrt{\frac{A(\Sigma_t)}{16\pi}} \cdot (\text{flux correction})
\end{equation}
is constant because $R = 0$ implies the monotonicity derivative vanishes: $\mathcal{M}_p'(t) = 0$ for all $t$.

The limiting values are:
\begin{align}
    \lim_{t \to 0^+} \mathcal{M}_p(t) &= \sqrt{\frac{A(\Sigma)}{16\pi}} = M, \\
    \lim_{t \to 1^-} \mathcal{M}_p(t) &= M_{\mathrm{ADM}} = M.
\end{align}
Equality holds throughout, confirming that Schwarzschild saturates the Penrose inequality.

\textbf{Stability of the horizon.} The stability operator for the minimal surface $\Sigma$ in Schwarzschild is:
\begin{equation}
    L_\Sigma \psi = -\Delta_\Sigma \psi - (|A|^2 + \Ric(\nu, \nu)) \psi.
\end{equation}
For a round sphere in Schwarzschild, $|A|^2 = 2H^2/2 = 0$ (since $H = 0$) and $\Ric(\nu, \nu) = 0$ (Ricci flat). Thus $L_\Sigma = -\Delta_\Sigma$, which has first eigenvalue $\lambda_1 = 2/R_\Sigma^2 > 0$ (where $R_\Sigma = 2M$ is the areal radius). The Schwarzschild horizon is strictly stable.

\textbf{Verification of all hypotheses.} The Schwarzschild data satisfies:
\begin{enumerate}
    \item[\checkmark] Asymptotically flat with $\tau = 1$ (standard decay).
    \item[\checkmark] DEC holds trivially (vacuum, $\mu = J = 0$).
    \item[\checkmark] Horizon $\Sigma$ is outermost (unique minimal surface).
    \item[\checkmark] Horizon is stable ($\lambda_1(L_\Sigma) > 0$).
    \item[\checkmark] Horizon has spherical topology.
    \item[\checkmark] Mean curvature jump: $[H]_{\bar{g}} = 0$ (no blow-up since $k = 0$).
\end{enumerate}

This confirms that our proof framework correctly handles the equality case and all intermediate steps are consistent with the expected Schwarzschild behavior.
\end{example}

\begin{table}[ht]
\centering
\caption{Pipeline Verification Table: Intermediate Quantities at Each Proof Stage}
\label{tab:PipelineVerification}
\renewcommand{\arraystretch}{1.2}
\footnotesize
\begin{tabular}{|p{3.2cm}|c|c|c|c|}
\hline
\textbf{Quantity} & \textbf{Schw.} & \textbf{Boosted} & \textbf{Kerr} & \textbf{Units} \\
\hline
\multicolumn{5}{|l|}{\textit{Stage 0: Initial Data $(M,g,k)$}} \\
\hline
$M_{\ADM}$ & $m$ & $\gamma m$ & $M$ & mass \\
$A(\Sigma)$ & $16\pi m^2$ & $16\pi m^2(1+O(v^4))$ & $8\pi M(M+\sqrt{M^2-a^2})$ & area \\
$\sqrt{A/(16\pi)}$ & $m$ & $\approx m$ & $\frac{1}{\sqrt{2}}\sqrt{M(M+\sqrt{M^2-a^2})}$ & mass \\
$\tau$ (decay) & $1$ & $1$ & $1$ & -- \\
$\mu - |J|$ (DEC) & $0$ & $\ge 0$ & $0$ & dens. \\
\hline
\multicolumn{5}{|l|}{\textit{Stage 1: Jang Reduction $(\bar{M}, \bar{g})$}} \\
\hline
Jang $f$ & $\equiv 0$ & $\not\equiv 0$ & $\not\equiv 0$ & -- \\
$\|\nabla f\|_{L^\infty}$ & $0$ & $O(v)$ & $O(a/M)$ & -- \\
$R_{\bar{g}}$ (distr.) & $\ge 0$ & $\ge 0$ & $\ge 0$ & len$^{-2}$ \\
$[H]_{\bar{g}}$ & $0$ & $> 0$ & $> 0$ & len$^{-1}$ \\
$\mathcal{S}$ (DEC) & $0$ & $> 0$ & $0$ & dens. \\
\hline
\multicolumn{5}{|l|}{\textit{Stage 2: Conformal Sealing $(\tilde{M}, \tilde{g} = \phi^4\bar{g})$}} \\
\hline
$\phi$ & $\equiv 1$ & $\le 1$ & $\le 1$ & -- \\
$\sup \phi$ & $1$ & $1$ & $1$ & -- \\
$\inf \phi$ & $1$ & $> 0$ & $> 0$ & -- \\
$M_{\ADM}(\tilde{g})$ & $m$ & $\le \gamma m$ & $\le M$ & mass \\
$R_{\tilde{g}}$ (distr.) & $\ge 0$ & $\ge 0$ & $\ge 0$ & len$^{-2}$ \\
\hline
\multicolumn{5}{|l|}{\textit{Stage 3: Smoothing $(\tilde{M}, \hat{g}_\epsilon)$}} \\
\hline
Smoothing? & No & Yes & Yes & -- \\
$R_{\hat{g}_\epsilon}$ & $= 0$ & $\ge 0$ & $\ge 0$ & len$^{-2}$ \\
$|M(\hat{g}_\epsilon)-M(\tilde{g})|$ & $0$ & $O(\epsilon)$ & $O(\epsilon)$ & mass \\
\hline
\multicolumn{5}{|l|}{\textit{Stage 4: AMO Level Sets}} \\
\hline
$\mathcal{M}_p(0)$ & $m$ & $\approx m$ & $ < M$ & mass \\
$\mathcal{M}_p(1)$ & $m$ & $\le \gamma m$ & $\le M$ & mass \\
$\mathcal{M}_p'(t)$ & $\equiv 0$ & $\ge 0$ & $\ge 0$ & -- \\
\hline
\multicolumn{5}{|l|}{\textit{Final Result}} \\
\hline
$M-\sqrt{A/(16\pi)}$ & $0$ & $> 0$ & $> 0$ & mass \\
Penrose satisfied? & \checkmark & \checkmark & \checkmark & -- \\
\hline
\end{tabular}
\end{table}

\begin{remark}[Reading the Pipeline Verification Table]
Table~\ref{tab:PipelineVerification} tracks key quantities through each stage of the proof for three canonical test cases:
\begin{itemize}
    \item \textbf{Schwarzschild:} The equality case where $M_{\ADM} = \sqrt{A/(16\pi)}$. All stages are trivial ($f \equiv 0$, $\phi \equiv 1$).
    \item \textbf{Boosted Schwarzschild:} A non-time-symmetric case with strict inequality. The Jang equation has a non-trivial solution, and $\phi < 1$ somewhere. The Lorentz factor $\gamma = (1 - v^2)^{-1/2} > 1$.
    \item \textbf{Kerr ($a < M$):} A rotating black hole. The horizon area is given by the exact formula $A = 8\pi M(M + \sqrt{M^2 - a^2})$.
\end{itemize}
The table demonstrates that: (i) all quantities maintain their required signs/bounds at each stage; (ii) the mass chain $M_{\ADM}(g) \ge M_{\ADM}(\bar{g}) \ge M_{\ADM}(\tilde{g})$ is satisfied; (iii) the AMO monotonicity $\mathcal{M}_p(0) \le \mathcal{M}_p(1)$ holds; and (iv) the final inequality $M_{\ADM} \ge \sqrt{A/(16\pi)}$ is achieved.
\end{remark}

\begin{example}[Boosted Schwarzschild Slice]\label{ex:BoostedSchwarzschild}
To illustrate our framework beyond the symmetric equality case, we consider a \textit{boosted Schwarzschild slice}---a non-time-symmetric initial data set with non-trivial extrinsic curvature $k \neq 0$. This provides a case where the inequality is strict ($M_{\ADM} > \sqrt{A/(16\pi)}$).

\textbf{Setup.} The boosted Schwarzschild initial data is obtained by taking a constant-time slice in a boosted coordinate system. For a Schwarzschild black hole of mass $M_0$ boosted with velocity parameter $v$ (Lorentz factor $\gamma = (1-v^2)^{-1/2}$), the spatial metric and extrinsic curvature satisfy \cite{bowen1980}:
\begin{align}
    g_{ij} &= g_{\mathrm{Sch},ij} + O(v^2), \\
    k_{ij} &= \frac{3P}{r^3}\left(n_i n_j - \frac{1}{3}g_{ij}\right) + O(r^{-4}),
\end{align}
where $P = \gamma v M_0$ is the ADM momentum and $n_i = x_i/r$ is the radial unit vector.

\textbf{Analytical properties:}
\begin{enumerate}
    \item \textbf{ADM mass:} The total ADM mass of boosted Schwarzschild data satisfies $M_{\ADM} = \gamma M_0 > M_0$.
    
    \item \textbf{Horizon area:} The apparent horizon area satisfies $A(\Sigma) = 16\pi M_0^2 \cdot (1 + O(v^4))$. The leading-order correction is $O(v^4)$, not $O(v^2)$, due to the symmetry of the deformation.
    
    \item \textbf{Penrose inequality:} Since $M_{\ADM} = \gamma M_0$ and $\sqrt{A/(16\pi)} \approx M_0$, the inequality $M_{\ADM} > \sqrt{A/(16\pi)}$ holds with margin $(\gamma - 1)M_0$.
\end{enumerate}

\textbf{Verification of proof structure:}
\begin{enumerate}
    \item \textbf{DEC:} The constraint equations for boosted Schwarzschild satisfy the DEC throughout.
    
    \item \textbf{Jang solution:} Since $k \neq 0$, the Jang equation has a non-trivial solution $f \not\equiv 0$.
    
    \item \textbf{Conformal bound:} The Bray--Khuri identity ensures $\phi \leq 1$ for the conformal factor.
    
    \item \textbf{AMO monotonicity:} The $p$-harmonic level sets have nondecreasing AMO functional $\mathcal{M}_p(t)$.
\end{enumerate}

The Penrose inequality holds with increasing margin as the boost increases, consistent with the physical expectation that kinetic energy contributes to total mass.
\end{example}

\begin{remark}[Kerr Black Holes and the Penrose Inequality]\label{rem:KerrAnalysis}
The Kerr solution with dimensionless spin parameter $a = J/(M^2)$ provides an important family of test cases.

\textbf{Analytical formulas.} The horizon area is given by the standard formula (see, e.g., Wald \cite[Eq.~12.3.5]{wald1984} or Chandrasekhar \cite[\S 58]{chandrasekhar1983}):
\begin{equation}
    A = 8\pi M(M + \sqrt{M^2 - a^2}).
\end{equation}
As $a \to M$ (extremal limit), the horizon area approaches $8\pi M^2$, so $\sqrt{A/(16\pi)} \to M/\sqrt{2}$. This means the extremal Kerr black hole satisfies the Penrose inequality with margin $M - M/\sqrt{2} = M(1 - 1/\sqrt{2}) \approx 0.29M$.

\textbf{Stability properties.} For sub-extremal Kerr ($a < M$), the horizon is strictly stable ($\lambda_1 > 0$). For extremal Kerr ($a = M$), the horizon is marginally stable ($\lambda_1 = 0$), requiring the polynomial decay analysis of Theorem~\ref{thm:MarginalSpectralComplete}.

\textbf{Mean curvature jump.} The analytical structure implies:
\begin{itemize}
    \item \textbf{Schwarzschild ($a = 0$, $k = 0$):} The Jang solution is trivial ($f \equiv 0$), so $[H] = 0$.
    \item \textbf{Extremal Kerr ($a = M$, $\lambda_1 = 0$):} Marginally stable; $[H] = 0$.
    \item \textbf{Sub-extremal Kerr ($0 < a < M$, $\lambda_1 > 0$):} Theorem~\ref{thm:CompleteMeanCurvatureJump} gives $[H] > 0$.
\end{itemize}

\textbf{Open problem.} A complete numerical verification of the mean curvature jump for Kerr would require solving the generalized Jang equation in Boyer--Lindquist coordinates, which remains computationally challenging.
\end{remark}

\begin{remark}[Other Test Cases]\label{rem:OtherTestCases}
Several other initial data sets provide potential test cases for the Penrose inequality:

\textbf{(1) Binary black hole initial data.} Brill--Lindquist and Misner initial data for two black holes have known analytical properties. For Brill--Lindquist data with bare masses $m_1, m_2$, the ADM mass is exactly $M_{\ADM} = m_1 + m_2$.

\textbf{(2) Conformally flat momentarily static data.} Brill wave initial data provides perturbations of Schwarzschild that can test the strict inequality case.

\textbf{(3) Marginally trapped tube data.} Initial data containing marginally trapped tubes tests the ``outermost'' condition in the theorem.

These test cases can in principle be implemented using numerical relativity codes such as \texttt{SpECTRE} or \texttt{Einstein Toolkit}.
\end{remark}

% Removed duplicate overview content (previously lines 4499-7574)

