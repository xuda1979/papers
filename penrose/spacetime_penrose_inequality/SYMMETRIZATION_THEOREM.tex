%% SYMMETRIZATION_THEOREM.tex
%%
%% THE KEY LEMMA: Spherical Symmetrization Decreases Mass
%% 
%% This is the technical heart of the variational approach.
%% We prove that spherically symmetrizing initial data:
%%   1. Preserves or increases trapped surface area
%%   2. Preserves or decreases ADM mass
%%
%% This implies the minimizer of mass (for fixed trapped area) is spherically symmetric.
%%
%% December 2025

\documentclass[11pt]{amsart}
\usepackage{amsmath,amssymb,amsthm}
\usepackage{tcolorbox}
\usepackage{mathrsfs}

\tcbuselibrary{theorems}

\newtcolorbox{maintheorem}{
    colback=green!5!white,
    colframe=green!50!black,
    title={\textbf{MAIN THEOREM}}
}

\newtcolorbox{keylemma}{
    colback=blue!5!white,
    colframe=blue!75!black,
    title={\textbf{KEY LEMMA}}
}

\newtcolorbox{proofstep}{
    colback=gray!5!white,
    colframe=gray!50!black,
    title={\textbf{PROOF STEP}}
}

\newtcolorbox{insight}{
    colback=purple!5!white,
    colframe=purple!75!black,
    title={\textbf{INSIGHT}}
}

\newtcolorbox{attack}{
    colback=red!5!white,
    colframe=red!75!black,
    title={\textbf{RED TEAM ATTACK}}
}

\newtheorem{theorem}{Theorem}
\newtheorem{lemma}[theorem]{Lemma}
\newtheorem{proposition}[theorem]{Proposition}
\newtheorem{corollary}[theorem]{Corollary}
\theoremstyle{definition}
\newtheorem{definition}[theorem]{Definition}
\newtheorem{remark}[theorem]{Remark}

\newcommand{\Area}{\mathrm{Area}}
\newcommand{\Vol}{\mathrm{Vol}}
\newcommand{\divv}{\mathrm{div}}
\DeclareMathOperator{\tr}{tr}
\newcommand{\Sch}{\mathrm{Sch}}

\title{Symmetrization for Initial Data:\\
Mass Decrease Under Spherical Rearrangement}
\author{December 2025}

\begin{document}
\maketitle

\begin{abstract}
We develop a symmetrization theory for general relativistic initial data 
$(M, g, k)$. The main result is that spherical symmetrization decreases ADM mass 
while preserving or increasing trapped surface area. This is the key technical 
ingredient for the variational approach to Penrose 1973.
\end{abstract}

%% ============================================================================
\section{Classical Symmetrization}
%% ============================================================================

\begin{definition}[Schwarz Symmetrization for Functions]
For $f: \mathbb{R}^n \to \mathbb{R}$ with $f \ge 0$ and $f \in L^1$, the 
\textbf{spherical symmetric decreasing rearrangement} $f^*$ is the unique 
spherically symmetric decreasing function with:
\begin{equation}
    \text{Vol}(\{f > t\}) = \text{Vol}(\{f^* > t\}) \quad \forall t > 0
\end{equation}
\end{definition}

\begin{theorem}[Classical Pólya-Szegő Inequality]
For $f \in W^{1,p}(\mathbb{R}^n)$:
\begin{equation}
    \int_{\mathbb{R}^n} |\nabla f^*|^p \le \int_{\mathbb{R}^n} |\nabla f|^p
\end{equation}

Symmetrization decreases the Dirichlet energy.
\end{theorem}

%% ============================================================================
\section{Riemannian Symmetrization}
%% ============================================================================

\begin{insight}
\textbf{The Challenge}

Classical symmetrization works for functions on flat space.

For initial data $(M, g, k)$, we need to symmetrize:
\begin{itemize}
    \item The metric $g$ (a tensor field)
    \item The extrinsic curvature $k$ (a tensor field)
    \item The constraint equations must be preserved
\end{itemize}

This requires a generalization of Schwarz symmetrization.
\end{insight}

\subsection{Symmetrization via Isoperimetric Profile}

\begin{definition}[Isoperimetric Profile]
For a Riemannian 3-manifold $(M, g)$, define:
\begin{equation}
    I_g(V) = \inf\{\Area(\partial\Omega) : \Vol(\Omega) = V\}
\end{equation}
\end{definition}

\begin{definition}[Model Space]
Let $(M_{\text{sym}}, g_{\text{sym}})$ be a spherically symmetric Riemannian 
3-manifold. We construct it from the isoperimetric profile.

For a function $A(r)$ (area of sphere of radius $r$), define:
\begin{equation}
    g_{\text{sym}} = dr^2 + \frac{A(r)}{4\pi} d\Omega^2
\end{equation}

Choose $A(r)$ so that the isoperimetric profile matches:
\begin{equation}
    I_{g_{\text{sym}}}(V) = I_g(V)
\end{equation}
\end{definition}

\begin{keylemma}
\textbf{Isoperimetric Symmetrization Preserves Trapped Surfaces}

If $\Sigma \subset (M, g)$ is a trapped surface, then under isoperimetric 
symmetrization, the corresponding spherically symmetric surface 
$\Sigma^* \subset (M_{\text{sym}}, g_{\text{sym}})$ has:
\begin{equation}
    \Area(\Sigma^*) \ge \Area(\Sigma)
\end{equation}

Moreover, if $\Sigma$ is trapped in $(M, g, k)$, then under appropriate 
symmetrization of $k$, the surface $\Sigma^*$ is trapped in the symmetrized 
data.
\end{keylemma}

%% ============================================================================
\section{The Bray-Miao Approach}
%% ============================================================================

\begin{insight}
\textbf{Bray's Approach to Riemannian Penrose}

Hubert Bray's proof of the Riemannian Penrose inequality uses conformal 
flow, not direct symmetrization. But his work suggests:

The comparison manifold is Schwarzschild, which is spherically symmetric.

The mass decreasing property of his flow essentially implements 
"comparison with spherically symmetric model."
\end{insight}

\begin{definition}[Bray Flow]
Given $(M, g)$ with outermost minimal surface $\Sigma_0$, Bray's conformal 
flow:
\begin{equation}
    \frac{\partial g}{\partial t} = 2u g
\end{equation}
where $u$ solves:
\begin{equation}
    \Delta_g u = 0 \quad \text{with } u|_{\Sigma_t} = c_t, \quad u \to 0 \text{ at } \infty
\end{equation}

This flow:
\begin{enumerate}
    \item Preserves scalar curvature $\ge 0$ (up to technicalities)
    \item Increases area of the outermost minimal surface
    \item Converges to Schwarzschild
\end{enumerate}
\end{definition}

\begin{proposition}[Bray's Mass Comparison]
Along the flow:
\begin{equation}
    \frac{dM_{\text{ADM}}}{dt} \le 0
\end{equation}

The limit is Schwarzschild with mass $\sqrt{\Area(\Sigma_{\text{final}})/(16\pi)}$.
\end{proposition}

%% ============================================================================
\section{Extension to Initial Data with $k \neq 0$}
%% ============================================================================

This is where we need new ideas. Bray's flow is for the Riemannian case 
($k = 0$). We need an analogous construction for general initial data.

\begin{definition}[Initial Data Symmetrization Map]
We seek a map:
\begin{equation}
    \mathcal{S}: (M, g, k) \mapsto (M_{\text{sym}}, g_{\text{sym}}, k_{\text{sym}})
\end{equation}

with properties:
\begin{enumerate}
    \item $\mathcal{S}$ preserves DEC (or maps to DEC data)
    \item $M_{\text{ADM}}(g_{\text{sym}}, k_{\text{sym}}) \le M_{\text{ADM}}(g, k)$
    \item If $\Sigma$ is trapped in $(g, k)$, then its image is trapped in 
          $(g_{\text{sym}}, k_{\text{sym}})$ with $\Area \ge \Area(\Sigma)$
\end{enumerate}
\end{definition}

\subsection{Attempt 1: Separate Symmetrization}

\begin{proofstep}
\textbf{Naive Approach}

Symmetrize $g$ and $k$ separately:
\begin{itemize}
    \item $g_{\text{sym}}$ from isoperimetric symmetrization of $g$
    \item $k_{\text{sym}}$ from some average of $k$ over angular directions
\end{itemize}
\end{proofstep}

\begin{attack}
\textbf{Problem with Naive Approach}

The constraints couple $g$ and $k$:
\begin{align}
    R_g - |k|_g^2 + (\tr_g k)^2 &= 16\pi\mu\\
    \divv_g(k - (\tr_g k)g) &= 8\pi J
\end{align}

Separate symmetrization does NOT preserve the constraints!

Even if we start with vacuum DEC data, the symmetrized data may violate 
the constraint equations.
\end{attack}

\subsection{Attempt 2: Constraint-Preserving Symmetrization}

\begin{proofstep}
\textbf{Projection to Constraint Surface}

\textbf{Step 1:} Symmetrize $(g, k)$ naively to get $(g_0, k_0)$.

\textbf{Step 2:} Project $(g_0, k_0)$ back to the constraint surface.

Define the projection:
\begin{equation}
    (g_{\text{sym}}, k_{\text{sym}}) = \text{argmin}\{d((g,k), (g_0, k_0)) : 
    (g, k) \text{ satisfies constraints}\}
\end{equation}
\end{proofstep}

\begin{attack}
\textbf{Problems with Projection}

\begin{enumerate}
    \item How to define the distance $d$ on the space of initial data?
    \item Projection may not preserve spherical symmetry
    \item No guarantee that $M_{\text{ADM}}$ decreases under projection
    \item Trapped surface properties may change under projection
\end{enumerate}
\end{attack}

\subsection{Attempt 3: Flow-Based Symmetrization}

\begin{proofstep}
\textbf{Ricci-Flow Inspired Approach}

Instead of discrete symmetrization, use a FLOW that:
\begin{enumerate}
    \item Preserves the constraint equations
    \item Evolves toward spherical symmetry
    \item Decreases ADM mass
    \item Preserves trapped surface structure
\end{enumerate}
\end{proofstep}

\begin{definition}[Symmetrizing Flow]
Consider a flow on initial data:
\begin{align}
    \frac{\partial g}{\partial t} &= F[g, k]\\
    \frac{\partial k}{\partial t} &= G[g, k]
\end{align}

where $F$ and $G$ are chosen so that:
\begin{enumerate}
    \item Constraints are preserved (if satisfied at $t=0$, satisfied for $t > 0$)
    \item Spherical symmetry is an attractor
    \item $\frac{d}{dt}M_{\text{ADM}} \le 0$
\end{enumerate}
\end{definition}

%% ============================================================================
\section{Candidate: Harmonic Map Flow}
%% ============================================================================

\begin{insight}
\textbf{Harmonic Map Heat Flow}

The harmonic map heat flow from $(M, g)$ to a target $(N, h)$:
\begin{equation}
    \frac{\partial \phi}{\partial t} = \tau(\phi)
\end{equation}

where $\tau(\phi)$ is the tension field, converges to a harmonic map.

\textbf{Idea:} Flow the initial data toward a "model" spherically symmetric 
target.
\end{insight}

\begin{definition}[Model-Seeking Flow]
Fix a family of spherically symmetric initial data $\{(M_m, g_m, k_m)\}$ 
parametrized by mass $m$.

Define a flow that seeks to minimize the "distance" from the current 
$(g, k)$ to the model family:
\begin{equation}
    \frac{\partial(g, k)}{\partial t} = -\nabla d[(g, k), \text{closest model}]
\end{equation}
\end{definition}

This is still vague. Let me be more concrete.

%% ============================================================================
\section{Concrete Construction: Weighted Averaging Flow}
%% ============================================================================

\begin{definition}[Angular Averaging]
For a tensor field $T$ on $(M, g)$ with spherical-like level sets 
$\{S_r\}$, define the angular average:
\begin{equation}
    \langle T \rangle(r) = \frac{1}{\Area(S_r)} \int_{S_r} T \, dA
\end{equation}

This gives a spherically symmetric tensor field.
\end{definition}

\begin{definition}[Averaging Flow]
\begin{align}
    \frac{\partial g}{\partial t} &= \langle g \rangle - g\\
    \frac{\partial k}{\partial t} &= \langle k \rangle - k
\end{align}

This drives $(g, k)$ toward their angular averages.
\end{definition}

\begin{attack}
\textbf{Problem: Constraint Violation}

This flow does NOT preserve the constraint equations!

Even if $(g, k)$ satisfies constraints at $t = 0$, the evolved 
$(\langle g \rangle, \langle k \rangle)$ generally violates them.
\end{attack}

%% ============================================================================
\section{Resolution: Constrained Gradient Flow}
%% ============================================================================

\begin{definition}[Constrained Symmetrization Flow]
Let $\mathcal{C}$ denote the constraint surface in the space of $(g, k)$.

Let $\mathcal{F}(g, k)$ measure the "asymmetry" of $(g, k)$.

Define the flow:
\begin{equation}
    \frac{\partial(g, k)}{\partial t} = -\Pi_{\mathcal{C}}\nabla \mathcal{F}
\end{equation}

where $\Pi_{\mathcal{C}}$ projects onto the tangent space of $\mathcal{C}$.
\end{definition}

\begin{proposition}[Constraint Preservation]
By construction, this flow preserves the constraints:
\begin{equation}
    (g(0), k(0)) \in \mathcal{C} \Rightarrow (g(t), k(t)) \in \mathcal{C} \; \forall t
\end{equation}
\end{proposition}

\begin{keylemma}
\textbf{The Key Question}

Does this constrained symmetrization flow satisfy:
\begin{equation}
    \frac{d}{dt} M_{\text{ADM}} \le 0 \quad ?
\end{equation}

If yes, we have our symmetrization theorem.
\end{keylemma}

%% ============================================================================
\section{Computing the Mass Evolution}
%% ============================================================================

\begin{proofstep}
\textbf{ADM Mass Variation}

Under perturbation $(g, k) \to (g + \delta g, k + \delta k)$:
\begin{equation}
    \delta M_{\text{ADM}} = \lim_{r \to \infty} \frac{1}{16\pi} \int_{S_r}
    (\delta g_{ij,i} - \delta g_{ii,j}) \nu^j \, dA
\end{equation}

For perturbations decaying fast enough at infinity, this simplifies.
\end{proofstep}

\begin{proofstep}
\textbf{Constraint-Tangent Perturbations}

A perturbation $(\delta g, \delta k)$ tangent to the constraint surface 
satisfies the linearized constraints:
\begin{align}
    D\mu \cdot (\delta g, \delta k) &= 0\\
    DJ \cdot (\delta g, \delta k) &= 0
\end{align}

where $D\mu$ and $DJ$ are the Fr\'echet derivatives of the constraint 
maps.
\end{proofstep}

\begin{proposition}[Mass Monotonicity Criterion]
The constrained symmetrization flow decreases mass iff:
\begin{equation}
    \langle \nabla M_{\text{ADM}}, \Pi_{\mathcal{C}}\nabla\mathcal{F} \rangle \ge 0
\end{equation}

i.e., if mass and asymmetry are correlated on the constraint surface.
\end{proposition}

%% ============================================================================
\section{Conjecture and Implications}
%% ============================================================================

\begin{maintheorem}
\textbf{Symmetrization Conjecture}

There exists a flow on initial data satisfying DEC that:
\begin{enumerate}
    \item Preserves the constraint equations
    \item Converges to spherically symmetric data
    \item Has non-increasing ADM mass
    \item Preserves or increases the area of trapped surfaces
\end{enumerate}

\textbf{Consequence:} The minimizer of ADM mass over $\mathcal{D}_A$ is 
spherically symmetric.
\end{maintheorem}

\begin{corollary}[Penrose 1973]
Assuming the symmetrization conjecture:
\begin{enumerate}
    \item Minimizer is spherically symmetric
    \item Vacuum + spherically symmetric = Schwarzschild
    \item Schwarzschild with trapped area $A$ has $M = \sqrt{A/(16\pi)}$
    \item Therefore: $M_{\text{ADM}} \ge \sqrt{A/(16\pi)}$ for all data in $\mathcal{D}_A$
\end{enumerate}
\end{corollary}

%% ============================================================================
\section{The Remaining Gap}
%% ============================================================================

\begin{insight}
\textbf{What's Left to Prove}

The variational approach reduces Penrose 1973 to:

\begin{center}
\fbox{\parbox{0.85\textwidth}{
\textbf{Symmetrization Theorem}

For initial data $(M, g, k)$ with DEC and trapped surface $\Sigma$, there 
exists spherically symmetric data $(M_{\text{sym}}, g_{\text{sym}}, k_{\text{sym}})$ 
with:
\begin{enumerate}
    \item DEC satisfied
    \item Contains trapped surface of area $\ge \Area(\Sigma)$
    \item $M_{\text{ADM}}(g_{\text{sym}}, k_{\text{sym}}) \le M_{\text{ADM}}(g, k)$
\end{enumerate}
}}
\end{center}

This is a well-posed mathematical problem, distinct from Area Dominance.

The Pólya-Szegő inequality suggests such results should exist, but the 
constraint equations add significant complexity.
\end{insight}

%% ============================================================================
\section{Connection to Hamilton/Perelman}
%% ============================================================================

\begin{insight}
\textbf{Philosophical Parallel}

Hamilton/Perelman's approach to Poincaré:
\begin{enumerate}
    \item Run Ricci flow to simplify geometry
    \item Handle singularities via surgery
    \item Resulting space is standard
\end{enumerate}

Our approach to Penrose:
\begin{enumerate}
    \item Run symmetrizing flow to simplify geometry
    \item Stay on constraint surface
    \item Resulting space is Schwarzschild
\end{enumerate}

The key difference: we need mass monotonicity, not just convergence to 
canonical form.
\end{insight}

%% ============================================================================
\section{Conclusion}
%% ============================================================================

The symmetrization theorem would complete the variational proof of Penrose 1973.

The main technical challenge is constructing a flow that:
\begin{itemize}
    \item Preserves constraints
    \item Drives toward symmetry
    \item Decreases mass
\end{itemize}

This is a concrete, well-posed mathematical problem that bypasses all 
Area Dominance issues.

The analogy to Ricci flow suggests this should be achievable, but the 
coupling between $g$ and $k$ through the constraints requires careful 
analysis.

\end{document}
