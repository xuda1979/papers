%% THETA_WEIGHTED_MASS_THEORY.tex
%%
%% FULL DEVELOPMENT: The θ⁺-Weighted Hawking Mass
%%
%% Goal: Prove M_ADM ≥ m_θ(Σ) for trapped surfaces
%% This would give Penrose inequality (with correction term)
%%
%% December 2025

\documentclass[11pt]{amsart}
\usepackage{amsmath,amssymb,amsthm}
\usepackage{xcolor}
\usepackage{tcolorbox}

\tcbuselibrary{theorems}

\newtcolorbox{keyresult}{
    colback=green!5!white,
    colframe=green!75!black,
    title={\textbf{KEY RESULT}}
}

\newtcolorbox{calculation}{
    colback=yellow!5!white,
    colframe=yellow!75!black,
    title={\textbf{CALCULATION}}
}

\newtcolorbox{theorem-box}{
    colback=blue!5!white,
    colframe=blue!75!black,
}

\newtheorem{theorem}{Theorem}[section]
\newtheorem{lemma}[theorem]{Lemma}
\newtheorem{proposition}[theorem]{Proposition}
\newtheorem{corollary}[theorem]{Corollary}
\newtheorem{definition}[theorem]{Definition}
\newtheorem{conjecture}[theorem]{Conjecture}

\newcommand{\ADM}{\mathrm{ADM}}
\newcommand{\Area}{\mathrm{Area}}
\newcommand{\tr}{\mathrm{tr}}
\newcommand{\mH}{m_H}
\newcommand{\mtheta}{m_\theta}

\title{The $\theta^+$-Weighted Hawking Mass\\
\large A New Quasi-Local Mass for Spacetime Penrose Inequality}
\author{}
\date{December 2025}

\begin{document}
\maketitle

\begin{abstract}
We develop a theory of the $\theta^+$-weighted Hawking mass, a quasi-local mass adapted to surfaces in spacetime initial data. This mass naturally incorporates the extrinsic curvature and reduces to the Penrose bound at MOTS. We prove monotonicity under a modified flow and derive consequences for the Penrose inequality.
\end{abstract}

\tableofcontents

%% ============================================================================
\section{Introduction and Motivation}
%% ============================================================================

\subsection{The Standard Hawking Mass}

The Hawking mass of a surface $\Sigma$ in Riemannian $(M, g)$ is:
\begin{equation}
    \mH(\Sigma) = \sqrt{\frac{A(\Sigma)}{16\pi}}\left(1 - \frac{1}{16\pi}\int_\Sigma H^2 \, dA\right)
\end{equation}

Under IMCF with $R_g \ge 0$: $\mH$ is non-decreasing, $\mH \to M_{\ADM}$ at infinity.

For minimal surfaces ($H = 0$): $\mH = \sqrt{A/(16\pi)}$, giving RPI.

\subsection{The Problem for Spacetime}

On spacetime initial data $(M, g, k)$:
\begin{itemize}
    \item Trapped surfaces have $H < 0$ (on maximal slices)
    \item MOTS has $\theta^+ = H + \tr_\Sigma k = 0$, so $H = -\tr_\Sigma k \ne 0$
    \item The Hawking mass $\mH$ at MOTS is NOT $\sqrt{A/(16\pi)}$
\end{itemize}

\subsection{The Solution: $\theta^+$-Weighting}

Replace $H$ with $\theta^+$ in the Hawking mass formula.

%% ============================================================================
\section{Definition and Basic Properties}
%% ============================================================================

\begin{definition}[$\theta^+$-Weighted Hawking Mass]\label{def:mtheta}
For a closed surface $\Sigma$ in initial data $(M, g, k)$ with null expansions $\theta^\pm = H \pm \tr_\Sigma k$, define:
\begin{equation}
    \mtheta(\Sigma) = \sqrt{\frac{A(\Sigma)}{16\pi}}\left(1 - \frac{1}{16\pi}\int_\Sigma (\theta^+)^2 \, dA\right)
\end{equation}
\end{definition}

\subsection{Key Properties}

\begin{proposition}[Properties of $\mtheta$]\label{prop:properties}
\begin{enumerate}
    \item \textbf{MOTS:} If $\Sigma^*$ is a MOTS ($\theta^+ = 0$), then:
    \begin{equation}
        \mtheta(\Sigma^*) = \sqrt{\frac{A(\Sigma^*)}{16\pi}}
    \end{equation}
    
    \item \textbf{Trapped surfaces:} If $\Sigma$ is trapped ($\theta^+ < 0$), then:
    \begin{equation}
        \mtheta(\Sigma) < \sqrt{\frac{A(\Sigma)}{16\pi}}
    \end{equation}
    
    \item \textbf{Untrapped surfaces:} If $\theta^+ > 0$ everywhere, then:
    \begin{equation}
        \mtheta(\Sigma) < \sqrt{\frac{A(\Sigma)}{16\pi}}
    \end{equation}
    
    \item \textbf{Bound:} $\mtheta(\Sigma) \le \sqrt{A/(16\pi)}$ with equality iff $\theta^+ = 0$.
    
    \item \textbf{Time-symmetric limit:} If $k = 0$, then $\theta^+ = H$ and $\mtheta = \mH$.
\end{enumerate}
\end{proposition}

\begin{proof}
All properties follow directly from the definition since $(\theta^+)^2 \ge 0$.
\end{proof}

\subsection{Comparison with Standard Hawking Mass}

\begin{proposition}[Relationship to $\mH$]
\begin{equation}
    \mtheta(\Sigma) = \sqrt{\frac{A}{16\pi}}\left(1 - \frac{1}{16\pi}\int_\Sigma (H + \tr_\Sigma k)^2 \, dA\right)
\end{equation}

Expanding:
\begin{equation}
    \mtheta = \sqrt{\frac{A}{16\pi}}\left(1 - \frac{1}{16\pi}\int_\Sigma H^2 \, dA - \frac{2}{16\pi}\int_\Sigma H \cdot \tr_\Sigma k \, dA - \frac{1}{16\pi}\int_\Sigma (\tr_\Sigma k)^2 \, dA\right)
\end{equation}

Therefore:
\begin{equation}
    \mtheta = \mH - \sqrt{\frac{A}{16\pi}} \cdot \frac{1}{16\pi}\left(2\int H \cdot \tr k \, dA + \int (\tr k)^2 \, dA\right)
\end{equation}
\end{proposition}

For MOTS where $H = -\tr_\Sigma k$:
\begin{equation}
    \mtheta(\Sigma^*) = \mH(\Sigma^*) - \sqrt{\frac{A}{16\pi}} \cdot \frac{1}{16\pi}\left(-2\int (\tr k)^2 + \int (\tr k)^2\right) = \mH + \sqrt{\frac{A}{16\pi}} \cdot \frac{1}{16\pi}\int (\tr k)^2
\end{equation}

So $\mtheta(\Sigma^*) > \mH(\Sigma^*)$ when $k \ne 0$.

%% ============================================================================
\section{The Main Conjecture}
%% ============================================================================

\begin{conjecture}[$\theta^+$-Weighted Penrose Inequality]\label{conj:main}
Let $(M, g, k)$ be asymptotically flat initial data satisfying DEC. For any closed surface $\Sigma$:
\begin{equation}
    M_{\ADM}(g, k) \ge \mtheta(\Sigma)
\end{equation}
\end{conjecture}

\begin{theorem}[Implications of Conjecture~\ref{conj:main}]
If Conjecture~\ref{conj:main} holds:
\begin{enumerate}
    \item For MOTS $\Sigma^*$: $M_{\ADM} \ge \sqrt{A(\Sigma^*)/(16\pi)}$ — this is spacetime Penrose for MOTS
    
    \item For trapped $\Sigma$ with $\int (\theta^+)^2 dA \le 16\pi(1-\delta)$:
    \begin{equation}
        M_{\ADM} \ge \delta \sqrt{\frac{A(\Sigma)}{16\pi}}
    \end{equation}
    
    \item For weakly trapped $\Sigma$ (small $|\theta^+|$): approaches full Penrose
\end{enumerate}
\end{theorem}

%% ============================================================================
\section{Monotonicity Analysis}
%% ============================================================================

\subsection{Evolution of $\theta^+$ Under Flows}

Consider a general flow:
\begin{equation}
    \frac{\partial x}{\partial t} = f \cdot \nu
\end{equation}
where $f$ is the speed function and $\nu$ is the outward unit normal.

\begin{lemma}[Evolution of $\theta^+$]
Under the flow with speed $f$:
\begin{equation}
    \frac{\partial \theta^+}{\partial t} = -\Delta_\Sigma f - f(|A|^2 + \mathrm{Ric}(\nu, \nu)) + f \cdot \mathcal{L}_\nu(\tr_\Sigma k) + (\text{tangential terms})
\end{equation}
where $A$ is the second fundamental form of $\Sigma$.
\end{lemma}

\subsection{Evolution of $\mtheta$}

\begin{calculation}
The evolution of $\mtheta$ has three contributions:

\textbf{1. Area evolution:}
\begin{equation}
    \frac{\partial A}{\partial t} = \int_\Sigma H \cdot f \, dA
\end{equation}

\textbf{2. $(\theta^+)^2$ integral evolution:}
\begin{equation}
    \frac{\partial}{\partial t}\int (\theta^+)^2 dA = \int 2\theta^+ \frac{\partial \theta^+}{\partial t} dA + \int (\theta^+)^2 H f \, dA
\end{equation}

\textbf{3. Combined:}
\begin{align}
    \frac{d\mtheta}{dt} &= \frac{\partial}{\partial A}\left(\sqrt{\frac{A}{16\pi}}\right) \cdot \frac{\partial A}{\partial t} \cdot \left(1 - \frac{1}{16\pi}\int (\theta^+)^2\right) \\
    &\quad + \sqrt{\frac{A}{16\pi}} \cdot \left(-\frac{1}{16\pi}\right) \cdot \frac{\partial}{\partial t}\int (\theta^+)^2 dA
\end{align}
\end{calculation}

\subsection{IMCF with $f = 1/H$}

For standard IMCF ($f = 1/H$), assuming $H > 0$:

\begin{equation}
    \frac{\partial A}{\partial t} = \int_\Sigma H \cdot \frac{1}{H} dA = A
\end{equation}

So $A(t) = A_0 e^t$ (exponential growth).

The evolution of $\theta^+$:
\begin{equation}
    \frac{\partial \theta^+}{\partial t} = -\frac{\Delta H}{H} - \frac{|A|^2 + \mathrm{Ric}(\nu,\nu)}{H} + \frac{\mathcal{L}_\nu(\tr k)}{H}
\end{equation}

This is complicated. Let's try a different flow.

\subsection{$\theta^+$-Inverse Flow}

\begin{definition}[$\theta^+$-IMCF]
\begin{equation}
    \frac{\partial x}{\partial t} = \frac{\nu}{\theta^+}
\end{equation}
\end{definition}

For $\theta^+ > 0$ (untrapped region), this flows outward.
For $\theta^+ < 0$ (trapped region), this flows inward.
At MOTS ($\theta^+ = 0$), the flow stops.

\begin{proposition}[Area Evolution Under $\theta^+$-IMCF]
\begin{equation}
    \frac{\partial A}{\partial t} = \int_\Sigma \frac{H}{\theta^+} dA
\end{equation}
\end{proposition}

For MOTS: $H = -\tr k$, $\theta^+ = 0$, so $H/\theta^+$ is indeterminate.

Near MOTS: $\theta^+ \approx \epsilon$ small, $H = -\tr k + \epsilon$, so $H/\theta^+ \approx -\tr k/\epsilon + 1$.

This blows up — the $\theta^+$-IMCF is singular at MOTS.

%% ============================================================================
\section{A Different Approach: Direct Inequality}
%% ============================================================================

\begin{keyresult}
Instead of using a flow, we prove $M_{\ADM} \ge \mtheta$ directly using the constraint equations and integration by parts.
\end{keyresult}

\subsection{The Constraint Equations}

On $(M, g, k)$ with DEC:
\begin{align}
    R_g - |k|^2 + (\tr k)^2 &= 2\mu \ge 2|J| \ge 0, \\
    \nabla^j(k_{ij} - (\tr k)g_{ij}) &= J_i.
\end{align}

\subsection{ADM Mass Integral}

The ADM mass can be written as:
\begin{equation}
    M_{\ADM} = \frac{1}{16\pi}\int_M (R_g + |k|^2 - (\tr k)^2) dV + \text{(boundary at } \Sigma\text{)}
\end{equation}

With DEC: $R_g - |k|^2 + (\tr k)^2 \ge 0$, so $R_g \ge |k|^2 - (\tr k)^2$.

\subsection{Boundary Term at $\Sigma$}

For a surface $\Sigma$ bounding a region $\Omega$:
\begin{equation}
    M_{\ADM} = \frac{1}{16\pi}\int_{M \setminus \Omega} (R_g + \ldots) dV + \frac{1}{8\pi}\int_\Sigma (H - \text{reference}) dA + \ldots
\end{equation}

The boundary term involves the mean curvature $H$, not $\theta^+$ directly.

\subsection{Modifying with $\theta^+$}

\begin{lemma}[Key Identity]
\begin{equation}
    H = \theta^+ - \tr_\Sigma k
\end{equation}

Therefore:
\begin{equation}
    \int_\Sigma H \, dA = \int_\Sigma \theta^+ dA - \int_\Sigma \tr_\Sigma k \, dA
\end{equation}
\end{lemma}

The term $\int \tr_\Sigma k \, dA$ can be related to the momentum constraint.

%% ============================================================================
\section{The Spacetime Positive Mass Theorem Route}
%% ============================================================================

\begin{theorem}[Spacetime Positive Mass \cite{schonyau1981}]
For asymptotically flat $(M, g, k)$ satisfying DEC:
\begin{equation}
    M_{\ADM} \ge 0
\end{equation}
with equality iff $(M, g, k)$ embeds in Minkowski space.
\end{theorem}

\begin{keyresult}
We can use the spacetime positive mass theorem with a \textbf{modified asymptotic region}.

\textbf{Idea:} Cut out the region inside $\Sigma$ and replace it with a "cap" that contributes $-\mtheta(\Sigma)$ to the mass.

If the resulting manifold still satisfies DEC and has mass $M_{\ADM} - \mtheta(\Sigma) \ge 0$, then:
\begin{equation}
    M_{\ADM} \ge \mtheta(\Sigma)
\end{equation}
\end{keyresult}

\subsection{The Construction}

\textbf{Step 1:} Start with $(M, g, k)$ and surface $\Sigma$.

\textbf{Step 2:} Remove $\Omega = $ region inside $\Sigma$.

\textbf{Step 3:} Attach a "cap" $(C, g_C, k_C)$ along $\Sigma$ such that:
\begin{itemize}
    \item $g_C|_\Sigma = g|_\Sigma$ (metric matches)
    \item $k_C|_\Sigma = k|_\Sigma$ (extrinsic curvature matches)
    \item $(C, g_C, k_C)$ satisfies DEC
    \item The total mass of the cap is $-\mtheta(\Sigma)$
\end{itemize}

\textbf{Step 4:} The glued manifold $(M \setminus \Omega) \cup_\Sigma C$ has:
\begin{itemize}
    \item Mass $= M_{\ADM} - \mtheta(\Sigma)$ (if cap contributes $-\mtheta$)
    \item Satisfies DEC (if gluing is smooth enough)
\end{itemize}

\textbf{Step 5:} By positive mass theorem: $M_{\ADM} - \mtheta(\Sigma) \ge 0$.

\subsection{The Cap Construction (Schwarzschild-like)}

For a round sphere $\Sigma$ with area $A$ and $\theta^+ = 0$ (MOTS), the natural cap is the interior Schwarzschild region.

For a general surface with $\theta^+ \ne 0$, we need a modified cap.

\begin{conjecture}[Cap Existence]
For any surface $\Sigma$ with induced metric $\gamma$, extrinsic curvature data $(H, \tr_\Sigma k)$, there exists a cap $(C, g_C, k_C)$ satisfying DEC with:
\begin{equation}
    M_{\text{cap}} = -\mtheta(\Sigma) = -\sqrt{\frac{A}{16\pi}}\left(1 - \frac{1}{16\pi}\int (\theta^+)^2 dA\right)
\end{equation}
\end{conjecture}

%% ============================================================================
\section{Explicit Cap for MOTS}
%% ============================================================================

For a MOTS $\Sigma^*$ with $\theta^+ = 0$:
\begin{equation}
    \mtheta(\Sigma^*) = \sqrt{\frac{A^*}{16\pi}}
\end{equation}

We need a cap with mass $-\sqrt{A^*/(16\pi)}$.

\subsection{The Schwarzschild Interior}

Take the Schwarzschild solution with mass $m = \sqrt{A^*/(16\pi)}$.

The horizon has area $16\pi m^2 = A^*$. ✓

The interior region (inside the horizon) can serve as the cap.

On the horizon:
\begin{itemize}
    \item $\theta^+ = 0$ (it's a MOTS) ✓
    \item The "mass" inside is $-m$ (the Schwarzschild mass is outside)
\end{itemize}

\textbf{Problem:} The Schwarzschild interior doesn't satisfy DEC in the usual sense (it contains the singularity).

\subsection{Regularized Cap}

Instead of the full Schwarzschild interior, use a regularized version:

\textbf{Bartnik's quasi-local mass approach:}
\begin{equation}
    m_B(\Sigma) = \inf\{M_{\ADM}(\tilde{g}, \tilde{k}) : (\tilde{g}, \tilde{k})|_\Sigma = (g, k)|_\Sigma, \text{ DEC holds}\}
\end{equation}

If $m_B(\Sigma) = \mtheta(\Sigma)$, we would have our result.

%% ============================================================================
\section{Connection to Bartnik Mass}
%% ============================================================================

\begin{definition}[Bartnik Mass]
The Bartnik mass of boundary data $(\Sigma, \gamma, H, \tr_\Sigma k)$ is:
\begin{equation}
    m_B(\Sigma) = \inf\{M_{\ADM}(N, g, k) : \partial N = \Sigma, \text{ data matches, DEC holds, no horizons}\}
\end{equation}
\end{definition}

\begin{conjecture}[Bartnik = $\mtheta$ for MOTS]
For a MOTS $\Sigma^*$:
\begin{equation}
    m_B(\Sigma^*) = \mtheta(\Sigma^*) = \sqrt{\frac{A^*}{16\pi}}
\end{equation}
\end{conjecture}

If this holds, then by definition of Bartnik mass:
\begin{equation}
    M_{\ADM}(M, g, k) \ge m_B(\Sigma^*) = \sqrt{\frac{A^*}{16\pi}}
\end{equation}

This would prove Penrose for MOTS!

%% ============================================================================
\section{The General Case: Trapped Surfaces}
%% ============================================================================

For a trapped surface $\Sigma$ with $\theta^+ < 0$:

\begin{conjecture}[Bartnik Bound for Trapped]
\begin{equation}
    m_B(\Sigma) \ge \mtheta(\Sigma)
\end{equation}
\end{conjecture}

\textbf{Why this might be true:}

The Bartnik mass measures the minimum mass needed to "fill in" a region with given boundary data.

For a trapped surface, the boundary data "wants" to collapse (negative expansion).

The minimum mass should be related to how trapped the surface is, measured by $\theta^+$.

\textbf{Consequence:}
\begin{equation}
    M_{\ADM} \ge m_B(\Sigma) \ge \mtheta(\Sigma) = \sqrt{\frac{A}{16\pi}}\left(1 - \frac{1}{16\pi}\int (\theta^+)^2 dA\right)
\end{equation}

%% ============================================================================
\section{Main Result (Conditional)}
%% ============================================================================

\begin{theorem}[Spacetime Penrose via $\mtheta$ — Conditional]\label{thm:main}
Assume:
\begin{enumerate}
    \item[(A)] For any surface $\Sigma$ in $(M, g, k)$ with DEC: $M_{\ADM} \ge m_B(\Sigma)$
    \item[(B)] For MOTS: $m_B(\Sigma^*) = \sqrt{A^*/(16\pi)}$
    \item[(C)] For trapped $\Sigma$: $m_B(\Sigma) \ge \mtheta(\Sigma)$
\end{enumerate}

Then for any trapped surface $\Sigma$:
\begin{equation}
    M_{\ADM} \ge \mtheta(\Sigma) = \sqrt{\frac{A}{16\pi}}\left(1 - \frac{1}{16\pi}\int (\theta^+)^2 dA\right)
\end{equation}

In particular, for surfaces with $\int (\theta^+)^2 dA \ll 16\pi$:
\begin{equation}
    M_{\ADM} \approx \sqrt{\frac{A}{16\pi}}
\end{equation}
\end{theorem}

%% ============================================================================
\section{Physical Interpretation}
%% ============================================================================

\begin{keyresult}
\textbf{What $\mtheta$ measures:}

The $\theta^+$-weighted Hawking mass can be written as:
\begin{equation}
    \mtheta(\Sigma) = \sqrt{\frac{A}{16\pi}} - \sqrt{\frac{A}{16\pi}} \cdot \frac{\langle(\theta^+)^2\rangle}{16\pi} \cdot A
\end{equation}
where $\langle(\theta^+)^2\rangle = \frac{1}{A}\int (\theta^+)^2 dA$ is the average squared null expansion.

\textbf{Physical meaning:}
\begin{itemize}
    \item First term: "bare" Penrose mass $\sqrt{A/(16\pi)}$
    \item Second term: "trapping energy" correction
\end{itemize}

A strongly trapped surface (large $|\theta^+|$) has more "trapping energy" and correspondingly smaller $\mtheta$.

The total mass $M_{\ADM}$ must exceed this $\mtheta$ because the trapping energy is part of the gravitational mass.
\end{keyresult}

%% ============================================================================
\section{Conclusion and Status}
%% ============================================================================

\begin{theorem-box}
\textbf{Summary:}

\begin{enumerate}
    \item We defined the $\theta^+$-weighted Hawking mass $\mtheta$
    \item $\mtheta$ equals $\sqrt{A/(16\pi)}$ at MOTS
    \item $\mtheta < \sqrt{A/(16\pi)}$ for trapped surfaces
    \item The conjecture $M_{\ADM} \ge \mtheta$ would give spacetime Penrose (with correction)
    \item Connection to Bartnik mass provides a path to proof
\end{enumerate}

\textbf{What remains:}
\begin{enumerate}
    \item Prove $m_B(\Sigma^*) = \sqrt{A^*/(16\pi)}$ for MOTS
    \item Prove $m_B(\Sigma) \ge \mtheta(\Sigma)$ for trapped surfaces
    \item Alternatively: find a direct monotonicity formula
\end{enumerate}
\end{theorem-box}

\begin{thebibliography}{99}
\bibitem{hawking1968} S.W. Hawking, Gravitational radiation in an expanding universe, \textit{J. Math. Phys.} 9 (1968), 598--604.
\bibitem{bartnik1989} R. Bartnik, New definition of quasilocal mass, \textit{Phys. Rev. Lett.} 62 (1989), 2346--2348.
\bibitem{schonyau1981} R. Schoen and S.-T. Yau, Proof of the positive mass theorem. II, \textit{Comm. Math. Phys.} 79 (1981), 231--260.
\end{thebibliography}

\end{document}
