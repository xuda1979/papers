%% ROUND3_SYNTHESIS.tex
%%
%% SYNTHESIS: Round 3 Blue/Red Team Analysis
%%
%% Summary of all attacks on Hull + Maximal Slice approach
%% and new attack vectors
%%
%% December 2025

\documentclass[11pt]{amsart}
\usepackage{amsmath,amssymb,amsthm}
\usepackage{xcolor}
\usepackage{tcolorbox}
\usepackage{booktabs}

\tcbuselibrary{theorems}

\newtcolorbox{summary}{
    colback=blue!5!white,
    colframe=blue!75!black,
    title={\textbf{SUMMARY}}
}

\newtcolorbox{fatal}{
    colback=red!10!white,
    colframe=red!75!black,
    title={\textbf{FATAL GAP}}
}

\newtcolorbox{promising}{
    colback=green!5!white,
    colframe=green!75!black,
    title={\textbf{PROMISING DIRECTION}}
}

\newtheorem{theorem}{Theorem}[section]

\newcommand{\ADM}{\mathrm{ADM}}
\newcommand{\Area}{\mathrm{Area}}
\newcommand{\tr}{\mathrm{tr}}
\newcommand{\Cap}{\mathrm{Cap}}

\title{Round 3 Synthesis:\\
\large Blue/Red Team Analysis of Penrose 1973}
\author{}
\date{December 2025}

\begin{document}
\maketitle

\begin{abstract}
We synthesize the results of Round 3 adversarial analysis, covering attacks on the Hull + Maximal Slice approach, the WCC + Hawking argument, perturbative methods, and capacity-based bounds. The analysis identifies promising new directions while confirming that the core area dominance problem remains unsolved.
\end{abstract}

\tableofcontents

%% ============================================================================
\section{Overview of Round 3}
%% ============================================================================

Round 3 tested the following approaches:

\begin{center}
\begin{tabular}{@{}lcc@{}}
\toprule
\textbf{Approach} & \textbf{Document} & \textbf{Status} \\
\midrule
Hull + Maximal Slice & BLUE\_RED\_TEAM\_ROUND3.tex & \textcolor{red}{FAILED} \\
WCC + Hawking Area & WCC\_HAWKING\_AREA\_ATTACK.tex & \textcolor{orange}{INCOMPLETE} \\
Perturbative & PERTURBATIVE\_AREA\_DOMINANCE.tex & \textcolor{orange}{PROMISING} \\
Capacity-based & DIRECT\_CONSTRAINT\_BOUND.tex & \textcolor{orange}{PROMISING} \\
\bottomrule
\end{tabular}
\end{center}

%% ============================================================================
\section{Attack 1: Hull + Maximal Slice (FAILED)}
%% ============================================================================

\subsection{The Approach}

PENROSE\_1973\_BREAKTHROUGH.tex proposed:
\begin{enumerate}
    \item Work on a maximal Cauchy surface (where $\tr k = 0$, hence $R_g \ge 0$)
    \item Construct the outer-area minimizing hull $\hat{\Sigma}$ of trapped surface $\Sigma$
    \item Apply RPI to the hull
\end{enumerate}

\subsection{Red Team Findings}

\begin{fatal}
\textbf{Not every trapped surface lies on a maximal slice.}

The set of trapped surfaces on maximal slices has measure zero in the space of all trapped surfaces.

Penrose's conjecture is about ALL trapped surfaces.
\end{fatal}

\subsection{Additional Issues}
\begin{itemize}
    \item Hull definition needs GMT formalization (fixable)
    \item IMCF starting condition needs case analysis (fixable)
    \item Jang + Hull has area going wrong direction (fatal)
\end{itemize}

\subsection{Verdict}
\textbf{The Hull + Maximal Slice approach cannot prove the full Penrose 1973 conjecture.}

%% ============================================================================
\section{Attack 2: WCC + Hawking Area Theorem (INCOMPLETE)}
%% ============================================================================

\subsection{The Approach}

Penrose's original argument:
\begin{enumerate}
    \item Trapped surface $\Sigma$ is inside the black hole (by NEC + WCC)
    \item Event horizon area is non-decreasing (Hawking)
    \item Final state is Kerr with $M = \sqrt{A_{\text{Kerr}}/(16\pi)}$
    \item Combine: $M \ge \sqrt{A(\Sigma)/(16\pi)}$
\end{enumerate}

\subsection{Red Team Findings}

The argument has a gap:
\begin{enumerate}
    \item Hawking theorem is about TIME evolution, not spatial comparison
    \item ``$\Sigma$ enclosed by $\mathcal{H}$'' does NOT imply ``$A(\Sigma) \le A(\mathcal{H})$''
    \item Counterexample: a surface can enclose another of larger area (tubes)
\end{enumerate}

\subsection{What WCC Does Give}
\begin{enumerate}
    \item Trapped $\Rightarrow$ inside black hole ✓
    \item Apparent horizon inside event horizon ✓
    \item Event horizon area non-decreasing in time ✓
\end{enumerate}

\subsection{What WCC Does NOT Give}
\begin{enumerate}
    \item $A(\Sigma) \le A(\Sigma^*)$ (trapped vs apparent) ✗
    \item $A(\Sigma^*) \le A(\mathcal{H})$ (apparent vs event) ✗
    \item Outer-minimizing property ✗
\end{enumerate}

\subsection{Verdict}
\textbf{The WCC + Hawking argument does not close the area dominance gap.}

The argument works for spherically symmetric collapse but not in general.

%% ============================================================================
\section{Attack 3: Perturbative Approach (PROMISING)}
%% ============================================================================

\subsection{The Approach}

Prove area dominance by perturbation from spherical symmetry:
\begin{enumerate}
    \item In spherical symmetry: $A(\Sigma) < A(\Sigma^*)$ trivially
    \item For small perturbations: inequality should persist by continuity
    \item Build up general case by density arguments
\end{enumerate}

\subsection{Results}

\textbf{First order:} For axisymmetric $\ell = 2$ perturbations:
\begin{itemize}
    \item MOTS area unchanged to $O(\varepsilon)$
    \item Trapped surface area unchanged to $O(\varepsilon)$
    \item Area dominance preserved ✓
\end{itemize}

\textbf{Higher order:} Analysis incomplete. The MOTS stability operator governs second-order effects.

\subsection{Key Observation}

\begin{promising}
No perturbative counterexample to area dominance has been found.

All explicit calculations support $A(\Sigma) \le A(\Sigma^*)$.

The inequality may be stable under all perturbations.
\end{promising}

\subsection{Open Problems}
\begin{enumerate}
    \item Complete second-order analysis
    \item Handle non-axisymmetric perturbations
    \item Prove uniform bound (independent of gap size)
\end{enumerate}

\subsection{Verdict}
\textbf{The perturbative approach is promising but incomplete.}

A full proof would require showing stability at all orders, which is technically challenging.

%% ============================================================================
\section{Attack 4: Capacity-Based Bound (PROMISING)}
%% ============================================================================

\subsection{The Approach}

Bypass area dominance using capacity:
\begin{enumerate}
    \item Define capacity: $\Cap(\Sigma) = \inf\{\int |\nabla u|^2 : u|_\Sigma = 1, u \to 0\}$
    \item Prove capacity-area isoperimetric: $A(\Sigma) \le \Cap^2/(4\pi)$
    \item Prove mass-capacity bound: $M_{\ADM} \ge \Cap/(8\pi)$
    \item Combine: $A(\Sigma) \le 16\pi M_{\ADM}^2$
\end{enumerate}

\subsection{Key Calculations}

In Schwarzschild with mass $M$:
\begin{itemize}
    \item $A(\text{horizon}) = 16\pi M^2$
    \item $\Cap(\text{horizon}) = 8\pi M$
    \item $A = \Cap^2/(4\pi)$ — exact equality!
\end{itemize}

\subsection{Conjectures Needed}

\begin{conjecture}[Capacity-Area Isoperimetric]
For any surface $\Sigma$ in $(M, g)$ with $R_g \ge 0$:
\begin{equation}
    A(\Sigma) \le \frac{\Cap(\Sigma)^2}{4\pi}
\end{equation}
\end{conjecture}

\begin{conjecture}[Mass-Capacity Inequality]
For outer-minimizing $\Sigma$ in asymptotically flat $(M, g)$ with $R_g \ge 0$:
\begin{equation}
    M_{\ADM} \ge \frac{\Cap(\Sigma)}{8\pi}
\end{equation}
\end{conjecture}

\subsection{Relation to Known Results}

\begin{itemize}
    \item The capacity-area isoperimetric is related to classical isoperimetric theory
    \item The mass-capacity inequality is related to the positive mass theorem
    \item Neither has been proven in full generality
\end{itemize}

\begin{promising}
The capacity approach provides a \textbf{clean framework} that:
\begin{enumerate}
    \item Avoids the MOTS/area dominance entirely
    \item Uses only global quantities (capacity, mass)
    \item Has exact equality in Schwarzschild
\end{enumerate}
\end{promising}

\subsection{Verdict}
\textbf{The capacity approach is promising but requires proving two new inequalities.}

%% ============================================================================
\section{Comparison of All Approaches}
%% ============================================================================

\begin{center}
\begin{tabular}{@{}p{3cm}p{4cm}p{4cm}p{2cm}@{}}
\toprule
\textbf{Approach} & \textbf{Key Idea} & \textbf{Obstacle} & \textbf{Status} \\
\midrule
Jang + RPI & Blow up at MOTS, apply RPI & Area dominance & Conditional \\
Hull + Maximal & Hull has $H \ge 0$, apply RPI & No maximal slice & Failed \\
WCC + Hawking & Dynamic area increase & Spatial comparison & Incomplete \\
Perturbative & Stability from spherical & Higher order analysis & Promising \\
Capacity & Direct $A \le C \cdot M^2$ & Two new inequalities & Promising \\
Spinor & Direct Witten bound & Boundary term sign & Open \\
\bottomrule
\end{tabular}
\end{center}

%% ============================================================================
\section{The Core Problem (Revisited)}
%% ============================================================================

After Round 3, we reconfirm the fundamental obstruction:

\begin{fatal}
\textbf{Area dominance: $A(\Sigma) \le A(\Sigma^*)$}

\textbf{Why it's hard:}
\begin{enumerate}
    \item Trapped surfaces have no variational characterization
    \item MOTS is not area-extremal ($H \ne 0$ generically)
    \item Null geometry defeats all flows
    \item No canonical geometric comparison
\end{enumerate}

\textbf{Why it might be true:}
\begin{enumerate}
    \item True in all known examples
    \item True in spherical symmetry
    \item Stable under perturbations (first order)
    \item No counterexample found after extensive search
\end{enumerate}
\end{fatal}

%% ============================================================================
\section{Remaining Viable Paths}
%% ============================================================================

\begin{enumerate}
    \item \textbf{Perturbative proof:} Show area dominance is stable at all orders, then use density.
    
    \item \textbf{Capacity theory:} Prove the two capacity conjectures. This would give a complete proof without area dominance.
    
    \item \textbf{Spinor methods:} Develop spinor boundary conditions for trapped surfaces (not MOTS).
    
    \item \textbf{Accept conditional:} The theorem holds under (OM). This is rigorous and publishable.
    
    \item \textbf{Find counterexample:} Construct initial data with $A(\Sigma) > A(\Sigma^*)$. This would disprove area dominance and show Penrose needs an additional hypothesis.
\end{enumerate}

%% ============================================================================
\section{What's Proven}
%% ============================================================================

\begin{theorem}[Conditional Spacetime Penrose Inequality]
Let $(M, g, k)$ be asymptotically flat initial data satisfying DEC. Assume:
\begin{itemize}
    \item There exists an outermost MOTS $\Sigma^*$
    \item \textbf{(OM)} $A(\Sigma_0) \le A(\Sigma^*)$ for all trapped surfaces $\Sigma_0$
\end{itemize}
Then for any trapped surface $\Sigma_0$:
\begin{equation}
    M_{\ADM}(g, k) \ge \sqrt{\frac{A(\Sigma_0)}{16\pi}}
\end{equation}
\end{theorem}

\begin{theorem}[Spherically Symmetric Penrose]
In spherically symmetric gravitational collapse satisfying NEC:
\begin{equation}
    M_{\ADM} \ge \sqrt{\frac{A(\Sigma)}{16\pi}}
\end{equation}
for any trapped surface $\Sigma$.
\end{theorem}

%% ============================================================================
\section{Documents Produced in Round 3}
%% ============================================================================

\begin{enumerate}
    \item \texttt{BLUE\_RED\_TEAM\_ROUND3.tex} — Attack on Hull + Maximal Slice
    \item \texttt{WCC\_HAWKING\_AREA\_ATTACK.tex} — WCC + Hawking analysis
    \item \texttt{PERTURBATIVE\_AREA\_DOMINANCE.tex} — Perturbative approach
    \item \texttt{DIRECT\_CONSTRAINT\_BOUND.tex} — Capacity-based approach
    \item \texttt{ROUND3\_SYNTHESIS.tex} — This synthesis document
\end{enumerate}

%% ============================================================================
\section{Conclusion}
%% ============================================================================

\begin{summary}
\textbf{Round 3 Status:}

After extensive adversarial analysis:

\begin{enumerate}
    \item The Hull + Maximal Slice approach \textbf{fails} (no maximal slice through general trapped surface)
    
    \item The WCC + Hawking approach is \textbf{incomplete} (spatial vs. temporal comparison)
    
    \item The perturbative approach is \textbf{promising} (stability at first order, no counterexample)
    
    \item The capacity approach is \textbf{promising} (clean framework, exact in Schwarzschild)
\end{enumerate}

\textbf{The 1973 Penrose Conjecture remains open.}

The conditional theorem (assuming OM) is proven. The full conjecture awaits either:
\begin{itemize}
    \item A proof of area dominance
    \item A proof via capacity theory
    \item A counterexample showing area dominance fails
\end{itemize}
\end{summary}

\end{document}
