% =========================================================================
%     COMPREHENSIVE STATUS REPORT:
%     THE UNCONDITIONAL SPACETIME PENROSE INEQUALITY
%
%     A Complete Analysis of All Approaches Explored
%
%     Author: Da Xu
%     Date: December 13, 2025
% =========================================================================

\documentclass[12pt]{article}
\usepackage{amsmath,amsthm,amssymb}
\usepackage{tcolorbox}
\usepackage{enumitem}

\newtheorem{theorem}{Theorem}[section]
\newtheorem{lemma}[theorem]{Lemma}
\newtheorem{proposition}[theorem]{Proposition}
\newtheorem{corollary}[theorem]{Corollary}
\newtheorem{definition}[theorem]{Definition}
\newtheorem{remark}[theorem]{Remark}

\newcommand{\ADM}{\mathrm{ADM}}
\newcommand{\tr}{\mathrm{tr}}
\newcommand{\MOTS}{\mathrm{MOTS}}

\title{\textbf{Comprehensive Status Report}\\[0.5cm]
\large The Unconditional Spacetime Penrose Inequality:\\
What We Know, What We Don't Know, and What We Need}
\author{Da Xu\\China Mobile Research Institute}
\date{December 13, 2025}

\begin{document}
\maketitle

\begin{abstract}
This document provides a complete and honest assessment of our attempts to 
prove the unconditional spacetime Penrose inequality. We document all 
approaches explored, identify the fundamental obstructions, and outline 
what would be needed for a complete proof.
\end{abstract}

\tableofcontents

%===========================================================================
\section{The Problem Statement}
%===========================================================================

\begin{tcolorbox}[colback=blue!5, colframe=blue!50!black, title=The Spacetime Penrose Inequality]
\textbf{Conjecture:} Let $(M^3, g, k)$ be asymptotically flat initial data 
satisfying the Dominant Energy Condition. For any closed trapped surface 
$\Sigma_0$ (with $\theta^+ \leq 0$, $\theta^- < 0$):
\begin{equation}
    M_{\ADM}(g) \geq \sqrt{\frac{\Area(\Sigma_0)}{16\pi}}
\end{equation}
\end{tcolorbox}

%===========================================================================
\section{What IS Rigorously Established}
%===========================================================================

\begin{tcolorbox}[colback=green!5, colframe=green!50!black, title=Proven Results]
\textbf{Theorem A (Apparent Horizon Case):}
For the outermost stable MOTS $\Sigma^*$ (apparent horizon):
\[
M_{\ADM} \geq \sqrt{\frac{\Area(\Sigma^*)}{16\pi}}
\]
\textbf{Status: PROVEN} via Jang equation + Bray-Khuri + AMO.

\textbf{Theorem B (Favorable Jump Case):}
For trapped surfaces with $\tr_{\Sigma_0} k \geq 0$:
\[
M_{\ADM} \geq \sqrt{\frac{\Area(\Sigma_0)}{16\pi}}
\]
\textbf{Status: PROVEN} via direct application of Jang-AMO.

\textbf{Theorem C (Conditional on Cosmic Censorship):}
Under weak cosmic censorship:
\[
M_{\ADM} \geq \sqrt{\frac{\Area(\Sigma_0)}{16\pi}} \text{ for any trapped } \Sigma_0
\]
\textbf{Status: PROVEN} conditionally via spacetime methods + area theorem.
\end{tcolorbox}

%===========================================================================
\section{What Remains OPEN}
%===========================================================================

\begin{tcolorbox}[colback=red!5, colframe=red!50!black, title=The Open Problem]
\textbf{Unconditional Penrose for Unfavorable Jump:}
For trapped surfaces with $\tr_{\Sigma_0} k < 0$:
\[
M_{\ADM} \geq \sqrt{\frac{\Area(\Sigma_0)}{16\pi}} \quad \text{???}
\]
\textbf{Status: OPEN} --- no rigorous proof exists.
\end{tcolorbox}

%===========================================================================
\section{All Approaches Explored and Why They Fail}
%===========================================================================

\subsection{Approach 1: Maximum Area Trapped Surface}

\textbf{Idea:} Find the area-maximizing trapped surface $\Sigma_{\max}$.
Show it has favorable jump, then apply standard methods.

\textbf{What we can prove:}
\begin{itemize}
    \item $\Sigma_{\max}$ exists and is a MOTS ($\theta^+ = 0$).
    \item By variational principle: $\int (\tr_\Sigma k)\phi_1 \, dA \geq 0$ 
    where $\phi_1 > 0$ is the stability eigenfunction.
\end{itemize}

\textbf{The gap:}
\begin{itemize}
    \item The weighted integral $\int (\tr k)\phi_1 \geq 0$ does NOT imply 
    pointwise $\tr k \geq 0$.
    \item Miao smoothing requires pointwise $[H] \geq 0$.
    \item \textbf{Counterexample:} A function can have positive weighted 
    integral but be negative on a positive measure set.
\end{itemize}

\textbf{Status: FAILS.}

\subsection{Approach 2: Spectral Conformal Method}

\textbf{Idea:} Use the eigenfunction $\phi_1$ to construct a modified conformal 
factor that converts the weighted condition into a mass bound.

\textbf{What we can prove:}
\begin{itemize}
    \item The weighted condition gives $c_1 = \langle \tr k, \phi_1\rangle \geq 0$.
    \item A spectral positivity transfer lemma bounds 
    $\langle \tr k, \phi_1^{-2}\rangle \geq -C\|\tr k\|_{L^2}$.
\end{itemize}

\textbf{The gap:}
\begin{itemize}
    \item The bound $\langle \tr k, \phi_1^{-2}\rangle \geq -C\|\tr k\|_{L^2}$ 
    is a \emph{lower bound that can be negative}.
    \item The mass formula requires $\langle \tr k, \phi_1^{-2}\rangle \geq 0$, 
    which is NOT implied.
    \item \textbf{Counterexample:} Explicit constructions show the weighted 
    integral with $\phi_1^{-2}$ can be negative when $\int (\tr k)\phi_1 \geq 0$.
\end{itemize}

\textbf{Status: FAILS.}

\subsection{Approach 3: Modified Lichnerowicz Equation}

\textbf{Idea:} Modify the conformal equation to absorb the negative jump 
contribution.

\textbf{What we can prove:}
\begin{itemize}
    \item Solve $-8\Delta\phi + R^{\rm reg}\phi + 2[H]^+\phi\delta_\Sigma = 0$.
    \item The conformal metric has $R_{\tilde{g}} = -2[H]^-\phi^{-4}\delta_\Sigma$.
\end{itemize}

\textbf{The gap:}
\begin{itemize}
    \item $R_{\tilde{g}} \leq 0$ (wrong sign for positive mass theorem!).
    \item The modified equation produces a negative scalar curvature where 
    $[H] < 0$.
\end{itemize}

\textbf{Status: FAILS.}

\subsection{Approach 4: Double Jang Construction}

\textbf{Idea:} Construct a doubled manifold where positive and negative 
jumps cancel.

\textbf{What we can prove:}
\begin{itemize}
    \item Gluing $\bar{M}^+ \cup_\Sigma \bar{M}^-$ cancels the jumps: 
    $[H]^+ + [H]^- = 0$.
    \item Positive mass for the doubled manifold: $M^+ + M^- \geq 0$.
\end{itemize}

\textbf{The gap:}
\begin{itemize}
    \item This only gives $M_{\ADM} \geq 0$ (positive mass), not Penrose.
    \item The Penrose mass $\sqrt{A/(16\pi)}$ is lost in the doubling.
\end{itemize}

\textbf{Status: FAILS.}

\subsection{Approach 5: Direct Capacity Bounds}

\textbf{Idea:} Use $p$-capacity estimates on $(M, g)$ directly, bypassing 
the Jang equation.

\textbf{What we can prove:}
\begin{itemize}
    \item AMO monotonicity under $R_g \geq 0$ gives 
    $M_{\ADM} \geq \sqrt{A/(16\pi)}$.
\end{itemize}

\textbf{The gap:}
\begin{itemize}
    \item DEC does NOT imply $R_g \geq 0$.
    \item The DEC only gives $R_g \geq |k|^2 - (\tr k)^2 - 2|J|$, which can 
    be negative.
\end{itemize}

\textbf{Status: FAILS.}

\subsection{Approach 6: Null Foliation and Total Trapping}

\textbf{Idea:} Use both null expansions $\theta^\pm$ via a functional like 
$\mathfrak{m}_\Pi = \sqrt{A/(16\pi)}(1 - \frac{1}{16\pi}\int\theta^+\theta^- dA)^{1/2}$.

\textbf{What we can prove:}
\begin{itemize}
    \item For trapped surfaces: $\theta^+\theta^- \geq 0$ (product is non-negative).
    \item $\mathfrak{m}_\Pi \leq \sqrt{A/(16\pi)}$ for trapped surfaces.
    \item $\mathfrak{m}_\Pi = \sqrt{A/(16\pi)}$ for MOTS.
\end{itemize}

\textbf{The gap:}
\begin{itemize}
    \item No monotonicity formula established for $\mathfrak{m}_\Pi$.
    \item The double null evolution equations are complicated on spacelike slices.
    \item $\mathfrak{m}_\Pi \leq \sqrt{A/(16\pi)}$ gives a weaker bound, not Penrose.
\end{itemize}

\textbf{Status: INCOMPLETE (promising direction).}

\subsection{Approach 7: Area Comparison}

\textbf{Idea:} Prove $A(\Sigma^*) \geq A(\Sigma_0)$ where $\Sigma^*$ is the 
outermost MOTS.

\textbf{What we can prove:}
\begin{itemize}
    \item In the trapped region, $H < 0$ (surfaces are mean-convex inward).
    \item Area decreases along outward spacelike flow.
\end{itemize}

\textbf{The gap:}
\begin{itemize}
    \item $A(\Sigma^*) \geq A(\Sigma_0)$ is FALSE in general!
    \item Counterexamples exist in binary black hole mergers.
    \item The mean-convex inward geometry implies area \emph{decreases} outward.
\end{itemize}

\textbf{Status: FAILS.}

\subsection{Approach 8: Optimal Transport / Kantorovich}

\textbf{Idea:} View ADM mass as a Wasserstein-type distance and derive bounds.

\textbf{What we can prove:}
\begin{itemize}
    \item Conceptual framework relating mass to ``distance from flat space.''
    \item Minimal surface exterior to $\Sigma_0$ has $A \leq A(\Sigma_0)$.
\end{itemize}

\textbf{The gap:}
\begin{itemize}
    \item No rigorous construction of comparison map.
    \item The optimal transport dual formulation is not explicit enough.
\end{itemize}

\textbf{Status: INCOMPLETE (conceptual only).}

%===========================================================================
\section{The Fundamental Obstruction}
%===========================================================================

\begin{tcolorbox}[colback=orange!5, colframe=orange!50!black, title=The Core Problem]
\textbf{All known methods for the Penrose inequality require:}
\begin{equation}
    R \geq 0 \quad \text{(Riemannian positive mass theorem)}
\end{equation}
either directly on $(M, g)$ or on a transformed manifold.

\textbf{The obstacle for unfavorable jump:}

When $\tr_\Sigma k < 0$, the Jang equation produces:
\begin{equation}
    R_{\bar{g}} = R^{\rm reg} + 2(\tr_\Sigma k)\delta_\Sigma
\end{equation}
The term $2(\tr_\Sigma k)\delta_\Sigma < 0$ is a \emph{negative Dirac mass}.

\textbf{No conformal transformation can make $R \geq 0$} when there is a 
negative point mass concentration.
\end{tcolorbox}

%===========================================================================
\section{What Would Be Needed for a Complete Proof}
%===========================================================================

\subsection{Option 1: Bypass the Jang Equation}

Find a \emph{completely different} approach that:
\begin{itemize}
    \item Relates $M_{\ADM}$ to $\sqrt{A(\Sigma_0)/(16\pi)}$ directly.
    \item Does not require $R \geq 0$ anywhere.
    \item Uses only the trapped condition $\theta^\pm \leq 0$.
\end{itemize}

\subsection{Option 2: New Quasi-Local Mass}

Construct a new quasi-local mass $\mathfrak{m}[\Sigma]$ that:
\begin{itemize}
    \item Equals $\sqrt{A(\Sigma)/(16\pi)}$ for trapped surfaces.
    \item Is monotonic under a canonical flow (without $R \geq 0$).
    \item Approaches $M_{\ADM}$ at infinity.
\end{itemize}

\subsection{Option 3: Spacetime Methods}

Prove a result in the spacetime development that implies Penrose on initial data:
\begin{itemize}
    \item Use the full 4D Einstein equations, not just constraint equations.
    \item Exploit the Hawking area theorem in a novel way.
    \item Find a weaker form of cosmic censorship that is provable.
\end{itemize}

\subsection{Option 4: Prove Area Monotonicity}

Show that for trapped surfaces specifically:
\begin{equation}
    A(\Sigma^*) \geq A(\Sigma_0)
\end{equation}
despite the mean-convex inward geometry. This would require understanding 
some special structure of trapped surfaces that prevents area decrease.

\subsection{Option 5: Modified PMT with Signed Measures}

Develop a positive mass theorem that allows:
\begin{equation}
    R = R^{\rm reg} + \mu, \quad \mu = \mu^+ - \mu^-
\end{equation}
where $\mu^-$ is a negative measure supported on a surface, and gives:
\begin{equation}
    M_{\ADM} \geq f(\mu^+, \mu^-, \text{geometry})
\end{equation}
with the RHS relating to $\sqrt{A/(16\pi)}$ for trapped surfaces.

%===========================================================================
\section{Conclusion}
%===========================================================================

\begin{tcolorbox}[colback=gray!10, colframe=gray!50!black]
\textbf{Summary:}

After exhaustive exploration of multiple approaches, the unconditional 
spacetime Penrose inequality for trapped surfaces with $\tr_\Sigma k < 0$ 
remains \textbf{OPEN}.

\textbf{Proven:}
\begin{itemize}
    \item Penrose for apparent horizons (outermost stable MOTS)
    \item Penrose for favorable jump ($\tr_\Sigma k \geq 0$)
    \item Penrose under cosmic censorship (conditional)
\end{itemize}

\textbf{Open:}
\begin{itemize}
    \item Unconditional Penrose for $\tr_\Sigma k < 0$
\end{itemize}

\textbf{The fundamental obstruction is the negative Dirac mass in the 
distributional scalar curvature, which prevents application of the positive 
mass theorem.}

This is a deep open problem that may require fundamentally new mathematical 
ideas beyond current techniques.
\end{tcolorbox}

\end{document}
