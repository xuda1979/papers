%% HONEST_SYNTHESIS_NEW_MATHEMATICS.tex
%% Honest Summary: The State of New Mathematical Approaches to Penrose 1973
%% 
%% This document honestly assesses all new approaches tried and identifies what's needed.

\documentclass[11pt]{amsart}
\usepackage{amsmath,amssymb,amsthm}
\usepackage{xcolor}
\usepackage{tcolorbox}

\theoremstyle{plain}
\newtheorem{theorem}{Theorem}
\newtheorem{proposition}[theorem]{Proposition}
\newtheorem{conjecture}[theorem]{Conjecture}

\theoremstyle{definition}
\newtheorem{definition}[theorem]{Definition}
\newtheorem{problem}[theorem]{Problem}

\theoremstyle{remark}
\newtheorem*{remark}{Remark}

\title{Honest Synthesis: New Mathematics for the Penrose 1973 Conjecture}
\author{}
\date{December 2025}

\begin{document}
\maketitle

\begin{abstract}
We provide an honest assessment of various novel mathematical approaches to the 1973 Penrose conjecture regarding trapped surface areas. Despite significant effort, all approaches encounter the same fundamental obstruction. We clearly identify what genuinely new mathematics would be needed for a complete proof.
\end{abstract}

%% ============================================================================
\section{The Problem}
%% ============================================================================

\begin{tcolorbox}[colback=blue!5!white,colframe=blue!75!black,title=The 1973 Penrose Conjecture (Area Monotonicity)]
\textbf{Statement (OM):} Let $\Sigma$ be a trapped surface in a globally hyperbolic spacetime $(M, g)$ satisfying DEC and cosmic censorship. Then:
\begin{equation}
    A(\Sigma) \le A(\mathcal{H}_\mathcal{C})
\end{equation}
where $\mathcal{H}_\mathcal{C}$ is the event horizon cross-section on a Cauchy surface containing $\Sigma$.
\end{tcolorbox}

\subsection{Why This Is Hard}

\begin{enumerate}
\item \textbf{Global vs. Local:} The event horizon $\mathcal{H}^+$ is defined globally (requires knowledge of $\mathscr{I}^+$), while trapped surfaces are defined locally.

\item \textbf{Sign Problem in Null Flows:}
\begin{itemize}
    \item \textit{Outgoing null:} $dA/d\lambda = \theta^+ A$. For trapped surfaces, $\theta^+ < 0$, so area \textit{decreases}. This gives $A(\Sigma) > A(\text{later})$, wrong direction.
    \item \textit{Ingoing null:} $dA/d\lambda = \theta^- A$. For trapped surfaces, $\theta^- < 0$, so area decreases. But ingoing null leaves the Cauchy surface.
\end{itemize}

\item \textbf{No Spacelike Flow:} There is no natural spacelike flow connecting $\Sigma$ to $\mathcal{H}_\mathcal{C}$ within a Cauchy surface.

\item \textbf{Jang Equation Limitations:} The Jang equation handles extrinsic curvature but has blow-up issues precisely at MOTS/trapped surfaces.
\end{enumerate}

%% ============================================================================
\section{Novel Approaches Attempted}
%% ============================================================================

\subsection{Approach 1: Lorentzian Causal Capacity}

\textbf{Idea:} Define a capacity-like functional using the causal structure.

\textbf{Definition:}
\begin{equation}
    \mathcal{C}_p(\Sigma) = \inf\left\{\int_{J^+(\Sigma)} |\nabla_g \phi|^p \, dV_g : \phi|_\Sigma = 1, \phi \to 0 \text{ at } \mathscr{I}^+\right\}
\end{equation}

\textbf{Hoped-for properties:}
\begin{enumerate}
    \item Monotonicity: $\Sigma_1 \subset J^-(\Sigma_2) \Rightarrow \mathcal{C}_p(\Sigma_1) \le \mathcal{C}_p(\Sigma_2)$
    \item Area-capacity: $A(\Sigma) \le C \cdot \mathcal{C}_p(\Sigma)^{\alpha}$ for suitable $\alpha$
    \item Horizon equality: $A(\mathcal{H}_\mathcal{C}) = C \cdot \mathcal{C}_p(\mathcal{H}_\mathcal{C})^{\alpha}$
\end{enumerate}

\textbf{Outcome:} \textcolor{red}{FAILS}
\begin{itemize}
    \item The gradient $|\nabla_g \phi|$ is computed in Lorentzian signature, making the integral non-positive-definite.
    \item The variational problem is ill-posed on Lorentzian manifolds.
    \item No clear relationship between Lorentzian capacity and area.
\end{itemize}

\subsection{Approach 2: $\theta$-Weighted Capacity}

\textbf{Idea:} Use a Riemannian capacity on the Cauchy surface, weighted by the null expansion $\theta^+$.

\textbf{Definition:}
\begin{equation}
    \widetilde{\text{Cap}}_\theta(\Sigma) = \inf_u \int_M \tilde{w}^2 |\nabla u|^2 \, dV_g, \quad \tilde{w} = \exp\left(-\int \theta^+/H\right)
\end{equation}

\textbf{Hoped-for properties:}
\begin{enumerate}
    \item In trapped regions: $\tilde{w} > 1$ (weight enhances capacity)
    \item Monotonicity: Inner surfaces have smaller capacity
    \item MOTS equality: $\widetilde{\text{Cap}}_\theta(\Sigma^*) = A(\Sigma^*)$
\end{enumerate}

\textbf{Outcome:} \textcolor{orange}{PARTIAL}
\begin{itemize}
    \item Monotonicity works (inner $\le$ outer)
    \item MOTS equality plausible but unproven
    \item \textbf{Critical gap:} Cannot prove $A(\Sigma) \le \widetilde{\text{Cap}}_\theta(\Sigma)$ for trapped surfaces
    \item The weight $\tilde{w} > 1$ increases capacity, but this doesn't obviously give an area bound
\end{itemize}

\subsection{Approach 3: Optimal Transport}

\textbf{Idea:} View area comparison as a transport inequality between trapped surface and horizon.

\textbf{Definition:} Causal Wasserstein distance with Lorentzian cost $c(x,y) = \tau(x,y)^2$.

\textbf{Hoped-for properties:}
\begin{enumerate}
    \item Jacobian bound from Raychaudhuri: $J_T \le 1$ for trapped surfaces
    \item Area comparison via change of variables
\end{enumerate}

\textbf{Outcome:} \textcolor{red}{FAILS}
\begin{itemize}
    \item Future-directed null flow: Jacobian bound goes wrong direction
    \item Past-directed flow: Leaves Cauchy surface
    \item Lorentzian optimal transport theory is immature
    \item Sign problems persist
\end{itemize}

\subsection{Approach 4: Causal Homology}

\textbf{Idea:} Define homology groups using only causal chains; area is a norm on homology.

\textbf{Definition:} $H_k^{\text{caus}}(M)$ with area norm $\|\alpha\|_A = \inf\{A(\Sigma) : [\Sigma] = \alpha\}$.

\textbf{Hoped-for properties:}
\begin{enumerate}
    \item Trapped surface and horizon in same homology class
    \item MOTS minimizes area in its class
    \item Penrose inequality follows
\end{enumerate}

\textbf{Outcome:} \textcolor{red}{WRONG DIRECTION}
\begin{itemize}
    \item Area norm gives \textit{lower} bounds: $A(\Sigma) \ge \|\alpha\|_A$
    \item If MOTS is the minimizer: $A(\Sigma) \ge A(\Sigma^*)$
    \item This is the \textbf{opposite} of the Penrose inequality!
\end{itemize}

\subsection{Approach 5: Focusing Functional}

\textbf{Idea:} Use the integrated expansion $\mathcal{F}(\Sigma) = \int_\Sigma \theta^+ dA$.

\textbf{Properties:}
\begin{enumerate}
    \item $\mathcal{F}(\Sigma) < 0$ for trapped surfaces
    \item $\mathcal{F}(\Sigma^*) = 0$ for MOTS
    \item $\mathcal{F}$ decreases along null flows (becomes more negative)
\end{enumerate}

\textbf{Outcome:} \textcolor{orange}{PROMISING BUT INCOMPLETE}
\begin{itemize}
    \item $\mathcal{F}$ captures "how trapped" a surface is
    \item No direct connection to area comparison established
    \item Differential inequality for $\mathcal{F}$ doesn't translate to area inequality
\end{itemize}

%% ============================================================================
\section{The Fundamental Obstruction}
%% ============================================================================

\begin{tcolorbox}[colback=red!5!white,colframe=red!75!black,title=Core Obstruction]
\textbf{All approaches fail for the same fundamental reason:}

The trapped condition $\theta^+ < 0$ implies area \textit{decrease} along outgoing null flow, while the Penrose inequality requires trapped surfaces to have \textit{smaller} area than the horizon.

These are compatible only if the trapped surface can flow \textit{inward} (area increasing) to reach the horizon. But:
\begin{enumerate}
    \item Inward null flow has $\theta^- < 0$ (also decreasing)
    \item There is no natural spacelike flow within the Cauchy surface
    \item The horizon is not "inside" the trapped surface on any Cauchy slice
\end{enumerate}
\end{tcolorbox}

%% ============================================================================
\section{What Would Constitute Genuine New Mathematics}
%% ============================================================================

Based on the above analysis, we identify what genuinely new mathematical structures might resolve the problem:

\subsection{Option A: A "Time-Reversed" Variational Principle}

\begin{problem}
Find a variational principle where trapped surfaces are \textit{area maximizers} (not minimizers) among surfaces with certain properties.
\end{problem}

\textbf{Why this might work:} If trapped surfaces maximize area among "outward-flowing" surfaces, and MOTS are the critical points, then $A(\Sigma) \le A(\Sigma^*)$ for trapped $\Sigma$.

\textbf{Challenge:} No such variational principle is currently known.

\subsection{Option B: A Dual Energy Condition}

\begin{problem}
Find a new energy condition or geometric inequality that "reverses" the Raychaudhuri focusing.
\end{problem}

\textbf{Idea:} Perhaps there exists a conformally related or dual metric where the roles of $\theta^+$ and area are exchanged.

\textbf{Challenge:} Conformal rescaling doesn't obviously help (it changes both $\theta$ and $A$ in linked ways).

\subsection{Option C: Non-Local Functional}

\begin{problem}
Define a functional on surfaces that is:
\begin{enumerate}
    \item Non-local (depends on the surface and its causal future)
    \item Monotonic under some flow
    \item Equals area for MOTS
    \item Bounds area from above for trapped surfaces
\end{enumerate}
\end{problem}

\textbf{Candidate:} The "causal volume" $\text{Vol}(J^+(\Sigma) \cap J^-(\mathcal{H}^+))$ -- the 4-volume of the region between the trapped surface and horizon.

\textbf{Challenge:} Relating 4-volume to 2-area requires new isoperimetric inequalities in Lorentzian geometry.

\subsection{Option D: Cosmic Censorship as a Variational Principle}

\begin{problem}
Reformulate cosmic censorship as a variational principle that implies (OM).
\end{problem}

\textbf{Idea:} Perhaps the statement "singularities are hidden behind horizons" can be expressed as a min-max principle that implies area bounds.

\textbf{Challenge:} Cosmic censorship is typically stated qualitatively, not variationally.

%% ============================================================================
\section{Most Promising Direction}
%% ============================================================================

\begin{tcolorbox}[colback=green!5!white,colframe=green!75!black,title=Recommended Approach]
\textbf{The Hull-Jang Method (for favorable cases):}

When the outer-area minimizing hull of a trapped surface $\Sigma$ lies entirely in the trapped region, the Jang equation approach succeeds:

\begin{enumerate}
    \item Hull has $\theta^+ < 0$ on its smooth parts
    \item This implies favorable sign in Jang equation
    \item Standard argument gives $A(\Sigma) \le A(\text{Hull}) \le A(\text{apparent horizon}) \le A(\mathcal{H}_\mathcal{C})$
\end{enumerate}

This handles a significant class of trapped surfaces, though not all.
\end{tcolorbox}

\subsection{What Remains for a Complete Proof}

To prove (OM) unconditionally, one needs to handle the case where:
\begin{itemize}
    \item The hull of $\Sigma$ exits the trapped region
    \item The extrinsic curvature on the hull has unfavorable sign ($\text{tr}_{\text{Hull}} k < 0$)
\end{itemize}

This case arises in binary black hole mergers where the "waist" connecting two horizons may be untrapped.

\textbf{No known approach handles this case.}

%% ============================================================================
\section{Conclusion}
%% ============================================================================

\textbf{Summary of findings:}
\begin{enumerate}
    \item Multiple genuinely new mathematical approaches were attempted: causal capacity, $\theta$-capacity, optimal transport, causal homology, focusing functional.
    
    \item All approaches encounter the same fundamental sign obstruction: the trapped condition implies area decrease, while the conjecture requires area comparison in the opposite direction.
    
    \item The Hull-Jang method provides a partial result for favorable cases.
    
    \item A complete proof requires either:
    \begin{itemize}
        \item A new variational principle where trapped surfaces are area maximizers
        \item A non-local functional with the right monotonicity
        \item A reformulation of cosmic censorship as a variational principle
        \item Genuinely new Lorentzian geometric inequalities
    \end{itemize}
\end{enumerate}

\textbf{Honest assessment:} The 1973 Penrose conjecture (OM) remains \textbf{open}. No current mathematical technique, including the novel approaches developed here, provides a complete proof. The conjecture may require fundamentally new ideas not yet discovered.

\end{document}
