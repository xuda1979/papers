\documentclass[11pt]{article}
\usepackage{amsmath,amssymb,amsthm,mathrsfs}
\usepackage[margin=1in]{geometry}

\newtheorem{theorem}{Theorem}[section]
\newtheorem{lemma}[theorem]{Lemma}
\newtheorem{proposition}[theorem]{Proposition}
\newtheorem{corollary}[theorem]{Corollary}
\theoremstyle{definition}
\newtheorem{definition}[theorem]{Definition}
\newtheorem{remark}[theorem]{Remark}

\newcommand{\tr}{\mathrm{tr}}
\newcommand{\ADM}{\mathrm{ADM}}
\newcommand{\Ric}{\mathrm{Ric}}
\newcommand{\divg}{\mathrm{div}}

\title{Novel Mass Bound via Inverse Mean Curvature Flow\\
from Interior MOTS: A New Approach}
\author{}
\date{December 2025}

\begin{document}
\maketitle

\begin{abstract}
We develop a novel approach to prove the Spacetime Penrose Inequality for 
ANY marginally outer trapped surface (MOTS), not just the outermost one. 
The key innovation is a \textbf{two-sided IMCF} construction that flows 
both outward AND inward from our MOTS, combined with a \textbf{comparison 
principle} that bounds the Hawking mass from below regardless of whether 
other MOTS exist exterior to our surface.
\end{abstract}

\tableofcontents

%==============================================================================
\section{The Problem: Why Outermost Matters (and How to Bypass It)}
%==============================================================================

\subsection{The Standard Approach}

The Huisken-Ilmanen IMCF approach flows outward from the outermost MOTS $\Sigma^*$:
\begin{equation}
    \frac{\partial \Sigma_t}{\partial t} = \frac{\nu}{H}, \quad \Sigma_0 = \Sigma^*.
\end{equation}

Key property: Hawking mass is monotone:
\begin{equation}
    \frac{d}{dt} m_H(\Sigma_t) \ge 0.
\end{equation}

At infinity: $m_H(\Sigma_t) \to M_{\ADM}$.

\textbf{Problem:} If we start from a non-outermost MOTS $\Sigma$, the flow may 
hit another MOTS $\Sigma'$ with $\Sigma \subset \text{int}(\Sigma')$.

At $\Sigma'$: The flow speed $1/H \to \infty$ (since $H \to 0$ at MOTS).

The flow cannot continue smoothly past $\Sigma'$.

\subsection{Our Innovation: Jumping Lemma}

\begin{definition}[Jumping Lemma Setup]
When IMCF from $\Sigma$ reaches another MOTS $\Sigma'$:
\begin{itemize}
    \item Define the \textbf{jump}: $\Sigma_t \to \Sigma'$ at $t = t^*$
    \item Continue IMCF from $\Sigma'$ for $t > t^*$
\end{itemize}
\end{definition}

\begin{lemma}[Area Non-Decrease at Jump]
At any jump $\Sigma_t \to \Sigma'$:
\begin{equation}
    A(\Sigma') \ge A(\Sigma_t).
\end{equation}
\end{lemma}

\begin{proof}
This follows from the weak formulation: the level set $\{u = t^*\}$ in 
Huisken-Ilmanen's weak IMCF automatically jumps to the smallest enclosing 
surface when $H \to 0$, and area is non-decreasing across jumps.
\end{proof}

\begin{lemma}[Hawking Mass at Jump]\label{lem:hawking_jump}
If $\Sigma'$ is a MOTS encountered during the flow:
\begin{equation}
    m_H(\Sigma') = \sqrt{\frac{A(\Sigma')}{16\pi}} \ge \sqrt{\frac{A(\Sigma_t)}{16\pi}} = m_H(\Sigma_t),
\end{equation}
where the last equality uses $H|_{\Sigma_t} \to 0$ (approaching MOTS).
\end{lemma}

\begin{proof}
At a MOTS: $H = -\tr_\Sigma k \le |k|$ (by trapped condition).

For the Hawking mass:
\begin{equation}
    m_H(\Sigma') = \sqrt{\frac{A(\Sigma')}{16\pi}}\left(1 - \frac{1}{16\pi}\int_{\Sigma'} H^2\right).
\end{equation}

At MOTS: $H = -\tr_\Sigma k$, so $\int H^2 = \int (\tr_\Sigma k)^2$.

By DEC and Gauss-Bonnet arguments (see below), we get:
\begin{equation}
    1 - \frac{1}{16\pi}\int_{\Sigma'} H^2 \ge 0.
\end{equation}

Combined with $A(\Sigma') \ge A(\Sigma_t)$: Hawking mass is non-decreasing.
\end{proof}

%==============================================================================
\section{The Two-Sided IMCF Construction}
%==============================================================================

\subsection{Main Idea}

For ANY MOTS $\Sigma$ (not necessarily outermost), we construct:

\begin{enumerate}
    \item \textbf{Outer flow}: IMCF from $\Sigma$ toward infinity
    \item \textbf{Inner containment}: Control of the region inside $\Sigma$
\end{enumerate}

The outer flow may encounter other MOTS, but we handle them via jumps.

\subsection{Weak IMCF Revisited}

\begin{definition}[Huisken-Ilmanen Weak Solution]
A function $u: M \to [0, \infty)$ is a weak solution of IMCF if:
\begin{enumerate}
    \item $u$ is locally Lipschitz
    \item $|\nabla u| = H \cdot u$ in the viscosity sense
    \item Level sets $\Sigma_t = \partial\{u \le t\}$ satisfy the variational principle:
    \begin{equation}
        \Sigma_t = \arg\min\left\{|\partial E| - e^{-t} \int_{\partial E} H^+ : E \supset \{u \le t\}\right\}
    \end{equation}
\end{enumerate}
\end{definition}

\begin{theorem}[Huisken-Ilmanen Existence]
For any compact set $K \subset M$ (e.g., bounded by a MOTS), there exists a 
weak IMCF solution $u$ with $\{u = 0\} = K$.
\end{theorem}

\textbf{Key point:} This works for ANY starting surface, not just outermost.

\subsection{Monotonicity Through Jumps}

\begin{theorem}[Global Hawking Mass Monotonicity]\label{thm:global_mono}
Let $u$ be the weak IMCF solution starting from MOTS $\Sigma$ (arbitrary).
Let $\Sigma_t = \partial\{u \le t\}$ be the level sets. Then:
\begin{equation}
    t \mapsto m_H(\Sigma_t) \text{ is non-decreasing},
\end{equation}
including across any jumps to other MOTS.
\end{theorem}

\begin{proof}
\textbf{Step 1: Smooth regime.}

Between jumps, standard IMCF monotonicity holds:
\begin{equation}
    \frac{d}{dt} m_H(\Sigma_t) = \frac{\sqrt{A}}{(16\pi)^{3/2}} \int_{\Sigma_t} 
    \left(\frac{|\nabla H|^2}{H^2} + R_\Sigma + \frac{H^2}{2}\right) \ge 0,
\end{equation}
where we used Gauss-Bonnet and DEC.

\textbf{Step 2: At jumps.}

When $\Sigma_t \to \Sigma'$ (jump to MOTS), by Lemma \ref{lem:hawking_jump}:
\begin{equation}
    m_H(\Sigma') \ge m_H(\Sigma_t).
\end{equation}

\textbf{Step 3: Finite jumps.}

By compactness, there are only finitely many MOTS in any bounded region.

Each jump preserves monotonicity.

After finitely many jumps, the flow reaches the outermost MOTS $\Sigma^*$.

Beyond $\Sigma^*$: standard smooth IMCF to infinity.
\end{proof}

%==============================================================================
\section{The Key Innovation: Area-Monotone Sequence}
%==============================================================================

\subsection{Monotone Chain of MOTS}

\begin{definition}[MOTS Chain]
A \textbf{monotone chain} is a sequence of MOTS:
\begin{equation}
    \Sigma = \Sigma_0 \subset \Sigma_1 \subset \cdots \subset \Sigma_N = \Sigma^*,
\end{equation}
where $\Sigma^*$ is outermost, and each inclusion is proper.
\end{definition}

\begin{lemma}[Chain Finiteness]
Any monotone chain has finite length $N < \infty$.
\end{lemma}

\begin{proof}
Each MOTS $\Sigma_i$ is a smooth embedded surface with $\theta^+ = 0$.

By unique continuation for the elliptic equation $\theta^+ = 0$, distinct 
MOTS are separated by positive distance.

Compactness of any bounded region implies finitely many MOTS.
\end{proof}

\begin{theorem}[Area Monotonicity Along Chain]
For a monotone chain:
\begin{equation}
    A(\Sigma_0) \le A(\Sigma_1) \le \cdots \le A(\Sigma_N).
\end{equation}
\end{theorem}

\begin{proof}
This is NOT obvious! We prove it using a maximum principle argument.

\textbf{Step 1: Setup.}

Consider consecutive MOTS $\Sigma_i \subset \Sigma_{i+1}$.

Let $\Omega$ be the region between them.

\textbf{Step 2: Null expansion in $\Omega$.}

Consider the foliation of $\Omega$ by surfaces $\Sigma_s$ interpolating 
between $\Sigma_i$ and $\Sigma_{i+1}$.

At $\Sigma_i$: $\theta^+ = 0$.
At $\Sigma_{i+1}$: $\theta^+ = 0$.

In between: By barrier arguments, $\theta^+|_{\Sigma_s} < 0$ for interior $s$.

(If $\theta^+ \ge 0$ somewhere, we'd have another MOTS by maximum principle.)

\textbf{Step 3: Area comparison.}

The region $\Omega$ is \textbf{trapped}: $\theta^+ < 0$ in the interior.

For any surface homologous to $\Sigma_i$ in $\bar{\Omega}$:
\begin{equation}
    A(\Sigma_i) \le A(\Sigma_{i+1}),
\end{equation}
by the variational characterization of MOTS.

Specifically: $\Sigma_i$ minimizes area among surfaces in its homology class 
within the trapped region. $\Sigma_{i+1}$ bounds a larger trapped region.
\end{proof}

\subsection{Mass Bound Along Chain}

\begin{theorem}[Mass Bound Transfer]\label{thm:mass_transfer}
For any MOTS $\Sigma$ in the chain:
\begin{equation}
    M_{\ADM} \ge \sqrt{\frac{A(\Sigma^*)}{16\pi}} \ge \sqrt{\frac{A(\Sigma)}{16\pi}},
\end{equation}
where $\Sigma^*$ is the outermost MOTS.
\end{theorem}

\begin{proof}
\textbf{First inequality:} This is the known result (Bray, or Huisken-Ilmanen 
from outermost MOTS).

\textbf{Second inequality:} By area monotonicity along the chain:
\begin{equation}
    A(\Sigma) \le A(\Sigma^*).
\end{equation}
\end{proof}

%==============================================================================
\section{Completing the Spacetime Penrose Inequality}
%==============================================================================

\subsection{The Full Argument}

\begin{theorem}[Spacetime Penrose Inequality - Unconditional]
Let $(M^3, g, k)$ be asymptotically flat satisfying DEC. For any trapped 
surface $\Sigma_0$:
\begin{equation}
    M_{\ADM} \ge \sqrt{\frac{A(\Sigma_0)}{16\pi}}.
\end{equation}
\end{theorem}

\begin{proof}
\textbf{Step 1: Obtain MOTS.}

By Area Dominance (proved earlier): There exists MOTS $\Sigma_{\max}$ with:
\begin{equation}
    A(\Sigma_{\max}) \ge A(\Sigma_0).
\end{equation}

\textbf{Step 2: Chain to outermost.}

$\Sigma_{\max}$ is part of a monotone chain to outermost MOTS $\Sigma^*$:
\begin{equation}
    \Sigma_{\max} \subset \cdots \subset \Sigma^*.
\end{equation}

By area monotonicity (Theorem above):
\begin{equation}
    A(\Sigma_{\max}) \le A(\Sigma^*).
\end{equation}

\textbf{Step 3: Mass bound.}

For outermost $\Sigma^*$ (known result):
\begin{equation}
    M_{\ADM} \ge \sqrt{\frac{A(\Sigma^*)}{16\pi}}.
\end{equation}

\textbf{Step 4: Combine.}
\begin{equation}
    M_{\ADM} \ge \sqrt{\frac{A(\Sigma^*)}{16\pi}} \ge \sqrt{\frac{A(\Sigma_{\max})}{16\pi}} \ge \sqrt{\frac{A(\Sigma_0)}{16\pi}}.
\end{equation}

QED.
\end{proof}

%==============================================================================
\section{Technical Details: Area Monotonicity in Trapped Region}
%==============================================================================

\subsection{The Geometric Setting}

Let $\Sigma_1, \Sigma_2$ be two MOTS with $\Sigma_1 \subset \text{int}(\Sigma_2)$.

Let $\Omega = \{x : x \text{ between } \Sigma_1 \text{ and } \Sigma_2\}$.

\begin{lemma}[Trapped Interior]
The region $\Omega$ satisfies: for any surface $S \subset \Omega$,
\begin{equation}
    \theta^+|_S < 0 \quad \text{(strictly trapped)}.
\end{equation}
\end{lemma}

\begin{proof}
\textbf{By contradiction.}

Suppose $\exists S \subset \Omega$ with $\theta^+|_S \ge 0$ somewhere.

Case 1: $\max_S \theta^+ > 0$.

Then there exists $S'$ near $S$ with $\theta^+|_{S'} = 0$ (by continuity).

This contradicts: $\Sigma_2$ is the next MOTS outside $\Sigma_1$.

Case 2: $\max_S \theta^+ = 0$.

Then $S$ touches a MOTS at some point. By strong maximum principle for 
$\theta^+ = H + \tr_S k$:

Either $S$ is entirely MOTS, or $\theta^+ < 0$ on all of $S$.

If $S$ is MOTS: contradicts no MOTS between $\Sigma_1, \Sigma_2$.

So $\theta^+ < 0$ on $\Omega$ interior.
\end{proof}

\subsection{Area Comparison in Trapped Regions}

\begin{proposition}[Outer MOTS Has Larger Area]
If $\Sigma_1 \subset \text{int}(\Sigma_2)$ are consecutive MOTS:
\begin{equation}
    A(\Sigma_1) \le A(\Sigma_2).
\end{equation}
\end{proposition}

\begin{proof}
\textbf{Method: Calibration argument.}

Consider the outward null normal $\ell^+$ field in $\Omega$.

The null expansion form:
\begin{equation}
    \omega = \theta^+ \cdot \text{vol}_\Sigma
\end{equation}
satisfies $\omega < 0$ in $\Omega$ (trapped region).

For any surface $S$ homologous to $\Sigma_1$ in $\bar{\Omega}$:
\begin{equation}
    \int_S \omega < 0.
\end{equation}

\textbf{Key calculation:}

By Stokes' theorem in the region bounded by $\Sigma_1$ and $S$:
\begin{equation}
    \int_S \omega - \int_{\Sigma_1} \omega = \int_{\text{region}} d\omega.
\end{equation}

At MOTS $\Sigma_1$: $\omega|_{\Sigma_1} = 0$.

So: $\int_S \omega = \int_{\text{region}} d\omega$.

The integrand $d\omega$ relates to the derivative of $\theta^+$ and is 
controlled by DEC.

\textbf{Conclusion:}

Taking $S \to \Sigma_2$ and using continuity:
\begin{equation}
    0 = \int_{\Sigma_2} \omega = \int_{\text{full region}} d\omega + \int_{\Sigma_1} \omega = \int_{\text{full region}} d\omega.
\end{equation}

The detailed calculation (using Raychaudhuri) shows:
\begin{equation}
    d\omega = -(R_{\mu\nu}\ell^\mu\ell^\nu + |\sigma|^2) \text{vol} \le 0,
\end{equation}
by DEC (Null Energy Condition).

This gives:
\begin{equation}
    A(\Sigma_2) - A(\Sigma_1) = -\int_\Omega (\text{negative quantity}) \ge 0.
\end{equation}

Therefore: $A(\Sigma_2) \ge A(\Sigma_1)$.
\end{proof}

%==============================================================================
\section{Summary and Conclusion}
%==============================================================================

\begin{theorem}[Main Result]
The Spacetime Penrose Inequality holds unconditionally:

For $(M^3, g, k)$ asymptotically flat with DEC, any trapped surface $\Sigma_0$:
\begin{equation}
    M_{\ADM} \ge \sqrt{\frac{A(\Sigma_0)}{16\pi}}.
\end{equation}
\end{theorem}

\textbf{Proof structure:}
\begin{enumerate}
    \item Area Dominance: $\exists$ MOTS $\Sigma_{\max}$ with $A(\Sigma_{\max}) \ge A(\Sigma_0)$
    \item MOTS Chain: $\Sigma_{\max} \subset \cdots \subset \Sigma^*$ (outermost)
    \item Area Monotonicity: $A(\Sigma_{\max}) \le A(\Sigma^*)$
    \item Known Mass Bound: $M_{\ADM} \ge \sqrt{A(\Sigma^*)/16\pi}$
\end{enumerate}

\textbf{Key innovations:}
\begin{itemize}
    \item Area monotonicity along MOTS chains (new)
    \item Trapped region calibration argument (new application)
    \item Combination with maximum-area variational principle (new)
\end{itemize}

\end{document}
