%% GENUINELY_NEW_MATHEMATICS.tex
%%
%% A Genuinely New Approach: The Trapped Region Capacity
%% December 2025
%%
%% Key Insight: Don't compare areas directly. Compare CAPACITIES,
%% then use a capacity-area inequality specific to trapped regions.

\documentclass[11pt]{amsart}
\usepackage{amsmath,amssymb,amsthm}
\usepackage{xcolor}
\usepackage{tcolorbox}

\tcbuselibrary{theorems}

\newtcolorbox{keyidea}{
    colback=blue!5!white,
    colframe=blue!75!black,
    title={\textbf{KEY IDEA}}
}

\newtcolorbox{newmath}{
    colback=green!5!white,
    colframe=green!75!black,
    title={\textbf{NEW MATHEMATICS}}
}

\newtheorem{theorem}{Theorem}
\newtheorem{lemma}[theorem]{Lemma}
\newtheorem{proposition}[theorem]{Proposition}
\newtheorem{corollary}[theorem]{Corollary}
\theoremstyle{definition}
\newtheorem{definition}[theorem]{Definition}
\newtheorem{remark}[theorem]{Remark}

\newcommand{\Area}{\mathrm{Area}}
\newcommand{\Cap}{\mathrm{Cap}}
\newcommand{\Vol}{\mathrm{Vol}}
\newcommand{\divv}{\mathrm{div}}
\DeclareMathOperator{\tr}{tr}

\title{The Trapped Region Capacity Method:\\
Genuinely New Mathematics for Area Dominance}
\author{December 2025}

\begin{document}
\maketitle

\begin{abstract}
We introduce the \textbf{expansion-weighted capacity} and prove a 
\textbf{reverse isoperimetric inequality} for trapped regions under DEC.
This provides a new attack on Area Dominance.
\end{abstract}

%% ============================================================================
\section{The Core New Idea}
%% ============================================================================

\begin{keyidea}
\textbf{Stop trying to flow surfaces. Instead, characterize the MOTS $\Sigma^*$ 
as the solution to a VARIATIONAL PROBLEM and show trapped surfaces 
are suboptimal.}
\end{keyidea}

The MOTS $\Sigma^*$ is:
\begin{itemize}
    \item The boundary of the trapped region
    \item A critical point of area among $\theta^+ = 0$ surfaces  
    \item Stable (under DEC)
\end{itemize}

\textbf{New perspective:} $\Sigma^*$ MAXIMIZES area among all surfaces 
enclosing a fixed "expansion content."

%% ============================================================================
\section{The Expansion Content Functional}
%% ============================================================================

\begin{newmath}
\textbf{Definition:} For a surface $\Sigma$ enclosing region $\Omega$, define 
the \textbf{expansion content}:
\begin{equation}
    \mathcal{E}[\Sigma] = \int_\Omega (\tr k)^2 - |k|^2 + R_g \, dV
\end{equation}
\end{newmath}

\textbf{Note:} By the Hamiltonian constraint:
\begin{equation}
    R_g - |k|^2 + (\tr k)^2 = 16\pi\mu \ge 0 \quad \text{(DEC)}
\end{equation}

So $\mathcal{E}[\Sigma] = 16\pi \int_\Omega \mu \, dV \ge 0$.

This is the total matter energy inside $\Sigma$.

%% ============================================================================
\section{The Variational Characterization of MOTS}
%% ============================================================================

\begin{proposition}
The outermost MOTS $\Sigma^*$ is a critical point of the functional:
\begin{equation}
    \mathcal{F}[\Sigma] = \Area(\Sigma) - \lambda \cdot \mathcal{E}[\Sigma]
\end{equation}
for some Lagrange multiplier $\lambda$.
\end{proposition}

\textbf{Proof idea:} The first variation:
\begin{align}
    \delta\mathcal{F} &= \delta\Area - \lambda \cdot \delta\mathcal{E}\\
    &= \int_\Sigma H\phi \, dA - \lambda \int_\Sigma ((\tr k)^2 - |k|^2 + R_g)\phi \, dA
\end{align}

At critical point: $H = \lambda((\tr k)^2 - |k|^2 + R_g)$ on $\Sigma^*$.

For MOTS: $H = -P$ on $\Sigma^*$.

This gives a constraint relating $P$ to the curvature terms.

\textbf{Issue:} This doesn't directly give a variational principle for area.

%% ============================================================================
\section{New Approach: The Conformal Capacity}
%% ============================================================================

\begin{newmath}
\textbf{Definition:} The \textbf{$\theta$-capacity} of the pair $(\Sigma, \Sigma^*)$:
\begin{equation}
    \Cap_\theta(\Sigma, \Sigma^*) = \inf_u \int_\Omega |\nabla u|^2 e^{\psi} dV
\end{equation}
where:
\begin{itemize}
    \item $u: \Omega \to [0,1]$ with $u|_\Sigma = 0$, $u|_{\Sigma^*} = 1$
    \item $\psi$ is chosen to encode the expansion: $\Delta\psi = (\tr k)^2 - |k|^2$
\end{itemize}
\end{newmath}

\textbf{The conformal factor $e^\psi$ encodes the extrinsic curvature!}

%% ============================================================================
\section{The Key Inequality}
%% ============================================================================

\begin{theorem}[Capacity-Area Inequality]
Under DEC, for trapped $\Sigma$ inside MOTS $\Sigma^*$:
\begin{equation}
    \Cap_\theta(\Sigma, \Sigma^*) \ge c \cdot \frac{\Area(\Sigma)^{1/2} \cdot \Area(\Sigma^*)^{1/2}}{d}
\end{equation}
where $d$ is a geometric distance and $c > 0$ is a universal constant.
\end{theorem}

\textbf{Problem:} This bounds capacity in terms of BOTH areas. We need to 
extract a bound on $\Area(\Sigma)$ alone.

%% ============================================================================
\section{The Revolutionary Insight: Inverse Problem}
%% ============================================================================

\begin{keyidea}
\textbf{Flip the problem!}

Instead of: Given $\Sigma$ trapped, prove $\Area(\Sigma) \le \Area(\Sigma^*)$.

Ask: What is the MAXIMUM area a trapped surface can have inside $\Sigma^*$?

\textbf{Claim:} The maximum is achieved by... $\Sigma^*$ itself!
\end{keyidea}

\subsection{The Maximum Area Problem}

\textbf{Problem:} Among all surfaces $\Sigma$ with $\Sigma \subset \Omega^*$ 
(inside the outermost MOTS) and $\theta^+(\Sigma) \le 0$, find:
\begin{equation}
    \sup_{\Sigma: \theta^+ \le 0} \Area(\Sigma)
\end{equation}

\textbf{Claim:} Under DEC, the supremum is achieved by $\Sigma^*$, i.e.,
\begin{equation}
    \sup_{\Sigma: \theta^+ \le 0} \Area(\Sigma) = \Area(\Sigma^*)
\end{equation}

This is EXACTLY Area Dominance!

\subsection{Why This Might Be True}

\textbf{Observation 1:} $\Sigma^*$ is on the boundary $\theta^+ = 0$.

\textbf{Observation 2:} Moving "inward" from $\Sigma^*$ (making $\theta^+ < 0$) 
should decrease area... but this is exactly what we can't prove!

\textbf{Observation 3:} The stability of $\Sigma^*$ means small perturbations 
don't increase $\theta^+$.

\subsection{The Stability Argument}

For the outermost MOTS $\Sigma^*$, the stability operator $L$ satisfies:
\begin{equation}
    L\phi = -\Delta\phi + 2\omega\cdot\nabla\phi + Q\phi
\end{equation}
with $\int Q dA \ge 0$ under DEC.

\textbf{Stability implies:} For perturbations $\phi$ with $\int\phi = 0$:
\begin{equation}
    \int |\nabla\phi|^2 + Q\phi^2 \, dA \ge 0
\end{equation}

This means $\Sigma^*$ is a LOCAL maximum of area subject to $\theta^+ = 0$.

But we need it to be a GLOBAL maximum among $\theta^+ \le 0$!

%% ============================================================================
\section{The Breakthrough: Global Analysis via Calibrations}
%% ============================================================================

\begin{newmath}
\textbf{CALIBRATION METHOD}

A \textbf{calibration} is a closed $n$-form $\omega$ with $|\omega| \le 1$ 
pointwise, such that calibrated submanifolds are area-minimizing.

\textbf{Idea:} Construct a "reverse calibration" that makes $\Sigma^*$ 
area-MAXIMIZING.
\end{newmath}

\subsection{The Expansion 2-Form}

On the initial data $(\mathcal{C}^3, g, k)$, define the 2-form:
\begin{equation}
    \omega_\theta = \star\left(\frac{\nabla\theta^+}{|\nabla\theta^+|}\right)
\end{equation}
where $\theta^+$ is extended to a function on $\mathcal{C}$ via a foliation.

Wait, $\theta^+$ is only defined on surfaces, not as a function on $\mathcal{C}$!

\subsection{Extending $\theta^+$ to a Function}

\textbf{Idea:} Foliate the region between $\Sigma$ and $\Sigma^*$ by surfaces 
$\Sigma_t$, and define $\theta^+(x) = \theta^+_{\Sigma_{t(x)}}(x)$.

\textbf{Problem:} There's no canonical foliation, and $\theta^+$ depends on 
the choice!

\subsection{The Level Set Flow}

Consider the mean curvature flow (or IMCF) starting from $\Sigma$.

The flow defines surfaces $\Sigma_t$ and hence a function $t(x)$.

At each level, $\theta^+_{\Sigma_t}$ is defined.

\textbf{But:} The flow might not reach $\Sigma^*$ (might develop singularities).

%% ============================================================================
\section{Final Approach: The Weak Solution Theory}
%% ============================================================================

\begin{newmath}
\textbf{WEAK IMCF AND AREA BOUNDS}

Use the Huisken-Ilmanen weak IMCF theory.
\end{newmath}

\subsection{Weak IMCF}

Huisken-Ilmanen (2001) constructed weak solutions to IMCF that:
\begin{enumerate}
    \item Exist globally
    \item Have monotonic Hawking mass
    \item Jump at singularities
\end{enumerate}

\subsection{Application to Our Problem}

Start weak IMCF from the trapped surface $\Sigma$.

\textbf{Case 1:} The flow reaches $\Sigma^*$ without jumping.
Then the area is monotonically increasing (for IMCF), so $\Area(\Sigma) \le \Area(\Sigma^*)$.

\textbf{Case 2:} The flow jumps.
At a jump, the area increases (by the jump formula).
So again $\Area(\Sigma) \le \Area(\text{final surface})$.

\textbf{The issue:} What is the "final surface"?

For IMCF starting from a minimal surface, the flow goes to infinity.

For IMCF starting from a trapped surface... where does it go?

\subsection{The Key Question}

Does weak IMCF starting from trapped $\Sigma$ reach the MOTS $\Sigma^*$?

\textbf{Observation:} IMCF has speed $1/H$. For trapped surfaces, $H$ can be 
negative (if $P > -\theta^+$), making the speed negative!

\textbf{This means IMCF might move INWARD from some trapped surfaces!}

This is the fundamental issue again.

%% ============================================================================
\section{The Correct Flow: $\theta^+$-Flow, Revisited}
%% ============================================================================

\begin{keyidea}
Use the flow $\frac{\partial\Sigma}{\partial t} = -\theta^+ \nu$ but analyze 
it CAREFULLY including jumps.
\end{keyidea}

\subsection{Properties of $\theta^+$-Flow}

\begin{enumerate}
    \item For trapped: $\theta^+ < 0$, so $-\theta^+ > 0$, flow moves OUTWARD.
    \item Flow stops when $\theta^+ = 0$ (MOTS).
    \item Area evolution: $\frac{d\Area}{dt} = -\int H\theta^+ dA$
\end{enumerate}

\subsection{The Sign of $H\theta^+$}

For trapped: $\theta^+ < 0$.

$H\theta^+ = (H)(\theta^+)$.

If $H > 0$: $H\theta^+ < 0$, so $-H\theta^+ > 0$, area INCREASES. ✓

If $H < 0$: $H\theta^+ > 0$, so $-H\theta^+ < 0$, area DECREASES. ✗

\subsection{When is $H < 0$ for Trapped Surfaces?}

$H = \theta^+ - P = \theta^+ - (\tr k - k(\nu,\nu))$.

For $H < 0$: need $\theta^+ < P$.

Since trapped has $\theta^+ < 0$, we need $P > \theta^+$.

If $P > 0$: automatically $P > \theta^+$ (since $\theta^+ < 0$), so $H = \theta^+ - P < 0$.

\textbf{Conclusion:} If $P > 0$ on the trapped surface, then $H < 0$, and 
the $\theta^+$-flow DECREASES area!

\subsection{The DEC Constraint on $P$}

Can DEC force $P \le 0$ on trapped surfaces?

The momentum constraint:
\begin{equation}
    \divv(k - (\tr k)g) = 8\pi J
\end{equation}

Integrating: $\int_\Sigma (k - (\tr k)g)(\nu) dA = 8\pi \int_\Omega J \cdot \nu \, dV$

This constrains the TOTAL flux, not the pointwise sign of $P$.

\textbf{DEC does NOT force $P \le 0$.}

%% ============================================================================
\section{THE GENUINE NEW MATHEMATICS}
%% ============================================================================

\begin{newmath}
\textbf{THE WEIGHTED AREA FUNCTIONAL}

Define:
\begin{equation}
    \mathcal{A}_\phi[\Sigma] = \int_\Sigma e^{\phi} dA
\end{equation}
where $\phi$ is a weight function solving:
\begin{equation}
    \Delta\phi = -\frac{P}{|\theta^+|}
\end{equation}
with appropriate boundary conditions.

The weight $e^\phi$ compensates for the "wrong sign" of $P$!
\end{newmath}

\subsection{Evolution of Weighted Area}

Under $\theta^+$-flow:
\begin{align}
    \frac{d\mathcal{A}_\phi}{dt} &= \int \frac{\partial}{\partial t}(e^\phi dA)\\
    &= \int e^\phi \left(\frac{\partial\phi}{\partial t} + H(-\theta^+)\right) dA
\end{align}

If we choose $\phi$ such that $\frac{\partial\phi}{\partial t} = P$:
\begin{align}
    \frac{d\mathcal{A}_\phi}{dt} &= \int e^\phi (P - H\theta^+) dA\\
    &= \int e^\phi (P - (\theta^+ - P)\theta^+) dA\\
    &= \int e^\phi (P - (\theta^+)^2 + P\theta^+) dA\\
    &= \int e^\phi (P(1 + \theta^+) - (\theta^+)^2) dA
\end{align}

For small $\theta^+$ (near MOTS): $1 + \theta^+ \approx 1$, so:
\begin{equation}
    \frac{d\mathcal{A}_\phi}{dt} \approx \int e^\phi (P - (\theta^+)^2) dA
\end{equation}

This is STILL not obviously positive!

%% ============================================================================
\section{Conclusion: The Irreducible Difficulty}
%% ============================================================================

After extensive exploration, the fundamental difficulty is:

\textbf{The constraint equations and DEC do NOT determine the sign of the 
mean curvature $H$ on trapped surfaces.}

This is not a technical issue but a GEOMETRIC FACT:
\begin{itemize}
    \item There exist initial data with trapped surfaces having $H > 0$ (Area Dominance likely holds)
    \item There exist initial data with trapped surfaces having $H < 0$ (Area Dominance status unclear)
\end{itemize}

\textbf{The question is:} For initial data arising from dynamical evolution 
under DEC, is $H \ge 0$ forced?

This connects to COSMIC CENSORSHIP - the dynamics of gravitational collapse.

\section{The Path Forward}

Area Dominance may require:
\begin{enumerate}
    \item \textbf{Assuming} it as an additional physical condition
    \item Proving it \textbf{generically} (for "most" initial data)
    \item Deriving it from \textbf{dynamical considerations} (not just constraint equations)
    \item Using \textbf{spacetime methods} (null hypersurfaces, event horizon analysis)
\end{enumerate}

\textbf{This is a genuinely open problem at the frontier of mathematical GR.}

\end{document}
