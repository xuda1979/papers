% =========================================================================
%     NOVEL EXPLORATION 2: THE MASS-ASPECT FUNCTION APPROACH
%
%     Using the mass aspect and its properties on trapped surfaces
%
%     Author: Da Xu
%     Date: December 2025
% =========================================================================

\documentclass[12pt]{article}
\usepackage{amsmath,amsthm,amssymb}
\usepackage{mathrsfs}
\usepackage{tcolorbox}

\theoremstyle{plain}
\newtheorem{theorem}{Theorem}[section]
\newtheorem{lemma}[theorem]{Lemma}
\newtheorem{proposition}[theorem]{Proposition}
\newtheorem{corollary}[theorem]{Corollary}
\newtheorem{conjecture}[theorem]{Conjecture}

\theoremstyle{definition}
\newtheorem{definition}[theorem]{Definition}
\newtheorem{remark}[theorem]{Remark}
\newtheorem{observation}[theorem]{Observation}

\newcommand{\ADM}{\mathrm{ADM}}
\newcommand{\tr}{\mathrm{tr}}
\newcommand{\Div}{\mathrm{div}}
\newcommand{\Area}{\mathrm{Area}}

\title{\textbf{Novel Exploration 2: The Mass Aspect Function}}
\author{Da Xu}
\date{December 2025}

\begin{document}
\maketitle

\section{The Mass Aspect Function}

\subsection{Definition}

On a 2-surface $\Sigma$ embedded in initial data $(M, g, k)$:

\begin{definition}
The \textbf{mass aspect function} is:
\[
    \mu_\Sigma = \frac{1}{8\pi}\left(R_\Sigma - \frac{1}{2}H^2 + \frac{1}{2}(\tr_\Sigma k)^2 
    - |A|^2 + |\chi|^2\right)
\]
where:
\begin{itemize}
    \item $R_\Sigma$ = intrinsic scalar curvature of $\Sigma$
    \item $H$ = mean curvature
    \item $A$ = traceless second fundamental form
    \item $\chi$ = traceless part of $k|_\Sigma$
\end{itemize}
\end{definition}

\subsection{Integral Formula}

\begin{theorem}
The integral of the mass aspect gives:
\[
    \int_\Sigma \mu_\Sigma \, dA = m_H(\Sigma) + \text{(corrections)}
\]
where $m_H$ is the Hawking mass.
\end{theorem}

\subsection{Positivity}

\begin{lemma}
Under DEC, for surfaces in regions with $R_g \geq 0$:
\[
    \int_\Sigma \mu_\Sigma \, dA \geq 0
\]
when $\Sigma$ is a stable MOTS.
\end{lemma}

\section{The Mass Aspect on Trapped Surfaces}

\subsection{Decomposition}

For a trapped surface with $\theta^+ = H + \tr_\Sigma k < 0$:

\[
    \mu_\Sigma = \frac{1}{8\pi}\left(R_\Sigma - \frac{(\theta^+ + \theta^-)^2}{8} 
    + \frac{(\theta^+ - \theta^-)^2}{8} - |A|^2 + |\chi|^2\right)
\]

Simplifying:
\[
    \mu_\Sigma = \frac{1}{8\pi}\left(R_\Sigma - \frac{\theta^+\theta^-}{2} - |A|^2 + |\chi|^2\right)
\]

\subsection{Sign Analysis}

For trapped surfaces:
\begin{itemize}
    \item $\theta^+\theta^- > 0$ (both negative)
    \item $-\theta^+\theta^-/2 < 0$ 
\end{itemize}

So this term is negative, but $R_\Sigma$ (by Gauss-Bonnet, for spheres) integrates to $8\pi$.

\begin{lemma}
For a topological sphere:
\[
    \int_\Sigma \mu_\Sigma \, dA = 1 - \frac{1}{16\pi}\int_\Sigma \theta^+\theta^- \, dA 
    - \frac{1}{8\pi}\int_\Sigma (|A|^2 - |\chi|^2) \, dA
\]
\end{lemma}

\section{A New Mass Definition}

\subsection{The Aspect Mass}

\begin{definition}
The \textbf{aspect mass} of a surface $\Sigma$ is:
\[
    m_A(\Sigma) = \sqrt{\frac{\Area(\Sigma)}{16\pi}} \cdot \int_\Sigma \mu_\Sigma \, dA
\]
\end{definition}

\subsection{Properties}

For large spheres at infinity:
\[
    m_A(S_r) \to M_{\ADM} \quad \text{as } r \to \infty
\]

For minimal surfaces ($H = 0$, $\tr_\Sigma k = 0$):
\[
    m_A(\Sigma) = \sqrt{\frac{\Area}{16\pi}} \cdot \left(1 - \frac{1}{8\pi}\int |A|^2 \, dA\right)
\]

\subsection{For Trapped Surfaces}

\begin{proposition}
For a trapped surface $\Sigma_0$:
\[
    m_A(\Sigma_0) = \sqrt{\frac{\Area(\Sigma_0)}{16\pi}} \cdot 
    \left(1 - \frac{1}{16\pi}\int \theta^+\theta^- \, dA + \cdots\right)
\]
\end{proposition}

Since $\theta^+\theta^- > 0$, the factor in parentheses is $< 1$.

So $m_A(\Sigma_0) < \sqrt{\Area(\Sigma_0)/(16\pi)}$.

\textbf{Wrong direction again!}

\section{Reversing the Logic}

\subsection{Key Insight}

We keep getting $m(\Sigma_0) < \sqrt{\Area/(16\pi)}$ for trapped surfaces.

But we want $M_{\ADM} > \sqrt{\Area/(16\pi)}$.

\textbf{New idea:} Instead of trying to show $m(\Sigma_0)$ is large, show that 
the ``gap'' $\sqrt{\Area/(16\pi)} - m(\Sigma_0)$ is bounded by something that
also appears in $M_{\ADM} - m(\Sigma_0)$!

\subsection{The Gap Approach}

Define:
\[
    \Delta(\Sigma) = \sqrt{\frac{\Area(\Sigma)}{16\pi}} - m(\Sigma)
\]

For trapped surfaces: $\Delta(\Sigma_0) > 0$.

\begin{conjecture}[Gap Bound]
For any monotone mass $m$:
\[
    M_{\ADM} - m(\Sigma_0) \geq \Delta(\Sigma_0) = \sqrt{\frac{\Area(\Sigma_0)}{16\pi}} - m(\Sigma_0)
\]
\end{conjecture}

If true:
\[
    M_{\ADM} \geq \sqrt{\frac{\Area(\Sigma_0)}{16\pi}}
\]

\textbf{This is Penrose!}

\subsection{Analysis}

For the Hawking mass:
\begin{align}
    \Delta(\Sigma_0) &= \sqrt{\frac{A}{16\pi}} - m_H(\Sigma_0) \\
    &= \sqrt{\frac{A}{16\pi}} - \sqrt{\frac{A}{16\pi}}\left(1 - \frac{1}{16\pi}\int H^2\right) \\
    &= \sqrt{\frac{A}{16\pi}} \cdot \frac{1}{16\pi}\int H^2 \, dA
\end{align}

And:
\[
    M_{\ADM} - m_H(\Sigma_0) = \int_{\Sigma_0}^\infty \frac{dm_H}{dt} \, dt
\]

Under IMCF with $R \geq 0$: $\frac{dm_H}{dt} \geq 0$.

\textbf{The question:} Is the total change $M_{\ADM} - m_H(\Sigma_0)$ at least $\Delta(\Sigma_0)$?

\section{The Defect Functional}

\subsection{Definition}

\begin{definition}
The \textbf{Penrose defect} of $\Sigma$ is:
\[
    D(\Sigma) = \sqrt{\frac{\Area(\Sigma)}{16\pi}} - M_{\ADM}
\]
\end{definition}

The Penrose inequality states: $D(\Sigma) \leq 0$ for all trapped $\Sigma$.

\subsection{Expressing the Defect}

\[
    D(\Sigma) = \sqrt{\frac{A}{16\pi}} - m_H(\Sigma) + m_H(\Sigma) - M_{\ADM}
\]
\[
    = \Delta(\Sigma) - (M_{\ADM} - m_H(\Sigma))
\]

For Penrose: need $\Delta(\Sigma) \leq M_{\ADM} - m_H(\Sigma)$.

\subsection{What We Know}

For trapped surfaces reaching out to MOTS $\Sigma^*$:
\begin{align}
    M_{\ADM} - m_H(\Sigma_0) &= \underbrace{(m_H(\Sigma^*) - m_H(\Sigma_0))}_{\text{trapped region}} 
    + \underbrace{(M_{\ADM} - m_H(\Sigma^*))}_{\text{exterior}}
\end{align}

The exterior contribution is non-negative (by known Penrose for MOTS).

The trapped region contribution is the mystery!

\section{Inside the Trapped Region}

\subsection{Evolution of Hawking Mass}

Moving from $\Sigma_0$ to $\Sigma^*$ (both in trapped region):

\textbf{Problem:} No good flow exists! 
\begin{itemize}
    \item IMCF requires $H > 0$
    \item MCF with $H < 0$ goes inward
\end{itemize}

\subsection{Direct Comparison}

\begin{lemma}
\[
    m_H(\Sigma^*) - m_H(\Sigma_0) = \sqrt{\frac{A^*}{16\pi}}\left(1 - \frac{1}{16\pi}\int_{\Sigma^*} H^2\right) 
    - \sqrt{\frac{A_0}{16\pi}}\left(1 - \frac{1}{16\pi}\int_{\Sigma_0} H^2\right)
\]
\end{lemma}

For $\Sigma^*$ (MOTS): $H = -\tr_\Sigma k$, so $H^2 = (\tr_\Sigma k)^2$.

For $\Sigma_0$ (trapped): $H = \theta^+ - \tr_\Sigma k < -\tr_\Sigma k$ if $\theta^+ < 0$.

\section{The Fundamental Question}

\begin{tcolorbox}[colback=yellow!10, colframe=orange!75!black]
\textbf{The Core Question:}

In the trapped region, as we move from $\Sigma_0$ to $\Sigma^*$:
\begin{enumerate}
    \item Area decreases (we know this)
    \item $H^2$ integral... changes how?
    \item Net effect on $m_H$?
\end{enumerate}

If $m_H(\Sigma^*) \geq m_H(\Sigma_0)$ despite $A^* < A_0$, then the $H^2$ term
must compensate!
\end{tcolorbox}

\subsection{Compensation Mechanism}

For $m_H(\Sigma^*) \geq m_H(\Sigma_0)$:
\[
    \sqrt{\frac{A^*}{16\pi}}\left(1 - \frac{\int H^2_{*}}{16\pi}\right) \geq 
    \sqrt{\frac{A_0}{16\pi}}\left(1 - \frac{\int H^2_0}{16\pi}\right)
\]

Let $A^* = A_0 - \delta A$ with $\delta A > 0$.

Need:
\[
    \sqrt{1 - \frac{\delta A}{A_0}}\left(1 - \frac{\int H^2_{*}}{16\pi}\right) \geq 
    \left(1 - \frac{\int H^2_0}{16\pi}\right)
\]

This can hold if $\int H^2_*$ is sufficiently smaller than $\int H^2_0$!

\subsection{Physical Interpretation}

On MOTS: $H = -\tr_\Sigma k$, so $|H|$ is controlled by $|k|$.

On deeply trapped surfaces: $H$ includes $\theta^+$ contribution, which can be large.

\textbf{Deeper trapped surfaces can have larger $|H|$!}

So moving from deeply trapped $\Sigma_0$ to marginally trapped $\Sigma^*$:
\begin{itemize}
    \item Area decreases (bad)
    \item $H^2$ integral decreases (good)
    \item These might compensate!
\end{itemize}

\section{A New Conjecture}

\begin{conjecture}[Trapped Hawking Monotonicity]
In the trapped region, the Hawking mass is non-decreasing as we move from
more trapped surfaces to less trapped surfaces (toward MOTS):
\[
    |\theta^+(\Sigma_1)| > |\theta^+(\Sigma_2)| \implies m_H(\Sigma_1) \leq m_H(\Sigma_2)
\]
\end{conjecture}

\subsection{Testing in Schwarzschild}

In Schwarzschild, for spheres at radius $r < 2M$:
\[
    \theta^+ = \frac{2}{r}\sqrt{1 - \frac{2M}{r}} - \frac{2M}{r^2\sqrt{1 - 2M/r}} \cdot (\text{k contribution})
\]

Actually, let me compute more carefully.

In standard Schwarzschild slicing:
\[
    H = \frac{2}{r}\sqrt{1 - \frac{2M}{r}}
\]

Wait, this is positive for $r > 2M$ and complex for $r < 2M$ in this slicing!

Need a different slicing for the interior.

\textbf{This requires careful analysis in appropriate coordinates.}

\section{Conclusion}

\begin{tcolorbox}[colback=blue!10, colframe=blue!75!black]
\textbf{Key New Ideas:}

\begin{enumerate}
    \item \textbf{Gap approach:} Instead of proving $m(\Sigma_0)$ large, prove
    $M_{\ADM} - m(\Sigma_0) \geq \sqrt{A/(16\pi)} - m(\Sigma_0)$
    
    \item \textbf{Compensation mechanism:} Area decrease from $\Sigma_0$ to $\Sigma^*$
    might be compensated by $H^2$ integral decrease
    
    \item \textbf{Trapped Hawking monotonicity:} $m_H$ might increase as trapping decreases
\end{enumerate}

\textbf{These ideas require detailed analysis in appropriate coordinate systems.}
\end{tcolorbox}

\end{document}
