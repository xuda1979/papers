%% CONSTRAINT_SOBOLEV_ATTACK.tex
%%
%% Using Constraint Equations with Sobolev Estimates
%% Direct Attack on Area Dominance
%%
%% December 2025

\documentclass[11pt]{amsart}
\usepackage{amsmath,amssymb,amsthm}
\usepackage{mathtools}
\usepackage{xcolor}
\usepackage{tcolorbox}

\tcbuselibrary{theorems}

\newtcolorbox{breakthrough}{
    colback=green!10!white,
    colframe=green!75!black,
    title={\textbf{BREAKTHROUGH}}
}

\newtcolorbox{gap}{
    colback=red!5!white,
    colframe=red!75!black,
    title={\textbf{GAP}}
}

\newtheorem{theorem}{Theorem}[section]
\newtheorem{lemma}[theorem]{Lemma}
\newtheorem{proposition}[theorem]{Proposition}
\newtheorem{corollary}[theorem]{Corollary}
\newtheorem{definition}[theorem]{Definition}

\newcommand{\ADM}{\mathrm{ADM}}
\newcommand{\Area}{\mathrm{Area}}
\newcommand{\tr}{\mathrm{tr}}
\newcommand{\dive}{\mathrm{div}}
\newcommand{\mtheta}{m_\theta}
\newcommand{\Hs}[1]{H^{#1}}
\newcommand{\Sob}[2]{W^{#1,#2}}

\title{Constraint Equation Analysis:\\
Sobolev Estimates for Penrose Inequality}
\author{}
\date{December 2025}

\begin{document}
\maketitle

\begin{abstract}
We use the constraint equations of general relativity combined with Sobolev embedding theorems to derive new estimates for the spacetime Penrose inequality. We establish sharp bounds relating the ADM mass to surface areas via integral identities derived from the constraints.
\end{abstract}

\tableofcontents

%% ============================================================================
\section{The Constraint Equations}
%% ============================================================================

\subsection{Hamiltonian and Momentum Constraints}

On initial data $(M^3, g, k)$:
\begin{align}
    R_g - |k|^2 + (\tr k)^2 &= 2\mu \label{eq:ham}\\
    \dive(k - (\tr k)g) &= J \label{eq:mom}
\end{align}

where $\mu \ge 0$ is energy density and $|J| \le \mu$ (DEC).

\subsection{Constraint Energy}

Define the \textbf{constraint energy density}:
\begin{equation}
    \mathcal{E} = \frac{1}{2}(R_g + (\tr k)^2 - |k|^2) = \mu \ge 0
\end{equation}

\subsection{Key Identity}

\begin{lemma}[Scalar Curvature Decomposition]\label{lem:scalar-decomp}
\begin{equation}
    R_g = 2\mu + |k|^2 - (\tr k)^2 = 2\mu + |k - \frac{1}{3}(\tr k)g|^2 - \frac{2}{3}(\tr k)^2
\end{equation}
\end{lemma}

For maximal slices ($\tr k = 0$):
\begin{equation}
    R_g = 2\mu + |k|^2 \ge 2\mu \ge 0
\end{equation}

%% ============================================================================
\section{ADM Mass via Constraint Integrals}
%% ============================================================================

\subsection{The Mass Integral}

\begin{theorem}[ADM Mass Identity]\label{thm:adm-identity}
\begin{equation}
    M_{\ADM} = \frac{1}{16\pi}\int_M (R_g + (\tr k)^2 - |k|^2)\, dV_g + \text{(boundary at infinity)}
\end{equation}

With DEC:
\begin{equation}
    M_{\ADM} = \frac{1}{8\pi}\int_M \mu\, dV_g + \text{(boundary)}
\end{equation}
\end{theorem}

\subsection{Mass with Inner Boundary}

For a region $\Omega$ with inner boundary $\partial\Omega = \Sigma$:

\begin{theorem}[Mass with Boundary]\label{thm:mass-boundary}
\begin{equation}
    M_{\ADM} = \frac{1}{8\pi}\int_{M\setminus\Omega} \mu\, dV_g + \frac{1}{8\pi}\int_\Sigma (H - \tr_\Sigma k)\nu \cdot n\, dA + \text{(terms)}
\end{equation}
where $\nu$ is outward normal to $\Sigma$, $n$ is future timelike normal.
\end{theorem}

\subsection{Null Expansion Appears}

Notice: $H - \tr_\Sigma k = \theta^-$ (ingoing null expansion).

And: $H + \tr_\Sigma k = \theta^+$ (outgoing null expansion).

So:
\begin{equation}
    H = \frac{1}{2}(\theta^+ + \theta^-), \quad \tr_\Sigma k = \frac{1}{2}(\theta^+ - \theta^-)
\end{equation}

%% ============================================================================
\section{Sobolev Estimates on 3-Manifolds}
%% ============================================================================

\subsection{Key Embeddings}

\begin{theorem}[Sobolev Embedding in 3D]\label{thm:sobolev-3d}
For $(M^3, g)$ asymptotically flat:
\begin{enumerate}
    \item $\Hs{1}(M) \hookrightarrow L^6(M)$
    \item $\Hs{2}(M) \hookrightarrow L^\infty(M)$
    \item $\Sob{1}{3/2}(M) \hookrightarrow L^3(M)$
\end{enumerate}

Constants depend on $(M, g)$ geometry.
\end{theorem}

\subsection{Trace Theorem}

\begin{theorem}[Trace Inequality]\label{thm:trace}
For $u \in \Hs{1}(M)$ and surface $\Sigma \subset M$:
\begin{equation}
    \|u|_\Sigma\|_{L^2(\Sigma)} \le C(\Sigma)\|u\|_{\Hs{1}(M)}
\end{equation}
\end{theorem}

%% ============================================================================
\section{New Estimate: $\theta^+$ via Constraints}
%% ============================================================================

\subsection{Main Technical Result}

\begin{theorem}[$L^2$ Control of $\theta^+$ via Energy]\label{thm:theta-energy}
For a surface $\Sigma$ bounding region $\Omega$:
\begin{equation}
    \int_\Sigma (\theta^+)^2 dA \le C\left(\int_\Omega \mu\, dV + \int_\Omega |k|^2 dV + \|\nabla k\|_{L^2(\Omega)}^2\right)
\end{equation}
\end{theorem}

\begin{proof}
\textbf{Step 1:} Express $\theta^+$ in terms of bulk quantities.

On $\Sigma$: $\theta^+ = H + \tr_\Sigma k$.

The mean curvature $H$ satisfies:
\begin{equation}
    H = -\Delta_M d|_\Sigma + O(|A|^2)
\end{equation}
where $d$ is distance function from $\Sigma$ (for small distances).

\textbf{Step 2:} Use constraint equations.

From (\ref{eq:ham}):
\begin{equation}
    \int_\Omega R_g\, dV = 2\int_\Omega\mu\, dV + \int_\Omega|k|^2 dV - \int_\Omega(\tr k)^2 dV
\end{equation}

By Gauss-Bonnet generalization:
\begin{equation}
    \int_\Omega R_g\, dV = \int_\Sigma H\, dA + \text{(curvature integrals)}
\end{equation}

\textbf{Step 3:} Bound $H$ in terms of bulk energy.

\begin{equation}
    \|H\|_{L^2(\Sigma)} \le C\|R_g\|_{L^1(\Omega)}^{1/2} + C' \le C\left(\int_\Omega\mu\, dV\right)^{1/2} + C'|k|_{L^2}
\end{equation}

\textbf{Step 4:} Bound $\tr_\Sigma k$.

By trace theorem:
\begin{equation}
    \|\tr_\Sigma k\|_{L^2(\Sigma)} \le C\|k\|_{\Hs{1}(\Omega)}
\end{equation}

\textbf{Step 5:} Combine.

\begin{equation}
    \|\theta^+\|_{L^2(\Sigma)} \le \|H\|_{L^2} + \|\tr_\Sigma k\|_{L^2} \le C\left(\|\mu\|_{L^1}^{1/2} + \|k\|_{\Hs{1}}\right)
\end{equation}

Squaring:
\begin{equation}
    \int_\Sigma(\theta^+)^2 dA \le C'\left(\int_\Omega\mu\, dV + \|k\|_{\Hs{1}(\Omega)}^2\right)
\end{equation}
\end{proof}

%% ============================================================================
\section{Consequence: Mass-Area-$\theta^+$ Inequality}
%% ============================================================================

\begin{theorem}[Main Inequality]\label{thm:main-ineq}
Let $(M, g, k)$ be asymptotically flat with DEC. For any surface $\Sigma$ bounding region $\Omega$:
\begin{equation}
    M_{\ADM} \ge \frac{1}{8\pi}\int_{M\setminus\Omega}\mu\, dV \ge 0
\end{equation}

If $\Sigma$ is trapped ($\theta^+ < 0$), then:
\begin{equation}
    M_{\ADM} \ge \frac{1}{8\pi}\int_{M\setminus\Omega}\mu\, dV + \mtheta(\Sigma) - \sqrt{\frac{A}{16\pi}}(1 - \frac{1}{16\pi}\int(\theta^+)^2 dA)
\end{equation}
\end{theorem}

Wait, this is circular. Let me try a different approach.

%% ============================================================================
\section{Direct Approach: Conformal Method with Constraints}
%% ============================================================================

\subsection{Conformal Transformation}

Let $\tilde{g} = u^4 g$ with $u > 0$, $u \to 1$ at infinity.

Under this transformation:
\begin{equation}
    R_{\tilde{g}} = u^{-5}(-8\Delta_g u + R_g u)
\end{equation}

\subsection{Lichnerowicz Equation}

To achieve $R_{\tilde{g}} = 0$ (for positive mass theorem):
\begin{equation}
    -8\Delta_g u + R_g u = 0
\end{equation}

With boundary condition on $\Sigma$:
\begin{equation}
    \frac{\partial u}{\partial\nu} = \frac{H}{4}u \quad \text{to make } \tilde{H} = 0
\end{equation}

\subsection{Mass Transformation}

\begin{lemma}[ADM Mass Under Conformal Change]\label{lem:conf-mass}
\begin{equation}
    M_{\ADM}(\tilde{g}) = M_{\ADM}(g) - 2\lim_{r\to\infty}\int_{S_r}(u-1)\, dA
\end{equation}

For $u = 1 + \frac{\alpha}{r} + O(r^{-2})$:
\begin{equation}
    M_{\ADM}(\tilde{g}) = M_{\ADM}(g) - 2\alpha
\end{equation}
\end{lemma}

\subsection{Solving for $\alpha$}

From the Lichnerowicz equation with inner boundary at $\Sigma$:
\begin{equation}
    \alpha = \frac{1}{8\pi}\int_\Sigma H u\, dA - \frac{1}{16\pi}\int_M R_g u\, dV
\end{equation}

Using $R_g = 2\mu + |k|^2 - (\tr k)^2$ and DEC:
\begin{equation}
    \int_M R_g u\, dV \ge \int_M 2\mu u\, dV - \int_M(\tr k)^2 u\, dV
\end{equation}

\subsection{The Key Estimate}

\begin{theorem}[Conformal Mass Bound]\label{thm:conf-bound}
If $\Sigma$ is minimal in $\tilde{g}$ (achieved by boundary condition), then:
\begin{equation}
    M_{\ADM}(g) \ge M_{\ADM}(\tilde{g}) + 2\alpha = M_{\ADM}(\tilde{g}) + \frac{1}{4\pi}\int_\Sigma H u\, dA - \frac{1}{8\pi}\int_M R_g u\, dV
\end{equation}

By positive mass theorem: $M_{\ADM}(\tilde{g}) \ge 0$.

By Riemannian Penrose on $(\tilde{M}, \tilde{g})$:
\begin{equation}
    M_{\ADM}(\tilde{g}) \ge \sqrt{\frac{\tilde{A}(\Sigma)}{16\pi}}
\end{equation}
\end{theorem}

\begin{gap}
\textbf{Issue:} The area $\tilde{A}$ in the conformal metric differs from $A$ in original metric:
\begin{equation}
    \tilde{A} = \int_\Sigma u^4 dA
\end{equation}

We need $u \le 1$ on $\Sigma$ to get $\tilde{A} \le A$, but the solution $u$ is determined by the PDE, not freely choosable.
\end{gap}

%% ============================================================================
\section{Maximum Principle for Conformal Factor}
%% ============================================================================

\subsection{Setup}

The conformal factor $u$ solves:
\begin{align}
    -8\Delta u + R_g u &= 0 \quad \text{in } M\setminus\Omega \\
    \frac{\partial u}{\partial\nu} &= \frac{H}{4}u \quad \text{on } \Sigma \\
    u &\to 1 \quad \text{at infinity}
\end{align}

\subsection{Sign of $u$}

\begin{lemma}[Positivity]\label{lem:positivity}
If $R_g \ge 0$ (from DEC on maximal slice), then $u > 0$ throughout.
\end{lemma}

\begin{proof}
Maximum principle for $-\Delta u + \frac{R_g}{8}u = 0$ with $R_g \ge 0$.
\end{proof}

\subsection{Bounds on $u$}

\begin{theorem}[$u$ Bound on $\Sigma$]\label{thm:u-bound}
On a maximal slice with DEC:
\begin{enumerate}
    \item If $H \ge 0$ on $\Sigma$: $u \le 1$ everywhere by maximum principle
    \item If $H < 0$ somewhere on $\Sigma$: $u$ may exceed 1
\end{enumerate}
\end{theorem}

\begin{proof}
At interior maximum of $u$: $\Delta u \le 0$, so $R_g u \le 0$. Since $R_g \ge 0$ and $u > 0$, this requires $R_g = 0$ at the maximum.

At boundary: if $u$ has maximum on $\Sigma$, then $\frac{\partial u}{\partial\nu} \ge 0$. By boundary condition: $H u \ge 0$.

If $H \ge 0$: consistent.
If $H < 0$: $u$ cannot have maximum on $\Sigma$ with $u > 0$, so maximum is at infinity.

Therefore if $H \ge 0$ everywhere: $\max u = 1$ (at infinity).

If $H < 0$ somewhere: maximum could be interior or at infinity, but $u$ could exceed 1 near $\Sigma$.
\end{proof}

\textbf{Critical:} Trapped surfaces have $\theta^+ = H + \tr k < 0$. On maximal slice $\tr k = 0$, so $H < 0$.

This means $u$ may be $> 1$ on $\Sigma$, giving $\tilde{A} > A$. This is the WRONG direction!

%% ============================================================================
\section{Alternative: Inverse Problem}
%% ============================================================================

\subsection{Idea}

Instead of making $\Sigma$ minimal in a new metric, find $u$ such that:
\begin{equation}
    \tilde{A} = u^4 A = A^* \quad \text{(area of MOTS)}
\end{equation}

Then:
\begin{equation}
    M_{\ADM}(\tilde{g}) \ge \sqrt{\frac{A^*}{16\pi}}
\end{equation}

\subsection{The Equation}

We need $u$ such that:
\begin{itemize}
    \item $\tilde{R} \ge 0$ (for positive mass)
    \item $\tilde{A}(\Sigma) = A^*$
    \item $u \to 1$ at infinity
\end{itemize}

The area condition: $\int_\Sigma u^4 dA = A^*$.

If $A < A^*$ (area dominance holds): need $u > 1$ on $\Sigma$ on average.
If $A > A^*$ (area dominance fails): need $u < 1$ on $\Sigma$ on average.

\textbf{But we don't know which case holds!} This is the whole problem.

%% ============================================================================
\section{Final Technical Attempt: Parabolic Flow}
%% ============================================================================

\subsection{Ricci-Type Flow}

Consider the flow:
\begin{equation}
    \partial_t g = -2(R_{ij} - \frac{R}{6}g_{ij}) = -2\mathring{R}_{ij}
\end{equation}

This is trace-free Ricci flow, preserving volume.

\subsection{Area Evolution}

For a surface $\Sigma$ fixed in coordinates:
\begin{equation}
    \frac{d\Area(\Sigma)}{dt} = -\int_\Sigma (R_{nn} + R_{\gamma})\, dA
\end{equation}

where $R_{nn}$ is normal-normal Ricci component and $R_\gamma$ is the trace over tangent directions.

\subsection{Using Constraint}

With $R_g = 2\mu + |k|^2 - (\tr k)^2$ and $\mu \ge 0$:

On maximal slice: $R_g \ge |k|^2 \ge 0$.

But the sign of $R_{nn} + R_\gamma$ (the components relevant for area) is not determined by total $R_g$.

%% ============================================================================
\section{Conclusion: Status of Hard Analysis}
%% ============================================================================

\begin{tcolorbox}[colback=yellow!10!white, colframe=yellow!75!black, title=\textbf{HARD ANALYSIS CONCLUSION}]

\textbf{Methods Applied:}
\begin{enumerate}
    \item Sobolev embedding and trace theorems
    \item Constraint equation integrals
    \item Conformal method with boundary conditions
    \item Maximum principle for conformal factor
    \item Parabolic flow analysis
\end{enumerate}

\textbf{Key Finding:}

The conformal method shows WHY the problem is hard:
\begin{itemize}
    \item For trapped surfaces: $H < 0$ on maximal slices
    \item This makes conformal factor $u > 1$ on $\Sigma$
    \item Therefore $\tilde{A} > A$, giving bound in wrong direction
\end{itemize}

\textbf{The Hard Analysis Reveals:}

There is no ``universal'' analytical argument that proves $A(\Sigma) \le A(\Sigma^*)$ for all trapped $\Sigma$. The sign of $H$ on trapped surfaces (which can be negative) fundamentally obstructs conformal approaches.

\textbf{What WOULD Work:}

If we had additional structure:
\begin{itemize}
    \item Axisymmetry (reduces to ODE)
    \item Spherical symmetry (explicit solution)
    \item Bounds on $|k|$ in terms of $\mu$ (extra energy condition)
\end{itemize}

The generic case remains open.
\end{tcolorbox}

\end{document}
