% =========================================================================
%     CRITICAL ANALYSIS OF THE SPECTRAL CONFORMAL APPROACH
%
%     Examining whether the spectral positivity transfer is valid
%
%     Author: Da Xu
%     Date: December 2025
% =========================================================================

\documentclass[12pt]{article}
\usepackage{amsmath,amsthm,amssymb}
\usepackage{tcolorbox}

\newtheorem{theorem}{Theorem}
\newtheorem{lemma}[theorem]{Lemma}
\newtheorem{proposition}[theorem]{Proposition}
\newtheorem{counterexample}[theorem]{Counterexample}
\newtheorem{remark}[theorem]{Remark}
\newtheorem{problem}[theorem]{Problem}

\newcommand{\tr}{\mathrm{tr}}

\title{\textbf{Critical Analysis: Does the Spectral Approach Work?}}
\author{Da Xu}
\date{December 2025}

\begin{document}
\maketitle

\section{The Claimed Lemma}

\begin{lemma}[Spectral Positivity Transfer -- CLAIMED]
Let $f \in L^2(\Sigma)$ with spectral decomposition $f = \sum_n c_n \phi_n$. If 
$\langle f, \phi_1 \rangle = c_1 \|\phi_1\|^2 \geq 0$, then:
\begin{equation}
    \langle f, \phi_1^{-2} \rangle \geq -C \|f\|_{L^2}
\end{equation}
for some constant $C > 0$.
\end{lemma}

\section{Analysis of the Proof}

The proof claims:
\begin{align}
    \langle f, \phi_1^{-2} \rangle &= c_1 \langle \phi_1, \phi_1^{-2} \rangle + \sum_{n \geq 2} c_n \langle \phi_n, \phi_1^{-2} \rangle \\
    &= c_1 \|\phi_1^{-1}\|_{L^1} + \sum_{n \geq 2} c_n \langle \phi_n, \phi_1^{-2} \rangle
\end{align}

\textbf{First term:} $c_1 \geq 0$ by assumption, $\|\phi_1^{-1}\|_{L^1} > 0$ since $\phi_1 > 0$. So first term $\geq 0$. \checkmark

\textbf{Second term:} The proof bounds this using Cauchy-Schwarz:
\begin{equation}
    \left|\sum_{n \geq 2} c_n \langle \phi_n, \phi_1^{-2} \rangle\right| \leq \|f - c_1\phi_1\|_{L^2} \cdot \|\phi_1^{-2}\|_{\mathcal{H}^{-1}}
\end{equation}

This gives: $\langle f, \phi_1^{-2}\rangle \geq -C\|f\|_{L^2}$.

\section{The Problem}

\textbf{The lemma is CORRECT but INSUFFICIENT!}

The lemma says $\langle f, \phi_1^{-2}\rangle \geq -C\|f\|_{L^2}$, which is a 
\textbf{lower bound that can be negative}.

For the proof to work, we need:
\begin{equation}
    \int_\Sigma (\tr_\Sigma k) \cdot \phi_1^{-2} \, dA \geq 0
\end{equation}
or at least
\begin{equation}
    \int_\Sigma (\tr_\Sigma k) \cdot \phi_1^{-2} \, dA \geq -\epsilon \cdot \sqrt{\frac{A(\Sigma)}{16\pi}}
\end{equation}
for small $\epsilon$.

\textbf{But the lemma only gives:}
\begin{equation}
    \int_\Sigma (\tr_\Sigma k) \cdot \phi_1^{-2} \, dA \geq -C \|\tr_\Sigma k\|_{L^2}
\end{equation}

The constant $C$ depends on the geometry of $\Sigma$ (specifically, on 
$\|\phi_1^{-2}\|$ in some norm). There is no guarantee that this is 
small relative to the Penrose mass!

\section{Explicit Counterexample}

Let me construct an example where the spectral approach fails.

\begin{counterexample}
Consider a MOTS $\Sigma = S^2$ with stability operator having:
\begin{itemize}
    \item Principal eigenfunction: $\phi_1(x) = 1 + 0.1 \cos\theta$ (nearly constant)
    \item Normalized so $\|\phi_1\|_{L^2} = 1$
\end{itemize}

Let:
\begin{equation}
    \tr_\Sigma k(x) = \begin{cases}
        +1 & \text{on } \theta \in [0, \pi/2) \\
        -0.9 & \text{on } \theta \in [\pi/2, \pi]
    \end{cases}
\end{equation}

Then:
\begin{equation}
    \int_\Sigma (\tr_\Sigma k) \cdot \phi_1 \, dA \approx (+1)(2\pi)(1.05) + (-0.9)(2\pi)(0.95) = 2\pi(1.05 - 0.855) > 0
\end{equation}
So the variational condition is satisfied.

But:
\begin{equation}
    \int_\Sigma (\tr_\Sigma k) \cdot \phi_1^{-2} \, dA \approx (+1)(2\pi)(0.91) + (-0.9)(2\pi)(1.11) = 2\pi(0.91 - 1) < 0
\end{equation}
The weighted integral with $\phi_1^{-2}$ is \textbf{NEGATIVE}!
\end{counterexample}

\section{The Fundamental Issue}

\begin{tcolorbox}[colback=red!10, colframe=red!50!black, title=THE REAL GAP]
The spectral conformal approach does NOT resolve the fundamental problem.

The variational condition $\int (\tr_\Sigma k)\phi_1 \, dA \geq 0$ gives 
control over the \emph{projection onto $\phi_1$}, but the mass formula 
requires control over $\int (\tr_\Sigma k)\phi_1^{-2} \, dA$, which involves 
a \emph{different weight function}.

There is no spectral relationship between $\langle f, \phi_1\rangle$ and 
$\langle f, \phi_1^{-2}\rangle$ that implies one is non-negative when the 
other is.
\end{tcolorbox}

\section{What Would Actually Work}

For the spectral approach to work, we would need one of:

\begin{enumerate}
    \item \textbf{Pointwise favorable:} $\tr_\Sigma k \geq 0$ everywhere. 
    Then both integrals are non-negative. But this is exactly what we're trying to avoid!
    
    \item \textbf{Spectral dominance:} $\tr_\Sigma k = c_1 \phi_1$ exactly 
    (no higher modes). Then $\langle \tr k, \phi_1^{-2}\rangle = c_1\|\phi_1^{-1}\|_{L^1} \geq 0$. 
    But there's no reason for this to hold in general.
    
    \item \textbf{Small higher modes:} $\|(\tr_\Sigma k) - c_1\phi_1\|_{L^2}$ 
    is small compared to $c_1$. But again, no general reason for this.
\end{enumerate}

\section{Conclusion}

\begin{tcolorbox}[colback=yellow!10, colframe=orange!50!black]
\textbf{Status of the Spectral Approach:} FAILS

The spectral conformal method does not provide a rigorous proof of the 
unconditional Penrose inequality. The ``spectral positivity transfer'' 
lemma is mathematically correct but insufficient --- it only provides a 
lower bound that can be negative.

The fundamental obstruction remains: we need pointwise control on the 
mean curvature jump, but variational methods only give weighted integral 
control.
\end{tcolorbox}

\section{The State of the Problem}

After exploring multiple approaches:
\begin{enumerate}
    \item Maximum area trapped surface → weighted integral, not pointwise
    \item Spectral conformal method → weighted integral doesn't transfer
    \item Modified Lichnerowicz → produces $R \leq 0$
    \item Double Jang → loses Penrose mass contribution
    \item Capacitary bounds → requires $R_g \geq 0$
\end{enumerate}

\textbf{The unconditional spacetime Penrose inequality remains open.}

\end{document}
