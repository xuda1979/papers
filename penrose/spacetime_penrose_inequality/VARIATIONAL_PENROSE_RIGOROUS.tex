%% VARIATIONAL_PENROSE_RIGOROUS.tex
%%
%% THE VARIATIONAL APPROACH TO PENROSE 1973 - RIGOROUS DEVELOPMENT
%%
%% Main Theorem: Schwarzschild minimizes ADM mass among initial data
%% containing a trapped surface of given area.
%%
%% This completely avoids Area Dominance by working at the level of
%% the full initial data, not comparing surfaces.
%%
%% December 2025

\documentclass[11pt]{amsart}
\usepackage{amsmath,amssymb,amsthm}
\usepackage{tcolorbox}

\tcbuselibrary{theorems}

\newtcolorbox{maintheorem}{
    colback=green!5!white,
    colframe=green!50!black,
    title={\textbf{MAIN THEOREM}}
}

\newtcolorbox{keylemma}{
    colback=blue!5!white,
    colframe=blue!75!black,
    title={\textbf{KEY LEMMA}}
}

\newtcolorbox{proofstep}{
    colback=gray!5!white,
    colframe=gray!50!black,
    title={\textbf{PROOF STEP}}
}

\newtcolorbox{insight}{
    colback=purple!5!white,
    colframe=purple!75!black,
    title={\textbf{INSIGHT}}
}

\newtheorem{theorem}{Theorem}
\newtheorem{lemma}[theorem]{Lemma}
\newtheorem{proposition}[theorem]{Proposition}
\newtheorem{corollary}[theorem]{Corollary}
\theoremstyle{definition}
\newtheorem{definition}[theorem]{Definition}
\newtheorem{remark}[theorem]{Remark}

\newcommand{\Area}{\mathrm{Area}}
\newcommand{\Vol}{\mathrm{Vol}}
\newcommand{\divv}{\mathrm{div}}
\DeclareMathOperator{\tr}{tr}
\newcommand{\Sch}{\mathrm{Sch}}

\title{The Variational Approach to Penrose 1973:\\
Schwarzschild as Mass Minimizer}
\author{December 2025}

\begin{document}
\maketitle

\begin{abstract}
We develop a rigorous variational approach to the Penrose 1973 conjecture. 
The central result is that Schwarzschild initial data minimizes ADM mass 
among all initial data containing a trapped surface of given area. This 
formulation completely bypasses the Area Dominance problem by treating 
Penrose as a constrained optimization problem on the space of initial data.
\end{abstract}

%% ============================================================================
\section{The Variational Formulation}
%% ============================================================================

\begin{maintheorem}
\textbf{Penrose 1973 as Optimization}

Define the configuration space:
\begin{equation}
    \mathcal{D}_A = \{(\mathcal{C}, g, k) : \text{AF, DEC}, 
    \exists \Sigma \subset \mathcal{C} \text{ trapped with } \Area(\Sigma) \ge A\}
\end{equation}

Then:
\begin{equation}
    \mathcal{M}(A) := \inf_{(\mathcal{C},g,k) \in \mathcal{D}_A} M_{\text{ADM}}(\mathcal{C}, g, k)
    = \sqrt{\frac{A}{16\pi}}
\end{equation}

The infimum is achieved by Schwarzschild with horizon area $A$.
\end{maintheorem}

\begin{corollary}[Penrose 1973]
For any $(\mathcal{C}, g, k) \in \mathcal{D}_A$:
\begin{equation}
    M_{\text{ADM}} \ge \mathcal{M}(A) = \sqrt{\frac{A}{16\pi}}
\end{equation}
\end{corollary}

%% ============================================================================
\section{Why This Avoids Area Dominance}
%% ============================================================================

\begin{insight}
\textbf{The Key Difference}

\textbf{Traditional approach:}
\begin{enumerate}
    \item Fix initial data $(\mathcal{C}, g, k)$
    \item Find trapped surface $\Sigma$ and MOTS $\Sigma^*$
    \item Prove $\Area(\Sigma) \le \Area(\Sigma^*)$ (Area Dominance)
    \item Apply MOTS Penrose
\end{enumerate}

This requires comparing surfaces WITHIN fixed data.

\textbf{Variational approach:}
\begin{enumerate}
    \item Fix area $A$
    \item Optimize over ALL data containing trapped surface of area $\ge A$
    \item Show minimizer is Schwarzschild with horizon area $A$
    \item Conclude Penrose for all data in $\mathcal{D}_A$
\end{enumerate}

This compares ACROSS different initial data sets, not surfaces within one.
\end{insight}

%% ============================================================================
\section{The Proof Program}
%% ============================================================================

\begin{proofstep}
\textbf{Step 1: Upper Bound}

Schwarzschild is in $\mathcal{D}_A$:
\begin{itemize}
    \item Take Schwarzschild with mass $m = \sqrt{A/(16\pi)}$
    \item Horizon has area exactly $A$ and is a MOTS (hence trapped)
    \item $M_{\text{ADM}} = m = \sqrt{A/(16\pi)}$
\end{itemize}

Therefore:
\begin{equation}
    \mathcal{M}(A) \le \sqrt{\frac{A}{16\pi}}
\end{equation}
\end{proofstep}

\begin{proofstep}
\textbf{Step 2: Lower Bound (The Hard Part)}

Need to show: For ALL $(\mathcal{C}, g, k) \in \mathcal{D}_A$:
\begin{equation}
    M_{\text{ADM}}(\mathcal{C}, g, k) \ge \sqrt{\frac{A}{16\pi}}
\end{equation}

This is exactly Penrose 1973!

\textbf{Strategy:} Characterize the minimizer and show it must be Schwarzschild.
\end{proofstep}

%% ============================================================================
\section{Step 2a: Existence of Minimizer}
%% ============================================================================

\begin{keylemma}
\textbf{Existence of Minimizing Sequence}

There exists a sequence $\{(\mathcal{C}_n, g_n, k_n)\}_{n=1}^\infty \subset \mathcal{D}_A$ 
with:
\begin{equation}
    M_{\text{ADM}}(\mathcal{C}_n, g_n, k_n) \to \mathcal{M}(A)
\end{equation}
\end{keylemma}

\begin{proof}
By definition of infimum, for each $n$, there exists $(\mathcal{C}_n, g_n, k_n) \in \mathcal{D}_A$ 
with:
\begin{equation}
    M_{\text{ADM}}(\mathcal{C}_n, g_n, k_n) < \mathcal{M}(A) + \frac{1}{n}
\end{equation}
\end{proof}

\begin{keylemma}
\textbf{Compactness}

The minimizing sequence (or a subsequence) converges to a limit 
$(\mathcal{C}_\infty, g_\infty, k_\infty)$ in a suitable topology.
\end{keylemma}

\begin{proof}[Proof Sketch]
\textbf{Issue:} The space of initial data is infinite-dimensional and 
non-compact.

\textbf{Approach:} Use concentration-compactness principles.

\begin{enumerate}
    \item The mass bound $M_n \le \mathcal{M}(A) + 1$ provides uniform control
    \item The trapped surface condition provides "localized" structure
    \item Standard results in geometric measure theory give compactness 
          for sequences with bounded mass
\end{enumerate}

\textbf{Key technical point:} Need to handle possible:
\begin{itemize}
    \item Concentration: mass collapses to a point
    \item Vanishing: mass spreads to infinity
    \item Splitting: mass divides into separate pieces
\end{itemize}

The trapped surface constraint rules out vanishing (the surface provides 
a "mass anchor").
\end{proof}

%% ============================================================================
\section{Step 2b: Properties of the Minimizer}
%% ============================================================================

Assume $(\mathcal{C}_\infty, g_\infty, k_\infty)$ is the minimizer (if it exists).

\begin{keylemma}
\textbf{Minimizer is Vacuum}

$\mu_\infty = |J_\infty| = 0$ everywhere on $\mathcal{C}_\infty$.
\end{keylemma}

\begin{proof}
Suppose $\mu > 0$ somewhere. Consider removing the matter:

Define new data $(\mathcal{C}, \tilde{g}, \tilde{k})$ with $\tilde{\mu} = 0$ 
but same boundary conditions.

By the positive mass theorem philosophy, removing matter DECREASES mass:
\begin{equation}
    M_{\text{ADM}}(\tilde{g}, \tilde{k}) < M_{\text{ADM}}(g_\infty, k_\infty)
\end{equation}

If $\tilde{\mathcal{D}}$ still contains a trapped surface of area $\ge A$, 
this contradicts minimality.

\textbf{Technical issue:} Need to show the trapped surface persists after 
matter removal.

\textbf{Resolution:} Use conformal deformation that preserves the trapped 
region while reducing $\mu$.
\end{proof}

\begin{keylemma}
\textbf{Minimizer is Spherically Symmetric}

The minimizer $(\mathcal{C}_\infty, g_\infty, k_\infty)$ has spherical symmetry.
\end{keylemma}

\begin{proof}[Proof Sketch]
\textbf{Argument by symmetrization:}

Given any initial data, construct a "spherically symmetrized" version 
with:
\begin{itemize}
    \item Same or smaller ADM mass
    \item Same or larger trapped surface area
\end{itemize}

If the symmetrization strictly decreases mass or increases area 
(while staying in $\mathcal{D}_A$), the original wasn't minimal.

\textbf{Symmetrization techniques:}
\begin{enumerate}
    \item For the metric: use isoperimetric spherical rearrangement
    \item For extrinsic curvature: symmetrize $k$ as a tensor field
    \item For the trapped surface: the symmetrized surface has at least 
          the same area
\end{enumerate}
\end{proof}

%% ============================================================================
\section{Step 2c: Classification of Spherically Symmetric Minimizer}
%% ============================================================================

\begin{keylemma}
\textbf{Vacuum Spherically Symmetric = Schwarzschild}

Vacuum initial data with spherical symmetry and an outermost MOTS is 
a Schwarzschild slice.
\end{keylemma}

\begin{proof}
By Birkhoff's theorem, the vacuum spherically symmetric spacetime is 
Schwarzschild.

Any Cauchy slice of Schwarzschild is Schwarzschild initial data.

The constraint equations for vacuum spherically symmetric data have 
Schwarzschild as the unique solution (up to gauge).
\end{proof}

\begin{theorem}[Minimizer Identification]
The minimizer of $M_{\text{ADM}}$ over $\mathcal{D}_A$ is Schwarzschild 
initial data with horizon area $A$.
\end{theorem}

\begin{proof}
Combining the lemmas:
\begin{enumerate}
    \item Minimizer is vacuum (removing matter decreases mass)
    \item Minimizer is spherically symmetric (symmetrization reduces mass)
    \item Vacuum + spherically symmetric = Schwarzschild
    \item The horizon area constraint fixes the Schwarzschild mass
\end{enumerate}
\end{proof}

%% ============================================================================
\section{Step 2d: Lower Bound Conclusion}
%% ============================================================================

\begin{theorem}[Lower Bound]
\begin{equation}
    \mathcal{M}(A) \ge \sqrt{\frac{A}{16\pi}}
\end{equation}
\end{theorem}

\begin{proof}
The minimizer is Schwarzschild with horizon area $A$.

For Schwarzschild: $M = \sqrt{A/(16\pi)}$ (by the mass-area relation).

Therefore: $\mathcal{M}(A) = \sqrt{A/(16\pi)}$.
\end{proof}

%% ============================================================================
\section{Gaps and Technical Issues}
%% ============================================================================

\begin{insight}
\textbf{Technical Gaps to Address}

\textbf{Gap 1: Compactness of minimizing sequence}
\begin{itemize}
    \item Need appropriate function spaces for $(g, k)$
    \item Need to handle possible degenerations
    \item Concentration-compactness may apply but needs verification
\end{itemize}

\textbf{Gap 2: Vacuum implies mass decrease}
\begin{itemize}
    \item True for positive mass theorem (vacuum $\Rightarrow M = 0$)
    \item For constrained problem, need to preserve trapped surface
    \item Conformal deformation may change trapped surface properties
\end{itemize}

\textbf{Gap 3: Spherical symmetry from optimization}
\begin{itemize}
    \item Symmetrization inequalities for initial data are not standard
    \item Need careful handling of $(g, k)$ together
    \item The trapped surface constraint complicates symmetrization
\end{itemize}

\textbf{Gap 4: Uniqueness of Schwarzschild}
\begin{itemize}
    \item Need to rule out other vacuum spherically symmetric data
    \item The trapped surface constraint (not just MOTS) adds subtlety
\end{itemize}
\end{insight}

%% ============================================================================
\section{Alternative: Direct Euler-Lagrange}
%% ============================================================================

Instead of the compactness argument, we can try to derive Euler-Lagrange 
equations directly.

\begin{definition}[Lagrangian Formulation]
Minimize:
\begin{equation}
    \mathcal{L}[g, k, \Sigma] = M_{\text{ADM}}[g, k] + \lambda_1(\Area(\Sigma) - A)
    + \lambda_2(\theta^+[\Sigma]) + \text{constraint terms}
\end{equation}

where:
\begin{itemize}
    \item $\lambda_1$ is the area constraint multiplier
    \item $\lambda_2$ enforces the trapped condition $\theta^+ \le 0$
    \item Constraint terms enforce Hamiltonian and momentum constraints
\end{itemize}
\end{definition}

\begin{proposition}[Euler-Lagrange Equations]
The critical points satisfy:
\begin{align}
    \frac{\delta M_{\text{ADM}}}{\delta g} + \lambda_1 \frac{\delta \Area}{\delta g}
    + \lambda_2 \frac{\delta \theta^+}{\delta g} &= 0\\
    \frac{\delta M_{\text{ADM}}}{\delta k} + \lambda_2 \frac{\delta \theta^+}{\delta k} &= 0\\
    \text{plus constraint equations}
\end{align}
\end{proposition}

These are complicated coupled PDEs. The hope is that analyzing them shows 
the only solution is Schwarzschild.

%% ============================================================================
\section{The Key New Ingredient}
%% ============================================================================

\begin{insight}
\textbf{What's New Here}

Traditional Penrose proofs need Area Dominance: a relationship BETWEEN 
surfaces in the SAME initial data.

The variational approach needs: Schwarzschild is minimal AMONG all initial 
data with given trapped surface area.

This is a different type of statement:
\begin{itemize}
    \item Area Dominance: geometric comparison within one $(g, k)$
    \item Variational Penrose: optimization over space of all $(g, k)$
\end{itemize}

The variational statement may be EASIER because it involves comparing 
entire configurations, not delicate surface properties.
\end{insight}

%% ============================================================================
\section{Connection to Positive Mass Theorem}
%% ============================================================================

\begin{proposition}[Positive Mass as Unconstrained Variational]
The positive mass theorem is:
\begin{equation}
    \inf\{M_{\text{ADM}} : \text{AF, DEC}\} = 0
\end{equation}

achieved by flat space (no trapped surfaces).
\end{proposition}

\begin{proposition}[Penrose as Constrained Variational]
The Penrose inequality is:
\begin{equation}
    \inf\{M_{\text{ADM}} : \text{AF, DEC, trapped surface of area } A\} = \sqrt{\frac{A}{16\pi}}
\end{equation}

achieved by Schwarzschild.

\textbf{The trapped surface constraint shifts the minimum from 0 to 
$\sqrt{A/(16\pi)}$.}
\end{proposition}

This perspective suggests Penrose is the "constrained positive mass theorem."

%% ============================================================================
\section{Proof Strategy Summary}
%% ============================================================================

\begin{enumerate}
    \item \textbf{Define the minimization problem:} $\mathcal{M}(A) = \inf M_{\text{ADM}}$ 
          over $\mathcal{D}_A$
    
    \item \textbf{Upper bound:} Schwarzschild achieves $M = \sqrt{A/(16\pi)}$, 
          so $\mathcal{M}(A) \le \sqrt{A/(16\pi)}$
    
    \item \textbf{Lower bound via minimizer characterization:}
    \begin{enumerate}
        \item Show minimizer exists (compactness)
        \item Show minimizer is vacuum (removing matter decreases mass)
        \item Show minimizer is spherically symmetric (symmetrization)
        \item Vacuum + spherical = Schwarzschild
        \item Schwarzschild has $M = \sqrt{A/(16\pi)}$
    \end{enumerate}
    
    \item \textbf{Conclusion:} $\mathcal{M}(A) = \sqrt{A/(16\pi)}$, proving Penrose
\end{enumerate}

%% ============================================================================
\section{Conclusion}
%% ============================================================================

The variational approach reformulates Penrose 1973 as:

\begin{center}
\fbox{\parbox{0.85\textwidth}{
\textbf{Schwarzschild Minimality Theorem}

Among all asymptotically flat initial data satisfying DEC and containing 
a trapped surface of area $\ge A$:
\begin{equation}
    M_{\text{ADM}} \ge \sqrt{\frac{A}{16\pi}}
\end{equation}

with equality iff the data is Schwarzschild with horizon area $A$.
}}
\end{center}

This is equivalent to Penrose 1973 but:
\begin{itemize}
    \item Avoids Area Dominance entirely
    \item Frames the problem as optimization over initial data
    \item Uses comparison across configurations, not within one
    \item Connects naturally to positive mass theorem (unconstrained case)
\end{itemize}

The main technical gaps are:
\begin{enumerate}
    \item Compactness/existence of minimizer
    \item Vacuum character of minimizer
    \item Spherical symmetry of minimizer
\end{enumerate}

Each of these is a well-posed mathematical question that may be 
addressable with modern techniques.

\end{document}
