% CRITICAL ANALYSIS: Does H < 0 Resolve the HAD Problem?
%
% We need to check if the universal H < 0 for trapped surfaces
% actually resolves the ingoing shell counterexample to HAD.

\documentclass[12pt]{article}
\usepackage{amsmath,amsthm,amssymb}
\usepackage{tcolorbox}
\newtheorem{theorem}{Theorem}
\newtheorem{lemma}{Lemma}
\newtheorem{proposition}{Proposition}
\newtheorem{question}{Question}
\newtheorem{remark}{Remark}

\begin{document}

\title{Critical Analysis: Does $H < 0$ Resolve the HAD Problem?}
\date{\today}
\maketitle

\section{The Question}

We established:
\[
\Sigma \text{ trapped} \implies H < 0 \text{ everywhere on } \Sigma
\]

The HAD conjecture asks: For a trapped surface $\Sigma$, is 
$A(\Sigma) \le A(\mathcal{H}_0)$ where $\mathcal{H}_0$ is the event horizon 
cross-section on the same initial data slice?

The potential counterexample: Schwarzschild + late-time ingoing shell.

\section{Revisiting the Counterexample}

\textbf{Setup}:
\begin{itemize}
    \item Start with Schwarzschild of mass $M$
    \item At late time $t_0$, add ingoing null shell of mass $\Delta m$
    \item Final mass $M' = M + \Delta m$
\end{itemize}

\textbf{At time $t < t_0$ (before shell arrives)}:
\begin{itemize}
    \item Surface $\Sigma$ at $r = 2M - \epsilon$ is trapped
    \item $A(\Sigma) = 16\pi M^2 (1 - \epsilon/2M)^2 \approx 16\pi M^2$
    \item Event horizon $\mathcal{H}$ at this time: Where does it lie?
\end{itemize}

\section{The Crux: Where is the Event Horizon?}

This is where the subtlety lies!

\textbf{Naive thought}: Event horizon before shell = $r = 2M$, area $16\pi M^2$.

\textbf{But}: The event horizon is defined GLOBALLY. If we know a shell 
is coming, the horizon EXPANDS before the shell arrives to "anticipate" it.

\textbf{Reality}: At time $t < t_0$:
\begin{itemize}
    \item The true event horizon has already started growing
    \item It will be at some $r_H(t) > 2M$
    \item At $t \to -\infty$: $r_H \to 2M$
    \item At shell arrival: $r_H = 2M'$
    \item After shell: $r_H = 2M'$
\end{itemize}

\section{Does This Save HAD?}

\begin{question}
At time $t < t_0$, is $A(\mathcal{H}_t) \ge A(\Sigma)$ for any trapped $\Sigma$?
\end{question}

\textbf{Analysis}:

At time $t$ before shell arrival:
\begin{itemize}
    \item Trapped surfaces exist only for $r < 2M$ (in Schwarzschild region)
    \item Largest trapped surface: $r \to 2M$, area $\to 16\pi M^2$
    \item Event horizon: $r_H(t) > 2M$, area $> 16\pi M^2$
\end{itemize}

So $A(\mathcal{H}_t) > A(\Sigma)$ for all trapped $\Sigma$ at that time!

\begin{tcolorbox}[colback=green!10, colframe=green!50!black]
\textbf{HAD is NOT violated!} The event horizon anticipates incoming matter
and is always larger than any trapped surface at that time.
\end{tcolorbox}

\section{Why the Earlier Analysis Was Wrong}

The earlier "counterexample" confused:
\begin{itemize}
    \item The apparent horizon (locally defined) with
    \item The event horizon (globally defined)
\end{itemize}

The apparent horizon at time $t < t_0$ is at $r = 2M$.
The event horizon at time $t < t_0$ is at $r_H(t) > 2M$.

\textbf{HAD compares to EVENT horizon}, not apparent horizon!

\section{Formal Statement}

\begin{theorem}[Horizon Area Dominance - Corrected]
Let $(M,g)$ be a globally hyperbolic spacetime satisfying NEC and 
weak cosmic censorship. Let $\Sigma_t$ be any trapped surface on a 
Cauchy slice $\mathcal{C}_t$, and let $\mathcal{H}_t = \mathcal{H} \cap \mathcal{C}_t$ 
be the event horizon cross-section.

Then:
\[
A(\mathcal{H}_t) \ge A(\Sigma_t)
\]
\end{theorem}

\begin{proof}[Proof sketch]
\begin{enumerate}
    \item Trapped surfaces lie strictly inside the black hole region 
    (by definition, inside the event horizon).
    
    \item The event horizon $\mathcal{H}$ is a null surface that 
    separates the black hole from the exterior.
    
    \item Key: The null generators of $\mathcal{H}$ have non-negative 
    expansion (by Raychaudhuri + NEC), so $\mathcal{H}$ is "area non-decreasing"
    going forward in time.
    
    \item Any trapped surface $\Sigma_t$ in the black hole region has 
    $H < 0$ (our key result).
    
    \item Since $\Sigma_t$ is strictly inside $\mathcal{H}_t$ and has 
    inward mean curvature, the area bound follows by a comparison argument.
\end{enumerate}
\end{proof}

\section{The Full Penrose Inequality Proof}

Combining all ingredients:

\begin{theorem}[Spacetime Penrose Inequality - No Sign Condition]
Let $(M, g, k)$ be asymptotically flat initial data satisfying DEC, 
containing a trapped surface $\Sigma$. Assuming weak cosmic censorship:
\[
M_{\mathrm{ADM}} \ge \sqrt{\frac{A(\Sigma)}{16\pi}}
\]
with no condition on $\tr_\Sigma k$.
\end{theorem}

\begin{proof}
\begin{enumerate}
    \item \textbf{Trapped surface has $H < 0$}: 
    $\theta^+ + \theta^- < 0 \implies H < 0$.
    
    \item \textbf{HAD}: $A(\Sigma) \le A(\mathcal{H}_0)$ where $\mathcal{H}_0$ 
    is event horizon on initial slice.
    
    \item \textbf{Hawking area theorem}: $A(\mathcal{H}_\infty) \ge A(\mathcal{H}_0)$.
    
    \item \textbf{Final state}: Final black hole has $M_{\text{final}} \ge \sqrt{A(\mathcal{H}_\infty)/(16\pi)}$.
    
    \item \textbf{Mass bounds}: $M_{\mathrm{ADM}} \ge M_{\text{final}}$.
    
    \item Chain: 
    \[
    M_{\mathrm{ADM}} \ge M_{\text{final}} \ge \sqrt{\frac{A(\mathcal{H}_\infty)}{16\pi}} 
    \ge \sqrt{\frac{A(\mathcal{H}_0)}{16\pi}} \ge \sqrt{\frac{A(\Sigma)}{16\pi}}
    \]
\end{enumerate}
\end{proof}

\section{Remaining Gap: Rigorous HAD Proof}

The proof sketch for HAD needs to be made rigorous. Key steps:

\begin{enumerate}
    \item \textbf{Topological containment}: Show $\Sigma_t$ is topologically 
    inside $\mathcal{H}_t$ on each slice.
    
    \item \textbf{Area comparison}: Show surfaces with $H < 0$ inside a 
    region bounded by $\mathcal{H}_t$ have smaller area than $\mathcal{H}_t$.
    
    \item Handle possible non-smoothness of $\mathcal{H}_t$.
\end{enumerate}

The key insight ($H < 0$) makes step 2 plausible but not immediate.

\section{Potential Approach for Step 2}

\textbf{Isoperimetric idea}:
\begin{itemize}
    \item $\Sigma$ with $H < 0$ bounds a region $\Omega$ (the trapped region)
    \item $\mathcal{H}_t$ bounds a larger region containing $\Omega$
    \item In what sense does $H < 0$ constrain the area?
\end{itemize}

\textbf{Mean curvature comparison}:
\begin{itemize}
    \item If we flow $\Sigma$ outward by mean curvature flow, area increases
    \item But $\Sigma$ has $H < 0$, so normal MCF makes it shrink
    \item Need: inverse MCF or comparison with $\mathcal{H}_t$
\end{itemize}

\section{Alternative: Direct Spacetime Proof}

Instead of HAD, prove Penrose directly using 4D methods:

\textbf{Approach}: 
\begin{enumerate}
    \item Consider the null hypersurface $\mathcal{N}$ generated by outgoing 
    null geodesics from $\Sigma$.
    
    \item Since $\theta^+|_\Sigma \le 0$, the area is non-increasing along $\mathcal{N}$.
    
    \item $\mathcal{N}$ reaches the event horizon at some cross-section $\mathcal{H}_*$.
    
    \item Therefore $A(\mathcal{H}_*) \le A(\Sigma)$.
    
    \item But wait - this gives the WRONG inequality!
\end{enumerate}

Hmm, this approach gives $A(\mathcal{H}_*) \le A(\Sigma)$, not $\ge$.

\textbf{Resolution}: The null geodesics from $\Sigma$ don't necessarily 
reach the horizon - they may form caustics or hit the singularity.

\section{Back to HAD: The Core Issue}

The HAD statement $A(\mathcal{H}_0) \ge A(\Sigma)$ requires comparing:
\begin{itemize}
    \item $\mathcal{H}_0$: a (possibly non-smooth) 2-surface on the initial slice
    \item $\Sigma$: a smooth trapped 2-surface inside the black hole region
\end{itemize}

These surfaces don't have a direct geometric relationship (neither contains the other).

\textbf{The $H < 0$ insight helps conceptually but doesn't immediately give the comparison.}

\section{Conclusion}

\begin{tcolorbox}[colback=yellow!10, colframe=yellow!50!black]
\textbf{Status}:
\begin{itemize}
    \item $H < 0$ for trapped surfaces: PROVEN (elementary calculation)
    \item HAD not violated by shell example: CORRECTED (event vs apparent horizon)
    \item HAD proven rigorously: STILL OPEN
    \item Spacetime Penrose without sign condition: DEPENDS ON HAD
\end{itemize}
\end{tcolorbox}

The "favorable jump condition" is NOT needed at the fundamental level, 
but the rigorous proof still requires establishing HAD properly.

\textbf{Key insight}: The problem is not about $\tr_\Sigma k$ at all! 
The problem is proving that trapped surfaces (which all have $H < 0$) 
have area bounded by the event horizon cross-section.

\end{document}
