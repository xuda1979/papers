\documentclass[11pt]{article}
\usepackage{amsmath,amssymb,amsthm}
\usepackage[margin=1in]{geometry}
\usepackage{tcolorbox}
\usepackage{xcolor}
\usepackage{hyperref}

\newtheorem{theorem}{Theorem}[section]
\newtheorem{lemma}[theorem]{Lemma}
\newtheorem{proposition}[theorem]{Proposition}
\newtheorem{corollary}[theorem]{Corollary}
\newtheorem{definition}[theorem]{Definition}
\newtheorem{conjecture}[theorem]{Conjecture}
\newtheorem{remark}[theorem]{Remark}

\newtcolorbox{keybox}[1][]{colback=blue!5!white,colframe=blue!75!black,fonttitle=\bfseries,title={#1}}
\newtcolorbox{gapbox}[1][]{colback=red!5!white,colframe=red!75!black,fonttitle=\bfseries,title={#1}}
\newtcolorbox{proofbox}[1][]{colback=green!5!white,colframe=green!65!black,fonttitle=\bfseries,title={#1}}
\newtcolorbox{ideabox}[1][]{colback=yellow!10!white,colframe=orange!75!black,fonttitle=\bfseries,title={#1}}

\title{\textbf{The Penrose Inequality via Trapping Product}\\
\large Exploiting $\theta^+\theta^- > 0$}
\author{Research Notes}
\date{December 2025}

\begin{document}
\maketitle

\begin{abstract}
We develop a novel approach to the Penrose inequality based on the \textbf{trapping product} $\mathcal{P} = \theta^+\theta^-$, which is strictly positive for trapped surfaces and vanishes exactly on MOTS. Unlike $\mathrm{tr}_\Sigma k = \frac{1}{2}(\theta^+ - \theta^-)$, this quantity has \textbf{definite sign}. We construct a modified Jang equation using this product and analyze its properties.
\end{abstract}

%==============================================================================
\section{The Trapping Product}
%==============================================================================

\begin{keybox}[Key Observation]
For a trapped surface $\Sigma$ with $\theta^+ \leq 0$ and $\theta^- < 0$:
\begin{align}
    \mathcal{P} &:= \theta^+\theta^- \geq 0 \quad (\text{always non-negative!})\\
    H &= \frac{1}{2}(\theta^+ + \theta^-) < 0 \quad (\text{definite sign})\\
    \mathrm{tr}_\Sigma k &= \frac{1}{2}(\theta^+ - \theta^-) \quad (\text{undetermined sign - THE PROBLEM})
\end{align}

Crucially:
\begin{itemize}
    \item $\mathcal{P} > 0$ for strictly trapped surfaces ($\theta^+ < 0$)
    \item $\mathcal{P} = 0$ iff $\theta^+ = 0$ (MOTS case)
    \item $\mathcal{P}$ is a \textbf{symmetric} function of $\theta^+$ and $\theta^-$
\end{itemize}
\end{keybox}

\subsection{Algebraic Identities}

\begin{lemma}[Product-Sum Relations]
For any surface with null expansions $\theta^\pm$:
\begin{align}
    \mathcal{P} &= \theta^+\theta^- = H^2 - (\mathrm{tr}_\Sigma k)^2 \label{eq:product-formula}\\
    \theta^\pm &= H \pm \mathrm{tr}_\Sigma k
\end{align}
\end{lemma}

\begin{proof}
Direct computation:
\begin{align}
    H^2 - (\mathrm{tr}_\Sigma k)^2 &= \frac{1}{4}(\theta^+ + \theta^-)^2 - \frac{1}{4}(\theta^+ - \theta^-)^2 \\
    &= \frac{1}{4}\left[(\theta^+)^2 + 2\theta^+\theta^- + (\theta^-)^2 - (\theta^+)^2 + 2\theta^+\theta^- - (\theta^-)^2\right]\\
    &= \theta^+\theta^- = \mathcal{P}
\end{align}
\end{proof}

\begin{corollary}[Sign of $\mathrm{tr}_\Sigma k$]
For trapped surfaces:
\begin{equation}
    (\mathrm{tr}_\Sigma k)^2 = H^2 - \mathcal{P}
\end{equation}
Since $H^2 > 0$ and $\mathcal{P} \geq 0$, we have:
\begin{equation}
    |\mathrm{tr}_\Sigma k| \leq |H|
\end{equation}
with equality iff $\mathcal{P} = 0$ (MOTS).
\end{corollary}

%==============================================================================
\section{The Product-Modified Jang Equation}
%==============================================================================

\subsection{Motivation}

The standard Jang equation:
\begin{equation}
    \mathcal{J}(f) = H_\Gamma + \mathrm{tr}_\Gamma k = \theta^+_\Gamma = 0
\end{equation}
produces scalar curvature:
\begin{equation}
    R_{\bar{g}} = R^{\mathrm{reg}} + 2(\mathrm{tr}_\Sigma k)\delta_\Sigma
\end{equation}

The problem: $\mathrm{tr}_\Sigma k$ can be negative.

\textbf{Idea:} Construct an equation using $\mathcal{P}$ instead.

\subsection{The Product-Modified Equation}

\begin{definition}[Product-Jang Equation]
Define the \textbf{product-Jang equation}:
\begin{equation}
    \boxed{\mathcal{J}_{\mathcal{P}}(f) := \theta^+_\Gamma \cdot \theta^-_\Gamma = \mathcal{P}_\Gamma = 0}
\end{equation}
where $\theta^\pm_\Gamma$ are the null expansions of the graph $\Gamma_f$.
\end{definition}

\begin{lemma}[Equivalent Formulations]
The product-Jang equation is equivalent to:
\begin{align}
    \mathcal{P}_\Gamma &= H_\Gamma^2 - (\mathrm{tr}_\Gamma k)^2 = 0\\
    |H_\Gamma| &= |\mathrm{tr}_\Gamma k|\\
    \theta^+_\Gamma &= 0 \text{ OR } \theta^-_\Gamma = 0
\end{align}
\end{lemma}

\begin{ideabox}[The Key Insight]
The equation $\mathcal{P}_\Gamma = 0$ has \textbf{two branches}:
\begin{enumerate}
    \item $\theta^+_\Gamma = 0$: Standard MOTS (Jang equation)
    \item $\theta^-_\Gamma = 0$: Past MOTS (dual Jang equation)
\end{enumerate}

For trapped surfaces, both branches are accessible, but the solution naturally chooses the branch with better regularity at each point.
\end{ideabox}

%==============================================================================
\section{Existence Theory for Product-Jang}
%==============================================================================

\subsection{The Variational Formulation}

The product-Jang equation can be written as a degenerate elliptic equation:
\begin{equation}
    (H_\Gamma - \mathrm{tr}_\Gamma k)(H_\Gamma + \mathrm{tr}_\Gamma k) = 0
\end{equation}

This is a \textbf{fully nonlinear} equation of the form:
\begin{equation}
    F[D^2f, Df, f] = 0
\end{equation}
where:
\begin{equation}
    F = \left(\frac{\text{div}(\nabla f/W)}{1 + |\nabla f|^2/W^2} + \frac{k(\nabla f, \nabla f)}{W} + H - \mathrm{tr} k\right) \times (\text{same with } k \to -k)
\end{equation}

\subsection{Viscosity Solution Approach}

\begin{definition}[Viscosity Solution]
A function $f \in C(\bar{M})$ is a \textbf{viscosity solution} of $\mathcal{P}_\Gamma = 0$ if:
\begin{enumerate}
    \item \textbf{Subsolution:} For any smooth $\phi$ such that $f - \phi$ has a local max at $x_0$:
    \begin{equation}
        \min(\theta^+_{\Gamma_\phi}(x_0), \theta^-_{\Gamma_\phi}(x_0)) \leq 0
    \end{equation}
    \item \textbf{Supersolution:} For any smooth $\phi$ such that $f - \phi$ has a local min at $x_0$:
    \begin{equation}
        \max(\theta^+_{\Gamma_\phi}(x_0), \theta^-_{\Gamma_\phi}(x_0)) \geq 0
    \end{equation}
\end{enumerate}
\end{definition}

\begin{theorem}[Existence - Conditional]\label{thm:product-existence}
Let $(M^3, g, k)$ be asymptotically flat with trapped surface $\Sigma_0$. Assume:
\begin{itemize}
    \item $\Sigma_0$ is strictly trapped: $\theta^+ < 0$, $\theta^- < 0$
    \item The DEC holds: $\mu \geq |J|$
\end{itemize}
Then there exists a viscosity solution $f$ to $\mathcal{P}_\Gamma = 0$ with:
\begin{enumerate}
    \item $f \to +\infty$ as we approach $\Sigma_0$ from the exterior
    \item $f \to 0$ at spatial infinity
\end{enumerate}
\end{theorem}

\begin{gapbox}[GAP: Existence Proof]
The existence theorem requires developing viscosity solution theory for the product equation. Key challenges:
\begin{itemize}
    \item The equation is \textbf{not} uniformly elliptic
    \item Branch switching between $\theta^+ = 0$ and $\theta^- = 0$
    \item Regularity at branch points
\end{itemize}

\textbf{Potential approach:} Perron's method using sub/supersolutions from the individual Jang equations.
\end{gapbox}

%==============================================================================
\section{Geometric Analysis of Product-Jang}
%==============================================================================

\subsection{Scalar Curvature of the Product-Jang Metric}

\begin{theorem}[Scalar Curvature Formula]\label{thm:scalar-product}
Let $f$ be a smooth solution to $\mathcal{P}_\Gamma = 0$ away from $\Sigma_0$. The induced metric $\bar{g}$ on the graph satisfies:
\begin{equation}
    R_{\bar{g}} \geq 2(\mu - |J|) - \frac{(\mathcal{P}_\Gamma)'}{W} \geq 0
\end{equation}
where the inequality holds because $\mathcal{P}_\Gamma = 0$ on the solution.
\end{theorem}

\begin{proofbox}[Key Calculation]
At any point where $\theta^+_\Gamma = 0$ (solution on standard branch):
\begin{align}
    R_{\bar{g}} &= 2\mu + \mathrm{tr}(k)^2 - |k|^2 - 2\langle X, J \rangle \\
    &\quad + 2|A - K|^2 + 2(\theta^+_\Gamma)(\theta^-_\Gamma) \\
    &= 2\mu + \mathrm{tr}(k)^2 - |k|^2 - 2\langle X, J \rangle + 2|A - K|^2
\end{align}
The last term $2\theta^+_\Gamma\theta^-_\Gamma = 0$ because we're on a branch.

By DEC and Cauchy-Schwarz:
\begin{equation}
    R_{\bar{g}} \geq 2(\mu - |J|) \geq 0
\end{equation}

The same holds on the $\theta^-_\Gamma = 0$ branch by symmetry.
\end{proofbox}

\subsection{Boundary Behavior}

\begin{lemma}[Blow-up at Trapped Surface]
As we approach $\Sigma_0$ from the exterior, the solution $f \to +\infty$ with rate:
\begin{equation}
    f(s, y) \sim C(y) \ln(s^{-1})
\end{equation}
where $s = \text{dist}(\cdot, \Sigma_0)$ and:
\begin{equation}
    C(y) = \begin{cases}
        \frac{|\theta^-|}{2} & \text{if solution is on } \theta^+ = 0 \text{ branch}\\
        \frac{|\theta^+|}{2} & \text{if solution is on } \theta^- = 0 \text{ branch}
    \end{cases}
\end{equation}
\end{lemma}

\begin{ideabox}[The Advantage]
The product equation allows the solution to \textbf{choose the favorable branch} at each point:
\begin{itemize}
    \item Where $|\theta^+| < |\theta^-|$: Use $\theta^+ = 0$ branch (faster blow-up)
    \item Where $|\theta^-| < |\theta^+|$: Use $\theta^- = 0$ branch (faster blow-up)
\end{itemize}
This \textbf{optimizes} the blow-up behavior over the surface.
\end{ideabox}

%==============================================================================
\section{The Interface Contribution}
%==============================================================================

\subsection{Mean Curvature Jump}

The key question: what is the interface term $[H]$ in the scalar curvature?

\begin{proposition}[Interface Analysis]
At the blow-up surface $\Sigma_0$, the mean curvature jump depends on the branch:
\begin{align}
    [H]_{\theta^+ = 0} &= \mathrm{tr}_\Sigma k\\
    [H]_{\theta^- = 0} &= -\mathrm{tr}_\Sigma k
\end{align}
\end{proposition}

\begin{gapbox}[THE FUNDAMENTAL ISSUE]
Even with the product equation, we cannot escape the sign of $\mathrm{tr}_\Sigma k$!

\textbf{If} the solution could smoothly transition between branches at each point of $\Sigma_0$, choosing:
\begin{equation}
    \text{Branch} = \begin{cases}
        \theta^+ = 0 & \text{where } \mathrm{tr}_\Sigma k \geq 0\\
        \theta^- = 0 & \text{where } \mathrm{tr}_\Sigma k < 0
    \end{cases}
\end{equation}
then:
\begin{equation}
    [H]_{\text{optimal}} = |\mathrm{tr}_\Sigma k| \geq 0
\end{equation}

\textbf{But:} The solution cannot freely switch branches - it must satisfy a global consistency condition.
\end{gapbox}

%==============================================================================
\section{Attempt at Resolution: The $\sqrt{\mathcal{P}}$ Method}
%==============================================================================

\subsection{Using the Square Root}

\begin{definition}[Root-Trapping Equation]
Define:
\begin{equation}
    \sqrt{\mathcal{P}_\Gamma} = \sqrt{\theta^+_\Gamma \cdot \theta^-_\Gamma}
\end{equation}
and consider the equation:
\begin{equation}
    \sqrt{\mathcal{P}_\Gamma} = \epsilon \to 0
\end{equation}
as a regularization that approaches MOTS.
\end{definition}

This is well-defined for strictly trapped surfaces where $\mathcal{P} > 0$.

\begin{lemma}[Gradient of $\sqrt{\mathcal{P}}$]
For the trapping product:
\begin{equation}
    \nabla \sqrt{\mathcal{P}} = \frac{\theta^- \nabla\theta^+ + \theta^+ \nabla\theta^-}{2\sqrt{\theta^+\theta^-}}
\end{equation}
\end{lemma}

\subsection{The $\sqrt{\mathcal{P}}$-Flow}

\begin{definition}[Trapping Flow]
Define the flow:
\begin{equation}
    \frac{\partial \Sigma}{\partial t} = -\sqrt{\mathcal{P}} \cdot \nu
\end{equation}
where $\nu$ is the outward normal.
\end{definition}

\begin{proposition}[Flow Properties]
The trapping flow satisfies:
\begin{enumerate}
    \item Stationary points are MOTS ($\theta^+ = 0$) or past-MOTS ($\theta^- = 0$)
    \item For trapped surfaces, $\sqrt{\mathcal{P}} > 0$, so the flow moves inward
    \item The flow is \textbf{well-defined} (no sign ambiguity)
\end{enumerate}
\end{proposition}

\begin{theorem}[Area Evolution - Conditional]\label{thm:area-evolution}
Under the trapping flow:
\begin{align}
    \frac{d A}{dt} &= -\int_\Sigma \sqrt{\mathcal{P}} \cdot H \, dA\\
    &= -\int_\Sigma \sqrt{\theta^+\theta^-} \cdot \frac{\theta^+ + \theta^-}{2} \, dA
\end{align}
For trapped surfaces ($\theta^+, \theta^- \leq 0$):
\begin{equation}
    \frac{dA}{dt} = -\int_\Sigma \sqrt{|\theta^+||\theta^-|} \cdot \frac{|\theta^+| + |\theta^-|}{2} \, dA \leq 0
\end{equation}
\textbf{Area is decreasing!}
\end{theorem}

%==============================================================================
\section{Connection to Penrose Inequality}
%==============================================================================

\subsection{The Strategy}

\begin{enumerate}
    \item Start with trapped surface $\Sigma_0$
    \item Flow by $\sqrt{\mathcal{P}}$ until reaching a MOTS $\Sigma^*$
    \item Apply known MOTS Penrose inequality to $\Sigma^*$
    \item Relate $A(\Sigma^*)$ to $A(\Sigma_0)$
\end{enumerate}

\begin{theorem}[Area Comparison - Conditional]\label{thm:area-comparison}
If the $\sqrt{\mathcal{P}}$-flow exists globally and converges to a smooth MOTS $\Sigma^*$:
\begin{equation}
    A(\Sigma^*) \leq A(\Sigma_0)
\end{equation}
\end{theorem}

\begin{proof}[Proof Sketch]
By Theorem~\ref{thm:area-evolution}, area is monotonically decreasing along the flow. At the limit $\Sigma^*$, we have $\sqrt{\mathcal{P}} = 0$, so $\theta^+ = 0$ (MOTS).
\end{proof}

\begin{gapbox}[Critical Gaps]

\textbf{GAP A: Flow Existence.}
Long-time existence of the $\sqrt{\mathcal{P}}$-flow is not established. Key issues:
\begin{itemize}
    \item Short-time existence (parabolic theory)
    \item Singularity formation
    \item Convergence to smooth limit
\end{itemize}

\textbf{GAP B: Wrong Direction!}
The area comparison goes the \textbf{wrong way}:
\begin{equation}
    A(\Sigma^*) \leq A(\Sigma_0)
\end{equation}
But for Penrose, we need:
\begin{equation}
    M_{\mathrm{ADM}} \geq \sqrt{\frac{A(\Sigma_0)}{16\pi}} \geq \sqrt{\frac{A(\Sigma^*)}{16\pi}}
\end{equation}
The area decreases, so knowing $M \geq \sqrt{A(\Sigma^*)/(16\pi)}$ does NOT imply $M \geq \sqrt{A(\Sigma_0)/(16\pi)}$.
\end{gapbox}

%==============================================================================
\section{Honest Assessment}
%==============================================================================

\begin{gapbox}[THE FUNDAMENTAL OBSTRUCTION]

After exploring multiple approaches using the trapping product, we find:

\textbf{The Sign Problem Cannot Be Avoided.}

\begin{enumerate}
    \item \textbf{Product-Jang:} The interface term $[H]$ still involves $\mathrm{tr}_\Sigma k$ with undetermined sign.

    \item \textbf{Branch Switching:} Even if we could switch branches optimally, this requires a discontinuous solution, violating regularity.

    \item \textbf{$\sqrt{\mathcal{P}}$-Flow:} Area decreases to MOTS, giving the wrong direction for Penrose.

    \item \textbf{Symmetric Quantities:} While $\mathcal{P} = \theta^+\theta^- > 0$ has definite sign, the \textbf{interface contribution} in any geometric construction still depends on $\mathrm{tr}_\Sigma k$.
\end{enumerate}

\textbf{Root Cause:}
The Penrose inequality for general trapped surfaces requires proving:
\begin{equation}
    \text{(ADM mass)} \geq \text{(function of area)}
\end{equation}
Any proof via positive mass theorem requires scalar curvature $\geq 0$, which for Jang-type constructions requires:
\begin{equation}
    R = R^{\mathrm{reg}} + 2(\text{interface term})\delta_\Sigma \geq 0
\end{equation}

The interface term is intrinsically related to $\mathrm{tr}_\Sigma k$, whose sign we cannot control for general trapped surfaces.
\end{gapbox}

%==============================================================================
\section{What Would Be Needed}
%==============================================================================

To prove the Penrose inequality for arbitrary trapped surfaces, one would need:

\begin{enumerate}
    \item \textbf{A geometric construction that bypasses the sign issue:}
    \begin{itemize}
        \item Use a quantity that depends only on $|\mathrm{tr}_\Sigma k|$, not its sign
        \item Relate this to mass via a new positive mass argument
    \end{itemize}

    \item \textbf{A flow that increases area to MOTS:}
    \begin{itemize}
        \item Current flows (IMCF, $\sqrt{\mathcal{P}}$) all decrease or fix area
        \item Need a flow moving \textbf{outward} from trapped to MOTS
    \end{itemize}

    \item \textbf{Cosmic censorship:}
    \begin{itemize}
        \item Assume trapped surface evolves to event horizon
        \item Event horizon has area $\geq$ trapped surface (area theorem)
        \item Apply MOTS inequality at late times
    \end{itemize}

    \item \textbf{Spacetime approach:}
    \begin{itemize}
        \item Prove directly using null hypersurface techniques
        \item Requires Lorentzian positive mass theorem (not available)
    \end{itemize}
\end{enumerate}

%==============================================================================
\section{Conclusion}
%==============================================================================

The trapping product $\mathcal{P} = \theta^+\theta^- > 0$ provides a symmetric, sign-definite quantity for trapped surfaces. However, exploiting this for the Penrose inequality faces fundamental obstacles:

\begin{itemize}
    \item The interface contribution in Jang-type constructions unavoidably involves $\mathrm{tr}_\Sigma k$
    \item Natural flows using $\mathcal{P}$ decrease area, giving the wrong direction
    \item The obstruction appears to be \textbf{structural}, not technical
\end{itemize}

\textbf{The 1973 Penrose conjecture for general trapped surfaces remains genuinely open}, and likely requires either cosmic censorship assumptions or fundamentally new mathematical tools.

\end{document}
