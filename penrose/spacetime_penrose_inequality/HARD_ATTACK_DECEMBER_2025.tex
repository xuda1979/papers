\documentclass[11pt]{article}
\usepackage{amsmath,amsthm,amssymb,mathrsfs}
\usepackage[margin=1in]{geometry}
\usepackage{xcolor}

\newtheorem{theorem}{Theorem}[section]
\newtheorem{lemma}[theorem]{Lemma}
\newtheorem{proposition}[theorem]{Proposition}
\newtheorem{corollary}[theorem]{Corollary}
\newtheorem{definition}[theorem]{Definition}
\theoremstyle{remark}
\newtheorem{remark}[theorem]{Remark}
\newtheorem*{claim}{Claim}
\newtheorem*{gap}{GAP}

\newcommand{\R}{\mathbb{R}}
\newcommand{\Sig}{\Sigma}
\newcommand{\tp}{\theta^+}
\newcommand{\tm}{\theta^-}
\newcommand{\Madm}{M_{\mathrm{ADM}}}
\newcommand{\Mirr}{M_{\mathrm{irr}}}

\title{\textbf{Hard Attack on Penrose 1973}\\
\large Weak IMCF + Viscosity Methods}
\author{Working Document}
\date{December 2025}

\begin{document}
\maketitle

\begin{abstract}
We attempt a rigorous proof of Penrose 1973 using weak inverse mean curvature flow 
with viscosity solution techniques. We identify precisely where each approach 
succeeds or fails, with explicit calculations.
\end{abstract}

\tableofcontents

%% ============================================================================
\section{Setup and Strategy}
%% ============================================================================

\textbf{Goal:} Prove $\Madm \geq \sqrt{A(\Sig_0)/(16\pi)}$ for any trapped surface $\Sig_0$.

\textbf{Given:}
\begin{itemize}
\item $(M^3, g, k)$ asymptotically flat initial data, DEC holds
\item $\Sig_0$ closed trapped surface: $\tp = H + \tr_\Sig k \leq 0$, $\tm = H - \tr_\Sig k < 0$
\item Hence $H = \frac{1}{2}(\tp + \tm) < 0$ on $\Sig_0$
\end{itemize}

\textbf{The Obstruction:} Standard IMCF $\partial_t \Sig = \frac{\nu}{H}$ undefined when $H < 0$.

\textbf{Strategy:} Construct a \emph{weak solution} that jumps over the $H < 0$ region.

%% ============================================================================
\section{Attempt 1: Elliptic Regularization}
\label{sec:elliptic}
%% ============================================================================

\subsection{The Regularized Problem}

Following Huisken-Ilmanen, we seek $u: M \setminus \Sig_0 \to \R$ with level sets 
$\Sig_t = \{u = t\}$ satisfying a regularized equation.

\textbf{Standard IMCF equation:} $\div\left(\frac{\nabla u}{|\nabla u|}\right) = |\nabla u|$

This is the level set formulation: if $\Sig_t = \{u = t\}$, then 
$H = |\nabla u| \div(\nabla u/|\nabla u|)$ and velocity $= 1/H$, giving the equation.

\textbf{Problem:} When $H < 0$, we need $|\nabla u| < 0$ which is impossible.

\subsection{Regularization with Sign Flip}

Define the \emph{signed regularization}:
\begin{equation}
\div\left(\frac{\nabla u}{\sqrt{|\nabla u|^2 + \epsilon^2}}\right) = \sqrt{|\nabla u|^2 + \epsilon^2}
\label{eq:regularized}
\end{equation}

\begin{lemma}[Existence for Regularized Problem]
For each $\epsilon > 0$, there exists a unique solution $u_\epsilon \in C^{2,\alpha}(M \setminus \Sig_0)$ 
to \eqref{eq:regularized} with $u_\epsilon|_{\Sig_0} = 0$ and $u_\epsilon \to \infty$ at spatial infinity.
\end{lemma}

\begin{proof}
This is a quasilinear elliptic equation with uniformly elliptic principal part 
(the $\epsilon^2$ removes degeneracy). Standard theory (Gilbarg-Trudinger, Theorem 13.8) 
gives existence and regularity.

\textbf{Ellipticity check:} The linearization at $u$ has principal symbol
\[
a^{ij}(\nabla u) = \frac{\delta^{ij}}{\sqrt{|\nabla u|^2 + \epsilon^2}} 
- \frac{\partial_i u \partial_j u}{(|\nabla u|^2 + \epsilon^2)^{3/2}}
\]
Eigenvalues: $\lambda_1 = (|\nabla u|^2 + \epsilon^2)^{-3/2} \epsilon^2 > 0$ (in $\nabla u$ direction), 
$\lambda_2 = \lambda_3 = (|\nabla u|^2 + \epsilon^2)^{-1/2}$ (perpendicular).
Uniform ellipticity: $\lambda_{\min}/\lambda_{\max} \geq \epsilon^2/(|\nabla u|^2 + \epsilon^2) > 0$.
\end{proof}

\subsection{The $\epsilon \to 0$ Limit}

\begin{proposition}[Compactness]
There exists a subsequence $\epsilon_j \to 0$ and $u \in W^{1,1}_{\mathrm{loc}}(M \setminus \Sig_0)$ 
such that $u_{\epsilon_j} \to u$ in $L^1_{\mathrm{loc}}$.
\end{proposition}

\begin{proof}
We need uniform bounds on $u_\epsilon$ and $\nabla u_\epsilon$.

\textbf{Step 1: Maximum principle bound on $u_\epsilon$.}
The maximum principle gives $0 \leq u_\epsilon \leq C \cdot \mathrm{dist}(\cdot, \Sig_0)$ near $\Sig_0$.

At infinity, the equation becomes approximately $\Delta u \approx |\nabla u|$, giving 
$u_\epsilon \sim \log r$ growth.

\textbf{Step 2: Gradient bound.}
Multiply \eqref{eq:regularized} by a test function $\phi^2$ and integrate:
\[
\int_M \phi^2 (|\nabla u|^2 + \epsilon^2) \, dV = \int_M \phi^2 \frac{\nabla u}{\sqrt{|\nabla u|^2 + \epsilon^2}} \cdot \nabla u \, dV + \text{boundary}
\]

This gives $\int |\nabla u_\epsilon|^2 \leq C$ uniformly in $\epsilon$.

\textbf{Step 3: Compactness.}
By Rellich-Kondrachov, $W^{1,2} \hookrightarrow L^2$ compactly, giving the result.
\end{proof}

\subsection{Critical Analysis: What is the Limit?}

\begin{gap}[Limit Behavior Near $H < 0$ Region]
The limit $u$ satisfies the IMCF equation weakly where $|\nabla u| > 0$. 
But what happens at points where $H < 0$?

\textbf{Possibilities:}
\begin{enumerate}
\item $|\nabla u| = 0$ on a set of positive measure (jump)
\item $u$ is constant on the $H < 0$ region
\item The limit develops a discontinuity
\end{enumerate}

\textbf{The Problem:} The regularized equation \eqref{eq:regularized} has RHS $\geq \epsilon > 0$, 
so level sets always move outward. But the \emph{rate} depends on the geometry.

In the $H < 0$ region, the regularized flow moves slowly (velocity $\sim 1/\epsilon$), 
and as $\epsilon \to 0$, the time to cross this region $\to \infty$.
\end{gap}

\textbf{Conclusion:} Elliptic regularization does NOT naturally produce the jump we need.

%% ============================================================================
\section{Attempt 2: Parabolic IMCF with Jump Prescription}
\label{sec:parabolic}
%% ============================================================================

\subsection{Modified Flow}

Consider the flow:
\begin{equation}
\frac{\partial \Sig_t}{\partial t} = \frac{\nu}{\max(H, \delta)}
\label{eq:modified-imcf}
\end{equation}
where $\delta > 0$ is a cutoff.

\textbf{Idea:} When $H < \delta$, the surface moves at rate $1/\delta$. As $\delta \to 0$, 
surfaces with $H \approx 0$ (MOTS) become stationary while $H < 0$ surfaces move fast.

\begin{lemma}[Short-time Existence]
For $\delta > 0$ fixed, the flow \eqref{eq:modified-imcf} exists for short time $t \in [0, T_\delta)$.
\end{lemma}

\begin{proof}
Standard parabolic theory. The velocity $V = 1/\max(H, \delta) \geq 1/\delta$ is bounded, 
so the flow is uniformly parabolic.
\end{proof}

\subsection{Area Evolution}

\begin{lemma}[Area Formula]
Along the modified flow:
\begin{equation}
\frac{dA}{dt} = \int_{\Sig_t} \frac{H}{\max(H, \delta)} \, dA
\label{eq:area-modified}
\end{equation}
\end{lemma}

\begin{proof}
First variation: $\frac{dA}{dt} = \int H \cdot V \, dA = \int \frac{H}{\max(H, \delta)} \, dA$.
\end{proof}

\textbf{Analysis of \eqref{eq:area-modified}:}
\begin{itemize}
\item Where $H \geq \delta$: contribution is $\int_{H \geq \delta} 1 \, dA = A(H \geq \delta) > 0$
\item Where $H < \delta$: contribution is $\int_{H < \delta} \frac{H}{\delta} \, dA$
  \begin{itemize}
  \item If $H > 0$: positive contribution
  \item If $H < 0$: \textbf{negative contribution} $= \frac{1}{\delta}\int_{H < 0} H \, dA < 0$
  \end{itemize}
\end{itemize}

\begin{gap}[Sign of Area Derivative]
For a trapped surface with $H < 0$ everywhere:
\[
\frac{dA}{dt} = \frac{1}{\delta}\int_{\Sig_t} H \, dA < 0
\]
\textbf{Area still decreases!} The modification doesn't help.
\end{gap}

%% ============================================================================
\section{Attempt 3: Null Flow Approach}
\label{sec:null}
%% ============================================================================

\subsection{The Null Mean Curvature Flow}

Instead of spacelike IMCF, use \emph{null} evolution. Given $\Sig_0$, flow along the 
outgoing null direction $\ell^+$ (the null normal with $\theta^+$ as expansion).

\begin{definition}[Null Flow]
The null mean curvature flow evolves $\Sig_t$ by:
\begin{equation}
\frac{\partial \Sig}{\partial t} = \frac{\ell^+}{\theta^+}
\label{eq:null-flow}
\end{equation}
when $\theta^+ < 0$.
\end{definition}

\textbf{Key difference from spacelike IMCF:} The denominator is $\theta^+ = H + \tr_\Sig k$, 
not $H$ alone.

\subsection{Area Evolution Under Null Flow}

\begin{lemma}[Null Raychaudhuri]
Along the null flow \eqref{eq:null-flow}:
\begin{equation}
\frac{dA}{dt} = \int_{\Sig_t} \frac{\theta^+}{\theta^+} \, dA - \text{shear terms} = A(\Sig_t) - \int \frac{|\sigma|^2}{\theta^+} \, dA
\end{equation}
where $\sigma$ is the null shear.
\end{lemma}

Wait, this isn't right. Let me recalculate.

\textbf{Correct calculation:}
Under null flow with velocity $\phi = 1/\theta^+$ along $\ell^+$:
\[
\frac{dA}{dt} = \int_{\Sig_t} \theta^+ \cdot \phi \, dA = \int_{\Sig_t} 1 \, dA = A(\Sig_t)
\]

\textbf{This looks promising!} Area grows linearly if we use the null flow.

\begin{gap}[Existence of Null Flow]
The flow \eqref{eq:null-flow} is singular when $\theta^+ = 0$ (MOTS). 
As the surface approaches the MOTS, $\theta^+ \to 0^-$ and velocity $\to -\infty$.

\textbf{Problem:} The flow accelerates to infinite speed as it approaches the MOTS, 
potentially overshooting or becoming ill-defined.
\end{gap}

\subsection{Viscosity Solution for Null Flow}

Define the null arrival time function $\tau: M \to \R$ by:
\[
\tau(p) = \inf\{t \geq 0 : p \in J^+(\Sig_t)\}
\]
where $J^+$ is the causal future.

\begin{definition}[Viscosity Solution]
$\tau$ is a \emph{viscosity solution} of the null flow if:
\begin{enumerate}
\item (Subsolution) At any point $p$ where $\tau$ has a smooth upper contact $\phi$:
\[
g^{\mu\nu}\partial_\mu \phi \partial_\nu \phi \leq 0 \quad \text{(null or timelike)}
\]
\item (Supersolution) At any point where $\tau$ has a smooth lower contact:
the analogous inequality.
\end{enumerate}
\end{definition}

\begin{theorem}[Existence of Viscosity Solution]
There exists a unique viscosity solution $\tau$ with $\tau|_{\Sig_0} = 0$.
\end{theorem}

\begin{proof}[Proof Sketch]
Use Perron's method. Define:
\[
\tau(p) = \sup\{v(p) : v \text{ is a subsolution}, v|_{\Sig_0} \leq 0\}
\]

\textbf{Step 1:} The function $\tau_0(p) = $ (Lorentzian distance from $\Sig_0$ to $p$) 
is a subsolution.

\textbf{Step 2:} Comparison principle holds for the eikonal equation 
$g^{\mu\nu}\partial_\mu \tau \partial_\nu \tau = 0$.

\textbf{Step 3:} The supremum is a viscosity solution.
\end{proof}

\begin{gap}[Connection to Area]
Even with a viscosity solution $\tau$, we need to prove:
\[
A(\{\tau = t\}) \geq A(\{\tau = 0\}) = A(\Sig_0)
\]

For smooth null flows, we showed $dA/dt = A$, giving $A(t) = A(0)e^t$. 
But for the viscosity solution:
\begin{enumerate}
\item Level sets $\{\tau = t\}$ may not be smooth
\item The ``area'' of a non-smooth set needs careful definition
\item Monotonicity may fail at jump points
\end{enumerate}
\end{gap}

%% ============================================================================
\section{Attempt 4: Capacity Method}
\label{sec:capacity}
%% ============================================================================

\subsection{The Key Insight}

The Riemannian Penrose inequality (Bray, Huisken-Ilmanen) uses:
\[
\Madm \geq \frac{C(\Sig)}{4\pi}
\]
where $C(\Sig)$ is the capacity. For MOTS, $C(\Sig^*) = \sqrt{4\pi A(\Sig^*)}$.

\textbf{Idea:} Find a modified capacity $\tilde{C}$ such that:
\begin{enumerate}
\item $\tilde{C}(\Sig_0) \geq \sqrt{4\pi A(\Sig_0)}$ for trapped $\Sig_0$
\item $\Madm \geq \tilde{C}(\Sig_0)/(4\pi)$
\end{enumerate}

\subsection{Weighted Capacity}

\begin{definition}[Trapping-Weighted Capacity]
\begin{equation}
\tilde{C}(\Sig) := \inf_{u \in \mathcal{A}} \int_M w(x)^2 |\nabla u|^2 \, dV
\end{equation}
where $\mathcal{A} = \{u : u|_\Sig = 1, u \to 0 \text{ at } \infty\}$ and 
$w(x) = e^{-\psi(x)}$ for some function $\psi$ to be determined.
\end{definition}

\textbf{Goal:} Choose $\psi$ so that $\tilde{C}(\Sig) \geq \sqrt{4\pi A(\Sig)}$.

\subsection{Euler-Lagrange and Monotonicity}

The minimizer $u$ satisfies:
\[
\div(w^2 \nabla u) = 0 \quad \text{in } M \setminus \Sig
\]

\begin{lemma}
If $\psi$ satisfies $\Delta \psi \geq |\nabla \psi|^2 + f$ for some $f \geq 0$, then 
$\tilde{C}$ has good monotonicity properties.
\end{lemma}

\begin{gap}[Choice of Weight]
The natural choice related to trapping would be $\psi \sim \int \theta^+$, but:
\begin{enumerate}
\item $\theta^+$ is only defined on surfaces, not in bulk
\item Any extension of $\theta^+$ to bulk is non-canonical
\item The PDE for $\psi$ may not have solutions with required sign
\end{enumerate}
\end{gap}

%% ============================================================================
\section{Attempt 5: Jang Equation with Different Blow-up}
\label{sec:jang}
%% ============================================================================

\subsection{Review of Jang Approach}

The Jang equation $H_{\text{graph}(f)} - \tr_{\text{graph}(f)} k = 0$ has solutions that blow up 
at MOTS. The induced metric $\hat{g}$ on the graph satisfies:
\[
R_{\hat{g}} \geq 2(\mu - J(\nu)) \geq 0 \quad \text{(DEC)}
\]

\textbf{The Problem:} At blow-up surface $\Sig$, the jump in mean curvature is:
\[
[H_{\hat{g}}] = 2|\tr_\Sig k|
\]
This has the WRONG SIGN when $\tr_\Sig k < 0$.

\subsection{Modified Jang Equation}

Consider the \emph{dual Jang equation}:
\begin{equation}
H_{\text{graph}(f)} + \tr_{\text{graph}(f)} k = 0
\label{eq:dual-jang}
\end{equation}

This blows up where $\theta^- = H - \tr k = 0$ (past MOTS).

\begin{lemma}
Solutions to \eqref{eq:dual-jang} blow up to $-\infty$ at surfaces where $\theta^- = 0$.
\end{lemma}

\textbf{Issue:} Trapped surfaces have $\theta^- < 0$, so they are NOT blow-up surfaces 
of the dual Jang equation.

\subsection{Interpolated Jang}

Try interpolating:
\begin{equation}
H_{\text{graph}(f)} - \lambda \tr_{\text{graph}(f)} k = 0
\label{eq:lambda-jang}
\end{equation}
for $\lambda \in [-1, 1]$.

Blow-up occurs where $H - \lambda \tr k = 0$, i.e., where $\lambda = H/\tr k$.

For a trapped surface: $\theta^+ = H + \tr k \leq 0$ and $\theta^- = H - \tr k < 0$.

If $\tr k > 0$: $H/\tr k < 1$ (from $\theta^- < 0$) and $H/\tr k \leq -1$ (from $\theta^+ \leq 0$), 
so $\lambda = H/\tr k \leq -1$.

If $\tr k < 0$: $H/\tr k > 1$ (from $\theta^- < 0$), so $\lambda = H/\tr k > 1$.

\begin{gap}[No $\lambda \in [-1,1]$ Works]
For a generic trapped surface, there is no $\lambda \in [-1,1]$ such that 
the $\lambda$-Jang equation blows up exactly at that surface.
\end{gap}

%% ============================================================================
\section{Attempt 6: Direct Spacetime Argument}
\label{sec:spacetime}
%% ============================================================================

\subsection{Setup}

Embed $(M, g, k)$ into spacetime $(N^4, \bar{g})$. Let $\Sig_0 \subset M$ be trapped.

\textbf{Assume:} Weak cosmic censorship holds, so there exists event horizon $\mathcal{H}^+$.

\textbf{Goal:} Prove $A(\Sig_0) \leq A(\mathcal{H}^+ \cap M)$.

\subsection{Causal Argument}

\begin{lemma}[Penrose's Original Observation]
If $\Sig_0$ is trapped, then $\Sig_0 \subset \overline{J^-(\mathcal{H}^+)}$.
\end{lemma}

\begin{proof}
By definition, trapped surfaces cannot communicate with $\mathscr{I}^+$, 
hence lie in the black hole region $B = N \setminus J^-(\mathscr{I}^+)$.
The event horizon $\mathcal{H}^+ = \partial B$, so $\Sig_0 \subset B \subset \overline{J^-(\mathcal{H}^+)}$.
\end{proof}

\subsection{Area Comparison}

We want: $A(\Sig_0) \leq A(\mathcal{H}^+ \cap M)$.

\textbf{Attempt via null geodesics:}
Fire null geodesics from $\Sig_0$ toward $\mathcal{H}^+$. 

If we use outgoing null geodesics (along $\ell^+$):
- Initial expansion $\theta^+ \leq 0$
- Raychaudhuri: $\frac{d\theta^+}{d\lambda} = -\frac{1}{2}(\theta^+)^2 - |\sigma|^2 - R_{\mu\nu}\ell^\mu\ell^\nu$
- Under NEC: $\frac{d\theta^+}{d\lambda} \leq -\frac{1}{2}(\theta^+)^2$
- So $\theta^+$ becomes more negative, area decreases.

\begin{gap}[Wrong Direction]
Outgoing null geodesics from a trapped surface go INTO the black hole, 
not toward the horizon. They don't reach $\mathcal{H}^+$.
\end{gap}

\textbf{Attempt via ingoing null geodesics (along $\ell^-$):}
- Initial expansion $\theta^- < 0$
- These go toward the horizon, but $\theta^-$ is already negative
- Area evolution: $\frac{dA}{d\lambda} = \int \theta^- \, dA < 0$
- Area DECREASES along ingoing null geodesics too!

\begin{gap}[Both Directions Fail]
For a trapped surface, BOTH null expansions are negative, 
so area decreases in BOTH null directions. There is no direction 
in which area increases!
\end{gap}

%% ============================================================================
\section{Attempt 7: Optimal Transport}
\label{sec:transport}
%% ============================================================================

\subsection{Riemannian Optimal Transport Review}

In Riemannian geometry, if $\text{Ric} \geq (n-1)K$, then for probability measures 
$\mu_0, \mu_1$ on $M$:
\[
W_2(\mu_0, \mu_1) \leq \text{(diameter bound)}
\]
and entropy is convex along geodesics.

\subsection{Lorentzian Optimal Transport}

For spacetime $(N, \bar{g})$ with timelike Ricci bound, Mondino-Suhr define 
a Lorentzian Wasserstein distance using the cost $c(x,y) = -\tau(x,y)^2$ 
where $\tau$ is the time separation.

\begin{theorem}[Mondino-Suhr, 2022]
If $(N, \bar{g})$ satisfies timelike curvature-dimension condition $\text{TCD}_p(K, N)$, 
then certain entropy functionals are convex along timelike geodesics.
\end{theorem}

\subsection{Application Attempt}

Let $\mu_0 = $ uniform measure on $\Sig_0$ (trapped surface).
Let $\mu_1 = $ uniform measure on $\mathcal{H}^+ \cap M$ (horizon cross-section).

\textbf{Idea:} Use optimal transport to compare $A(\Sig_0)$ and $A(\mathcal{H}^+ \cap M)$.

\begin{gap}[Technical Issues]
\begin{enumerate}
\item TCD requires \textbf{timelike} geodesics between supports. But $\Sig_0$ and 
$\mathcal{H}^+ \cap M$ may not be connected by timelike geodesics (the horizon is null).
\item The curvature-dimension condition TCD requires something like SEC 
(strong energy condition), not just DEC or NEC.
\item The entropy functional in Lorentzian OT is not directly related to area.
\end{enumerate}
\end{gap}

%% ============================================================================
\section{Attempt 8: Variational Approach}
\label{sec:variational}
%% ============================================================================

\subsection{The Maximum Area Problem}

Consider:
\[
A_{\max} := \sup\{A(\Sig) : \Sig \text{ trapped}, \Sig \supset \Sig_0\}
\]

If the supremum is achieved by some $\Sig_{\max}$, what can we say?

\begin{lemma}[First Variation]
If $\Sig_{\max}$ achieves the supremum among trapped surfaces, then:
\[
H = 0 \quad \text{at any point where } \theta^+ = 0, \theta^- < 0
\]
\end{lemma}

\begin{proof}
Vary in the direction $\phi \nu$. First variation of area: $\delta A = \int H \phi \, dA$.
For $\Sig_{\max}$ to be critical among trapped surfaces, we need 
$\delta A = 0$ for all variations preserving trappedness.

If $\theta^+ = 0$ (boundary of trapped condition) and we vary inward ($\phi < 0$), 
then $\theta^+$ can increase (become positive), violating trappedness.
So we can only vary outward at such points, giving $H \geq 0$.
But $H = \frac{1}{2}(\theta^+ + \theta^-) = \frac{1}{2}\theta^- < 0$ at such points. Contradiction.

Thus $\Sig_{\max}$ cannot have $\theta^+ = 0$ points interior to the trapped region. 
It must be a MOTS itself.
\end{proof}

\begin{gap}[Existence of Maximum]
The supremum $A_{\max}$ may not be achieved! 

\textbf{Example:} The trapped region may be unbounded, with surfaces of 
arbitrarily large area. Or the maximizing sequence may ``escape to infinity'' 
or develop singularities.
\end{gap}

%% ============================================================================
\section{Synthesis: The Core Obstruction}
\label{sec:synthesis}
%% ============================================================================

After all attempts, the obstruction is clear:

\begin{theorem}[Fundamental Obstruction]
For a trapped surface $\Sig_0$ with $\theta^+ < 0$ and $\theta^- < 0$ everywhere:
\begin{enumerate}
\item Any smooth outward evolution decreases area (since $H < 0$)
\item Any smooth inward evolution also decreases area (since $\theta^- < 0$)
\item Any null evolution decreases area in both directions
\item The Jang equation cannot produce favorable jump
\end{enumerate}
There is NO smooth flow that increases area starting from $\Sig_0$.
\end{theorem}

\subsection{What Would Be Needed}

To prove Penrose 1973 unconditionally, one would need:

\textbf{Option A: Discontinuous Flow}
A weak solution theory where:
\begin{itemize}
\item Area can ``jump up'' at certain instants
\item The jump is controlled by the geometry
\item The final area equals $A(\text{MOTS})$
\end{itemize}

\textbf{Option B: Different Monotone Quantity}
A functional $F(\Sig)$ such that:
\begin{itemize}
\item $F$ is monotone along some flow
\item $F(\Sig_0) = $ something depending only on $A(\Sig_0)$
\item $F(\text{MOTS}) = $ something giving $\Madm \geq \sqrt{A/(16\pi)}$
\end{itemize}

\textbf{Option C: Direct Argument}
A proof that bypasses flows entirely, perhaps using:
\begin{itemize}
\item Spinorial methods (Witten-style)
\item Index theory
\item Algebraic/topological arguments
\end{itemize}

%% ============================================================================
\section{A New Attempt: Conformal Flow}
\label{sec:conformal}
%% ============================================================================

\subsection{Idea}

Instead of moving the surface, conformally change the metric to make $H > 0$.

Let $\tilde{g} = e^{2\phi} g$ for some function $\phi$. The mean curvature transforms as:
\[
\tilde{H} = e^{-\phi}(H + 2\partial_\nu \phi)
\]

To make $\tilde{H} > 0$, we need $\partial_\nu \phi > -H/2 > 0$ (since $H < 0$).

\subsection{The Conformal Factor}

Solve:
\begin{equation}
\Delta \phi = f, \quad \phi|_\infty = 0, \quad \partial_\nu \phi|_{\Sig_0} = -H/2 + \epsilon
\label{eq:conformal}
\end{equation}

This is a mixed boundary value problem. 

\begin{lemma}
There exists $\phi$ solving \eqref{eq:conformal} for appropriate $f$.
\end{lemma}

\subsection{Effect on Mass}

Under conformal change $\tilde{g} = e^{2\phi} g$, the ADM mass transforms.

\begin{gap}[Mass Change]
The ADM mass is NOT conformally invariant. 
\[
\tilde{M} = M + \text{(terms involving $\phi$)}
\]
If $\phi$ is significant near infinity, the mass can increase or decrease arbitrarily.
\end{gap}

%% ============================================================================
\section{Final Assessment}
\label{sec:final}
%% ============================================================================

\textbf{Status after hard analysis:}

All eight approaches encounter fundamental obstructions:

\begin{center}
\begin{tabular}{|l|l|}
\hline
\textbf{Approach} & \textbf{Obstruction} \\
\hline
Elliptic regularization & Limit doesn't produce jump \\
Parabolic IMCF & Area still decreases \\
Null flow & Singular at MOTS, direction issues \\
Capacity method & No canonical weight function \\
Modified Jang & No $\lambda$ produces blow-up at trapped surface \\
Spacetime causal & Both null directions decrease area \\
Optimal transport & TCD needs SEC, not applicable \\
Variational & Existence of maximum unproven \\
Conformal & Changes mass, not helpful \\
\hline
\end{tabular}
\end{center}

\textbf{Honest Conclusion:}

The 1973 Penrose conjecture for arbitrary trapped surfaces remains \textbf{OPEN}. 
The fundamental issue is geometric: trapped surfaces have $H < 0$, and no known 
technique can overcome this without additional assumptions.

The most promising directions for future work:
\begin{enumerate}
\item Viscosity solutions for null flows with rigorous existence theory
\item New monotone quantities beyond area
\item Spinorial methods avoiding flows entirely
\end{enumerate}

\end{document}
