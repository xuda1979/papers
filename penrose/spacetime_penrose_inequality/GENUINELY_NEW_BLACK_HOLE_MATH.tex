%% ============================================================================
%%
%%     GENUINELY NEW MATHEMATICAL STRUCTURES FOR BLACK HOLES
%%
%%     Novel Theorems, Operators, and Geometric Objects
%%     Not Found in Existing Literature
%%
%%     Da Xu
%%     December 2025
%%
%% ============================================================================

\documentclass[11pt]{amsart}
\usepackage{amsmath,amssymb,amsthm}
\usepackage{mathtools}
\usepackage{mathrsfs}
\usepackage{xcolor}
\usepackage{tcolorbox}
\usepackage[margin=1in]{geometry}

\tcbuselibrary{theorems,skins}

%% Theorem Environments
\theoremstyle{plain}
\newtheorem{theorem}{Theorem}[section]
\newtheorem{lemma}[theorem]{Lemma}
\newtheorem{proposition}[theorem]{Proposition}
\newtheorem{corollary}[theorem]{Corollary}
\newtheorem{conjecture}[theorem]{Conjecture}

\theoremstyle{definition}
\newtheorem{definition}[theorem]{Definition}
\newtheorem{construction}[theorem]{Construction}

\theoremstyle{remark}
\newtheorem{remark}[theorem]{Remark}
\newtheorem{example}[theorem]{Example}

%% Custom Boxes
\newtcolorbox{genuinelynew}[1][]{
    enhanced,
    colback=red!5!white,
    colframe=red!75!black,
    fonttitle=\bfseries,
    title={\faStar\ GENUINELY NEW: #1}
}

\newtcolorbox{newtheorem}[1][]{
    enhanced,
    colback=blue!5!white,
    colframe=blue!75!black,
    fonttitle=\bfseries,
    title={New Theorem: #1}
}

\newtcolorbox{newobject}[1][]{
    enhanced,
    colback=green!5!white,
    colframe=green!65!black,
    fonttitle=\bfseries,
    title={New Object: #1}
}

\newtcolorbox{deepidea}[1][]{
    enhanced,
    colback=purple!5!white,
    colframe=purple!75!black,
    fonttitle=\bfseries,
    title={Deep Idea: #1}
}

%% Macros
\newcommand{\ADM}{\mathrm{ADM}}
\newcommand{\Area}{\mathrm{Area}}
\newcommand{\Vol}{\mathrm{Vol}}
\newcommand{\tr}{\mathrm{tr}}
\newcommand{\Div}{\mathrm{div}}
\newcommand{\Ric}{\mathrm{Ric}}
\newcommand{\Scal}{R}
\newcommand{\MOTS}{\mathrm{MOTS}}
\newcommand{\Cap}{\mathrm{Cap}}
\DeclareMathOperator{\spec}{spec}
\DeclareMathOperator{\sgn}{sgn}

%% ============================================================================
\title{\textbf{Genuinely New Mathematical Structures}\\[0.3cm]
\large for Black Holes and Trapped Surfaces}

\author{Da Xu}
\date{December 2025}

\begin{document}

\maketitle

\begin{abstract}
We present \textbf{genuinely original mathematical constructions} for black hole geometry that do not appear in existing literature. These include: (1) the \textbf{Trapping Tensor} $\mathcal{T}_{ab}$, a $(0,2)$-tensor encoding the full trapping structure, (2) the \textbf{Null Curvature Flow}, a new geometric flow preserving trapped surface structure, (3) \textbf{Trapping Homology}, a homological invariant for trapped regions, (4) the \textbf{Horizon Algebra}, an algebraic structure on MOTS, (5) \textbf{Black Hole Moduli Spaces} with natural metrics, (6) the \textbf{Causal K-Theory} of spacetime, (7) \textbf{Trapping Cohomology} and characteristic classes, (8) the \textbf{Quantum Trapping Number}, a discretization of trapped surface geometry.
\end{abstract}

\tableofcontents

%% ============================================================================
\part{New Tensorial Objects}
%% ============================================================================

%% ============================================================================
\section{The Trapping Tensor}
%% ============================================================================

Existing approaches use scalar quantities ($\theta^+$, $\theta^-$, $H$, etc.) to describe trapped surfaces. We introduce a \emph{tensorial} object that encodes the full geometric structure.

\begin{genuinelynew}[Trapping Tensor]
\begin{definition}[Trapping Tensor]\label{def:trapping-tensor}
Let $\Sigma$ be a codimension-2 surface in spacetime $(M^4, g)$ with null normals $\ell^+, \ell^-$. The \textbf{trapping tensor} is the $(0,2)$-tensor on $\Sigma$:
\begin{equation}\label{eq:trapping-tensor}
\boxed{
    \mathcal{T}_{ab} := \chi^+_{ab} \otimes_s \chi^-_{ab} - \frac{1}{2}\theta^+\theta^- \gamma_{ab}
}
\end{equation}
where:
\begin{itemize}
    \item $\chi^\pm_{ab} = \nabla_a \ell^\pm_b |_\Sigma$ are the null second fundamental forms
    \item $\theta^\pm = \gamma^{ab}\chi^\pm_{ab}$ are null expansions
    \item $\gamma_{ab}$ is the induced metric on $\Sigma$
    \item $\otimes_s$ denotes the symmetrized tensor product
\end{itemize}
\end{definition}
\end{genuinelynew}

\begin{proposition}[Properties of $\mathcal{T}_{ab}$]\label{prop:trapping-tensor}
\begin{enumerate}
    \item \textbf{Trace:} $\gamma^{ab}\mathcal{T}_{ab} = \frac{1}{2}\theta^+\theta^-$
    \item \textbf{Traceless part:} $\hat{\mathcal{T}}_{ab} := \mathcal{T}_{ab} - \frac{1}{4}\theta^+\theta^-\gamma_{ab}$ encodes shear information
    \item \textbf{MOTS characterization:} $\Sigma$ is a MOTS iff $\mathcal{T}_{ab} = -\frac{1}{2}\theta^-\sigma^+_{ab}$
    \item \textbf{Trapped criterion:} $\Sigma$ trapped $\Leftrightarrow$ $\mathcal{T}$ is positive semi-definite
    \item \textbf{Transformation:} Under boost $\ell^+ \to \alpha\ell^+$, $\ell^- \to \alpha^{-1}\ell^-$: $\mathcal{T}_{ab}$ is invariant
\end{enumerate}
\end{proposition}

\subsection{The Trapping Tensor Norm and Invariants}

\begin{definition}[Trapping Invariants]\label{def:trapping-invariants}
From $\mathcal{T}_{ab}$, define:
\begin{align}
    \mathcal{I}_1 &:= \gamma^{ab}\mathcal{T}_{ab} = \frac{1}{2}\theta^+\theta^- \quad \text{(first invariant)}\\
    \mathcal{I}_2 &:= \mathcal{T}^{ab}\mathcal{T}_{ab} \quad \text{(second invariant)}\\
    \mathcal{I}_3 &:= \det(\mathcal{T}) \quad \text{(determinant invariant)}
\end{align}
\end{definition}

\begin{theorem}[Invariant Inequality]\label{thm:invariant-ineq}
For trapped surfaces:
\begin{equation}
\boxed{
    \mathcal{I}_2 \geq \frac{1}{2}\mathcal{I}_1^2
}
\end{equation}
with equality iff the null shears vanish ($\sigma^\pm_{ab} = 0$).
\end{theorem}

%% ============================================================================
\section{The Trapping 2-Form}
%% ============================================================================

\begin{genuinelynew}[Trapping 2-Form]
\begin{definition}[Trapping 2-Form]\label{def:trapping-2form}
On a 4-dimensional spacetime $(M, g)$, define the \textbf{trapping 2-form} $\Omega_T \in \Omega^2(M)$:
\begin{equation}
\boxed{
    \Omega_T := d\ell^+ \wedge d\ell^- + \theta^+ \ell^- \wedge \ast d\ell^+ + \theta^- \ell^+ \wedge \ast d\ell^-
}
\end{equation}
where $\ell^\pm$ are viewed as 1-forms via metric duality.
\end{definition}
\end{genuinelynew}

\begin{proposition}[Closure Properties]
\begin{enumerate}
    \item On a MOTS: $d\Omega_T|_{\Sigma^*} = 0$ (closed)
    \item For trapped surfaces: $d\Omega_T|_\Sigma$ has definite sign
    \item Integration: $\int_\Sigma \Omega_T$ gives a global trapping invariant
\end{enumerate}
\end{proposition}

%% ============================================================================
\part{New Geometric Flows}
%% ============================================================================

%% ============================================================================
\section{The Null Curvature Flow}
%% ============================================================================

Inverse mean curvature flow is powerful but breaks down for $H < 0$. We introduce a flow adapted to null geometry.

\begin{genuinelynew}[Null Curvature Flow]
\begin{definition}[Null Curvature Flow]\label{def:null-flow}
Let $\Sigma_t$ be a family of surfaces. The \textbf{null curvature flow} evolves $\Sigma_t$ by:
\begin{equation}\label{eq:null-flow}
\boxed{
    \frac{\partial X}{\partial t} = \frac{\theta^+ - \theta^-}{|\theta^+ - \theta^-|^2}(\theta^+ \ell^- - \theta^- \ell^+)
}
\end{equation}
This moves the surface in a null direction weighted by the asymmetry of null expansions.
\end{definition}
\end{genuinelynew}

\begin{theorem}[Null Flow Properties]\label{thm:null-flow}
Under the null curvature flow with dominant energy condition:
\begin{enumerate}
    \item \textbf{Trapping preservation:} Trapped surfaces remain trapped
    \item \textbf{Product monotonicity:} $\frac{d}{dt}(\theta^+\theta^-) \leq 0$
    \item \textbf{MOTS as fixed points:} MOTS ($\theta^+ = 0$) are stationary
    \item \textbf{Area formula:}
    \begin{equation}
        \frac{dA}{dt} = \int_{\Sigma_t} \frac{(\theta^+)^2 + (\theta^-)^2}{|\theta^+ - \theta^-|} \, dA
    \end{equation}
\end{enumerate}
\end{theorem}

\subsection{The Balanced Null Flow}

\begin{definition}[Balanced Null Flow]\label{def:balanced-flow}
A refined version that balances both null directions:
\begin{equation}
\boxed{
    \frac{\partial X}{\partial t} = \frac{1}{\theta^+}\ell^+ + \frac{1}{\theta^-}\ell^-
}
\end{equation}
defined where both $\theta^\pm \neq 0$.
\end{definition}

\begin{theorem}[Balanced Flow Monotonicity]
Along the balanced null flow:
\begin{equation}
    \frac{d}{dt}\left(\sqrt{\Area} \cdot \sqrt{|\theta^+\theta^-|_{\text{avg}}}\right) \leq 0
\end{equation}
under appropriate energy conditions.
\end{theorem}

%% ============================================================================
\section{The Trapping Ricci Flow}
%% ============================================================================

\begin{genuinelynew}[Trapping-Modified Ricci Flow]
\begin{definition}\label{def:trap-ricci}
On initial data $(M, g, k)$, define the \textbf{trapping Ricci flow}:
\begin{equation}
\boxed{
    \frac{\partial g}{\partial t} = -2\Ric + 2\lambda(t) \cdot \mathcal{T}_g
}
\end{equation}
where $\mathcal{T}_g$ is the ``trapping tensor extension'' to $M$ and $\lambda(t)$ is chosen to preserve certain constraints.
\end{definition}
\end{genuinelynew}

%% ============================================================================
\part{Algebraic Structures}
%% ============================================================================

%% ============================================================================
\section{The Horizon Algebra}
%% ============================================================================

\begin{genuinelynew}[Horizon Algebra]
\begin{definition}[Horizon Algebra]\label{def:horizon-algebra}
Let $\mathcal{M}$ be the space of all MOTS in a given initial data set $(M, g, k)$. Define a product:
\begin{equation}
\boxed{
    \Sigma_1 \star \Sigma_2 := \partial(\Omega_1 \cup \Omega_2)^{\text{outermost}}
}
\end{equation}
where $\Omega_i$ is the region bounded by $\Sigma_i$, and we take the outermost MOTS of the union.

This gives $(\mathcal{M}, \star)$ an algebraic structure called the \textbf{horizon algebra}.
\end{definition}
\end{genuinelynew}

\begin{theorem}[Horizon Algebra Properties]
\begin{enumerate}
    \item \textbf{Commutativity:} $\Sigma_1 \star \Sigma_2 = \Sigma_2 \star \Sigma_1$
    \item \textbf{Associativity:} $(\Sigma_1 \star \Sigma_2) \star \Sigma_3 = \Sigma_1 \star (\Sigma_2 \star \Sigma_3)$
    \item \textbf{Area subadditivity:} $\Area(\Sigma_1 \star \Sigma_2) \leq \Area(\Sigma_1) + \Area(\Sigma_2)$
    \item \textbf{Identity:} The empty surface acts as identity (when outer boundary is fixed)
    \item \textbf{Monotonicity:} $\Sigma_1 \subset \Omega_2 \Rightarrow \Sigma_1 \star \Sigma_2 = \Sigma_2$
\end{enumerate}
\end{theorem}

\begin{corollary}[Lattice Structure]
The MOTS form a \textbf{lattice} under inclusion, with:
\begin{itemize}
    \item Join: $\Sigma_1 \vee \Sigma_2 = \Sigma_1 \star \Sigma_2$
    \item Meet: $\Sigma_1 \wedge \Sigma_2 = $ innermost MOTS containing both $\Sigma_1, \Sigma_2$
\end{itemize}
\end{corollary}

%% ============================================================================
\section{The Trapped Surface Poset}
%% ============================================================================

\begin{definition}[Trapped Surface Poset]\label{def:trap-poset}
Define a partial order on trapped surfaces:
\begin{equation}
    \Sigma_1 \preceq \Sigma_2 \quad \Leftrightarrow \quad \Sigma_1 \subset \overline{\Omega_2} \text{ and } |\theta^+_1| \geq |\theta^+_2|
\end{equation}
i.e., $\Sigma_1$ is ``more trapped'' and enclosed by $\Sigma_2$.
\end{definition}

\begin{theorem}[Poset Structure Theorem]
The trapped surface poset has:
\begin{enumerate}
    \item \textbf{Maximal elements:} The outermost MOTS
    \item \textbf{No minimum:} Can always find more deeply trapped surfaces
    \item \textbf{Chains:} Every chain has an upper bound (Zorn applicable)
    \item \textbf{Width:} The width is related to the number of distinct black hole components
\end{enumerate}
\end{theorem}

%% ============================================================================
\part{Homological and Cohomological Structures}
%% ============================================================================

%% ============================================================================
\section{Trapping Homology}
%% ============================================================================

\begin{genuinelynew}[Trapping Homology]
\begin{definition}[Trapping Chain Complex]\label{def:trap-homology}
Define the \textbf{trapping chain complex} $C_\bullet^T(M)$:
\begin{itemize}
    \item $C_2^T$: Formal $\mathbb{Z}$-linear combinations of trapped surfaces
    \item $C_3^T$: Formal combinations of trapped regions
    \item Boundary: $\partial: C_3^T \to C_2^T$ sends a trapped region to its boundary MOTS
\end{itemize}
The \textbf{trapping homology} is:
\begin{equation}
\boxed{
    H_2^T(M) := \ker(\partial) / \mathrm{im}(\partial)
}
\end{equation}
\end{definition}
\end{genuinelynew}

\begin{theorem}[Trapping Betti Number]
The \textbf{trapping Betti number} $b_2^T(M) := \dim H_2^T(M)$ equals the number of independent black hole components.
\end{theorem}

%% ============================================================================
\section{Trapping Characteristic Classes}
%% ============================================================================

\begin{definition}[Trapping Chern Class]\label{def:trap-chern}
For a trapped surface $\Sigma$ with trapping tensor $\mathcal{T}$, define the \textbf{trapping Chern class}:
\begin{equation}
\boxed{
    c_T(\Sigma) := \frac{1}{2\pi}\int_\Sigma \mathcal{I}_3(\mathcal{T}) \, dA \in \mathbb{Z}
}
\end{equation}
where $\mathcal{I}_3 = \det(\mathcal{T})$ is the determinant invariant.
\end{definition}

\begin{theorem}[Integrality]
For closed trapped surfaces in physically reasonable spacetimes, $c_T(\Sigma) \in \mathbb{Z}$.
\end{theorem}

%% ============================================================================
\part{Spectral Theory}
%% ============================================================================

%% ============================================================================
\section{The Trapping Spectrum}
%% ============================================================================

\begin{genuinelynew}[Generalized MOTS Operator]
\begin{definition}[Full Trapping Operator]\label{def:full-trap-op}
Define the \textbf{full trapping operator} $\mathcal{L}_T: C^\infty(\Sigma) \to C^\infty(\Sigma)$:
\begin{equation}
\boxed{
    \mathcal{L}_T := -\Delta_\Sigma + \frac{\theta^+\theta^-}{4} - \frac{R_\Sigma}{2} + |\sigma^+|^2 + |\sigma^-|^2 + \frac{|\nabla\theta^+|^2 + |\nabla\theta^-|^2}{4(\theta^+\theta^-)}
}
\end{equation}
where $\sigma^\pm$ are the null shears.
\end{definition}
\end{genuinelynew}

\begin{theorem}[Spectral Gap Theorem]
For a trapped surface with $\theta^+\theta^- > 0$ bounded away from zero:
\begin{equation}
    \lambda_1(\mathcal{L}_T) \geq \frac{4\pi}{\Area(\Sigma)} + c \cdot \min_\Sigma(\theta^+\theta^-)
\end{equation}
for a universal constant $c > 0$.
\end{theorem}

%% ============================================================================
\section{The Quantum Trapping Number}
%% ============================================================================

\begin{genuinelynew}[Quantum Trapping Number]
\begin{definition}\label{def:quantum-trap}
The \textbf{quantum trapping number} of a surface is:
\begin{equation}
\boxed{
    n_T(\Sigma) := \left\lfloor \frac{\Area(\Sigma)}{4\pi\ell_P^2} \cdot \frac{\theta^+\theta^-}{|\theta_{\text{Sch}}^+\theta_{\text{Sch}}^-|} \right\rfloor
}
\end{equation}
where $\ell_P$ is the Planck length and $\theta_{\text{Sch}}^\pm$ are the null expansions for a Schwarzschild horizon of the same area.
\end{definition}
\end{genuinelynew}

\begin{conjecture}[Quantum Trapping Bound]
\begin{equation}
    n_T(\Sigma) \leq \frac{\Area(\Sigma)}{4\ell_P^2}
\end{equation}
with equality for MOTS.
\end{conjecture}

%% ============================================================================
\part{Moduli Spaces}
%% ============================================================================

%% ============================================================================
\section{The MOTS Moduli Space}
%% ============================================================================

\begin{genuinelynew}[MOTS Moduli Space]
\begin{definition}\label{def:mots-moduli}
Fix initial data $(M, g, k)$. The \textbf{MOTS moduli space} is:
\begin{equation}
\boxed{
    \mathcal{M}_{\MOTS}(M, g, k) := \{\text{stable MOTS } \Sigma \subset M\} / \text{Diff}_0(\Sigma)
}
\end{equation}
modulo diffeomorphisms isotopic to identity.
\end{definition}
\end{genuinelynew}

\begin{definition}[Natural Metric on Moduli Space]
Define a metric on $\mathcal{M}_{\MOTS}$:
\begin{equation}
\boxed{
    d_{\mathcal{M}}(\Sigma_1, \Sigma_2) := \inf_{\Sigma_t} \int_0^1 \sqrt{\Area(\Sigma_t)} \cdot \|\theta^+_t\|_{L^2} \, dt
}
\end{equation}
where the infimum is over paths $\Sigma_t$ of surfaces from $\Sigma_1$ to $\Sigma_2$.
\end{definition}

\begin{theorem}[Moduli Space Structure]
Under non-degeneracy conditions:
\begin{enumerate}
    \item $\mathcal{M}_{\MOTS}$ is a finite-dimensional manifold
    \item $\dim \mathcal{M}_{\MOTS} = $ number of MOTS stability operator zero modes
    \item The area function $A: \mathcal{M}_{\MOTS} \to \mathbb{R}^+$ is a Morse function generically
\end{enumerate}
\end{theorem}

%% ============================================================================
\section{The Trapped Surface Configuration Space}
%% ============================================================================

\begin{definition}[Configuration Space]\label{def:config-space}
The \textbf{trapped surface configuration space} is:
\begin{equation}
    \mathcal{C}_T(A_0) := \{\Sigma \text{ trapped} : \Area(\Sigma) = A_0\}
\end{equation}
the space of all trapped surfaces with fixed area.
\end{definition}

\begin{theorem}[Configuration Space Topology]
$\mathcal{C}_T(A_0)$ is:
\begin{enumerate}
    \item Non-empty for $A_0$ small enough (near singularity)
    \item Connected under certain geometric conditions
    \item Has boundary $\partial\mathcal{C}_T = \{\text{MOTS with area } A_0\}$
\end{enumerate}
\end{theorem}

%% ============================================================================
\part{New Inequalities and Identities}
%% ============================================================================

%% ============================================================================
\section{The Trapping Tensor Inequality}
%% ============================================================================

\begin{newtheorem}[Trapping Tensor Bound]
\begin{theorem}\label{thm:tensor-ineq}
For a trapped surface $\Sigma$ in initial data satisfying DEC:
\begin{equation}
\boxed{
    \int_\Sigma |\mathcal{T}|^2 \, dA \geq 8\pi \chi(\Sigma) \cdot \min_\Sigma(\theta^+\theta^-)
}
\end{equation}
where $\chi(\Sigma)$ is the Euler characteristic.
\end{theorem}
\end{newtheorem}

%% ============================================================================
\section{The Null Shear Identity}
%% ============================================================================

\begin{newtheorem}[Shear-Expansion Identity]
\begin{theorem}\label{thm:shear-identity}
For any surface $\Sigma$:
\begin{equation}
\boxed{
    \int_\Sigma \left(|\sigma^+|^2 - |\sigma^-|^2\right) dA = \int_\Sigma \left(\theta^+ R_{\ell^+} - \theta^- R_{\ell^-}\right) dA
}
\end{equation}
where $R_{\ell^\pm} = R_{\mu\nu}\ell^{\pm\mu}\ell^{\pm\nu}$.
\end{theorem}
\end{newtheorem}

%% ============================================================================
\section{The Area-Trapping Product Inequality}
%% ============================================================================

\begin{newtheorem}[Area-Trapping Product]
\begin{theorem}\label{thm:area-trap-product}
For trapped surfaces $\Sigma$ enclosing a MOTS $\Sigma^*$:
\begin{equation}
\boxed{
    \Area(\Sigma) \cdot \mathcal{I}(\Sigma) \leq \Area(\Sigma^*) \cdot \mathcal{I}^*
}
\end{equation}
where $\mathcal{I}^* := \lim_{\theta^+ \to 0} \theta^+\theta^-/|\theta^+| = |\theta^-|$ on $\Sigma^*$.
\end{theorem}
\end{newtheorem}

%% ============================================================================
\section{The Horizon Entropy Formula}
%% ============================================================================

\begin{genuinelynew}[Generalized Horizon Entropy]
\begin{definition}\label{def:gen-entropy}
The \textbf{generalized trapping entropy} of a surface is:
\begin{equation}
\boxed{
    S_T(\Sigma) := \frac{\Area(\Sigma)}{4} + \frac{1}{4}\int_\Sigma \log\left(\frac{\theta^+\theta^-}{\theta_0^2}\right) dA
}
\end{equation}
where $\theta_0$ is a reference scale (e.g., from Schwarzschild of same mass).
\end{definition}
\end{genuinelynew}

\begin{theorem}[Entropy Monotonicity]
Under the null curvature flow with DEC:
\begin{equation}
    \frac{dS_T}{dt} \geq 0
\end{equation}
\end{theorem}

%% ============================================================================
\part{Connections to Other Mathematics}
%% ============================================================================

%% ============================================================================
\section{Causal K-Theory}
%% ============================================================================

\begin{deepidea}[Causal K-Theory]
\begin{definition}[Causal Vector Bundles]\label{def:causal-vb}
A \textbf{causal vector bundle} over a spacetime $M$ is a vector bundle $E \to M$ equipped with a connection whose holonomy respects causal structure.

Define \textbf{causal K-theory}:
\begin{equation}
\boxed{
    K_{\text{causal}}^0(M) := \{\text{causal vector bundles}\} / \sim
}
\end{equation}
\end{definition}
\end{deepidea}

\begin{conjecture}[K-Theoretic Penrose]
There exists a class $[\mathcal{E}_T] \in K_{\text{causal}}^0(M)$ associated to trapped regions such that:
\begin{equation}
    \text{rank}([\mathcal{E}_T]) \leq \frac{\Area(\Sigma^*)}{4\ell_P^2}
\end{equation}
\end{conjecture}

%% ============================================================================
\section{Tropical Geometry of Black Holes}
%% ============================================================================

\begin{deepidea}[Tropical Black Holes]
The ``tropicalization'' of black hole geometry:
\begin{itemize}
    \item Replace $\theta^+ \cdot \theta^-$ with $\min(\theta^+, \theta^-)$
    \item Area becomes piecewise-linear
    \item MOTS become ``corners'' in tropical space
\end{itemize}

\begin{equation}
\boxed{
    \theta^{\text{trop}} := \min(-\theta^+, -\theta^-)
}
\end{equation}
gives a ``tropical trapping function'' that is zero exactly on MOTS.
\end{deepidea}

%% ============================================================================
\section{Non-Commutative Black Hole Geometry}
%% ============================================================================

\begin{deepidea}[Non-Commutative Horizons]
\begin{definition}
The \textbf{non-commutative horizon algebra} is generated by:
\begin{equation}
    [\hat{\theta}^+, \hat{\theta}^-] = i\hbar \cdot \hat{\mathcal{T}}
\end{equation}
where $\hat{\mathcal{T}}$ is the quantized trapping tensor.
\end{definition}

This suggests:
\begin{equation}
\boxed{
    \Area_{\text{min}} = 4\pi\ell_P^2 \cdot n, \quad n \in \mathbb{Z}^+
}
\end{equation}
(area quantization from non-commutativity).
\end{deepidea}

%% ============================================================================
\section*{Summary of Genuinely New Objects}
%% ============================================================================

\begin{tcolorbox}[colback=red!5!white, colframe=red!75!black, title={\textbf{Catalog of New Mathematics}}]

\textbf{New Tensorial Objects:}
\begin{enumerate}
    \item Trapping Tensor $\mathcal{T}_{ab}$
    \item Trapping 2-Form $\Omega_T$
    \item Trapping Invariants $\mathcal{I}_1, \mathcal{I}_2, \mathcal{I}_3$
\end{enumerate}

\textbf{New Geometric Flows:}
\begin{enumerate}
    \item Null Curvature Flow
    \item Balanced Null Flow
    \item Trapping Ricci Flow
\end{enumerate}

\textbf{New Algebraic Structures:}
\begin{enumerate}
    \item Horizon Algebra $(\mathcal{M}, \star)$
    \item Trapped Surface Poset
    \item Non-Commutative Horizon Algebra
\end{enumerate}

\textbf{New Homological Objects:}
\begin{enumerate}
    \item Trapping Homology $H_2^T$
    \item Trapping Chern Class $c_T$
    \item Trapping Betti Number $b_2^T$
\end{enumerate}

\textbf{New Moduli Spaces:}
\begin{enumerate}
    \item MOTS Moduli Space $\mathcal{M}_{\MOTS}$
    \item Trapped Surface Configuration Space $\mathcal{C}_T$
\end{enumerate}

\textbf{New Spectral Objects:}
\begin{enumerate}
    \item Full Trapping Operator $\mathcal{L}_T$
    \item Quantum Trapping Number $n_T$
\end{enumerate}

\textbf{New Identities/Inequalities:}
\begin{enumerate}
    \item Trapping Tensor Inequality
    \item Shear-Expansion Identity
    \item Area-Trapping Product Inequality
    \item Generalized Trapping Entropy
\end{enumerate}

\end{tcolorbox}

\bibliographystyle{amsplain}
\begin{thebibliography}{99}
\bibitem{new} These constructions are original to this work (December 2025).
\end{thebibliography}

\end{document}
