%% JANG_SURFACE_AREA_DOMINANCE.tex
%%
%% THE JANG SURFACE APPROACH TO AREA DOMINANCE
%%
%% Key Innovation: The Jang equation creates a graph over C where the
%% relationship between areas is more tractable.
%%
%% December 2025

\documentclass[11pt]{amsart}
\usepackage{amsmath,amssymb,amsthm}
\usepackage{tcolorbox}

\tcbuselibrary{theorems}

\newtcolorbox{key}{
    colback=blue!5!white,
    colframe=blue!75!black,
    title={\textbf{KEY INSIGHT}}
}

\newtcolorbox{theorem_box}{
    colback=green!5!white,
    colframe=green!50!black,
    title={\textbf{MAIN RESULT}}
}

\newtcolorbox{gap_box}{
    colback=red!5!white,
    colframe=red!75!black,
    title={\textbf{GAP/ISSUE}}
}

\newtheorem{theorem}{Theorem}
\newtheorem{lemma}[theorem]{Lemma}
\newtheorem{proposition}[theorem]{Proposition}
\newtheorem{corollary}[theorem]{Corollary}
\theoremstyle{definition}
\newtheorem{definition}[theorem]{Definition}
\newtheorem{remark}[theorem]{Remark}

\newcommand{\Area}{\mathrm{Area}}
\newcommand{\Vol}{\mathrm{Vol}}
\newcommand{\divv}{\mathrm{div}}
\DeclareMathOperator{\tr}{tr}

\title{The Jang Surface Approach to Area Dominance}
\author{December 2025}

\begin{document}
\maketitle

\begin{abstract}
We develop an approach to Area Dominance using the Jang equation, which 
transforms the initial data $(\mathcal{C}, g, k)$ into a Riemannian 
manifold $(\hat{\mathcal{C}}, \hat{g})$ where area comparisons are 
more tractable. The key insight is that on the Jang surface, the 
relationship between trapped surfaces and MOTS becomes geometric.
\end{abstract}

%% ============================================================================
\section{Review: The Jang Equation}
%% ============================================================================

\begin{definition}[Jang Equation]
Given initial data $(\mathcal{C}, g, k)$, the Jang equation for a 
function $f: \mathcal{C} \to \mathbb{R}$ is:
\begin{equation}
    H_f - \tr_{\Gamma_f}(k) = 0
\end{equation}

where:
\begin{itemize}
    \item $\Gamma_f = \{(x, f(x)) : x \in \mathcal{C}\} \subset 
          \mathcal{C} \times \mathbb{R}$ is the graph of $f$
    \item $H_f$ is the mean curvature of $\Gamma_f$ in 
          $(\mathcal{C} \times \mathbb{R}, g + dt^2)$
    \item $\tr_{\Gamma_f}(k)$ is the trace of $k$ restricted to $\Gamma_f$
\end{itemize}
\end{definition}

\begin{proposition}[Jang Equation in Coordinates]
\begin{equation}
    \divv\left(\frac{\nabla f}{\sqrt{1 + |\nabla f|^2}}\right) 
    - \frac{k_{ij}(g^{ij} - \frac{\nabla^i f \nabla^j f}{1 + |\nabla f|^2})}
    {\sqrt{1 + |\nabla f|^2}} = 0
\end{equation}

Or equivalently:
\begin{equation}
    (g^{ij} - \frac{\nabla^i f \nabla^j f}{1 + |\nabla f|^2})
    (\nabla_i\nabla_j f - k_{ij}\sqrt{1 + |\nabla f|^2}) = 0
\end{equation}
\end{proposition}

%% ============================================================================
\section{The Key Relationship}
%% ============================================================================

\begin{key}
\textbf{Why the Jang Equation is Relevant}

On the Jang surface $\Gamma_f$, the combination
\begin{equation}
    H_f - P_f = H_f - \tr_{\Gamma_f}(k) = 0
\end{equation}

vanishes! This is exactly the condition that makes $\Gamma_f$ behave like 
a MOTS with respect to the induced geometry.

More precisely:
\begin{equation}
    \theta^+|_{\Gamma_f} = H_f + P_f = 2P_f
\end{equation}

So on $\Gamma_f$: $\theta^+ = 2P$ and $H = P$.
\end{key}

%% ============================================================================
\section{Blow-Up at MOTS}
%% ============================================================================

\begin{theorem}[Jang Blow-Up - Schoen-Yau]
The Jang equation solution $f$ blows up (i.e., $f \to +\infty$ or $f \to -\infty$) 
precisely at MOTS.

More precisely:
\begin{enumerate}
    \item If $\Sigma^*$ is an outermost MOTS with $\theta^- < 0$ (future MOTS), 
          then $f \to +\infty$ as we approach $\Sigma^*$
    \item If $\Sigma'$ is a past MOTS with $\theta^+ = 0$, $\theta^- > 0$, 
          then $f \to -\infty$ near $\Sigma'$
\end{enumerate}
\end{theorem}

\begin{key}
\textbf{The Geometric Picture}

The Jang surface $\Gamma_f$ is a graph over $\mathcal{C}$ that:
\begin{itemize}
    \item Blows up to $+\infty$ at the outermost MOTS $\Sigma^*$
    \item Forms a "cylinder" over $\Sigma^*$ (going to infinity)
    \item The region inside $\Sigma^*$ maps to a region of $\Gamma_f$
\end{itemize}
\end{key}

%% ============================================================================
\section{Area on the Jang Surface}
%% ============================================================================

\begin{proposition}[Area Comparison on Jang Surface]
Let $\Sigma \subset \mathcal{C}$ be a surface inside $\Sigma^*$.

Let $\hat{\Sigma} = \{(x, f(x)) : x \in \Sigma\} \subset \Gamma_f$ be its 
lift to the Jang surface.

Then:
\begin{equation}
    \Area_{\hat{g}}(\hat{\Sigma}) = \int_\Sigma \sqrt{1 + |\nabla f|^2} \, dA_g
    \ge \Area_g(\Sigma)
\end{equation}

with equality if and only if $\nabla f|_\Sigma = 0$.
\end{proposition}

\begin{proof}
The induced metric on $\hat{\Sigma}$ is:
\begin{equation}
    \hat{\gamma}_{ab} = \gamma_{ab} + \nabla_a f \nabla_b f
\end{equation}

The area element is:
\begin{equation}
    d\hat{A} = \sqrt{\det(\gamma + df \otimes df)} = 
    \sqrt{\det(\gamma)(1 + |df|^2_\gamma)} = \sqrt{1 + |\nabla f|^2} \, dA_g
\end{equation}
\end{proof}

%% ============================================================================
\section{The Cylinder Contribution}
%% ============================================================================

\begin{proposition}[MOTS Lift]
As $f \to +\infty$ near $\Sigma^*$, the Jang surface forms a cylinder:
\begin{equation}
    \Gamma_f|_{\text{near } \Sigma^*} \approx \Sigma^* \times [T, \infty)
\end{equation}

The cross-sectional area of this cylinder is $\Area_g(\Sigma^*)$.
\end{proposition}

\begin{key}
\textbf{The Key Observation}

On the Jang surface:
\begin{enumerate}
    \item Trapped surfaces $\Sigma$ lift to $\hat{\Sigma}$ with 
          $\Area(\hat{\Sigma}) \ge \Area(\Sigma)$
    \item The outermost MOTS $\Sigma^*$ lifts to a cylinder with 
          cross-section $\Sigma^*$
    \item The cylinder has area $= \Area(\Sigma^*)$ per cross-section
\end{enumerate}

If we could show $\Area(\hat{\Sigma}) \le \Area(\Sigma^*)$, we'd have:
\begin{equation}
    \Area(\Sigma) \le \Area(\hat{\Sigma}) \le \Area(\Sigma^*)
\end{equation}

But we need $\Area(\hat{\Sigma}) \le \Area(\Sigma^*)$!
\end{key}

%% ============================================================================
\section{The Riemannian Structure on $\Gamma_f$}
%% ============================================================================

\begin{theorem}[Schoen-Yau Scalar Curvature]
The induced metric $\hat{g}$ on the Jang surface $\Gamma_f$ satisfies:
\begin{equation}
    \hat{R} \ge 2(\mu - |J|) - 2|H_f - \tr k|^2 + \text{lower order}
\end{equation}

Under DEC ($\mu \ge |J|$) and using $H_f = \tr k$ (Jang equation):
\begin{equation}
    \hat{R} \ge 0
\end{equation}

modulo possible cylindrical ends.
\end{theorem}

\begin{key}
\textbf{This is the crucial point!}

The Jang surface $(\Gamma_f, \hat{g})$ has NON-NEGATIVE scalar curvature!

This brings us into the realm of Riemannian geometry where we have powerful 
tools like the Riemannian Penrose inequality.
\end{key}

%% ============================================================================
\section{The Area Dominance on $\Gamma_f$}
%% ============================================================================

Now we can apply Riemannian techniques on $(\Gamma_f, \hat{g})$.

\begin{proposition}[IMCF on Jang Surface]
Consider Inverse Mean Curvature Flow starting from $\hat{\Sigma}$ on 
$(\Gamma_f, \hat{g})$.

If $\hat{R} \ge 0$, the Hawking mass:
\begin{equation}
    m_H(\hat{\Sigma}_t) = \sqrt{\frac{\Area(\hat{\Sigma}_t)}{16\pi}}
    \left(1 - \frac{1}{16\pi}\int_{\hat{\Sigma}_t} \hat{H}^2 d\hat{A}\right)
\end{equation}

is NON-DECREASING along the flow.
\end{proposition}

\begin{gap_box}
\textbf{The Issue:}

IMCF starting from $\hat{\Sigma}$ may not reach the cylinder over $\Sigma^*$.

The flow may:
\begin{enumerate}
    \item Exit through a different boundary
    \item Develop singularities before reaching the cylinder
    \item Go to infinity without meeting the cylinder
\end{enumerate}

We need to control the TOPOLOGY of the Jang surface.
\end{gap_box}

%% ============================================================================
\section{The Trapped Region on $\Gamma_f$}
%% ============================================================================

\begin{lemma}[Trapped Surface on Jang Surface]
If $\Sigma \subset \mathcal{C}$ is trapped ($\theta^+ < 0$, $\theta^- < 0$), 
then its lift $\hat{\Sigma} \subset \Gamma_f$ has:
\begin{equation}
    \hat{H} = \sqrt{1 + |\nabla f|^2}(H_f - P_f)|_{\hat{\Sigma}}
\end{equation}

Using $H_f - P_f = 0$ on $\Gamma_f$ (Jang equation... wait, no!)

Actually, the Jang equation holds for $\Gamma_f$ as a whole, but $\hat{\Sigma}$ 
is a SURFACE in $\Gamma_f$, not $\Gamma_f$ itself!
\end{lemma}

Let me reconsider. The Jang equation is:
\begin{equation}
    H_{\Gamma_f} - P_{\Gamma_f} = 0
\end{equation}

where $H_{\Gamma_f}$ and $P_{\Gamma_f}$ are computed for $\Gamma_f$ as a 
3-surface in $(\mathcal{C} \times \mathbb{R}, g + dt^2)$.

For a 2-surface $\hat{\Sigma} \subset \Gamma_f$, we have a DIFFERENT mean 
curvature: the mean curvature of $\hat{\Sigma}$ as a surface in $(\Gamma_f, \hat{g})$.

%% ============================================================================
\section{Clarifying the Geometry}
%% ============================================================================

Let me set up the geometry carefully.

\textbf{Ambient space:} $(M^4, g + dt^2)$ where $M^4 = \mathcal{C} \times \mathbb{R}$

\textbf{Jang surface:} $\Gamma_f^3 \subset M^4$ is the graph of $f$

\textbf{Original surface:} $\Sigma^2 \subset \mathcal{C}^3$ is trapped

\textbf{Lifted surface:} $\hat{\Sigma}^2 = \{(x, f(x)) : x \in \Sigma\} \subset \Gamma_f^3$

We have:
\begin{itemize}
    \item $H_\Sigma$ = mean curvature of $\Sigma$ in $(\mathcal{C}, g)$
    \item $\hat{H}_{\hat{\Sigma}}$ = mean curvature of $\hat{\Sigma}$ in $(\Gamma_f, \hat{g})$
    \item $\theta^+_\Sigma$ = outward null expansion of $\Sigma$ in spacetime
\end{itemize}

\begin{proposition}[Mean Curvature Relationship]
The mean curvature $\hat{H}$ of $\hat{\Sigma}$ in $(\Gamma_f, \hat{g})$ is 
related to the original geometry by:
\begin{equation}
    \hat{H} = \frac{H_\Sigma + \text{(gradient terms)}}{\sqrt{1 + |\nabla f|^2}}
\end{equation}

(The exact formula involves second derivatives of $f$ restricted to $\Sigma$.)
\end{proposition}

%% ============================================================================
\section{The Key Question}
%% ============================================================================

\begin{key}
\textbf{What we need:}

Show that on the Jang surface $(\Gamma_f, \hat{g})$ with $\hat{R} \ge 0$:

$\Area_{\hat{g}}(\hat{\Sigma}) \le \Area_{\hat{g}}(\text{cylinder cross-section})
= \Area_g(\Sigma^*)$

This would give:
\begin{equation}
    \Area(\Sigma) \le \Area(\hat{\Sigma}) \le \Area(\Sigma^*)
\end{equation}
\end{key}

\begin{gap_box}
\textbf{The Problem:}

We can't directly compare $\Area(\hat{\Sigma})$ with the cylinder area!

\begin{enumerate}
    \item $\hat{\Sigma}$ and the cylinder may not be "comparable" in 
          any geometric sense
    \item There's no obvious foliation or flow connecting them
    \item The Jang surface topology near the cylinder is singular
\end{enumerate}
\end{gap_box}

%% ============================================================================
\section{A New Idea: The Capped Jang Surface}
%% ============================================================================

\begin{definition}[Capped Jang Surface]
Modify the Jang surface near $\Sigma^*$ by "capping off" the cylinder.

At height $T$ (large), glue a minimal surface cap $D$ across the cylinder:
\begin{equation}
    \tilde{\Gamma}_f = \Gamma_f|_{f < T} \cup D
\end{equation}

where $D$ is a minimal surface spanning the circle $\Sigma^* \times \{T\}$.
\end{definition}

\begin{proposition}[Cap Area]
The minimal cap $D$ has area satisfying:
\begin{equation}
    \Area(D) = \Area(\Sigma^*)
\end{equation}

(In the limit $T \to \infty$, the cap becomes isometric to a flat disk if 
the cylinder is exactly cylindrical.)

Actually, for a true cylinder $\Sigma^* \times [0, \infty)$, any cross-section 
has area $\Area(\Sigma^*)$.
\end{proposition}

%% ============================================================================
\section{Applying Riemannian Penrose on $\tilde{\Gamma}_f$}
%% ============================================================================

\begin{theorem_box}
\textbf{Strategy}

\begin{enumerate}
    \item $(\tilde{\Gamma}_f, \hat{g})$ is a Riemannian 3-manifold with 
          $\hat{R} \ge 0$
    \item The boundary (or "horizon") is the cap $D$ with $\Area(D) = 
          \Area(\Sigma^*)$
    \item $\hat{\Sigma}$ is a surface inside $\tilde{\Gamma}_f$
    \item If $\hat{\Sigma}$ is MINIMAL (or has $\hat{H} \ge 0$), then by 
          standard Riemannian arguments: $\Area(\hat{\Sigma}) \le \Area(D)$
\end{enumerate}
\end{theorem_box}

\begin{gap_box}
\textbf{Critical Issue:}

Is $\hat{\Sigma}$ minimal (or $\hat{H} \ge 0$) in $(\Gamma_f, \hat{g})$?

$\Sigma$ is TRAPPED in spacetime: $\theta^+ < 0$, $\theta^- < 0$.

This does NOT directly imply $\hat{H} \ge 0$ on the Jang surface!
\end{gap_box}

%% ============================================================================
\section{Computing $\hat{H}$}
%% ============================================================================

Let's compute more carefully.

On the Jang surface $\Gamma_f$, the Jang equation says:
\begin{equation}
    H_{\Gamma} = P_{\Gamma} = \tr_{\Gamma}(k)
\end{equation}

where $H_\Gamma$ and $P_\Gamma$ are computed viewing $\Gamma_f$ as a 
hypersurface in $(M^4, g + dt^2)$.

For a surface $\hat{\Sigma} \subset \Gamma_f$, its mean curvature 
$\hat{H}$ in $(\Gamma_f, \hat{g})$ is different.

Using the Gauss equation:
\begin{equation}
    \hat{H} = H_{\hat{\Sigma} \subset M^4} - H_{\Gamma}|_{\hat{\Sigma}}
\end{equation}

where:
\begin{itemize}
    \item $H_{\hat{\Sigma} \subset M^4}$ = mean curvature of $\hat{\Sigma}$ 
          in $M^4$
    \item $H_\Gamma|_{\hat{\Sigma}}$ = normal curvature of $\Gamma$ at 
          $\hat{\Sigma}$
\end{itemize}

Actually, this is getting complicated. Let me think differently.

%% ============================================================================
\section{Direct Computation}
%% ============================================================================

The induced metric on $\Gamma_f$ is $\hat{g} = g + df \otimes df$.

The induced metric on $\hat{\Sigma}$ from $\hat{g}$ is:
\begin{equation}
    \hat{\gamma}_{ab} = \gamma_{ab} + (\partial_a f)(\partial_b f)
\end{equation}

where $\gamma$ is the induced metric on $\Sigma$ from $g$.

The mean curvature $\hat{H}$ of $\hat{\Sigma}$ in $(\Gamma_f, \hat{g})$ is:
\begin{equation}
    \hat{H} = \hat{g}^{ij}(\hat{\nabla}_i \hat{\nu}_j)
\end{equation}

where $\hat{\nu}$ is the unit normal to $\hat{\Sigma}$ in $\Gamma_f$.

\begin{lemma}[Unit Normal]
If $\nu$ is the unit outward normal to $\Sigma$ in $\mathcal{C}$, then 
the unit normal $\hat{\nu}$ to $\hat{\Sigma}$ in $\Gamma_f$ is:
\begin{equation}
    \hat{\nu} = \frac{\nu - (\nu \cdot \nabla f)\frac{\partial_t}
    {\sqrt{1 + |\nabla f|^2}}}{\sqrt{1 + |\nabla f|^2 - (\nu \cdot \nabla f)^2
    /(1 + |\nabla f|^2)}}
\end{equation}

This simplifies if $\nabla f$ is tangent to $\Sigma$ (i.e., $\nu \cdot \nabla f = 0$).
\end{lemma}

%% ============================================================================
\section{Special Case: $\nabla f \perp \nu$}
%% ============================================================================

\begin{proposition}[Mean Curvature when $\nabla f \perp \nu$]
If $\nabla f$ is tangent to $\Sigma$ (i.e., $f$ is constant along normal 
geodesics from $\Sigma$), then:
\begin{equation}
    \hat{H} = H_\Sigma
\end{equation}

The mean curvature is preserved!
\end{proposition}

\begin{key}
In general, this won't hold. The Jang solution $f$ depends on the global 
geometry, not just $\Sigma$.

But near the MOTS $\Sigma^*$ (where $f \to \infty$), the gradient 
$\nabla f$ becomes large and PERPENDICULAR to $\Sigma^*$.

So the lift of $\Sigma^*$ has $\hat{H} \approx H_{\Sigma^*}$.

For MOTS: $H_{\Sigma^*} = \theta^+ - P = 0 - P = -P$.

If $P < 0$: $H_{\Sigma^*} > 0$ (convex surface, smaller inside)

If $P > 0$: $H_{\Sigma^*} < 0$ (concave surface, larger inside)
\end{key}

%% ============================================================================
\section{Conclusion}
%% ============================================================================

The Jang surface approach transforms the problem:

\textbf{Original:} Compare $\Area(\Sigma)$ to $\Area(\Sigma^*)$ in 
$(\mathcal{C}, g, k)$

\textbf{Transformed:} Compare $\Area(\hat{\Sigma})$ to $\Area(D)$ in 
$(\tilde{\Gamma}_f, \hat{g})$ with $\hat{R} \ge 0$

The transformation gains:
\begin{enumerate}
    \item Non-negative scalar curvature $\hat{R} \ge 0$
    \item Riemannian tools available
\end{enumerate}

The transformation loses:
\begin{enumerate}
    \item Direct relationship between $\hat{H}$ and trapped condition
    \item Simple topology (cylindrical ends are singular)
\end{enumerate}

\begin{gap_box}
\textbf{Remaining Gap:}

The sign of $\hat{H}$ for $\hat{\Sigma}$ (lift of trapped surface) is 
NOT determined by the trapped condition.

This mirrors the original problem: the sign of $H_\Sigma$ is not determined 
by $\theta^+ < 0$.

The Jang transformation doesn't SOLVE the fundamental sign ambiguity - 
it merely REFORMULATES it in a different geometric setting.
\end{gap_box}

\textbf{The fundamental obstruction persists.}

\end{document}
