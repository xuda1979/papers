%% OM_PROOF_PDE.tex  
%% Rigorous PDE-based proof of the outer-minimizing property (OM)
%% Key innovation: Use the constraint equations + spacetime reconstruction

\documentclass[11pt]{amsart}
\usepackage{amsmath,amssymb,amsthm}
\usepackage{mathtools}
\usepackage{xcolor}

\newtheorem{theorem}{Theorem}[section]
\newtheorem{lemma}[theorem]{Lemma}
\newtheorem{proposition}[theorem]{Proposition}
\newtheorem{corollary}[theorem]{Corollary}
\newtheorem{definition}[theorem]{Definition}
\newtheorem{remark}[theorem]{Remark}
\newtheorem{claim}[theorem]{Claim}

\newcommand{\ADM}{\mathrm{ADM}}
\newcommand{\Area}{\mathrm{Area}}
\newcommand{\tr}{\mathrm{tr}}

\title{Rigorous Proof of (OM) via PDE Methods and Spacetime Reconstruction}
\author{Analysis of the Penrose Inequality}
\date{December 2025}

\begin{document}
\maketitle

\section{The Problem}

\textbf{Goal:} Prove the outer-minimizing property (OM) under weak cosmic censorship (WCC):
\begin{equation}\label{eq:OM}
    A(\Sigma) \le A(\mathcal{H}_\mathcal{C})
\end{equation}
for any trapped surface $\Sigma$ on Cauchy surface $\mathcal{C}$, where $\mathcal{H}_\mathcal{C} = \mathcal{H} \cap \mathcal{C}$ is the event horizon cross-section.

\textbf{The fundamental difficulty:}
\begin{itemize}
    \item The event horizon $\mathcal{H}$ is a \textbf{global} construct requiring knowledge of the entire future spacetime.
    \item The Cauchy surface $\mathcal{C}$ only provides \textbf{local} initial data $(M, g, k)$.
    \item Cosmic censorship tells us the spacetime exists and is ``nice,'' but doesn't directly give geometric bounds.
\end{itemize}

\section{Key Insight: Rigidity of the Event Horizon}

The crucial observation is that the event horizon has a special PDE characterization:

\begin{lemma}[Event Horizon as a Barrier]\label{lem:horizon-barrier}
Under weak cosmic censorship, the event horizon cross-section $\mathcal{H}_\mathcal{C}$ satisfies:
\begin{equation}
    \theta^+_{\mathcal{H}_\mathcal{C}} = 0 \quad \text{(marginally trapped in outgoing direction)}.
\end{equation}
Moreover, $\mathcal{H}_\mathcal{C}$ is the \textbf{outermost} surface with this property on $\mathcal{C}$ (in the sense that any trapped surface lies in the region bounded by $\mathcal{H}_\mathcal{C}$).
\end{lemma}

\begin{proof}
By the Penrose--Hawking theorems, the event horizon $\mathcal{H}$ is a null hypersurface with $\theta^+_{\mathcal{H}} = 0$. On any Cauchy slice $\mathcal{C}$, the cross-section $\mathcal{H}_\mathcal{C}$ inherits this marginal trapping.

The outermost property follows from causality: if a trapped surface $\Sigma$ were outside $\mathcal{H}_\mathcal{C}$, then by the focusing theorem, the outgoing null geodesics from $\Sigma$ would develop caustics before reaching future null infinity, contradicting cosmic censorship.
\end{proof}

\section{The Spacetime Reconstruction Approach}

\textbf{Strategy:} Use the constraint equations and evolution to construct the spacetime, then analyze the event horizon.

\begin{theorem}[Spacetime Evolution]\label{thm:evolution}
Given initial data $(M^3, g, k)$ satisfying DEC and the constraint equations:
\begin{align}
    R_g - |k|^2 + (\tr_g k)^2 &= 2\mu \ge 0, \\
    \nabla^j(k_{ij} - (\tr_g k) g_{ij}) &= J_i, \quad |J| \le \mu,
\end{align}
there exists a maximal globally hyperbolic development $(N^{3+1}, \bar{g})$ with Cauchy surface $\mathcal{C} = M$.
\end{theorem}

\begin{proof}
Standard result from Choquet-Bruhat--Geroch.
\end{proof}

\textbf{Problem:} WCC asserts this development extends to $\mathscr{I}^+$ without naked singularities, but this is a \textbf{conjecture} we are assuming, not proving.

\section{New Approach: Trapped Region Analysis}

Since the event horizon is defined globally, we use a \textbf{local proxy}: the trapped region $\mathcal{T}$.

\begin{definition}[Trapped Region]
The trapped region on $\mathcal{C}$ is:
\begin{equation}
    \mathcal{T} = \{p \in \mathcal{C} : p \text{ lies on some trapped surface}\}.
\end{equation}
Its boundary $\Sigma^* = \partial\mathcal{T}$ is the \textbf{apparent horizon} (outermost MOTS).
\end{definition}

\textbf{Key relationship under WCC:}
\begin{equation}
    \Sigma^* \subseteq \overline{\mathcal{H}_\mathcal{C}} \quad \text{(apparent horizon inside event horizon)}.
\end{equation}

\section{Main Theorem: Area Comparison via Constrained Optimization}

\begin{theorem}[Area Comparison Under WCC]\label{thm:main}
Let $(M^3, g, k)$ be asymptotically flat initial data satisfying DEC. Assume:
\begin{enumerate}
    \item \textbf{(WCC)} The maximal development extends to future null infinity without naked singularities.
    \item \textbf{(FS)} The spacetime settles to a Kerr solution asymptotically.
\end{enumerate}
Let $\Sigma$ be any closed trapped surface on $M$. Then:
\begin{equation}
    A(\Sigma) \le A(\mathcal{H}_\mathcal{C}).
\end{equation}
\end{theorem}

\begin{proof}
The proof proceeds in several steps.

\textbf{Step 1: Trapped surfaces lie inside the black hole region.}

Under WCC, the black hole region $\mathcal{B} = N \setminus J^-(\mathscr{I}^+)$ is nonempty (assuming $\Sigma$ exists), and any trapped surface lies in $\mathcal{B}$. This is the Penrose singularity theorem applied in reverse: if $\Sigma \not\subset \mathcal{B}$, then null geodesics from $\Sigma$ would reach $\mathscr{I}^+$ while having negative expansion, contradicting their eventual escape.

\textbf{Step 2: The event horizon cross-section bounds the trapped region.}

On the Cauchy surface $\mathcal{C}$:
\begin{equation}
    \Sigma \subset \mathcal{T} \subset \text{Int}(\mathcal{H}_\mathcal{C}),
\end{equation}
where $\mathcal{T}$ is the trapped region and $\text{Int}(\mathcal{H}_\mathcal{C})$ is the interior of the region bounded by $\mathcal{H}_\mathcal{C}$.

\textbf{Step 3: Area bound from the future evolution.}

This is the key step. Consider the null hypersurface $\mathcal{N}^+$ generated by outgoing null geodesics from $\Sigma$. Under NEC, by the Raychaudhuri equation:
\begin{equation}
    \frac{d\theta^+}{d\lambda} = -\frac{1}{2}(\theta^+)^2 - |\sigma|^2 - R_{\mu\nu}\ell^\mu\ell^\nu \le 0.
\end{equation}

Since $\theta^+(\Sigma) \le 0$, the expansion remains nonpositive. Two scenarios:

\textbf{Case A:} $\mathcal{N}^+$ hits a caustic before reaching $\mathcal{H}$. Then $\theta^+ \to -\infty$ at some point.

\textbf{Case B:} $\mathcal{N}^+$ reaches the event horizon $\mathcal{H}$.

In Case B, let $\Sigma' = \mathcal{N}^+ \cap \mathcal{H}$ be the intersection. By the Raychaudhuri evolution:
\begin{equation}
    A(\Sigma') \le A(\Sigma) \quad \text{(area decreases along outgoing null with } \theta^+ \le 0\text{)}.
\end{equation}

\textbf{Step 4: Hawking area theorem on the event horizon.}

The event horizon $\mathcal{H}$ is a null hypersurface with $\theta^+_\mathcal{H} = 0$. By the Hawking area theorem, cross-sectional area is non-decreasing toward the future:
\begin{equation}
    A(\mathcal{H}_\mathcal{C}) \le A(\mathcal{H}_{\mathcal{C}'}) \quad \text{for } \mathcal{C}' \text{ to the future of } \mathcal{C}.
\end{equation}

However, we need to compare $A(\Sigma')$ on $\mathcal{H}$ with $A(\mathcal{H}_\mathcal{C})$ on $\mathcal{C}$.

\textbf{Step 5: Combining the bounds.}

\textcolor{red}{\textbf{GAP:}} Step 4 gives area evolution along $\mathcal{H}$ (future direction), but $\Sigma'$ lies on $\mathcal{H}$ at a \textbf{later time} than $\mathcal{H}_\mathcal{C}$. We have:
\begin{align}
    A(\Sigma) &\ge A(\Sigma') \quad \text{(by Step 3, if Case B)}, \\
    A(\Sigma') &\le A(\mathcal{H}_{\text{final}}) \quad \text{(since $\Sigma' \subset \mathcal{H}$)}.
\end{align}
But $A(\mathcal{H}_{\text{final}}) \ge A(\mathcal{H}_\mathcal{C})$ by Hawking, so this gives:
\begin{equation}
    A(\Sigma) \ge A(\Sigma') \;\; \text{and} \;\; A(\Sigma') \le A(\mathcal{H}_{\text{final}}) \ge A(\mathcal{H}_\mathcal{C}).
\end{equation}
This doesn't close the loop!
\end{proof}

\section{Correct Approach: Use the Characteristic Initial Value Problem}

The above attempt fails because we're flowing in the wrong direction. Let me try a different approach.

\begin{theorem}[Area Comparison via Characteristic Data]\label{thm:char}
Under WCC + NEC + FS, for any trapped surface $\Sigma$ on $\mathcal{C}$:
\begin{equation}
    A(\Sigma) \le A(\mathcal{H}_\mathcal{C}).
\end{equation}
\end{theorem}

\begin{proof}
Consider the \textbf{ingoing} null hypersurface $\mathcal{N}^-$ from the event horizon cross-section $\mathcal{H}_\mathcal{C}$, directed toward the interior of the black hole.

The ingoing null expansion on $\mathcal{H}_\mathcal{C}$ is $\theta^-_{\mathcal{H}_\mathcal{C}} < 0$ (the event horizon is \textbf{not} trapped in the ingoing direction, but typically has negative ingoing expansion for dynamical black holes).

Actually, wait. For a \textbf{stationary} black hole (Kerr), the event horizon has $\theta^- < 0$ (ingoing light is converging). For a \textbf{dynamical} black hole, the situation is more complex.

\textbf{Key observation:} Under the final state assumption (FS), the spacetime settles to Kerr. At late times, the event horizon approaches a Kerr horizon with $\theta^+ = 0$ and $\theta^- < 0$.

\textcolor{red}{\textbf{GAP:}} On the initial Cauchy surface $\mathcal{C}$, the event horizon cross-section $\mathcal{H}_\mathcal{C}$ may not have $\theta^- < 0$. For example, during rapid accretion, the horizon could be expanding in the ingoing direction too.
\end{proof}

\section{Fundamental Obstruction}

After multiple attempts, I've identified the fundamental obstruction:

\begin{remark}[Why (OM) is Hard]
The (OM) assumption $A(\Sigma) \le A(\mathcal{H}_\mathcal{C})$ relates two objects:
\begin{enumerate}
    \item $\Sigma$: An arbitrary trapped surface on $\mathcal{C}$ (local data).
    \item $\mathcal{H}_\mathcal{C}$: The event horizon cross-section (global data).
\end{enumerate}

The difficulty is that:
\begin{itemize}
    \item $\mathcal{H}_\mathcal{C}$ is defined by the \textbf{future} behavior of the spacetime.
    \item $\Sigma$ is specified on the \textbf{present} slice.
    \item There is no direct geometric flow connecting them within $\mathcal{C}$.
\end{itemize}

The natural flows (IMCF, MCF) don't preserve the trapped property or have the wrong monotonicity. The null flows leave the Cauchy surface.
\end{remark}

\section{Possible Resolution: Topological Constraint}

One potential approach uses \textbf{topology} rather than PDE:

\begin{proposition}[Topological Constraint]\label{prop:topology}
Under WCC + FS, if $\Sigma$ is a trapped surface on $\mathcal{C}$ and $\mathcal{H}_\mathcal{C}$ is the event horizon cross-section, then:
\begin{enumerate}
    \item Both $\Sigma$ and $\mathcal{H}_\mathcal{C}$ are 2-spheres (Galloway--Schoen for $\Sigma$; topological censorship for $\mathcal{H}_\mathcal{C}$).
    \item $\Sigma$ is contained in the ball bounded by $\mathcal{H}_\mathcal{C}$ on $\mathcal{C}$.
\end{enumerate}
\end{proposition}

However, containment doesn't imply area comparison! A small sphere can have larger area than a large sphere if the metric is sufficiently distorted (e.g., near a very massive object).

\section{Honest Conclusion}

\textbf{Status:} The (OM) assumption remains \textbf{unproven} even under WCC + NEC + FS.

\textbf{What we can prove:}
\begin{itemize}
    \item $\Sigma \subset \text{Int}(\mathcal{H}_\mathcal{C})$ (topological containment).
    \item The apparent horizon $\Sigma^*$ satisfies $\Sigma^* \subseteq \overline{\mathcal{H}_\mathcal{C}}$.
    \item For stable MOTS, the Penrose inequality holds via the Direct Trapped Surface Construction.
\end{itemize}

\textbf{What remains open:}
\begin{itemize}
    \item Area comparison $A(\Sigma) \le A(\mathcal{H}_\mathcal{C})$ for arbitrary trapped surfaces.
    \item A direct relationship between the local geometry of $\Sigma$ and the global geometry of $\mathcal{H}_\mathcal{C}$.
\end{itemize}

\textbf{Promising directions:}
\begin{enumerate}
    \item \textbf{Lorentzian optimal transport:} Develop a ``Lorentzian Wasserstein distance'' that connects trapped surfaces to the event horizon.
    \item \textbf{Dynamical horizon methods:} Use the quasi-local structure of dynamical horizons to interpolate between $\Sigma$ and $\mathcal{H}_\mathcal{C}$.
    \item \textbf{Spectral methods:} Analyze the stability operator spectrum to constrain area differences.
\end{enumerate}

\end{document}
