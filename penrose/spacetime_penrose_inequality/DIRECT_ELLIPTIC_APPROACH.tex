\documentclass[11pt]{article}
\usepackage[margin=1in]{geometry}
\usepackage{amsmath,amsthm,amssymb,mathrsfs}
\usepackage{mathtools}
\usepackage{enumitem}
\usepackage{hyperref}
\usepackage{xcolor}

\newtheorem{theorem}{Theorem}[section]
\newtheorem{lemma}[theorem]{Lemma}
\newtheorem{proposition}[theorem]{Proposition}
\newtheorem{corollary}[theorem]{Corollary}
\newtheorem{definition}[theorem]{Definition}
\newtheorem{remark}[theorem]{Remark}
\newtheorem{claim}{Claim}

\newcommand{\tr}{\mathrm{tr}}
\newcommand{\Ric}{\mathrm{Ric}}
\newcommand{\Rm}{\mathrm{Rm}}
\newcommand{\Vol}{\mathrm{Vol}}
\newcommand{\diam}{\mathrm{diam}}
\newcommand{\supp}{\mathrm{supp}}
\newcommand{\dist}{\mathrm{dist}}
\newcommand{\sgn}{\mathrm{sgn}}
\newcommand{\Hess}{\mathrm{Hess}}
\newcommand{\Div}{\mathrm{div}}
\newcommand{\ADM}{\mathrm{ADM}}
\newcommand{\MOTS}{\mathrm{MOTS}}
\newcommand{\irr}{\mathrm{irr}}

\title{\textbf{The Direct Constraint Equation Approach}\\[0.5em]
\large Bypassing Flows via Elliptic Methods}
\author{}
\date{December 2025}

\begin{document}
\maketitle

\begin{abstract}
We develop a direct approach to the spacetime Penrose inequality using the constraint equations themselves, avoiding geometric flows. The idea: construct a harmonic function adapted to the trapped surface that directly yields the mass-area bound.
\end{abstract}

%% ============================================================================
\section{The Direct Strategy}
%% ============================================================================

\subsection{Motivation}

The flow-based approaches (Jang + IMCF/Bray) work but require area comparison:
$$A(\MOTS) \geq A(\text{trapped surface})$$
which is \textbf{false} in general.

\textbf{New idea:} Don't compare areas. Instead, construct a function directly relating $M_{\ADM}$ to $A(\Sigma_0)$.

\subsection{The Positive Mass Argument}

Recall Witten's proof of positive mass:
\begin{enumerate}
\item Solve Dirac equation: $D\psi = 0$
\item Integrate: $0 \leq \int |\nabla\psi|^2 + \frac{R}{4}|\psi|^2 = M_{\ADM} \cdot (\text{boundary term})$
\item Conclude: $M_{\ADM} \geq 0$
\end{enumerate}

\textbf{Idea:} Modify boundary conditions to extract $\sqrt{A/16\pi}$ instead of $0$.

%% ============================================================================
\section{The Adapted Harmonic Function}
%% ============================================================================

\subsection{Setup}

Let $\Sigma_0$ be a trapped surface. Define:
\begin{equation}
\Omega = M \setminus \overline{B_{\Sigma_0}}
\end{equation}
(exterior of $\Sigma_0$).

\begin{definition}[Capacity Function]
Let $u: \Omega \to [0,1]$ solve:
\begin{equation}
\begin{cases}
\Delta_g u = 0 & \text{in } \Omega \\
u|_{\Sigma_0} = 1 \\
u \to 0 & \text{at infinity}
\end{cases}
\end{equation}
\end{definition}

\begin{lemma}[Capacity and Mass]
The capacity of $\Sigma_0$ is:
\begin{equation}
\text{Cap}(\Sigma_0) = \int_\Omega |Du|^2 dV_g = -\int_{\Sigma_0} \frac{\partial u}{\partial \nu} dA
\end{equation}
\end{lemma}

\subsection{The Key Identity}

\begin{theorem}[Bray's Identity]
On $(M, g)$ with $R \geq 0$:
\begin{equation}
M_{\ADM} = \frac{1}{16\pi}\int_M R \cdot u^2 \, dV + \frac{1}{16\pi}\text{Cap}(\Sigma_0)^2 \cdot (\text{geometric factor})
\end{equation}
\end{theorem}

For our case with $R_g$ not necessarily non-negative (due to $k$), we need to use the constraint equations.

%% ============================================================================
\section{Using the Constraint Equations}
%% ============================================================================

\subsection{The Constraint Equations}

On initial data $(M, g, k)$:
\begin{align}
R_g - |k|^2 + (\tr k)^2 &= 2\mu \quad \text{(Hamiltonian)}\\
\Div_g(k - (\tr k)g) &= J \quad \text{(Momentum)}
\end{align}

DEC: $\mu \geq |J|$.

\subsection{Scalar Curvature Decomposition}

\begin{equation}
R_g = 2\mu + |k|^2 - (\tr k)^2 = 2\mu + |k - \frac{\tr k}{3}g|^2 - \frac{2}{3}(\tr k)^2
\end{equation}

Let $\mathring{k} = k - \frac{\tr k}{3}g$ (traceless part). Then:
\begin{equation}
R_g = 2\mu + |\mathring{k}|^2 - \frac{2}{3}(\tr k)^2
\end{equation}

\begin{corollary}
If $\mu \geq 0$ (from DEC) and $\tr k = 0$ (maximal slice):
\begin{equation}
R_g \geq |\mathring{k}|^2 \geq 0
\end{equation}
\end{corollary}

\subsection{Non-Maximal Slices}

For general $\tr k \neq 0$:
\begin{equation}
R_g = 2\mu + |\mathring{k}|^2 - \frac{2}{3}(\tr k)^2
\end{equation}

This can be negative even with DEC!

%% ============================================================================
\section{The Jang-Modified Metric}
%% ============================================================================

\subsection{Conformal Jang}

The Jang equation produces $\bar{g} = g + df \otimes df$ with:
\begin{equation}
R_{\bar{g}} \geq 2(\mu - |J|) \geq 0 \quad \text{(DEC)}
\end{equation}

\begin{theorem}
On $(\bar{M}, \bar{g})$ (Jang graph over $M \setminus \Sigma^*$):
\begin{enumerate}
\item $R_{\bar{g}} \geq 0$
\item $\bar{M}$ is asymptotically flat with $M_{\ADM}(\bar{g}) = M_{\ADM}(g)$
\item $\bar{M}$ has a cylindrical end at $\Sigma^*$
\end{enumerate}
\end{theorem}

\subsection{Conformal Compactification}

Let $\phi$ be a conformal factor with $\phi \sim s$ near $\Sigma^*$. Define:
\begin{equation}
\hat{g} = \phi^4 \bar{g}
\end{equation}

Then $(\hat{M}, \hat{g})$ has:
\begin{itemize}
\item Minimal boundary $\hat{\Sigma}$ with area $A(\Sigma^*)$
\item $R_{\hat{g}} \geq 0$ (if $\phi$ is chosen correctly)
\item Same ADM mass
\end{itemize}

%% ============================================================================
\section{The Direct Bound}
%% ============================================================================

\subsection{Harmonic Function on Jang Manifold}

On $(\hat{M}, \hat{g})$, let $\hat{u}$ solve:
\begin{equation}
\begin{cases}
\Delta_{\hat{g}} \hat{u} = 0 \\
\hat{u}|_{\hat{\Sigma}} = 1 \\
\hat{u} \to 0 \text{ at } \infty
\end{cases}
\end{equation}

\begin{theorem}[Mass-Capacity Inequality]
\begin{equation}
M_{\ADM} \geq \frac{\text{Cap}_{\hat{g}}(\hat{\Sigma})^2}{16\pi}
\end{equation}
\end{theorem}

\begin{proof}
Use Bray's argument on $(\hat{M}, \hat{g})$ with $R_{\hat{g}} \geq 0$.
\end{proof}

\subsection{Capacity vs Area}

\begin{lemma}[Isoperimetric Capacity Bound]
\begin{equation}
\text{Cap}(\Sigma) \geq 4\pi r_\Sigma = \sqrt{4\pi A(\Sigma)}
\end{equation}
\end{lemma}

\begin{corollary}
\begin{equation}
M_{\ADM} \geq \frac{16\pi \cdot A(\hat{\Sigma})}{16\pi} = \sqrt{\frac{A(\hat{\Sigma})}{16\pi}} = \sqrt{\frac{A(\Sigma^*)}{16\pi}}
\end{equation}
\end{corollary}

\textbf{Same result:} We get the MOTS area, not the trapped surface area.

%% ============================================================================
\section{Attempt: Direct Bound on Trapped Surface}
%% ============================================================================

\subsection{The Problem}

We need a way to relate $M_{\ADM}$ directly to $A(\Sigma_0)$ without going through the MOTS.

\subsection{Harmonic Function from Trapped Surface}

Define $u_0$ solving:
\begin{equation}
\begin{cases}
\Delta_g u_0 = 0 & \text{in } M \setminus \Sigma_0 \\
u_0|_{\Sigma_0} = 1 \\
u_0 \to 0 & \text{at } \infty
\end{cases}
\end{equation}

\begin{lemma}[Boundary Flux]
\begin{equation}
\int_{\Sigma_0} \frac{\partial u_0}{\partial \nu} dA = -\text{Cap}(\Sigma_0)
\end{equation}
\end{lemma}

\subsection{The Obstruction}

To get $M \geq \sqrt{A(\Sigma_0)/16\pi}$ directly, we'd need:
\begin{equation}
M_{\ADM} \geq \frac{A(\Sigma_0)}{16\pi}
\end{equation}
(not squared!)

But the Penrose inequality is:
\begin{equation}
M_{\ADM} \geq \sqrt{\frac{A(\Sigma_0)}{16\pi}}
\end{equation}

These differ by the square root. The capacity argument gives:
\begin{equation}
M \geq \frac{\text{Cap}^2}{16\pi} \geq \frac{4\pi A}{16\pi} = \frac{A}{4}
\end{equation}
which is \textbf{weaker} than Penrose!

%% ============================================================================
\section{The Geometric Mean Approach}
%% ============================================================================

\subsection{Idea}

Use the \textbf{geometric mean} of capacity and area:
\begin{equation}
M_{\ADM} \stackrel{?}{\geq} \sqrt{\text{Cap}(\Sigma_0) \cdot \sqrt{\frac{A(\Sigma_0)}{16\pi}}}
\end{equation}

No known theorem gives this.

\subsection{The Hawking Mass}

\begin{definition}
\begin{equation}
m_H(\Sigma) = \sqrt{\frac{A}{16\pi}}\left(1 - \frac{1}{16\pi}\int_\Sigma H^2 dA\right)
\end{equation}
\end{definition}

\begin{theorem}[Hawking Mass Bound]
On $(M, g)$ with $R \geq 0$:
\begin{equation}
M_{\ADM} \geq m_H(\Sigma) \quad \text{for minimal } \Sigma
\end{equation}
\end{theorem}

For minimal $\Sigma$ ($H = 0$): $m_H = \sqrt{A/16\pi}$, giving Penrose.

For trapped $\Sigma$ with $H < 0$: $m_H < \sqrt{A/16\pi}$, giving a \textbf{weaker} bound.

%% ============================================================================
\section{Analysis of the Obstruction}
%% ============================================================================

\subsection{Why Direct Methods Fail}

\begin{enumerate}
\item \textbf{Capacity:} Gives $M \geq \text{Cap}^2/16\pi$, not $M \geq \sqrt{A/16\pi}$
\item \textbf{Hawking mass:} Gives $M \geq m_H$, which is smaller than $\sqrt{A/16\pi}$ for trapped surfaces
\item \textbf{Harmonic functions:} The boundary conditions on trapped surfaces don't give the right inequality
\end{enumerate}

\subsection{The Fundamental Issue}

The Penrose inequality involves a \textbf{square root}:
\begin{equation}
M \geq \sqrt{\frac{A}{16\pi}} \sim A^{1/2}
\end{equation}

But elliptic methods (capacity, Dirichlet energy) give:
\begin{equation}
M \sim \text{Cap}^2 \sim A \quad \text{or} \quad M \sim m_H < A^{1/2}
\end{equation}

The flow methods (IMCF, Bray) achieve the square root by:
\begin{itemize}
\item Starting from minimal surface ($H = 0$)
\item Using monotonicity of Hawking mass along the flow
\item The square root emerges from the Geroch formula
\end{itemize}

\textbf{Key insight:} The square root comes from the \textbf{minimal surface condition} $H = 0$, which trapped surfaces violate!

%% ============================================================================
\section{A Potential New Approach}
%% ============================================================================

\subsection{The Modified Hawking Mass}

\begin{definition}[Trapping-Corrected Hawking Mass]
\begin{equation}
\tilde{m}_H(\Sigma) = \sqrt{\frac{A}{16\pi}}\left(1 - \frac{1}{16\pi}\int_\Sigma \theta^+\theta^- dA\right)
\end{equation}
\end{definition}

\begin{lemma}
For trapped surfaces: $\theta^+\theta^- > 0$, so $\tilde{m}_H < \sqrt{A/16\pi}$.

For MOTS: $\theta^+ = 0$, so $\tilde{m}_H = \sqrt{A/16\pi}$.
\end{lemma}

\begin{proposition}[Evolution under $\theta^+$-flow]
\begin{equation}
\frac{d\tilde{m}_H}{dt} = \text{(complicated expression involving } R, k, A, \nabla\theta^\pm\text{)}
\end{equation}
\end{proposition}

No clear sign, so no monotonicity.

\subsection{The $\theta$-Capacity}

\begin{definition}
\begin{equation}
\text{Cap}_\theta(\Sigma) = \inf_{u|_\Sigma = 1} \int_M w_\theta |Du|^2 dV
\end{equation}
where $w_\theta = e^{\int \theta^+/H}$ is a weight adapted to trapping.
\end{definition}

\textbf{Problem:} $H < 0$ for trapped surfaces, so $\theta^+/H > 0$, making $w_\theta > 1$, which increases capacity, not helps.

%% ============================================================================
\section{Honest Conclusion}
%% ============================================================================

\subsection{Summary of Attempts}

\begin{center}
\begin{tabular}{|l|l|l|}
\hline
\textbf{Method} & \textbf{Gives} & \textbf{Needed} \\
\hline
Capacity & $M \geq \text{Cap}^2/16\pi$ & $M \geq A^{1/2}$ \\
Hawking mass & $M \geq m_H < A^{1/2}$ & $M \geq A^{1/2}$ \\
Jang + IMCF & $M \geq A(\MOTS)^{1/2}$ & $M \geq A(\text{trapped})^{1/2}$ \\
Direct elliptic & No clean inequality & — \\
\hline
\end{tabular}
\end{center}

\subsection{The Real Obstruction}

The square root in the Penrose inequality:
\begin{equation}
M \geq \sqrt{\frac{A}{16\pi}}
\end{equation}
arises from:
\begin{enumerate}
\item The Geroch monotonicity formula for Hawking mass
\item Which requires $H = 0$ (minimal surface) as the starting point
\item Trapped surfaces have $H < 0$, breaking the argument
\end{enumerate}

\textbf{To solve 1973:} We need either:
\begin{itemize}
\item A new monotone quantity that works for $H < 0$
\item A way to ``correct'' the Hawking mass for trapping
\item A completely different approach (spinors? optimal transport?)
\end{itemize}

\subsection{Current State}

The spacetime Penrose inequality for arbitrary trapped surfaces remains \textbf{OPEN}.

All known methods give the inequality for MOTS only:
\begin{equation}
M_{\ADM} \geq \sqrt{\frac{A(\Sigma^*)}{16\pi}}
\end{equation}
where $\Sigma^*$ is the outermost MOTS.

The gap $A(\Sigma^*) \stackrel{?}{\geq} A(\Sigma_0)$ is \textbf{not provable} in general.

\begin{thebibliography}{10}
\bibitem{Bray2001} H. Bray, J. Differ. Geom. \textbf{59}, 177 (2001).
\bibitem{HuiskenIlmanen2001} G. Huisken, T. Ilmanen, J. Differ. Geom. \textbf{59}, 353 (2001).
\end{thebibliography}

\end{document}
