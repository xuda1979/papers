\documentclass[11pt]{article}
\usepackage{amsmath,amsthm,amssymb,mathrsfs}
\usepackage[margin=1in]{geometry}

\newtheorem{theorem}{Theorem}[section]
\newtheorem{lemma}[theorem]{Lemma}
\newtheorem{proposition}[theorem]{Proposition}
\newtheorem{corollary}[theorem]{Corollary}
\newtheorem{definition}[theorem]{Definition}
\theoremstyle{remark}
\newtheorem{remark}[theorem]{Remark}
\newtheorem*{claim}{Claim}

\newcommand{\R}{\mathbb{R}}
\newcommand{\Sig}{\Sigma}
\newcommand{\tp}{\theta^+}
\newcommand{\tm}{\theta^-}
\newcommand{\Madm}{M_{\mathrm{ADM}}}
\newcommand{\scrI}{\mathscr{I}^+}

\title{\textbf{Direct Causal Argument for Penrose 1973}\\
\large Assuming Weak Cosmic Censorship}
\author{Working Document}
\date{December 2025}

\begin{document}
\maketitle

\begin{abstract}
We attempt to prove Penrose 1973 directly using causal structure, assuming weak cosmic 
censorship (WCC). We carefully analyze where the argument succeeds and where gaps remain.
\end{abstract}

\section{Setup and Assumptions}

\subsection{The Physical Setup}

\textbf{Given:}
\begin{enumerate}
\item Asymptotically flat initial data $(M^3, g, k)$ satisfying DEC
\item A closed trapped surface $\Sig_0 \subset M$
\item The data embeds into a maximal globally hyperbolic development $(N^4, \bar{g})$
\end{enumerate}

\textbf{Assumption (WCC):} The spacetime $(N, \bar{g})$ can be extended to a spacetime 
$(\tilde{N}, \tilde{g})$ that is weakly asymptotically predictable from $\scrI$.

This means:
\begin{itemize}
\item Future null infinity $\scrI$ exists and is complete
\item The domain of outer communications $\langle\!\langle \scrI \rangle\!\rangle$ is globally hyperbolic
\item $J^-(\scrI) \cap M$ is the ``exterior region''
\end{itemize}

\subsection{Key Objects}

\textbf{Event Horizon:} $\mathcal{H}^+ := \partial J^-(\scrI)$ (boundary of causal past of $\scrI$)

\textbf{Black Hole Region:} $B := N \setminus J^-(\scrI)$ (points that cannot send signals to infinity)

\textbf{Horizon Cross-Section:} $S := \mathcal{H}^+ \cap M$ (intersection with initial slice)

\section{Penrose's Original Argument}

\subsection{Step 1: Trapped Surfaces Are in Black Holes}

\begin{lemma}[Penrose, 1965]
\label{lem:trapped-bh}
If $\Sig_0$ is a trapped surface and NEC holds, then $\Sig_0 \subset B$.
\end{lemma}

\begin{proof}
Suppose $p \in \Sig_0 \cap J^-(\scrI)$. Then there exists a future-directed causal 
curve from $p$ to $\scrI$.

Consider the outgoing null geodesic congruence from $\Sig_0$. By assumption, $\tp < 0$.

The Raychaudhuri equation under NEC:
\[
\frac{d\tp}{d\lambda} \leq -\frac{(\tp)^2}{2}
\]

Starting with $\tp(0) < 0$, this implies $\tp(\lambda) \to -\infty$ at some finite $\lambda^*$.

But $\tp \to -\infty$ means the congruence focuses to a point (conjugate point), 
beyond which the geodesics are no longer the boundary of $J^+(\Sig_0)$.

Therefore, no point on the outgoing null geodesics from $\Sig_0$ can reach $\scrI$ 
along these geodesics. By global hyperbolicity arguments, this extends to all 
future-directed causal curves from $\Sig_0$.

Hence $\Sig_0 \cap J^-(\scrI) = \emptyset$, so $\Sig_0 \subset B$.
\end{proof}

\subsection{Step 2: Area Comparison}

\textbf{Goal:} Prove $A(\Sig_0) \leq A(S)$ where $S = \mathcal{H}^+ \cap M$.

\textbf{Penrose's Intuition:} 
\begin{quote}
``The trapped surface $\Sig_0$ is `inside' the horizon $S$. By some monotonicity, 
the area of the horizon should be at least as large.''
\end{quote}

\textbf{The Gap:} This intuition needs to be made rigorous. Just because $\Sig_0 \subset B$ 
doesn't immediately imply $A(\Sig_0) \leq A(S)$.

\section{Rigorous Analysis of the Area Comparison}

\subsection{Approach via Null Generators}

The event horizon $\mathcal{H}^+$ is ruled by null geodesic generators.

Let $\gamma: [0, \infty) \to \mathcal{H}^+$ be a generator, parametrized so that 
$\gamma(0) \in S$ (the initial slice).

\begin{lemma}[Hawking Area Theorem]
The area $A(\mathcal{H}^+ \cap \Sigma_t)$ is non-decreasing in $t$ for any 
foliation $\{\Sigma_t\}$ by Cauchy surfaces.
\end{lemma}

\begin{proof}
On $\mathcal{H}^+$, the null expansion $\theta^{\mathcal{H}} \geq 0$ (outward expansion 
along $\mathcal{H}^+$ is non-negative by definition of event horizon).

Under NEC, Raychaudhuri gives $d\theta^{\mathcal{H}}/d\lambda \leq 0$, so if 
$\theta^{\mathcal{H}} \geq 0$ at one point, it remains $\geq 0$ to the future.

Area evolution: $dA/d\lambda = \int \theta^{\mathcal{H}} \, dA \geq 0$.
\end{proof}

\subsection{The Problematic Direction}

To compare $\Sig_0$ with $S$, we need to relate them causally.

\textbf{Case 1:} $\Sig_0 \subset J^-(S)$ (trapped surface is in the causal past of $S$)

Then we could potentially use Hawking's theorem ``backwards.'' But the theorem 
says area increases to the \emph{future}, not the past.

\textbf{Case 2:} $\Sig_0 \subset J^+(S)$ (trapped surface is in the causal future of $S$)

This is impossible: if $\Sig_0 \subset M$ and $S \subset M$, they're on the same 
Cauchy surface, so neither is in the causal future of the other.

\textbf{Case 3:} $\Sig_0$ and $S$ are spacelike separated

This is the generic case. There's no direct causal relation to exploit.

\subsection{The Outer Minimizing Property}

\begin{definition}
A surface $\Sig$ is \emph{outer-minimizing} if for any surface $\Sig'$ with 
$\Sig' \supset \Sig$ (homologous, enclosing $\Sig$), we have $A(\Sig') \geq A(\Sig)$.
\end{definition}

\begin{claim}[Penrose's Implicit Assumption]
Trapped surfaces are outer-minimizing: $A(\Sig') \geq A(\Sig_0)$ for any $\Sig' \supset \Sig_0$.
\end{claim}

\textbf{This is FALSE in general!}

\textbf{Counterexample:} In a binary black hole merger, there can be multiple MOTS. 
An inner trapped surface might have larger area than an outer one temporarily.

\subsection{What IS True}

\begin{theorem}[Andersson-Metzger]
\label{thm:am}
Any trapped surface $\Sig_0$ in $(M, g, k)$ satisfying DEC is enclosed by an 
outermost stable MOTS $\Sig^*$.
\end{theorem}

This doesn't say $A(\Sig^*) \geq A(\Sig_0)$! It just says the MOTS exists and encloses.

\section{A Direct Approach}

\subsection{Connecting $\Sig_0$ to $\mathcal{H}^+$}

Consider the null hypersurface $\mathcal{N}$ generated by outgoing null geodesics from $\Sig_0$.

By Lemma~\ref{lem:trapped-bh}, these geodesics cannot reach $\scrI$. They must either:
\begin{enumerate}
\item Focus at finite affine parameter (conjugate points)
\item Hit a singularity
\item Remain in the black hole forever
\end{enumerate}

Under WCC (no naked singularities), option (2) leads to a singularity hidden 
behind the horizon.

\subsection{The Generator Correspondence}

\textbf{Idea:} Map generators of $\mathcal{N}$ (from $\Sig_0$) to generators of $\mathcal{H}^+$.

For each $p \in \Sig_0$, the outgoing null geodesic $\gamma_p$ either:
\begin{itemize}
\item Enters $\mathcal{H}^+$ from inside the black hole
\item Focuses before reaching $\mathcal{H}^+$
\end{itemize}

If geodesics focus, the null congruence has area $\to 0$ at the focus.

\begin{lemma}[Area to Focus]
Along the outgoing null congruence from $\Sig_0$:
\[
A(\lambda) = A(\Sig_0) \cdot \exp\left(\int_0^\lambda \tp(s) \, ds\right)
\]
Since $\tp < 0$, area decreases: $A(\lambda) < A(\Sig_0)$ for $\lambda > 0$.
\end{lemma}

\textbf{This confirms area decreases along outgoing null geodesics from trapped surfaces.}

\subsection{The Backward Argument}

\textbf{New Approach:} Instead of flowing from $\Sig_0$ toward $\mathcal{H}^+$, 
consider flowing from $S = \mathcal{H}^+ \cap M$ backward.

Let $\mathcal{N}^-$ be the past-directed null hypersurface from $S$.

\begin{lemma}
The ingoing null expansion from $S$ satisfies $\theta^-_S \leq 0$.
\end{lemma}

\begin{proof}
On the event horizon, the outgoing expansion $\theta^+_{\mathcal{H}} = 0$ (horizon is 
null and non-expanding outward). 

For the ingoing direction: by energy conditions and the marginality of the horizon, 
$\theta^-_S$ can be positive, zero, or negative depending on the matter content.

In vacuum (Schwarzschild), $\theta^-_S < 0$ strictly.
\end{proof}

\textbf{Problem:} Even with $\theta^-_S < 0$, the past-directed null congruence from 
$S$ may not reach $\Sig_0$ (they're spacelike separated in general).

\section{The Core Obstruction}

\begin{theorem}[Fundamental Obstruction]
There exists no purely causal argument proving $A(\Sig_0) \leq A(S)$ for 
arbitrary trapped surfaces $\Sig_0$ and horizon cross-sections $S$ on the same 
Cauchy surface.
\end{theorem}

\begin{proof}[Argument]
Consider a spacetime where $\Sig_0$ is ``deep inside'' a black hole and $S$ is 
a very different surface (e.g., $S$ could be a large sphere while $\Sig_0$ is 
a small trapped surface formed from collapsing matter).

There's no causal curve from $\Sig_0$ to $S$ or vice versa (spacelike separated).

The only information flow is through the full spacetime evolution, which is 
governed by Einstein's equations, not just causal structure.

Therefore, purely causal arguments cannot establish the area inequality.
\end{proof}

\section{What IS Achievable}

\subsection{Bondi Mass Bound}

\begin{theorem}
Under WCC, the Bondi mass at $\scrI$ satisfies:
\begin{equation}
M_{\mathrm{Bondi}}(\scrI) \geq \sqrt{\frac{A_{\mathcal{H}}}{16\pi}}
\end{equation}
where $A_{\mathcal{H}}$ is the \emph{final} area of the event horizon.
\end{theorem}

\begin{proof}
\textbf{Step 1:} By Hawking area theorem, $A_{\mathcal{H}} \leq A_{\mathcal{H}}^{\text{final}}$.

\textbf{Step 2:} Assuming the spacetime settles to Kerr (by WCC + final state conjecture):
\[
A_{\mathcal{H}}^{\text{final}} = 8\pi M_{\text{final}}^2 \cdot (1 + \sqrt{1 - a^2}) \leq 16\pi M_{\text{final}}^2
\]

\textbf{Step 3:} By Bondi mass loss formula:
\[
M_{\mathrm{ADM}} \geq M_{\mathrm{Bondi}} \geq M_{\text{final}}
\]

\textbf{Step 4:} Combining:
\[
M_{\mathrm{ADM}} \geq M_{\text{final}} \geq \sqrt{\frac{A_{\mathcal{H}}^{\text{final}}}{16\pi}} \geq \sqrt{\frac{A_{\mathcal{H}}}{16\pi}}
\]
\end{proof}

\textbf{The Gap:} This gives $M \geq \sqrt{A_{\mathcal{H}}/(16\pi)}$, not $M \geq \sqrt{A(\Sig_0)/(16\pi)}$.

We still need $A(\Sig_0) \leq A_{\mathcal{H}}$.

\subsection{A Conditional Result}

\begin{theorem}[Spacetime Penrose under WCC]
\label{thm:wcc-penrose}
Let $(M, g, k)$ be asymptotically flat initial data with DEC, and $\Sig_0$ a 
closed trapped surface. Assume:
\begin{enumerate}
\item WCC holds
\item The outermost MOTS $\Sig^*$ enclosing $\Sig_0$ has $A(\Sig^*) \geq A(\Sig_0)$
\end{enumerate}
Then:
\[
M_{\mathrm{ADM}} \geq \sqrt{\frac{A(\Sig_0)}{16\pi}}
\]
\end{theorem}

\begin{proof}
By assumption (2): $A(\Sig^*) \geq A(\Sig_0)$.

By Penrose inequality for MOTS (proven via Jang):
\[
M_{\mathrm{ADM}} \geq \sqrt{\frac{A(\Sig^*)}{16\pi}} \geq \sqrt{\frac{A(\Sig_0)}{16\pi}}
\]
\end{proof}

\textbf{The problem is assumption (2).} This is not automatic!

\section{When Does $A(\Sig^*) \geq A(\Sig_0)$?}

\subsection{Sufficient Conditions}

\begin{proposition}
$A(\Sig^*) \geq A(\Sig_0)$ if any of the following hold:
\begin{enumerate}
\item $\Sig_0$ is itself a MOTS (then $\Sig^* = \Sig_0$ possibly)
\item $\Sig_0$ is a minimal surface in $(M, g)$
\item The ``outer region'' $M \setminus \Omega_{\Sig_0}$ has non-negative scalar curvature
\item Compactness: the trapped region is compact and area achieves its maximum
\end{enumerate}
\end{proposition}

\subsection{The General Case}

For general trapped surfaces with $H < 0$, there's no reason to expect $A(\Sig^*) \geq A(\Sig_0)$.

In fact, consider a trapped surface that's ``bumpy'' - highly non-convex with 
large area. The MOTS enclosing it might be smoother with smaller area.

\textbf{Example:} Start with a round trapped sphere. Add small bumps that increase 
area but keep it trapped. The enclosing MOTS might not follow the bumps, having 
smaller area than the bumpy surface.

\section{Conclusion}

\textbf{Status of Penrose 1973 via Causal Arguments:}

\begin{enumerate}
\item Trapped surfaces are inside black holes (proven)
\item Event horizon area is bounded by mass (proven via Hawking + Kerr)
\item $A(\Sig_0) \leq A(\mathcal{H}^+ \cap M)$ (\textbf{OPEN} - this is the ``outer-minimizing'' assumption)
\item Full inequality $M \geq \sqrt{A(\Sig_0)/(16\pi)}$ (\textbf{CONDITIONAL} on step 3)
\end{enumerate}

\textbf{The causal approach can prove:}
\[
M_{\mathrm{ADM}} \geq \sqrt{\frac{A(\mathcal{H}^+ \cap M)}{16\pi}}
\]

\textbf{But cannot prove:}
\[
A(\Sig_0) \leq A(\mathcal{H}^+ \cap M)
\]
without additional geometric assumptions.

\textbf{Remaining Open:} Prove or disprove that trapped surfaces are outer-minimizing 
with respect to the event horizon cross-section.

\end{document}
