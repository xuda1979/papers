\documentclass[11pt,a4paper]{article}
\usepackage[utf8]{inputenc}
\usepackage{amsmath,amssymb,amsthm}
\usepackage{geometry}
\usepackage{booktabs}
\usepackage{array}
\usepackage{xcolor}
\usepackage{tcolorbox}
\usepackage{tikz}

\geometry{margin=2.5cm}

\newtheorem{theorem}{Theorem}
\newtheorem{lemma}[theorem]{Lemma}
\newtheorem{proposition}[theorem]{Proposition}
\newtheorem{definition}[theorem]{Definition}
\newtheorem{remark}[theorem]{Remark}

\newcommand{\good}[1]{\textcolor{green!60!black}{#1}}
\newcommand{\bad}[1]{\textcolor{red!70!black}{#1}}
\newcommand{\neutral}[1]{\textcolor{blue!70!black}{#1}}
\newcommand{\danger}[1]{\textcolor{orange!80!black}{#1}}

\title{\textbf{手术刀分析:Boost-不变准局部质量的配平方机制}}
\author{Spacetime Penrose Inequality Program}
\date{December 2025}

\begin{document}
\maketitle

\begin{abstract}
本文档从"手术刀"视角分析 boost-不变准局部质量 $\mathcal{Q}$ 的变分公式,明确标注:(1) 必须被配平方消灭的坏项;(2) 各修正项的最小形式;(3) 危险点(MOTS/caustic/无穷远)的行为。这是证明时空 Penrose 不等式的核心技术路线图。
\end{abstract}

\tableofcontents

\section{核心手术刀:Boost-不变准局部质量}

\subsection{原始 Hawking 质量的问题}

Hawking 质量定义为:
\begin{equation}
m_H(\Sigma) = \sqrt{\frac{|\Sigma|}{16\pi}}\left(1 - \frac{1}{16\pi}\int_\Sigma \theta^+\theta^- \, dA\right)
\end{equation}

\begin{tcolorbox}[colback=red!5!white, colframe=red!75!black, title=Hawking 质量的致命缺陷]
沿 null 方向的变分:
\begin{equation}
\frac{dm_H}{ds} = \frac{\sqrt{|\Sigma|/16\pi}}{16\pi} \int_\Sigma \left[\mu - |J| \cdot (\text{something}) - \bad{2\sigma^+:\sigma^-} - \bad{|\zeta|^2}\right] dA
\end{equation}

\textbf{坏项分析}:
\begin{itemize}
\item $\bad{-2\sigma^+:\sigma^-}$:符号不定!$\sigma^+:\sigma^- = \text{tr}(\sigma^+_{ab}\sigma^{-ab})$ 可正可负
\item $\bad{-|\zeta|^2}$:负定,直接破坏单调性
\end{itemize}

\textbf{结论}:Hawking 质量沿 null 方向\textbf{不单调},无法直接用于 Penrose 不等式。
\end{tcolorbox}

\subsection{手术刀 \#1:平方完成恒等式}

\begin{tcolorbox}[colback=green!5!white, colframe=green!60!black, title=核心代数恒等式]
对于任意两个对称无迹张量 $\sigma^+, \sigma^-$:
\begin{equation}
\boxed{-\sigma^+:\sigma^- = -\frac{1}{4}|\sigma^+ + \sigma^-|^2 + \frac{1}{4}|\sigma^+ - \sigma^-|^2}
\end{equation}

\textbf{效果}:
\begin{itemize}
\item 左边:符号不定的双线性项
\item 右边:两个平方项的差,各自符号确定
\end{itemize}
\end{tcolorbox}

\subsection{手术刀 \#2:Boost 不变性要求}

在 null 框架 $(\ell, n)$ 下,boost 变换为:
\begin{equation}
\ell \mapsto \lambda \ell, \quad n \mapsto \lambda^{-1} n
\end{equation}

各量的变换性质:
\begin{center}
\begin{tabular}{c|c|c}
\toprule
\textbf{量} & \textbf{变换} & \textbf{boost 权重} \\
\midrule
$\theta^+$ & $\lambda \theta^+$ & $+1$ \\
$\theta^-$ & $\lambda^{-1} \theta^-$ & $-1$ \\
$\sigma^+$ & $\lambda \sigma^+$ & $+1$ \\
$\sigma^-$ & $\lambda^{-1} \sigma^-$ & $-1$ \\
$\theta^+\theta^-$ & 不变 & $0$ \\
$\sigma^+:\sigma^-$ & 不变 & $0$ \\
$\sigma^+/\theta^+$ & 不变 & $0$ \\
\bottomrule
\end{tabular}
\end{center}

\begin{tcolorbox}[colback=blue!5!white, colframe=blue!60!black, title=Boost 不变性的代价]
要构造 boost 不变的组合,必须用 $\theta^\pm$ 做"归一化":
\begin{equation}
\frac{\sigma^+}{\theta^+} - \frac{\sigma^-}{\theta^-} \quad \text{(boost 不变)}
\end{equation}

\textbf{代价}:当 $\theta^+ \to 0$(MOTS)或 $\theta^- \to 0$ 时,这个组合\textbf{发散}!
\end{tcolorbox}

\section{修正后的质量泛函 $\mathcal{Q}$}

\subsection{定义}

\begin{definition}[Boost-不变准局部质量]
\begin{equation}
\mathcal{Q}(\Sigma) = \sqrt{\frac{|\Sigma|}{16\pi}}\left(1 - \frac{1}{16\pi}\int_\Sigma \left[\theta^+\theta^- + |\zeta|^2 + \frac{1}{4}\left|\frac{\sigma^+}{\theta^+} - \frac{\sigma^-}{\theta^-}\right|^2\theta^+\theta^-\right] dA\right)
\end{equation}
\end{definition}

\subsection{变分公式(沿 outgoing null 方向)}

\begin{theorem}[Q 的单调性]
沿满足 DEC 的时空中的 outgoing null 方向:
\begin{equation}
\frac{d\mathcal{Q}}{ds} = \frac{\sqrt{|\Sigma|/16\pi}}{16\pi} \int_\Sigma \Phi \, dA
\end{equation}
其中被积函数 $\Phi$ 分解为:
\begin{align}
\Phi &= \underbrace{\good{(\mu - |J|) \cdot (\text{正系数})}}_{\text{DEC 项:} \geq 0} \\
&\quad + \underbrace{\good{\frac{1}{4}|\sigma^+ - \sigma^-|^2 \cdot (\text{正系数})}}_{\text{配平方好项:} \geq 0} \\
&\quad + \underbrace{\neutral{(\text{边界/渐近项})}}_{\text{需要 caustic 手术处理}}
\end{align}
\end{theorem}

\section{坏项-配平方-危险点对照表}

\begin{center}
\renewcommand{\arraystretch}{1.8}
\begin{tabular}{|>{\raggedright}p{3cm}|>{\raggedright}p{4cm}|>{\raggedright}p{4cm}|>{\raggedright\arraybackslash}p{3.5cm}|}
\hline
\textbf{坏项} & \textbf{配平方机制} & \textbf{最小修正项} & \textbf{危险点} \\
\hline
\hline
$\bad{-2\sigma^+:\sigma^-}$ \newline (符号不定的剪切耦合) & 
$-\sigma^+:\sigma^- = -\frac{1}{4}|\sigma^++\sigma^-|^2 + \frac{1}{4}|\sigma^+-\sigma^-|^2$ &
添加 $+\frac{1}{4}|\sigma^++\sigma^-|^2$ 项消掉负平方 &
\danger{MOTS}:需要 $\theta^+$ 归一化 \\
\hline
$\bad{-|\zeta|^2}$ \newline (扭转的负贡献) &
直接吸收进修正项 &
$+|\zeta|^2$ 抵消 &
\danger{Caustic}:$\zeta$ 可能发散 \\
\hline
$\bad{\sigma^+/\theta^+}$ 发散 \newline (MOTS 奇点) &
无法配平方消除!\newline 这是\textbf{本质奇点} &
必须用\textbf{跳跃}:在 $|\theta^+| < \delta$ 前跳到 outer hull &
\danger{MOTS}:$\theta^+ = 0$ \\
\hline
$\bad{\theta^+ \to -\infty}$ \newline (Caustic 发散) &
无法配平方消除!\newline 这是\textbf{几何奇点} &
必须用\textbf{跳跃}:Huisken-Ilmanen 式的 outer hull 手术 &
\danger{Caustic}:共轭点 \\
\hline
$\neutral{\text{边界项}}$ \newline (渐近行为) &
$\mathcal{Q} \to M_B$ 需要渐近分析 &
Bondi 坐标展开 + 衰减估计 &
\danger{$\mathscr{I}^+$}:无穷远 \\
\hline
\end{tabular}
\end{center}

\section{两类奇点的统一处理:MOTS-回避弱 null 流}

\subsection{核心洞察}

\begin{tcolorbox}[colback=yellow!5!white, colframe=yellow!75!black, title=Gap 1+2 的统一]
原本以为是两个独立问题:
\begin{itemize}
\item \textbf{Gap 1}:Caustic($\theta^+ \to -\infty$)
\item \textbf{Gap 2}:MOTS 穿越($\theta^+ \to 0$)
\end{itemize}

\textbf{统一洞察}:两者都是 $\mathcal{Q}$ 中 $\theta^+$ 出现在分母的后果。解决方案相同:
\begin{center}
\fbox{\textbf{在到达奇点之前跳跃到 outer hull}}
\end{center}
\end{tcolorbox}

\subsection{MOTS-回避弱 null 流的定义}

\begin{definition}[MOTS-Avoiding Weak Null Flow]
从陷获面 $\Sigma$ 出发的 \textbf{MOTS-回避弱 null 流} $\{\Sigma_s\}_{s \geq 0}$ 满足:

\begin{enumerate}
\item[(WA1)] \textbf{初始}:$\Sigma_0 = \Sigma$,且 $\theta^+(\Sigma_0) < 0$(陷获);

\item[(WA2)] \textbf{光滑段}:在跳跃时间之外,$\Sigma_s$ 沿 outgoing null 方向光滑演化,且 $|\theta^\pm| \geq \delta > 0$;

\item[(WA3)] \textbf{Caustic 跳跃}:当 $\theta^+ \to -\infty$ 时,跳跃到 outward minimizing hull;

\item[(WA4)] \textbf{MOTS-接近跳跃}:当 $|\theta^+| < \delta$ 时,跳跃到 outward minimizing hull;

\item[(WA5)] \textbf{终点}:$\Sigma_s \to \mathscr{I}^+$(null 无穷远)。
\end{enumerate}
\end{definition}

\subsection{关键引理:跳跃处的单调性}

\begin{tcolorbox}[colback=red!5!white, colframe=red!75!black, title=核心开放问题]
\begin{conjecture}[跳跃单调性]
设 $\Sigma^-$ 是跳跃前的曲面,$\Sigma^+$ 是跳跃后的 outer hull。则:
\begin{equation}
\mathcal{Q}(\Sigma^+) \geq \mathcal{Q}(\Sigma^-)
\end{equation}
\end{conjecture}

\textbf{困难}:
\begin{itemize}
\item Outer hull 的定义需要 Lorentzian 几何测度论
\item $\mathcal{Q}$ 在 $\Sigma^-$(接近 MOTS 或 caustic)可能发散
\item 需要证明"望远镜式误差吸收"
\end{itemize}
\end{tcolorbox}

\section{变分公式的完整展开}

\subsection{Raychaudhuri 方程}

沿 outgoing null 方向 $\ell$:
\begin{align}
\frac{d\theta^+}{ds} &= -\frac{1}{2}(\theta^+)^2 - |\sigma^+|^2 - R_{\mu\nu}\ell^\mu\ell^\nu \\
&= -\frac{1}{2}(\theta^+)^2 - |\sigma^+|^2 - 8\pi(\mu - J \cdot \ell)
\end{align}

沿 ingoing null 方向 $n$:
\begin{equation}
\frac{d\theta^-}{ds} = -\frac{1}{2}(\theta^-)^2 - |\sigma^-|^2 - 8\pi(\mu - J \cdot n)
\end{equation}

\subsection{$\theta^+\theta^-$ 的演化}

\begin{align}
\frac{d(\theta^+\theta^-)}{ds} &= \theta^- \frac{d\theta^+}{ds} + \theta^+ \frac{d\theta^-}{ds} \\
&= -\frac{1}{2}\theta^-(\theta^+)^2 - \theta^-|\sigma^+|^2 - 8\pi\theta^-(\mu - J\cdot\ell) \\
&\quad - \frac{1}{2}\theta^+(\theta^-)^2 - \theta^+|\sigma^-|^2 - 8\pi\theta^+(\mu - J\cdot n)
\end{align}

\subsection{剪切项的处理}

原始坏项:
\begin{equation}
-\theta^-|\sigma^+|^2 - \theta^+|\sigma^-|^2 - 2\sigma^+:\sigma^-
\end{equation}

应用平方完成:
\begin{align}
&-\theta^-|\sigma^+|^2 - \theta^+|\sigma^-|^2 - 2\sigma^+:\sigma^- \\
&= -\theta^-|\sigma^+|^2 - \theta^+|\sigma^-|^2 + \frac{1}{2}|\sigma^+ + \sigma^-|^2 - \frac{1}{2}|\sigma^+ - \sigma^-|^2
\end{align}

\begin{tcolorbox}[colback=green!5!white, colframe=green!60!black, title=配平方后的符号分析]
定义 $\Delta\sigma = \sigma^+ - \sigma^-$(boost 权重 $+1 - (-1) = +2$,不是 boost 不变)。

为了得到 boost 不变的组合:
\begin{equation}
\frac{\sigma^+}{\theta^+} - \frac{\sigma^-}{\theta^-} \quad \text{(boost 不变)}
\end{equation}

则:
\begin{equation}
\left|\frac{\sigma^+}{\theta^+} - \frac{\sigma^-}{\theta^-}\right|^2 \theta^+\theta^- = \frac{|\sigma^+\theta^- - \sigma^-\theta^+|^2}{\theta^+\theta^-}
\end{equation}

\textbf{符号}:当 $\theta^+\theta^- < 0$(untrapped 区域),这一项为\textbf{负}!

\textbf{但是}:在 trapped 区域 $\theta^+\theta^- > 0$,这一项为\textbf{正}。

这就是为什么我们需要从 trapped surface 出发!
\end{tcolorbox}

\section{危险点的详细分析}

\subsection{MOTS ($\theta^+ = 0$)}

当 $\theta^+ \to 0^-$:
\begin{itemize}
\item $\sigma^+/\theta^+ \to \pm\infty$(除非 $\sigma^+ = 0$)
\item $\mathcal{Q}$ 中的修正项 $\left|\frac{\sigma^+}{\theta^+} - \frac{\sigma^-}{\theta^-}\right|^2\theta^+\theta^- \to -\infty$(因为 $\theta^- < 0$)
\item 这是\textbf{本质奇点},不可能通过重新定义 $\mathcal{Q}$ 来消除
\end{itemize}

\textbf{手术方案}:在 $|\theta^+| < \delta$ 时跳跃。

\subsection{Caustic ($\theta^+ \to -\infty$)}

当光线聚焦形成 caustic:
\begin{itemize}
\item $\theta^+ \to -\infty$
\item 曲面退化(面积 $\to 0$)
\item $\mathcal{Q}$ 的定义域失效
\end{itemize}

\textbf{手术方案}:Huisken-Ilmanen 式的 outer hull 跳跃。

\subsection{Null Infinity ($\mathscr{I}^+$)}

渐近行为:
\begin{itemize}
\item $|\Sigma_r| \sim 4\pi r^2$
\item $\theta^+ \sim 2/r$, $\theta^- \sim -1/r$
\item $\sigma^\pm \sim O(r^{-2})$(news function 衰减)
\item $\zeta \sim O(r^{-2})$
\end{itemize}

\textbf{结论}:$\mathcal{Q}(\Sigma_r) = M_B + O(r^{-1})$,趋近于 Bondi 质量。

\section{条件主定理}

\begin{tcolorbox}[colback=blue!5!white, colframe=blue!75!black, title=\textbf{Main Conditional Theorem}]
\begin{theorem}[时空 Penrose 不等式——条件版]
设 $(M^4, g)$ 是满足 DEC 的全局双曲、渐近平坦时空,Bondi 质量为 $M_B$,$\Sigma$ 是球面拓扑的闭合最外层陷获面。

\textbf{若}存在 MOTS-回避弱 null 流 $\{\Sigma_s\}_{s \in [0,\infty)}$ 满足:
\begin{enumerate}
\item[(H1)] 条件 (WA1)--(WA5);
\item[(H2)] 跳跃单调性:在每个跳跃处 $\mathcal{Q}(\Sigma^+) \geq \mathcal{Q}(\Sigma^-)$;
\end{enumerate}
\textbf{则}:
\begin{equation}
\boxed{M_B \geq \sqrt{\frac{|\Sigma|}{16\pi}}}
\end{equation}
\end{theorem}
\end{tcolorbox}

\begin{proof}[证明概要]
\begin{enumerate}
\item \textbf{初始值}:$\mathcal{Q}(\Sigma_0) = \sqrt{|\Sigma|/16\pi}$(对于 outermost trapped surface)

\item \textbf{光滑段单调}:由 DEC + 配平方机制,$d\mathcal{Q}/ds \geq 0$

\item \textbf{跳跃单调}:由假设 (H2),$\mathcal{Q}$ 在跳跃处不减

\item \textbf{渐近极限}:$\lim_{s\to\infty} \mathcal{Q}(\Sigma_s) = M_B$

\item \textbf{结论}:$M_B \geq \mathcal{Q}(\Sigma_0) = \sqrt{|\Sigma|/16\pi}$
\end{enumerate}
\end{proof}

\section{开放问题清单}

\begin{enumerate}
\item \textbf{弱 null 流存在性}:是否总存在满足 (WA1)--(WA5) 的流?

\item \textbf{Outer hull 的 Lorentzian 定义}:如何在 null 超曲面上定义"outward minimizing hull"?

\item \textbf{跳跃单调性}:如何证明 $\mathcal{Q}(\Sigma^+) \geq \mathcal{Q}(\Sigma^-)$?

\item \textbf{球面拓扑的保持}:流是否保持拓扑?

\item \textbf{刚性}:等号成立是否蕴含 Schwarzschild?
\end{enumerate}

\section{与王虹"手术刀"的类比}

\begin{center}
\renewcommand{\arraystretch}{1.5}
\begin{tabular}{|>{\raggedright}p{5cm}|>{\raggedright}p{5cm}|>{\raggedright\arraybackslash}p{5cm}|}
\hline
\textbf{方面} & \textbf{Kakeya 问题(王虹)} & \textbf{Penrose 1973(我们)} \\
\hline
\hline
核心坏项 & 多尺度/多线性的 Kakeya 构型 & 符号不定的剪切耦合 $\sigma^+:\sigma^-$ \\
\hline
手术刀 & 精细化调和分析(多尺度分解 + 正交性) & 平方完成 + boost 不变归一化 \\
\hline
切割机制 & 把坏构型分解成可控的小块 & 把坏符号项变成平方项 \\
\hline
几何奇点 & 无 & Caustic, MOTS \\
\hline
配套手术 & 无 & 弱 null 流 + outer hull 跳跃 \\
\hline
尖锐常数 & 精细估计闭合 & DEC + 配平方自动闭合 \\
\hline
开放问题 & 已解决 & 跳跃单调性、流存在性 \\
\hline
\end{tabular}
\end{center}

\section{结论}

\begin{tcolorbox}[colback=green!5!white, colframe=green!60!black, title=两把手术刀]
\textbf{第一把刀}:\textbf{Boost-不变准局部质量 $\mathcal{Q}$ 的平方完成机制}
\begin{itemize}
\item 把 $-\sigma^+:\sigma^-$ 变成 $+\frac{1}{4}|\sigma^+-\sigma^-|^2 - \frac{1}{4}|\sigma^++\sigma^-|^2$
\item DEC 提供 $(\mu - |J|) \geq 0$
\item 尖锐常数自动闭合
\end{itemize}

\textbf{第二把刀}:\textbf{弱 null 流的 caustic/MOTS 手术}
\begin{itemize}
\item 统一处理 $\theta^+ \to -\infty$(caustic)和 $\theta^+ \to 0$(MOTS)
\item 在奇点前跳跃到 outer hull
\item 需要证明跳跃单调性(核心开放问题)
\end{itemize}
\end{tcolorbox}

\end{document}
