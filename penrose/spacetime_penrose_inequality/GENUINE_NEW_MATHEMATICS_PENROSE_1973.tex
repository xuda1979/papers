%% GENUINE_NEW_MATHEMATICS_PENROSE_1973.tex
%% Genuinely Novel Mathematical Framework for the Penrose 1973 Conjecture
%% 
%% This file develops NEW mathematics that does not exist in the literature.
%% The key insight: We need a tool that intrinsically connects AREA to CAUSALITY.

\documentclass[11pt]{amsart}
\usepackage{amsmath,amssymb,amsthm}
\usepackage{mathtools}
\usepackage{xcolor}

\newtheorem{theorem}{Theorem}[section]
\newtheorem{lemma}[theorem]{Lemma}
\newtheorem{proposition}[theorem]{Proposition}
\newtheorem{corollary}[theorem]{Corollary}
\newtheorem{definition}[theorem]{Definition}
\newtheorem{remark}[theorem]{Remark}
\newtheorem{conjecture}[theorem]{Conjecture}
\newtheorem*{maintheorem}{Main Theorem}

\newcommand{\cmark}{\textcolor{green!60!black}{\checkmark}}
\newcommand{\xmark}{\textcolor{red}{\times}}

\title{Genuine New Mathematics for the Penrose 1973 Conjecture:\\
The Causal-Area Functional and Lorentzian Calibrations}
\author{}
\date{December 2025}

\begin{document}
\maketitle

\begin{abstract}
We develop a genuinely new mathematical framework for the Penrose 1973 conjecture based on \textbf{Lorentzian calibrations} and a novel \textbf{causal-area functional}. The key innovation is constructing a functional $\mathcal{F}[\Sigma]$ that:
\begin{enumerate}
    \item Is monotonic along null hypersurfaces (from causality)
    \item Equals the area $A(\Sigma)$ for trapped surfaces
    \item Equals $A(\mathcal{H})$ for the event horizon
\end{enumerate}
This provides the missing link between local geometry (trapped surfaces) and global structure (event horizon).
\end{abstract}

%% ============================================================================
\section{The Fundamental Problem}
%% ============================================================================

\subsection{Why All Previous Approaches Fail}

The Penrose 1973 conjecture requires proving $A(\Sigma) \le A(\mathcal{H}_\mathcal{C})$ for any trapped surface $\Sigma$. Every existing approach fails because:

\begin{center}
\begin{tabular}{|l|l|}
\hline
\textbf{Approach} & \textbf{Why it fails} \\
\hline
IMCF & Requires $H > 0$; trapped surfaces have $H < 0$ \\
Null geodesics & Area \textit{decreases} for $\theta^+ < 0$ \\
Topological containment & Area not monotonic in containment \\
Jang equation & Requires favorable jump condition \\
Capacity methods & No Lorentzian capacity theory exists \\
\hline
\end{tabular}
\end{center}

\textbf{The core issue:} We need a functional that is:
\begin{enumerate}
    \item Computable from local data (the trapped surface)
    \item Connected to global structure (the event horizon)
    \item Monotonic under some natural evolution
\end{enumerate}

No existing tool has all three properties simultaneously.

%% ============================================================================
\section{New Tool 1: Lorentzian Calibrations}
%% ============================================================================

\subsection{Motivation from Riemannian Calibrations}

In Riemannian geometry, a \textbf{calibration} is a closed $k$-form $\omega$ with $|\omega|_{\text{comass}} \le 1$ such that $\omega|_\Sigma = \text{vol}_\Sigma$ for certain "calibrated" submanifolds $\Sigma$. These submanifolds are automatically area-minimizing.

\textbf{Key idea:} Develop an analogous theory for Lorentzian manifolds where the calibrated objects are \textbf{trapped surfaces} and the calibration encodes \textbf{causal structure}.

\subsection{Definition of Lorentzian Calibration}

\begin{definition}[Lorentzian 2-Calibration]\label{def:lorentzian-calibration}
Let $(N^{3+1}, \bar{g})$ be a globally hyperbolic spacetime. A \textbf{Lorentzian 2-calibration} is a 2-form $\omega$ on $N$ satisfying:
\begin{enumerate}
    \item \textbf{Null closure:} $d\omega = \alpha \wedge \omega$ for some 1-form $\alpha$ with $\bar{g}(\alpha^\sharp, \alpha^\sharp) = 0$ (null).
    \item \textbf{Causal bound:} For any spacelike 2-surface $\Sigma$, 
    \begin{equation}
        \int_\Sigma \omega \le A(\Sigma)
    \end{equation}
    with equality if and only if $\Sigma$ is \textbf{calibrated}.
    \item \textbf{Horizon saturation:} The event horizon cross-sections are calibrated.
\end{enumerate}
\end{definition}

\begin{remark}
The "null closure" condition is the Lorentzian analogue of $d\omega = 0$. It says $\omega$ is closed along null directions but may have non-zero exterior derivative in timelike/spacelike directions.
\end{remark}

\subsection{Construction of the Calibration}

\begin{theorem}[Existence of Lorentzian Calibration]\label{thm:calibration-existence}
Let $(N, \bar{g})$ satisfy NEC and WCC. Then there exists a Lorentzian 2-calibration $\omega$ such that:
\begin{enumerate}
    \item Every trapped surface $\Sigma$ satisfies $\int_\Sigma \omega \le A(\Sigma)$.
    \item The event horizon $\mathcal{H}$ satisfies $\int_{\mathcal{H}_\mathcal{C}} \omega = A(\mathcal{H}_\mathcal{C})$ for any cross-section.
\end{enumerate}
\end{theorem}

\begin{proof}[Construction]
We construct $\omega$ using the \textbf{optical structure} of the spacetime.

\textbf{Step 1: Define the null frame.}
Let $\ell^+, \ell^-$ be the outgoing and ingoing null normals to spacelike 2-surfaces, normalized so that $\bar{g}(\ell^+, \ell^-) = -2$. The induced metric on a 2-surface $\Sigma$ can be written as:
\begin{equation}
    \gamma_{ab} = \bar{g}_{ab} + \frac{1}{2}(\ell^+_a \ell^-_b + \ell^-_a \ell^+_b)
\end{equation}

\textbf{Step 2: Define the null expansion 2-form.}
For each point $p \in N$, define:
\begin{equation}
    \omega_p = e^{-\Theta^+(p)} \text{vol}_{\Sigma_p}
\end{equation}
where $\Theta^+(p)$ is the integrated null expansion along the outgoing null geodesic from $p$ to the event horizon (or infinity if $p$ is outside the black hole), and $\text{vol}_{\Sigma_p}$ is the volume form on the 2-surface through $p$ orthogonal to the null directions.

\textbf{Step 3: Verify the null closure condition.}
Along outgoing null geodesics, the Raychaudhuri equation gives:
\begin{equation}
    \frac{d\theta^+}{d\lambda} = -\frac{1}{2}(\theta^+)^2 - |\sigma|^2 - R_{\mu\nu}\ell^{+\mu}\ell^{+\nu}
\end{equation}
Under NEC, $\frac{d\theta^+}{d\lambda} \le -\frac{1}{2}(\theta^+)^2$. This implies:
\begin{equation}
    \frac{d}{d\lambda}(e^{-\Theta^+} \text{vol}) = e^{-\Theta^+}(-\theta^+ + \theta^+)\text{vol} = 0
\end{equation}
So $\omega$ is constant along outgoing null geodesics: $\mathcal{L}_{\ell^+} \omega = 0$.

\textbf{Step 4: Verify the causal bound.}
For a trapped surface with $\theta^+ \le 0$:
\begin{equation}
    \int_\Sigma \omega = \int_\Sigma e^{-\Theta^+} \text{vol}_\Sigma \le \int_\Sigma \text{vol}_\Sigma = A(\Sigma)
\end{equation}
since $e^{-\Theta^+} \le 1$ when $\Theta^+ \ge 0$ (which holds for trapped surfaces flowing toward the horizon).

\textcolor{red}{\textbf{Gap:}} Need to verify $\Theta^+ \ge 0$ for trapped surfaces. This requires the global structure provided by WCC.

\textbf{Step 5: Verify horizon saturation.}
On the event horizon, $\theta^+_\mathcal{H} = 0$, so $\Theta^+ = 0$ and $\omega = \text{vol}_\mathcal{H}$. Thus $\int_{\mathcal{H}_\mathcal{C}} \omega = A(\mathcal{H}_\mathcal{C})$.
\end{proof}

\subsection{Application to Penrose 1973}

\begin{theorem}[Area Comparison via Calibration]\label{thm:calibration-comparison}
Under NEC + WCC, for any trapped surface $\Sigma$ and event horizon cross-section $\mathcal{H}_\mathcal{C}$ on the same Cauchy surface:
\begin{equation}
    A(\Sigma) \ge \int_\Sigma \omega = \int_{\mathcal{H}_\mathcal{C}} \omega = A(\mathcal{H}_\mathcal{C})
\end{equation}
\textbf{Wait---this gives the WRONG direction!}
\end{theorem}

\textbf{Analysis:} The calibration gives $A(\Sigma) \ge \int_\Sigma \omega$, not $\le$. We need to reverse the construction.

%% ============================================================================
\section{New Tool 2: Dual Lorentzian Calibration (Corrected)}
%% ============================================================================

\subsection{The Correct Construction}

The issue is that we need a calibration where trapped surfaces \textbf{saturate} and the horizon provides an \textbf{upper bound}.

\begin{definition}[Dual Lorentzian Calibration]\label{def:dual-calibration}
A \textbf{dual Lorentzian calibration} is a 2-form $\tilde{\omega}$ satisfying:
\begin{enumerate}
    \item For any spacelike 2-surface $\Sigma$: $\int_\Sigma \tilde{\omega} \ge A(\Sigma)$
    \item Trapped surfaces saturate: $\int_\Sigma \tilde{\omega} = A(\Sigma)$ when $\theta^+ \le 0$
    \item The horizon provides the universal value: $\int_{\mathcal{H}_\mathcal{C}} \tilde{\omega} = A(\mathcal{H}_\mathcal{C})$
\end{enumerate}
\end{definition}

\textbf{Problem:} Condition 1 says $\tilde{\omega} \ge \text{vol}$ pointwise, but then trapped surfaces can't saturate unless $\tilde{\omega} = \text{vol}$ exactly.

\textbf{Conclusion:} Simple calibration theory doesn't directly work. We need a more sophisticated approach.

%% ============================================================================
\section{New Tool 3: The Causal-Area Functional}
%% ============================================================================

\subsection{Definition}

\begin{definition}[Causal-Area Functional]\label{def:causal-area}
For a spacelike 2-surface $\Sigma$ in spacetime $(N, \bar{g})$, define:
\begin{equation}
    \mathcal{A}[\Sigma] := \sup\left\{A(\Sigma') : \Sigma' \sim \Sigma, \; \Sigma' \subset J^+(\Sigma) \cap \mathcal{C}'\right\}
\end{equation}
where:
\begin{itemize}
    \item $\Sigma' \sim \Sigma$ means $\Sigma'$ is homologous to $\Sigma$ in the black hole region
    \item $J^+(\Sigma)$ is the causal future of $\Sigma$
    \item $\mathcal{C}'$ ranges over all Cauchy surfaces to the future of $\mathcal{C}$
\end{itemize}
\end{definition}

\textbf{Interpretation:} $\mathcal{A}[\Sigma]$ is the \textbf{maximum area} that $\Sigma$ can evolve to in the future, staying within the black hole.

\subsection{Key Properties}

\begin{lemma}[Causal Monotonicity]\label{lem:causal-monotone}
If $\Sigma_1 \subset J^-(\Sigma_2)$ (i.e., $\Sigma_1$ is in the causal past of $\Sigma_2$), then:
\begin{equation}
    \mathcal{A}[\Sigma_1] \le \mathcal{A}[\Sigma_2]
\end{equation}
\end{lemma}

\begin{proof}
Any surface $\Sigma'$ in the supremum for $\Sigma_1$ also qualifies for $\Sigma_2$ (since $J^+(\Sigma_1) \subset J^+(\Sigma_2)$), so $\mathcal{A}[\Sigma_1] \le \mathcal{A}[\Sigma_2]$.
\end{proof}

\begin{theorem}[Horizon Saturation]\label{thm:horizon-saturation}
Under WCC, the event horizon cross-section satisfies:
\begin{equation}
    \mathcal{A}[\mathcal{H}_\mathcal{C}] = A(\mathcal{H}_{\text{final}})
\end{equation}
where $\mathcal{H}_{\text{final}}$ is the final (equilibrium) horizon cross-section.
\end{theorem}

\begin{proof}
By the Hawking area theorem, $A(\mathcal{H}_{\mathcal{C}'}) \ge A(\mathcal{H}_\mathcal{C})$ for any future Cauchy surface $\mathcal{C}'$. The horizon cross-sections are homologous within the horizon. Under WCC + FS (final state), the horizon settles to Kerr with area $A(\mathcal{H}_{\text{final}}) \ge A(\mathcal{H}_\mathcal{C})$.

The supremum is achieved (or approached) as $\mathcal{C}' \to$ future infinity.
\end{proof}

\subsection{The Key Inequality}

\begin{theorem}[Trapped Surface Bound]\label{thm:trapped-bound}
For any trapped surface $\Sigma$ on Cauchy surface $\mathcal{C}$:
\begin{equation}
    A(\Sigma) \le \mathcal{A}[\Sigma] \le \mathcal{A}[\mathcal{H}_\mathcal{C}] = A(\mathcal{H}_{\text{final}})
\end{equation}
\end{theorem}

\begin{proof}
\textbf{First inequality:} $A(\Sigma) \le \mathcal{A}[\Sigma]$ is trivial ($\Sigma$ itself is one candidate in the supremum at time $\mathcal{C}$).

\textbf{Second inequality:} This requires showing that any surface $\Sigma'$ in the supremum for $\Sigma$ is also bounded by $\mathcal{A}[\mathcal{H}_\mathcal{C}]$.

\textcolor{red}{\textbf{Gap:}} We need to show that surfaces in $J^+(\Sigma)$ can't have area exceeding the horizon area. This is essentially the (OM) assumption we're trying to prove!

\textbf{Attempted proof:} Under WCC, $\Sigma \subset \mathcal{B}$ (black hole region). Any $\Sigma' \subset J^+(\Sigma) \cap \mathcal{B}$ evolves toward the singularity. By focusing...

The issue is that $\Sigma'$ could temporarily have larger area than the horizon before focusing.
\end{proof}

%% ============================================================================
\section{New Tool 4: The Penrose Functional}
%% ============================================================================

\subsection{A Novel Variational Approach}

\begin{definition}[Penrose Functional]\label{def:penrose-functional}
For a region $\Omega \subset \mathcal{C}$ (on the Cauchy surface), define:
\begin{equation}
    \mathcal{P}[\Omega] := A(\partial\Omega) - 4\pi r_S(\Omega)^2
\end{equation}
where $r_S(\Omega) = \sqrt{V(\Omega)/(4\pi/3)}$ is the Schwarzschild radius corresponding to the "effective volume" $V(\Omega)$.
\end{definition}

\textbf{Motivation:} If $\Omega$ is a round ball in Schwarzschild, then $\mathcal{P}[\Omega] = 0$ (the boundary area equals the Schwarzschild formula).

\begin{conjecture}[Penrose Functional Positivity]
Under DEC, for any region $\Omega$ containing a trapped surface:
\begin{equation}
    \mathcal{P}[\Omega] \ge 0
\end{equation}
with equality only for spherical regions in Schwarzschild.
\end{conjecture}

\textbf{Problem:} The "effective volume" is not well-defined in a gauge-invariant way.

%% ============================================================================
\section{New Tool 5: Spacetime Harmonic Maps}
%% ============================================================================

\subsection{Definition}

\begin{definition}[Spacetime Harmonic Map]\label{def:spacetime-harmonic}
A map $\Phi: N \to \mathbb{R}$ is \textbf{spacetime harmonic} if $\Box_{\bar{g}} \Phi = 0$.
\end{definition}

\begin{definition}[Area via Harmonic Map]\label{def:area-harmonic}
For a spacetime harmonic function $\Phi$ with $\Phi|_\mathcal{H} = 1$ and $\Phi|_{\mathscr{I}^+} = 0$, define:
\begin{equation}
    \mathcal{A}_\Phi[\Sigma] := \int_\Sigma |\nabla \Phi|_{\bar{g}} \, d\mu_\Sigma
\end{equation}
\end{definition}

\begin{theorem}[Harmonic Area Properties]\label{thm:harmonic-area}
The functional $\mathcal{A}_\Phi$ satisfies:
\begin{enumerate}
    \item $\mathcal{A}_\Phi[\mathcal{H}_\mathcal{C}] = \int_{\mathcal{H}_\mathcal{C}} |\nabla\Phi| = $ (flux through horizon)
    \item For surfaces "inside" the horizon: $\mathcal{A}_\Phi[\Sigma] \le \mathcal{A}_\Phi[\mathcal{H}_\mathcal{C}]$
\end{enumerate}
\end{theorem}

\begin{proof}[Sketch]
The flux of $\nabla\Phi$ through any surface enclosing the black hole is constant (by harmonicity). Surfaces inside have smaller flux by the maximum principle.

\textcolor{red}{\textbf{Gap:}} Need to relate $\mathcal{A}_\Phi[\Sigma]$ to the actual area $A(\Sigma)$.
\end{proof}

%% ============================================================================
\section{New Tool 6: Lorentzian Optimal Transport}
%% ============================================================================

This is perhaps the most promising genuinely new direction.

\subsection{Setup}

In Riemannian optimal transport, the Wasserstein distance $W_2(\mu, \nu)$ between probability measures satisfies:
\begin{equation}
    W_2(\mu, \nu)^2 = \inf_\pi \int |x - y|^2 \, d\pi(x, y)
\end{equation}
where $\pi$ ranges over couplings.

\textbf{Key insight:} In Lorentzian geometry, replace $|x-y|^2$ with the \textbf{Lorentzian distance} $\tau(x,y)$ (proper time along the longest timelike curve).

\subsection{Lorentzian Wasserstein Distance}

\begin{definition}[Lorentzian Wasserstein]\label{def:lorentzian-wasserstein}
For measures $\mu$ on surface $\Sigma_1$ and $\nu$ on surface $\Sigma_2$ with $\Sigma_1 \subset J^-(\Sigma_2)$:
\begin{equation}
    \mathcal{W}_\tau(\mu, \nu) := \sup_\pi \int \tau(x, y) \, d\pi(x, y)
\end{equation}
where $\tau(x,y)$ is the Lorentzian distance and the supremum is over causal couplings.
\end{definition}

\begin{remark}
We take \textbf{sup} instead of inf because in Lorentzian geometry, longer paths are more "efficient" (geodesics maximize proper time).
\end{remark}

\subsection{Transport Inequality}

\begin{theorem}[Lorentzian Transport Inequality]\label{thm:lorentzian-transport}
Under SEC (or a suitable curvature condition), for surfaces $\Sigma_1 \subset J^-(\Sigma_2)$:
\begin{equation}
    A(\Sigma_2) \ge A(\Sigma_1) \cdot F\left(\frac{\mathcal{W}_\tau(\mu_1, \mu_2)}{D}\right)
\end{equation}
where $\mu_i$ are normalized area measures, $D$ is a characteristic length scale, and $F$ is a function with $F(0) = 1$.
\end{theorem}

\textbf{Physical interpretation:} If two surfaces are "close" in Lorentzian transport distance, their areas are comparable.

\begin{proof}[Idea]
Use the Lorentzian Jacobi field equation along optimal geodesics. The area evolution is controlled by the Ricci curvature, which under SEC satisfies $\text{Ric}(\dot\gamma, \dot\gamma) \ge 0$ for timelike $\gamma$.

\textcolor{red}{\textbf{Gap:}} Need to develop the full theory of Lorentzian optimal transport, which is an active research area (Mondino, Suhr, McCann, etc.).
\end{proof}

\subsection{Application to Penrose 1973}

\begin{theorem}[Penrose via Lorentzian Transport]\label{thm:penrose-transport}
Assume:
\begin{enumerate}
    \item The Lorentzian transport theory is developed with area-controlling estimates.
    \item Under WCC, there exists a "shortest" causal path from $\Sigma$ to $\mathcal{H}$.
\end{enumerate}
Then $A(\Sigma) \le A(\mathcal{H}_\mathcal{C})$ follows from the transport inequality.
\end{theorem}

\textcolor{red}{\textbf{Status:}} This is a research program, not a complete proof. The Lorentzian optimal transport theory is being developed by several groups but is not yet mature enough to prove Penrose.

%% ============================================================================
\section{New Tool 7: The $\theta$-Capacity}
%% ============================================================================

Here is a genuinely new definition that could work.

\subsection{Definition}

\begin{definition}[$\theta$-Capacity]\label{def:theta-capacity}
For a trapped surface $\Sigma$ in initial data $(M, g, k)$, define:
\begin{equation}
    \text{Cap}_\theta(\Sigma) := \inf_u \int_M |\nabla u|^2 \cdot e^{2\Psi(u)} \, dV_g
\end{equation}
where:
\begin{itemize}
    \item $u: M \to [0, 1]$ with $u|_\Sigma = 1$ and $u \to 0$ at infinity
    \item $\Psi(u) = \int_0^u \max(0, -\theta^+(s)) \, ds$ (integrated null expansion along level sets)
\end{itemize}
\end{definition}

\textbf{Key property:} The weight $e^{2\Psi}$ penalizes regions where $\theta^+ < 0$ (trapped regions), making the capacity "see" the trapped structure.

\subsection{Capacity-Area Inequality}

\begin{theorem}[$\theta$-Capacity Bounds Area]\label{thm:theta-cap-area}
For any trapped surface $\Sigma$:
\begin{equation}
    A(\Sigma) \le C \cdot \text{Cap}_\theta(\Sigma)
\end{equation}
for some universal constant $C$.
\end{theorem}

\begin{proof}[Sketch]
The co-area formula gives:
\begin{equation}
    \int_M |\nabla u|^2 e^{2\Psi} = \int_0^1 \left(\int_{\{u = t\}} |\nabla u| e^{2\Psi}\right) dt
\end{equation}
The inner integral is related to the area of level sets weighted by $e^{2\Psi}$.

For the level set $\Sigma = \{u = 1\}$, the weight $e^{2\Psi(1)}$ depends on how much "trapping" was accumulated.

\textcolor{red}{\textbf{Gap:}} Need to rigorously connect the capacity to the area.
\end{proof}

\subsection{Capacity of the Horizon}

\begin{theorem}[Horizon Capacity]\label{thm:horizon-cap}
For the event horizon cross-section $\mathcal{H}_\mathcal{C}$:
\begin{equation}
    \text{Cap}_\theta(\mathcal{H}_\mathcal{C}) = A(\mathcal{H}_\mathcal{C})
\end{equation}
(The horizon is a "perfect conductor" with capacity equal to area.)
\end{theorem}

\begin{proof}[Idea]
On the horizon, $\theta^+ = 0$, so $\Psi = 0$ and the weight is $e^0 = 1$. The capacity reduces to the standard capacity, which for the horizon equals the area by the isoperimetric property.
\end{proof}

\subsection{The Main Comparison}

\begin{theorem}[Penrose via $\theta$-Capacity]\label{thm:penrose-capacity}
Under WCC + DEC:
\begin{equation}
    A(\Sigma) \le C \cdot \text{Cap}_\theta(\Sigma) \le C \cdot \text{Cap}_\theta(\mathcal{H}_\mathcal{C}) = C \cdot A(\mathcal{H}_\mathcal{C})
\end{equation}
If $C = 1$, this proves Penrose 1973.
\end{theorem}

\textcolor{red}{\textbf{Gap:}} The constant $C$ and the middle inequality need rigorous proof.

%% ============================================================================
\section{Honest Assessment}
%% ============================================================================

\textbf{Tools with genuine potential:}
\begin{enumerate}
    \item \textbf{Lorentzian optimal transport} (Section 6): Research program, not yet mature.
    \item \textbf{$\theta$-Capacity} (Section 7): New definition, needs development.
    \item \textbf{Lorentzian calibrations} (Section 2): Wrong direction, needs modification.
\end{enumerate}

\textbf{The fundamental obstruction:}

The Penrose 1973 conjecture requires connecting \textbf{local} geometric information (trapped surface area) to \textbf{global} causal structure (event horizon). All existing mathematical tools are either:
\begin{itemize}
    \item Local (Riemannian methods): Don't see the horizon.
    \item Global (causal structure): Don't directly control area.
\end{itemize}

A proof requires genuinely new mathematics that bridges this gap.

\textbf{Most promising direction:} Develop the $\theta$-capacity theory rigorously. The key steps are:
\begin{enumerate}
    \item Prove the capacity-area inequality (Theorem~\ref{thm:theta-cap-area}).
    \item Prove capacity monotonicity under causal structure (middle inequality in Theorem~\ref{thm:penrose-capacity}).
    \item Show the constant $C = 1$.
\end{enumerate}

\end{document}
