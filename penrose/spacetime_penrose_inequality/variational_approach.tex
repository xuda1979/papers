% A POTENTIALLY RIGOROUS NEW APPROACH
%
% Using area-minimization inside the trapped region

\documentclass{article}
\usepackage{amsmath,amsthm,amssymb}
\newtheorem{theorem}{Theorem}
\newtheorem{lemma}{Lemma}
\newtheorem{proposition}{Proposition}
\newtheorem{corollary}{Corollary}
\newtheorem{conjecture}{Conjecture}
\newtheorem{remark}{Remark}
\newtheorem{claim}{Claim}

\begin{document}

\title{Variational Approach to the Unfavorable Jump Case}
\date{\today}
\maketitle

\section{Setup}

Let $(M^3, g, k)$ be asymptotically flat initial data satisfying DEC with decay $\tau > 1$.

Let $\Sigma_0$ be a strictly trapped surface: $\theta^+(\Sigma_0) < 0$, $\theta^-(\Sigma_0) < 0$.

Let $\Sigma^*$ be the outermost MOTS enclosing $\Sigma_0$ (exists by Andersson-Metzger).

Assume $\tr_{\Sigma_0} k < 0$ (unfavorable jump).

\section{The Key Observation}

Inside the trapped region $\mathcal{T}$ (bounded by $\Sigma^*$), all surfaces have:
\[
H = \frac{1}{2}(\theta^+ + \theta^-) < 0
\]
since both null expansions are negative or zero.

\textbf{Mean convexity:} The trapped region is mean-convex toward its interior. Surfaces "want to shrink" as you move outward.

\section{Geometric Lemma}

\begin{lemma}[Existence of area-minimizing surface]
Let $\Omega$ be the region between $\Sigma_0$ and $\Sigma^*$. There exists a surface $\Sigma_{\min} \subset \bar{\Omega}$ that minimizes area among surfaces homologous to $\Sigma_0$ in $\bar{\Omega}$.

Moreover, $\Sigma_{\min}$ is either:
\begin{enumerate}
    \item A minimal surface ($H = 0$) in the interior of $\Omega$, or
    \item Equal to $\Sigma^*$ (if $A(\Sigma^*) \le A(\Sigma)$ for all competitors $\Sigma$).
\end{enumerate}
\end{lemma}

\begin{proof}
Standard geometric measure theory. The existence follows from compactness and lower semicontinuity of area. The alternative (i) or (ii) follows from the maximum principle: if $\Sigma_{\min}$ touches $\Sigma^*$ tangentially from inside, it must equal $\Sigma^*$ by uniqueness of MOTS.
\end{proof}

\section{Analysis of the Minimizer}

\textbf{Case (ii):} $\Sigma_{\min} = \Sigma^*$.

Then $A(\Sigma^*) \le A(\Sigma_0)$ (by definition of minimizer), and we have:
\[
M_{\mathrm{ADM}} \ge \sqrt{\frac{A(\Sigma^*)}{16\pi}} \le \sqrt{\frac{A(\Sigma_0)}{16\pi}}
\]
using the Penrose inequality for the stable MOTS $\Sigma^*$ (which has favorable jump automatically).

But wait—this gives the WRONG direction! We get $M \ge \sqrt{A(\Sigma^*)/16\pi}$ which is SMALLER than what we want.

\textbf{Case (i):} $\Sigma_{\min}$ is a minimal surface in the interior.

On $\Sigma_{\min}$: $H = 0$.

The null expansions satisfy:
\[
\theta^+ = H + \tr_{\Sigma_{\min}} k = \tr_{\Sigma_{\min}} k, \quad \theta^- = H - \tr_{\Sigma_{\min}} k = -\tr_{\Sigma_{\min}} k.
\]

\textbf{Key question:} What is the sign of $\tr_{\Sigma_{\min}} k$?

\subsection{Constraint on $\tr k$ for minimal surfaces in trapped regions}

Since $\Sigma_{\min}$ lies inside the trapped region $\mathcal{T}$, it is enclosed by a trapped surface (namely $\Sigma_0$ or some surface between $\Sigma_0$ and $\Sigma_{\min}$).

\textbf{Question:} Does $\Sigma_{\min}$ satisfy $\theta^+ \le 0$?

If $\theta^+ = \tr_{\Sigma_{\min}} k \le 0$, then:
\begin{itemize}
    \item $\Sigma_{\min}$ is a MOTS (if $\theta^+ = 0$) or trapped (if $\theta^+ < 0$)
    \item Since $\theta^- = -\tr_{\Sigma_{\min}} k \ge 0$, we have $\theta^- \ge 0$
    \item If $\theta^+ \le 0$ and $\theta^- \ge 0$, then $\Sigma_{\min}$ is a MOTS or marginally outer trapped
\end{itemize}

But actually, $\Sigma_{\min}$ being minimal does NOT automatically make it trapped. A minimal surface has $H = 0$, so:
\[
\theta^+ = \tr_{\Sigma_{\min}} k, \quad \theta^- = -\tr_{\Sigma_{\min}} k.
\]

These have opposite signs! So $\Sigma_{\min}$ cannot be strictly trapped unless $\tr_{\Sigma_{\min}} k = 0$.

\textbf{Observation:} An interior minimal surface in the trapped region satisfies:
\begin{itemize}
    \item $\theta^+ \theta^- = -(\tr_{\Sigma_{\min}} k)^2 \le 0$
    \item Either $\theta^+ \le 0, \theta^- \ge 0$ (if $\tr k \le 0$) or $\theta^+ \ge 0, \theta^- \le 0$ (if $\tr k \ge 0$)
\end{itemize}

\section{The Bootstrap Argument}

\textbf{Case (i.a):} $\tr_{\Sigma_{\min}} k \ge 0$ (favorable jump on the minimizer).

Then we can apply the Jang equation at $\Sigma_{\min}$:
\begin{enumerate}
    \item $\Sigma_{\min}$ is minimal, so $H = 0$ and $\theta^+ = \tr_{\Sigma_{\min}} k \ge 0$.
    \item If $\theta^+ = 0$, it's a MOTS with $\tr k \ge 0$ (stable, favorable).
    \item If $\theta^+ > 0$, it's NOT a trapped surface—problem!
\end{enumerate}

Wait—if $\tr_{\Sigma_{\min}} k > 0$, then $\theta^+ > 0$ on $\Sigma_{\min}$, which means $\Sigma_{\min}$ is NOT trapped. But $\Sigma_{\min}$ is supposed to be inside the trapped region...

\textbf{Resolution:} The trapped region $\mathcal{T}$ is defined by the existence of enclosing trapped surfaces, not by each point being on a trapped surface. A minimal surface can exist inside $\mathcal{T}$ without being trapped.

\textbf{Case (i.b):} $\tr_{\Sigma_{\min}} k < 0$ (unfavorable jump persists).

Then $\theta^+ = \tr_{\Sigma_{\min}} k < 0$, so $\Sigma_{\min}$ is an outer trapped surface (a MOTS with $\theta^+ < 0$? No—MOTS has $\theta^+ = 0$ by definition).

Actually, $\Sigma_{\min}$ has $H = 0$ and $\theta^+ < 0$, which means:
\[
\theta^+ = H + \tr_{\Sigma_{\min}} k = \tr_{\Sigma_{\min}} k < 0.
\]

So $\Sigma_{\min}$ is an "outer trapped minimal surface"—it has $H = 0$ but $\theta^+ < 0$.

This is a TRAPPED surface (since $\theta^+ < 0$) but NOT a MOTS (since $\theta^+ \neq 0$).

\textbf{Key insight:} $\Sigma_{\min}$ is trapped with SMALLER area than $\Sigma_0$ (by definition of minimizer), but still has unfavorable jump.

\section{Iteration}

Starting from $\Sigma_0$, we found $\Sigma_{\min}$ with:
\begin{itemize}
    \item $A(\Sigma_{\min}) \le A(\Sigma_0)$
    \item $H_{\Sigma_{\min}} = 0$
    \item $\tr_{\Sigma_{\min}} k < 0$ (unfavorable jump persists)
\end{itemize}

Can we iterate? Starting from $\Sigma_{\min}$, find a new minimizer...

But $\Sigma_{\min}$ is already minimal! There's no room to minimize further (it's a fixed point of the minimization process).

\section{Alternative: Use the Stability of $\Sigma^*$}

The outermost MOTS $\Sigma^*$ is stable. By Andersson-Mars-Simon, stability implies:
\[
\tr_{\Sigma^*} k \ge 0.
\]

\textbf{Idea:} Use the stability of $\Sigma^*$ and the geometry of the trapped region to derive constraints on $A(\Sigma_0)$.

\subsection{Penrose inequality for $\Sigma^*$}

We know:
\[
M_{\mathrm{ADM}} \ge \sqrt{\frac{A(\Sigma^*)}{16\pi}}.
\]

To prove the Penrose inequality for $\Sigma_0$, we need to show:
\[
\sqrt{\frac{A(\Sigma^*)}{16\pi}} \ge \sqrt{\frac{A(\Sigma_0)}{16\pi}},
\]
i.e., $A(\Sigma^*) \ge A(\Sigma_0)$.

\textbf{This is the area comparison we failed to prove!}

\section{A New Geometric Principle?}

\begin{conjecture}[Trapped Surface Area Bound]
Let $(M, g, k)$ satisfy DEC. Let $\Sigma_0$ be strictly trapped and let $\Sigma^*$ be the outermost MOTS enclosing it. Then:
\[
A(\Sigma^*) \ge A(\Sigma_0) \cdot \frac{|\theta^+(\Sigma_0)|}{|\theta^-(\Sigma_0)|}.
\]
\end{conjecture}

This would give $A(\Sigma^*) \ge A(\Sigma_0)$ when $|\theta^+| \le |\theta^-|$.

For the unfavorable case: $\tr_{\Sigma_0} k < 0$ means $\theta^+ = H + \tr k < H - \tr k = \theta^-$ (since $\tr k < 0$).

Wait—$\theta^+$ and $\theta^-$ are both negative on a trapped surface. The question is which is MORE negative.

$\theta^+ - \theta^- = 2\tr k < 0$ when $\tr k < 0$. So $\theta^+ < \theta^-$, meaning $|\theta^+| > |\theta^-|$.

The conjecture would give $A(\Sigma^*) \ge A(\Sigma_0) \cdot \frac{|\theta^+|}{|\theta^-|} > A(\Sigma_0)$.

\textbf{This is exactly what we need!}

But the conjecture is unproven...

\section{Toward a Proof of the Conjecture}

The conjecture says that when outgoing light converges faster than ingoing light ($|\theta^+| > |\theta^-|$), the area must grow as you move outward to the boundary.

\textbf{Physical intuition:} Faster outgoing convergence means more "focusing" toward the interior. The surface is "deeper" in the gravitational well. The area must increase as you climb out.

\textbf{Mathematical approach:} Use the focusing equation for null geodesics.

Along outgoing null rays with tangent $\ell^+$:
\[
\frac{d\theta^+}{d\lambda} = -\frac{1}{2}(\theta^+)^2 - |\sigma^+|^2 - R_{\mu\nu}\ell^+{}^\mu \ell^+{}^\nu \le -\frac{1}{2}(\theta^+)^2
\]
by NEC.

This is the Raychaudhuri equation. It shows that $\theta^+$ becomes MORE negative as you move along outgoing null rays (if $\theta^+ < 0$ initially).

But we want to move from $\Sigma_0$ (inside) to $\Sigma^*$ (outside), which is AGAINST the outgoing null direction inside the horizon!

\section{Conclusion}

The variational approach leads to:
\begin{enumerate}
    \item An area-minimizing surface $\Sigma_{\min}$ that is either $\Sigma^*$ or an interior minimal surface.
    \item If $\Sigma_{\min} = \Sigma^*$, we get $A(\Sigma^*) \le A(\Sigma_0)$—the WRONG inequality.
    \item If $\Sigma_{\min}$ is interior and minimal, it may still have unfavorable jump.
\end{enumerate}

\textbf{The fundamental issue:} Area can either increase OR decrease from $\Sigma_0$ to $\Sigma^*$, depending on the geometry. There is no universal monotonicity principle.

\textbf{The Penrose inequality for unfavorable jump remains OPEN.}

\textbf{Possible resolution:} The Penrose inequality might be FALSE in general for trapped surfaces with $\tr_\Sigma k < 0$. A counterexample would require:
\begin{itemize}
    \item Initial data $(M, g, k)$ satisfying DEC
    \item A trapped surface $\Sigma_0$ with $\tr_{\Sigma_0} k < 0$
    \item $M_{\mathrm{ADM}} < \sqrt{A(\Sigma_0)/(16\pi)}$
\end{itemize}

No such counterexample is known, but it cannot be ruled out with current techniques.

\end{document}
