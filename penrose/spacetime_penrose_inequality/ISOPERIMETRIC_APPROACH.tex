% =========================================================================
%     ISOPERIMETRIC AND GEOMETRIC MEASURE THEORY APPROACHES
%
%     Using isoperimetric profiles and GMT for the Penrose inequality
%
%     Author: Da Xu
%     Date: December 2025
% =========================================================================

\documentclass[12pt]{article}
\usepackage{amsmath,amsthm,amssymb}
\usepackage{mathrsfs}
\usepackage{tcolorbox}

\theoremstyle{plain}
\newtheorem{theorem}{Theorem}[section]
\newtheorem{lemma}[theorem]{Lemma}
\newtheorem{proposition}[theorem]{Proposition}
\newtheorem{corollary}[theorem]{Corollary}

\theoremstyle{definition}
\newtheorem{definition}[theorem]{Definition}
\newtheorem{remark}[theorem]{Remark}

\newcommand{\ADM}{\mathrm{ADM}}
\newcommand{\tr}{\mathrm{tr}}
\newcommand{\Div}{\mathrm{div}}
\newcommand{\Area}{\mathrm{Area}}
\newcommand{\Vol}{\mathrm{Vol}}

\title{\textbf{Isoperimetric Approaches to the Penrose Inequality}}
\author{Da Xu}
\date{December 2025}

\begin{document}
\maketitle

\section{Isoperimetric Inequalities and Mass}

\subsection{The Euclidean Isoperimetric Inequality}

In $\mathbb{R}^3$:
\[
    36\pi \Vol(\Omega)^2 \leq \Area(\partial\Omega)^3
\]
with equality for balls.

\subsection{Isoperimetric Inequality on Asymptotically Flat Manifolds}

\begin{theorem}[Fan-Shi-Tam, Miao]
On an asymptotically flat 3-manifold $(M^3, g)$ with $R_g \geq 0$:
\[
    \Area(\Sigma)^{3/2} \geq 6\sqrt{\pi} \Vol(\Omega_\Sigma) - C \cdot M_{\ADM} \cdot \Area(\Sigma)
\]
for large surfaces $\Sigma$ enclosing volume $\Omega_\Sigma$.
\end{theorem}

\subsection{Isoperimetric Mass}

\begin{definition}
The \textbf{isoperimetric mass} is:
\[
    m_{\text{iso}}(\Sigma) = \frac{1}{2}\left(\frac{\Area(\Sigma)}{4\pi}\right)^{1/2} 
    \left(1 - \frac{36\pi \Vol(\Omega_\Sigma)^2}{\Area(\Sigma)^3}\right)
\]
\end{definition}

\begin{theorem}
For large coordinate spheres in asymptotically flat space:
\[
    \lim_{r \to \infty} m_{\text{iso}}(S_r) = M_{\ADM}
\]
\end{theorem}

\section{Application to Penrose Inequality}

\subsection{Strategy}

\begin{enumerate}
    \item Define isoperimetric mass on trapped surfaces
    \item Show it's bounded below by $\sqrt{\Area/(16\pi)}$
    \item Show it's monotonic to $M_{\ADM}$
\end{enumerate}

\subsection{The Isoperimetric Profile}

\begin{definition}
The \textbf{isoperimetric profile} of $(M, g)$ is:
\[
    I(V) = \inf\{\Area(\partial\Omega) : \Vol(\Omega) = V\}
\]
\end{definition}

For asymptotically flat manifolds with $R \geq 0$:
\[
    I(V) \leq (36\pi)^{1/3} V^{2/3} + O(V^{1/3})
\]

\subsection{Trapped Surfaces and Isoperimetry}

\textbf{Question:} How does trappedness affect isoperimetric properties?

For a trapped surface $\Sigma_0$ bounding region $\Omega_0$:
\begin{itemize}
    \item $\Sigma_0$ has $H < 0$ (mean curvature points inward)
    \item But $\Sigma_0$ is NOT necessarily isoperimetric (area-minimizing for volume)
\end{itemize}

\begin{lemma}
If $\Sigma$ is isoperimetric (minimizes area for enclosed volume), then $H$ is constant.
For trapped surfaces, $H$ varies and can have either sign locally.
\end{lemma}

\textbf{Problem:} Trapped surfaces are not isoperimetric in general.

\section{Approach 1: Convex Hull Method}

\subsection{Idea}

Replace $\Sigma_0$ by its ``isoperimetric hull'':

\begin{definition}
The \textbf{isoperimetric hull} of $\Sigma_0$ is the surface $\Sigma_{\text{iso}}$
that minimizes area among all surfaces enclosing $\Omega_0$.
\end{definition}

\begin{lemma}
The isoperimetric hull satisfies $H = $ const and $\Area(\Sigma_{\text{iso}}) \leq \Area(\Sigma_0)$.
\end{lemma}

\subsection{Application}

If we could prove:
\[
    M_{\ADM} \geq \sqrt{\frac{\Area(\Sigma_{\text{iso}})}{16\pi}}
\]
this would give a \textbf{weaker} bound:
\[
    M_{\ADM} \geq \sqrt{\frac{\Area(\Sigma_{\text{iso}})}{16\pi}} \leq \sqrt{\frac{\Area(\Sigma_0)}{16\pi}}
\]

This is the \textbf{wrong direction}!

\section{Approach 2: The Bray Isoperimetric Method}

\subsection{Bray's Approach for Minimal Surfaces}

Bray proved the Riemannian Penrose inequality using:
\begin{enumerate}
    \item Conformal flow $g_t = u_t^4 g$
    \item $\Area_t(\Sigma_0)$ is preserved
    \item $m_{\text{iso}}$ is non-decreasing
    \item At $t = \infty$, the metric is Schwarzschild
\end{enumerate}

\subsection{Extension to Spacetime}

For spacetime data $(g, k)$:
\begin{itemize}
    \item Need a coupled flow on $(g, k)$
    \item Constraints must be preserved
    \item Area of trapped surface should be controlled
\end{itemize}

\textbf{Problem:} Bray's conformal flow doesn't extend to spacetime because:
\begin{enumerate}
    \item Conformal changes of $g$ alter the constraints
    \item The area of trapped surfaces (not minimal) changes unpredictably
    \item The isoperimetric profile depends on $g$, not $(g, k)$
\end{enumerate}

\section{Approach 3: Currents and Geometric Measure Theory}

\subsection{Trapped Surfaces as Currents}

In GMT, surfaces are represented as currents (linear functionals on differential forms).

\begin{definition}
A surface $\Sigma$ defines a 2-current $[\Sigma]$ by:
\[
    [\Sigma](\omega) = \int_\Sigma \omega
\]
\end{definition}

\subsection{The Mass of a Current}

For a current $T$, the \textbf{mass} is:
\[
    \mathbf{M}(T) = \sup\{T(\omega) : |\omega| \leq 1\}
\]

For smooth surfaces, $\mathbf{M}([\Sigma]) = \Area(\Sigma)$.

\subsection{Trapped Currents?}

\textbf{Idea:} Define a notion of ``trapped current'' incorporating the spacetime structure.

Let $\omega_k$ be the 2-form associated with $\tr k$:
\[
    \omega_k = k \wedge \text{(volume element)}
\]

A current $T$ is \textbf{k-trapped} if:
\[
    T(\omega_H + \omega_k) \leq 0
\]
where $\omega_H$ is the mean curvature 2-form.

\textbf{Problem:} This doesn't lead to useful inequalities because the mass
$\mathbf{M}(T)$ doesn't have a direct relationship to $M_{\ADM}$.

\section{Approach 4: The Geroch-Huisken-Ilmanen Functional}

\subsection{Definition}

Define the functional:
\[
    \mathcal{F}(\Sigma) = \sqrt{\frac{\Area(\Sigma)}{16\pi}} \exp\left(-\frac{1}{16\pi}\int_\Sigma H^2 \, dA\right)
\]

This satisfies:
\begin{itemize}
    \item $\mathcal{F}(\Sigma) \leq \sqrt{\Area/(16\pi)}$ always
    \item Under IMCF with $R \geq 0$: $\mathcal{F}$ is non-decreasing
    \item At infinity: $\mathcal{F} \to M_{\ADM}$
\end{itemize}

\subsection{Modified for Trapped Surfaces}

Define:
\[
    \mathcal{F}_{\text{trap}}(\Sigma) = \sqrt{\frac{\Area(\Sigma)}{16\pi}} \exp\left(-\frac{1}{16\pi}\int_\Sigma \theta^+\theta^- \, dA\right)
\]

For trapped surfaces, $\theta^+\theta^- > 0$, so:
\[
    \mathcal{F}_{\text{trap}}(\Sigma_0) < \sqrt{\frac{\Area(\Sigma_0)}{16\pi}}
\]

\textbf{Problem:} This gives a bound on $\mathcal{F}_{\text{trap}}$, not on $\sqrt{\Area/(16\pi)}$.

\section{Approach 5: Optimal Isoperimetric Constant}

\subsection{The Cheeger Constant}

\begin{definition}
The \textbf{Cheeger constant} of $M$ is:
\[
    h(M) = \inf_\Sigma \frac{\Area(\Sigma)}{\min(\Vol(\Omega_1), \Vol(\Omega_2))}
\]
where $\Sigma$ divides $M$ into $\Omega_1$ and $\Omega_2$.
\end{definition}

\subsection{Relation to Spectrum}

\begin{theorem}[Cheeger Inequality]
\[
    \lambda_1(-\Delta) \geq \frac{h(M)^2}{4}
\]
\end{theorem}

\subsection{Application Attempt}

For trapped surfaces, can we bound $h(M)$ in terms of ADM mass?

\textbf{Analysis:} On asymptotically flat manifolds:
\[
    h(M) \sim \frac{1}{r_{\text{max}}}
\]
where $r_{\text{max}}$ is the scale of the manifold.

This doesn't directly involve $M_{\ADM}$ or $\Area(\Sigma_0)$.

\begin{tcolorbox}[colback=yellow!10, colframe=orange!75!black, title=\textbf{Interesting Direction}]
\textbf{Observation:} In Schwarzschild with mass $M$, the horizon has:
\[
    \Area = 16\pi M^2, \quad h(M) \sim \frac{1}{4M}
\]

So:
\[
    h(M)^2 \cdot \Area \sim \frac{16\pi M^2}{16M^2} = \pi
\]

\textbf{Question:} Is there a universal bound $h(M)^2 \cdot \Area(\Sigma_0) \geq C$ for trapped surfaces?

This would give $\Area(\Sigma_0) \geq C/h(M)^2 \sim C \cdot (M)^2$, which is close to Penrose!

\textbf{Status:} This is speculative and would require significant new analysis.
\end{tcolorbox}

\section{Approach 6: Willmore Energy}

\subsection{Definition}

The \textbf{Willmore energy} of $\Sigma$ in $\mathbb{R}^3$:
\[
    W(\Sigma) = \frac{1}{4}\int_\Sigma H^2 \, dA
\]

\begin{theorem}[Willmore Inequality]
For any closed surface $\Sigma$ in $\mathbb{R}^3$:
\[
    W(\Sigma) \geq 4\pi
\]
with equality for round spheres.
\end{theorem}

\subsection{Generalized Willmore}

On a Riemannian manifold:
\[
    W_g(\Sigma) = \frac{1}{4}\int_\Sigma (H^2 - K_{\text{ext}} + K_{\text{int}}) \, dA
\]
where $K_{\text{ext}}, K_{\text{int}}$ are extrinsic and intrinsic curvatures.

\subsection{For Trapped Surfaces}

\[
    W_{\text{trap}}(\Sigma) = \frac{1}{4}\int_\Sigma (\theta^+)^2 \, dA
\]

For trapped surfaces, $\theta^+ < 0$, so this is positive.

\textbf{Problem:} No direct link to $M_{\ADM}$ or the Penrose bound.

\begin{tcolorbox}[colback=red!10, colframe=red!75!black, title=\textbf{Conclusion: Isoperimetric Methods}]
\textbf{Summary:} Isoperimetric and GMT approaches face these issues:

\begin{enumerate}
    \item \textbf{Isoperimetric hull:} Gives weaker bound (wrong direction)
    \item \textbf{Bray's method:} Doesn't extend to spacetime
    \item \textbf{GMT currents:} No direct mass connection
    \item \textbf{Cheeger/spectral:} Interesting but unproven
\end{enumerate}

\textbf{Status:} No isoperimetric method currently resolves the unconditional case.
\end{tcolorbox}

\end{document}
