%% NOVEL_ATTACK_PENROSE_1973.tex
%%
%% A GENUINELY NEW APPROACH TO THE 1973 CONJECTURE
%%
%% Key insight: Use the constraint equations to derive area monotonicity
%% directly, without flows or capacity or spinors
%%
%% Blue/Red Team development
%%
%% Author: Mathematical Analysis for Penrose 1973
%% Date: December 2025

\documentclass[11pt]{amsart}
\usepackage{amsmath,amssymb,amsthm}
\usepackage{mathtools}
\usepackage{xcolor}
\usepackage{tcolorbox}

\tcbuselibrary{theorems}

\newtcolorbox{redattack}{
    colback=red!5!white,
    colframe=red!75!black,
    title={\textbf{RED TEAM ATTACK}}
}

\newtcolorbox{bluedefense}{
    colback=blue!5!white,
    colframe=blue!75!black,
    title={\textbf{BLUE TEAM DEFENSE}}
}

\newtcolorbox{keyidea}{
    colback=cyan!5!white,
    colframe=cyan!75!black,
    title={\textbf{KEY NEW IDEA}}
}

\newtcolorbox{breakthrough}{
    colback=yellow!10!white,
    colframe=yellow!50!black,
    title={\textbf{BREAKTHROUGH}}
}

\newtheorem{theorem}{Theorem}[section]
\newtheorem{lemma}[theorem]{Lemma}
\newtheorem{proposition}[theorem]{Proposition}
\newtheorem{corollary}[theorem]{Corollary}
\theoremstyle{definition}
\newtheorem{definition}[theorem]{Definition}
\newtheorem{conjecture}[theorem]{Conjecture}
\theoremstyle{remark}
\newtheorem{remark}[theorem]{Remark}

\newcommand{\bR}{\mathbb{R}}
\newcommand{\cT}{\mathcal{T}}
\newcommand{\cM}{\mathcal{M}}
\newcommand{\ADM}{\mathrm{ADM}}
\newcommand{\Area}{\mathrm{Area}}
\newcommand{\Vol}{\mathrm{Vol}}
\newcommand{\tr}{\mathrm{tr}}
\newcommand{\divg}{\mathrm{div}}

\title{Novel Attack on Penrose 1973:\\
\large Constraint-Based Area Monotonicity}
\author{}
\date{December 2025}

\begin{document}
\maketitle

\begin{abstract}
We develop a new approach to the area dominance problem for the spacetime Penrose inequality. The key idea is to use the Einstein constraint equations directly to establish a relationship between the areas of nested surfaces, without using flows or capacity theory. We introduce a ``constraint potential'' whose level sets provide a natural foliation of the trapped region, with area monotonicity built in.
\end{abstract}

\tableofcontents

%% ============================================================================
\section{The Core Insight}
%% ============================================================================

\begin{keyidea}
\textbf{The constraint equations contain area information.}

The Hamiltonian and momentum constraints:
\begin{align}
    R + (\tr k)^2 - |k|^2 &= 2\mu, \\
    \divg(k - (\tr k)g) &= J,
\end{align}
determine the \textbf{local} geometry. But they also contain \textbf{global} information about areas via integration.

\textbf{New idea:} Construct a scalar function $\Phi$ from the constraints such that:
\begin{equation}
    A(\Sigma) = F(\Phi|_\Sigma)
\end{equation}
for some monotonic function $F$. Then level sets of $\Phi$ automatically give area ordering.
\end{keyidea}

%% ============================================================================
\section{The Constraint Potential}
%% ============================================================================

\subsection{Definition}

\begin{definition}[Constraint Potential]
Define the \textbf{constraint potential} $\Phi: M \to \bR$ by:
\begin{equation}
    \Delta \Phi = R + (\tr k)^2 - |k|^2 = 2\mu
\end{equation}
with boundary conditions:
\begin{itemize}
    \item $\Phi \to 0$ at infinity
    \item $\Phi|_{\Sigma^*} = \Phi_0$ (constant on outermost MOTS)
\end{itemize}
\end{definition}

\begin{redattack}
\textbf{Attack 1: This is just a restatement.}

Solving $\Delta \Phi = 2\mu$ with these boundary conditions is equivalent to solving a Poisson equation. The solution exists but doesn't obviously give area information.

\textbf{Attack 2: Why should $\Phi$ know about areas?}

The equation $\Delta \Phi = 2\mu$ relates $\Phi$ to the energy density $\mu$, not to surface areas. There's no a priori reason for level sets of $\Phi$ to have monotonic area.
\end{redattack}

\begin{bluedefense}
\textbf{Defense via coarea formula:}

The coarea formula gives:
\begin{equation}
    \int_M f |\nabla \Phi| \, dV = \int_{-\infty}^{\infty} \left( \int_{\{\Phi = t\}} f \, dA \right) dt.
\end{equation}

If we choose $f = 1$:
\begin{equation}
    \int_M |\nabla \Phi| \, dV = \int_{-\infty}^{\infty} A(\{\Phi = t\}) \, dt.
\end{equation}

This relates the total ``gradient energy'' to the areas of level sets.

\textbf{Key observation:} If $|\nabla \Phi|$ is approximately constant, then:
\begin{equation}
    A(\{\Phi = t\}) \approx \frac{\int_M |\nabla \Phi| dV}{\text{range of } \Phi}.
\end{equation}

But we want \textbf{monotonicity}, not just average behavior.
\end{bluedefense}

\subsection{Refined Construction}

\begin{bluedefense}
\textbf{Improved potential:} Instead of $\Delta \Phi = 2\mu$, use the \textbf{null expansion} directly.

Define $\Phi$ by:
\begin{equation}
    \Delta \Phi + H |\nabla \Phi| = 2\mu
\end{equation}
where $H$ is the mean curvature of level sets of $\Phi$.

This is a \textbf{mean curvature-type equation}. Level sets evolve by mean curvature in a generalized sense.

\textbf{Alternative:} Define $\Phi$ implicitly via the level set equation:
\begin{equation}
    \theta^+(\{\Phi = t\}) = f(t)
\end{equation}
for some prescribed function $f(t)$ with $f(t_0) = 0$ at the MOTS.
\end{bluedefense}

\begin{redattack}
\textbf{Attack: This is circular.}

Defining $\Phi$ via $\theta^+(\text{level sets}) = f(t)$ requires knowing which level sets to take. But level sets are determined by $\Phi$, which is what we're trying to define!

This is an \textbf{implicit} definition that may not have a solution, or may have multiple solutions.
\end{redattack}

%% ============================================================================
\section{Alternative: The Mass Aspect Function}
%% ============================================================================

\begin{keyidea}
\textbf{The Hawking mass of surfaces encodes area and curvature.}

For a surface $\Sigma$:
\begin{equation}
    m_H(\Sigma) = \sqrt{\frac{A(\Sigma)}{16\pi}} \left( 1 - \frac{1}{16\pi} \int_\Sigma H^2 dA \right).
\end{equation}

This combines area $A$ with mean curvature $H$.

\textbf{Idea:} If Hawking mass is monotonic along some foliation, we get area control.
\end{keyidea}

\begin{theorem}[Hawking Mass Monotonicity under IMCF]
Under inverse mean curvature flow (IMCF) in a manifold with $R \ge 0$:
\begin{equation}
    \frac{d m_H}{dt} \ge 0.
\end{equation}
Moreover, $m_H \to M_{\ADM}$ as the flow goes to infinity.
\end{theorem}

\begin{redattack}
\textbf{Attack: IMCF doesn't work for trapped surfaces.}

IMCF requires $H > 0$ to be well-defined. The flow velocity is $V = 1/H$.

For trapped surfaces: $H < 0$ (by our earlier lemma). IMCF cannot start from a trapped surface!

This is why Huisken-Ilmanen's proof works for the \textbf{Riemannian} Penrose inequality (minimal surfaces with $H = 0$), but not for the spacetime case (trapped surfaces with $H < 0$).
\end{redattack}

\begin{bluedefense}
\textbf{Modify IMCF for trapped region:}

Consider the flow:
\begin{equation}
    \frac{\partial \Sigma}{\partial t} = \frac{1}{|H|} \nu
\end{equation}
which is well-defined even when $H < 0$ (just use the absolute value).

Area evolution:
\begin{equation}
    \frac{dA}{dt} = \int_\Sigma \frac{H}{|H|} dA = \int_\Sigma \text{sgn}(H) dA.
\end{equation}

For $H < 0$ uniformly (trapped region): $\frac{dA}{dt} = -A < 0$.

\textbf{Area decreases}, which is the wrong direction for reaching MOTS.
\end{bluedefense}

%% ============================================================================
\section{The Dual Flow Approach}
%% ============================================================================

\begin{keyidea}
\textbf{Run a flow BACKWARDS from MOTS into the trapped region.}

If we can't flow from trapped surface to MOTS with increasing area, maybe we can flow from MOTS into the trapped region with \textbf{decreasing} area.

Then: For any trapped surface $\Sigma_0$, there exists a flow path from $\Sigma^*$ to some surface $\Sigma'$ with $\Sigma_0 \subset \text{interior of } \Sigma'$.

Area decreases along flow: $A(\Sigma') < A(\Sigma^*)$.

If $\Sigma_0 \subset \Sigma'$, then... wait, this doesn't directly give $A(\Sigma_0) \le A(\Sigma^*)$.
\end{keyidea}

\begin{redattack}
\textbf{Attack: Containment doesn't imply area comparison.}

Even if $\Sigma_0 \subset \text{interior bounded by } \Sigma'$, we could have $A(\Sigma_0) > A(\Sigma')$.

Example: A small sphere inside a large dumbell-shaped surface. The sphere has smaller volume but the dumbell has smaller area.

\textbf{We need a different approach.}
\end{redattack}

%% ============================================================================
\section{The Isoperimetric Approach}
%% ============================================================================

\begin{keyidea}
\textbf{Use isoperimetric properties of the trapped region.}

In a region $\Omega$ with suitable properties, the isoperimetric inequality states:
\begin{equation}
    A(\partial \Omega) \ge C \cdot \Vol(\Omega)^{2/3}
\end{equation}
for some constant $C$ depending on the geometry.

\textbf{If the trapped region $\cT$ satisfies an isoperimetric inequality, then surfaces inside have area controlled by the boundary.}
\end{keyidea}

\begin{definition}[Trapped Region]
The trapped region is:
\begin{equation}
    \cT = \{ x \in M : x \text{ lies inside some trapped surface} \}.
\end{equation}
Its boundary (if smooth) is the outermost MOTS: $\partial \cT = \Sigma^*$.
\end{definition}

\begin{proposition}[Trapped Region Isoperimetric]\label{prop:isop}
Let $(M, g, k)$ satisfy DEC with trapped region $\cT$ and outermost MOTS $\Sigma^* = \partial \cT$. Suppose $\cT$ satisfies the isoperimetric condition:
\begin{equation}
    A(\Sigma) \le A(\partial \Omega) \quad \text{for all } \Sigma = \partial \Omega \subset \cT.
\end{equation}
Then for any trapped surface $\Sigma_0$:
\begin{equation}
    A(\Sigma_0) \le A(\Sigma^*).
\end{equation}
\end{proposition}

\begin{proof}
The trapped surface $\Sigma_0$ bounds a region $\Omega_0$ with $\Omega_0 \subset \cT$ (since $\Sigma_0$ is trapped, its interior is in the trapped region).

Apply the isoperimetric condition with $\Omega = \cT$ (the full trapped region):
\begin{equation}
    A(\Sigma_0) = A(\partial \Omega_0) \le A(\partial \cT) = A(\Sigma^*).
\end{equation}
\end{proof}

\begin{redattack}
\textbf{Attack: The isoperimetric condition is assumed, not proven.}

Proposition \ref{prop:isop} says: ``If $\cT$ satisfies isoperimetric, then area dominance holds.''

But the isoperimetric condition for $\cT$ is exactly what we need to prove! This is circular.

\textbf{Question:} Does DEC imply the trapped region satisfies an isoperimetric inequality?
\end{redattack}

\begin{bluedefense}
\textbf{Partial answer:} The DEC gives $\mu \ge |J| \ge 0$. By the Hamiltonian constraint:
\begin{equation}
    R + (\tr k)^2 - |k|^2 = 2\mu \ge 0.
\end{equation}

This is a \textbf{lower bound on scalar curvature} (in a generalized sense).

\textbf{Classical result:} In a manifold with $R \ge -6\kappa^2$, the isoperimetric inequality holds with constant depending on $\kappa$.

\textbf{For DEC:} We have $R \ge |k|^2 - (\tr k)^2$. If $|k|$ is bounded, this gives $R \ge -C$ for some $C$.

\textbf{Problem:} Near the MOTS, $|k|$ may blow up (the trapped region might have large extrinsic curvature).
\end{bluedefense}

%% ============================================================================
\section{The Stability Approach}
%% ============================================================================

\begin{keyidea}
\textbf{Use the stability of MOTS.}

A MOTS $\Sigma^*$ is \textbf{stable} if the principal eigenvalue of the stability operator $L$ is non-negative:
\begin{equation}
    L \phi = -\Delta_\Sigma \phi - (|A|^2 + \text{Ric}(\nu, \nu) + \divg \chi) \phi, \quad \lambda_1(L) \ge 0.
\end{equation}

Stability means: small perturbations don't create trapped surfaces outside $\Sigma^*$.

\textbf{This should imply that $\Sigma^*$ is an ``outer barrier'' for the trapped region.}
\end{keyidea}

\begin{theorem}[Stability and Outermost Property]\label{thm:stability}
Let $\Sigma^*$ be a stable MOTS. Then no trapped surface exists in the exterior of $\Sigma^*$ (the region where $\theta^+ > 0$ before $\Sigma^*$).
\end{theorem}

\begin{proof}[Proof sketch]
By stability, small outward perturbations of $\Sigma^*$ have $\theta^+ > 0$. By the maximum principle for $\theta^+$ (Andersson-Mars-Simon), any surface with $\theta^+ \le 0$ must lie inside $\Sigma^*$.
\end{proof}

\begin{bluedefense}
\textbf{This proves containment, not area comparison.}

From Theorem \ref{thm:stability}: Any trapped surface $\Sigma_0$ satisfies $\Sigma_0 \subset \overline{\cT}$ (closure of trapped region).

But we need $A(\Sigma_0) \le A(\Sigma^*)$, which doesn't follow from containment alone.

\textbf{New idea:} Use stability more directly.

The stability operator $L$ has $\lambda_1 \ge 0$. Consider the ``area functional'':
\begin{equation}
    A(\Sigma_\epsilon) = A(\Sigma^*) + \epsilon \int_{\Sigma^*} H_{\Sigma^*} \phi \, dA + \frac{\epsilon^2}{2} \int_{\Sigma^*} \phi L \phi \, dA + O(\epsilon^3)
\end{equation}
for perturbations $\Sigma_\epsilon$ in direction $\phi \nu$.

At a MOTS: $H_{\Sigma^*} = -\tr_{\Sigma^*} k \ne 0$ generally.

\textbf{The first-order term doesn't vanish!} So $\Sigma^*$ is not a critical point of area.
\end{bluedefense}

\begin{redattack}
\textbf{Attack: MOTS is not area-extremal.}

The MOTS condition $\theta^+ = 0$ is not the same as $H = 0$ (minimal).

At a minimal surface: $H = 0$, so first variation of area vanishes.

At a MOTS: $H = -\tr k \ne 0$ generally, so first variation of area is non-zero.

\textbf{MOTS can be locally increased or decreased in area by perturbations.}

This is why area dominance is hard: MOTS is NOT an area extremum.
\end{redattack}

%% ============================================================================
\section{The Null Hypersurface Approach}
%% ============================================================================

\begin{keyidea}
\textbf{Embed the problem in 4D and use null geometry.}

The initial data $(M, g, k)$ sits inside a spacetime $(N^4, \bar{g})$. The trapped surface $\Sigma_0$ generates null hypersurfaces $\cN^\pm$.

\textbf{Raychaudhuri equation:} Along null geodesics:
\begin{equation}
    \frac{d\theta}{d\lambda} = -\frac{\theta^2}{2} - \sigma^2 - R_{\mu\nu} \ell^\mu \ell^\nu
\end{equation}
where $\sigma$ is shear and $R_{\mu\nu}$ is Ricci tensor.

Under null energy condition: $R_{\mu\nu} \ell^\mu \ell^\nu \ge 0$, so $\frac{d\theta}{d\lambda} \le -\theta^2/2$.

\textbf{For trapped surface:} $\theta|_{\Sigma_0} < 0$. By Raychaudhuri, $\theta$ becomes more negative along future-directed null geodesics.

\textbf{This leads to focusing theorem and singularities, not area comparison.}
\end{keyidea}

\begin{bluedefense}
\textbf{Use Raychaudhuri for area evolution on null hypersurface:}

On the outgoing null hypersurface $\cN^+$ from $\Sigma_0$:
\begin{equation}
    \frac{dA}{d\lambda} = \int_{\Sigma_\lambda} \theta^+ dA
\end{equation}
where $\Sigma_\lambda$ is the cross-section at affine parameter $\lambda$.

For trapped $\Sigma_0$: $\theta^+|_{\Sigma_0} < 0$, so initially $\frac{dA}{d\lambda} < 0$.

Area \textbf{decreases} along the outgoing null direction.

\textbf{Wait:} This is the Hawking area theorem in reverse! For a \textbf{black hole event horizon}, area increases to the future.

But we're looking at the \textbf{past} of the horizon (inside the trapped region).
\end{bluedefense}

\begin{breakthrough}
\textbf{Key insight:} The null hypersurface from $\Sigma_0$ going \textbf{inward} (decreasing $\lambda$) has \textbf{increasing} area.

More precisely: shoot null geodesics \textbf{inward} from $\Sigma_0$. Since $\theta^+ < 0$ means the outgoing null rays converge, the \textbf{ingoing} null rays diverge.

Let $\cN^-$ be the ingoing null hypersurface. Then:
\begin{equation}
    \frac{dA}{d\lambda} = \int \theta^- dA.
\end{equation}

For trapped surface: $\theta^- < 0$, so area still decreases!

\textbf{Both null directions give decreasing area for trapped surfaces.} This is what makes them ``trapped.''
\end{breakthrough}

\begin{redattack}
\textbf{Fundamental obstruction:}

For a trapped surface, \textbf{both} null expansions are negative. Area decreases in both null directions.

The only way to \textbf{increase} area is to move in a \textbf{spacelike} direction.

But spacelike directions don't have a canonical choice, and moving spacelike doesn't preserve the null structure that defines trapped surfaces.

\textbf{This is why the problem is hard:} The null geometry that defines trapped surfaces is exactly what prevents area from increasing.
\end{redattack}

%% ============================================================================
\section{A New Monotonicity Formula}
%% ============================================================================

\begin{keyidea}
\textbf{Find a quantity that IS monotonic along some path from $\Sigma_0$ to $\Sigma^*$.}

Instead of trying to make area monotonic, find a functional $F(\Sigma)$ with:
\begin{enumerate}
    \item $F$ is monotonic along some path
    \item $F(\Sigma) \ge c \cdot A(\Sigma)$ for some $c > 0$
    \item $F(\Sigma^*) = c' \cdot A(\Sigma^*)$ with $c' \le c$
\end{enumerate}

Then: $c \cdot A(\Sigma_0) \le F(\Sigma_0) \le F(\Sigma^*) = c' \cdot A(\Sigma^*) \le c \cdot A(\Sigma^*)$, giving $A(\Sigma_0) \le A(\Sigma^*)$.
\end{keyidea}

\begin{bluedefense}
\textbf{Candidate:} The generalized Hawking mass:
\begin{equation}
    m_\theta(\Sigma) = \sqrt{\frac{A(\Sigma)}{16\pi}} \left( 1 - \frac{1}{16\pi} \int_\Sigma (\theta^+)^2 dA \right).
\end{equation}

For trapped surface: $\theta^+ < 0$, so $(\theta^+)^2 > 0$, giving $m_\theta < \sqrt{A/(16\pi)}$.

For MOTS: $\theta^+ = 0$, so $m_\theta = \sqrt{A/(16\pi)}$.

\textbf{Property:} $m_\theta(\Sigma) \le \sqrt{A(\Sigma)/(16\pi)}$ with equality iff $\Sigma$ is MOTS.
\end{bluedefense}

\begin{redattack}
\textbf{Attack: No monotonicity for $m_\theta$.}

There is no known flow or foliation along which $m_\theta$ is monotonic.

The standard Hawking mass $m_H$ is monotonic under IMCF, but $m_\theta$ is a different quantity.

\textbf{Also:} The inequality $m_\theta \le \sqrt{A/(16\pi)}$ goes the wrong way. We want:
\begin{equation}
    A(\Sigma_0) \le A(\Sigma^*).
\end{equation}

From $m_\theta \le \sqrt{A/(16\pi)}$, we get an \textbf{upper} bound on $m_\theta$, not on $A$.
\end{redattack}

%% ============================================================================
\section{Direct Area Bound via Constraints}
%% ============================================================================

\begin{keyidea}
\textbf{Use the constraints to directly bound $A(\Sigma_0)$ in terms of mass.}

The positive mass theorem says $M \ge 0$ under DEC.

\textbf{Can we prove $M \ge f(A(\Sigma_0))$ for trapped surfaces directly?}

If yes: $A(\Sigma_0) \le f^{-1}(M) \le f^{-1}(M) = \sqrt{16\pi} M$ (for Penrose).

This would give the inequality without needing area dominance!
\end{keyidea}

\begin{theorem}[Proposed Direct Bound]\label{thm:direct}
Let $(M, g, k)$ satisfy DEC with a trapped surface $\Sigma_0$. Then:
\begin{equation}
    M_{\ADM} \ge \sqrt{\frac{A(\Sigma_0)}{16\pi}}.
\end{equation}
\end{theorem}

\begin{proof}[Attempted proof via constraints]
\textbf{Step 1:} The ADM mass can be written as:
\begin{equation}
    M_{\ADM} = \frac{1}{16\pi} \int_M (R - |k|^2 + (\tr k)^2) dV + \text{boundary terms}.
\end{equation}

Under DEC: $R - |k|^2 + (\tr k)^2 = 2\mu \ge 0$.

\textbf{Step 2:} For a trapped surface $\Sigma_0$, use the Gauss-Bonnet theorem:
\begin{equation}
    \int_{\Sigma_0} K dA = 2\pi \chi(\Sigma_0) = 4\pi \quad \text{(for sphere)}.
\end{equation}

The Gaussian curvature $K$ relates to $R$ via Gauss equation.

\textbf{Step 3:} Combine to get... 

\textbf{[This is where the proof breaks down. We cannot directly relate $\int_{\Sigma_0}$ to $\int_M$.]}
\end{proof}

\begin{redattack}
\textbf{Attack: The proof doesn't close.}

The ADM mass involves a bulk integral over $M$. The area of $\Sigma_0$ is a boundary quantity.

There's no direct way to bound a boundary term by a bulk term without additional structure (like a Green's function or harmonic coordinates).

\textbf{The constraint equations relate $R$, $k$, $\mu$, $J$ but don't directly give area bounds.}
\end{redattack}

%% ============================================================================
\section{Summary and Honest Assessment}
%% ============================================================================

\subsection{What We Tried}

\begin{center}
\begin{tabular}{|l|c|p{5cm}|}
\hline
\textbf{Approach} & \textbf{Status} & \textbf{Failure Mode} \\
\hline
Constraint potential & \textcolor{red}{FAILED} & Doesn't give area info \\
Hawking mass monotonicity & \textcolor{red}{FAILED} & IMCF needs $H > 0$ \\
Dual flow from MOTS & \textcolor{red}{FAILED} & Containment $\ne$ area bound \\
Trapped region isoperimetric & \textcolor{orange}{CIRCULAR} & Assumes what we prove \\
MOTS stability & \textcolor{orange}{PARTIAL} & Gives containment, not area \\
Null hypersurface & \textcolor{red}{FAILED} & Area decreases both ways \\
Generalized Hawking mass & \textcolor{red}{FAILED} & Wrong direction inequality \\
Direct constraint bound & \textcolor{red}{FAILED} & Can't relate bulk to boundary \\
\hline
\end{tabular}
\end{center}

\subsection{The Fundamental Obstruction}

\begin{breakthrough}
\textbf{Why is area dominance so hard?}

\begin{enumerate}
    \item \textbf{Trapped surfaces have no variational characterization.} Unlike minimal surfaces (area extrema) or CMC surfaces (constrained extrema), trapped surfaces are defined by an inequality $\theta^\pm < 0$. There's no associated variational problem.
    
    \item \textbf{MOTS is not area-extremal.} The condition $\theta^+ = 0$ is not the same as critical point of area. MOTS can be perturbed to increase or decrease area.
    
    \item \textbf{Null geometry defeats flows.} Any flow that respects the null structure (expansions $\theta^\pm$) has area changing in the ``wrong'' direction. Flows that increase area don't respect the trapped/MOTS structure.
    
    \item \textbf{No canonical comparison.} Unlike the Riemannian case where minimal surfaces are outer-minimizing, there's no geometric reason why MOTS should have maximal area among trapped surfaces.
\end{enumerate}

\textbf{Conclusion:} Area dominance $A(\Sigma_0) \le A(\Sigma^*)$ may require either:
\begin{itemize}
    \item A fundamentally new geometric insight
    \item Additional physical assumptions (like WCC)
    \item Acceptance as a hypothesis in the theorem statement
\end{itemize}
\end{breakthrough}

\subsection{Remaining Hope: Spinor/Jang Bypass}

The most promising path remains:
\begin{enumerate}
    \item Use Jang equation to blow up at MOTS $\Sigma^*$
    \item Apply Riemannian Penrose on Jang manifold: $M \ge \sqrt{A(\Sigma^*)/(16\pi)}$
    \item Develop spinor methods to directly prove $M \ge \sqrt{A(\Sigma_0)/(16\pi)}$ for trapped $\Sigma_0$
\end{enumerate}

If the spinor approach can be made to work for trapped surfaces (not just MOTS), we bypass area dominance entirely.

\begin{thebibliography}{99}

\bibitem{anderssonmarssimion2008} L. Andersson, M. Mars, and W. Simon, Stability of marginally outer trapped surfaces and existence of marginally outer trapped tubes, \textit{Adv. Theor. Math. Phys.} 12 (2008), 853--888.

\bibitem{huiskenilmanen2001} G. Huisken and T. Ilmanen, The inverse mean curvature flow and the Riemannian Penrose inequality, \textit{J. Differential Geom.} 59 (2001), 353--437.

\bibitem{mars2009} M. Mars, Present status of the Penrose inequality, \textit{Classical Quantum Gravity} 26 (2009), 193001.

\end{thebibliography}

\end{document}
