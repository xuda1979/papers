%% SURGERY_METHOD_AREA_DOMINANCE.tex
%%
%% The Surgery Method: A Perelman-Inspired Attack
%% December 2025
%%
%% Key Insight: Instead of flowing, SURGICALLY MODIFY the geometry
%% to make Area Dominance obvious, while controlling how the surgery
%% affects the relevant quantities.

\documentclass[11pt]{amsart}
\usepackage{amsmath,amssymb,amsthm}
\usepackage{xcolor}
\usepackage{tcolorbox}

\tcbuselibrary{theorems}

\newtcolorbox{key_idea}{
    colback=blue!5!white,
    colframe=blue!75!black,
    title={\textbf{KEY IDEA}}
}

\newtcolorbox{breakthrough}{
    colback=yellow!10!white,
    colframe=red!75!black,
    title={\textbf{BREAKTHROUGH}}
}

\newtcolorbox{proof_step}{
    colback=green!5!white,
    colframe=green!75!black,
}

\newtheorem{theorem}{Theorem}
\newtheorem{lemma}[theorem]{Lemma}
\newtheorem{proposition}[theorem]{Proposition}
\newtheorem{corollary}[theorem]{Corollary}
\theoremstyle{definition}
\newtheorem{definition}[theorem]{Definition}
\newtheorem{remark}[theorem]{Remark}

\newcommand{\Area}{\mathrm{Area}}
\newcommand{\Vol}{\mathrm{Vol}}
\newcommand{\divv}{\mathrm{div}}
\newcommand{\Ric}{\mathrm{Ric}}
\DeclareMathOperator{\tr}{tr}

\title{The Surgery Method for Area Dominance:\\
Geometric Modification and Monotonicity}
\author{December 2025}

\begin{document}
\maketitle

\begin{abstract}
Inspired by Perelman's surgery techniques in the proof of the Poincaré 
conjecture, we develop a SURGERY-BASED approach to Area Dominance.
The method involves geometrically modifying the initial data while 
controlling the area ratio.
\end{abstract}

%% ============================================================================
\section{The Surgery Philosophy}
%% ============================================================================

\begin{key_idea}
\textbf{Perelman's approach to Poincaré:}
\begin{enumerate}
    \item Run Ricci flow until a singularity forms
    \item Identify the singular region (a "neck")
    \item Cut out the neck and cap off with standard geometry
    \item Continue the flow on the modified manifold
    \item Repeat until the manifold simplifies
\end{enumerate}

\textbf{Our approach to Area Dominance:}
\begin{enumerate}
    \item Start with initial data having trapped $\Sigma$ and MOTS $\Sigma^*$
    \item Identify "bad" regions where $H < 0$ (causing area decrease)
    \item Surgically modify these regions to have $H \ge 0$
    \item Show the surgery doesn't decrease $\Area(\Sigma^*) - \Area(\Sigma)$
    \item In the modified geometry, Area Dominance is straightforward
\end{enumerate}
\end{key_idea}

%% ============================================================================
\section{The Geometric Setup}
%% ============================================================================

\subsection{The Region of Interest}

Let $\Omega$ be the region between $\Sigma$ (inner) and $\Sigma^*$ (outer).

On $\Sigma$: $\theta^+ < 0$ (trapped)
On $\Sigma^*$: $\theta^+ = 0$ (MOTS)

\subsection{Decomposition by Sign of $H$}

Decompose $\Sigma = \Sigma_+ \cup \Sigma_-$ where:
\begin{itemize}
    \item $\Sigma_+ = \{p \in \Sigma : H(p) \ge 0\}$
    \item $\Sigma_- = \{p \in \Sigma : H(p) < 0\}$
\end{itemize}

On $\Sigma_+$: $H \ge 0$ and $\theta^+ = H + P < 0$ implies $P < -H \le 0$.

On $\Sigma_-$: $H < 0$, and $P = \theta^+ - H > \theta^+$.

\subsection{The Problem Region}

The "problem" is $\Sigma_-$ where $H < 0$.

On $\Sigma_-$: moving outward DECREASES area (since $dA/dt = H\phi \cdot dA < 0$ for $\phi > 0$).

%% ============================================================================
\section{The Surgery Procedure}
%% ============================================================================

\begin{breakthrough}
\textbf{THE SURGERY:}

Replace the region near $\Sigma_-$ with a "standard model" where $H \ge 0$.

This is analogous to Perelman's neck surgery where singular regions are 
replaced with caps.
\end{breakthrough}

\subsection{Step 1: Identify the Bad Region}

Let $U_- \subset \mathcal{C}$ be a neighborhood of $\Sigma_-$.

$U_-$ is where the geometry causes problems.

\subsection{Step 2: Cut and Cap}

Remove $U_-$ from $\mathcal{C}$.

This creates a boundary.

Glue in a "cap" - a region with controlled geometry where $H \ge 0$ on all 
surfaces.

\subsection{Step 3: The Standard Cap}

\begin{definition}[Standard Cap]
A standard cap is a region diffeomorphic to $B^3$ (3-ball) with:
\begin{itemize}
    \item Metric $g_{cap}$ with $R_{cap} \ge 0$
    \item All surfaces have $H \ge 0$ (mean-convex)
    \item Boundary matches the cut boundary of $\mathcal{C} \setminus U_-$
\end{itemize}
\end{definition}

\textbf{Example:} A portion of a round 3-sphere.

\subsection{Step 4: Matching Conditions}

At the gluing boundary, we need:
\begin{itemize}
    \item Continuous metric
    \item Continuous (or controlled jump in) extrinsic curvature $k$
    \item The constraint equations still satisfied
\end{itemize}

%% ============================================================================
\section{How Surgery Affects Areas}
%% ============================================================================

\begin{lemma}[Area Change Under Surgery]
Let $\Sigma'$ be the surface in the surgered manifold corresponding to $\Sigma$.

If the surgery removes the region where $H < 0$ and replaces with $H \ge 0$:
\begin{equation}
    \Area(\Sigma') \le \Area(\Sigma) + C \cdot \text{(surgery parameters)}
\end{equation}

where $C$ depends on the geometry of the cap.
\end{lemma}

\textbf{Key point:} We need to control how much area is added by the surgery.

\subsection{The Area Balance}

Before surgery: $\Area(\Sigma) = \Area(\Sigma_+) + \Area(\Sigma_-)$

After surgery: $\Area(\Sigma') = \Area(\Sigma_+) + \Area(\text{cap boundary})$

For Area Dominance, we need:
\begin{equation}
    \Area(\text{cap boundary}) \le \Area(\Sigma_-)
\end{equation}

OR we need to track how $\Sigma^*$ also changes and maintain $\Area(\Sigma') \le \Area(\Sigma^{*'})$.

%% ============================================================================
\section{The Monotonicity Under Surgery}
%% ============================================================================

\begin{proposition}
If the surgery satisfies:
\begin{enumerate}
    \item The cap has $H \ge 0$ on all internal surfaces
    \item The cap has scalar curvature $R \ge 0$
    \item The surgery does not affect $\Sigma^*$ (or affects it favorably)
\end{enumerate}

Then $\Area(\Sigma') \le \Area(\Sigma^{*'})$ in the surgered manifold implies 
(under certain conditions) the original inequality.
\end{proposition}

\textbf{Issue:} The surgery might disconnect $\Sigma$ from $\Sigma^*$!

If we cut out a region between them, the topological relationship changes.

%% ============================================================================
\section{A Better Approach: Conformal Surgery}
%% ============================================================================

\begin{key_idea}
Instead of cutting and gluing, use a CONFORMAL CHANGE to modify the metric 
in the bad region.

This preserves topology while changing geometry.
\end{key_idea}

\subsection{Conformal Modification}

Let $\phi: \mathcal{C} \to \mathbb{R}_{>0}$ be a conformal factor.

New metric: $\tilde{g} = \phi^4 g$.

Under conformal change:
\begin{itemize}
    \item Area: $\Area_{\tilde{g}}(\Sigma) = \int_\Sigma \phi^4 dA_g$
    \item Mean curvature: $\tilde{H} = \phi^{-2}(H + 4\nu(\log\phi))$
    \item Scalar curvature: $\tilde{R} = \phi^{-5}(-8\Delta\phi + R\phi)$
\end{itemize}

\subsection{Choosing $\phi$ to Fix $H$}

We want $\tilde{H} \ge 0$ on $\Sigma$.

This requires:
\begin{equation}
    H + 4\nu(\log\phi) \ge 0
\end{equation}
\begin{equation}
    \nu(\log\phi) \ge -\frac{H}{4}
\end{equation}

On $\Sigma_-$ where $H < 0$: $-H/4 > 0$, so we need $\phi$ INCREASING in the 
outward direction.

\subsection{Constructing $\phi$}

Near $\Sigma$, let $r$ be the signed distance to $\Sigma$ (positive outward).

Define:
\begin{equation}
    \phi(r) = 1 + \epsilon f(r)
\end{equation}

where $f(r)$ is chosen so that $\phi'(0)/\phi(0) \ge -H/4$ on $\Sigma_-$.

On $\Sigma_+$ (where $H \ge 0$), we can take $\phi = 1$.

%% ============================================================================
\section{The Conformal Area Bound}
%% ============================================================================

\begin{lemma}
With the conformal factor $\phi$ as above:
\begin{equation}
    \Area_{\tilde{g}}(\Sigma) = \int_{\Sigma_+} \phi^4 dA + \int_{\Sigma_-} \phi^4 dA
\end{equation}

On $\Sigma_+$: $\phi = 1$, so contribution is $\Area(\Sigma_+)$.

On $\Sigma_-$: $\phi > 1$ (since $\phi' > 0$), so contribution is $> \Area(\Sigma_-)$.

Therefore:
\begin{equation}
    \Area_{\tilde{g}}(\Sigma) > \Area_g(\Sigma)
\end{equation}
\end{lemma}

\textbf{This is the WRONG direction!}

We've made $\Sigma$ LARGER in the conformal metric.

\subsection{The Resolution}

The key is that $\Sigma^*$ also changes!

\begin{equation}
    \Area_{\tilde{g}}(\Sigma^*) = \int_{\Sigma^*} \phi^4 dA
\end{equation}

If $\phi > 1$ on $\Sigma^*$ as well (and more so than on $\Sigma$), then:
\begin{equation}
    \frac{\Area_{\tilde{g}}(\Sigma)}{\Area_{\tilde{g}}(\Sigma^*)} < \frac{\Area_g(\Sigma)}{\Area_g(\Sigma^*)}
\end{equation}

This would give Area Dominance improvement!

%% ============================================================================
\section{The Main Theorem Attempt}
%% ============================================================================

\begin{theorem}[Conformal Area Dominance - Conditional]
Let $(\mathcal{C}, g, k)$ be initial data with trapped $\Sigma$ and MOTS $\Sigma^*$.

Suppose there exists a conformal factor $\phi$ such that:
\begin{enumerate}
    \item $\tilde{H} \ge 0$ on $\Sigma$ (in the conformal metric)
    \item $\phi|_{\Sigma} \le \phi|_{\Sigma^*}$ (pointwise or in average)
    \item The conformal metric $\tilde{g}$ satisfies $\tilde{R} \ge 0$
\end{enumerate}

Then $\Area(\Sigma) \le \Area(\Sigma^*)$.
\end{theorem}

\begin{proof_step}
\textbf{Proof:}

In the conformal metric $\tilde{g}$:
\begin{itemize}
    \item $\tilde{H} \ge 0$ on $\Sigma$ (by construction)
    \item $\tilde{R} \ge 0$ (assumed)
\end{itemize}

Since $\Sigma$ has $\tilde{H} \ge 0$, flowing outward from $\Sigma$ INCREASES area.

The flow from $\Sigma$ to $\Sigma^*$ (suitably defined) has increasing area.

Therefore in the $\tilde{g}$ metric:
\begin{equation}
    \Area_{\tilde{g}}(\Sigma) \le \Area_{\tilde{g}}(\Sigma^*)
\end{equation}

By condition 2:
\begin{equation}
    \frac{\Area_{\tilde{g}}(\Sigma)}{\Area_g(\Sigma)} = \int \phi^4/\Area \le \int \phi^4/\Area = \frac{\Area_{\tilde{g}}(\Sigma^*)}{\Area_g(\Sigma^*)}
\end{equation}

Hmm, this doesn't directly give the result. Let me reconsider.

\textbf{Alternative:}

If $\phi \equiv 1$ near $\Sigma^*$ and $\phi \ge 1$ near $\Sigma$:
\begin{align}
    \Area_{\tilde{g}}(\Sigma) &\ge \Area_g(\Sigma)\\
    \Area_{\tilde{g}}(\Sigma^*) &= \Area_g(\Sigma^*)
\end{align}

And in the conformal metric, Area Dominance gives:
\begin{equation}
    \Area_{\tilde{g}}(\Sigma) \le \Area_{\tilde{g}}(\Sigma^*) = \Area_g(\Sigma^*)
\end{equation}

But we have $\Area_{\tilde{g}}(\Sigma) \ge \Area_g(\Sigma)$, so:
\begin{equation}
    \Area_g(\Sigma) \le \Area_{\tilde{g}}(\Sigma) \le \Area_g(\Sigma^*)
\end{equation}

\textbf{This works IF we can prove Area Dominance in the conformal metric!}
\end{proof_step}

%% ============================================================================
\section{Why This Might Work}
%% ============================================================================

In the conformal metric with $\tilde{H} \ge 0$ on $\Sigma$:

\textbf{Claim:} Area Dominance holds for surfaces with $\tilde{H} \ge 0$ inside 
a MOTS (or minimal surface).

\textbf{Reason:} With $\tilde{H} \ge 0$, moving outward increases area.

But wait - $\Sigma$ is no longer trapped in the conformal metric! (We made $\tilde{H} \ge 0$.)

The relationship between $\Sigma$ and $\Sigma^*$ changes.

\subsection{What Happens to $\Sigma^*$?}

In the original metric: $\Sigma^*$ has $\theta^+ = H + P = 0$.

In conformal metric: 
\begin{equation}
    \tilde{\theta}^+ = \tilde{H} + \tilde{P}
\end{equation}

The trace $\tilde{P}$ also transforms:
\begin{equation}
    \tilde{P} = \phi^{-2}P \quad \text{(approximately)}
\end{equation}

So $\tilde{\theta}^+ = \phi^{-2}(H + 4\nu(\log\phi)) + \phi^{-2}P = \phi^{-2}(H + P + 4\nu(\log\phi))$

At $\Sigma^*$ with $H + P = 0$:
\begin{equation}
    \tilde{\theta}^+|_{\Sigma^*} = \phi^{-2} \cdot 4\nu(\log\phi)|_{\Sigma^*}
\end{equation}

If $\phi$ is chosen appropriately, $\Sigma^*$ might still be a MOTS or become trapped or untrapped.

%% ============================================================================
\section{The Complete Strategy}
%% ============================================================================

\begin{enumerate}
    \item \textbf{Find conformal factor $\phi$} such that:
    \begin{itemize}
        \item $\tilde{H} \ge 0$ on $\Sigma$ (fixes the sign problem)
        \item $\phi = 1$ on $\Sigma^*$ (preserves MOTS)
        \item $\tilde{R} \ge 0$ (non-negative scalar curvature)
    \end{itemize}
    
    \item \textbf{Prove Area Dominance in $\tilde{g}$:}
    
    With $\tilde{H} \ge 0$ on $\Sigma$ and $\Sigma^*$ still a MOTS, flowing from 
    $\Sigma$ to $\Sigma^*$ increases area.
    
    \item \textbf{Transfer back:}
    
    Since $\phi = 1$ on $\Sigma^*$: $\Area_{\tilde{g}}(\Sigma^*) = \Area_g(\Sigma^*)$.
    
    Since $\phi \ge 1$ on $\Sigma$: $\Area_{\tilde{g}}(\Sigma) \ge \Area_g(\Sigma)$.
    
    Area Dominance in $\tilde{g}$: $\Area_{\tilde{g}}(\Sigma) \le \Area_{\tilde{g}}(\Sigma^*) = \Area_g(\Sigma^*)$.
    
    Therefore: $\Area_g(\Sigma) \le \Area_{\tilde{g}}(\Sigma) \le \Area_g(\Sigma^*)$.
\end{enumerate}

%% ============================================================================
\section{The Technical Challenge}
%% ============================================================================

\textbf{The main technical challenge:}

Finding $\phi$ with:
\begin{enumerate}
    \item $\phi = 1$ on $\Sigma^*$
    \item $\phi \ge 1$ everywhere with $\phi > 1$ where $H < 0$
    \item $\nu(\log\phi) \ge -H/4$ on $\Sigma$ (to ensure $\tilde{H} \ge 0$)
    \item $-8\Delta\phi + R\phi \ge 0$ (to ensure $\tilde{R} \ge 0$)
\end{enumerate}

This is an elliptic PDE problem with boundary conditions and inequalities.

\subsection{The PDE Problem}

Find $\phi$ solving:
\begin{align}
    -8\Delta\phi + R\phi &\ge 0 \quad \text{in } \Omega\\
    \phi &= 1 \quad \text{on } \Sigma^*\\
    \nu(\phi) &\ge -\frac{H\phi}{4} \quad \text{on } \Sigma
\end{align}

This is a variational inequality / obstacle problem.

%% ============================================================================
\section{Conclusion}
%% ============================================================================

The surgery/conformal method offers a path to Area Dominance:

\textbf{Strengths:}
\begin{itemize}
    \item Directly addresses the sign-of-$H$ problem
    \item Reduces to a PDE problem for $\phi$
    \item Preserves the overall structure of the proof
\end{itemize}

\textbf{Challenges:}
\begin{itemize}
    \item Finding $\phi$ satisfying all constraints
    \item Proving Area Dominance in the conformal metric
    \item Handling cases where $\phi$ cannot exist
\end{itemize}

\textbf{This approach warrants further development.}

\end{document}
