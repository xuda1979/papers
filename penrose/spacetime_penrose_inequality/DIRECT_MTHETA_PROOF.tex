%% DIRECT_MTHETA_PROOF.tex
%%
%% Direct Proof Attempt: M_ADM ≥ m_θ(Σ) for Any Surface Σ
%%
%% This bypasses the area dominance problem entirely!
%%
%% December 2025

\documentclass[11pt]{amsart}
\usepackage{amsmath,amssymb,amsthm}
\usepackage{xcolor}
\usepackage{tcolorbox}

\tcbuselibrary{theorems}

\newtcolorbox{keyresult}{
    colback=green!5!white,
    colframe=green!75!black,
    title={\textbf{KEY RESULT}}
}

\newtcolorbox{gap}{
    colback=red!5!white,
    colframe=red!75!black,
    title={\textbf{GAP}}
}

\newtcolorbox{proofbox}{
    colback=blue!5!white,
    colframe=blue!75!black,
}

\newtheorem{theorem}{Theorem}[section]
\newtheorem{lemma}[theorem]{Lemma}
\newtheorem{proposition}[theorem]{Proposition}
\newtheorem{corollary}[theorem]{Corollary}
\newtheorem{conjecture}[theorem]{Conjecture}
\newtheorem{definition}[theorem]{Definition}

\newcommand{\ADM}{\mathrm{ADM}}
\newcommand{\Area}{\mathrm{Area}}
\newcommand{\tr}{\mathrm{tr}}
\newcommand{\mH}{m_H}
\newcommand{\mtheta}{m_\theta}
\newcommand{\mB}{m_B}
\newcommand{\dive}{\mathrm{div}}

\title{Direct Proof of $M_{\ADM} \ge m_\theta(\Sigma)$\\
\large The $\theta^+$-Weighted Spacetime Penrose Inequality}
\author{}
\date{December 2025}

\begin{document}
\maketitle

\begin{abstract}
We attempt a direct proof that $M_{\ADM} \ge m_\theta(\Sigma)$ for any closed surface $\Sigma$ in asymptotically flat initial data satisfying DEC, where $m_\theta$ is the $\theta^+$-weighted Hawking mass. This would imply the spacetime Penrose inequality without requiring the problematic area dominance step.
\end{abstract}

\tableofcontents

%% ============================================================================
\section{The Goal}
%% ============================================================================

\begin{definition}[$\theta^+$-Weighted Hawking Mass]
For a closed surface $\Sigma$ in initial data $(M, g, k)$:
\begin{equation}
    \mtheta(\Sigma) := \sqrt{\frac{\Area(\Sigma)}{16\pi}}\left(1 - \frac{1}{16\pi}\int_\Sigma (\theta^+)^2 dA\right)
\end{equation}
where $\theta^+ = H + \tr_\Sigma k$ is the outgoing null expansion.
\end{definition}

\textbf{Properties:}
\begin{itemize}
    \item For MOTS ($\theta^+ = 0$): $\mtheta(\Sigma^*) = \sqrt{A^*/(16\pi)}$
    \item For trapped ($\theta^+ < 0$): $\mtheta(\Sigma) < \sqrt{A/(16\pi)}$
    \item For untrapped: $\mtheta$ could be negative (but we focus on trapped region)
\end{itemize}

\begin{conjecture}[Main Result]
For any asymptotically flat $(M, g, k)$ satisfying DEC and any closed surface $\Sigma$:
\begin{equation}
    M_{\ADM}(g, k) \ge \mtheta(\Sigma)
\end{equation}
\end{conjecture}

%% ============================================================================
\section{Strategy: Conformal Flow}
%% ============================================================================

\textbf{Idea:} Conformally deform the metric $g$ to $\tilde{g} = u^4 g$ such that:
\begin{enumerate}
    \item $\Sigma$ becomes minimal in $\tilde{g}$
    \item $M_{\ADM}(\tilde{g}) = \mtheta(\Sigma)$
    \item DEC on $(M, g, k)$ implies $R_{\tilde{g}} \ge 0$
\end{enumerate}

Then by positive mass theorem: $M_{\ADM}(\tilde{g}) \ge 0$ (but we need more).

\subsection{Conformal Factor Equation}

To make $\Sigma$ minimal in $\tilde{g} = u^4 g$, we need:
\begin{equation}
    \tilde{H} = u^{-2}\left(H + 4\frac{\partial u}{\partial \nu}/u\right) = 0
\end{equation}

So:
\begin{equation}
    \frac{\partial u}{\partial \nu} = -\frac{H}{4}u \quad \text{on } \Sigma
\end{equation}

\subsection{Mass Transformation}

Under conformal change $\tilde{g} = u^4 g$ with $u \to 1$ at infinity:
\begin{equation}
    M_{\ADM}(\tilde{g}) = M_{\ADM}(g) - \frac{1}{2\pi}\lim_{r\to\infty}\int_{S_r} \frac{\partial u}{\partial r} dA
\end{equation}

For $u = 1 + \frac{\alpha}{r} + O(r^{-2})$:
\begin{equation}
    M_{\ADM}(\tilde{g}) = M_{\ADM}(g) + 2\alpha
\end{equation}

To get $M_{\ADM}(\tilde{g}) = \mtheta(\Sigma)$, we need:
\begin{equation}
    \alpha = \frac{1}{2}\left[\mtheta(\Sigma) - M_{\ADM}(g)\right]
\end{equation}

\begin{gap}
We want to prove $M_{\ADM}(g) \ge \mtheta(\Sigma)$, which would require $\alpha \le 0$.

But $\alpha$ is determined by the solution to the conformal equation, not freely choosable.
\end{gap}

%% ============================================================================
\section{Strategy: Geroch-Style Monotonicity}
%% ============================================================================

\textbf{Idea:} Find a flow from $\Sigma$ to infinity where some mass-like quantity is monotonic.

\subsection{Standard Geroch Flow}

For IMCF $\Sigma_t$ with $\partial_t \mathbf{x} = H^{-1}\nu$, the Hawking mass satisfies:
\begin{equation}
    \frac{d\mH(\Sigma_t)}{dt} = \frac{\sqrt{A}}{(16\pi)^{3/2}} \int_{\Sigma_t} \left[\frac{|\nabla H|^2}{H^2} + (\mu - J \cdot \nu) H^{-1}\right] dA
\end{equation}

With DEC: $\mu - |J| \ge 0$, so the second term has good sign if $H > 0$.

\textbf{Problem:} Requires $H > 0$. Trapped surfaces can have $H \le 0$.

\subsection{Modified Flow for Trapped Surfaces}

Try flow with speed $\theta^+ = H + \tr_\Sigma k$ instead of $H$.

\textbf{Problem:} For trapped surfaces, $\theta^+ < 0$, so this flows inward!

\subsection{θ⁺-Weighted Mass Monotonicity}

Consider the evolution of $\mtheta$ along a general flow $\partial_t \mathbf{x} = v\nu$:

\begin{align}
    \frac{d}{dt}\mtheta(\Sigma_t) &= \frac{d}{dt}\left[\sqrt{\frac{A}{16\pi}}\left(1 - \frac{1}{16\pi}\int (\theta^+)^2 dA\right)\right]
\end{align}

\textbf{This is very complicated!} The $(\theta^+)^2$ term evolves non-trivially.

%% ============================================================================
\section{Strategy: Jang Surface Extension}
%% ============================================================================

\textbf{Idea:} Use Jang surface to convert the problem to Riemannian.

\subsection{Jang Equation Review}

The Jang equation $\bar{H}_f = \tr_{\Sigma_f} k$ transforms $(M, g, k)$ to $(\bar{M}, \bar{g})$ where:
\begin{itemize}
    \item $R_{\bar{g}} \ge 0$ (distributionally)
    \item MOTS become minimal surfaces
    \item $M_{\ADM}(g, k) \ge M_{\ADM}(\bar{g})$
\end{itemize}

\subsection{Trapped Surface on Jang Surface}

What happens to a trapped surface $\Sigma$ (with $\theta^+ < 0$) on the Jang surface?

The Jang surface blows up only at MOTS. For trapped surfaces:
\begin{itemize}
    \item $\Sigma$ remains a finite surface in $(\bar{M}, \bar{g})$
    \item The mean curvature $\bar{H}$ of $\Sigma$ in $\bar{g}$ is NOT zero
\end{itemize}

\subsection{Hawking Mass on Jang Surface}

In the Jang metric:
\begin{equation}
    \bar{m}_H(\Sigma) = \sqrt{\frac{\bar{A}}{16\pi}}\left(1 - \frac{1}{16\pi}\int_\Sigma \bar{H}^2 d\bar{A}\right)
\end{equation}

We have:
\begin{itemize}
    \item $\bar{A} = A$ (same area)
    \item $\bar{H} = H - (\text{Jang contribution})$
\end{itemize}

The Jang contribution relates to $\tr_\Sigma k$, and for MOTS: $\bar{H} = 0$.

For trapped surfaces: $\bar{H}$ has no definite sign.

%% ============================================================================
\section{Strategy: Direct Spinor Approach}
%% ============================================================================

The most promising approach uses Witten spinors with modified boundary conditions.

\subsection{Witten Identity}

For a spinor $\psi$ on $(M, g, k)$:
\begin{equation}
    M_{\ADM} = \frac{1}{4\pi}\int_M |D_A \psi|^2 + \frac{1}{8\pi}\int_M (\mu - J \cdot \nu) |\psi|^2 + \text{boundary}_\Sigma
\end{equation}

where:
\begin{itemize}
    \item $D_A$ is the Sen connection
    \item $\mu = \frac{1}{2}(R_g - |k|^2 + (\tr k)^2)$ (energy density)
    \item $J = \dive(k - (\tr k)g)$ (momentum density)
    \item DEC: $\mu \ge |J|$
\end{itemize}

\subsection{Boundary Term at Σ}

The boundary contribution at $\Sigma$ is:
\begin{equation}
    \text{boundary}_\Sigma = \frac{1}{4\pi}\int_\Sigma \langle \psi, \gamma_\nu D_\nu \psi \rangle dA
\end{equation}

With the Sen connection:
\begin{equation}
    D_\nu \psi = \nabla_\nu \psi + \frac{i}{2}k_{\nu\mu}\gamma^\mu \psi
\end{equation}

\subsection{Boundary Condition}

Choose boundary condition:
\begin{equation}
    (D_\nu - \frac{\theta^+}{2}\gamma_\nu)\psi = 0 \quad \text{on } \Sigma
\end{equation}

Then:
\begin{equation}
    \text{boundary}_\Sigma = \frac{1}{8\pi}\int_\Sigma \theta^+ |\psi|^2 dA
\end{equation}

\subsection{Main Identity}

\begin{equation}
    M_{\ADM} = \frac{1}{4\pi}\int_M |D_A \psi|^2 + \frac{1}{8\pi}\int_M (\mu - J \cdot \nu) |\psi|^2 + \frac{1}{8\pi}\int_\Sigma \theta^+ |\psi|^2 dA
\end{equation}

With DEC: first two terms are $\ge 0$.

For MOTS ($\theta^+ = 0$): boundary term vanishes, giving $M_{\ADM} \ge 0$.

For trapped ($\theta^+ < 0$): boundary term is negative!

\begin{gap}
The boundary term $\frac{1}{8\pi}\int_\Sigma \theta^+ |\psi|^2 dA < 0$ for trapped surfaces.

The identity becomes:
\begin{equation}
    M_{\ADM} \ge \text{(positive bulk)} + \frac{1}{8\pi}\int_\Sigma \theta^+ |\psi|^2 dA
\end{equation}

This could give $M_{\ADM} > $ (negative number), which is not useful.
\end{gap}

\subsection{Resolution Attempt: Optimize Over ψ}

The spinor $\psi$ is not arbitrary—it must satisfy the Dirac equation with boundary conditions.

Key observation: the optimal $|\psi|^2$ on $\Sigma$ should relate to the area.

\textbf{Conjecture:} There exists a spinor solution with:
\begin{equation}
    \frac{1}{8\pi}\int_\Sigma \theta^+ |\psi|^2 dA \ge -\sqrt{\frac{A}{16\pi}} + \mtheta(\Sigma)
\end{equation}

If true:
\begin{equation}
    M_{\ADM} \ge \text{(positive)} + \mtheta(\Sigma) - \sqrt{\frac{A}{16\pi}}
\end{equation}

This is still not quite right...

%% ============================================================================
\section{Strategy: Two-Surface Approach}
%% ============================================================================

\textbf{Key insight:} Use BOTH $\Sigma$ and $\Sigma^*$ in the argument.

\subsection{Setup}

Let $\Sigma$ be trapped, $\Sigma^*$ be MOTS with $\Sigma$ inside $\Sigma^*$.

Apply Witten identity to the region $M \setminus \Omega$ where $\Omega$ is bounded by $\Sigma$.

\subsection{Modified Identity}

\begin{equation}
    M_{\ADM} = \frac{1}{4\pi}\int_{M\setminus\Omega} |D_A\psi|^2 + \frac{1}{8\pi}\int_{M\setminus\Omega} (\mu - |J|)|\psi|^2 + \text{boundary}_\Sigma
\end{equation}

The boundary term at $\Sigma$ (with outward normal pointing INTO $\Omega$):
\begin{equation}
    \text{boundary}_\Sigma = -\frac{1}{8\pi}\int_\Sigma \theta^+ |\psi|^2 dA > 0
\end{equation}

(Positive because $\theta^+ < 0$ and we have a minus sign from orientation!)

\begin{keyresult}
With proper orientation: boundary term at trapped surface is POSITIVE!
\end{keyresult}

\subsection{Bound}

\begin{equation}
    M_{\ADM} \ge -\frac{1}{8\pi}\int_\Sigma \theta^+ |\psi|^2 dA
\end{equation}

With $|\psi|^2 = $ constant $= c$ on $\Sigma$:
\begin{equation}
    M_{\ADM} \ge -\frac{c}{8\pi}\int_\Sigma \theta^+ dA
\end{equation}

\begin{gap}
This gives a bound in terms of $\int \theta^+ dA$, not $\int (\theta^+)^2 dA$.

For MOTS: $\int \theta^+ dA = 0$, giving $M_{\ADM} \ge 0$.

For trapped: $\int \theta^+ dA < 0$, giving $M_{\ADM} \ge $ (positive)!

But how does this relate to $\sqrt{A/(16\pi)}$?
\end{gap}

\subsection{Optimal Choice of c}

The constant $c = |\psi|^2$ on $\Sigma$ is not free—determined by the Dirac equation.

However, by normalizing: if $\psi$ solves the equation, so does $\lambda\psi$ for any $\lambda$.

Choose normalization such that the boundary term gives the best bound.

\textbf{Dimensional analysis:} 
\begin{itemize}
    \item $\int \theta^+ dA$ has dimension $[\text{length}]$
    \item $|\psi|^2$ has dimension $[\text{length}]^{-2}$ (in geometric units)
    \item Bound has dimension $[\text{mass}]$ ✓
\end{itemize}

%% ============================================================================
\section{Key Calculation}
%% ============================================================================

Let's compute more carefully.

\subsection{Spinor Normalization}

The Witten spinor $\psi$ satisfies:
\begin{itemize}
    \item $D_A \psi = 0$ (Dirac equation with Sen connection)
    \item $|\psi| \to 1$ at infinity
\end{itemize}

This fixes the normalization.

\subsection{Maximum Principle for |ψ|²}

By maximum principle for harmonic spinors:
\begin{equation}
    \max_\Sigma |\psi|^2 \ge 1 \quad \text{(value at infinity)}
\end{equation}

In fact, $|\psi|^2$ achieves its maximum on $\Sigma$ if $\Sigma$ is the boundary.

\subsection{Rigorous Lower Bound}

From $M_{\ADM} \ge -\frac{1}{8\pi}\int_\Sigma \theta^+ |\psi|^2 dA$:

If we can show $|\psi|^2 \ge c_0$ on $\Sigma$ for some $c_0 > 0$:
\begin{equation}
    M_{\ADM} \ge -\frac{c_0}{8\pi}\int_\Sigma \theta^+ dA = \frac{c_0}{8\pi}|\theta^+|_{\text{avg}} \cdot A
\end{equation}

where $|\theta^+|_{\text{avg}} = -\frac{1}{A}\int \theta^+ dA > 0$.

\begin{gap}
This bound involves $|\theta^+|_{\text{avg}} \cdot A$, not $\sqrt{A}$.

For dimensional consistency with Penrose: need $\sqrt{A/(16\pi)}$.

The bound $\frac{c_0}{8\pi}|\theta^+|_{\text{avg}} \cdot A$ has wrong dimension unless $|\theta^+|_{\text{avg}} \sim A^{-1/2}$.

For round spheres: $|\theta^+| \sim r^{-1} \sim A^{-1/2}$, so this could work!
\end{gap}

%% ============================================================================
\section{Spherical Symmetry Check}
%% ============================================================================

\subsection{Setup}

In Schwarzschild with mass $m$, consider a trapped sphere at area radius $r < 2m$.

We have:
\begin{itemize}
    \item Area $A = 4\pi r^2$
    \item Outgoing null expansion: $\theta^+ = \frac{2}{r}(1 - \frac{2m}{r}) < 0$
    \item Average: $|\theta^+|_{\text{avg}} = \frac{2}{r}(\frac{2m}{r} - 1)$
\end{itemize}

\subsection{Check the Bound}

Our bound: $M_{\ADM} \ge \frac{c_0}{8\pi} \cdot \frac{2}{r}(\frac{2m}{r} - 1) \cdot 4\pi r^2 = c_0 \cdot r \cdot (\frac{2m}{r} - 1) = c_0(2m - r)$

With $c_0 = 1$: $m \ge 2m - r \Rightarrow r \ge m$.

This is satisfied for $r > m$ but NOT for $r < m$. Gap!

\textbf{The bound is wrong for very trapped surfaces.}

\subsection{What We Need}

We need:
\begin{equation}
    M_{\ADM} = m \ge \sqrt{\frac{4\pi r^2}{16\pi}} = \frac{r}{2}
\end{equation}

Since $r < 2m$, we have $\frac{r}{2} < m$. ✓ Penrose inequality holds.

But our spinor bound gives $m \ge c_0(2m - r)$, which fails for small $r$.

\begin{gap}
The direct spinor bound with boundary $\int \theta^+ |\psi|^2 dA$ is too weak for very trapped surfaces.

The $\theta^+$-\textit{weighted mass} includes a $(\theta^+)^2$ correction, not just $\theta^+$.
\end{gap}

%% ============================================================================
\section{Refined Approach: Use Both θ⁺ and (θ⁺)²}
%% ============================================================================

\subsection{Modified Boundary Condition}

Instead of $D_\nu \psi = \frac{\theta^+}{2}\gamma_\nu \psi$, consider:
\begin{equation}
    D_\nu \psi = \left(\frac{\theta^+}{2} + f(\theta^+)\right)\gamma_\nu \psi
\end{equation}

where $f$ is chosen to produce $(\theta^+)^2$ terms.

\subsection{Boundary Term}

The boundary term becomes:
\begin{equation}
    \text{boundary} = \frac{1}{8\pi}\int_\Sigma \left(\theta^+ + 2f(\theta^+)\right) |\psi|^2 dA
\end{equation}

\subsection{Choose f to Get Penrose}

We want the bound:
\begin{equation}
    M_{\ADM} \ge \sqrt{\frac{A}{16\pi}}\left(1 - \frac{1}{16\pi}\int (\theta^+)^2 dA\right) = \mtheta(\Sigma)
\end{equation}

This requires a very specific relationship between $|\psi|^2$, $\theta^+$, and $A$.

\textbf{This is the core unsolved problem.}

%% ============================================================================
\section{Conclusion and Status}
%% ============================================================================

\begin{tcolorbox}[colback=yellow!10!white, colframe=yellow!75!black, title=\textbf{DIRECT PROOF STATUS}]

\textbf{Attempted approaches:}
\begin{enumerate}
    \item Conformal flow: Works for MOTS, fails for trapped
    \item Geroch-style monotonicity: Requires $H > 0$
    \item Jang surface: Gives MOTS result, unclear for trapped
    \item Spinor (basic): Wrong power of $\theta^+$
    \item Two-surface: Promising but gives $\theta^+$, not $(\theta^+)^2$
\end{enumerate}

\textbf{Key insight:} The $\theta^+$-weighted mass $\mtheta$ with the $(\theta^+)^2$ term is geometrically natural but hard to capture with standard techniques.

\textbf{What remains:}
\begin{itemize}
    \item Find spinor/conformal construction that produces $(\theta^+)^2$ term
    \item Or prove area dominance via spacetime methods (WCC)
    \item Or find entirely new approach
\end{itemize}
\end{tcolorbox}

\end{document}
