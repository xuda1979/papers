% =========================================================================
%     FINDING UNFAVORABLE TRAPPED SURFACES IN SCHWARZSCHILD
%
%     Constructing test cases with tr_Σ k < 0
%
%     Author: Da Xu
%     Date: December 2025
% =========================================================================

\documentclass[12pt]{article}
\usepackage{amsmath,amsthm,amssymb}
\usepackage{mathrsfs}
\usepackage{tcolorbox}

\theoremstyle{plain}
\newtheorem{theorem}{Theorem}[section]
\newtheorem{lemma}[theorem]{Lemma}
\newtheorem{proposition}[theorem]{Proposition}

\theoremstyle{definition}
\newtheorem{definition}[theorem]{Definition}
\newtheorem{example}[theorem]{Example}
\newtheorem{calculation}[theorem]{Calculation}

\newcommand{\ADM}{\mathrm{ADM}}
\newcommand{\tr}{\mathrm{tr}}

\title{\textbf{Constructing Unfavorable Trapped Surfaces in Schwarzschild}}
\author{Da Xu}
\date{December 2025}

\begin{document}
\maketitle

\section{Goal}

Find trapped surfaces $\Sigma$ in Schwarzschild spacetime with:
\begin{itemize}
    \item $\theta^+ = H + \tr_\Sigma k \leq 0$
    \item $\theta^- = H - \tr_\Sigma k < 0$
    \item $\tr_\Sigma k < 0$ (the unfavorable case)
\end{itemize}

\section{Time-Reversed PG Coordinates}

\subsection{Outgoing PG Coordinates}

Instead of ingoing Painlevé-Gullstrand, use OUTGOING:
\[
    ds^2 = -d\tau^2 + \left(dr - \sqrt{\frac{2M}{r}}d\tau\right)^2 + r^2 d\Omega^2
\]

This is the time-reversal of ingoing PG.

\subsection{Extrinsic Curvature}

By time-reversal symmetry:
\[
    k_{ij}^{\text{out}} = -k_{ij}^{\text{in}}
\]

So in outgoing PG:
\[
    \tr k = -\frac{3}{2}\sqrt{\frac{2M}{r^3}} < 0
\]

\textbf{This gives $\tr k < 0$!}

\subsection{Spheres in Outgoing PG}

For a sphere at $r = r_0$:

\textbf{Mean curvature:} $H = 2/r_0 > 0$ (same as before, this is intrinsic to 3-geometry).

\textbf{$\tr_\Sigma k$:} By time-reversal:
\[
    \tr_\Sigma k = -2\sqrt{\frac{2M}{r_0^3}} < 0
\]

\subsection{Null Expansions}

\[
    \theta^+ = H + \tr_\Sigma k = \frac{2}{r_0} - 2\sqrt{\frac{2M}{r_0^3}}
\]

\[
    \theta^- = H - \tr_\Sigma k = \frac{2}{r_0} + 2\sqrt{\frac{2M}{r_0^3}}
\]

Note: $\theta^- > 0$ always!

\subsection{Trapped Condition}

For trapped (in future sense): need $\theta^+ \leq 0$.

\[
    \theta^+ \leq 0 \Leftrightarrow \frac{2}{r_0} \leq 2\sqrt{\frac{2M}{r_0^3}} \Leftrightarrow r_0 \leq 2M
\]

So spheres with $r_0 \leq 2M$ in outgoing PG have $\theta^+ \leq 0$.

But $\theta^- > 0$!

\textbf{This means: the surface is MOTS or marginally anti-trapped, not fully trapped!}

\subsection{Interpretation}

In outgoing PG:
\begin{itemize}
    \item $\theta^+ \leq 0$ for $r \leq 2M$ (outgoing light converges or is stationary)
    \item $\theta^- > 0$ always (ingoing light diverges)
\end{itemize}

The sphere at $r = 2M$ has $\theta^+ = 0$ (MOTS) and $\theta^- > 0$.

This is the EVENT HORIZON viewed from "past" perspective.

\section{Finding Fully Trapped Surfaces}

\subsection{The Issue}

In time-symmetric ($k = 0$) slicing: spheres at $r < 2M$ are fully trapped.

In PG slicings: spheres are only "half-trapped" (one expansion negative).

\subsection{Need Non-Spherical Surfaces?}

Perhaps non-spherical surfaces can be fully trapped with $\tr_\Sigma k < 0$.

\subsection{Tilted Spheres}

Consider a sphere that is "tilted" in spacetime.

In coordinates $(t, r, \theta, \phi)$, consider:
\[
    \Sigma = \{(t, r, \theta, \phi) : r = r_0, t = f(\theta)\}
\]
for some function $f$.

This surface is tilted in the $t$-$\theta$ plane.

\subsection{Analysis}

The induced metric on $\Sigma$ depends on $f$.

The extrinsic curvature in the spacelike slice depends on how we slice.

This becomes complicated...

\section{A Different Approach: Boosted Slices}

\subsection{Lorentz-Boosted Coordinates}

In flat spacetime, a Lorentz boost gives new time slices.

In Schwarzschild, we can do something similar using the isometry group.

\subsection{Boosted Schwarzschild}

Consider Schwarzschild in isotropic coordinates:
\[
    ds^2 = -\left(\frac{1 - M/(2\rho)}{1 + M/(2\rho)}\right)^2 dt^2 + \left(1 + \frac{M}{2\rho}\right)^4(d\rho^2 + \rho^2 d\Omega^2)
\]

Apply a "boost": $t' = \gamma(t - vz)$, $z' = \gamma(z - vt)$.

The $t' = $ const slices are not time-symmetric.

\subsection{Extrinsic Curvature}

The extrinsic curvature of the boosted slice has components that depend on the boost velocity.

For a sphere centered at the "boosted origin", we can have $\tr_\Sigma k$ with either sign depending on boost direction and magnitude.

\section{Explicit Construction}

\subsection{Two-Slice Setup}

Consider two time-symmetric slices $\Sigma_1$ at $t = t_1$ and $\Sigma_2$ at $t = t_2$.

Interpolate between them with a family of slices $\Sigma_s$ for $s \in [1, 2]$.

At intermediate $s$, the slice has $k \neq 0$.

\subsection{The Sign of $k$}

If the slice is "tilted toward the future" (moving forward in $t$ as we go outward in $r$), then:
\begin{itemize}
    \item Material is "falling in" relative to the slice
    \item This gives $\tr k > 0$ (expansion of normals)
\end{itemize}

If tilted toward the past:
\begin{itemize}
    \item Material is "expanding out" relative to the slice
    \item This gives $\tr k < 0$ (contraction of normals)
\end{itemize}

\subsection{A Concrete Example}

Consider the slice defined by:
\[
    t = -\alpha r \quad \text{for } r > 2M
\]
with some smooth extension inside.

For $\alpha > 0$: slice tilts toward the past as $r$ increases.

This should give $\tr k < 0$ in the exterior.

\section{The Key Point}

\begin{tcolorbox}[colback=yellow!20, colframe=yellow!75!black]
\textbf{OBSERVATION:}

In Schwarzschild, we can construct slices with either sign of $\tr k$.

For slices tilted toward the PAST (in the exterior): $\tr k < 0$.

For slices tilted toward the FUTURE (in the exterior): $\tr k > 0$.

The trapped surfaces in these slices have:
\begin{itemize}
    \item Future tilted: $\tr_\Sigma k > 0$ (favorable)
    \item Past tilted: $\tr_\Sigma k < 0$ (unfavorable)
\end{itemize}
\end{tcolorbox}

\section{Verification of Penrose}

\subsection{Setup}

Take a past-tilted slice in Schwarzschild with a trapped sphere at $r = r_0 < 2M$.

\subsection{The Data}

\begin{itemize}
    \item $A = 4\pi r_0^2$ (area is geometric, doesn't depend on slicing)
    \item $M_{\ADM} = M$ (mass is independent of slicing)
    \item $\tr_\Sigma k < 0$ (by construction)
\end{itemize}

\subsection{Penrose Check}

\[
    M \geq \sqrt{\frac{4\pi r_0^2}{16\pi}} = \frac{r_0}{2}
\]

Since $r_0 < 2M$:
\[
    M > \frac{r_0}{2} \quad \checkmark
\]

\textbf{Penrose is satisfied!}

\section{What About the $\theta^+$-Flow?}

\subsection{Starting Point}

Start with a trapped sphere at $r_0 < 2M$ in a past-tilted slice.

\begin{itemize}
    \item $\theta^+ < 0$ (by trapped condition)
    \item $\theta^- < 0$ (by trapped condition)
    \item $\tr_\Sigma k < 0$
\end{itemize}

\subsection{The Flow}

$\dot{\Sigma} = -\theta^+ \nu$

Since $\theta^+ < 0$: flow is outward.

By our earlier result: area is non-decreasing.

\subsection{Destination}

The flow should reach a MOTS (where $\theta^+ = 0$).

In Schwarzschild, the outermost MOTS is at $r = 2M$.

So the flow should take us from $r_0$ to $r = 2M$ (with area increasing from $4\pi r_0^2$ to $16\pi M^2$).

\subsection{At the MOTS}

At $r = 2M$:
\begin{itemize}
    \item $A = 16\pi M^2$
    \item Penrose: $M \geq \sqrt{16\pi M^2/(16\pi)} = M$ ✓
\end{itemize}

Equality holds at the horizon!

\section{The Complete Picture}

\begin{tcolorbox}[colback=green!20, colframe=green!75!black]
\textbf{THE $\theta^+$-FLOW IN SCHWARZSCHILD:}

\textbf{Starting:} Trapped sphere at $r_0 < 2M$ (any slicing)
\begin{itemize}
    \item $\theta^+ < 0$, $\theta^- < 0$
    \item $A_0 = 4\pi r_0^2$
\end{itemize}

\textbf{Flow:} $\dot{\Sigma} = -\theta^+\nu$ (outward)
\begin{itemize}
    \item Area increases: $\frac{dA}{dt} \geq 0$
    \item Surface expands toward horizon
\end{itemize}

\textbf{Ending:} MOTS at $r = 2M$
\begin{itemize}
    \item $\theta^+ = 0$
    \item $A_* = 16\pi M^2$
\end{itemize}

\textbf{Penrose:}
\[
    M = \sqrt{\frac{A_*}{16\pi}} \geq \sqrt{\frac{A_0}{16\pi}}
\]

\textbf{QED for Schwarzschild!}
\end{tcolorbox}

\section{Generalization}

\subsection{The Argument Works Because:}

\begin{enumerate}
    \item Area increases along $\theta^+$-flow (general theorem)
    \item Flow reaches MOTS (in Schwarzschild, the horizon)
    \item At MOTS: Penrose is satisfied (with equality for Schwarzschild)
\end{enumerate}

\subsection{For General Spacetimes:}

Need to verify:
\begin{enumerate}
    \item $\theta^+$-flow has global existence (or can be continued through surgeries)
    \item Flow reaches a MOTS (not guaranteed in general!)
    \item Penrose holds for the limiting MOTS (the remaining gap)
\end{enumerate}

\section{Conclusion}

\begin{tcolorbox}[colback=blue!20, colframe=blue!75!black]
\textbf{SUMMARY:}

\textbf{In Schwarzschild:}
\begin{itemize}
    \item Can construct unfavorable trapped surfaces ($\tr_\Sigma k < 0$)
    \item The $\theta^+$-flow takes them to the horizon (MOTS)
    \item Area increases along the flow
    \item Penrose is satisfied (with equality at horizon)
\end{itemize}

\textbf{This provides strong evidence that the $\theta^+$-flow approach works!}

\textbf{Remaining for general case:}
\begin{itemize}
    \item Flow existence theory
    \item Convergence to MOTS
    \item Penrose for MOTS in general data
\end{itemize}
\end{tcolorbox}

\end{document}
