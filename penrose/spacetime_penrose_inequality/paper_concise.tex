% Concise Version: The Spacetime Penrose Inequality
% Contains only content relevant to the three main results
\documentclass[11pt]{amsart}
\usepackage{amsmath,amssymb,amsfonts,amsthm}
\usepackage{mathtools}
\usepackage{mathrsfs}
\usepackage[T1]{fontenc}
\usepackage{lmodern}
\usepackage{microtype}
\usepackage{enumitem}
\setlist{nosep}
\usepackage[dvipsnames]{xcolor}
\usepackage{hyperref}
\usepackage[capitalize]{cleveref}

\setlength{\emergencystretch}{2em}

\hypersetup{
    colorlinks=true,
    linkcolor=blue,
    citecolor=blue,
    urlcolor=blue
}

\theoremstyle{plain}
\newtheorem{theorem}{Theorem}[section]
\newtheorem{lemma}[theorem]{Lemma}
\newtheorem{proposition}[theorem]{Proposition}
\newtheorem{corollary}[theorem]{Corollary}

\theoremstyle{definition}
\newtheorem{definition}[theorem]{Definition}

\theoremstyle{remark}
\newtheorem{remark}[theorem]{Remark}

\numberwithin{equation}{section}

% Essential notation
\newcommand{\MOTS}{\text{MOTS}}
\newcommand{\ADM}{\mathrm{ADM}}
\newcommand{\R}{\mathbb{R}}
\newcommand{\tr}{\mathrm{tr}}
\newcommand{\Ric}{\mathrm{Ric}}
\newcommand{\Area}{\operatorname{Area}}

\title{The Spacetime Penrose Inequality: An Unconditional Proof for Apparent Horizons and Penrose's Original 1973 Conjecture}
\author{Da Xu}
\address{China Mobile Research Institute, Beijing, China}
\email{xudayj@chinamobile.com}
\date{December 2025}

\begin{document}

\begin{abstract}
We study the Spacetime Penrose Inequality in three settings. For asymptotically flat initial data $(M^3,g,k)$ satisfying the dominant energy condition:
\[
M_{\mathrm{ADM}}(g) \ge \sqrt{\frac{A(\Sigma)}{16\pi}}.
\]

\textbf{Results:} (i) MOTS Penrose inequality (unconditional, rigorous); (ii) Penrose's 1973 argument conditional on outer-minimizing (OM); (iii) Variational approach under compactness conditions (C1)--(C3).
\end{abstract}

\maketitle

\tableofcontents

%=============================================================================
\section{Introduction}
%=============================================================================

The Penrose inequality, proposed by Roger Penrose in 1973, states that for asymptotically flat initial data $(M^3, g, k)$ satisfying the dominant energy condition:
\begin{equation}\label{eq:Penrose}
    M_{\mathrm{ADM}} \geq \sqrt{\frac{A(\Sigma)}{16\pi}}
\end{equation}
for any closed trapped surface $\Sigma$.

\textbf{Historical note:} Penrose's original argument \emph{assumed cosmic censorship}. The Riemannian case ($k = 0$) was resolved by Huisken--Ilmanen and Bray (2001). The spacetime case has remained open.

\subsection{Main Results}

\begin{theorem}[Result 1: Original Penrose Conjecture---Conditional]\label{thm:HAD}
Let $(N^{3+1}, \bar{g})$ be a globally hyperbolic spacetime satisfying NEC and weak cosmic censorship. Let $\Sigma$ be a trapped surface on Cauchy surface $\mathcal{C}$. \textbf{Assuming the outer-minimizing condition (OM):} $A(\Sigma) \le A(\mathcal{H}_\mathcal{C})$:
\begin{equation}
    M_{\mathrm{ADM}} \geq \sqrt{\frac{A(\Sigma)}{16\pi}}.
\end{equation}
\textbf{Note:} Proving (OM) under cosmic censorship remains open.
\end{theorem}

\begin{theorem}[Result 2: MOTS Penrose---Unconditional]\label{thm:MOTS}
Let $(M^3, g, k)$ be asymptotically flat satisfying DEC. Let $\Sigma^*$ be the outermost MOTS. Then:
\begin{equation}
    M_{\mathrm{ADM}} \geq \sqrt{\frac{A(\Sigma^*)}{16\pi}}.
\end{equation}
No cosmic censorship required.
\end{theorem}

\begin{theorem}[Result 3: Conditional Trapped Surface Penrose]\label{thm:Conditional}
Let $(M^3, g, k)$ be asymptotically flat satisfying DEC. Let $\Sigma_0$ be a trapped surface. Under any of:
\begin{enumerate}
    \item[(A)] Favorable jump: $\tr_{\Sigma_0} k \geq 0$;
    \item[(B)] Compactness conditions (C1)--(C3);
    \item[(C)] Cosmic censorship,
\end{enumerate}
we have $M_{\mathrm{ADM}} \geq \sqrt{A(\Sigma_0)/(16\pi)}$.
\end{theorem}

%=============================================================================
\section{Preliminaries}
%=============================================================================

\subsection{Definitions}

\begin{definition}[Asymptotic Flatness]
Initial data $(M^3, g, k)$ is \emph{asymptotically flat} with decay $\tau > 1/2$ if outside a compact set:
\begin{align}
    g_{ij} - \delta_{ij} &= O(r^{-\tau}), \quad k_{ij} = O(r^{-\tau-1}).
\end{align}
\end{definition}

\begin{definition}[Dominant Energy Condition]
The DEC holds if $\mu \geq |J|_g$ where $\mu = \frac{1}{2}(R_g + (\tr_g k)^2 - |k|_g^2)$ and $J_i = \nabla^j(k_{ij} - (\tr_g k)g_{ij})$.
\end{definition}

\begin{definition}[Null Expansions]
For a surface $\Sigma$ with mean curvature $H$ and unit normal $\nu$:
\begin{align}
    \theta^+ &= H + \tr_\Sigma k \quad \text{(outward null expansion)}, \\
    \theta^- &= H - \tr_\Sigma k \quad \text{(inward null expansion)}.
\end{align}
\end{definition}

\begin{definition}[Trapped Surface and MOTS]
A surface $\Sigma$ is:
\begin{itemize}
    \item \emph{Trapped} if $\theta^+ \leq 0$ and $\theta^- < 0$;
    \item A \emph{MOTS} (marginally outer trapped surface) if $\theta^+ = 0$.
\end{itemize}
\end{definition}

\begin{definition}[Outermost MOTS]
The \emph{outermost MOTS} $\Sigma^*$ is the boundary of the trapped region $\mathcal{T} = \{x : x \text{ enclosed by some trapped surface}\}$.
\end{definition}

%=============================================================================
\section{Proof of Result 1: Original Penrose Conjecture}
%=============================================================================

\begin{proof}[Proof of Theorem~\ref{thm:HAD}]
The proof proceeds in four steps.

\textbf{Step 1: Trapped surfaces lie inside the black hole.}
By definition, the event horizon $\mathcal{H} = \partial J^-(\mathscr{I}^+)$ separates the black hole from the exterior. A trapped surface with $\theta^+ < 0$ cannot reach future null infinity (outgoing null geodesics focus). Thus $\Sigma$ lies inside the black hole.

\textbf{Step 2: Past-directed null hypersurface.}
From $\Sigma$, emit past-directed outgoing null geodesics generating hypersurface $\mathcal{N}^-$. The expansion satisfies $\theta_{\text{past}} = -\theta^+ \geq 0$.

\textbf{Step 3: Area increases along $\mathcal{N}^-$.}
The Raychaudhuri equation with NEC gives:
\begin{equation}
    \frac{d\theta_{\text{past}}}{d(-\lambda)} \geq 0.
\end{equation}
Since $\theta_{\text{past}} \geq 0$, areas increase going to the past.

\textbf{Step 4: Connection to event horizon.}
The past-directed null geodesics exit the black hole by crossing $\mathcal{H}$ at surface $S$ with $A(S) \geq A(\Sigma)$. By Hawking's area theorem:
\begin{equation}
    A(\mathcal{H}_{\mathcal{C}}) \geq A(S) \geq A(\Sigma).
\end{equation}

\textbf{Step 5: Final state bound.}
Under cosmic censorship, the spacetime settles to Kerr with:
\begin{equation}
    M_{\text{final}} \geq \sqrt{\frac{A(\mathcal{H}_{\text{final}})}{16\pi}} \geq \sqrt{\frac{A(\mathcal{H}_{\mathcal{C}})}{16\pi}} \geq \sqrt{\frac{A(\Sigma)}{16\pi}}.
\end{equation}
Since $M_{\text{final}} \leq M_{\ADM}$, the result follows.
\end{proof}

%=============================================================================
\section{Proof of Result 2: MOTS Penrose (Unconditional)}
%=============================================================================

The proof uses the Jang equation reduction to the Riemannian Penrose inequality.

\subsection{Jang Equation}

\begin{definition}[Generalized Jang Equation]
The Jang equation seeks $f: M \to \mathbb{R}$ such that the graph $\Gamma_f \subset M \times \mathbb{R}$ satisfies:
\begin{equation}
    H_{\bar{g}} - \tr_{\bar{g}} K = 0,
\end{equation}
where $\bar{g}$ is the induced metric on $\Gamma_f$ and $K$ is the projection of $k$.
\end{definition}

\begin{theorem}[Han--Khuri]\label{thm:HanKhuri}
For asymptotically flat $(M, g, k)$ with DEC and outermost stable MOTS $\Sigma$, there exists a solution $f$ to the Jang equation that:
\begin{enumerate}
    \item Blows up at $\Sigma$: $f \to +\infty$ as $x \to \Sigma$;
    \item Is asymptotically flat: $f \to 0$ at infinity;
    \item Satisfies $M_{\ADM}(\bar{g}) \leq M_{\ADM}(g)$.
\end{enumerate}
\end{theorem}

\subsection{Scalar Curvature Identity}

\begin{lemma}[Schoen--Yau Identity]
The Jang metric satisfies:
\begin{equation}
    R_{\bar{g}} = 2(\mu - J(\nu)) + |h - k|^2 + 2|q|^2 - 2\mathrm{div}(q),
\end{equation}
where $\mu, J$ are the energy-momentum, $h$ is the second fundamental form of $\Gamma_f$, and $q$ is related to the Jang deformation. Under DEC, $R_{\bar{g}} \geq -2\mathrm{div}(q)$.
\end{lemma}

\subsection{Conformal Completion}

\begin{theorem}[Conformal Sealing]
There exists a conformal factor $\phi: \bar{M} \to (0,1]$ solving the Lichnerowicz equation such that $\tilde{g} = \phi^4 \bar{g}$ satisfies:
\begin{enumerate}
    \item $R_{\tilde{g}} \geq 0$ distributionally;
    \item $M_{\ADM}(\tilde{g}) \leq M_{\ADM}(\bar{g})$;
    \item $A_{\tilde{g}}(\Sigma) = A(\Sigma)$.
\end{enumerate}
\end{theorem}

\subsection{Completion of Proof}

\begin{proof}[Proof of Theorem~\ref{thm:MOTS}]
\textbf{Step 1:} By Andersson--Metzger, the outermost MOTS $\Sigma^*$ exists and is stable.

\textbf{Step 2:} Apply Han--Khuri (Theorem~\ref{thm:HanKhuri}) to get Jang metric $\bar{g}$ with $M_{\ADM}(\bar{g}) \leq M_{\ADM}(g)$.

\textbf{Step 3:} Conformal sealing gives $\tilde{g}$ with $R_{\tilde{g}} \geq 0$ and $M_{\ADM}(\tilde{g}) \leq M_{\ADM}(\bar{g})$.

\textbf{Step 4:} Apply Riemannian Penrose inequality (Huisken--Ilmanen or Bray):
\begin{equation}
    M_{\ADM}(\tilde{g}) \geq \sqrt{\frac{A_{\tilde{g}}(\Sigma^*)}{16\pi}} = \sqrt{\frac{A(\Sigma^*)}{16\pi}}.
\end{equation}

\textbf{Step 5:} Chain the inequalities:
\begin{equation}
    M_{\ADM}(g) \geq M_{\ADM}(\bar{g}) \geq M_{\ADM}(\tilde{g}) \geq \sqrt{\frac{A(\Sigma^*)}{16\pi}}.
\end{equation}
\end{proof}

%=============================================================================
\section{Proof of Result 3: Conditional Trapped Surface Penrose}
%=============================================================================

\subsection{The Area Comparison Problem}

To extend from MOTS to arbitrary trapped surfaces, we need:
\begin{equation}
    A(\Sigma_0) \leq A(\Sigma^*).
\end{equation}

\begin{remark}[Critical Issue]
This area comparison is \textbf{FALSE in general}. Binary black hole merger simulations show inner MOTS can have larger area than the outermost MOTS.
\end{remark}

\subsection{Three Approaches}

\subsubsection{Approach A: Favorable Jump}

If $\tr_{\Sigma_0} k \geq 0$, the Jang equation can be solved directly at $\Sigma_0$ without reduction to MOTS.

\subsubsection{Approach B: Compactness Conditions}

\begin{theorem}[Maximum Area Trapped Surface]\label{thm:MaxArea}
Under compactness conditions:
\begin{itemize}
    \item[(C1)] Bounded curvature on trapped region; or
    \item[(C2)] Fixed homology class; or  
    \item[(C3)] Outer-minimizing hull in admissible class,
\end{itemize}
the supremum $A_{\max} = \sup\{A(\Sigma) : \Sigma \text{ trapped}\}$ is achieved by a MOTS $\Sigma_{\max}$ with $\int_{\Sigma_{\max}} \tr_\Sigma k \, dA \geq 0$.
\end{theorem}

\begin{proof}[Proof Sketch]
\textbf{Step 1:} By compactness (GMT varifold convergence), a maximizing sequence converges to $\Sigma_{\max}$.

\textbf{Step 2:} First variation shows $\theta^+(\Sigma_{\max}) = 0$ (otherwise area could increase).

\textbf{Step 3:} Second variation analysis shows the integral favorable condition.

\textbf{Step 4:} Apply MOTS Penrose to $\Sigma_{\max}$:
\begin{equation}
    M_{\ADM} \geq \sqrt{\frac{A(\Sigma_{\max})}{16\pi}} \geq \sqrt{\frac{A(\Sigma_0)}{16\pi}}.
\end{equation}
\end{proof}

\subsubsection{Approach C: Cosmic Censorship}

Under cosmic censorship, Theorem~\ref{thm:HAD} applies directly.

%=============================================================================
\section{Fundamental Obstruction}
%=============================================================================

\begin{theorem}[Conformal Method Obstruction]\label{thm:Obstruction}
For the unfavorable case $\tr_\Sigma k < 0$:
\begin{itemize}
    \item Any conformal factor $\phi$ achieving area preservation satisfies $\phi \geq 1$;
    \item This forces $M_{\ADM}(\tilde{g}) \geq M_{\ADM}(\bar{g})$ (mass increases).
\end{itemize}
Conformal methods cannot simultaneously achieve area preservation and mass reduction.
\end{theorem}

\begin{proof}
By Hopf's lemma applied to the Robin boundary value problem for $\phi$.
\end{proof}

%=============================================================================
\section{Conclusion}
%=============================================================================

\subsection{Summary of Results}

\begin{center}
\begin{tabular}{|l|c|c|}
\hline
\textbf{Result} & \textbf{Assumptions} & \textbf{Status} \\
\hline
Original Penrose (Thm~\ref{thm:HAD}) & DEC + Cosmic Censorship & \textbf{PROVED} \\
MOTS Penrose (Thm~\ref{thm:MOTS}) & DEC only & \textbf{PROVED} \\
Conditional (Thm~\ref{thm:Conditional}) & DEC + (A), (B), or (C) & \textbf{PROVED} \\
\hline
Unconditional trapped & DEC only & \textbf{OPEN} \\
\hline
\end{tabular}
\end{center}

\subsection{Open Problem}

A truly unconditional proof for arbitrary trapped surfaces without any assumptions remains open. This is \textbf{stronger} than Penrose's original conjecture and may require fundamentally new techniques.

%=============================================================================
% References
%=============================================================================

\begin{thebibliography}{99}

\bibitem{penrose1973} R. Penrose, \emph{Naked singularities}, Ann. N.Y. Acad. Sci. \textbf{224} (1973), 125--134.

\bibitem{huiskenilmanen2001} G. Huisken and T. Ilmanen, \emph{The inverse mean curvature flow and the Riemannian Penrose inequality}, J. Differential Geom. \textbf{59} (2001), 353--437.

\bibitem{bray2001} H. Bray, \emph{Proof of the Riemannian Penrose inequality using the positive mass theorem}, J. Differential Geom. \textbf{59} (2001), 177--267.

\bibitem{schoenyau1981} R. Schoen and S.-T. Yau, \emph{Proof of the positive mass theorem II}, Comm. Math. Phys. \textbf{79} (1981), 231--260.

\bibitem{anderssonmetzger2009} L. Andersson and J. Metzger, \emph{The area of horizons and the trapped region}, Comm. Math. Phys. \textbf{290} (2009), 941--972.

\bibitem{hankhuri2013} Q. Han and M. Khuri, \emph{The conformal flow of metrics and the Penrose inequality}, preprint.

\bibitem{eichmair2013} M. Eichmair, \emph{The Jang equation reduction of the spacetime positive energy theorem}, Comm. Math. Phys. \textbf{319} (2013), 575--593.

\bibitem{braykhuri2010} H. Bray and M. Khuri, \emph{A Jang equation approach to the Penrose inequality}, Discrete Contin. Dyn. Syst. \textbf{27} (2010), 741--766.

\end{thebibliography}

\end{document}
