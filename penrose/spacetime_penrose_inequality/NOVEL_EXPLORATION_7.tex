% =========================================================================
%     NOVEL EXPLORATION 7: DYNAMICAL SYSTEMS AND ERGODIC THEORY
%
%     Flows, Attractors, and Stability
%
%     Author: Da Xu
%     Date: December 2025
% =========================================================================

\documentclass[12pt]{article}
\usepackage{amsmath,amsthm,amssymb}
\usepackage{mathrsfs}
\usepackage{tcolorbox}

\theoremstyle{plain}
\newtheorem{theorem}{Theorem}[section]
\newtheorem{lemma}[theorem]{Lemma}
\newtheorem{proposition}[theorem]{Proposition}
\newtheorem{corollary}[theorem]{Corollary}
\newtheorem{conjecture}[theorem]{Conjecture}

\theoremstyle{definition}
\newtheorem{definition}[theorem]{Definition}
\newtheorem{remark}[theorem]{Remark}
\newtheorem{observation}[theorem]{Key Observation}

\newcommand{\ADM}{\mathrm{ADM}}
\newcommand{\tr}{\mathrm{tr}}
\newcommand{\Div}{\mathrm{div}}
\newcommand{\Area}{\mathrm{Area}}

\title{\textbf{Novel Exploration 7: Dynamical Systems Approach}}
\author{Da Xu}
\date{December 2025}

\begin{document}
\maketitle

\section{Motivation}

Dynamical systems theory studies:
\begin{itemize}
    \item Flows and their fixed points
    \item Attractors and stability
    \item Lyapunov functions and monotonicity
    \item Ergodic theory and long-time behavior
\end{itemize}

All geometric flows (MCF, IMCF, Ricci flow) are dynamical systems on the space of geometries.

\section{The Space of Surfaces}

\subsection{Configuration Space}

Let $\mathcal{S}$ be the space of embedded surfaces in $(M, g)$.

A point in $\mathcal{S}$ is a surface $\Sigma$.

\subsection{The Trapped Region}

\begin{definition}
The \textbf{trapped subset} is:
\[
    \mathcal{T} = \{\Sigma \in \mathcal{S} : \theta^+(\Sigma) \leq 0, \theta^-(\Sigma) < 0\}
\]
\end{definition}

\subsection{The MOTS Boundary}

The boundary of $\mathcal{T}$ (in $\mathcal{S}$) includes:
\[
    \partial\mathcal{T} \supset \{\Sigma : \theta^+ = 0\} = \text{MOTS}
\]

\section{Flows on Surface Space}

\subsection{Mean Curvature Flow}

\[
    \dot{\Sigma} = -H \nu
\]

Fixed points: minimal surfaces ($H = 0$).

But $H = 0$ inside trapped region is impossible (since $H < 0$).

\subsection{Inverse Mean Curvature Flow}

\[
    \dot{\Sigma} = \frac{\nu}{H}
\]

Only defined for $H \neq 0$. For $H < 0$, surface moves INWARD.

\subsection{Null Expansion Flow}

\[
    \dot{\Sigma} = -\theta^+ \nu
\]

Fixed points: MOTS ($\theta^+ = 0$).

This flow makes sense for trapped surfaces!

\section{MOTS as Attractor}

\subsection{Conjecture}

\begin{conjecture}[MOTS Attractor]
Under the null expansion flow:
\[
    \dot{\Sigma} = -\theta^+ \nu
\]
trapped surfaces flow toward MOTS.
\end{conjecture}

\subsection{Stability Analysis}

At a MOTS ($\theta^+ = 0$), linearize:
\[
    \delta\dot{\theta}^+ = L_{\text{MOTS}} \delta\theta^+ + \cdots
\]

where $L_{\text{MOTS}}$ is the MOTS stability operator.

If $L_{\text{MOTS}}$ has positive eigenvalues: MOTS is stable attractor.

\subsection{The Stability Operator}

The MOTS stability operator is:
\[
    L\phi = -\Delta\phi - 2\Omega \cdot \nabla\phi + \left(\frac{1}{2}R_\Sigma + \Div\Omega - |\Omega|^2 - G_{\mu\nu}\ell^+_\mu\ell^+_\nu\right)\phi
\]

where $\Omega$ is the connection 1-form and $\ell^+$ is the outgoing null vector.

A MOTS is \textbf{stable} if $\lambda_1(L) \geq 0$.

\section{Lyapunov Functions}

\subsection{Definition}

A \textbf{Lyapunov function} for a flow is $V: \mathcal{S} \to \mathbb{R}$ such that:
\[
    \dot{V}(\Sigma(t)) \leq 0
\]
along flow lines.

\subsection{Area as (Anti-)Lyapunov Function}

Under $\dot{\Sigma} = f\nu$:
\[
    \frac{d\Area}{dt} = -\int_\Sigma Hf \, dA
\]

For MCF ($f = -H$): $\dot{A} = \int H^2 \, dA > 0$ for $H \neq 0$. Area INCREASES!

Wait, that's wrong. For MCF with $H < 0$ (trapped):
\[
    \dot{\Sigma} = -H\nu = |H|\nu \quad (\text{outward})
\]
So the surface expands? Let me reconsider.

If $H < 0$ and $\nu$ points outward, then $-H\nu$ points outward.

Actually, for closed surfaces in Euclidean space, $H > 0$ (convex outward) and MCF shrinks.

In our case, $H < 0$ means the surface is ``inside-out'' convex. MCF would expand it!

\subsection{Corrected Analysis}

For trapped surfaces with $H < 0$:
\begin{itemize}
    \item MCF: $\dot{\Sigma} = -H\nu = |H|\nu$ (outward), area increases
    \item IMCF: $\dot{\Sigma} = \frac{\nu}{H} = -\frac{\nu}{|H|}$ (inward), area decreases
\end{itemize}

So IMCF on trapped surfaces shrinks them!

\section{A Novel Flow}

\subsection{The Hawking Mass Flow}

\begin{definition}
The \textbf{Hawking mass flow} is:
\[
    \dot{\Sigma} = (m_H - M_{\text{target}}) \nu
\]
where $M_{\text{target}} = \sqrt{A(\Sigma)/(16\pi)}$.
\end{definition}

The flow stops when $m_H = M_{\text{target}}$, i.e., $\int H^2 = 0$.

\subsection{Analysis}

For trapped surfaces: $m_H < \sqrt{A/(16\pi)} = M_{\text{target}}$.

So $m_H - M_{\text{target}} < 0$, and the surface moves inward.

As it moves inward:
\begin{itemize}
    \item Area might decrease
    \item $\int H^2$ might change
    \item $m_H$ might increase or decrease
\end{itemize}

Not clear if this flow has good properties!

\section{The Gradient Flow Perspective}

\subsection{Gradient Flows}

A gradient flow minimizes a functional:
\[
    \dot{\Sigma} = -\nabla E(\Sigma)
\]

The energy $E$ decreases along the flow.

\subsection{Which Functional?}

For Penrose inequality, we want a functional $E$ such that:
\begin{enumerate}
    \item $E$ decreases under some flow
    \item At the minimum, $\sqrt{A/(16\pi)} = M_{\ADM}$
\end{enumerate}

\begin{conjecture}
Define:
\[
    E(\Sigma) = \Area(\Sigma) - 16\pi M_{\ADM}^2
\]

Penrose = $E(\Sigma) \leq 0$ for all trapped $\Sigma$.
\end{conjecture}

But this $E$ depends on $M_{\ADM}$, which is fixed!

\subsection{A Better Functional}

\begin{definition}
The \textbf{Penrose functional} is:
\[
    P(\Sigma) = \Area(\Sigma) - 16\pi m_H(\Sigma)^2
\]
\end{definition}

We have:
\begin{align}
    P(\Sigma) &= A - 16\pi \cdot \frac{A}{16\pi}\left(1 - \frac{\int H^2}{16\pi}\right)^2 \\
    &= A - A\left(1 - \frac{\int H^2}{16\pi}\right)^2 \\
    &= A\left[1 - \left(1 - \frac{\int H^2}{16\pi}\right)^2\right]
\end{align}

For small $\int H^2$:
\[
    P \approx A \cdot \frac{2\int H^2}{16\pi} = \frac{A \int H^2}{8\pi}
\]

$P \geq 0$ always (as long as $\int H^2 \leq 16\pi$).

\section{Ergodic Theory Perspective}

\subsection{Time Averages}

For a dynamical system, ergodic theory studies time averages:
\[
    \bar{f} = \lim_{T\to\infty} \frac{1}{T}\int_0^T f(\Sigma(t)) \, dt
\]

\subsection{Application}

Consider a spacetime evolution.

Time average of area:
\[
    \bar{A} = \lim_{T\to\infty} \frac{1}{T}\int_0^T \Area(\Sigma_t) \, dt
\]

\begin{conjecture}
If the spacetime ``equilibrates'' to Schwarzschild:
\[
    \bar{A} \leq 16\pi M_{\ADM}^2
\]
\end{conjecture}

This would be a time-averaged Penrose inequality!

\section{Attractor Analysis}

\subsection{The Final State}

Under gravitational collapse (Cosmic Censorship):
\[
    \text{Initial data} \to \text{Kerr-Newman black hole}
\]

For Kerr: $A = 8\pi(M^2 + \sqrt{M^4 - J^2}) \leq 16\pi M^2$.

The attractor satisfies Penrose!

\subsection{Reverse Time Argument}

\begin{observation}
If:
\begin{enumerate}
    \item Initial surface has $A_0$
    \item Final (Kerr) has $A_\infty \leq 16\pi M^2$
    \item Area is non-decreasing (Second Law)
\end{enumerate}
Then $A_0 \leq A_\infty \leq 16\pi M^2$. Penrose!
\end{observation}

But wait: the Second Law is for horizons, not arbitrary trapped surfaces!

\subsection{The Issue}

For a trapped surface $\Sigma_0$ that is NOT a horizon:
\begin{itemize}
    \item It might not evolve to the final horizon
    \item Area might decrease before increasing
    \item No direct application of Second Law
\end{itemize}

\section{Basin of Attraction}

\subsection{Definition}

The \textbf{basin of attraction} of MOTS is:
\[
    \mathcal{B} = \{\Sigma : \text{flow from } \Sigma \text{ reaches MOTS}\}
\]

\subsection{Conjecture}

\begin{conjecture}[Basin Coverage]
Under the null expansion flow, every trapped surface is in the basin of attraction of some MOTS:
\[
    \mathcal{T} \subset \mathcal{B}
\]
\end{conjecture}

If true: trapped surfaces flow to MOTS.

\section{Morse Theory on Surface Space}

\subsection{Critical Points}

For a functional $F: \mathcal{S} \to \mathbb{R}$, critical points are where $\nabla F = 0$.

\subsection{The Area Functional}

Critical points of Area: surfaces with $H = 0$ (minimal/maximal).

Index: number of negative eigenvalues of second variation.

\subsection{Morse Inequalities}

\begin{proposition}
The Morse inequalities relate:
\begin{itemize}
    \item Number of critical points of each index
    \item Topology of $\mathcal{S}$
\end{itemize}
\end{proposition}

This might constrain possible surface configurations!

\section{A Conley Index Approach}

\subsection{The Conley Index}

The Conley index is a topological invariant of isolated invariant sets under a flow.

\subsection{Application}

\begin{conjecture}
The Conley index of the trapped region $\mathcal{T}$ under an appropriate flow constrains the relationship between $A$ and $M$.
\end{conjecture}

\section{Conclusion}

\begin{tcolorbox}[colback=blue!10, colframe=blue!75!black]
\textbf{Dynamical Systems Insights:}

\begin{enumerate}
    \item MOTS are natural ``fixed points'' of null expansion flow
    \item Trapped surfaces might flow to MOTS (attractor conjecture)
    \item Area can increase or decrease depending on flow direction
    \item Time-averaged inequalities might be provable
    \item Attractor analysis via Cosmic Censorship gives Penrose for final state
\end{enumerate}

\textbf{Most Promising:}

The \textbf{MOTS attractor conjecture}: if all trapped surfaces flow to MOTS, and we can control the area change, Penrose might follow.

\textbf{Key Issue:}

Finding a Lyapunov function that:
\begin{itemize}
    \item Decreases along the flow
    \item Gives area bound at the fixed point
\end{itemize}
\end{tcolorbox}

\end{document}
