% WEAK FORMULATION APPROACH: ABSORBING NEGATIVE DIRAC MASS
%
% Key insight: The p-harmonic method in AMO uses WEAK formulations.
% Perhaps the weak formulation can naturally handle the negative [H].

\documentclass{article}
\usepackage{amsmath,amsthm,amssymb}
\newtheorem{theorem}{Theorem}
\newtheorem{lemma}{Lemma}
\newtheorem{proposition}{Proposition}
\newtheorem{corollary}{Corollary}
\newtheorem{conjecture}{Conjecture}
\newtheorem{remark}{Remark}
\newtheorem{definition}{Definition}
\newtheorem{claim}{Claim}
\newtheorem{idea}{Idea}
\newtheorem{observation}{Observation}

\begin{document}

\title{Weak Formulation and the Negative Jump}
\author{Mathematical Exploration}
\date{\today}
\maketitle

\section{The AMO Method and Weak Curvature}

\subsection{Review of AMO}

The Agostiniani-Mazzieri-Oronzio method uses $p$-harmonic functions $u_p$ satisfying:
\begin{equation}
    \div(|\nabla u_p|^{p-2} \nabla u_p) = 0
\end{equation}
with $u_p = 0$ on the inner boundary $\Sigma$ and $u_p \to \infty$ at spatial infinity.

The key monotonicity formula involves the mass functional:
\begin{equation}
    \mathfrak{m}_p(t) = \frac{1}{c_p}\int_{\{u_p = t\}} |\nabla u_p|^{p-1} \, dA - \frac{1}{c_p}\int_{\{u_p > t\}} R \cdot |\nabla u_p|^{p-2} \, dV
\end{equation}

Under the assumption $R \ge 0$, this is monotone: $\mathfrak{m}_p'(t) \ge 0$.

\subsection{The Weak Curvature Condition}

In the AMO paper, they work with $R \ge 0$ in a \emph{weak} sense:
\begin{equation}
    \int_M R \cdot \varphi \, dV \ge 0 \quad \text{for all } \varphi \ge 0 \text{ with } \varphi \in C_c^\infty(M)
\end{equation}

For our problem:
\begin{equation}
    R = R^{\text{reg}} + 2[H]\delta_\Sigma
\end{equation}

The weak formulation gives:
\begin{equation}
    \int_M R \cdot \varphi \, dV = \int_M R^{\text{reg}} \cdot \varphi \, dV + 2[H]\int_\Sigma \varphi|_\Sigma \, dA
\end{equation}

If $[H] < 0$ and $\varphi|_\Sigma > 0$, the second term is NEGATIVE.

\section{A Potential Loophole: Test Functions Vanishing on $\Sigma$}

\subsection{The Observation}

What if we could restrict to test functions with $\varphi|_\Sigma = 0$?

Then:
\begin{equation}
    \int_M R \cdot \varphi \, dV = \int_M R^{\text{reg}} \cdot \varphi \, dV \ge 0
\end{equation}
since $R^{\text{reg}} \ge 0$ by DEC!

\textbf{The negative $[H]$ would be invisible!}

\subsection{Why This Might Work}

In the AMO method, the test functions come from variations of the $p$-harmonic function $u_p$.

\textbf{Key point:} $u_p = 0$ on $\Sigma$ by the boundary condition!

So perhaps the relevant test functions DO vanish on $\Sigma$...

\subsection{Checking the AMO Proof}

The monotonicity formula involves:
\begin{equation}
    \mathfrak{m}_p(t) = \frac{1}{c_p}\int_{\{u_p = t\}} |\nabla u_p|^{p-1} \, dA - \frac{1}{c_p}\int_{\{u_p > t\}} R \cdot |\nabla u_p|^{p-2} \, dV
\end{equation}

The level set $\{u_p = t\}$ does NOT intersect $\Sigma$ for $t > 0$ (since $u_p = 0$ on $\Sigma$).

The integrand $|\nabla u_p|^{p-2}$ in the volume integral is used to weight the curvature.

\textbf{Critical question:} Does $|\nabla u_p|^{p-2}$ vanish on $\Sigma$?

\subsection{Behavior of $u_p$ Near $\Sigma$}

Since $u_p = 0$ on $\Sigma$ and $u_p > 0$ in $M \setminus \Sigma$, we have $\nabla u_p|_\Sigma$ pointing into the interior.

Near $\Sigma$, with $s = \text{dist}(x, \Sigma)$:
\begin{equation}
    u_p \sim s^\alpha \quad \text{for some } \alpha > 0
\end{equation}

If $\alpha = 1$ (Neumann-type): $|\nabla u_p| \sim 1$ near $\Sigma$.

If $\alpha > 1$ (faster growth): $|\nabla u_p| \sim s^{\alpha-1} \to 0$ at $\Sigma$.

\textbf{Question:} What is the value of $\alpha$ for the $p$-harmonic function?

For the $p$-Laplacian with Dirichlet boundary conditions, the gradient typically does NOT vanish at the boundary.

\section{A More Careful Analysis}

\subsection{The Bochner-Type Identity}

The AMO monotonicity comes from a Bochner-type identity:
\begin{equation}
    \div(|\nabla u|^{p-2}\nabla u \cdot |\nabla u|) = |\nabla u|^{p-2}\left(|\nabla^2 u|^2 + \text{Ric}(\nabla u, \nabla u) + \ldots\right)
\end{equation}

Integrating this over a region $\{t < u < T\}$ gives the monotonicity formula.

\subsection{The Interface Contribution}

When integrating by parts near $\Sigma$:
\begin{equation}
    \int_{\{u > t\}} \div(\ldots) \, dV = \int_{\{u = t\}} (\ldots) \cdot \nu \, dA - \int_{\Sigma} (\ldots) \cdot \nu \, dA
\end{equation}

The boundary term on $\Sigma$ is:
\begin{equation}
    \int_\Sigma |\nabla u|^{p-1} \cdot \partial_\nu u \, dA
\end{equation}

Since $u = 0$ on $\Sigma$ and increases away from $\Sigma$, we have $\partial_\nu u|_\Sigma > 0$.

This boundary term is POSITIVE and appears in the monotonicity formula.

\subsection{The Curvature Term}

The volume integral over $R$ is:
\begin{equation}
    \int_{\{u > t\}} R \cdot |\nabla u|^{p-2} \, dV = \int_{\{u > t\}} R^{\text{reg}} \cdot |\nabla u|^{p-2} \, dV + 2[H]\int_\Sigma |\nabla u|^{p-2} \, dA
\end{equation}

\textbf{The interface term does NOT vanish!}

The factor $|\nabla u|^{p-2}|_\Sigma$ is finite and positive.

\section{A New Idea: Modified Test Function}

\subsection{The Problem}

The standard $p$-harmonic function $u$ has $|\nabla u||_\Sigma > 0$, so the negative $[H]$ term contributes.

\textbf{Idea:} Use a MODIFIED $p$-harmonic function that vanishes faster at $\Sigma$.

\subsection{Construction}

Let $u$ be the standard $p$-harmonic function with $u|_\Sigma = 0$.

Let $\eta$ be a cutoff function with $\eta = 0$ near $\Sigma$ and $\eta = 1$ away from $\Sigma$.

Consider $v = u \cdot \eta^\beta$ for some $\beta > 0$.

Near $\Sigma$: $v \sim s^\alpha \cdot (s/\epsilon)^\beta = s^{\alpha + \beta}/\epsilon^\beta$ where $\eta \sim s/\epsilon$ in an $\epsilon$-neighborhood.

\textbf{Problem:} $v$ is no longer $p$-harmonic!

\subsection{A Different Modification}

Consider solving the $p$-harmonic equation with a DIFFERENT boundary condition:
\begin{equation}
    u|_\Sigma = \epsilon > 0 \quad \text{(small)}
\end{equation}

Then the level set $\{u = 0\}$ lies OUTSIDE $\Sigma$ in the sealed metric...

This doesn't seem to help either.

\section{The Fundamental Issue}

\subsection{Why the Standard Approach Fails}

The AMO method computes:
\begin{equation}
    \lim_{t \to 0^+} \mathfrak{m}_p(t) = \sqrt{\frac{A(\Sigma)}{16\pi}}
\end{equation}

This limit captures the area of $\Sigma$ by:
\begin{equation}
    \lim_{t \to 0^+} \int_{\{u = t\}} |\nabla u|^{p-1} \, dA \to |\nabla u|_\Sigma|^{p-1} \cdot A(\Sigma)
\end{equation}

The gradient $|\nabla u|_\Sigma|$ appears in the area term.

\textbf{The same gradient appears in the curvature term!}

There's no way to decouple them.

\subsection{A Quantitative Analysis}

Suppose:
\begin{itemize}
    \item $|\nabla u|_\Sigma| = c_0 > 0$
    \item $R^{\text{reg}} \ge 0$ in a neighborhood of $\Sigma$
    \item $[H] = -\delta < 0$
\end{itemize}

The monotonicity formula near $t = 0$:
\begin{align}
    \mathfrak{m}_p(t) &= \frac{1}{c_p}c_0^{p-1}A(\Sigma) - \frac{1}{c_p}\int_{\{u > t\}} R^{\text{reg}} \cdot |\nabla u|^{p-2} - \frac{2\delta}{c_p}c_0^{p-2}A(\Sigma) \\
    &= \frac{c_0^{p-1}}{c_p}\left(1 - \frac{2\delta}{c_0}\right)A(\Sigma) - \text{(positive terms)}
\end{align}

For $\delta > c_0/2$: the effective "area" is REDUCED by the negative $[H]$.

\textbf{This is actually what SHOULD happen physically!}

The negative $[H]$ means the interface contributes negative mass, which should REDUCE the lower bound in the Penrose inequality.

\section{A Revelation: Maybe the Inequality is Different!}

\subsection{The Modified Penrose Inequality}

Perhaps the correct statement for unfavorable jump is:
\begin{equation}
    M_{\text{ADM}} \ge \sqrt{\frac{A(\Sigma)}{16\pi}} - \frac{|[H]|}{8\pi}A(\Sigma)
\end{equation}

or some similar modification!

\subsection{Checking the Physics}

The Penrose inequality is motivated by:
\begin{itemize}
    \item Cosmic censorship: trapped surfaces form black holes
    \item Area-mass relation: $M_{\text{BH}} = \sqrt{A/(16\pi)}$ for Schwarzschild
\end{itemize}

For a trapped surface with $[H] < 0$:
\begin{itemize}
    \item The surface is "actively contracting" (even the ingoing null expansion is negative)
    \item The area might DECREASE before settling into a final black hole
\end{itemize}

\textbf{Physical interpretation:} A trapped surface with unfavorable jump might have LESS "effective area" than its geometric area because it's in a more dynamical state.

\subsection{A Conjectured Modified Inequality}

\begin{conjecture}[Modified Penrose Inequality for Unfavorable Jump]
Let $(M, g, k)$ satisfy DEC. Let $\Sigma$ be a trapped surface with $[H] = \tr_\Sigma k < 0$. Then:
\begin{equation}
    M_{\text{ADM}} \ge \sqrt{\frac{A(\Sigma) + 2[H]A(\Sigma)}{16\pi}} = \sqrt{\frac{(1 + 2\tr_\Sigma k)A(\Sigma)}{16\pi}}
\end{equation}
\end{conjecture}

Note: When $\tr_\Sigma k = -1/2$, the RHS vanishes. This corresponds to a "maximally contracting" trapped surface.

\section{Testing the Conjecture}

\subsection{Schwarzschild Example}

In Schwarzschild with maximal slicing ($\tr k = 0$ everywhere), we have $[H] = 0$ on all surfaces. The standard Penrose inequality holds.

\subsection{Boosted Schwarzschild}

Consider Schwarzschild viewed from a boosted frame. The extrinsic curvature $k \neq 0$, and trapped surfaces may have $\tr_\Sigma k \neq 0$.

\textbf{Question:} Can we find a trapped surface with $\tr_\Sigma k < 0$ in boosted Schwarzschild and check the inequality?

\subsection{Kerr}

In Kerr spacetime with angular momentum, $\tr_\Sigma k$ on the horizon depends on the slice.

For extremal Kerr, the horizon has $A = 8\pi M^2$ (not $16\pi M^2$). This might provide a test case.

\section{Conclusion}

The weak formulation analysis reveals:

\begin{enumerate}
    \item The negative $[H]$ term DOES contribute to the monotonicity formula.
    \item It cannot be eliminated by choosing special test functions.
    \item The contribution REDUCES the effective area in the Penrose inequality.
    \item This suggests a MODIFIED Penrose inequality for unfavorable jump.
\end{enumerate}

\textbf{The modified conjecture:}
\begin{equation}
    M_{\text{ADM}} \ge \sqrt{\frac{(1 + 2\tr_\Sigma k)A(\Sigma)}{16\pi}}
\end{equation}

This is a \textbf{weaker} inequality that accounts for the negative mass contribution from the unfavorable jump.

\textbf{Next steps:}
\begin{enumerate}
    \item Verify the modified inequality in examples (Kerr, boosted Schwarzschild).
    \item Attempt to prove the modified inequality using AMO.
    \item Understand the physical meaning of the $(1 + 2\tr_\Sigma k)$ factor.
\end{enumerate}

\end{document}
