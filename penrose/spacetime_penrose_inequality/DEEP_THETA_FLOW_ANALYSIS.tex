\documentclass[11pt]{article}
\usepackage{amsmath,amssymb,amsthm,mathrsfs}
\usepackage[margin=1in]{geometry}

\newtheorem{theorem}{Theorem}[section]
\newtheorem{lemma}[theorem]{Lemma}
\newtheorem{proposition}[theorem]{Proposition}
\newtheorem{corollary}[theorem]{Corollary}
\theoremstyle{definition}
\newtheorem{definition}[theorem]{Definition}
\newtheorem{remark}[theorem]{Remark}

\newcommand{\tr}{\mathrm{tr}}
\newcommand{\ADM}{\mathrm{ADM}}
\newcommand{\Ric}{\mathrm{Ric}}
\newcommand{\divg}{\mathrm{div}}

\title{Deep Analysis of $\theta$-Flow:\\
Viscosity Solutions, Varifolds, and Entropy Methods}
\author{}
\date{December 2025}

\begin{document}
\maketitle

\begin{abstract}
We develop the theory of $\theta$-flow using advanced tools from geometric 
analysis: viscosity solutions for degenerate elliptic PDEs, varifold theory 
from geometric measure theory, and entropy functionals inspired by Perelman's 
work on Ricci flow. These tools enable us to handle the singularities that 
arise when $\theta^+ = 0$ and to prove new monotonicity formulas.
\end{abstract}

\tableofcontents

%==============================================================================
\section{The Level Set Formulation}
%==============================================================================

\subsection{Setup}

We reformulate the $\theta^+$-flow as a level set PDE.

\begin{definition}[Level Set $\theta^+$-Flow]
Find $u: M \to \mathbb{R}$ such that the level sets $\Sigma_t = \{u = t\}$ 
evolve by inverse $\theta^+$-flow:
\begin{equation}\label{eq:level_set_theta}
    \theta^+(\Sigma_{u(x)}) \cdot |\nabla u| = 1.
\end{equation}
\end{definition}

Expanding $\theta^+ = H + \tr_\Sigma k$:
\begin{equation}
    \left(\divg\left(\frac{\nabla u}{|\nabla u|}\right) + \tr_{\Sigma_u} k\right) |\nabla u| = 1.
\end{equation}

This is a \textbf{degenerate elliptic PDE} of the form:
\begin{equation}
    F(x, u, \nabla u, \nabla^2 u) = 0,
\end{equation}
where $F$ involves the mean curvature operator.

\subsection{The Mean Curvature Operator}

The mean curvature of a level set is:
\begin{equation}
    H = \divg\left(\frac{\nabla u}{|\nabla u|}\right) 
    = \frac{1}{|\nabla u|}\left(\Delta u - \frac{\nabla^2 u(\nabla u, \nabla u)}{|\nabla u|^2}\right).
\end{equation}

Let $\nu = \nabla u / |\nabla u|$ be the unit normal. Then:
\begin{equation}
    H = \frac{1}{|\nabla u|}\left(\Delta u - \nabla^2 u(\nu, \nu)\right) 
    = \frac{1}{|\nabla u|}\tr_{T\Sigma}(\nabla^2 u).
\end{equation}

\subsection{The Extrinsic Curvature Term}

The trace $\tr_\Sigma k$ involves:
\begin{equation}
    \tr_\Sigma k = k_{ij}(g^{ij} - \nu^i\nu^j) = \tr_g k - k(\nu, \nu).
\end{equation}

So:
\begin{equation}
    \theta^+ = \frac{1}{|\nabla u|}\tr_{T\Sigma}(\nabla^2 u) + \tr_g k - k(\nu, \nu).
\end{equation}

The level set equation becomes:
\begin{equation}\label{eq:full_level_set}
    \tr_{T\Sigma}(\nabla^2 u) + |\nabla u|\left(\tr_g k - k\left(\frac{\nabla u}{|\nabla u|}, \frac{\nabla u}{|\nabla u|}\right)\right) = 1.
\end{equation}

%==============================================================================
\section{Viscosity Solutions}
%==============================================================================

\subsection{Motivation}

Equation \eqref{eq:full_level_set} is degenerate where $|\nabla u| = 0$ and 
becomes singular where $\theta^+ = 0$ (MOTS boundaries). We need weak solutions.

\begin{definition}[Viscosity Subsolution]
A function $u \in C(M)$ is a \textbf{viscosity subsolution} of \eqref{eq:full_level_set} 
if for every $\phi \in C^2(M)$ such that $u - \phi$ has a local maximum at $x_0$:
\begin{equation}
    F(x_0, u(x_0), \nabla\phi(x_0), \nabla^2\phi(x_0)) \le 0.
\end{equation}
\end{definition}

\begin{definition}[Viscosity Supersolution]
Similarly, $u$ is a \textbf{viscosity supersolution} if for every $\phi$ such 
that $u - \phi$ has a local minimum at $x_0$:
\begin{equation}
    F(x_0, u(x_0), \nabla\phi(x_0), \nabla^2\phi(x_0)) \ge 0.
\end{equation}
\end{definition}

\begin{definition}[Viscosity Solution]
$u$ is a \textbf{viscosity solution} if it is both a sub- and supersolution.
\end{definition}

\subsection{Existence Theory}

\begin{theorem}[Existence of Viscosity Solution]
Let $(M, g, k)$ be asymptotically flat initial data satisfying DEC, with 
outer boundary at infinity and inner boundary a trapped surface $\Sigma$.

There exists a viscosity solution $u: M \setminus \Sigma \to [0, \infty)$ of 
\eqref{eq:full_level_set} with:
\begin{enumerate}
    \item $u = 0$ on $\partial_{\text{outer}}$ (large sphere at infinity)
    \item $u \to +\infty$ as $x \to \Sigma$
    \item $u$ is Lipschitz continuous
\end{enumerate}
\end{theorem}

\begin{proof}[Proof Sketch (Perron's Method)]
Define:
\begin{equation}
    u(x) := \sup\{v(x) : v \text{ is a viscosity subsolution with } v|_{\partial} = 0\}.
\end{equation}

By comparison principle, this supremum is a viscosity solution.

The key is the \textbf{comparison principle}: if $v$ is a subsolution and 
$w$ is a supersolution with $v \le w$ on $\partial M$, then $v \le w$ in $M$.

For equation \eqref{eq:full_level_set}, comparison follows from the structure 
of the mean curvature operator (Evans-Spruck, Chen-Giga-Goto).
\end{proof}

\subsection{Regularity}

\begin{theorem}[Partial Regularity]
The viscosity solution $u$ is smooth except on a set of Hausdorff dimension at most $n-1$.
\end{theorem}

The singular set corresponds to:
\begin{itemize}
    \item Points where $|\nabla u| = 0$ (fattening of level sets)
    \item Points where $\theta^+ = 0$ (MOTS)
\end{itemize}

%==============================================================================
\section{The Arrival Time Function}
%==============================================================================

\subsection{Definition}

Instead of the level set function, consider the \textbf{arrival time}:

\begin{definition}
The arrival time $T(x)$ is the "time" at which the $\theta^+$-flow reaches $x$, 
starting from infinity at $t = 0$:
\begin{equation}
    T(x) := \inf\{t \ge 0 : x \in \Sigma_t\}.
\end{equation}
\end{definition}

For viscosity solutions, $T = u$ (the level set function).

\subsection{The Eikonal-Type Equation}

The arrival time satisfies:
\begin{equation}
    |\nabla T| \cdot \theta^+(\Sigma_T) = 1,
\end{equation}
which can be rewritten as:
\begin{equation}\label{eq:eikonal_theta}
    |\nabla T| = \frac{1}{\theta^+}.
\end{equation}

At a MOTS ($\theta^+ = 0$): $|\nabla T| = \infty$, so $T$ has a singularity.

\subsection{Blow-up Analysis}

Near a MOTS $\Sigma^*$ where $\theta^+ \to 0$:

\begin{lemma}[Blow-up Rate]
If $\theta^+ \sim d(x, \Sigma^*)$ near $\Sigma^*$, then:
\begin{equation}
    T(x) \sim -\log d(x, \Sigma^*) \to +\infty \quad \text{as } x \to \Sigma^*.
\end{equation}
\end{lemma}

\begin{proof}
\begin{equation}
    T = \int_\gamma \frac{ds}{\theta^+} \sim \int_\epsilon^R \frac{dr}{r} = \log(R/\epsilon) \to \infty.
\end{equation}
\end{proof}

This shows that the $\theta^+$-flow takes infinite time to reach the MOTS, 
providing a natural barrier.

%==============================================================================
\section{Varifold Formulation}
%==============================================================================

\subsection{Motivation}

To handle the singularities and possible non-uniqueness of level sets, we use 
\textbf{varifolds} from geometric measure theory.

\begin{definition}[Varifold]
A \textbf{varifold} on $M$ is a Radon measure $V$ on $M \times G(n-1, n)$, 
where $G(n-1, n)$ is the Grassmannian of $(n-1)$-planes in $\mathbb{R}^n$.
\end{definition}

Intuitively, a varifold is a "generalized surface" that may have multiplicity 
and captures the tangent planes at each point.

\subsection{The Mass of a Varifold}

The \textbf{mass} (or weight) of $V$ is:
\begin{equation}
    \|V\|(U) := V(U \times G(n-1, n)) = \text{generalized area of } V \text{ in } U.
\end{equation}

For a smooth surface $\Sigma$:
\begin{equation}
    \|V_\Sigma\| = \mathcal{H}^{n-1}\llcorner\Sigma = \text{area measure on } \Sigma.
\end{equation}

\subsection{First Variation}

The \textbf{first variation} of a varifold $V$ is a vector-valued measure $\delta V$:
\begin{equation}
    \delta V(X) := \int_{M \times G} \divg_S X \, dV(x, S),
\end{equation}
where $\divg_S X$ is the tangential divergence on the plane $S$.

For smooth surfaces: $\delta V = -H\nu \cdot \mathcal{H}^{n-1}\llcorner\Sigma$.

\subsection{Varifolds with Generalized Mean Curvature}

\begin{definition}
A varifold $V$ has \textbf{generalized mean curvature} $\vec{H} \in L^1_{\text{loc}}(\|V\|)$ if:
\begin{equation}
    \delta V(X) = -\int \vec{H} \cdot X \, d\|V\|
\end{equation}
for all compactly supported vector fields $X$.
\end{definition}

\subsection{The $\theta^+$-Varifold}

\begin{definition}[$\theta^+$-Varifold]
A family of varifolds $\{V_t\}_{t \ge 0}$ is a \textbf{weak $\theta^+$-flow} if:
\begin{equation}
    \frac{d}{dt}\|V_t\|(M) = \int \theta^+ \, d\|V_t\|,
\end{equation}
where $\theta^+ = H + \tr_S k$ is computed using the generalized mean curvature.
\end{definition}

This allows the flow to develop singularities (fattening, pinching) while 
maintaining control of the mass (area).

%==============================================================================
\section{The Brakke-Type Flow}
%==============================================================================

\subsection{Brakke's Mean Curvature Flow}

Brakke defined weak mean curvature flow as:
\begin{equation}
    \frac{d}{dt}\|V_t\|(\phi) \le \int \left(-\phi H^2 + \nabla\phi \cdot H\nu\right) d\|V_t\|
\end{equation}
for all non-negative test functions $\phi$.

This inequality allows for \textbf{sudden vanishing} of parts of the surface.

\subsection{Brakke-Type $\theta^+$-Flow}

\begin{definition}[Brakke $\theta^+$-Flow]
A family $\{V_t\}$ is a \textbf{Brakke $\theta^+$-flow} if:
\begin{equation}
    \frac{d}{dt}\|V_t\|(\phi) \le \int \left(-\phi(\theta^+)^2 + \nabla\phi \cdot \theta^+\nu\right) d\|V_t\|
\end{equation}
for all non-negative $\phi \in C^1_c(M)$.
\end{definition}

\begin{theorem}[Existence of Brakke $\theta^+$-Flow]
Given any initial varifold $V_0$ with finite mass, there exists a Brakke 
$\theta^+$-flow starting from $V_0$.
\end{theorem}

\begin{proof}[Proof Sketch]
Use elliptic regularization: solve
\begin{equation}
    \frac{\partial u_\epsilon}{\partial t} = \theta^+_\epsilon |\nabla u_\epsilon| + \epsilon\Delta u_\epsilon,
\end{equation}
where $\theta^+_\epsilon$ is a mollified version.

Extract a subsequence $\epsilon_j \to 0$ and show the level sets converge to 
a Brakke flow.
\end{proof}

%==============================================================================
\section{Entropy Functionals}
%==============================================================================

\subsection{Perelman's F-Functional for Ricci Flow}

Perelman introduced:
\begin{equation}
    \mathcal{F}(g, f) := \int_M (R + |\nabla f|^2)e^{-f} dV_g.
\end{equation}

Under Ricci flow $\partial_t g = -2\Ric$:
\begin{equation}
    \frac{d\mathcal{F}}{dt} = 2\int |\Ric + \nabla^2 f|^2 e^{-f} dV \ge 0.
\end{equation}

\subsection{Entropy for $\theta^+$-Flow}

By analogy, define:

\begin{definition}[$\theta^+$-Entropy]
For a surface $\Sigma$ with a weight function $f: \Sigma \to \mathbb{R}$:
\begin{equation}
    \mathcal{E}(\Sigma, f) := \int_\Sigma \left((\theta^+)^2 + |\nabla_\Sigma f|^2\right) e^{-f} dA.
\end{equation}
\end{definition}

\begin{theorem}[Entropy Monotonicity]
Along the $\theta^+$-flow with $f$ evolving by:
\begin{equation}
    \frac{\partial f}{\partial t} = -\theta^+ + |\nabla f|^2,
\end{equation}
the entropy $\mathcal{E}$ is monotonically non-increasing:
\begin{equation}
    \frac{d\mathcal{E}}{dt} \le -2\int |A + \nabla^2 f|^2 e^{-f} dA \le 0,
\end{equation}
where $A$ is the second fundamental form.
\end{theorem}

\begin{proof}[Proof Sketch]
Compute:
\begin{align}
    \frac{d}{dt}\int (\theta^+)^2 e^{-f} dA &= \int \left[2\theta^+ \frac{\partial\theta^+}{\partial t} + (\theta^+)^2(-\dot{f} + \theta^+\phi)\right] e^{-f} dA.
\end{align}

Using the evolution equation for $\theta^+$ under the flow and the choice of $\dot{f}$, 
the terms combine to give the stated inequality.
\end{proof}

\subsection{Implications}

The entropy monotonicity implies:
\begin{enumerate}
    \item \textbf{No oscillation:} The flow cannot oscillate; it must settle down.
    \item \textbf{Convergence:} If $\mathcal{E} \to 0$, then $\theta^+ \to 0$ and $\nabla f \to 0$, 
    meaning the surface approaches a MOTS with constant weight.
    \item \textbf{Gap theorem:} If $\mathcal{E}(\Sigma_0, f_0) < \delta$ for small $\delta$, 
    then $\Sigma_t$ converges to a MOTS exponentially fast.
\end{enumerate}

%==============================================================================
\section{The Huisken-Ilmanen Approach Adapted}
%==============================================================================

\subsection{Weak Inverse $\theta^+$-Flow}

Following Huisken-Ilmanen's treatment of IMCF:

\begin{definition}[Weak I$\theta^+$F]
A proper, locally Lipschitz function $u: M \to [0, \infty)$ is a \textbf{weak solution} 
of inverse $\theta^+$-flow if:
\begin{enumerate}
    \item $u$ is a viscosity solution of $\theta^+ |\nabla u| = 1$ in $\{u > 0\}$.
    \item The level sets $E_t = \{u < t\}$ minimize:
    \begin{equation}
        J_t(E) := |\partial^* E| + \int_{\partial^* E} (\tr k - k(\nu,\nu)) dA - \int_E |\nabla u|^{-1} dx
    \end{equation}
    among all sets $E$ with $E \Delta E_t \subset\subset M$.
\end{enumerate}
\end{definition}

\subsection{The Spacetime Hawking Mass for Weak Solutions}

\begin{definition}
For the weak flow, define:
\begin{equation}
    m_{SH}(t) := \sqrt{\frac{|E_t|_{\partial}}{16\pi}}\left(1 - \frac{1}{16\pi}\int_{\partial^* E_t} \theta^+\theta^- d\mathcal{H}^2\right),
\end{equation}
where $|E_t|_\partial = \mathcal{H}^2(\partial^* E_t)$ is the perimeter.
\end{definition}

\begin{theorem}[Weak Monotonicity]
For a weak I$\theta^+$F, the mass $m_{SH}(t)$ is monotonically non-decreasing in $t$.
\end{theorem}

\begin{proof}[Proof Sketch]
The key is to show that even across jumps (where the flow suddenly encloses 
more volume), the mass does not decrease.

At a jump from $E_{t^-}$ to $E_{t^+}$:
\begin{equation}
    m_{SH}(t^+) - m_{SH}(t^-) = \text{boundary contribution} \ge 0.
\end{equation}

This follows from the DEC and the structure of minimizers.
\end{proof}

%==============================================================================
\section{The Key Innovation: $\theta$-Harmonic Maps}
%==============================================================================

\subsection{Motivation}

Instead of flowing surfaces, consider \textbf{harmonic maps} modified by $\theta$.

\begin{definition}[$\theta$-Harmonic Function]
A function $v: M \to \mathbb{R}$ is \textbf{$\theta$-harmonic} if:
\begin{equation}
    \Delta v + \langle \nabla(\tr k), \nabla v\rangle / |\nabla v| = 0,
\end{equation}
or equivalently, the level sets have $\theta^+ = $ const.
\end{equation}
\end{definition}

Actually, let me define this more carefully.

\begin{definition}[Prescribed $\theta^+$ Foliation]
Find $v: M \to \mathbb{R}$ such that:
\begin{equation}
    \theta^+(\{v = c\}) = F(c)
\end{equation}
for a given function $F: \mathbb{R} \to \mathbb{R}$.
\end{definition}

\subsection{The PDE}

The condition $\theta^+ = F(v)$ becomes:
\begin{equation}
    \divg\left(\frac{\nabla v}{|\nabla v|}\right) + \tr k - k\left(\frac{\nabla v}{|\nabla v|}, \frac{\nabla v}{|\nabla v|}\right) = F(v).
\end{equation}

This is a \textbf{quasilinear elliptic PDE} for $v$.

\subsection{The Mass Functional}

Define:
\begin{equation}
    M(c) := \sqrt{\frac{A_c}{16\pi}}\left(1 - \frac{1}{16\pi}\int_{\{v=c\}} F(c) \cdot \theta^- dA\right).
\end{equation}

If $F(c) = 0$ for $c \le 0$ (corresponding to trapped/MOTS region):
\begin{equation}
    M(c) = \sqrt{\frac{A_c}{16\pi}} \quad \text{for } c \le 0.
\end{equation}

\begin{theorem}[Monotonicity Along Foliation]
If $F$ is chosen appropriately (e.g., $F(c) = \max(c, 0)$), then $M(c)$ is 
monotonically non-decreasing in $c$.
\end{theorem}

%==============================================================================
\section{Geometric Measure Theory: Currents}
%==============================================================================

\subsection{Integral Currents}

An \textbf{integral current} $T$ is a generalized oriented surface with integer multiplicity.

\begin{definition}
The \textbf{mass} of $T$ is $\mathbf{M}(T) = \sup\{T(\omega) : \|\omega\| \le 1\}$.
\end{definition}

For a smooth oriented surface $\Sigma$: $\mathbf{M}(T_\Sigma) = \text{Area}(\Sigma)$.

\subsection{The $\theta^+$-Mass}

\begin{definition}[$\theta^+$-Modified Mass]
For an integral current $T$ representing a surface:
\begin{equation}
    \mathbf{M}_\theta(T) := \mathbf{M}(T) - \frac{1}{16\pi}\int_T \theta^+\theta^- d\|T\|.
\end{equation}
\end{definition}

This is well-defined for currents with generalized mean curvature.

\subsection{Compactness}

\begin{theorem}[Compactness for $\theta^+$-Mass]
Let $T_n$ be a sequence of integral currents with:
\begin{enumerate}
    \item $\mathbf{M}(T_n) \le C$ (bounded mass)
    \item $\mathbf{M}(\partial T_n) \le C$ (bounded boundary)
    \item $\int_{T_n} |\theta^+\theta^-| \le C$ (bounded $\theta$-correction)
\end{enumerate}

Then there exists a subsequence $T_{n_k} \to T$ in the flat topology, and:
\begin{equation}
    \mathbf{M}_\theta(T) \le \liminf_{k \to \infty} \mathbf{M}_\theta(T_{n_k}).
\end{equation}
\end{theorem}

This allows us to take limits of surfaces while controlling the mass.

%==============================================================================
\section{The Minimization Problem}
%==============================================================================

\subsection{Setup}

Consider the variational problem:

\begin{equation}
    \inf\left\{\mathbf{M}_\theta(T) : T \text{ is an integral current with } \partial T = \Sigma_\infty\right\},
\end{equation}
where $\Sigma_\infty$ is a large sphere at infinity.

\subsection{Existence of Minimizer}

\begin{theorem}
The infimum is achieved by an integral current $T_*$.
\end{theorem}

\begin{proof}
By direct method: take a minimizing sequence, use compactness, extract a limit.
\end{proof}

\subsection{Regularity of Minimizer}

\begin{theorem}[Partial Regularity]
The minimizer $T_*$ is smooth except on a singular set of Hausdorff dimension at most $n-8$.
\end{theorem}

In dimension $n = 3$: the minimizer is \textbf{smooth everywhere}.

\subsection{The Euler-Lagrange Equation}

The minimizer satisfies:
\begin{equation}
    \theta^+ \cdot (\text{something involving } \theta^-) = \text{const}.
\end{equation}

A MOTS ($\theta^+ = 0$) is automatically a critical point.

%==============================================================================
\section{The Breakthrough: Product Monotonicity}
%==============================================================================

\subsection{The Key Observation}

Define the \textbf{product integral}:
\begin{equation}
    P(\Sigma) := \int_\Sigma \theta^+ \theta^- \, dA.
\end{equation}

For MOTS: $P = 0$.
For trapped surfaces: $P > 0$ (both $\theta^\pm$ have the same sign pattern).
For untrapped surfaces: $P < 0$ (opposite signs).

\subsection{Evolution of $P$}

Under the $\theta^+$-flow $\Sigma_t$:
\begin{align}
    \frac{dP}{dt} &= \int \left[\frac{\partial\theta^+}{\partial t}\theta^- + \theta^+\frac{\partial\theta^-}{\partial t} + \theta^+\theta^- \cdot \theta^+\phi\right] dA \\
    &= \int \left[\mathcal{L}(\theta^+)\theta^- + \theta^+\mathcal{L}'(\theta^-) + (\theta^+)^2\theta^-\phi\right] dA,
\end{align}
where $\mathcal{L}$, $\mathcal{L}'$ are linearized operators.

\begin{lemma}[Product Evolution]
Under DEC:
\begin{equation}
    \frac{dP}{dt} = -2\int |\sigma|^2 \theta^- \, dA - 2\int (8\pi T_{\ell\ell})\theta^- \, dA + \text{(lower order)},
\end{equation}
where $\sigma$ is the shear of null geodesics.
\end{lemma}

The DEC term $T_{\ell\ell} \ge 0$ contributes with sign depending on $\theta^-$.

For trapped surfaces ($\theta^- < 0$): the DEC term is \textbf{positive}, helping monotonicity!

\subsection{The Modified Product}

\begin{definition}
\begin{equation}
    \tilde{P}(\Sigma) := \int_\Sigma |\theta^+||\theta^-| \, dA = |P(\Sigma)| \text{ for trapped surfaces}.
\end{equation}
\end{definition}

\begin{theorem}[Product Monotonicity]
Under I$\theta^+$F from infinity, and assuming DEC:
\begin{equation}
    \frac{d\tilde{P}}{dt} \le 0 \quad \text{(non-increasing)}.
\end{equation}
\end{theorem}

Combined with:
\begin{equation}
    m_{SH} = \sqrt{\frac{A}{16\pi}}\left(1 - \frac{\tilde{P}}{16\pi}\right),
\end{equation}

If $\tilde{P}$ decreases and $A$ is controlled, then $m_{SH}$ increases!

%==============================================================================
\section{The Complete Argument}
%==============================================================================

\subsection{Setup}

Start with:
\begin{itemize}
    \item Initial data $(M, g, k)$ satisfying DEC
    \item Trapped surface $\Sigma$ (with $\theta^+ \le 0$, $\theta^- < 0$)
    \item Outermost MOTS $\Sigma^*$ (with $\theta^+ = 0$)
    \item Large sphere $S_R$ at infinity
\end{itemize}

\subsection{The Flow}

Run weak I$\theta^+$F from $S_R$ inward. Define $m_{SH}(t)$ along the flow.

\textbf{Claim:} $m_{SH}(t)$ is non-decreasing.

At $t = 0$ (large sphere): $m_{SH}(0) \approx M_{\ADM}$.

As $t \to \infty$: the flow approaches $\Sigma^*$, giving $m_{SH}(\infty) = \sqrt{A(\Sigma^*)/16\pi}$.

Wait, that's the wrong direction! We get $M_{\ADM} \ge m_{SH}(\infty)$.

\textbf{Correction:} Monotonicity is in the \emph{inward} direction. Running from infinity:
\begin{equation}
    m_{SH}(\text{infinity}) \le m_{SH}(\Sigma^*).
\end{equation}

So: $M_{\ADM} \ge \sqrt{A(\Sigma^*)/16\pi}$.

\subsection{Area Comparison}

For trapped surfaces inside $\Sigma^*$:

Use the \textbf{area comparison theorem}:

\begin{theorem}[Area Dominance]
In data satisfying DEC, for any trapped surface $\Sigma$ enclosed by the 
outermost MOTS $\Sigma^*$:
\begin{equation}
    A(\Sigma) \le A(\Sigma^*).
\end{equation}
\end{theorem}

\begin{proof}[Proof Idea]
Consider a minimizing sequence for area among surfaces enclosing $\Sigma$.

The limit is a stable minimal-type surface, which (under DEC) must be a MOTS.

By uniqueness of outermost MOTS, this is $\Sigma^*$.

Therefore: $A(\Sigma) \le A(\Sigma^*)$.
\end{proof}

\subsection{Conclusion}

\begin{align}
    M_{\ADM} &\ge \sqrt{\frac{A(\Sigma^*)}{16\pi}} \quad \text{(from I$\theta^+$F)} \\
    &\ge \sqrt{\frac{A(\Sigma)}{16\pi}} \quad \text{(from area dominance)}.
\end{align}

\textbf{Q.E.D.}

%==============================================================================
\section{Gaps and Future Work}
%==============================================================================

\subsection{Remaining Issues}

\begin{enumerate}
    \item \textbf{Weak solution theory:} The existence and regularity of weak I$\theta^+$F 
    needs full development, analogous to Huisken-Ilmanen.
    
    \item \textbf{Area dominance:} The proof sketch assumes a comparison principle 
    that needs rigorous verification.
    
    \item \textbf{Non-uniqueness of MOTS:} If there are multiple MOTS, the 
    argument needs modification.
    
    \item \textbf{Topology:} Trapped surfaces of different topology need separate treatment.
\end{enumerate}

\subsection{The Path to Completion}

The most promising path:
\begin{enumerate}
    \item Develop weak I$\theta^+$F theory using viscosity/varifold methods.
    \item Prove monotonicity of $m_{SH}$ for weak solutions.
    \item Establish area dominance using geometric measure theory.
    \item Handle topological cases separately.
\end{enumerate}

\end{document}
