% =========================================================================
%     MIN-MAX AND VARIATIONAL APPROACHES TO THE PENROSE INEQUALITY
%
%     Using Almgren-Pitts theory and variational methods
%
%     Author: Da Xu
%     Date: December 2025
% =========================================================================

\documentclass[12pt]{article}
\usepackage{amsmath,amsthm,amssymb}
\usepackage{mathrsfs}
\usepackage{tcolorbox}

\theoremstyle{plain}
\newtheorem{theorem}{Theorem}[section]
\newtheorem{lemma}[theorem]{Lemma}
\newtheorem{proposition}[theorem]{Proposition}
\newtheorem{corollary}[theorem]{Corollary}

\theoremstyle{definition}
\newtheorem{definition}[theorem]{Definition}
\newtheorem{remark}[theorem]{Remark}

\newcommand{\ADM}{\mathrm{ADM}}
\newcommand{\tr}{\mathrm{tr}}
\newcommand{\Div}{\mathrm{div}}
\newcommand{\Area}{\mathrm{Area}}
\newcommand{\width}{\mathrm{width}}

\title{\textbf{Min-Max Methods for the Penrose Inequality}}
\author{Da Xu}
\date{December 2025}

\begin{document}
\maketitle

\section{Min-Max Theory Background}

\subsection{The Almgren-Pitts Width}

\begin{definition}
The \textbf{width} of a 3-manifold $M$ is:
\[
    W(M) = \inf_\Phi \sup_{t \in [0,1]} \Area(\Phi(t))
\]
where $\Phi: [0,1] \to \mathcal{Z}_2(M)$ is a sweepout by 2-cycles.
\end{definition}

\begin{theorem}[Almgren-Pitts]
For a closed Riemannian 3-manifold, there exists a closed embedded minimal
surface $\Sigma$ with $\Area(\Sigma) = W(M)$.
\end{theorem}

\subsection{Applications to Mass}

\begin{theorem}[Marques-Neves]
For certain manifolds with positive scalar curvature:
\[
    W(M) \leq C(R, \text{topology})
\]
\end{theorem}

\section{Application to Penrose Inequality}

\subsection{Strategy}

\begin{enumerate}
    \item Define a min-max problem involving the trapped surface $\Sigma_0$
    \item Show the width is bounded below by $\sqrt{16\pi} \cdot M_{\ADM}$
    \item Show the width equals or bounds $\Area(\Sigma_0)$
\end{enumerate}

\subsection{The Trapped Width}

\begin{definition}
The \textbf{trapped width} is:
\[
    W_{\text{trap}}(M, \Sigma_0) = \inf_\Phi \sup_{t} \Area(\Phi(t))
\]
where $\Phi$ sweeps out starting from $\Sigma_0$ and ending at a point.
\end{definition}

\begin{lemma}
$W_{\text{trap}}(M, \Sigma_0) \geq \Area(\Sigma_0)$ by definition (the sweepout
starts at $\Sigma_0$).
\end{lemma}

\subsection{Upper Bound on Width}

For asymptotically flat $M$ with mass $M_{\ADM}$:

\begin{proposition}
$W(M) \leq C \cdot M_{\ADM}^2$ for some universal constant $C$.
\end{proposition}

\textbf{Proof idea:} Take a sweepout by large coordinate spheres. The maximum
area is achieved at intermediate scales where $\Area \sim 4\pi r^2$ with 
$r \sim M_{\ADM}$.

\subsection{The Problem}

We have:
\begin{align}
    \Area(\Sigma_0) &\leq W_{\text{trap}}(M, \Sigma_0) \leq W(M) \leq C \cdot M_{\ADM}^2
\end{align}

This gives:
\[
    \Area(\Sigma_0) \leq C \cdot M_{\ADM}^2
\]

which is:
\[
    M_{\ADM} \geq \frac{1}{\sqrt{C}} \sqrt{\Area(\Sigma_0)}
\]

\textbf{But:} This requires $C = 16\pi$ with the right constant!

\textbf{Problem:} The standard min-max doesn't give $C = 16\pi$. The width
depends on global geometry, not just mass.

\section{The Area-Minimizing Approach}

\subsection{Outermost Minimal Area Enclosure}

\begin{definition}
The \textbf{outermost minimal area enclosure} (OMAE) of $\Sigma_0$ is the surface
$\Sigma_{\text{out}}$ that:
\begin{enumerate}
    \item Encloses $\Sigma_0$
    \item Has minimal area among all such surfaces
    \item Is outermost (no other such surface encloses it)
\end{enumerate}
\end{definition}

\begin{lemma}
The OMAE exists and is a smooth minimal surface ($H = 0$).
\end{lemma}

\subsection{Comparison}

For the OMAE $\Sigma_{\text{out}}$:
\begin{itemize}
    \item $\Area(\Sigma_{\text{out}}) \leq \Area(\Sigma_0)$ (by minimality)
    \item $M_{\ADM} \geq \sqrt{\Area(\Sigma_{\text{out}})/(16\pi)}$ (Riemannian Penrose for minimal)
\end{itemize}

Therefore:
\[
    M_{\ADM} \geq \sqrt{\frac{\Area(\Sigma_{\text{out}})}{16\pi}}
\]

But $\Area(\Sigma_{\text{out}}) \leq \Area(\Sigma_0)$, so this gives:
\[
    M_{\ADM} \geq \sqrt{\frac{\Area(\Sigma_{\text{out}})}{16\pi}} \leq \sqrt{\frac{\Area(\Sigma_0)}{16\pi}}
\]

\textbf{Wrong direction!} We need $M_{\ADM} \geq \sqrt{\Area(\Sigma_0)/(16\pi)}$,
not $\leq$.

\section{The Maximum Area Approach Revisited}

\subsection{Maximum Trapped Surface}

\begin{definition}
The \textbf{maximum trapped surface} is:
\[
    \Sigma_{\max} = \arg\max\{\Area(\Sigma) : \Sigma \text{ is trapped}\}
\]
\end{definition}

From previous analysis: $\Sigma_{\max}$ is a MOTS with 
$\int_{\Sigma_{\max}} (\tr_\Sigma k) \phi_1 \, dA \geq 0$.

\subsection{The Area Relationship}

By maximality: $\Area(\Sigma_{\max}) \geq \Area(\Sigma_0)$.

If we could prove Penrose for $\Sigma_{\max}$:
\[
    M_{\ADM} \geq \sqrt{\frac{\Area(\Sigma_{\max})}{16\pi}} \geq \sqrt{\frac{\Area(\Sigma_0)}{16\pi}}
\]

\textbf{But:} Penrose for $\Sigma_{\max}$ requires $\tr_\Sigma k \geq 0$ pointwise,
which we only have in a weighted average sense.

\section{The Schoen Conjecture Approach}

\subsection{Schoen's Conjecture}

\begin{conjecture}[Schoen]
For an asymptotically flat 3-manifold $(M, g)$ with $R \geq 0$:
\[
    M_{\ADM} \geq \sqrt{\frac{W(M)}{16\pi}}
\]
where $W(M)$ is the min-max width.
\end{conjecture}

If true, and if $W(M) \geq \Area(\Sigma_0)$ for trapped surfaces, we'd have Penrose.

\subsection{Status}

The conjecture is \textbf{open}. Known results:
\begin{itemize}
    \item True for spherically symmetric data
    \item True for small perturbations of Schwarzschild
    \item Unknown in general
\end{itemize}

\section{The Simon-Smith Approach}

\subsection{Unstable Minimal Surfaces}

Simon-Smith theory produces \textbf{unstable} minimal surfaces via min-max.

For an unstable minimal surface $\Sigma$:
\begin{itemize}
    \item There exist variations decreasing area
    \item The second variation has negative eigenvalues
\end{itemize}

\subsection{Connection to Trapped Surfaces}

\textbf{Observation:} Trapped surfaces are ``more negative'' than unstable minimal surfaces
in the sense that $H < 0$ everywhere.

\textbf{Question:} Is there a min-max theory for trapped surfaces directly?

\subsection{Trapped Surface Min-Max}

Define:
\[
    T(M, k) = \inf_\Phi \sup_t \Area(\Phi(t))
\]
where $\Phi$ sweeps through trapped surfaces.

\textbf{Problem:} The set of trapped surfaces is not a linear space, making
standard min-max theory inapplicable.

\section{The Barrier Method}

\subsection{Idea}

Use $\Sigma_0$ as a barrier:
\begin{itemize}
    \item $\Sigma_0$ is trapped (innermost barrier to outgoing light)
    \item Any surface outside $\Sigma_0$ in the trapped region has area $\leq \Area(\Sigma_0)$?
\end{itemize}

\textbf{Wrong:} Area can increase or decrease in the trapped region; the trapped
condition only constraints null expansions, not area directly.

\subsection{The Hawking Area Theorem}

In a spacetime satisfying NEC:
\begin{itemize}
    \item Horizon area is non-decreasing to the future
    \item But this requires a horizon (global structure)
\end{itemize}

For initial data alone, no area theorem applies to arbitrary trapped surfaces.

\begin{tcolorbox}[colback=yellow!10, colframe=orange!75!black, title=\textbf{Key Insight}]
\textbf{The min-max approach reveals:}

For minimal surfaces (Riemannian case):
\[
    \Area(\text{outermost minimal}) \leq \Area(\text{any enclosing surface})
\]

For trapped surfaces:
\[
    \Area(\Sigma_0) \text{ has no relation to } \Area(\text{outermost MOTS})
\]

The minimizing property of minimal surfaces is essential for the Riemannian Penrose
inequality. Trapped surfaces don't have this property.
\end{tcolorbox}

\begin{tcolorbox}[colback=red!10, colframe=red!75!black, title=\textbf{Conclusion: Min-Max Methods}]
\textbf{Summary:} Min-max approaches fail because:

\begin{enumerate}
    \item \textbf{Width bounds:} Don't give the precise $16\pi$ constant
    \item \textbf{Area-minimizing:} Works in wrong direction for trapped surfaces
    \item \textbf{Maximum trapped:} Only weighted integral condition
    \item \textbf{Schoen conjecture:} Open
    \item \textbf{Barrier methods:} Trapped surfaces don't form barriers
\end{enumerate}

\textbf{Status:} No min-max method resolves the unconditional case.
\end{tcolorbox}

\end{document}
