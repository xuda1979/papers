% =========================================================================
%     MOTS PENROSE: THE FINAL THEOREM
%
%     Proving Penrose for Marginally Outer Trapped Surfaces
%
%     This is the "Poincaré Conjecture" of our flow program
%
%     Author: Da Xu
%     Date: December 2025
% =========================================================================

\documentclass[12pt]{article}
\usepackage{amsmath,amsthm,amssymb}
\usepackage{mathrsfs}
\usepackage{tcolorbox}

\theoremstyle{plain}
\newtheorem{theorem}{Theorem}[section]
\newtheorem{lemma}[theorem]{Lemma}
\newtheorem{proposition}[theorem]{Proposition}
\newtheorem{corollary}[theorem]{Corollary}
\newtheorem{conjecture}[theorem]{Conjecture}

\theoremstyle{definition}
\newtheorem{definition}[theorem]{Definition}
\newtheorem{remark}[theorem]{Remark}

\newcommand{\ADM}{\mathrm{ADM}}
\newcommand{\tr}{\mathrm{tr}}
\newcommand{\Div}{\mathrm{div}}
\newcommand{\Area}{\mathrm{Area}}
\newcommand{\Ric}{\mathrm{Ric}}
\newcommand{\Cap}{\mathrm{Cap}}

\title{\textbf{The MOTS Penrose Theorem:\\Completing the Program}}
\author{Da Xu}
\date{December 2025}

\begin{document}
\maketitle

\begin{abstract}
We attack the MOTS Penrose inequality: $M_{\ADM} \geq \sqrt{A/(16\pi)}$ for Marginally Outer Trapped Surfaces. This is the final piece needed to complete the $\theta^+$-flow program for the unconditional spacetime Penrose inequality.
\end{abstract}

% =========================================================================
\section{The Statement}
% =========================================================================

\begin{theorem}[MOTS Penrose - To Prove]
Let $(M^3, g, k)$ be asymptotically flat initial data satisfying DEC. Let $\Sigma$ be a stable MOTS (i.e., $\theta^+ = 0$ with stability). Then:
\begin{equation}
    M_{\ADM} \geq \sqrt{\frac{\Area(\Sigma)}{16\pi}}
\end{equation}
\end{theorem}

\subsection{The MOTS Condition}

On $\Sigma$: $\theta^+ = H + \tr_\Sigma k = 0$, so:
\begin{equation}
    H = -\tr_\Sigma k
\end{equation}

\textbf{Two cases:}
\begin{itemize}
    \item \textbf{Type I:} $\tr_\Sigma k \leq 0 \Rightarrow H \geq 0$ (favorable)
    \item \textbf{Type II:} $\tr_\Sigma k > 0 \Rightarrow H < 0$ (unfavorable)
\end{itemize}

% =========================================================================
\section{Type I: MOTS with $H \geq 0$}
% =========================================================================

\subsection{Strategy: Modified IMCF}

When $H > 0$, we can flow outward by IMCF!

\begin{theorem}[IMCF from MOTS]
Let $\Sigma$ be a MOTS with $H > 0$. Run IMCF $\dot{\Sigma} = \nu/H$ from $\Sigma$. Under DEC, the Hawking mass is non-decreasing:
\[
    m_H(t) \leq m_H(t') \quad \text{for } t < t'
\]
and $m_H \to M_{\ADM}$ as $t \to \infty$.
\end{theorem}

\subsection{The Hawking Mass Issue}

For MOTS with $\theta^+ = 0$:
\[
    m_H(\Sigma) = \sqrt{\frac{A}{16\pi}}\left(1 - \frac{1}{16\pi}\int_\Sigma H^2 \, dA\right)
\]

Since $H = -\tr_\Sigma k$ on MOTS:
\[
    m_H(\Sigma) = \sqrt{\frac{A}{16\pi}}\left(1 - \frac{1}{16\pi}\int_\Sigma (\tr_\Sigma k)^2 \, dA\right)
\]

This is LESS than $\sqrt{A/(16\pi)}$ unless $\tr_\Sigma k \equiv 0$.

\subsection{Resolution: The Correct Monotonicity}

\begin{lemma}[Key Observation]
For IMCF starting from MOTS:
\[
    M_{\ADM} \geq m_H(\Sigma_\infty) \geq m_H(\Sigma)
\]
But we need $M_{\ADM} \geq \sqrt{A/(16\pi)}$, not $M_{\ADM} \geq m_H(\Sigma)$.
\end{lemma}

\textbf{The gap:} $\sqrt{A/(16\pi)} - m_H(\Sigma) = \sqrt{A/(16\pi)} \cdot \frac{\int H^2}{16\pi} > 0$.

\subsection{New Idea: Track the Defect}

Define:
\[
    D(\Sigma) = \sqrt{\frac{A}{16\pi}} - m_H(\Sigma) = \sqrt{\frac{A}{16\pi}} \cdot \frac{\int H^2}{16\pi}
\]

\begin{proposition}[Defect Evolution]
Along IMCF from MOTS:
\[
    \frac{d}{dt}\left(\sqrt{\frac{A}{16\pi}}\right) \leq \frac{dm_H}{dt} + \frac{dD}{dt}
\]
If $D$ decreases faster than $m_H$ increases, we get Penrose!
\end{proposition}

\textbf{Analysis needed:} How does $\int H^2$ evolve along IMCF?

% =========================================================================
\section{Type II: MOTS with $H < 0$}
% =========================================================================

This is the critical case. IMCF fails since $H < 0$.

\subsection{Approach 1: Capacity Method}

\begin{definition}[Capacity]
\[
    \Cap(\Sigma) = \inf\left\{\int_M |\nabla u|^2 \, dV : u|_\Sigma = 1, u \to 0 \text{ at } \infty\right\}
\]
\end{definition}

\begin{theorem}[Bray-Miao Inequality]
For asymptotically flat $(M^3, g)$ with non-negative scalar curvature:
\[
    M_{\ADM} \geq \frac{\Cap(\Sigma)}{4\pi}
\]
\end{theorem}

\begin{theorem}[Capacity-Area Inequality]
For any surface $\Sigma$ in $(M^3, g)$:
\[
    \Cap(\Sigma) \geq 2\sqrt{\pi \Area(\Sigma)}
\]
Equality holds for round sphere in flat space.
\end{theorem}

\textbf{Combining:}
\begin{align}
    M_{\ADM} &\geq \frac{\Cap(\Sigma)}{4\pi} \\
    &\geq \frac{2\sqrt{\pi A}}{4\pi} \\
    &= \frac{1}{2\sqrt{\pi}}\sqrt{A} \\
    &= \sqrt{\frac{A}{4\pi}}
\end{align}

This gives $M \geq \sqrt{A/(4\pi)}$, which is STRONGER than $M \geq \sqrt{A/(16\pi)}$!

\begin{corollary}
If both Bray-Miao and capacity-area hold for our MOTS, then Penrose follows with room to spare.
\end{corollary}

\subsection{Issue with Capacity Method}

\textbf{Problem:} The Bray-Miao inequality requires:
\begin{itemize}
    \item $R \geq 0$ (scalar curvature non-negative)
    \item But DEC only gives $R \geq 2|J|^2 - \mu$ in general
\end{itemize}

For initial data $(M, g, k)$:
\[
    R = 2\mu - |k|^2 + (\tr k)^2
\]

DEC: $\mu \geq |J|$.

\textbf{Resolution:} Use Jang equation to pass to $R \geq 0$ setting.

\subsection{Approach 2: Jang Equation Analysis}

The Jang equation surface $(M, \bar{g})$ has $\bar{R} \geq 0$ away from MOTS.

\begin{theorem}[Schoen-Yau/EHLS]
The Jang surface with appropriate boundary conditions satisfies:
\[
    M_{\ADM}(\bar{g}) \geq M_{\ADM}(g)
\]
and has $\bar{R} \geq 0$.
\end{theorem}

\textbf{Strategy:}
\begin{enumerate}
    \item Form Jang surface $(M, \bar{g})$
    \item MOTS $\Sigma$ becomes minimal in $\bar{g}$
    \item Apply Riemannian Penrose to minimal $\Sigma$ in $\bar{g}$
    \item Get $M_{\ADM}(\bar{g}) \geq \sqrt{A_{\bar{g}}(\Sigma)/(16\pi)}$
    \item Need: $M_{\ADM}(g) \approx M_{\ADM}(\bar{g})$ and $A_g(\Sigma) \approx A_{\bar{g}}(\Sigma)$
\end{enumerate}

\subsection{Approach 3: Generalized Jang (Bray-Khuri)}

\begin{theorem}[Bray-Khuri Framework]
The generalized Jang equation with warping function $\phi$ gives:
\[
    M_{\ADM} \geq \sqrt{\frac{A(\Sigma_0)}{16\pi}} + \text{error terms}
\]
where error terms depend on the geometry of MOTS.
\end{theorem}

For MOTS as boundary, the error terms may vanish!

% =========================================================================
\section{The Key Insight: Stability}
% =========================================================================

\subsection{MOTS Stability Operator}

\begin{definition}
The MOTS stability operator is:
\[
    L = -\Delta - 2\omega \cdot \nabla - \frac{1}{2}R_\Sigma + \Div \omega - |\omega|^2 - |\chi^+|^2 + \frac{1}{2}G(\ell, \ell)
\]
where $\omega$ is the connection 1-form and $\chi^+$ is the outgoing shear.
\end{definition}

\begin{definition}
MOTS $\Sigma$ is stable if $\lambda_1(L) \geq 0$ (principal eigenvalue non-negative).
\end{definition}

\subsection{Why Stability Matters}

\begin{theorem}[Andersson-Mars-Simon]
Stable MOTS are outermost in a precise sense: no other MOTS can lie outside a stable MOTS in the same direction.
\end{theorem}

\begin{proposition}
The MOTS reached by $\theta^+$-flow is necessarily stable (or the flow would continue).
\end{proposition}

\subsection{Stability Implies Area Bounds}

\begin{theorem}[Potential - Using Stability]
For stable MOTS $\Sigma$:
\[
    \int_\Sigma \left(\frac{1}{2}R_\Sigma + |\chi^+|^2\right) dA \leq \int_\Sigma \left(\Div \omega - |\omega|^2 + \frac{1}{2}G(\ell,\ell)\right) dA
\]
\end{theorem}

Using Gauss-Bonnet and DEC, this constrains the geometry of $\Sigma$.

% =========================================================================
\section{A Direct Proof Attempt}
% =========================================================================

\subsection{Setup}

Let $\Sigma$ be a stable MOTS with:
\begin{itemize}
    \item $\theta^+ = 0$
    \item $H = -\tr_\Sigma k$
    \item Stability: $\lambda_1(L) \geq 0$
\end{itemize}

\subsection{The Mass Integral}

Consider the mass integral:
\[
    M_{\ADM} = \frac{1}{16\pi}\lim_{r \to \infty} \int_{S_r} (g_{ij,i} - g_{ii,j})\nu^j dA
\]

\subsection{Connecting to $\Sigma$}

Using the divergence theorem and constraint equations:
\[
    M_{\ADM} = \frac{1}{16\pi}\int_\Sigma H_0 \, dA + \frac{1}{16\pi}\int_{M \setminus \Sigma} (\mu - J \cdot \nu) dV + \text{boundary}
\]

where $H_0$ is the mean curvature in the ambient metric sense.

\subsection{For MOTS}

For MOTS, we have $H = -\tr_\Sigma k$, so:
\[
    H_0 = H + (\text{correction from } k)
\]

The exact relationship depends on the embedding.

% =========================================================================
\section{The Schwarzschild Verification}
% =========================================================================

\subsection{Setup}

Schwarzschild in Painlevé-Gullstrand:
\[
    ds^2 = -dt^2 + \left(dr + \sqrt{\frac{2M}{r}}dt\right)^2 + r^2 d\Omega^2
\]

$t = 0$ slice is flat: $ds^2_\Sigma = dr^2 + r^2 d\Omega^2$.

\subsection{The Horizon}

At $r = 2M$:
\begin{itemize}
    \item $\theta^+ = 0$ (MOTS condition)
    \item $\Area = 16\pi M^2$
    \item $H = 1/M > 0$ (positive mean curvature in flat slice)
    \item $\tr_\Sigma k = -1/M < 0$
\end{itemize}

\textbf{Penrose check:}
\[
    M_{\ADM} = M \stackrel{?}{\geq} \sqrt{\frac{16\pi M^2}{16\pi}} = M \quad \checkmark
\]

Equality holds!

\subsection{Type of MOTS}

Since $\tr_\Sigma k = -1/M < 0$, this is Type I (favorable).

$H = 1/M > 0$, so IMCF can proceed.

% =========================================================================
\section{What About Type II in Schwarzschild?}
% =========================================================================

\subsection{Different Slicing}

To get Type II MOTS ($H < 0$) in Schwarzschild, we need a different slicing.

Consider ingoing Eddington-Finkelstein:
\[
    ds^2 = -\left(1 - \frac{2M}{r}\right)dv^2 + 2dv\,dr + r^2 d\Omega^2
\]

The $v = $ const slices give different induced geometry.

\subsection{Analysis}

On $v = $ const slice, compute:
\begin{itemize}
    \item Induced metric
    \item Extrinsic curvature $k$
    \item Mean curvature $H$ of sphere at $r = 2M$
    \item Trace $\tr_\Sigma k$
\end{itemize}

\textbf{Expectation:} By choosing appropriate slicing, we can make $\tr_\Sigma k > 0$ at horizon, giving Type II MOTS.

\subsection{But Penrose Still Holds!}

Regardless of slicing:
\begin{itemize}
    \item $M_{\ADM} = M$ (unchanged)
    \item $\Area(\text{horizon}) = 16\pi M^2$ (unchanged, intrinsic)
    \item Penrose: $M \geq M$ ✓
\end{itemize}

\textbf{Key point:} Penrose inequality depends only on $M_{\ADM}$ and $\Area$, both of which are slicing-independent!

% =========================================================================
\section{The Universal Argument}
% =========================================================================

\subsection{Key Observation}

\begin{proposition}
For any MOTS $\Sigma$:
\begin{enumerate}
    \item $M_{\ADM}$ is a property of the spacetime, not the slice
    \item $\Area(\Sigma)$ is intrinsic to $\Sigma$
    \item Penrose inequality is a statement about these invariants
\end{enumerate}
\end{proposition}

\subsection{Reduction to Favorable Case}

\begin{theorem}[Potential Key Result]
Given initial data $(M^3, g, k)$ with MOTS $\Sigma$, there exists a different slice $(M^3, g', k')$ with the same $M_{\ADM}$ and $\Area(\Sigma)$, but where $\Sigma$ is Type I MOTS.
\end{theorem}

If true, we can always reduce to the favorable case!

\subsection{Sketch of Proof}

The freedom in choosing the slice corresponds to the lapse and shift.

By choosing appropriate lapse:
\[
    k'_{ij} = k_{ij} + \text{lapse contribution}
\]

We can adjust $\tr_\Sigma k$ while preserving the MOTS condition $\theta^+ = 0$.

\textbf{Technical issue:} Need to verify this preserves DEC and asymptotic flatness.

% =========================================================================
\section{Conclusion and Roadmap}
% =========================================================================

\subsection{Summary of Approaches}

\begin{enumerate}
    \item \textbf{Type I:} IMCF monotonicity (technical issues with Hawking mass)
    \item \textbf{Type II:} Capacity method (need $R \geq 0$)
    \item \textbf{Type II:} Jang equation (MOTS becomes minimal)
    \item \textbf{Universal:} Slice change to reduce to Type I
\end{enumerate}

\subsection{Most Promising Path}

\begin{tcolorbox}[title=Recommended Strategy]
\begin{enumerate}
    \item Show that for any MOTS, there exists a slice where it's Type I
    \item Apply IMCF from Type I MOTS
    \item Prove Penrose using mass monotonicity
    \item Transfer back to original slice (invariants preserved)
\end{enumerate}
\end{tcolorbox}

\subsection{The Complete Program}

Once MOTS Penrose is established:
\begin{enumerate}
    \item Start with any trapped surface $\Sigma_0$
    \item Run $\theta^+$-flow: area increases
    \item Reach MOTS $\Sigma^*$ with $\Area(\Sigma^*) \geq \Area(\Sigma_0)$
    \item Apply MOTS Penrose: $M \geq \sqrt{A(\Sigma^*)/(16\pi)}$
    \item Conclude: $M \geq \sqrt{A(\Sigma_0)/(16\pi)}$
\end{enumerate}

\textbf{THE UNCONDITIONAL SPACETIME PENROSE INEQUALITY!}

\end{document}
