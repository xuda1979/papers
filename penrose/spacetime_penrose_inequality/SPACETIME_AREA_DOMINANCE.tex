%% SPACETIME_AREA_DOMINANCE.tex
%%
%% THE SPACETIME APPROACH: Using Null Geometry Directly
%% December 2025
%%
%% The Key: Work with the EVENT HORIZON, not just the initial data.

\documentclass[11pt]{amsart}
\usepackage{amsmath,amssymb,amsthm}
\usepackage{xcolor}
\usepackage{tcolorbox}

\tcbuselibrary{theorems}

\newtcolorbox{breakthrough}{
    colback=yellow!10!white,
    colframe=red!75!black,
    title={\textbf{BREAKTHROUGH}}
}

\newtcolorbox{theorem_box}{
    colback=blue!5!white,
    colframe=blue!75!black,
}

\newtheorem{theorem}{Theorem}
\newtheorem{lemma}[theorem]{Lemma}
\newtheorem{proposition}[theorem]{Proposition}
\newtheorem{corollary}[theorem]{Corollary}
\theoremstyle{definition}
\newtheorem{definition}[theorem]{Definition}
\newtheorem{remark}[theorem]{Remark}

\newcommand{\Area}{\mathrm{Area}}
\newcommand{\Vol}{\mathrm{Vol}}
\newcommand{\divv}{\mathrm{div}}
\DeclareMathOperator{\tr}{tr}

\title{The Spacetime Approach to Area Dominance:\\
Event Horizon Geometry and Trapped Surfaces}
\author{December 2025}

\begin{document}
\maketitle

%% ============================================================================
\section{The Revolutionary Perspective}
%% ============================================================================

\textbf{Previous approaches:} Work on the 3D initial data surface $\mathcal{C}$.

\textbf{New approach:} Work in the full 4D spacetime, using the EVENT HORIZON.

\begin{breakthrough}
\textbf{KEY INSIGHT:}

The trapped surface $\Sigma$ and MOTS $\Sigma^*$ both lie on the initial data 
slice $\mathcal{C}$. But they are ALSO cross-sections of the EVENT HORIZON $\mathcal{H}$.

The Hawking Area Theorem says: Cross-sections of $\mathcal{H}$ have 
NON-DECREASING area toward the future!

\textbf{Use the PAST direction to get Area Dominance!}
\end{breakthrough}

%% ============================================================================
\section{Setup and Key Structures}
%% ============================================================================

\subsection{The Spacetime}

$(M^4, g_{\mu\nu})$ is a spacetime satisfying:
\begin{itemize}
    \item DEC: $T_{\mu\nu}V^\mu W^\nu \ge 0$ for future-causal $V, W$
    \item Asymptotically flat
    \item Contains a black hole with event horizon $\mathcal{H}$
\end{itemize}

\subsection{The Event Horizon}

$\mathcal{H}$ is a null hypersurface - the boundary of the black hole region.

$\mathcal{H}$ is generated by null geodesics with tangent $\ell^\mu$.

Cross-sections $S \subset \mathcal{H}$ are 2-surfaces.

\subsection{The Initial Data Slice}

$\mathcal{C}$ intersects $\mathcal{H}$ in a 2-surface.

By assumption (WCC), this intersection is the MOTS $\Sigma^*$.

The trapped surface $\Sigma$ lies inside $\Sigma^*$ on $\mathcal{C}$.

%% ============================================================================
\section{The Hawking Area Theorem}
%% ============================================================================

\begin{theorem}[Hawking 1971]
Let $\mathcal{H}$ be the event horizon in a spacetime satisfying NEC.

Let $S_1, S_2$ be cross-sections of $\mathcal{H}$ with $S_1$ to the past of $S_2$.

Then $\Area(S_1) \le \Area(S_2)$.
\end{theorem}

\textbf{Proof:} The expansion $\theta$ of the null generators satisfies 
Raychaudhuri:
\begin{equation}
    \frac{d\theta}{d\lambda} = -\frac{\theta^2}{2} - \sigma^2 - R_{\mu\nu}\ell^\mu\ell^\nu
\end{equation}

Under NEC: $R_{\mu\nu}\ell^\mu\ell^\nu \ge 0$, so $\frac{d\theta}{d\lambda} \le 0$.

On the event horizon: $\theta \ge 0$ (by definition - no focusing).

Area formula: $\frac{d\Area}{d\lambda} = \int \theta \, dA \ge 0$.

%% ============================================================================
\section{The Key Question}
%% ============================================================================

\textbf{Question:} Can we relate the trapped surface $\Sigma$ to a cross-section 
of $\mathcal{H}$?

\textbf{Issue:} $\Sigma$ is NOT on the event horizon! It's inside the black hole.

\subsection{The Apparent Horizon vs Event Horizon}

\begin{itemize}
    \item Event Horizon $\mathcal{H}$: Global, teleological (depends on future)
    \item Apparent Horizon / MOTS $\Sigma^*$: Local, on each slice
\end{itemize}

Under cosmic censorship: The apparent horizon lies inside or on the event horizon.

\textbf{Key fact:} $\Sigma^* \subseteq \mathcal{H} \cap \mathcal{C}$ or 
$\Sigma^*$ inside $\mathcal{H}$.

%% ============================================================================
\section{The New Construction: Past Light Cone}
%% ============================================================================

\begin{breakthrough}
\textbf{CONSTRUCTION:}

From the MOTS $\Sigma^*$, shoot INGOING past-directed null geodesics.

These form a null hypersurface $\mathcal{N}^-(\Sigma^*)$.

The trapped surface $\Sigma$ (being inside $\Sigma^*$) may lie on or inside 
$\mathcal{N}^-(\Sigma^*)$.
\end{breakthrough}

\subsection{Properties of $\mathcal{N}^-(\Sigma^*)$}

$\mathcal{N}^-$ is generated by ingoing null geodesics from $\Sigma^*$.

At $\Sigma^*$: the ingoing expansion $\theta^-|_{\Sigma^*} < 0$ (since MOTS has 
$\theta^+ = 0$ but $\theta^- < 0$ for trapped).

Wait, MOTS has $\theta^+ = 0$. What about $\theta^-$?

\textbf{For MOTS:} $\theta^+ = 0$, but $\theta^-$ can be any sign!

If $\Sigma^*$ is the outermost MOTS, it's not necessarily trapped 
(doesn't require $\theta^- < 0$).

\subsection{Refinement}

Let me reconsider. For the outermost MOTS:
\begin{itemize}
    \item $\theta^+ = 0$ (definition)
    \item $\theta^-$ is typically negative (physical expectation)
\end{itemize}

\textbf{Assume:} $\theta^-(\Sigma^*) < 0$.

Then the ingoing null hypersurface from $\Sigma^*$ has focusing.

%% ============================================================================
\section{Area Along Ingoing Null from MOTS}
%% ============================================================================

Consider the ingoing null hypersurface $\mathcal{N}^-$ from $\Sigma^*$.

Let $S_\lambda$ be cross-sections at affine parameter $\lambda$ (increasing 
toward the past/interior).

At $\lambda = 0$: $S_0 = \Sigma^*$.

\subsection{Area Evolution}

\begin{equation}
    \frac{d\Area(S_\lambda)}{d\lambda} = \int_{S_\lambda} \theta^- dA
\end{equation}

If $\theta^- < 0$ along $\mathcal{N}^-$:
\begin{equation}
    \Area(S_\lambda) < \Area(S_0) = \Area(\Sigma^*) \quad \text{for } \lambda > 0
\end{equation}

\textbf{The cross-sections DECREASE in area as we go inward!}

\subsection{The Connection to $\Sigma$}

\textbf{Question:} Is the trapped surface $\Sigma$ a cross-section of 
$\mathcal{N}^-$?

\textbf{Generally: NO.} $\Sigma$ is a spacelike surface in $\mathcal{C}$, while 
$S_\lambda$ are cross-sections of the null hypersurface.

\textbf{But:} We can compare $\Sigma$ to the cross-section of $\mathcal{N}^-$ 
at the same "level."

%% ============================================================================
\section{The Comparison Argument}
%% ============================================================================

\subsection{Setup}

Let $\Sigma$ be trapped, lying inside $\Sigma^*$ on the slice $\mathcal{C}$.

The ingoing null geodesics from $\Sigma^*$ enter the interior.

Let $S$ be the cross-section of $\mathcal{N}^-$ that "corresponds to" $\Sigma$.

\subsection{Defining the Correspondence}

\textbf{Option 1:} $S$ is the cross-section at the same affine distance.

\textbf{Option 2:} $S$ is the cross-section reached by null geodesics that 
hit $\Sigma$.

\textbf{Option 3:} Compare at the same "time" on some auxiliary foliation.

\subsection{The Area Inequality}

By the focusing along $\mathcal{N}^-$:
\begin{equation}
    \Area(S) \le \Area(\Sigma^*)
\end{equation}

\textbf{Now we need:} $\Area(\Sigma) \le \Area(S)$.

This requires comparing $\Sigma$ (on $\mathcal{C}$) with $S$ (on $\mathcal{N}^-$).

%% ============================================================================
\section{The Spacelike-Null Comparison}
%% ============================================================================

\begin{theorem}[Proposed]
Let $\Sigma$ be a trapped surface on $\mathcal{C}$, and $S$ a cross-section 
of the ingoing null hypersurface from $\Sigma^*$.

If $\Sigma$ and $S$ enclose the same "spatial region", then:
\begin{equation}
    \Area(\Sigma) \le \Area(S)
\end{equation}
\end{theorem}

\textbf{Justification:} $S$ is a "null slice" while $\Sigma$ is a "spacelike 
slice" of the same region.

In Schwarzschild: both are spheres of the same area coordinate $r$!

But in general: the shapes differ.

\subsection{The Issue}

The comparison between spacelike and null cross-sections is NOT straightforward.

They don't enclose the same region in general.

%% ============================================================================
\section{Alternative: The Doubled Manifold Trick}
%% ============================================================================

\begin{breakthrough}
\textbf{NEW IDEA: DOUBLED INITIAL DATA}

Consider the "doubled" initial data: $\mathcal{C}$ glued to its mirror along 
$\Sigma^*$.

The MOTS $\Sigma^*$ becomes a minimal surface in the doubled manifold!

Apply the Riemannian Penrose inequality to the doubled manifold.
\end{breakthrough}

\subsection{The Doubling}

Take $(\mathcal{C}, g, k)$ and its reflection $(\mathcal{C}, g, -k)$.

Glue along $\Sigma^*$ (where $\theta^+ = 0$).

The result: $(\hat{\mathcal{C}}, \hat{g})$ with $\Sigma^*$ as a minimal surface 
(since $H_{\Sigma^*} + P = 0$ and $H_{\Sigma^*} - P = 0$ after doubling).

Wait, this doesn't work directly because $k$ changes sign.

\subsection{Revised Doubling}

The Bray-Khuri approach uses the Jang equation to create a Riemannian manifold 
from the initial data.

On the Jang surface, the MOTS corresponds to a minimal surface or asymptotic 
cylinder.

%% ============================================================================
\section{The Final Picture}
%% ============================================================================

After all these explorations, here's the situation:

\textbf{What we HAVE:}
\begin{enumerate}
    \item Hawking Area Theorem: Areas increase along event horizon toward future
    \item MOTS on slice corresponds to cross-section of horizon (under WCC)
    \item Trapped surfaces lie inside the MOTS
    \item Ingoing null from MOTS has DECREASING area
\end{enumerate}

\textbf{What we NEED:}
\begin{enumerate}
    \item A comparison between trapped $\Sigma$ and null cross-section $S$
    \item Specifically: $\Area(\Sigma) \le \Area(S) \le \Area(\Sigma^*)$
\end{enumerate}

\textbf{The gap:} The middle comparison $\Area(\Sigma) \le \Area(S)$ is not 
established.

%% ============================================================================
\section{A Potential Resolution: The Maximum Principle}
%% ============================================================================

\begin{proposition}[Attempted]
Among all surfaces in the region enclosed by $\Sigma^*$ (inside the MOTS), 
the cross-sections of the ingoing null from $\Sigma^*$ have MAXIMAL area 
for their "position."
\end{proposition}

\textbf{Intuition:} Null surfaces are "less curved" than spacelike surfaces 
in a sense, so their cross-sections have larger area.

\textbf{Counter-intuition:} Spacelike surfaces can be positioned to have 
arbitrarily large area (e.g., spiral around many times).

\textbf{With trapped condition:} If $\theta^+ < 0$, the surface is 
"contracting" in the outgoing null direction, which limits its area.

\subsection{The Maximum Principle Argument}

Let $\Omega$ be the region inside $\Sigma^*$.

Consider all surfaces in $\Omega$ with $\theta^+ \le 0$.

\textbf{Claim:} The maximum area is achieved on $\partial\Omega = \Sigma^*$.

\textbf{Proof attempt:}

If $\Sigma \subset \Omega$ has $\theta^+ < 0$, we can flow it outward.

Under the $\theta^+$-flow ($\phi = -\theta^+$):
\begin{equation}
    \frac{d\theta^+}{dt} = L(-\theta^+)
\end{equation}

The linearized operator $L$ has the property that... [this needs detailed analysis].

%% ============================================================================
\section{Honest Conclusion}
%% ============================================================================

The spacetime approach provides:
\begin{itemize}
    \item A clear geometric picture
    \item Connection to Hawking Area Theorem
    \item A potential path via null geometry
\end{itemize}

But it does NOT directly prove Area Dominance because:
\begin{itemize}
    \item Trapped surfaces are NOT cross-sections of null hypersurfaces
    \item The comparison between spacelike and null surfaces is non-trivial
    \item The constraint equations alone don't control the comparison
\end{itemize}

\section{The Status}

\textbf{Area Dominance remains open.}

It is:
\begin{itemize}
    \item True in all explicitly computed examples (Schwarzschild, Kerr, etc.)
    \item Expected from physical intuition
    \item Consistent with cosmic censorship
    \item NOT proven from DEC + constraint equations alone
\end{itemize}

\textbf{A complete proof likely requires EITHER:}
\begin{enumerate}
    \item New geometric insights connecting spacelike and null geometry
    \item Additional assumptions (beyond DEC)
    \item Dynamical arguments using spacetime evolution
    \item Genuinely new mathematical techniques yet to be discovered
\end{enumerate}

\end{document}
