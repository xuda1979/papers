% =========================================================================
%     NOVEL EXPLORATION: THE TRAPPED REGION GEOMETRY
%
%     New mathematical structures in the trapped region
%
%     Author: Da Xu
%     Date: December 2025
% =========================================================================

\documentclass[12pt]{article}
\usepackage{amsmath,amsthm,amssymb}
\usepackage{mathrsfs}
\usepackage{tcolorbox}

\theoremstyle{plain}
\newtheorem{theorem}{Theorem}[section]
\newtheorem{lemma}[theorem]{Lemma}
\newtheorem{proposition}[theorem]{Proposition}
\newtheorem{corollary}[theorem]{Corollary}
\newtheorem{conjecture}[theorem]{Conjecture}

\theoremstyle{definition}
\newtheorem{definition}[theorem]{Definition}
\newtheorem{remark}[theorem]{Remark}
\newtheorem{observation}[theorem]{Observation}

\newcommand{\ADM}{\mathrm{ADM}}
\newcommand{\tr}{\mathrm{tr}}
\newcommand{\Div}{\mathrm{div}}
\newcommand{\Area}{\mathrm{Area}}
\newcommand{\Vol}{\mathrm{Vol}}

\title{\textbf{Novel Exploration: Geometry of the Trapped Region}}
\author{Da Xu}
\date{December 2025}

\begin{document}
\maketitle

\section{A New Perspective: The Trapped Region as a Whole}

\subsection{The Key Insight}

Previous approaches focus on:
\begin{itemize}
    \item Individual surfaces ($\Sigma_0$, $\Sigma^*$)
    \item Flows between surfaces
    \item Local quantities on surfaces
\end{itemize}

\textbf{New perspective:} Consider the \textbf{entire trapped region} $\mathcal{T}$ as a geometric object.

\begin{definition}
The \textbf{trapped region} is:
\[
    \mathcal{T} = \{p \in M : p \text{ lies on some trapped surface}\}
\]
\end{definition}

\subsection{Properties of $\mathcal{T}$}

\begin{proposition}
The trapped region $\mathcal{T}$ is:
\begin{enumerate}
    \item Bounded (contained in a compact set)
    \item Has boundary $\partial\mathcal{T}$ consisting of MOTS
    \item Contains all trapped surfaces
\end{enumerate}
\end{proposition}

\section{The Volume of the Trapped Region}

\subsection{Definition}

\begin{definition}
The \textbf{trapped volume} is:
\[
    V_{\mathcal{T}} = \Vol(\mathcal{T}) = \int_{\mathcal{T}} dV_g
\]
\end{definition}

\subsection{A Volume-Area Inequality?}

\begin{conjecture}[Volume-Area Bound]
For the trapped region:
\[
    V_{\mathcal{T}} \leq C \cdot \Area(\partial\mathcal{T})^{3/2}
\]
for some universal constant $C$.
\end{conjecture}

In Schwarzschild with $r_+ = 2M$:
\begin{align}
    V_{\mathcal{T}} &= \int_0^{2M} 4\pi r^2 \sqrt{\frac{1}{1-2M/r}} \, dr \\
    &\approx \frac{4\pi}{3}(2M)^3 \cdot (\text{factor}) \\
    &\sim M^3
\end{align}

And $\Area(\partial\mathcal{T}) = 16\pi M^2$, so $\Area^{3/2} \sim M^3$. Consistent!

\subsection{Connecting Volume to Mass}

\begin{observation}
If we could prove:
\begin{enumerate}
    \item $V_{\mathcal{T}} \leq C_1 \cdot M_{\ADM}^3$
    \item $\Area(\Sigma_0) \leq C_2 \cdot V_{\mathcal{T}}^{2/3}$
\end{enumerate}
then combining gives $\Area(\Sigma_0) \leq C \cdot M_{\ADM}^2$, which is Penrose!
\end{observation}

\section{The Intrinsic Geometry of $\mathcal{T}$}

\subsection{Curvature in the Trapped Region}

Inside $\mathcal{T}$, what can we say about curvature?

\begin{lemma}
At any point $p \in \mathcal{T}$, there exists a trapped surface through $p$.
This implies constraints on the local geometry.
\end{lemma}

\subsection{The Trapped Scalar Curvature}

Define the \textbf{trapped scalar curvature}:
\[
    R_{\mathcal{T}} = R_g - |\theta^+||\theta^-| \cdot (\text{localization})
\]

This combines the intrinsic curvature with the trapped condition.

\section{The Boundary $\partial\mathcal{T}$: The MOTS}

\subsection{Structure of the Boundary}

The boundary $\partial\mathcal{T}$ consists of MOTS (marginally outer trapped surfaces).

\begin{theorem}[Andersson-Metzger]
The outermost MOTS $\Sigma^* = \partial_{\text{out}}\mathcal{T}$ is:
\begin{enumerate}
    \item Smooth (generically)
    \item Stable
    \item Unique (outermost)
\end{enumerate}
\end{theorem}

\subsection{The Inner Boundary}

What about the \textbf{inner boundary} of $\mathcal{T}$?

\begin{definition}
The \textbf{innermost trapped boundary} is:
\[
    \partial_{\text{in}}\mathcal{T} = \{p \in \partial\mathcal{T} : p \text{ is not on outermost MOTS}\}
\]
\end{definition}

In many cases, $\partial_{\text{in}}\mathcal{T} = \emptyset$ (the trapped region extends to a singularity).

\section{A New Functional: The Trapped Mass}

\subsection{Definition}

\begin{definition}
The \textbf{trapped mass} of a trapped surface $\Sigma$ is:
\[
    m_{\text{trap}}(\Sigma) = \sqrt{\frac{\Area(\Sigma)}{16\pi}} \cdot 
    \exp\left(\frac{1}{8\pi}\int_\Sigma \frac{\theta^+\theta^-}{|\theta^+| + |\theta^-|} \, dA\right)
\]
\end{definition}

\subsection{Properties}

For trapped surfaces:
\begin{itemize}
    \item $\theta^+, \theta^- < 0$, so $\theta^+\theta^- > 0$
    \item The exponent is positive
    \item $m_{\text{trap}}(\Sigma) > \sqrt{\Area/(16\pi)}$
\end{itemize}

For MOTS ($\theta^+ = 0$):
\[
    m_{\text{trap}}(\Sigma^*) = \sqrt{\frac{\Area(\Sigma^*)}{16\pi}} \cdot e^0 = \sqrt{\frac{\Area(\Sigma^*)}{16\pi}}
\]

\begin{conjecture}[Trapped Mass Monotonicity]
$m_{\text{trap}}$ is non-decreasing from trapped surfaces to MOTS:
\[
    m_{\text{trap}}(\Sigma_0) \leq m_{\text{trap}}(\Sigma^*) = \sqrt{\frac{\Area(\Sigma^*)}{16\pi}} \leq M_{\ADM}
\]
\end{conjecture}

If true, this would give:
\[
    \sqrt{\frac{\Area(\Sigma_0)}{16\pi}} < m_{\text{trap}}(\Sigma_0) \leq M_{\ADM}
\]

\textbf{This is STRONGER than Penrose!}

Wait... this gives the wrong direction. Let me reconsider.

\subsection{Corrected Definition}

\begin{definition}[Corrected Trapped Mass]
\[
    \tilde{m}_{\text{trap}}(\Sigma) = \sqrt{\frac{\Area(\Sigma)}{16\pi}} \cdot 
    \left(1 + \frac{1}{16\pi}\int_\Sigma \theta^+\theta^- \, dA / \Area(\Sigma)\right)^{-1/2}
\]
\end{definition}

For trapped surfaces: $\theta^+\theta^- > 0$, so the factor is $< 1$.

Therefore: $\tilde{m}_{\text{trap}}(\Sigma) < \sqrt{\Area/(16\pi)}$.

This goes the wrong way again!

\section{The Key Realization}

\begin{tcolorbox}[colback=yellow!10, colframe=orange!75!black]
\textbf{Observation:} Any local modification of $\sqrt{\Area/(16\pi)}$ using $\theta^+, \theta^-$
gives a quantity that is either:
\begin{itemize}
    \item $> \sqrt{\Area/(16\pi)}$ (wrong direction for Penrose from inequality standpoint)
    \item $< \sqrt{\Area/(16\pi)}$ (correct direction but doesn't help prove Penrose)
\end{itemize}

The Penrose inequality is NOT about local corrections---it's about a GLOBAL relationship
between trapped surfaces and ADM mass.
\end{tcolorbox}

\section{New Approach: The Capacity of $\mathcal{T}$}

\subsection{Definition}

\begin{definition}
The \textbf{capacity of the trapped region} is:
\[
    \text{Cap}(\mathcal{T}) = \inf\left\{\int_{M \setminus \mathcal{T}} |\nabla u|^2 \, dV : 
    u|_{\partial\mathcal{T}} = 1, u \to 0 \text{ at } \infty\right\}
\]
\end{definition}

\subsection{Relation to Mass}

\begin{theorem}[Capacity-Mass Inequality]
For asymptotically flat manifolds with $R \geq 0$:
\[
    M_{\ADM} \geq \frac{\text{Cap}(\mathcal{T})}{4\pi}
\]
\end{theorem}

\subsection{Relation to Area}

\begin{lemma}[Capacity-Area for Convex Sets]
For convex sets in $\mathbb{R}^3$:
\[
    \text{Cap}(K) \geq \sqrt{\frac{\Area(\partial K)}{4\pi}}
\]
\end{lemma}

\textbf{The problem:} $\mathcal{T}$ is not convex, and we're not in flat space!

\section{The Conformal Volume}

\subsection{Definition}

\begin{definition}
The \textbf{conformal volume} of a region $\Omega$:
\[
    V_{\text{conf}}(\Omega) = \sup_{\phi} \int_\Omega e^{3\phi} \, dV
\]
where $\phi$ ranges over conformal factors with $\phi = 0$ on $\partial\Omega$.
\end{definition}

\subsection{Why This Might Help}

Conformal changes can improve scalar curvature properties.

If we conformally map $\mathcal{T}$ to a standard region, area and volume transform as:
\begin{align}
    \Area &\to e^{2\phi} \Area \\
    \Vol &\to e^{3\phi} \Vol
\end{align}

\section{Exploring the $\theta^+/\theta^-$ Ratio}

\subsection{The Expansion Ratio}

Define:
\[
    \rho(p) = \frac{\theta^+(p)}{\theta^-(p)}
\]

For trapped surfaces: both negative, so $\rho > 0$.

\subsection{Level Sets of $\rho$}

Consider the level sets $\{\rho = c\}$ within the trapped region.

\begin{itemize}
    \item $\rho = 1$: $\theta^+ = \theta^-$, i.e., $\tr_\Sigma k = 0$ (time-symmetric)
    \item $\rho \to 0$: approaching outer MOTS ($\theta^+ \to 0$)
    \item $\rho \to \infty$: approaching inner MOTS ($\theta^- \to 0$, if exists)
\end{itemize}

\subsection{Foliation by $\rho$-Surfaces}

Can we foliate $\mathcal{T}$ by surfaces of constant $\rho$?

This would give a canonical way to connect $\Sigma_0$ to $\Sigma^*$.

\textbf{Problem:} $\rho$ is defined on surfaces, not points. Need a way to extend.

\section{The Weyl Tensor Approach}

\subsection{Motivation}

The Weyl tensor $W_{ijkl}$ encodes gravitational radiation.

In vacuum ($R_{ij} = 0$), all curvature is in the Weyl tensor.

\subsection{Weyl Invariants}

Define:
\[
    \mathcal{W} = W_{ijkl}W^{ijkl} \quad \text{(Kretschmann-like)}
\]

\subsection{Integrated Weyl}

\begin{definition}
The \textbf{Weyl content} of $\mathcal{T}$:
\[
    \mathcal{W}(\mathcal{T}) = \int_{\mathcal{T}} \sqrt{\mathcal{W}} \, dV
\]
\end{definition}

\begin{question}
Is there a bound: $\mathcal{W}(\mathcal{T}) \leq C \cdot M_{\ADM}$?
\end{question}

In Schwarzschild: $\mathcal{W} = 48M^2/r^6$.
\[
    \mathcal{W}(\mathcal{T}) = \int_0^{2M} 4\pi r^2 \cdot \frac{\sqrt{48}M}{r^3} \cdot (\text{factor}) \, dr
    \sim M \int_0^{2M} \frac{dr}{r} \to \infty
\]

The integral diverges at $r = 0$! So this doesn't work directly.

\section{A Promising Direction: The Null Energy Integral}

\subsection{Definition}

Along null geodesics through $\mathcal{T}$:
\[
    E_{\text{null}} = \int_\gamma R_{\mu\nu} \ell^\mu \ell^\nu \, d\lambda
\]
where $\gamma$ is a null geodesic and $\ell$ is its tangent.

\subsection{DEC Constraint}

Under DEC: $R_{\mu\nu}\ell^\mu\ell^\nu \geq 0$ for null $\ell$.

So $E_{\text{null}} \geq 0$ for each null geodesic.

\subsection{Total Null Energy}

\begin{definition}
The \textbf{total null energy through} $\Sigma$ is:
\[
    \mathcal{E}(\Sigma) = \int_\Sigma \left(\int_{\gamma_p} R_{\mu\nu}\ell^\mu\ell^\nu \, d\lambda\right) dA_p
\]
where $\gamma_p$ is the null geodesic starting at $p \in \Sigma$ in direction $\ell$.
\end{definition}

This integrates the matter/curvature ``seen'' by outgoing light rays from $\Sigma$.

\begin{conjecture}
\[
    \mathcal{E}(\Sigma_0) \leq C \cdot M_{\ADM}
\]
\end{conjecture}

\begin{conjecture}
\[
    \Area(\Sigma_0) \leq C' \cdot \mathcal{E}(\Sigma_0)^2
\]
\end{conjecture}

If both hold: $\Area(\Sigma_0) \leq C'' \cdot M_{\ADM}^2$, which is Penrose!

\section{Summary of New Directions}

\begin{tcolorbox}[colback=blue!10, colframe=blue!75!black]
\textbf{Potentially promising new approaches:}

\begin{enumerate}
    \item \textbf{Trapped region volume:} Relate $\Vol(\mathcal{T})$ to $M_{\ADM}^3$
    
    \item \textbf{Capacity of $\mathcal{T}$:} Use potential theory on the trapped region
    
    \item \textbf{Null energy integrals:} Integrate $R_{\mu\nu}\ell^\mu\ell^\nu$ along null geodesics
    
    \item \textbf{Conformal methods:} Map $\mathcal{T}$ to standard region
    
    \item \textbf{$\rho$-foliation:} Use $\theta^+/\theta^-$ to foliate trapped region
\end{enumerate}

\textbf{Each requires significant development to become a full approach.}
\end{tcolorbox}

\end{document}
