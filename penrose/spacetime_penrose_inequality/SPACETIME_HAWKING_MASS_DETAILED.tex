%% SPACETIME_HAWKING_MASS_DETAILED.tex
%%
%% DETAILED DEVELOPMENT: The Spacetime Hawking Mass Approach
%%
%% Goal: Prove m_H^{ST}(Σ) ≤ M_ADM, which implies Penrose 1973
%%
%% December 2025

\documentclass[11pt]{amsart}
\usepackage{amsmath,amssymb,amsthm}
\usepackage{tcolorbox}

\tcbuselibrary{theorems}

\newtcolorbox{maintheorem}{
    colback=green!5!white,
    colframe=green!50!black,
    title={\textbf{MAIN THEOREM}}
}

\newtcolorbox{keylemma}{
    colback=blue!5!white,
    colframe=blue!75!black,
    title={\textbf{KEY LEMMA}}
}

\newtcolorbox{calculation}{
    colback=gray!5!white,
    colframe=gray!50!black,
    title={\textbf{CALCULATION}}
}

\newtcolorbox{insight}{
    colback=purple!5!white,
    colframe=purple!75!black,
    title={\textbf{INSIGHT}}
}

\newtheorem{theorem}{Theorem}
\newtheorem{lemma}[theorem]{Lemma}
\newtheorem{proposition}[theorem]{Proposition}
\newtheorem{corollary}[theorem]{Corollary}
\theoremstyle{definition}
\newtheorem{definition}[theorem]{Definition}
\newtheorem{remark}[theorem]{Remark}

\newcommand{\Area}{\mathrm{Area}}
\newcommand{\Vol}{\mathrm{Vol}}
\newcommand{\divv}{\mathrm{div}}
\DeclareMathOperator{\tr}{tr}

\title{The Spacetime Hawking Mass:\\
Detailed Development and Proof Strategy}
\author{December 2025}

\begin{document}
\maketitle

\begin{abstract}
We develop in detail the spacetime Hawking mass approach to the Penrose 
1973 conjecture. The key is proving $m_H^{ST}(\Sigma) \le M_{\text{ADM}}$ 
for surfaces in initial data satisfying DEC. Combined with the automatic 
lower bound $m_H^{ST} > \sqrt{A/(16\pi)}$ for trapped surfaces, this 
yields Penrose without Area Dominance.
\end{abstract}

%% ============================================================================
\section{Setup and Definitions}
%% ============================================================================

Let $(\mathcal{C}, g, k)$ be asymptotically flat initial data for the 
Einstein equations, satisfying the Dominant Energy Condition (DEC).

\begin{definition}[Null Expansions]
For a closed surface $\Sigma \subset \mathcal{C}$, let $\nu$ be the 
outward unit normal. Define null normals:
\begin{align}
    \ell &= \frac{1}{\sqrt{2}}(\nu + n_t) \quad \text{(outgoing)}\\
    n &= \frac{1}{\sqrt{2}}(-\nu + n_t) \quad \text{(ingoing)}
\end{align}
where $n_t$ is the future unit timelike normal to $\mathcal{C}$.

The null expansions are:
\begin{align}
    \theta^+ &= H + P = H + \tr_\Sigma k\\
    \theta^- &= H - P = H - \tr_\Sigma k
\end{align}
where $H = \divv_\Sigma \nu$ is the mean curvature and $P = k_{ij}\gamma^{ij}$ 
is the trace of $k$ on $\Sigma$.
\end{definition}

\begin{definition}[Spacetime Hawking Mass]
\begin{equation}
    m_H^{ST}(\Sigma) = \sqrt{\frac{\Area(\Sigma)}{16\pi}}
    \left(1 + \frac{1}{8\pi}\int_\Sigma \theta^+\theta^- dA\right)
\end{equation}
\end{definition}

%% ============================================================================
\section{The Fundamental Property}
%% ============================================================================

\begin{keylemma}
\textbf{Lower Bound for Trapped Surfaces}

If $\Sigma$ is trapped ($\theta^+ < 0$, $\theta^- < 0$), then:
\begin{equation}
    m_H^{ST}(\Sigma) > \sqrt{\frac{\Area(\Sigma)}{16\pi}}
\end{equation}
\end{keylemma}

\begin{proof}
For trapped surfaces, $\theta^+\theta^- = (\text{negative})(\text{negative}) > 0$.

Therefore:
\begin{equation}
    1 + \frac{1}{8\pi}\int_\Sigma \theta^+\theta^- dA > 1
\end{equation}

Multiplying by $\sqrt{\Area/(16\pi)} > 0$ gives the result.
\end{proof}

\begin{insight}
This lower bound is AUTOMATIC - no proof needed beyond the definition!

The entire Penrose conjecture reduces to proving the UPPER bound:
\begin{equation}
    m_H^{ST}(\Sigma) \le M_{\text{ADM}}
\end{equation}
\end{insight}

%% ============================================================================
\section{Alternative Form of $m_H^{ST}$}
%% ============================================================================

\begin{proposition}[H-P Form]
\begin{equation}
    m_H^{ST}(\Sigma) = \sqrt{\frac{\Area}{16\pi}}
    \left(1 + \frac{1}{8\pi}\int_\Sigma (H^2 - P^2) dA\right)
\end{equation}
\end{proposition}

\begin{proof}
$\theta^+\theta^- = (H+P)(H-P) = H^2 - P^2$.
\end{proof}

\begin{proposition}[Comparison with Classical Hawking Mass]
Classical: $m_H(\Sigma) = \sqrt{\frac{A}{16\pi}}\left(1 - \frac{1}{16\pi}\int H^2 dA\right)$

Spacetime: $m_H^{ST}(\Sigma) = \sqrt{\frac{A}{16\pi}}\left(1 + \frac{1}{8\pi}\int (H^2 - P^2) dA\right)$

Difference:
\begin{equation}
    m_H^{ST} - m_H = \sqrt{\frac{A}{16\pi}} \cdot \frac{1}{16\pi}
    \int_\Sigma (3H^2 - 2P^2) dA
\end{equation}
\end{proposition}

%% ============================================================================
\section{Asymptotic Behavior}
%% ============================================================================

\begin{proposition}[Asymptotics at Infinity]
For coordinate spheres $S_r$ at large $r$ in asymptotically flat data:
\begin{equation}
    \lim_{r \to \infty} m_H^{ST}(S_r) = M_{\text{ADM}}
\end{equation}
\end{proposition}

\begin{proof}
In asymptotically flat coordinates:
\begin{align}
    g_{ij} &= \delta_{ij} + \frac{2M}{r}\delta_{ij} + O(r^{-2})\\
    k_{ij} &= O(r^{-2})
\end{align}

For $S_r$:
\begin{align}
    \Area(S_r) &= 4\pi r^2 + O(r)\\
    H &= \frac{2}{r} - \frac{2M}{r^2} + O(r^{-3})\\
    P &= O(r^{-2})
\end{align}

Therefore:
\begin{align}
    \int_{S_r} H^2 dA &= \int_{S_r} \frac{4}{r^2} dA + O(r^{-1}) = 16\pi + O(r^{-1})\\
    \int_{S_r} P^2 dA &= O(r^{-2})
\end{align}

And:
\begin{align}
    m_H^{ST}(S_r) &= \sqrt{\frac{4\pi r^2}{16\pi}}\left(1 + \frac{1}{8\pi}
    \cdot 16\pi + O(r^{-1})\right)\\
    &= \frac{r}{2}(1 + 2 + O(r^{-1}))\\
    &= \frac{3r}{2} + O(1)
\end{align}

Hmm, this doesn't converge to $M$. Let me recalculate more carefully.
\end{proof}

\begin{calculation}
\textbf{Careful Asymptotic Analysis}

For a sphere $S_r$ of coordinate radius $r$:

\textbf{Area:}
\begin{equation}
    \Area(S_r) = 4\pi r^2\left(1 + \frac{2M}{r}\right)^2 + O(r) = 4\pi r^2 + 16\pi Mr + O(1)
\end{equation}

\textbf{Mean curvature:} In the physical metric,
\begin{equation}
    H = \frac{2}{r}\left(1 - \frac{M}{r}\right) + O(r^{-3}) = \frac{2}{r} - \frac{2M}{r^2} + O(r^{-3})
\end{equation}

\textbf{Expansion:}
\begin{equation}
    P = \tr_{S_r} k = O(r^{-2})
\end{equation}

\textbf{Integral:}
\begin{align}
    \int_{S_r} H^2 dA &= \int_{S_r} \left(\frac{2}{r} - \frac{2M}{r^2}\right)^2 dA\\
    &= \frac{4}{r^2}\Area(S_r) - \frac{8M}{r^3}\Area(S_r) + O(r^{-2})\\
    &= \frac{4}{r^2} \cdot 4\pi r^2 - \frac{8M}{r^3} \cdot 4\pi r^2 + O(r^{-1})\\
    &= 16\pi - \frac{32\pi M}{r} + O(r^{-1})
\end{align}

\textbf{Spacetime Hawking mass:}
\begin{align}
    m_H^{ST}(S_r) &= \sqrt{\frac{\Area}{16\pi}}\left(1 + \frac{1}{8\pi}\int H^2 dA + O(r^{-2})\right)\\
    &= \sqrt{\frac{4\pi r^2 + 16\pi Mr}{16\pi}}\left(1 + \frac{16\pi - 32\pi M/r}{8\pi} + O(r^{-1})\right)\\
    &= \sqrt{\frac{r^2 + 4Mr}{4}}\left(1 + 2 - \frac{4M}{r} + O(r^{-1})\right)\\
    &= \frac{r}{2}\sqrt{1 + \frac{4M}{r}}\left(3 - \frac{4M}{r} + O(r^{-1})\right)
\end{align}

Expanding $\sqrt{1 + 4M/r} \approx 1 + 2M/r$:
\begin{align}
    m_H^{ST}(S_r) &\approx \frac{r}{2}\left(1 + \frac{2M}{r}\right)\left(3 - \frac{4M}{r}\right)\\
    &= \frac{r}{2}\left(3 - \frac{4M}{r} + \frac{6M}{r} + O(r^{-2})\right)\\
    &= \frac{3r}{2} + M + O(r^{-1})
\end{align}

This DIVERGES as $r \to \infty$! The definition needs adjustment.
\end{calculation}

%% ============================================================================
\section{Corrected Definition}
%% ============================================================================

\begin{insight}
The spacetime Hawking mass as defined diverges at infinity.

We need a NORMALIZED version that has the correct asymptotics.
\end{insight}

\begin{definition}[Renormalized Spacetime Hawking Mass]
\begin{equation}
    \tilde{m}_H^{ST}(\Sigma) = \sqrt{\frac{\Area}{16\pi}}
    \left(1 - \frac{1}{16\pi}\int_\Sigma (H^2 - 2P^2) dA\right)
\end{equation}
\end{definition}

Let me check the asymptotics of this.

\begin{calculation}
For $S_r$:
\begin{align}
    \tilde{m}_H^{ST}(S_r) &= \sqrt{\frac{4\pi r^2}{16\pi}}
    \left(1 - \frac{1}{16\pi}(16\pi - 32\pi M/r + O(r^{-1}))\right)\\
    &= \frac{r}{2}\left(1 - 1 + \frac{2M}{r} + O(r^{-1})\right)\\
    &= M + O(r^{-1})
\end{align}

This converges to $M_{\text{ADM}}$! ✓
\end{calculation}

But wait, does this have the right lower bound for trapped surfaces?

\begin{proposition}[Lower Bound Check]
For trapped surface ($\theta^+ < 0$, $\theta^- < 0$):
\begin{align}
    H^2 - 2P^2 &= (H^2 - P^2) - P^2 = \theta^+\theta^- - P^2
\end{align}

For trapped: $\theta^+\theta^- > 0$, but $P^2 \ge 0$.

So $H^2 - 2P^2$ could be positive or negative!
\end{proposition}

The renormalized version doesn't automatically give the lower bound. 
We need a different approach.

%% ============================================================================
\section{The Correct Mass Functional}
%% ============================================================================

\begin{insight}
\textbf{The Geroch Mass (for null hypersurfaces)}

On a null hypersurface with cross-sections $S_\lambda$:
\begin{equation}
    m_G(S_\lambda) = \sqrt{\frac{\Area}{16\pi}}\left(1 - \frac{1}{16\pi}
    \oint_S \theta\bar{\theta} dA\right)
\end{equation}

where $\theta, \bar{\theta}$ are the expansions of the two null normals.

This is monotonic under NEC and converges to Bondi mass.
\end{insight}

For initial data, we're not on a null hypersurface, so Geroch mass 
doesn't directly apply.

\begin{definition}[The Correct Functional]
Let's try:
\begin{equation}
    m^*(\Sigma) = \sqrt{\frac{\Area}{16\pi}} - \frac{1}{16\pi}
    \sqrt{\frac{1}{16\pi\Area}}\int_\Sigma \theta^+\theta^- dA
\end{equation}

Simplifying:
\begin{equation}
    m^*(\Sigma) = \sqrt{\frac{\Area}{16\pi}}\left(1 - \frac{1}{16\pi\Area}
    \int_\Sigma \theta^+\theta^- dA\right)
\end{equation}

Hmm, this is different from before. Let me reconsider.
\end{definition}

%% ============================================================================
\section{Back to Basics: What Do We Need?}
%% ============================================================================

\begin{keylemma}
\textbf{The Requirements}

We need a functional $m(\Sigma)$ such that:
\begin{enumerate}
    \item For trapped $\Sigma$: $m(\Sigma) \ge \sqrt{\Area/(16\pi)}$
    \item For large spheres: $m(S_r) \to M_{\text{ADM}}$
    \item There exists a flow where $m$ is monotonic
\end{enumerate}

If (1), (2), (3) hold, then:
\begin{equation}
    M_{\text{ADM}} = \lim m(S_r) \ge m(\Sigma) \ge \sqrt{\frac{\Area(\Sigma)}{16\pi}}
\end{equation}

This is Penrose!
\end{keylemma}

%% ============================================================================
\section{The Liu-Yau Approach}
%% ============================================================================

\begin{definition}[Liu-Yau Quasi-Local Mass]
For a surface $\Sigma$ with positive Gaussian curvature:
\begin{equation}
    m_{LY}(\Sigma) = \frac{1}{8\pi}\int_\Sigma (H_0 - H) dA
\end{equation}

where $H_0$ is the mean curvature of the isometric embedding in $\mathbb{R}^3$.
\end{definition}

\begin{theorem}[Liu-Yau Properties]
\begin{enumerate}
    \item $m_{LY} \ge 0$
    \item $m_{LY}(S_r) \to M_{\text{ADM}}$ as $r \to \infty$
    \item $m_{LY} \le M_{\text{ADM}}$ for surfaces satisfying certain conditions
\end{enumerate}
\end{theorem}

But Liu-Yau uses mean curvature $H$, not expansions $\theta^\pm$.

For trapped surfaces, we need to incorporate $P$.

%% ============================================================================
\section{The Wang-Yau Mass}
%% ============================================================================

\begin{definition}[Wang-Yau Quasi-Local Mass]
The Wang-Yau mass is defined via an optimization:
\begin{equation}
    m_{WY}(\Sigma) = \inf_{\tau, X} E(\Sigma, \tau, X)
\end{equation}

where the infimum is over time functions $\tau$ and isometric embeddings $X$ 
into Minkowski space, and $E$ is a certain energy functional.
\end{definition}

The Wang-Yau mass incorporates both $H$ and $P$ naturally, but is 
complicated to work with directly.

%% ============================================================================
\section{New Construction: The Trapped Surface Mass}
%% ============================================================================

\begin{definition}[Trapped Surface Mass]
For a surface $\Sigma$ in initial data $(\mathcal{C}, g, k)$, define:
\begin{equation}
    m_T(\Sigma) = \sqrt{\frac{\Area}{16\pi}}\left(1 - \frac{1}{16\pi}
    \int_\Sigma H^2 dA + \frac{1}{16\pi}\int_\Sigma P^2 dA\right)
\end{equation}

This can be rewritten as:
\begin{equation}
    m_T(\Sigma) = \sqrt{\frac{\Area}{16\pi}}\left(1 - \frac{1}{16\pi}
    \int_\Sigma (H^2 - P^2) dA\right)
\end{equation}

Or using $\theta^+\theta^- = H^2 - P^2$:
\begin{equation}
    m_T(\Sigma) = \sqrt{\frac{\Area}{16\pi}}\left(1 - \frac{1}{16\pi}
    \int_\Sigma \theta^+\theta^- dA\right)
\end{equation}
\end{definition}

\begin{proposition}[Properties of $m_T$]
\textbf{(1) Lower bound for trapped surfaces:}

For trapped: $\theta^+\theta^- > 0$, so:
\begin{equation}
    1 - \frac{1}{16\pi}\int \theta^+\theta^- dA < 1
\end{equation}

This gives $m_T(\Sigma) < \sqrt{A/(16\pi)}$... WRONG direction!

\textbf{(2) Asymptotics:}
\begin{align}
    m_T(S_r) &= \sqrt{\frac{4\pi r^2}{16\pi}}\left(1 - \frac{16\pi - 32\pi M/r}{16\pi}\right)\\
    &= \frac{r}{2}\left(\frac{2M}{r}\right) = M
\end{align}

Good asymptotics! But wrong lower bound.
\end{proposition}

%% ============================================================================
\section{The Sign Issue}
%% ============================================================================

\begin{insight}
\textbf{The Fundamental Tension}

For good asymptotics: need $-\int H^2$ term (like classical Hawking mass)

For trapped lower bound: need $+\int \theta^+\theta^-$ term

These have OPPOSITE signs!

$\theta^+\theta^- = H^2 - P^2$

\begin{itemize}
    \item Classical Hawking: $1 - \frac{1}{16\pi}\int H^2$
    \item For asymptotics: need the $-H^2$ term
    \item For trapped bound: $\theta^+\theta^- > 0$, so need $+\theta^+\theta^-$ term
\end{itemize}
\end{insight}

%% ============================================================================
\section{Resolution: The Two-Part Argument}
%% ============================================================================

\begin{keylemma}
\textbf{The Key Observation}

We don't need a SINGLE functional with both properties!

We can use DIFFERENT arguments for:
\begin{enumerate}
    \item Showing mass bounds the Penrose quantity
    \item Showing trapped surfaces achieve the lower bound
\end{enumerate}
\end{keylemma}

\begin{maintheorem}
\textbf{Theorem (Two-Part Penrose Proof)}

\textbf{Part A:} For any surface $\Sigma$ in DEC-satisfying data:
\begin{equation}
    M_{\text{ADM}} \ge m_H(\Sigma) = \sqrt{\frac{\Area}{16\pi}}
    \left(1 - \frac{1}{16\pi}\int H^2 dA\right)
\end{equation}

(This is the classical Hawking mass bound, provable via IMCF.)

\textbf{Part B:} For trapped surfaces:
\begin{equation}
    m_H(\Sigma) \ge \sqrt{\frac{\Area(\Sigma)}{16\pi}} \cdot f(\Sigma)
\end{equation}

where $f(\Sigma) \ge 1$ when $\Sigma$ is trapped.

\textbf{Combining:}
\begin{equation}
    M_{\text{ADM}} \ge m_H(\Sigma) \ge \sqrt{\frac{\Area}{16\pi}} \cdot f(\Sigma)
    \ge \sqrt{\frac{\Area}{16\pi}}
\end{equation}
\end{maintheorem}

Now we need to find $f(\Sigma)$.

%% ============================================================================
\section{Finding the Correction Factor}
%% ============================================================================

\begin{proposition}[Hawking Mass for Trapped Surface]
\begin{align}
    m_H(\Sigma) &= \sqrt{\frac{A}{16\pi}}\left(1 - \frac{1}{16\pi}\int H^2 dA\right)\\
    &= \sqrt{\frac{A}{16\pi}}\left(1 - \frac{1}{16\pi}\int (\theta^+ - P)^2 dA\right)
\end{align}

For $m_H \ge \sqrt{A/(16\pi)}$, we need:
\begin{equation}
    \int (\theta^+ - P)^2 dA \le 0
\end{equation}

But $(\theta^+ - P)^2 \ge 0$, so this requires $\theta^+ = P$ everywhere.

For trapped: $\theta^+ < 0$, so need $P < 0$ and $P = \theta^+$.

This is NOT generally true!
\end{proposition}

\begin{insight}
The classical Hawking mass does NOT give the Penrose lower bound for 
arbitrary trapped surfaces!

$m_H(\Sigma) < \sqrt{A/(16\pi)}$ is possible when $\int H^2 dA$ is large.
\end{insight}

%% ============================================================================
\section{The Real Solution: Flow to MOTS}
%% ============================================================================

\begin{keylemma}
\textbf{Flow Strategy (Revised)}

\begin{enumerate}
    \item Start with trapped surface $\Sigma_0$ of area $A_0$
    \item Flow OUTWARD (toward larger surfaces)
    \item Reach the outermost MOTS $\Sigma^*$
    \item For MOTS: $\theta^+ = 0$, so $H = -P$
    \item Hawking mass of MOTS: $m_H(\Sigma^*) = \sqrt{\frac{A^*}{16\pi}}
          \left(1 - \frac{1}{16\pi}\int P^2 dA\right)$
\end{enumerate}

Key question: Is area increasing along this flow?
\end{keylemma}

This brings us back to Area Dominance! The flow from $\Sigma$ to $\Sigma^*$ 
is exactly where Area Dominance enters.

%% ============================================================================
\section{Avoiding Area Dominance: The Mass Route}
%% ============================================================================

\begin{insight}
\textbf{The Fundamental Realization}

Every approach that tries to bound $M$ in terms of $\Area(\Sigma)$ 
MUST eventually compare $\Sigma$ to $\Sigma^*$ (or something larger).

The question is: can we do this comparison via MASS rather than AREA?
\end{insight}

\begin{maintheorem}
\textbf{Mass Comparison Theorem (Conjectured)}

For trapped $\Sigma$ inside MOTS $\Sigma^*$:
\begin{equation}
    m_H(\Sigma) \le m_H(\Sigma^*) + \text{correction}
\end{equation}

where the correction accounts for the matter between $\Sigma$ and $\Sigma^*$.
\end{maintheorem}

Let me develop this.

%% ============================================================================
\section{The Integrated Constraint Approach}
%% ============================================================================

\begin{proposition}[Mass in a Region]
For a region $\Omega$ with boundary $\partial\Omega = \Sigma^* \cup (-\Sigma)$:
\begin{equation}
    \int_\Omega (R_g - |k|^2 + (\tr k)^2) dV = 16\pi \int_\Omega \mu \, dV
\end{equation}

This integral relates to the "mass content" of $\Omega$.
\end{proposition}

\begin{proposition}[Boundary Integrals]
By the divergence theorem and constraint equations:
\begin{equation}
    \int_{\Sigma^*} (\cdots) - \int_\Sigma (\cdots) = 16\pi \int_\Omega \mu \, dV
\end{equation}

where $(\cdots)$ involves $H$, $P$, and their derivatives.
\end{proposition}

\begin{keylemma}
\textbf{The Positive Mass Contribution}

Under DEC, $\mu \ge 0$, so:
\begin{equation}
    16\pi \int_\Omega \mu \, dV \ge 0
\end{equation}

This positive contribution should HELP, not hurt, the mass bound.
\end{keylemma}

%% ============================================================================
\section{Detailed Calculation}
%% ============================================================================

\begin{calculation}
\textbf{Relating Boundary Hawking Masses}

The Hamiltonian constraint:
\begin{equation}
    R_g = |k|^2 - (\tr k)^2 + 16\pi\mu
\end{equation}

Integrate over $\Omega$:
\begin{equation}
    \int_\Omega R_g \, dV = \int_\Omega (|k|^2 - (\tr k)^2) dV + 16\pi\int_\Omega \mu \, dV
\end{equation}

Use Gauss-Bonnet for the LHS:
\begin{equation}
    \int_\Omega R_g \, dV = 2\int_{\partial\Omega} H \, dA + \text{topology term}
\end{equation}

Actually, this is getting complicated. Let me try a different approach.
\end{calculation}

%% ============================================================================
\section{The Direct Inequality Approach}
%% ============================================================================

\begin{maintheorem}
\textbf{Direct Bound (To Prove)}

For any surface $\Sigma$ in asymptotically flat DEC-satisfying data:
\begin{equation}
    \sqrt{\frac{\Area(\Sigma)}{16\pi}}\left(1 - \frac{1}{16\pi}\int_\Sigma 
    (\theta^+)^2 dA\right) \le M_{\text{ADM}}
\end{equation}
\end{maintheorem}

\begin{proof}[Proof Attempt]
For trapped surfaces, $\theta^+ < 0$, so $(\theta^+)^2 > 0$ and:
\begin{equation}
    1 - \frac{1}{16\pi}\int (\theta^+)^2 dA < 1
\end{equation}

This gives:
\begin{equation}
    \sqrt{\frac{A}{16\pi}}\left(1 - \frac{1}{16\pi}\int (\theta^+)^2 dA\right)
    < \sqrt{\frac{A}{16\pi}}
\end{equation}

So even if the LHS $\le M$, we don't get $M \ge \sqrt{A/(16\pi)}$.

Wrong approach again!
\end{proof}

%% ============================================================================
\section{Conclusion: The State of the Approach}
%% ============================================================================

After detailed analysis:

\textbf{The core difficulty:}
\begin{enumerate}
    \item Classical Hawking mass has good asymptotics ($\to M$)
    \item Classical Hawking mass has good monotonicity (under IMCF with $R \ge 0$)
    \item But classical Hawking mass is NOT $\ge \sqrt{A/(16\pi)}$ for trapped surfaces
\end{enumerate}

\textbf{The sign issue:}
\begin{itemize}
    \item For asymptotics: need $-H^2$ in the formula
    \item For trapped bound: $H^2 - P^2 = \theta^+\theta^- > 0$
    \item Can't have both with a single functional
\end{itemize}

\textbf{Resolution needed:}

A fundamentally new quasi-local mass that:
\begin{enumerate}
    \item Incorporates both $\theta^+$ and $\theta^-$
    \item Has correct asymptotics
    \item Has a monotonicity property under some flow
    \item Automatically satisfies lower bound for trapped surfaces
\end{enumerate}

This may require NEW MATHEMATICS beyond current quasi-local mass theory.

\end{document}
