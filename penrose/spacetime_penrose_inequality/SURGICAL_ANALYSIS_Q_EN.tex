\documentclass[11pt,a4paper]{article}
\usepackage[utf8]{inputenc}
\usepackage{amsmath,amssymb,amsthm}
\usepackage{geometry}
\usepackage{booktabs}
\usepackage{array}
\usepackage{xcolor}
\usepackage{tcolorbox}
\usepackage{tikz}

\geometry{margin=2.5cm}

\newtheorem{theorem}{Theorem}
\newtheorem{lemma}[theorem]{Lemma}
\newtheorem{proposition}[theorem]{Proposition}
\newtheorem{definition}[theorem]{Definition}
\newtheorem{remark}[theorem]{Remark}
\newtheorem{conjecture}[theorem]{Conjecture}

\newcommand{\good}[1]{\textcolor{green!60!black}{#1}}
\newcommand{\bad}[1]{\textcolor{red!70!black}{#1}}
\newcommand{\neutral}[1]{\textcolor{blue!70!black}{#1}}
\newcommand{\danger}[1]{\textcolor{orange!80!black}{#1}}

\title{\textbf{Surgical Analysis: The Completing-the-Square Mechanism\\for Boost-Invariant Quasi-Local Mass}}
\author{Spacetime Penrose Inequality Program}
\date{December 2025}

\begin{document}
\maketitle

\begin{abstract}
This document analyzes the variational formula for the boost-invariant quasi-local mass $\mathcal{Q}$ from a ``surgical'' perspective, explicitly identifying: (1) bad terms that must be eliminated by completing the square; (2) the minimal correction terms required; (3) dangerous points (MOTS/caustic/null infinity) where the construction fails. This provides a technical roadmap for proving the spacetime Penrose inequality.
\end{abstract}

\tableofcontents

\section{The Core Surgical Tool: Boost-Invariant Quasi-Local Mass}

\subsection{The Problem with Hawking Mass}

The Hawking mass is defined as:
\begin{equation}
m_H(\Sigma) = \sqrt{\frac{|\Sigma|}{16\pi}}\left(1 - \frac{1}{16\pi}\int_\Sigma \theta^+\theta^- \, dA\right)
\end{equation}

\begin{tcolorbox}[colback=red!5!white, colframe=red!75!black, title=Fatal Defects of Hawking Mass]
Variation along the null direction:
\begin{equation}
\frac{dm_H}{ds} = \frac{\sqrt{|\Sigma|/16\pi}}{16\pi} \int_\Sigma \left[\mu - |J| \cdot (\text{something}) - \bad{2\sigma^+:\sigma^-} - \bad{|\zeta|^2}\right] dA
\end{equation}

\textbf{Bad term analysis}:
\begin{itemize}
\item $\bad{-2\sigma^+:\sigma^-}$: Indefinite sign! $\sigma^+:\sigma^- = \text{tr}(\sigma^+_{ab}\sigma^{-ab})$ can be positive or negative
\item $\bad{-|\zeta|^2}$: Negative definite, directly destroys monotonicity
\end{itemize}

\textbf{Conclusion}: Hawking mass is \textbf{not monotonic} along null directions; cannot be directly used for Penrose inequality.
\end{tcolorbox}

\subsection{Surgical Tool \#1: Completing-the-Square Identity}

\begin{tcolorbox}[colback=green!5!white, colframe=green!60!black, title=Core Algebraic Identity]
For any two symmetric trace-free tensors $\sigma^+, \sigma^-$:
\begin{equation}
\boxed{-\sigma^+:\sigma^- = -\frac{1}{4}|\sigma^+ + \sigma^-|^2 + \frac{1}{4}|\sigma^+ - \sigma^-|^2}
\end{equation}

\textbf{Effect}:
\begin{itemize}
\item LHS: Bilinear term with indefinite sign
\item RHS: Difference of two squared terms, each with definite sign
\end{itemize}
\end{tcolorbox}

\subsection{Surgical Tool \#2: Boost Invariance Requirement}

Under the null frame $(\ell, n)$, the boost transformation is:
\begin{equation}
\ell \mapsto \lambda \ell, \quad n \mapsto \lambda^{-1} n
\end{equation}

Transformation properties of various quantities:
\begin{center}
\begin{tabular}{c|c|c}
\toprule
\textbf{Quantity} & \textbf{Transformation} & \textbf{Boost weight} \\
\midrule
$\theta^+$ & $\lambda \theta^+$ & $+1$ \\
$\theta^-$ & $\lambda^{-1} \theta^-$ & $-1$ \\
$\sigma^+$ & $\lambda \sigma^+$ & $+1$ \\
$\sigma^-$ & $\lambda^{-1} \sigma^-$ & $-1$ \\
$\theta^+\theta^-$ & invariant & $0$ \\
$\sigma^+:\sigma^-$ & invariant & $0$ \\
$\sigma^+/\theta^+$ & invariant & $0$ \\
\bottomrule
\end{tabular}
\end{center}

\begin{tcolorbox}[colback=blue!5!white, colframe=blue!60!black, title=The Price of Boost Invariance]
To construct boost-invariant combinations, we must normalize by $\theta^\pm$:
\begin{equation}
\frac{\sigma^+}{\theta^+} - \frac{\sigma^-}{\theta^-} \quad \text{(boost invariant)}
\end{equation}

\textbf{Price}: When $\theta^+ \to 0$ (MOTS) or $\theta^- \to 0$, this combination \textbf{diverges}!
\end{tcolorbox}

\section{The Corrected Mass Functional $\mathcal{Q}$}

\subsection{Definition}

\begin{definition}[Boost-Invariant Quasi-Local Mass]
\begin{equation}
\mathcal{Q}(\Sigma) = \sqrt{\frac{|\Sigma|}{16\pi}}\left(1 - \frac{1}{16\pi}\int_\Sigma \left[\theta^+\theta^- + |\zeta|^2 + \frac{1}{4}\left|\frac{\sigma^+}{\theta^+} - \frac{\sigma^-}{\theta^-}\right|^2\theta^+\theta^-\right] dA\right)
\end{equation}
\end{definition}

\subsection{Variational Formula (Along Outgoing Null Direction)}

\begin{theorem}[$\mathcal{Q}$ Monotonicity]
Along an outgoing null direction in a spacetime satisfying DEC:
\begin{equation}
\frac{d\mathcal{Q}}{ds} = \frac{\sqrt{|\Sigma|/16\pi}}{16\pi} \int_\Sigma \Phi \, dA
\end{equation}
where the integrand $\Phi$ decomposes as:
\begin{align}
\Phi &= \underbrace{\good{(\mu - |J|) \cdot (\text{positive coefficient})}}_{\text{DEC term: } \geq 0} \\
&\quad + \underbrace{\good{\frac{1}{4}|\sigma^+ - \sigma^-|^2 \cdot (\text{positive coefficient})}}_{\text{Good squared term: } \geq 0} \\
&\quad + \underbrace{\neutral{(\text{boundary/asymptotic terms})}}_{\text{Requires caustic surgery}}
\end{align}
\end{theorem}

\section{Bad Terms -- Completing Square -- Danger Points: Reference Table}

\begin{center}
\renewcommand{\arraystretch}{1.8}
\begin{tabular}{|>{\raggedright}p{3cm}|>{\raggedright}p{4cm}|>{\raggedright}p{4cm}|>{\raggedright\arraybackslash}p{3.5cm}|}
\hline
\textbf{Bad Term} & \textbf{Completing Square} & \textbf{Minimal Correction} & \textbf{Danger Point} \\
\hline
\hline
$\bad{-2\sigma^+:\sigma^-}$ \newline (indefinite shear coupling) & 
$-\sigma^+:\sigma^- = -\frac{1}{4}|\sigma^++\sigma^-|^2 + \frac{1}{4}|\sigma^+-\sigma^-|^2$ &
Add $+\frac{1}{4}|\sigma^++\sigma^-|^2$ to cancel negative square &
\danger{MOTS}: needs $\theta^+$ normalization \\
\hline
$\bad{-|\zeta|^2}$ \newline (negative twist) &
Absorb into correction &
$+|\zeta|^2$ cancels &
\danger{Caustic}: $\zeta$ may diverge \\
\hline
$\bad{\sigma^+/\theta^+}$ divergence \newline (MOTS singularity) &
Cannot be squared away! \newline \textbf{Essential singularity} &
Must use \textbf{jump}: jump to outer hull before $|\theta^+| < \delta$ &
\danger{MOTS}: $\theta^+ = 0$ \\
\hline
$\bad{\theta^+ \to -\infty}$ \newline (Caustic divergence) &
Cannot be squared away! \newline \textbf{Geometric singularity} &
Must use \textbf{jump}: Huisken-Ilmanen style outer hull surgery &
\danger{Caustic}: conjugate points \\
\hline
$\neutral{\text{Boundary terms}}$ \newline (asymptotic behavior) &
$\mathcal{Q} \to M_B$ needs asymptotic analysis &
Bondi coordinate expansion + decay estimates &
\danger{$\mathscr{I}^+$}: null infinity \\
\hline
\end{tabular}
\end{center}

\section{Unified Treatment: MOTS-Avoiding Weak Null Flow}

\subsection{Core Insight}

\begin{tcolorbox}[colback=yellow!5!white, colframe=yellow!75!black, title=Unification of Gap 1 and Gap 2]
Originally thought to be two separate problems:
\begin{itemize}
\item \textbf{Gap 1}: Caustic ($\theta^+ \to -\infty$)
\item \textbf{Gap 2}: MOTS crossing ($\theta^+ \to 0$)
\end{itemize}

\textbf{Unified insight}: Both are consequences of $\theta^+$ appearing in denominators of $\mathcal{Q}$. The solution is the same:
\begin{center}
\fbox{\textbf{Jump to outer hull BEFORE reaching the singularity}}
\end{center}
\end{tcolorbox}

\subsection{Definition of MOTS-Avoiding Weak Null Flow}

\begin{definition}[MOTS-Avoiding Weak Null Flow]
A \textbf{MOTS-avoiding weak null flow} $\{\Sigma_s\}_{s \geq 0}$ from trapped surface $\Sigma$ satisfies:

\begin{enumerate}
\item[(WA1)] \textbf{Initial}: $\Sigma_0 = \Sigma$ with $\theta^+(\Sigma_0) < 0$ (trapped);

\item[(WA2)] \textbf{Smooth segments}: Between jumps, $\Sigma_s$ evolves smoothly along outgoing null with $|\theta^\pm| \geq \delta > 0$;

\item[(WA3)] \textbf{Caustic jump}: When $\theta^+ \to -\infty$, jump to outward minimizing hull;

\item[(WA4)] \textbf{MOTS-approach jump}: When $|\theta^+| < \delta$, jump to outward minimizing hull;

\item[(WA5)] \textbf{Endpoint}: $\Sigma_s \to \mathscr{I}^+$ (null infinity).
\end{enumerate}
\end{definition}

\subsection{Key Lemma: Monotonicity at Jumps}

\begin{tcolorbox}[colback=red!5!white, colframe=red!75!black, title=Core Open Problem]
\begin{conjecture}[Jump Monotonicity]
Let $\Sigma^-$ be the surface before jump, $\Sigma^+$ be the outer hull after jump. Then:
\begin{equation}
\mathcal{Q}(\Sigma^+) \geq \mathcal{Q}(\Sigma^-)
\end{equation}
\end{conjecture}

\textbf{Difficulties}:
\begin{itemize}
\item Outer hull definition requires Lorentzian geometric measure theory
\item $\mathcal{Q}$ may diverge at $\Sigma^-$ (near MOTS or caustic)
\item Need to prove ``telescoping error absorption''
\end{itemize}
\end{tcolorbox}

\section{Complete Expansion of Variational Formula}

\subsection{Raychaudhuri Equations}

Along outgoing null direction $\ell$:
\begin{align}
\frac{d\theta^+}{ds} &= -\frac{1}{2}(\theta^+)^2 - |\sigma^+|^2 - R_{\mu\nu}\ell^\mu\ell^\nu \\
&= -\frac{1}{2}(\theta^+)^2 - |\sigma^+|^2 - 8\pi(\mu - J \cdot \ell)
\end{align}

Along ingoing null direction $n$:
\begin{equation}
\frac{d\theta^-}{ds} = -\frac{1}{2}(\theta^-)^2 - |\sigma^-|^2 - 8\pi(\mu - J \cdot n)
\end{equation}

\subsection{Evolution of $\theta^+\theta^-$}

\begin{align}
\frac{d(\theta^+\theta^-)}{ds} &= \theta^- \frac{d\theta^+}{ds} + \theta^+ \frac{d\theta^-}{ds} \\
&= -\frac{1}{2}\theta^-(\theta^+)^2 - \theta^-|\sigma^+|^2 - 8\pi\theta^-(\mu - J\cdot\ell) \\
&\quad - \frac{1}{2}\theta^+(\theta^-)^2 - \theta^+|\sigma^-|^2 - 8\pi\theta^+(\mu - J\cdot n)
\end{align}

\subsection{Handling Shear Terms}

Original bad terms:
\begin{equation}
-\theta^-|\sigma^+|^2 - \theta^+|\sigma^-|^2 - 2\sigma^+:\sigma^-
\end{equation}

Applying completing the square:
\begin{align}
&-\theta^-|\sigma^+|^2 - \theta^+|\sigma^-|^2 - 2\sigma^+:\sigma^- \\
&= -\theta^-|\sigma^+|^2 - \theta^+|\sigma^-|^2 + \frac{1}{2}|\sigma^+ + \sigma^-|^2 - \frac{1}{2}|\sigma^+ - \sigma^-|^2
\end{align}

\begin{tcolorbox}[colback=green!5!white, colframe=green!60!black, title=Sign Analysis After Completing Square]
Define $\Delta\sigma = \sigma^+ - \sigma^-$ (boost weight $+1 - (-1) = +2$, not boost invariant).

For boost-invariant combination:
\begin{equation}
\frac{\sigma^+}{\theta^+} - \frac{\sigma^-}{\theta^-} \quad \text{(boost invariant)}
\end{equation}

Then:
\begin{equation}
\left|\frac{\sigma^+}{\theta^+} - \frac{\sigma^-}{\theta^-}\right|^2 \theta^+\theta^- = \frac{|\sigma^+\theta^- - \sigma^-\theta^+|^2}{\theta^+\theta^-}
\end{equation}

\textbf{Sign}: When $\theta^+\theta^- < 0$ (untrapped region), this term is \textbf{negative}!

\textbf{But}: In trapped region where $\theta^+\theta^- > 0$, this term is \textbf{positive}.

This is why we must start from a trapped surface!
\end{tcolorbox}

\section{Detailed Analysis of Danger Points}

\subsection{MOTS ($\theta^+ = 0$)}

As $\theta^+ \to 0^-$:
\begin{itemize}
\item $\sigma^+/\theta^+ \to \pm\infty$ (unless $\sigma^+ = 0$)
\item Correction term in $\mathcal{Q}$: $\left|\frac{\sigma^+}{\theta^+} - \frac{\sigma^-}{\theta^-}\right|^2\theta^+\theta^- \to -\infty$ (since $\theta^- < 0$)
\item This is an \textbf{essential singularity}, cannot be removed by redefining $\mathcal{Q}$
\end{itemize}

\textbf{Surgical solution}: Jump when $|\theta^+| < \delta$.

\subsection{Caustic ($\theta^+ \to -\infty$)}

When null rays focus to form caustic:
\begin{itemize}
\item $\theta^+ \to -\infty$
\item Surface degenerates (area $\to 0$)
\item Domain of $\mathcal{Q}$ fails
\end{itemize}

\textbf{Surgical solution}: Huisken-Ilmanen style outer hull jump.

\subsection{Null Infinity ($\mathscr{I}^+$)}

Asymptotic behavior:
\begin{itemize}
\item $|\Sigma_r| \sim 4\pi r^2$
\item $\theta^+ \sim 2/r$, $\theta^- \sim -1/r$
\item $\sigma^\pm \sim O(r^{-2})$ (news function decay)
\item $\zeta \sim O(r^{-2})$
\end{itemize}

\textbf{Conclusion}: $\mathcal{Q}(\Sigma_r) = M_B + O(r^{-1})$, approaches Bondi mass.

\section{Main Conditional Theorem}

\begin{tcolorbox}[colback=blue!5!white, colframe=blue!75!black, title=\textbf{Main Conditional Theorem}]
\begin{theorem}[Spacetime Penrose Inequality -- Conditional Version]
Let $(M^4, g)$ be a globally hyperbolic, asymptotically flat spacetime satisfying DEC, with Bondi mass $M_B$ and $\Sigma$ a closed outermost trapped surface with spherical topology.

\textbf{IF} there exists a MOTS-avoiding weak null flow $\{\Sigma_s\}_{s \in [0,\infty)}$ satisfying:
\begin{enumerate}
\item[(H1)] Conditions (WA1)--(WA5);
\item[(H2)] Jump monotonicity: at each jump, $\mathcal{Q}(\Sigma^+) \geq \mathcal{Q}(\Sigma^-)$;
\end{enumerate}
\textbf{THEN}:
\begin{equation}
\boxed{M_B \geq \sqrt{\frac{|\Sigma|}{16\pi}}}
\end{equation}
\end{theorem}
\end{tcolorbox}

\begin{proof}[Proof outline]
\begin{enumerate}
\item \textbf{Initial value}: $\mathcal{Q}(\Sigma_0) = \sqrt{|\Sigma|/16\pi}$ (for outermost trapped surface)

\item \textbf{Smooth segment monotonicity}: By DEC + completing square, $d\mathcal{Q}/ds \geq 0$

\item \textbf{Jump monotonicity}: By hypothesis (H2), $\mathcal{Q}$ does not decrease at jumps

\item \textbf{Asymptotic limit}: $\lim_{s\to\infty} \mathcal{Q}(\Sigma_s) = M_B$

\item \textbf{Conclusion}: $M_B \geq \mathcal{Q}(\Sigma_0) = \sqrt{|\Sigma|/16\pi}$
\end{enumerate}
\end{proof}

\section{Open Problems}

\begin{enumerate}
\item \textbf{Weak null flow existence}: Does a flow satisfying (WA1)--(WA5) always exist?

\item \textbf{Lorentzian definition of outer hull}: How to define ``outward minimizing hull'' on null hypersurfaces?

\item \textbf{Jump monotonicity}: How to prove $\mathcal{Q}(\Sigma^+) \geq \mathcal{Q}(\Sigma^-)$?

\item \textbf{Topology preservation}: Does the flow preserve spherical topology?

\item \textbf{Rigidity}: Does equality imply Schwarzschild?
\end{enumerate}

\section{Analogy with Wang Hong's ``Surgical Knife''}

\begin{center}
\renewcommand{\arraystretch}{1.5}
\begin{tabular}{|>{\raggedright}p{4cm}|>{\raggedright}p{5cm}|>{\raggedright\arraybackslash}p{5cm}|}
\hline
\textbf{Aspect} & \textbf{Kakeya Problem (Wang Hong)} & \textbf{Penrose 1973 (Our approach)} \\
\hline
\hline
Core bad term & Multi-scale/multilinear Kakeya configurations & Indefinite shear coupling $\sigma^+:\sigma^-$ \\
\hline
Surgical knife & Refined harmonic analysis (multi-scale decomposition + orthogonality) & Completing square + boost-invariant normalization \\
\hline
Cutting mechanism & Decompose bad configurations into controllable pieces & Convert indefinite terms into squared terms \\
\hline
Geometric singularities & None & Caustic, MOTS \\
\hline
Auxiliary surgery & None & Weak null flow + outer hull jump \\
\hline
Sharp constants & Close via refined estimates & DEC + completing square auto-closes \\
\hline
Open problems & Solved & Jump monotonicity, flow existence \\
\hline
\end{tabular}
\end{center}

\section{Conclusion}

\begin{tcolorbox}[colback=green!5!white, colframe=green!60!black, title=The Two Surgical Knives]
\textbf{First knife}: \textbf{Completing-the-square mechanism for boost-invariant quasi-local mass $\mathcal{Q}$}
\begin{itemize}
\item Converts $-\sigma^+:\sigma^-$ into $+\frac{1}{4}|\sigma^+-\sigma^-|^2 - \frac{1}{4}|\sigma^++\sigma^-|^2$
\item DEC provides $(\mu - |J|) \geq 0$
\item Sharp constants close automatically
\end{itemize}

\textbf{Second knife}: \textbf{Caustic/MOTS surgery for weak null flow}
\begin{itemize}
\item Unified treatment of $\theta^+ \to -\infty$ (caustic) and $\theta^+ \to 0$ (MOTS)
\item Jump to outer hull before singularity
\item Need to prove jump monotonicity (core open problem)
\end{itemize}
\end{tcolorbox}

\end{document}
