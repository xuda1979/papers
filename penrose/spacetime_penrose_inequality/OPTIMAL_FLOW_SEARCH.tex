% =========================================================================
%     OPTIMAL FLOW-FUNCTIONAL PAIRS FOR SPACETIME PENROSE
%
%     Searching for (f, F) such that dF/dt ≥ 0 without sign obstruction
%
%     Author: Da Xu
%     Date: December 2025
% =========================================================================

\documentclass[12pt]{article}
\usepackage{amsmath,amsthm,amssymb}
\usepackage{tcolorbox}
\usepackage{xcolor}

\theoremstyle{plain}
\newtheorem{theorem}{Theorem}[section]
\newtheorem{lemma}[theorem]{Lemma}
\newtheorem{proposition}[theorem]{Proposition}
\newtheorem{corollary}[theorem]{Corollary}

\theoremstyle{definition}
\newtheorem{definition}[theorem]{Definition}
\newtheorem{remark}[theorem]{Remark}

\newcommand{\tr}{\mathrm{tr}}
\newcommand{\Div}{\mathrm{div}}
\newcommand{\Ric}{\mathrm{Ric}}
\newcommand{\Area}{\mathrm{Area}}
\newcommand{\ADM}{\mathrm{ADM}}

\title{\textbf{Optimal Flow-Functional Pairs\\for the Spacetime Penrose Inequality}}
\author{Da Xu}
\date{December 2025}

\begin{document}
\maketitle

\begin{abstract}
We systematically search for a flow speed $f$ and mass functional $\mathcal{M}$
such that $d\mathcal{M}/dt \geq 0$ under DEC, independent of the sign of
$\tr_\Sigma k$. This would give an unconditional proof of the spacetime
Penrose inequality.
\end{abstract}

%===========================================================================
\section{The General Framework}
%===========================================================================

\subsection{Setup}

For a surface flow $\partial\Sigma/\partial t = f\nu$, we have:
\begin{align}
    \frac{dA}{dt} &= \int_\Sigma Hf \, dA \\
    \frac{dH}{dt} &= -\Delta_\Sigma f - (|A|^2 + \Ric(\nu,\nu))f \\
    \frac{d(\tr_\Sigma k)}{dt} &= \text{complicated spacetime terms} \times f
\end{align}

\subsection{The Geroch Miracle (Review)}

For IMCF ($f = 1/H$, with $H < 0$), the Geroch functional is:
\begin{equation}
    m_G = \sqrt{\frac{A}{16\pi}}\left(1 - \frac{1}{16\pi}\int_\Sigma H^2 \, dA\right)
\end{equation}

Under IMCF:
\begin{equation}
    \frac{dm_G}{dt} = \sqrt{\frac{A}{16\pi}} \cdot \frac{1}{8\pi} \int_\Sigma \left[\frac{|\nabla H|^2}{H^3} + \frac{\Ric(\nu,\nu) + \frac{1}{2}|A^\circ|^2}{H}\right] dA
\end{equation}

For $H < 0$, the $|\nabla H|^2/H^3$ term is $\leq 0$.
For $H < 0$ and $\Ric(\nu,\nu) \geq 0$, the second term is $\leq 0$.

So $dm_G/dt \leq 0$? No wait---the signs need careful tracking.

Actually, the correct formula (with $H < 0$ throughout) gives $dm_G/dt \geq 0$
because area increases and the $H^2$ integral decreases appropriately.

%===========================================================================
\section{Constraints on the Flow Speed}
%===========================================================================

\subsection{What We Need}

For the Penrose inequality, we need:
\begin{enumerate}
    \item Area non-decreasing: $\frac{dA}{dt} = \int Hf \geq 0$
    \item Mass functional non-decreasing: $\frac{d\mathcal{M}}{dt} \geq 0$
    \item $\mathcal{M} \to M_{\ADM}$ as surfaces go to infinity
    \item $\mathcal{M}(\text{MOTS}) = M_P(\text{MOTS}) = \sqrt{A/(16\pi)}$
\end{enumerate}

\subsection{Sign Constraints}

For trapped surfaces: $H < 0$, $\theta^+ \leq 0$, $\theta^- < 0$.

For area non-decreasing: $\int Hf \geq 0$ with $H < 0$ requires $f \leq 0$ (or weighted appropriately).

\textbf{Natural candidates for $f$:}
\begin{itemize}
    \item $f = 1/H$ (IMCF): $\int Hf = \int 1 = A > 0$ \checkmark
    \item $f = 1/\theta^+$ (null mean curvature flow): $\int Hf = \int H/\theta^+$
    \item $f = -\sqrt{|\theta^+\theta^-|}$ (product flow): $\int Hf = -\int H\sqrt{|\theta^+\theta^-|} > 0$ \checkmark
    \item $f = 1/\sqrt{|\theta^+\theta^-|}$ (inverse product flow): $\int Hf = \int H/\sqrt{|\theta^+\theta^-|}$
\end{itemize}

%===========================================================================
\section{Candidate 1: Inverse Product Flow}
%===========================================================================

\subsection{Definition}

\begin{definition}[Inverse Product Mean Curvature Flow]
\begin{equation}
    \frac{\partial\Sigma}{\partial t} = \frac{1}{\mathcal{H}_P}\nu = \frac{1}{\sqrt{|\theta^+\theta^-|}}\nu
\end{equation}
for trapped surfaces with $\theta^+\theta^- > 0$.
\end{definition}

\subsection{Area Evolution}

\begin{equation}
    \frac{dA}{dt} = \int_\Sigma \frac{H}{\sqrt{|\theta^+\theta^-|}} \, dA
\end{equation}

For trapped surfaces: $H < 0$, $\sqrt{|\theta^+\theta^-|} > 0$, so the integrand is negative.

\textbf{Area is DECREASING.} This is the wrong direction.

\subsection{Conclusion}

The inverse product flow has area decreasing---not useful for Penrose.

%===========================================================================
\section{Candidate 2: Geometric Mean Flow}
%===========================================================================

\subsection{Key Observation}

For trapped surfaces:
\begin{align}
    H &= \frac{1}{2}(\theta^+ + \theta^-) < 0 \\
    \sqrt{|\theta^+\theta^-|} &= \sqrt{|\theta^+||\theta^-|} > 0 \quad (\text{when $\theta^+ \neq 0$})
\end{align}

Note: $|H| \leq \sqrt{|\theta^+||\theta^-|}$ by AM-GM? Let's check.

$|H| = |\frac{\theta^+ + \theta^-}{2}|$

For $\theta^+ \leq 0$, $\theta^- < 0$:
$|H| = \frac{|\theta^+| + |\theta^-|}{2}$

By AM-GM: $\frac{|\theta^+| + |\theta^-|}{2} \geq \sqrt{|\theta^+||\theta^-|}$

So $|H| \geq \sqrt{|\theta^+\theta^-|}$, i.e., $-H \geq \sqrt{|\theta^+\theta^-|}$.

\subsection{The Ratio}

Define:
\begin{equation}
    \rho := \frac{\sqrt{|\theta^+\theta^-|}}{|H|} = \frac{\sqrt{|\theta^+\theta^-|}}{-H} \leq 1
\end{equation}

This ratio satisfies $0 \leq \rho \leq 1$, with:
\begin{itemize}
    \item $\rho = 1$ when $|\theta^+| = |\theta^-|$, i.e., $\tr_\Sigma k = 0$
    \item $\rho \to 0$ as $\theta^+ \to 0$ (approaching MOTS)
\end{itemize}

%===========================================================================
\section{Candidate 3: Weighted Flow}
%===========================================================================

\subsection{Idea}

Consider a flow with speed that interpolates based on the local geometry:
\begin{equation}
    f = \frac{\alpha}{H} + \frac{\beta}{\sqrt{|\theta^+\theta^-|}}
\end{equation}
where $\alpha, \beta$ are chosen to achieve monotonicity.

\subsection{Area Evolution}

\begin{equation}
    \frac{dA}{dt} = \int_\Sigma \left(\alpha + \frac{\beta H}{\sqrt{|\theta^+\theta^-|}}\right) dA
    = \alpha A + \beta \int_\Sigma \frac{H}{\sqrt{|\theta^+\theta^-|}} dA
\end{equation}

For this to be non-negative with $H < 0$:
\begin{equation}
    \alpha A \geq -\beta \int_\Sigma \frac{H}{\sqrt{|\theta^+\theta^-|}} dA = \beta \int_\Sigma \frac{|H|}{\sqrt{|\theta^+\theta^-|}} dA
\end{equation}

Since $|H|/\sqrt{|\theta^+\theta^-|} \geq 1$, we need $\alpha \geq \beta$.

%===========================================================================
\section{The Fundamental Obstruction Revisited}
%===========================================================================

\subsection{Why This Is Hard}

The Geroch monotonicity for IMCF is ``miraculous'' in the sense that:
\begin{equation}
    \frac{d}{dt}\left(\sqrt{A}(1 - \frac{1}{16\pi}\int H^2)\right) = \text{DEC-controlled}
\end{equation}

This uses:
\begin{enumerate}
    \item $f = 1/H$ cancels certain terms in $dH/dt$
    \item The term $\int H^2$ has the right structure
    \item Scalar curvature $R \geq 0$ (DEC in Riemannian) gives the sign
\end{enumerate}

For spacetime:
\begin{itemize}
    \item We have $\tr_\Sigma k$ appearing in $\theta^\pm$
    \item The DEC involves both $\mu$ and $J$
    \item No single flow speed cancels all the ``bad'' terms
\end{itemize}

\subsection{The Coupled System}

The evolution of $(H, \tr_\Sigma k)$ under a flow $f\nu$ is a coupled system:
\begin{align}
    \frac{dH}{dt} &= -\Delta f - (|A|^2 + \Ric^{(4)}(\nu,\nu) + k_{ij}k^{ij} - (\tr k)k_{\nu\nu} + \ldots)f \\
    \frac{d(\tr_\Sigma k)}{dt} &= (\text{spacetime terms})f
\end{align}

The $\tr_\Sigma k$ evolution involves $\nabla k$ (derivatives of extrinsic curvature), which don't have a simple sign under DEC.

%===========================================================================
\section{A New Idea: The Null Expansion Ratio}
%===========================================================================

\subsection{The Ratio $\theta^+/\theta^-$}

For trapped surfaces: $\theta^+ \leq 0$, $\theta^- < 0$.

Define:
\begin{equation}
    \lambda := \frac{\theta^+}{\theta^-} = \frac{H + \tr_\Sigma k}{H - \tr_\Sigma k} \in [0, \infty)
\end{equation}

Properties:
\begin{itemize}
    \item $\lambda = 0$ at MOTS ($\theta^+ = 0$)
    \item $\lambda = 1$ when $\tr_\Sigma k = 0$
    \item $\lambda \to \infty$ as $\theta^- \to 0$ (rare)
\end{itemize}

\subsection{The Ratio Flow}

Consider:
\begin{equation}
    f = \frac{1}{\theta^-} = \frac{1}{H - \tr_\Sigma k}
\end{equation}

For trapped surfaces, $\theta^- < 0$, so $f < 0$ (inward motion).

Area evolution:
\begin{equation}
    \frac{dA}{dt} = \int_\Sigma \frac{H}{\theta^-} dA = \int_\Sigma \frac{H}{H - \tr_\Sigma k} dA
\end{equation}

Using $H = \frac{1}{2}(\theta^+ + \theta^-)$:
\begin{equation}
    \frac{H}{\theta^-} = \frac{\theta^+ + \theta^-}{2\theta^-} = \frac{1}{2}\left(\frac{\theta^+}{\theta^-} + 1\right) = \frac{1}{2}(\lambda + 1)
\end{equation}

So:
\begin{equation}
    \frac{dA}{dt} = \frac{1}{2}\int_\Sigma (\lambda + 1) dA > 0
\end{equation}

\textbf{Area is increasing!} Good sign.

\subsection{A New Functional?}

Define:
\begin{equation}
    m_-(\Sigma) := \sqrt{\frac{A}{16\pi}}\left(1 - \frac{1}{16\pi}\int_\Sigma (\theta^-)^2 dA\right)
\end{equation}

Is this monotonic under the $1/\theta^-$ flow?

At a MOTS: $\theta^+ = 0$, $\theta^- = H - \tr_\Sigma k$... this doesn't vanish in general.

At infinity: $\theta^- \to 0$, so $m_- \to \sqrt{A/(16\pi)}$.

Hmm, the asymptotics don't match $M_{\ADM}$ correctly.

%===========================================================================
\section{Conclusion: The Search Continues}
%===========================================================================

\begin{tcolorbox}[colback=yellow!10, colframe=yellow!50!black, title=Status]
We have explored several flow-functional pairs:
\begin{enumerate}
    \item \textbf{Product flow + Product Geroch:} Area increases, but monotonicity not proven
    \item \textbf{Inverse product flow:} Area decreases---wrong direction
    \item \textbf{$1/\theta^-$ flow:} Area increases, but functional unclear
    \item \textbf{Weighted flows:} No clear winner
\end{enumerate}

\textbf{The fundamental challenge:} Finding $(f, \mathcal{M})$ such that:
\begin{itemize}
    \item $d\mathcal{M}/dt \geq 0$ under DEC
    \item $\mathcal{M}(\text{MOTS}) = M_P$
    \item $\mathcal{M} \to M_{\ADM}$ at infinity
    \item Works for ANY sign of $\tr_\Sigma k$
\end{itemize}

This is an \textbf{overdetermined problem}---no solution may exist in the
classical Geroch framework.
\end{tcolorbox}

\begin{tcolorbox}[colback=green!5, colframe=green!75!black, title=Future Direction]
\textbf{Non-flow approaches:} Perhaps the solution is not a geometric flow at all.

\textbf{Alternatives to consider:}
\begin{itemize}
    \item Optimal transport methods
    \item Variational characterization of mass
    \item Rigidity/stability arguments
    \item Weak solutions and limiting arguments
\end{itemize}
\end{tcolorbox}

\end{document}
