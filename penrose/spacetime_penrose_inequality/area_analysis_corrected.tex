% CORRECTED ANALYSIS: Area comparison in trapped regions
% 
% Key: Understanding the geometry correctly

\documentclass{article}
\usepackage{amsmath,amsthm,amssymb}
\begin{document}

\section{Schwarzschild Example (Corrected)}

In Schwarzschild with mass $M$, horizon at $r = 2M$:
\begin{itemize}
    \item Area at $r = 2M$: $A = 16\pi M^2$ (the MOTS)
    \item Area at $r = 1.5M$: $A = 9\pi M^2$ (inside, trapped)
    \item Area at $r = M$: $A = 4\pi M^2$ (deeper inside)
\end{itemize}

Surfaces INSIDE the horizon have SMALLER area than the horizon itself!

\section{Reconciling with Null Expansion}

$\theta^+ < 0$ on trapped surfaces means:
\begin{quote}
"Outgoing null rays are converging"
\end{quote}

In Schwarzschild coordinates, outgoing null rays have $dr/dt > 0$ outside horizon but $dr/dt < 0$ inside horizon!

Inside the horizon:
\begin{itemize}
    \item "Outgoing" null rays actually move to SMALLER $r$
    \item They converge and hit the singularity
    \item Area DECREASES along these rays (toward singularity)
\end{itemize}

So along the outgoing null direction (which points to smaller $r$ inside the horizon):
\[
\frac{dA}{d\lambda} = \theta^+ \cdot A < 0
\]
Area decreases as we follow outgoing null rays toward the singularity.

Going from small $r$ (inner trapped surface) to larger $r$ (outermost MOTS) is going AGAINST the outgoing null direction inside the horizon!

\section{Correct Interpretation}

Let $\Sigma_0$ be at smaller $r$, $\Sigma^*$ at the horizon (larger $r$).

The "outgoing" null direction points from large $r$ to small $r$ (inside horizon).

To go from $\Sigma_0$ to $\Sigma^*$, we must go along INGOING null direction.

Along ingoing null direction with $\theta^-$:
\[
\theta^- < 0 \text{ (trapped condition)}
\]

So $\theta^- < 0$ means area also decreases along ingoing null rays!

Both $\theta^+$ and $\theta^-$ are negative on trapped surfaces. Area decreases in BOTH null directions.

But wait - how do we go from $\Sigma_0$ to $\Sigma^*$?

\section{The Key Insight: Spacelike Separation}

$\Sigma_0$ and $\Sigma^*$ are both in the SAME spatial slice!

They are NOT connected by null rays (which would leave the initial data surface).

They ARE connected by spacelike curves in $M$.

The initial data constraint $\theta^+ \le 0$ at each surface says nothing direct about how areas compare along spacelike paths.

\section{Area Comparison via Spacelike Paths}

To compare $A(\Sigma_0)$ and $A(\Sigma^*)$, we need a spacelike interpolation.

If there exists a foliation $\{\Sigma_t\}_{t \in [0,1]}$ with $\Sigma_0$ inner and $\Sigma_1 = \Sigma^*$:
\[
\frac{dA}{dt} = \int_{\Sigma_t} H\phi \, dA
\]
where $\phi > 0$ is the outward speed of the foliation.

Inside the trapped region, $H = \frac{1}{2}(\theta^+ + \theta^-) < 0$.

So $dA/dt < 0$, meaning area DECREASES as we go outward!

\textbf{This says $A(\Sigma_0) > A(\Sigma^*)$!}

\section{Contradiction with Schwarzschild?}

In Schwarzschild:
\begin{itemize}
    \item $\Sigma_0$: sphere at $r = 1.5M$, $A = 9\pi M^2$
    \item $\Sigma^*$: horizon at $r = 2M$, $A = 16\pi M^2$
\end{itemize}

Here $A(\Sigma^*) > A(\Sigma_0)$. But mean curvature $H < 0$ inside horizon?

Let me check the mean curvature of $r = \mathrm{const}$ surfaces in Schwarzschild.

In standard coordinates, the induced metric on $t = \mathrm{const}$, $r = \mathrm{const}$ is:
\[
ds^2 = r^2(d\theta^2 + \sin^2\theta \, d\phi^2)
\]

The 3-metric on $t = \mathrm{const}$ slice:
\[
ds_3^2 = \left(1 - \frac{2M}{r}\right)^{-1} dr^2 + r^2 d\Omega^2
\]

The mean curvature of $r = r_0$ in this 3-metric:

The unit outward normal is $n = \sqrt{1-2M/r}\, \partial_r$.

$H = -\mathrm{div}_\Sigma n = $ ... this requires careful calculation.

Actually, for $r > 2M$, $H > 0$ (surfaces are mean-convex toward infinity).
For $r < 2M$, the coordinate $r$ is timelike! The slice is not the same.

\section{Correct Initial Data for Schwarzschild}

The standard maximal slice of Schwarzschild is NOT the $t = \mathrm{const}$ slice inside the horizon.

Various choices of initial data slice:
\begin{enumerate}
    \item Time-symmetric slice (Brill-Lindquist coordinates): $k = 0$
    \item Maximal slice: $\tr k = 0$ but $k \neq 0$
\end{enumerate}

For the time-symmetric slice, $\theta^\pm = H \pm \tr_\Sigma k = H$ (since $k = 0$).

So trapped surface requires $H \le 0$. In isotropic coordinates:
\[
ds^2 = \Psi^4(dr^2 + r^2 d\Omega^2), \quad \Psi = 1 + \frac{M}{2r}
\]

The horizon is at $r = M/2$ in isotropic coordinates.

Apparent horizons on time-symmetric slices: the outermost minimal surface (MOTS with $H=0$).

In Schwarzschild with $k=0$: the throat at $r = M/2$ (isotropic) is a minimal surface with $A = 16\pi M^2$.

There are NO trapped surfaces on the time-symmetric slice extending to one asymptotic end! 

\section{Key Realization}

On a time-symmetric ($k = 0$) initial data set, $\theta^\pm = H$.

A trapped surface requires $H \le 0$ (minimal or "inward-pointing").

In Schwarzschild time-symmetric slice:
- Outside throat: $H > 0$ (mean-convex toward infinity)
- At throat: $H = 0$ (minimal surface = horizon)
- No "inside" in the single-end picture

The Schwarzschild time-symmetric slice with ONE asymptotic end has the throat as a minimal surface boundary.

There are no trapped surfaces on this data except the throat itself!

For trapped surfaces with $\tr_\Sigma k \neq 0$, we need non-time-symmetric data.

\section{Non-time-symmetric case}

On initial data with $k \neq 0$:

$\theta^+ = H + \tr_\Sigma k$ and $\theta^- = H - \tr_\Sigma k$.

A MOTS has $\theta^+ = 0$, so $H = -\tr_\Sigma k$.

If $\tr_\Sigma k > 0$, then $H < 0$ on MOTS.
If $\tr_\Sigma k < 0$, then $H > 0$ on MOTS.

The outermost MOTS $\Sigma^*$ satisfies:
- $\theta^+ = 0$
- stability implies $\tr_\Sigma k \ge 0$ (Andersson-Mars-Simon)

So on the outermost MOTS: $H \le 0$ (mean-convex toward interior).

\section{Revised Area Comparison}

$\Sigma^*$ = outermost MOTS:
\begin{itemize}
    \item $\theta^+(\Sigma^*) = 0$
    \item $H(\Sigma^*) = -\tr_{\Sigma^*} k \le 0$ (by stability + Andersson-Mars-Simon)
\end{itemize}

$\Sigma_0$ = inner trapped surface:
\begin{itemize}
    \item $\theta^+(\Sigma_0) < 0$ (strictly trapped)
    \item $H(\Sigma_0) = $ could be anything
\end{itemize}

For a foliation from $\Sigma_0$ to $\Sigma^*$:

Near $\Sigma^*$: $H \approx 0$, so $dA/dt \approx 0$

The sign of $H$ in the interior of the trapped region is NOT constrained uniformly.

\section{Conclusion}

Whether $A(\Sigma^*) \ge A(\Sigma_0)$ or not depends on the geometry of the trapped region.

\textbf{There is no universal inequality!}

The Penrose inequality for general trapped surfaces with $\tr_\Sigma k < 0$ requires new ideas beyond area comparison.

\end{document}
