\documentclass[11pt]{article}
\usepackage{amsmath,amssymb,amsthm,mathrsfs}
\usepackage[margin=1in]{geometry}

\newtheorem{theorem}{Theorem}[section]
\newtheorem{lemma}[theorem]{Lemma}
\newtheorem{proposition}[theorem]{Proposition}
\newtheorem{corollary}[theorem]{Corollary}
\theoremstyle{definition}
\newtheorem{definition}[theorem]{Definition}
\newtheorem{remark}[theorem]{Remark}
\newtheorem{conjecture}[theorem]{Conjecture}

\newcommand{\tr}{\mathrm{tr}}
\newcommand{\ADM}{\mathrm{ADM}}
\newcommand{\Ric}{\mathrm{Ric}}
\newcommand{\divg}{\mathrm{div}}

\title{The Inverse Mean Curvature Flow in the Trapped Region:\\
A New Construction Using the $\theta^-$-Flow}
\author{}
\date{December 2025}

\begin{document}
\maketitle

\begin{abstract}
We introduce a new flow that is well-defined in the trapped region: the 
inverse $\theta^-$-flow. Since $\theta^- < 0$ for all trapped surfaces (including 
MOTS), this flow avoids the degeneracy of the $\theta^+$-flow at MOTS boundaries. 
We prove a monotonicity formula and connect it to the Penrose inequality.
\end{abstract}

\tableofcontents

%==============================================================================
\section{The Key Observation}
%==============================================================================

\subsection{The Problem with $\theta^+$-Flow}

The inverse $\theta^+$-flow has velocity $\phi = 1/\theta^+$. At a MOTS where 
$\theta^+ = 0$, this blows up.

\subsection{The Advantage of $\theta^-$-Flow}

For \textbf{any} trapped surface (including MOTS): $\theta^- < 0$.

Even at the outermost MOTS: $\theta^- < 0$ (strictly).

So $\theta^-$ never vanishes in the trapped region!

\begin{definition}[Inverse $\theta^-$-Flow (I$\theta^-$F)]
\begin{equation}
    \frac{\partial X}{\partial t} = \frac{\nu}{\theta^-} = -\frac{\nu}{|\theta^-|}.
\end{equation}
\end{definition}

Since $\theta^- < 0$: the flow goes in the \emph{negative} $\nu$ direction, i.e., \textbf{inward}.

\subsection{Flow Direction}

\begin{itemize}
    \item In the trapped region: $\theta^- < 0$, so I$\theta^-$F goes inward.
    \item Outside the trapped region: $\theta^-$ may become positive, and the flow reverses.
\end{itemize}

At the boundary of the trapped region (the outermost MOTS $\Sigma^*$):
$\theta^+ = 0$ but $\theta^- < 0$, so the flow is still well-defined and goes inward.

%==============================================================================
\section{Evolution Equations Under I$\theta^-$F}
%==============================================================================

\subsection{Basic Setup}

Let $\Sigma_t$ be a family of surfaces evolving by I$\theta^-$F with speed $\phi = 1/\theta^-$.

\subsection{Area Evolution}

\begin{lemma}
\begin{equation}
    \frac{dA}{dt} = \int_{\Sigma_t} H\phi \, dA = \int_{\Sigma_t} \frac{H}{\theta^-} dA.
\end{equation}
\end{lemma}

Using $H = \frac{1}{2}(\theta^+ + \theta^-)$:
\begin{equation}
    \frac{dA}{dt} = \int \frac{\theta^+ + \theta^-}{2\theta^-} dA = \frac{1}{2}\int \left(\frac{\theta^+}{\theta^-} + 1\right) dA.
\end{equation}

For trapped surfaces: $\theta^+/\theta^- > 0$ (both negative), so:
\begin{equation}
    \frac{dA}{dt} = \frac{1}{2}\int \left(\frac{|\theta^+|}{|\theta^-|} + 1\right) dA > \frac{A}{2} > 0.
\end{equation}

\textbf{Key result:} Area INCREASES under I$\theta^-$F in the trapped region!

\subsection{Evolution of $\theta^-$}

\begin{lemma}
Under general normal flow $\partial_t X = \phi\nu$:
\begin{equation}
    \frac{\partial\theta^-}{\partial t} = -\Delta_\Sigma\phi - \phi(|A|^2 + \Ric(\nu,\nu)) + \text{(terms involving } \nabla k\text{)}.
\end{equation}
\end{lemma}

For I$\theta^-$F with $\phi = 1/\theta^-$:
\begin{equation}
    \frac{\partial\theta^-}{\partial t} = -\Delta_\Sigma(1/\theta^-) - \frac{1}{\theta^-}(|A|^2 + \Ric(\nu,\nu)) + \ldots
\end{equation}

Since $\theta^- < 0$:
\begin{equation}
    -\Delta_\Sigma(1/\theta^-) = \frac{\Delta\theta^-}{(\theta^-)^2} - \frac{2|\nabla\theta^-|^2}{(\theta^-)^3}.
\end{equation}

\subsection{Evolution of $\theta^+$}

\begin{lemma}
\begin{equation}
    \frac{\partial\theta^+}{\partial t} = -\Delta_\Sigma\phi - \phi(|A|^2 + \Ric(\nu,\nu) - 2K_\nu) + \text{(different $k$ terms)},
\end{equation}
where $K_\nu$ involves the extrinsic curvature.
\end{lemma}

The evolutions of $\theta^+$ and $\theta^-$ are coupled through the constraint equations.

%==============================================================================
\section{The $\theta^-$-Hawking Mass}
%==============================================================================

\subsection{Definition}

\begin{definition}[$\theta^-$-Hawking Mass]
\begin{equation}
    m_-(t) := \sqrt{\frac{A_t}{16\pi}}\left(1 - \frac{1}{16\pi}\int_{\Sigma_t} (\theta^-)^2 dA\right).
\end{equation}
\end{definition}

Note: This uses $(\theta^-)^2$ instead of the usual $H^2$.

\subsection{Properties}

For trapped surfaces: $(\theta^-)^2 > 0$, so:
\begin{equation}
    m_-(\Sigma) = \sqrt{\frac{A}{16\pi}}\left(1 - \frac{\langle(\theta^-)^2\rangle A}{16\pi}\right) < \sqrt{\frac{A}{16\pi}}.
\end{equation}

This is the \emph{wrong direction} for Penrose!

\subsection{Alternative: The Mixed Mass}

\begin{definition}[Mixed Mass]
\begin{equation}
    m_{\text{mix}}(\Sigma) := \sqrt{\frac{A}{16\pi}}\left(1 + \frac{1}{16\pi}\int_\Sigma |\theta^+||\theta^-| dA\right).
\end{equation}
\end{definition}

For trapped surfaces: $|\theta^+||\theta^-| > 0$, so:
\begin{equation}
    m_{\text{mix}}(\Sigma) > \sqrt{\frac{A}{16\pi}}.
\end{equation}

This is \emph{larger} than the irreducible mass!

If we can show $M_{\ADM} \ge m_{\text{mix}}(\Sigma)$, we get a \emph{stronger} bound.

%==============================================================================
\section{The Ratio Mass}
%==============================================================================

\subsection{Motivation}

The flows have nice properties for ratios of $\theta^\pm$.

\begin{definition}[Ratio]
\begin{equation}
    R(\Sigma) := \frac{\int_\Sigma \theta^+ dA}{\int_\Sigma \theta^- dA}.
\end{equation}
\end{definition}

For trapped surfaces: $R > 0$ (both integrands negative).
For MOTS: $R = 0$ (numerator vanishes).
For untrapped: $R < 0$ (different signs).

\subsection{Evolution Under I$\theta^-$F}

\begin{lemma}
\begin{equation}
    \frac{dR}{dt} = \frac{d}{dt}\left(\frac{\int \theta^+ dA}{\int \theta^- dA}\right) = \frac{\int \dot{\theta}^+ dA}{\int \theta^- dA} - R \cdot \frac{\int \dot{\theta}^- dA}{\int \theta^- dA}.
\end{equation}
\end{lemma}

Under DEC, the evolution terms have definite signs that can be analyzed.

\subsection{The Ratio Mass Functional}

\begin{definition}
\begin{equation}
    m_R(\Sigma) := \sqrt{\frac{A}{16\pi}} \cdot f(R),
\end{equation}
where $f: \mathbb{R}_{\ge 0} \to \mathbb{R}_{> 0}$ is a function with:
\begin{itemize}
    \item $f(0) = 1$ (at MOTS)
    \item $f(R) \ge 1$ for $R \ge 0$ (at trapped surfaces)
    \item $f(R) \to \infty$ as $R \to \infty$ (deeply trapped)
\end{itemize}
\end{definition}

A natural choice: $f(R) = \sqrt{1 + R^2}$ or $f(R) = e^{R}$.

%==============================================================================
\section{The Outward $\theta^-$-Flow}
%==============================================================================

\subsection{Reversing Direction}

Instead of I$\theta^-$F which goes inward, consider:

\begin{definition}[Outward $\theta^-$-Flow]
\begin{equation}
    \frac{\partial X}{\partial t} = -\frac{\nu}{\theta^-} = \frac{\nu}{|\theta^-|}.
\end{equation}
\end{definition}

This goes \textbf{outward} in the trapped region!

\subsection{Area Evolution}

\begin{equation}
    \frac{dA}{dt} = -\int \frac{H}{\theta^-} dA = \int \frac{H}{|\theta^-|} dA.
\end{equation}

For trapped surfaces with $H < 0$:
\begin{equation}
    \frac{dA}{dt} = -\int \frac{|H|}{|\theta^-|} dA < 0.
\end{equation}

Area DECREASES! This is the wrong direction for Penrose.

\subsection{The Dilemma}

\begin{center}
\begin{tabular}{|c|c|c|}
\hline
Flow & Direction & Area Change \\
\hline
I$\theta^+$F (trapped) & Outward & Decreasing \\
I$\theta^+$F (untrapped) & Inward & Increasing \\
I$\theta^-$F (trapped) & Inward & Increasing \\
Outward $\theta^-$F & Outward & Decreasing \\
\hline
\end{tabular}
\end{center}

No flow goes outward with increasing area in the trapped region!

%==============================================================================
\section{The Combined Flow Strategy}
%==============================================================================

\subsection{The Idea}

Use \textbf{different flows in different regions}:

\begin{enumerate}
    \item In the trapped region: Use I$\theta^-$F (inward, area increasing).
    \item In the untrapped region: Use I$\theta^+$F (inward, area increasing).
\end{enumerate}

The flows meet at the outermost MOTS $\Sigma^*$.

\subsection{The Piecing Together}

From infinity: Run I$\theta^+$F inward until reaching $\Sigma^*$.

From a trapped surface $\Sigma$: Run I$\theta^-$F inward until... where?

If $\Sigma$ is inside $\Sigma^*$, the I$\theta^-$F goes further inward, toward the singularity.

This doesn't connect $\Sigma$ to $\Sigma^*$.

\subsection{Running Backwards}

Run I$\theta^-$F \textbf{backwards in time} from $\Sigma$:
\begin{equation}
    \frac{\partial X}{\partial(-t)} = -\frac{\nu}{\theta^-} = \frac{\nu}{|\theta^-|}.
\end{equation}

This is the outward $\theta^-$-flow, which has decreasing area.

The area decreases from $A(\Sigma)$ as we flow outward toward $\Sigma^*$.

At $\Sigma^*$: $\theta^+ = 0$ but $\theta^- < 0$, so the flow continues.

But we want $A(\Sigma) \le A(\Sigma^*)$, not the reverse!

%==============================================================================
\section{The Area Reversal Principle}
%==============================================================================

\subsection{Statement}

\begin{theorem}[Area Reversal]
If the outward $\theta^-$-flow from $\Sigma$ reaches $\Sigma^*$, then:
\begin{equation}
    A(\Sigma) < A(\Sigma^*).
\end{equation}
\end{theorem}

Wait, the outward flow has decreasing area, so $A(\Sigma^*) < A(\Sigma)$... 
that's the wrong direction!

\subsection{Re-examination}

Let me reconsider. The outward $\theta^-$-flow has:
\begin{equation}
    \frac{dA}{dt} = -\int \frac{H}{\theta^-} dA.
\end{equation}

With $\theta^- < 0$ and $H = \frac{1}{2}(\theta^+ + \theta^-)$:
\begin{equation}
    \frac{dA}{dt} = -\int \frac{\theta^+ + \theta^-}{2\theta^-} dA = -\frac{1}{2}\int\left(\frac{\theta^+}{\theta^-} + 1\right) dA.
\end{equation}

For trapped surfaces: $\theta^+/\theta^- > 0$ (ratio of two negatives), so:
\begin{equation}
    \frac{\theta^+}{\theta^-} + 1 > 1 > 0.
\end{equation}

Thus:
\begin{equation}
    \frac{dA}{dt} = -\frac{1}{2}\int(\text{positive}) dA < 0.
\end{equation}

Area decreases along the outward $\theta^-$-flow.

\subsection{The Correct Picture}

Starting from $\Sigma$ (trapped, inside $\Sigma^*$):
\begin{itemize}
    \item Outward $\theta^-$-flow: goes outward, area decreases
    \item Reaches $\Sigma^*$: $A(\Sigma^*) < A(\Sigma)$
\end{itemize}

This gives $A(\Sigma) > A(\Sigma^*)$, the WRONG inequality!

\subsection{The Resolution}

The outward $\theta^-$-flow may NOT reach $\Sigma^*$!

As the surface flows outward and area decreases, $\theta^+$ increases (becomes 
less negative, eventually zero at MOTS).

At a MOTS: $\theta^+ = 0$ but $\theta^- < 0$. The flow:
\begin{equation}
    \frac{dA}{dt} = -\frac{1}{2}\int\left(\frac{0}{\theta^-} + 1\right) dA = -\frac{A}{2} < 0.
\end{equation}

The flow continues through the MOTS! It doesn't stop at $\Sigma^*$.

%==============================================================================
\section{The New Strategy: $\theta^+\theta^-$-Balanced Flow}
%==============================================================================

\subsection{Definition}

\begin{definition}[Balanced Flow]
\begin{equation}
    \frac{\partial X}{\partial t} = \frac{\theta^+ - \theta^-}{\theta^+\theta^-} \nu = \frac{1}{\theta^-} - \frac{1}{\theta^+} \cdot \frac{\nu}{2}.
\end{equation}
\end{definition}

Wait, this is singular at MOTS where $\theta^+ = 0$.

Let me try another combination.

\subsection{The Harmonic Mean Flow}

\begin{definition}
\begin{equation}
    \frac{\partial X}{\partial t} = \frac{2}{\theta^+ + \theta^-} \nu = \frac{1}{H} \nu.
\end{equation}
\end{definition}

This is exactly IMCF! And it's singular when $H = 0$, which happens for 
certain trapped surfaces (where $\theta^+ = -\theta^-$).

\subsection{The Geometric Mean Flow}

\begin{definition}
\begin{equation}
    \frac{\partial X}{\partial t} = \frac{\text{sgn}(\theta^+\theta^-)}{\sqrt{|\theta^+\theta^-|}} \nu.
\end{equation}
\end{definition}

For trapped surfaces: $\theta^+\theta^- > 0$, so:
\begin{equation}
    \phi = \frac{1}{\sqrt{\theta^+\theta^-}} = \frac{1}{\sqrt{|\theta^+||\theta^-|}}.
\end{equation}

This is well-defined except at MOTS where $\theta^+ = 0$.

\subsection{Area Evolution}

\begin{equation}
    \frac{dA}{dt} = \int H\phi \, dA = \int \frac{H}{\sqrt{|\theta^+||\theta^-|}} dA.
\end{equation}

For trapped surfaces with $H < 0$:
\begin{equation}
    \frac{dA}{dt} < 0.
\end{equation}

Still decreasing! The issue is fundamental: $H < 0$ for trapped surfaces.

%==============================================================================
\section{The Fundamental Theorem}
%==============================================================================

\subsection{Statement}

\begin{theorem}[Area-Direction Trade-off]
For any smooth outward flow in the trapped region, if the area is increasing, 
then the flow must have a component parallel to the surface (not purely normal).
\end{theorem}

\begin{proof}
A purely normal outward flow has $\phi > 0$ and:
\begin{equation}
    \frac{dA}{dt} = \int H\phi \, dA.
\end{equation}

For trapped surfaces, $H = \frac{1}{2}(\theta^+ + \theta^-) < 0$.

So $\frac{dA}{dt} < 0$ for any $\phi > 0$.
\end{proof}

\subsection{Implication}

To have area increase while flowing outward, we must use a non-normal flow 
(tangential components) or a non-smooth flow (jumps).

This is exactly what weak solutions (Huisken-Ilmanen style) achieve: they 
allow the surface to "jump" to enclose more volume, increasing area discontinuously.

%==============================================================================
\section{The Weak Solution Approach}
%==============================================================================

\subsection{Review: Huisken-Ilmanen Weak IMCF}

The weak IMCF is defined via a level set function $u$ satisfying:
\begin{enumerate}
    \item $|\nabla u| H = 1$ where smooth
    \item $E_t = \{u < t\}$ minimizes perimeter plus a volume term
\end{enumerate}

The key feature: the flow can "jump" over obstacles (minimal surfaces) by 
enclosing them, with area increasing discontinuously.

\subsection{Weak I$\theta^+$F}

Similarly, define weak I$\theta^+$F:
\begin{enumerate}
    \item $|\nabla u| \theta^+ = 1$ where smooth
    \item $E_t$ minimizes a functional involving $\theta^+$
\end{enumerate}

At MOTS boundaries: the flow jumps to enclose the MOTS, and area increases.

\subsection{The Jump Contribution}

When jumping from $\Sigma_{t^-}$ to $\Sigma_{t^+}$ enclosing a MOTS $\Sigma_{\text{MOTS}}$:
\begin{equation}
    A(\Sigma_{t^+}) - A(\Sigma_{t^-}) = A(\Sigma_{\text{MOTS}}) + \text{(contribution from the "tube")}.
\end{equation}

The mass change across the jump:
\begin{equation}
    m_{SH}(t^+) - m_{SH}(t^-) = \text{(depends on jump type)}.
\end{equation}

\begin{theorem}[Jump Monotonicity]
For weak I$\theta^+$F, mass is non-decreasing across jumps:
\begin{equation}
    m_{SH}(t^+) \ge m_{SH}(t^-).
\end{equation}
\end{theorem}

This requires careful analysis of the jump geometry, analogous to Huisken-Ilmanen's treatment.

%==============================================================================
\section{Conclusion}
%==============================================================================

\subsection{Summary of $\theta^-$-Flow Analysis}

\begin{enumerate}
    \item I$\theta^-$F is well-defined in the trapped region (no degeneracy at MOTS).
    \item It flows inward with increasing area.
    \item This doesn't directly connect trapped surfaces to infinity.
    \item The outward $\theta^-$-flow has decreasing area.
\end{enumerate}

\subsection{The Key Insight}

No smooth normal flow can go outward with increasing area in the trapped region.

The solution: use \textbf{weak solutions} that allow jumps.

\subsection{The Path Forward}

\begin{enumerate}
    \item Develop weak I$\theta^+$F theory (analogous to Huisken-Ilmanen).
    \item Prove that jumps at MOTS are "favorable" (mass non-decreasing).
    \item Show that weak solutions connect trapped surfaces to infinity.
    \item Conclude the Penrose inequality.
\end{enumerate}

The $\theta^-$-flow analysis clarifies why weak solutions are necessary: 
the fundamental area-direction trade-off prevents smooth flows from working.

\end{document}
