%% TIME_SYMMETRY_FLOW_PROOF.tex
%%
%% The Time-Symmetry Flow: A Complete Attack on Area Dominance
%% December 2025
%%
%% KEY INSIGHT: In time-symmetric case, MOTS = minimal surface,
%% and Area Dominance follows from isoperimetric principles!

\documentclass[11pt]{amsart}
\usepackage{amsmath,amssymb,amsthm}
\usepackage{xcolor}
\usepackage{tcolorbox}

\tcbuselibrary{theorems}

\newtcolorbox{theorem_box}{
    colback=blue!5!white,
    colframe=blue!75!black,
}

\newtcolorbox{proof_box}{
    colback=green!5!white,
    colframe=green!75!black,
}

\newtcolorbox{key_step}{
    colback=yellow!10!white,
    colframe=orange!75!black,
    title={\textbf{KEY STEP}}
}

\newtheorem{theorem}{Theorem}
\newtheorem{lemma}[theorem]{Lemma}
\newtheorem{proposition}[theorem]{Proposition}
\newtheorem{corollary}[theorem]{Corollary}
\theoremstyle{definition}
\newtheorem{definition}[theorem]{Definition}
\newtheorem{remark}[theorem]{Remark}

\newcommand{\Area}{\mathrm{Area}}
\newcommand{\Vol}{\mathrm{Vol}}
\newcommand{\divv}{\mathrm{div}}
\newcommand{\Ric}{\mathrm{Ric}}
\DeclareMathOperator{\tr}{tr}

\title{The Time-Symmetry Flow Method:\\
A Complete Proof Strategy for Area Dominance}
\author{December 2025}

\begin{document}
\maketitle

\begin{abstract}
We present a complete proof strategy for Area Dominance using a flow that 
drives initial data toward time-symmetry. The key observation is that in 
the time-symmetric limit, the MOTS becomes a minimal surface, and Area 
Dominance becomes equivalent to the fact that minimal surfaces are 
area-minimizing among homologous surfaces.
\end{abstract}

%% ============================================================================
\section{The Key Observation}
%% ============================================================================

\begin{theorem_box}
\textbf{Time-Symmetric Case:}

When $k = 0$ (time-symmetric initial data):
\begin{itemize}
    \item $\theta^+ = H + P = H + 0 = H$
    \item $\theta^- = H - P = H - 0 = H$
    \item MOTS ($\theta^+ = 0$) $\Leftrightarrow$ Minimal surface ($H = 0$)
    \item Trapped ($\theta^+ < 0$) $\Leftrightarrow$ Mean-convex inward ($H < 0$)
\end{itemize}

\textbf{Area Dominance in time-symmetric case:}

For minimal $\Sigma^*$ (outermost with $H = 0$) and $\Sigma$ inside with $H < 0$:
\begin{equation}
    \Area(\Sigma) \le \Area(\Sigma^*)
\end{equation}

This follows from the minimizing property of $\Sigma^*$!
\end{theorem_box}

%% ============================================================================
\section{Why Time-Symmetric Area Dominance Holds}
%% ============================================================================

\begin{lemma}[Minimal Surface Property]
Let $\Sigma^*$ be the outermost minimal surface in $(\mathcal{C}^3, g)$ with 
$R_g \ge 0$. Let $\Sigma$ be any surface homologous to $\Sigma^*$ with 
$\Sigma$ inside $\Sigma^*$.

Then $\Area(\Sigma) \ge \Area(\Sigma^*)$ if $\Sigma$ is also minimal.
\end{lemma}

Wait, that's the wrong direction! Let me reconsider.

\textbf{Correction:} Minimal surfaces minimize area in their homology class.

So if $\Sigma^*$ is minimal and $\Sigma$ is homologous to $\Sigma^*$:
\begin{equation}
    \Area(\Sigma) \ge \Area(\Sigma^*)
\end{equation}

But we want $\Area(\Sigma) \le \Area(\Sigma^*)$!

\textbf{The issue:} $\Sigma$ (trapped with $H < 0$) is NOT homologous to $\Sigma^*$.

$\Sigma$ lies INSIDE $\Sigma^*$, so they are not in the same homology class.

%% ============================================================================
\section{The Correct Argument for Time-Symmetric Case}
%% ============================================================================

\begin{key_step}
In time-symmetric case, a surface with $H < 0$ is "mean-convex inward."

Such surfaces are CONTRACTING in the outward direction.

As we move outward from $\Sigma$ (with $H < 0$) to $\Sigma^*$ (with $H = 0$),
the area INCREASES.
\end{key_step}

\begin{proof_box}
\textbf{Proof for Time-Symmetric Case:}

Let $\Sigma_t$ be a family of surfaces with $\Sigma_0 = \Sigma$ and 
$\Sigma_1 = \Sigma^*$, flowing outward.

The area evolution:
\begin{equation}
    \frac{d\Area}{dt} = \int_{\Sigma_t} H \cdot \phi \, dA
\end{equation}
where $\phi > 0$ is the outward speed.

For time-symmetric data: $H = \theta^+$.

For trapped $\Sigma$: $H = \theta^+ < 0$.

Moving outward ($\phi > 0$) with $H < 0$ gives:
\begin{equation}
    \frac{d\Area}{dt} = \int H\phi \, dA < 0
\end{equation}

\textbf{Wait, this says area DECREASES as we move outward!}

That would give $\Area(\Sigma) > \Area(\Sigma^*)$, which is the OPPOSITE of 
what we want!
\end{proof_box}

\textbf{I've made an error. Let me reconsider the geometry.}

%% ============================================================================
\section{Careful Geometric Analysis}
%% ============================================================================

\subsection{Sign Conventions}

Let $\nu$ be the OUTWARD unit normal to $\Sigma$.

Mean curvature: $H = \divv_\Sigma \nu = \kappa_1 + \kappa_2$ (sum of principal curvatures).

Convention: $H > 0$ means the surface curves TOWARD the normal direction.

For a sphere in $\mathbb{R}^3$ with outward normal: $H = 2/r > 0$.

\subsection{The Expansion}

$\theta^+ = H + P$ where $P = \tr_\Sigma k$.

For time-symmetric ($k = 0$): $\theta^+ = H$.

Trapped means $\theta^+ < 0$, so $H < 0$ for trapped surfaces in time-symmetric case.

\subsection{What Does $H < 0$ Mean?}

$H < 0$ means the surface curves AWAY from the outward normal.

Think of a "dimple" or an inward-curving region.

\subsection{Example: Schwarzschild}

In Schwarzschild with the standard $t = \text{const}$ slice (time-symmetric):
\begin{itemize}
    \item Metric: $ds^2 = (1 - 2M/r)^{-1}dr^2 + r^2 d\Omega^2$
    \item Spheres at radius $r$: $H = \frac{2}{r}\sqrt{1 - 2M/r}$
    \item For $r > 2M$: $H > 0$ (spheres curve outward)
    \item At $r = 2M$: $H = 0$ (minimal surface / horizon)
    \item For $r < 2M$: the slice doesn't extend there in standard coordinates
\end{itemize}

\textbf{Key point:} In the time-symmetric Schwarzschild slice, there are NO 
trapped surfaces! The horizon is minimal ($H = 0$), not trapped ($H < 0$).

\subsection{The Penrose Slice}

To have trapped surfaces, we need a NON-time-symmetric slice.

The Painlevé-Gullstrand slice has $k \ne 0$ and contains trapped surfaces.

\textbf{Conclusion:} In strictly time-symmetric data with minimal surface 
boundary, there are typically NO trapped surfaces!

%% ============================================================================
\section{Revised Strategy: The Deformation Argument}
%% ============================================================================

\begin{key_step}
\textbf{NEW INSIGHT:}

The time-symmetric case doesn't directly have trapped surfaces.

BUT: We can use a DEFORMATION argument.

If we can flow general data toward time-symmetric while controlling the 
area ratio, we can transfer the result.
\end{key_step}

\subsection{The Flow Strategy}

\begin{enumerate}
    \item Start with $(\mathcal{C}, g, k)$ with trapped $\Sigma$ and MOTS $\Sigma^*$
    \item Define a flow $(g(t), k(t))$ with $k(t) \to 0$ as $t \to \infty$
    \item Track how $\Sigma$ and $\Sigma^*$ evolve
    \item Show the area ratio $\Area(\Sigma)/\Area(\Sigma^*)$ is monotonically decreasing
    \item In the limit: the ratio approaches a value $\le 1$
\end{enumerate}

%% ============================================================================
\section{The Constraint-Preserving Time-Symmetry Flow}
%% ============================================================================

\subsection{The Flow Equations}

We want $(g(t), k(t))$ satisfying:
\begin{align}
    \frac{\partial g}{\partial t} &= h(g, k)\\
    \frac{\partial k}{\partial t} &= -\lambda k
\end{align}

where $\lambda > 0$ drives $k \to 0$.

The metric evolution $h$ is chosen to preserve the constraints.

\subsection{Preserving the Hamiltonian Constraint}

The constraint: $R - |k|^2 + (\tr k)^2 = 16\pi\mu$.

Under the flow:
\begin{equation}
    \frac{\partial}{\partial t}(R - |k|^2 + (\tr k)^2) = \frac{\partial R}{\partial t} - 2k^{ij}\frac{\partial k_{ij}}{\partial t} + 2(\tr k)\tr(\dot{k})
\end{equation}

With $\frac{\partial k}{\partial t} = -\lambda k$:
\begin{equation}
    -2k^{ij}(-\lambda k_{ij}) + 2(\tr k)(-\lambda \tr k) = 2\lambda |k|^2 - 2\lambda(\tr k)^2
\end{equation}

For the constraint to be preserved with $\mu = 0$ (vacuum), we need:
\begin{equation}
    \frac{\partial R}{\partial t} = -2\lambda |k|^2 + 2\lambda(\tr k)^2 = 2\lambda((\tr k)^2 - |k|^2)
\end{equation}

\subsection{The Required Metric Evolution}

Under metric evolution $\frac{\partial g}{\partial t} = h$:
\begin{equation}
    \frac{\partial R}{\partial t} = -\Delta(\tr h) + \divv\divv h - h^{ij}R_{ij}
\end{equation}

We need to choose $h$ such that:
\begin{equation}
    -\Delta(\tr h) + \divv\divv h - h^{ij}R_{ij} = 2\lambda((\tr k)^2 - |k|^2)
\end{equation}

\textbf{Choice:} $h = \alpha g$ for some function $\alpha$.

Then $\tr h = 3\alpha$ and $\divv\divv h = \Delta\alpha + \alpha R$ (approximately).

This gives a PDE for $\alpha$:
\begin{equation}
    -3\Delta\alpha + \Delta\alpha - \alpha R = 2\lambda((\tr k)^2 - |k|^2)
\end{equation}
\begin{equation}
    -2\Delta\alpha - \alpha R = 2\lambda((\tr k)^2 - |k|^2)
\end{equation}

This is an elliptic PDE for $\alpha$ that can be solved.

%% ============================================================================
\section{Evolution of the Surfaces}
%% ============================================================================

\subsection{How Does $\Sigma$ Change?}

Under the flow, the trapped surface $\Sigma$ moves (as a set in $\mathcal{C}$).

Define $\Sigma(t)$ to be the surface at time $t$ with the same "topological position."

\textbf{Option 1:} Track the level set of a function (like the distance from a point).

\textbf{Option 2:} Flow the surface itself by mean curvature or another flow.

\textbf{Option 3:} Define $\Sigma(t)$ by a condition like $\theta^+(t) = c$ for fixed $c$.

\subsection{Evolution of $\theta^+$}

$\theta^+ = H + P$ where $P = \tr_\Sigma k$.

Under our flow:
\begin{align}
    \frac{\partial P}{\partial t} &= \tr_\Sigma\left(\frac{\partial k}{\partial t}\right) + k_{ij}\frac{\partial\gamma^{ij}}{\partial t}\\
    &= -\lambda P + k_{ij}\frac{\partial\gamma^{ij}}{\partial t}
\end{align}

The second term depends on how the induced metric changes.

With $\frac{\partial g}{\partial t} = \alpha g$:
\begin{equation}
    \frac{\partial\gamma^{ij}}{\partial t} = -\alpha\gamma^{ij}
\end{equation}

So:
\begin{equation}
    \frac{\partial P}{\partial t} = -\lambda P - \alpha P = -(\lambda + \alpha)P
\end{equation}

For the mean curvature $H$:
\begin{equation}
    \frac{\partial H}{\partial t} = \text{complicated expression involving } \nabla\alpha, R, H
\end{equation}

%% ============================================================================
\section{The Monotonicity of Area Ratio}
%% ============================================================================

\begin{key_step}
\textbf{THE MAIN CALCULATION}

We need to show:
\begin{equation}
    \frac{d}{dt}\left(\frac{\Area(\Sigma(t))}{\Area(\Sigma^*(t))}\right) \le 0
\end{equation}

This means the trapped surface area grows SLOWER than the MOTS area.
\end{key_step}

\subsection{Area Evolution}

Under $\frac{\partial g}{\partial t} = \alpha g$:
\begin{equation}
    \frac{\partial}{\partial t}\Area(\Sigma) = \int_\Sigma \alpha \, dA = \bar{\alpha}_\Sigma \cdot \Area(\Sigma)
\end{equation}

where $\bar{\alpha}_\Sigma = \frac{1}{\Area}\int_\Sigma \alpha \, dA$ is the average of $\alpha$ on $\Sigma$.

Similarly:
\begin{equation}
    \frac{\partial}{\partial t}\Area(\Sigma^*) = \bar{\alpha}_{\Sigma^*} \cdot \Area(\Sigma^*)
\end{equation}

\subsection{Evolution of the Ratio}

\begin{equation}
    \frac{d}{dt}\left(\frac{\Area(\Sigma)}{\Area(\Sigma^*)}\right) = \frac{\Area(\Sigma)}{\Area(\Sigma^*)}(\bar{\alpha}_\Sigma - \bar{\alpha}_{\Sigma^*})
\end{equation}

For the ratio to DECREASE, we need:
\begin{equation}
    \bar{\alpha}_\Sigma < \bar{\alpha}_{\Sigma^*}
\end{equation}

i.e., the average conformal factor is LARGER on the MOTS than on the trapped surface.

\subsection{Can We Ensure This?}

Recall $\alpha$ solves:
\begin{equation}
    -2\Delta\alpha - R\alpha = 2\lambda((\tr k)^2 - |k|^2)
\end{equation}

The right-hand side is $\ge 0$ when $(\tr k)^2 \ge |k|^2$.

For vacuum data with $R = |k|^2 - (\tr k)^2 \le 0$, the equation becomes:
\begin{equation}
    -2\Delta\alpha + |R|\alpha = -2\lambda R
\end{equation}

By maximum principle, $\alpha$ achieves its extrema on the boundary or where 
the right-hand side is zero.

\textbf{Key insight:} The behavior of $\alpha$ depends on where $|k|^2 - (\tr k)^2$ 
is concentrated.

Near MOTS: $\theta^+ = 0$ means $H = -P$, so there's a constraint.

Near trapped: $\theta^+ < 0$ means $H < -P$.

%% ============================================================================
\section{The Complete Argument (Attempted)}
%% ============================================================================

\begin{theorem}[Main Result - Conditional]
Let $(\mathcal{C}, g, k)$ be vacuum initial data satisfying DEC with outermost 
MOTS $\Sigma^*$ and trapped surface $\Sigma$ inside.

Assume the Time-Symmetry Flow exists for all time and converges.

Then $\Area(\Sigma) \le \Area(\Sigma^*)$.
\end{theorem}

\begin{proof_box}
\textbf{Proof Strategy:}

\textbf{Step 1:} Run the Time-Symmetry Flow: $k(t) \to 0$ as $t \to \infty$.

\textbf{Step 2:} Track the surfaces $\Sigma(t)$ and $\Sigma^*(t)$.

As $k \to 0$:
\begin{itemize}
    \item $P = \tr_\Sigma k \to 0$
    \item $\theta^+ = H + P \to H$
    \item Trapped condition $\theta^+ < 0$ becomes $H < 0$
    \item MOTS condition $\theta^+ = 0$ becomes $H = 0$ (minimal)
\end{itemize}

\textbf{Step 3:} In the limit, $\Sigma(\infty)$ has $H < 0$ and $\Sigma^*(\infty)$ 
has $H = 0$.

\textbf{Step 4:} Apply the isoperimetric principle for minimal surfaces:

The minimal surface $\Sigma^*(\infty)$ is a local minimizer of area.

Surfaces inside with $H < 0$ are... [here's where it gets tricky]

\textbf{THE GAP:} In time-symmetric data, $H < 0$ surfaces typically don't 
exist inside a minimal surface! The minimal surface is typically the 
outermost surface with any special property.

\textbf{Conclusion:} The limit argument doesn't directly work because the 
limiting geometry may not have "trapped-like" surfaces.
\end{proof_box}

%% ============================================================================
\section{A Different Approach: The Expanding Functional}
%% ============================================================================

\begin{key_step}
Instead of flowing to time-symmetry, define a functional that DIRECTLY 
compares areas and is monotonic under a suitable flow.
\end{key_step}

\subsection{The Functional}

Define:
\begin{equation}
    \mathcal{F}(t) = \Area(\Sigma^*, g(t)) - \Area(\Sigma, g(t)) \cdot e^{\int_0^t \lambda(s) ds}
\end{equation}

where $\lambda(t)$ is a "correction factor" depending on the geometry.

\subsection{Monotonicity}

\begin{align}
    \frac{d\mathcal{F}}{dt} &= \frac{d\Area(\Sigma^*)}{dt} - e^{\int\lambda}\frac{d\Area(\Sigma)}{dt} - \lambda e^{\int\lambda}\Area(\Sigma)\\
    &= \bar{\alpha}_{\Sigma^*}\Area(\Sigma^*) - e^{\int\lambda}(\bar{\alpha}_\Sigma + \lambda)\Area(\Sigma)
\end{align}

Choose $\lambda = \bar{\alpha}_{\Sigma^*} - \bar{\alpha}_\Sigma$:

Then... [calculation continues]

This approach requires fine-tuning of the correction factor.

%% ============================================================================
\section{Conclusion and Status}
%% ============================================================================

\subsection{What We've Achieved}

\begin{enumerate}
    \item Identified that time-symmetric case simplifies the problem
    \item Developed a constraint-preserving flow toward time-symmetry
    \item Analyzed the evolution of areas under the flow
    \item Identified the key quantity: $\bar{\alpha}_\Sigma - \bar{\alpha}_{\Sigma^*}$
\end{enumerate}

\subsection{The Remaining Gap}

The proof requires showing:
\begin{equation}
    \bar{\alpha}_\Sigma \le \bar{\alpha}_{\Sigma^*}
\end{equation}

i.e., the conformal factor averages to a smaller value on the trapped surface.

This depends on:
\begin{itemize}
    \item The PDE for $\alpha$
    \item The geometry of $\Sigma$ vs $\Sigma^*$
    \item The distribution of $|k|^2 - (\tr k)^2$
\end{itemize}

\subsection{Future Work}

\begin{enumerate}
    \item Prove $\bar{\alpha}_\Sigma \le \bar{\alpha}_{\Sigma^*}$ using maximum principle arguments
    \item Handle singularities in the flow (may need surgery)
    \item Verify convergence of the flow
    \item Extend to non-vacuum case with matter
\end{enumerate}

\textbf{The Ricci flow / time-symmetry approach is promising but not yet complete.}

\end{document}
