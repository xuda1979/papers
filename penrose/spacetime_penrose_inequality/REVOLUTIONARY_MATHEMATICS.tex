%% REVOLUTIONARY_MATHEMATICS.tex
%%
%% Revolutionary New Mathematics for Area Dominance
%% December 2025
%%
%% The Key Insight: We've been thinking about this WRONG.

\documentclass[11pt]{amsart}
\usepackage{amsmath,amssymb,amsthm}
\usepackage{xcolor}
\usepackage{tcolorbox}

\tcbuselibrary{theorems}

\newtcolorbox{insight}{
    colback=green!5!white,
    colframe=green!75!black,
    title={\textbf{KEY INSIGHT}}
}

\newtcolorbox{revolution}{
    colback=purple!5!white,
    colframe=purple!75!black,
    title={\textbf{REVOLUTIONARY IDEA}}
}

\newtheorem{theorem}{Theorem}
\newtheorem{lemma}[theorem]{Lemma}
\newtheorem{proposition}[theorem]{Proposition}
\newtheorem{corollary}[theorem]{Corollary}
\theoremstyle{definition}
\newtheorem{definition}[theorem]{Definition}
\newtheorem{remark}[theorem]{Remark}

\newcommand{\Area}{\mathrm{Area}}
\newcommand{\Vol}{\mathrm{Vol}}
\newcommand{\divv}{\mathrm{div}}
\DeclareMathOperator{\tr}{tr}

\title{Revolutionary Mathematics for Area Dominance:\\
The Spacetime Divergence Method}
\author{December 2025}

\begin{document}
\maketitle

\begin{abstract}
We develop a revolutionary new approach to Area Dominance using the 
\textbf{spacetime divergence theorem} applied to carefully constructed 
tensor fields. The key insight is to work in \textbf{4D spacetime} rather 
than the 3D initial data surface, exploiting the null structure directly.
\end{abstract}

%% ============================================================================
\section{The Fundamental Obstruction (Why Previous Approaches Fail)}
%% ============================================================================

\textbf{The Problem:} Prove $\Area(\Sigma) \le \Area(\Sigma^*)$ for trapped 
$\Sigma$ inside MOTS $\Sigma^*$.

\textbf{Why flows fail:}
\begin{itemize}
    \item Any flow from $\Sigma$ to $\Sigma^*$ has $\frac{d\Area}{dt} = \int H\phi \, dA$
    \item The sign of $H$ is NOT determined by $\theta^+ < 0$
    \item For trapped: $H = \theta^+ - P$, and $P$ can dominate
\end{itemize}

\textbf{Why functionals fail:}
\begin{itemize}
    \item Hawking mass gives bounds with wrong constants
    \item Exponential functionals lack monotonicity proofs
    \item The constraint equations don't directly control area
\end{itemize}

\begin{insight}
\textbf{The obstruction is trying to work on the 3D slice.}

On a spacelike slice, the trapped condition involves BOTH null expansions 
($\theta^+$ and $\theta^-$), which encode different information than 
mean curvature $H$.

\textbf{We need to work in 4D spacetime where the null structure is natural.}
\end{insight}

%% ============================================================================
\section{Revolutionary Idea: The Null Cone Comparison}
%% ============================================================================

\begin{revolution}
\textbf{Don't flow on the 3D slice. Flow along the NULL CONE.}

The past light cone from infinity naturally "sees" both the trapped surface 
$\Sigma$ and the MOTS $\Sigma^*$. The area along null generators decreases 
toward the past (Raychaudhuri!).
\end{revolution}

\subsection{Setup}

Consider the spacetime $(M^4, g_{\mu\nu})$ satisfying:
\begin{itemize}
    \item DEC: $T_{\mu\nu}V^\mu W^\nu \ge 0$ for future causal $V, W$
    \item Asymptotically flat
    \item Contains trapped surface $\Sigma$ and outermost MOTS $\Sigma^*$
\end{itemize}

Let $\mathcal{N}^-$ be the \textbf{past null cone} from future null infinity, 
truncated at the initial data surface $\mathcal{C}$.

\subsection{The Null Generator Picture}

Each point on $\Sigma^*$ lies on a null generator of $\mathcal{N}^-$ 
(or can be connected to one).

Each point on $\Sigma$ lies further in the past along these generators.

\textbf{Key:} Area along null generators satisfies:
\begin{equation}
    \frac{d\Area}{d\lambda} = \int_S \theta \, dA
\end{equation}
where $\lambda$ is affine parameter and $\theta$ is the null expansion.

For outgoing null cone from $\Sigma$: $\theta|_\Sigma = \theta^+ < 0$ (trapped).

%% ============================================================================
\section{The Spacetime Divergence Theorem}
%% ============================================================================

\begin{revolution}
\textbf{Use the 4D divergence theorem with a carefully constructed current.}
\end{revolution}

\subsection{The Key Current}

Define the \textbf{area current}:
\begin{equation}
    J^\mu = \rho \, \ell^\mu
\end{equation}
where:
\begin{itemize}
    \item $\ell^\mu$ is the outgoing null normal (with $\ell^\mu k_\mu = -1$ for ingoing $k$)
    \item $\rho$ is a scalar weight function to be determined
\end{itemize}

\subsection{Divergence Calculation}

\begin{equation}
    \nabla_\mu J^\mu = \nabla_\mu(\rho \ell^\mu) = \ell^\mu \nabla_\mu \rho + \rho \nabla_\mu \ell^\mu
\end{equation}

The key term:
\begin{equation}
    \nabla_\mu \ell^\mu = \theta^+
\end{equation}

So:
\begin{equation}
    \nabla_\mu J^\mu = \ell(\rho) + \rho \theta^+
\end{equation}

\subsection{The Choice of $\rho$}

\textbf{Choose:} $\rho$ such that $\ell(\rho) = -\rho\theta^+$, i.e.,
\begin{equation}
    \frac{d\rho}{d\lambda} = -\rho\theta^+
\end{equation}

Solution: $\rho(\lambda) = \rho_0 \exp\left(-\int_0^\lambda \theta^+ d\lambda'\right)$

With this choice:
\begin{equation}
    \nabla_\mu J^\mu = 0
\end{equation}

$J^\mu$ is a \textbf{conserved current}!

%% ============================================================================
\section{The Area Comparison via Flux Conservation}
%% ============================================================================

\subsection{The Spacetime Region}

Consider the 4D region $\mathcal{R}$ bounded by:
\begin{itemize}
    \item $\Sigma$: trapped surface (inner boundary on the slice)
    \item $\Sigma^*$: MOTS (outer boundary on the slice)
    \item $\mathcal{N}^+$: outgoing null hypersurface from $\Sigma$
    \item $\mathcal{N}^-$: ingoing null hypersurface from $\Sigma^*$
\end{itemize}

(This is a "causal diamond" type region.)

\subsection{Flux Conservation}

By the divergence theorem:
\begin{equation}
    0 = \int_{\mathcal{R}} \nabla_\mu J^\mu \, dV = \oint_{\partial\mathcal{R}} J^\mu n_\mu \, d\Sigma
\end{equation}

The boundary contributions:
\begin{align}
    \int_\Sigma J^\mu n_\mu &= -\int_\Sigma \rho \, dA \quad (\text{with } n_\mu \propto -\ell_\mu)\\
    \int_{\Sigma^*} J^\mu n_\mu &= \int_{\Sigma^*} \rho \, dA\\
    \int_{\mathcal{N}^+} J^\mu n_\mu &= 0 \quad (\ell^\mu \text{ tangent to } \mathcal{N}^+)\\
    \int_{\mathcal{N}^-} J^\mu n_\mu &= 0 \quad (\ell^\mu \text{ orthogonal to } k^\mu)
\end{align}

Wait, the null boundary terms need more care.

\subsection{Revised: 3+1 Foliation Approach}

Let me instead use a foliation approach within the initial data surface.

Consider a family of surfaces $\Sigma_t$ with $\Sigma_0 = \Sigma$ and 
$\Sigma_1 = \Sigma^*$, parametrized by $t \in [0,1]$.

For each surface, define:
\begin{equation}
    F(t) = \int_{\Sigma_t} \rho_t \, dA_t
\end{equation}

where $\rho_t$ encodes the "null expansion memory":
\begin{equation}
    \rho_t = \exp\left(-\int_0^t \bar{\theta}^+(s) ds\right)
\end{equation}

with $\bar{\theta}^+$ being an averaged expansion along the path.

Hmm, this is getting complicated. Let me try a different revolutionary idea.

%% ============================================================================
\section{Revolutionary Idea 2: The Inverse Mean Curvature Flow Trick}
%% ============================================================================

\begin{revolution}
\textbf{Use IMCF but with a MODIFIED speed that accounts for $P$.}

Standard IMCF: $\frac{\partial x}{\partial t} = \frac{\nu}{H}$

New flow: $\frac{\partial x}{\partial t} = \frac{\nu}{\theta^+}$

This is the "$\theta^+$-inverse flow" - it blows up at the MOTS!
\end{revolution}

\subsection{The $\theta^+$-Inverse Flow}

Define the flow:
\begin{equation}
    \frac{\partial x}{\partial t} = \frac{\nu}{\theta^+}
\end{equation}

Since $\theta^+ < 0$ for trapped surfaces, this moves INWARD (opposite to normal).

Wait, that's the wrong direction. Let me reconsider.

\textbf{Corrected:} Define
\begin{equation}
    \frac{\partial x}{\partial t} = -\frac{\nu}{\theta^+}
\end{equation}

Since $\theta^+ < 0$, we have $-1/\theta^+ > 0$, so this moves OUTWARD.

\subsection{Area Evolution}

\begin{equation}
    \frac{d\Area}{dt} = \int_{\Sigma_t} H \cdot \left(-\frac{1}{\theta^+}\right) dA = -\int_{\Sigma_t} \frac{H}{\theta^+} dA
\end{equation}

Write $H = \theta^+ - P$:
\begin{equation}
    \frac{d\Area}{dt} = -\int_{\Sigma_t} \frac{\theta^+ - P}{\theta^+} dA = -\int_{\Sigma_t} \left(1 - \frac{P}{\theta^+}\right) dA
\end{equation}

\begin{equation}
    \frac{d\Area}{dt} = -\Area + \int_{\Sigma_t} \frac{P}{\theta^+} dA
\end{equation}

Since $\theta^+ < 0$ and $P$ can have either sign:
\begin{itemize}
    \item If $P < 0$: $\frac{P}{\theta^+} > 0$, so $\frac{d\Area}{dt} > -\Area$
    \item If $P > 0$: $\frac{P}{\theta^+} < 0$, so $\frac{d\Area}{dt} < -\Area$
\end{itemize}

This doesn't give clean monotonicity either!

%% ============================================================================
\section{Revolutionary Idea 3: The Optimal Transport Approach}
%% ============================================================================

\begin{revolution}
\textbf{View Area Dominance as an OPTIMAL TRANSPORT problem.}

The trapped surface $\Sigma$ must "transport" its area to $\Sigma^*$ along 
paths respecting the causal structure. The "cost" of transport is related 
to the expansion.
\end{revolution}

\subsection{Setup}

Let $\mu_\Sigma$ be the area measure on $\Sigma$ and $\mu_{\Sigma^*}$ be the 
area measure on $\Sigma^*$.

Define the \textbf{causal transport cost}:
\begin{equation}
    c(x, y) = \begin{cases}
        d(x,y)^2 & \text{if } x \prec y \text{ (causally precedes)}\\
        +\infty & \text{otherwise}
    \end{cases}
\end{equation}

\subsection{The Optimal Transport Problem}

Find the transport plan $\pi$ minimizing:
\begin{equation}
    \mathcal{T}[\pi] = \int_{\Sigma \times \Sigma^*} c(x,y) \, d\pi(x,y)
\end{equation}
subject to:
\begin{align}
    \int_{\Sigma^*} d\pi(x,y) &= d\mu_\Sigma(x)\\
    \int_\Sigma d\pi(x,y) &= d\mu_{\Sigma^*}(y)
\end{align}

\textbf{Key insight:} If the transport is possible with finite cost, then 
there's a mass-preserving map from $\Sigma$ to $\Sigma^*$.

But mass-preserving means $\Area(\Sigma) = \Area(\Sigma^*)$ only if the 
measures have the same total mass!

We need: $\Area(\Sigma) \le \Area(\Sigma^*)$, which means transport with 
\textbf{expansion}.

\subsection{Transport with Expansion}

Modify: allow the transport to \textbf{create} area, not destroy it.

The null expansion along the transport path creates area:
\begin{equation}
    \frac{d(\text{area element})}{d\lambda} = \theta \cdot (\text{area element})
\end{equation}

Along outgoing null geodesics from trapped $\Sigma$:
\begin{itemize}
    \item Initially: $\theta^+ < 0$, area decreases
    \item After reaching MOTS: $\theta^+ = 0$, area constant
\end{itemize}

\textbf{Problem:} Null geodesics from trapped surfaces go INTO the black hole, 
not toward the MOTS!

This is the fundamental causal obstruction.

%% ============================================================================
\section{Revolutionary Idea 4: The Dual Flow}
%% ============================================================================

\begin{revolution}
\textbf{Instead of flowing FROM $\Sigma$ TO $\Sigma^*$, flow FROM $\Sigma^*$ TO $\Sigma$.}

If we can show that flowing INWARD from MOTS DECREASES area, then 
$\Area(\Sigma) \le \Area(\Sigma^*)$.
\end{revolution}

\subsection{Inward Flow from MOTS}

Start at $\Sigma^*$ with $\theta^+ = 0$.

Flow inward: $\frac{\partial x}{\partial t} = -\phi \nu$ with $\phi > 0$.

\textbf{Area evolution:}
\begin{equation}
    \frac{d\Area}{dt} = -\int_{\Sigma_t} H\phi \, dA
\end{equation}

At $\Sigma^* = \Sigma_0$: $H = -P$ (since $\theta^+ = H + P = 0$).

\begin{equation}
    \left.\frac{d\Area}{dt}\right|_{t=0} = -\int_{\Sigma^*} (-P)\phi \, dA = \int_{\Sigma^*} P\phi \, dA
\end{equation}

\textbf{Sign analysis:}
\begin{itemize}
    \item If $P > 0$ on $\Sigma^*$: $\frac{d\Area}{dt} > 0$, moving inward INCREASES area! BAD.
    \item If $P < 0$ on $\Sigma^*$: $\frac{d\Area}{dt} < 0$, moving inward DECREASES area. GOOD.
\end{itemize}

The sign of $P$ on the MOTS is not determined!

\subsection{Using DEC to Control $P$}

The constraint equations under DEC:

\begin{equation}
    \divv(k - (\tr k)g) = 8\pi J, \quad |J| \le \mu
\end{equation}

Integrate over the region inside $\Sigma^*$:
\begin{equation}
    \int_{\Sigma^*} (k - (\tr k)g)(\nu, \cdot) = \int_{\text{inside}} 8\pi J
\end{equation}

The normal-tangential part gives constraints on $P = \tr_{\Sigma^*} k$.

\textbf{But:} This constrains the TOTAL $P$, not its pointwise sign.

%% ============================================================================
\section{Revolutionary Idea 5: The Bray-Khuri Method Revisited}
%% ============================================================================

\begin{revolution}
\textbf{Use the GENERALIZED Jang equation with a free boundary condition 
at the trapped surface.}
\end{revolution}

\subsection{The Generalized Jang Equation}

Solve:
\begin{equation}
    H_{\text{graph}(f)} - P_{\text{graph}(f)} = 0
\end{equation}

where the graph is in the warped product $(\mathcal{C} \times \mathbb{R}, g + e^{2\phi}dt^2)$.

\textbf{Boundary conditions:}
\begin{itemize}
    \item At infinity: $f \to 0$
    \item At $\Sigma$: Free boundary (the graph meets $\Sigma$ at some height)
    \item At $\Sigma^*$: The graph blows up (since $\theta^+ = 0$)
\end{itemize}

\subsection{The Area Bound}

The Jang surface $\hat{\Sigma}$ above $\Sigma$ has:
\begin{equation}
    \Area(\hat{\Sigma}) \ge \Area(\Sigma)
\end{equation}
(graphs have larger area than base).

At the MOTS, the Jang surface blows up and approaches a cylinder over $\Sigma^*$.

\textbf{Key:} The Riemannian Penrose inequality applied to the Jang surface gives:
\begin{equation}
    M_{\text{ADM}}(\hat{g}) \ge \sqrt{\frac{\Area(\text{horizon in Jang})}{16\pi}}
\end{equation}

The "horizon in Jang" is related to $\Sigma^*$, not $\Sigma$!

This is why we get the MOTS Penrose inequality, not the trapped surface one.

%% ============================================================================
\section{THE BREAKTHROUGH: Generalized Geroch Monotonicity}
%% ============================================================================

\begin{insight}
\textbf{The key is to find a flow that doesn't depend on $H$ alone, but on 
a combination that's controlled by the constraint equations.}
\end{insight}

\begin{revolution}
\textbf{THE SPACETIME HARMONIC FUNCTION METHOD}

Instead of flowing surfaces, use a harmonic function in the 4D spacetime 
and analyze its level sets.
\end{revolution}

\subsection{The Spacetime Harmonic Function}

Consider the 4D spacetime and solve:
\begin{equation}
    \Box u = 0 \quad \text{(wave equation)}
\end{equation}
with:
\begin{itemize}
    \item $u = 0$ on the event horizon $\mathcal{H}$
    \item $u \to 1$ at infinity
\end{itemize}

The level sets of $u$ foliate the exterior region.

\subsection{The Null Energy Condition and Level Sets}

Under NEC (implied by DEC):
\begin{equation}
    R_{\mu\nu} \ell^\mu \ell^\nu \ge 0
\end{equation}

For the wave equation solution:
\begin{equation}
    \Box u = g^{\mu\nu}\nabla_\mu\nabla_\nu u = 0
\end{equation}

The level sets $\{u = c\}$ have mean curvature related to $\nabla u$.

\subsection{Monotonicity of Area}

For level sets $S_c = \{u = c\}$:
\begin{equation}
    \frac{d\Area(S_c)}{dc} = \int_{S_c} \frac{H}{|\nabla u|} dA
\end{equation}

where $H$ is the mean curvature of $S_c$ in the spacetime.

\textbf{Key observation:} The wave equation $\Box u = 0$ implies:
\begin{equation}
    H = |\nabla u| \cdot K + \text{terms involving } \nabla^2 u
\end{equation}

where $K$ involves the spacetime curvature.

Under DEC, certain combinations are signed!

\subsection{The Connection to Area Dominance}

If the trapped surface $\Sigma$ lies on level set $u = c_1$ and MOTS $\Sigma^*$ 
lies on $u = c_2 > c_1$, then:

\textbf{Claim:} Under DEC, $\Area(S_{c_1}) \le \Area(S_{c_2})$ if the level 
sets have controlled geometry.

This would give $\Area(\Sigma) \le \Area(\Sigma^*)$!

\textbf{BUT:} The trapped surface $\Sigma$ is in the 3D slice, not a level 
set of the spacetime function $u$.

We need to relate the two.

%% ============================================================================
\section{THE ACTUAL BREAKTHROUGH: The Canonical Null Foliation}
%% ============================================================================

\begin{revolution}
\textbf{Use the NULL HYPERSURFACES generated by $\Sigma$ and $\Sigma^*$.}

The key is that both surfaces generate null hypersurfaces, and the 
AREA ALONG NULL GENERATORS is monotonic under NEC/DEC!
\end{revolution}

\subsection{Construction}

\begin{enumerate}
    \item From $\Sigma^*$: shoot ingoing null geodesics. These form $\mathcal{N}^-(\Sigma^*)$.
    \item The null expansion $\theta^-$ of $\Sigma^*$ satisfies $\theta^- < 0$ (MOTS).
    \item By Raychaudhuri: $\frac{d\theta^-}{d\lambda} \le -\frac{(\theta^-)^2}{2}$ under NEC.
    \item So $\theta^-$ becomes more negative: the null generators focus.
\end{enumerate}

\subsection{Area Along Ingoing Null from MOTS}

\begin{equation}
    \frac{d\Area}{d\lambda} = \int \theta^- dA < 0
\end{equation}

So area DECREASES along ingoing null from $\Sigma^*$!

\subsection{Where Does $\Sigma$ Lie?}

The trapped surface $\Sigma$ lies in the interior of the MOTS.

The ingoing null geodesics from $\Sigma^*$ may or may not hit $\Sigma$.

\textbf{Key geometric fact:} The trapped region is bounded by the MOTS. Any 
surface in the trapped region can be connected to the MOTS by curves.

\textbf{But:} The connection is NOT along null geodesics in general!

\subsection{The Resolution: Cross-Sectional Areas}

Consider the ingoing null hypersurface $\mathcal{N}^-$ from $\Sigma^*$.

Let $S_\lambda$ be the cross-section at affine parameter $\lambda$.

$S_0 = \Sigma^*$ with $\Area(S_0) = \Area(\Sigma^*)$.

As $\lambda$ increases (going inward):
\begin{equation}
    \Area(S_\lambda) < \Area(S_0) = \Area(\Sigma^*)
\end{equation}

\textbf{Claim:} The trapped surface $\Sigma$ has area bounded by $\Area(S_\lambda)$ 
for some $\lambda$.

\textbf{Issue:} $\Sigma$ is not a cross-section of $\mathcal{N}^-$ in general!

%% ============================================================================
\section{FINAL BREAKTHROUGH: The Isoperimetric Characterization}
%% ============================================================================

\begin{revolution}
\textbf{The MOTS is the OUTERMOST surface with $\theta^+ \le 0$.}

By definition, any surface with $\theta^+ < 0$ (trapped) lies INSIDE the MOTS.

Use the ISOPERIMETRIC property of the MOTS within the constraint-satisfying geometry.
\end{revolution}

\subsection{The Key Observation}

Let $\Omega^*$ be the region enclosed by the outermost MOTS $\Sigma^*$.

Any trapped surface $\Sigma$ satisfies $\Sigma \subset \Omega^*$.

\subsection{The Isoperimetric Principle}

\textbf{Theorem (Isoperimetric for Trapped Regions):}
Under DEC, the outermost MOTS $\Sigma^*$ has the LARGEST area among all 
surfaces in $\Omega^*$ with $\theta^+ \le 0$.

\textbf{Proof Idea:}
\begin{enumerate}
    \item $\Sigma^*$ is a critical point of area among $\theta^+ = 0$ surfaces (stability).
    \item Under DEC, the second variation is non-negative (strict stability of outermost).
    \item Any $\theta^+ < 0$ surface can be deformed outward, increasing $\theta^+$.
    \item The deformation increases area until reaching $\theta^+ = 0$.
    \item The outermost such surface is $\Sigma^*$.
\end{enumerate}

\subsection{Making This Rigorous}

\textbf{Step 1:} For trapped $\Sigma$, consider deformations with $\phi > 0$.
\begin{equation}
    \delta\theta^+ = L(\phi) = -\Delta\phi + 2\omega\cdot\nabla\phi + Q\phi
\end{equation}

Under DEC: $Q \ge $ (curvature terms that can be bounded).

\textbf{Step 2:} If $\theta^+ < 0$ and we choose $\phi$ to solve $L(\phi) = -\theta^+$:
\begin{equation}
    \delta\theta^+ = -\theta^+ > 0
\end{equation}

This increases $\theta^+$ toward zero.

\textbf{Step 3:} The area change:
\begin{equation}
    \delta\Area = \int H\phi \, dA
\end{equation}

If the flow is controlled until $\theta^+ = 0$, we reach a MOTS.

\textbf{Step 4:} The outermost MOTS has the largest area among all MOTS.

\textbf{Step 5:} Therefore $\Area(\Sigma) \le \Area(\text{deformed to MOTS}) \le \Area(\Sigma^*)$.

\subsection{THE GAP}

\textbf{Step 3 is not controlled:} The area change $\int H\phi \, dA$ depends on $H$, 
which can be negative!

Even if we flow to a MOTS, the area might INCREASE or DECREASE.

\textbf{We cannot conclude $\Area(\Sigma) \le \Area(\Sigma^*)$ this way.}

%% ============================================================================
\section{Honest Conclusion}
%% ============================================================================

After exploring multiple revolutionary ideas:
\begin{enumerate}
    \item Spacetime divergence theorem
    \item Null cone comparison
    \item $\theta^+$-inverse flow
    \item Optimal transport
    \item Dual flow from MOTS
    \item Generalized Jang
    \item Spacetime harmonic functions
    \item Null foliation
    \item Isoperimetric characterization
\end{enumerate}

\textbf{None of them directly prove Area Dominance.}

The fundamental obstruction remains: \textbf{the sign of mean curvature $H$ 
is not determined by the trapped condition $\theta^+ < 0$}.

\section{What Would Actually Work}

A successful proof would need to show ONE of:
\begin{enumerate}
    \item DEC + constraint equations imply $H \ge 0$ on trapped surfaces (FALSE in general)
    \item There exists a path from $\Sigma$ to $\Sigma^*$ along which area is non-decreasing (UNKNOWN)
    \item A monotonic functional interpolating between $\Area(\Sigma)$ and $\Area(\Sigma^*)$ (NOT FOUND)
    \item A variational principle making $\Sigma^*$ area-maximizing among trapped surfaces (NOT PROVEN)
\end{enumerate}

\textbf{Area Dominance remains a genuine open problem.}

\end{document}
