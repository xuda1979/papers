%% BLUE_RED_TEAM_ROUND3.tex
%%
%% ROUND 3: Attack on the Hull + Maximal Slice Approach
%%
%% Target: PENROSE_1973_BREAKTHROUGH.tex
%%
%% Red Team: Find fatal flaws
%% Blue Team: Fix or circumvent
%%
%% December 2025

\documentclass[11pt]{amsart}
\usepackage{amsmath,amssymb,amsthm}
\usepackage{xcolor}
\usepackage{tcolorbox}
\usepackage{booktabs}

\tcbuselibrary{theorems}

\newtcolorbox{redattack}{
    colback=red!5!white,
    colframe=red!75!black,
    title={\textbf{RED TEAM ATTACK}}
}

\newtcolorbox{bluedefense}{
    colback=blue!5!white,
    colframe=blue!75!black,
    title={\textbf{BLUE TEAM DEFENSE}}
}

\newtcolorbox{verdict}{
    colback=yellow!10!white,
    colframe=orange!75!black,
    title={\textbf{VERDICT}}
}

\newtcolorbox{newattack}{
    colback=purple!5!white,
    colframe=purple!75!black,
    title={\textbf{NEW ATTACK VECTOR}}
}

\newtheorem{theorem}{Theorem}[section]
\newtheorem{lemma}[theorem]{Lemma}
\newtheorem{claim}[theorem]{Claim}

\newcommand{\ADM}{\mathrm{ADM}}
\newcommand{\Area}{\mathrm{Area}}
\newcommand{\tr}{\mathrm{tr}}

\title{Blue/Red Team Analysis: Round 3\\
\large Attack on the Hull + Maximal Slice Approach}
\author{}
\date{December 2025}

\begin{document}
\maketitle

\begin{abstract}
We subject the ``Hull + Maximal Slice'' approach from PENROSE\_1973\_BREAKTHROUGH.tex to rigorous adversarial analysis. We identify \textbf{three critical gaps} and attempt to fix them.
\end{abstract}

\tableofcontents

%% ============================================================================
\section{Summary of the Proposed Proof}
%% ============================================================================

The proposed proof in PENROSE\_1973\_BREAKTHROUGH.tex claims:

\begin{theorem}[Claimed]
On a maximal Cauchy surface $(M, g, k)$ with DEC, for any trapped surface $\Sigma$:
\begin{equation}
    M_{\ADM} \ge \sqrt{\frac{A(\Sigma)}{16\pi}}
\end{equation}
\end{theorem}

\textbf{Proof strategy:}
\begin{enumerate}
    \item On maximal slice: $\tr k = 0 \Rightarrow R_g = |k|^2 + 2\mu \ge 0$.
    \item Construct outer-area minimizing hull $\hat{\Sigma}$ of trapped surface $\Sigma$.
    \item Claim: $A(\hat{\Sigma}) \le A(\Sigma)$ and $H_{\hat{\Sigma}} \ge 0$.
    \item Apply Riemannian Penrose inequality to $\hat{\Sigma}$.
    \item Conclude $M_{\ADM} \ge \sqrt{A(\hat{\Sigma})/(16\pi)} \ge \sqrt{A(\Sigma)/(16\pi)}$.
\end{enumerate}

%% ============================================================================
\section{Critical Gap \#1: Hull Definition and Existence}
%% ============================================================================

\begin{redattack}
\textbf{Attack:} The ``outer-area minimizing hull'' is not properly defined for a \textbf{non-minimal surface}.

The paper defines:
\begin{equation}
    \hat{\Omega} := \bigcap\{\Omega' : \Omega \subset \Omega', \; \partial\Omega' \text{ is area-minimizing among surfaces enclosing } \Omega\}
\end{equation}

\textbf{Problem 1:} The set of surfaces enclosing $\Sigma$ and minimizing area is \textbf{empty} if there is no global area minimizer.

\textbf{Problem 2:} Even if a minimizer exists, the intersection of regions is not well-defined (the boundary of the intersection may not be area-minimizing).

\textbf{Problem 3:} In the standard theory, the ``outer hull'' is defined for \textbf{minimal surfaces}, not arbitrary surfaces.
\end{redattack}

\begin{bluedefense}
\textbf{Corrected definition:} Use the \textbf{outward minimizing hull} from geometric measure theory.

\begin{definition}[Outward Minimizing Hull - Correct Version]
For a compact set $K \subset M$, define:
\begin{equation}
    \hat{K} := \{x \in M : \text{every surface separating } x \text{ from } \infty \text{ has area} \ge A^*\}
\end{equation}
where $A^* = \inf\{A(\Sigma') : \Sigma' \text{ encloses } K\}$.
\end{definition}

\textbf{Existence:} By the direct method in geometric measure theory, a minimizing sequence converges to a varifold, and regularity theory (Allard, Simon) shows the limit is smooth almost everywhere.

\textbf{Properties:}
\begin{enumerate}
    \item $\partial\hat{K}$ has $H \ge 0$ (by first variation)
    \item $A(\partial\hat{K}) \le A(\Sigma)$ if $\Sigma$ encloses $K$
    \item $\partial\hat{K}$ may be empty if no finite-area surface encloses $K$
\end{enumerate}

\textbf{For trapped surfaces:} Take $K = \Sigma$ (treating $\Sigma$ as a 2D subset). Then $\partial\hat{K}$ is the outer hull.
\end{bluedefense}

\begin{verdict}
\textbf{Status: FIXABLE}

The hull construction is valid with proper GMT definition. The key property $A(\hat{\Sigma}) \le A(\Sigma)$ follows from minimization.

\textbf{However:} New issue arises - see Critical Gap \#2.
\end{verdict}

%% ============================================================================
\section{Critical Gap \#2: Hull May Be Empty or Disconnected}
%% ============================================================================

\begin{redattack}
\textbf{Attack:} The outer hull $\hat{\Sigma}$ of a trapped surface may have \textbf{zero area} or be \textbf{disconnected}.

\textbf{Scenario 1 (Zero Area):} If the trapped surface $\Sigma$ is inside a ``neck'' region, the area-minimizing surface enclosing it might have arbitrarily small area (by pinching the neck).

\textbf{Scenario 2 (Disconnected):} The minimizer might have multiple components, some of which don't enclose $\Sigma$.

\textbf{Scenario 3 (Non-existence):} In non-compact manifolds, the infimum might not be achieved.
\end{redattack}

\begin{bluedefense}
\textbf{Defense against Scenario 1:}

On a maximal slice with $R_g \ge 0$, the isoperimetric inequality bounds area from below. A surface enclosing volume $V$ has:
\begin{equation}
    A \ge c \cdot V^{2/3}
\end{equation}
for some $c > 0$ depending on scalar curvature bounds.

If $\Sigma$ is trapped, it bounds a non-trivial region (by the Jordan-Brouwer theorem). Thus $\hat{\Sigma}$ has positive area.

\textbf{Defense against Scenario 2:}

Take the component of the minimizer that encloses $\Sigma$. This component alone has $A \le A(\Sigma)$ (otherwise, removing other components would decrease area).

\textbf{Defense against Scenario 3:}

On asymptotically flat manifolds, surfaces at infinity have area $\to \infty$, so minimizing sequences are confined to a compact region. Standard compactness gives existence.
\end{bluedefense}

\begin{verdict}
\textbf{Status: PARTIALLY FIXED}

Scenarios 2 and 3 are resolved. Scenario 1 requires more care:

\textbf{Remaining Issue:} The ``volume enclosed'' argument assumes $\Sigma$ is a topological sphere bounding a region. What if $\Sigma$ is a torus or higher genus?

\textbf{For genus 0:} Argument works.

\textbf{For higher genus:} Need separate analysis (see Critical Gap \#3).
\end{verdict}

%% ============================================================================
\section{Critical Gap \#3: Maximal Slice Through Trapped Surface}
%% ============================================================================

\begin{redattack}
\textbf{Attack:} Can we always find a maximal Cauchy surface containing a given trapped surface?

The proof assumes:
\begin{enumerate}
    \item Given any trapped surface $\Sigma$ in spacetime
    \item There exists a maximal Cauchy surface $\mathcal{C}_{\max}$ with $\Sigma \subset \mathcal{C}_{\max}$
\end{enumerate}

\textbf{This is FALSE in general.}

A maximal surface satisfies $\tr K = 0$ where $K$ is the extrinsic curvature of the slice in spacetime. This is a \textbf{global} constraint on the slice.

A given trapped surface $\Sigma$ is a 2D submanifold. There's no reason a global maximal slice must pass through $\Sigma$.

\textbf{Counterexample:} In Schwarzschild, the constant-$t$ slices are maximal, but these are just one family. A trapped surface near the singularity cannot lie on any constant-$t$ slice.
\end{redattack}

\begin{bluedefense}
\textbf{Partial defense:}

\textbf{Approach 1 (Deformation):} Given any Cauchy surface $\mathcal{C}$ containing $\Sigma$, deform it to a maximal surface. The deformed surface $\mathcal{C}_{\max}$ will contain a surface isotopic to $\Sigma$.

\textbf{Problem:} The deformed surface may not contain the original $\Sigma$, and the isotopic surface may not be trapped.

\textbf{Approach 2 (Comparison):} Use the spacetime constraint equations to relate $\Sigma$ on $\mathcal{C}$ to a surface on $\mathcal{C}_{\max}$.

\textbf{Problem:} This doesn't preserve the trapped condition.

\textbf{Approach 3 (Accept non-maximal):} Work on general Cauchy surfaces using the Jang equation to achieve $R \ge 0$.

\textbf{This is the standard approach}, which brings us back to the original area dominance problem.
\end{bluedefense}

\begin{verdict}
\textbf{Status: FATAL GAP}

The maximal slice approach \textbf{cannot be applied} to a general trapped surface because:
\begin{enumerate}
    \item There may be no maximal slice through the trapped surface
    \item Deforming to a maximal slice doesn't preserve the trapped condition
    \item The Penrose inequality is about \textbf{the original trapped surface}, not some deformed version
\end{enumerate}

\textbf{Conclusion:} The maximal slice approach works only for trapped surfaces that \textit{happen} to lie on a maximal slice, which is a measure-zero condition.
\end{verdict}

%% ============================================================================
\section{Critical Gap \#4: The IMCF Starting Condition}
%% ============================================================================

\begin{redattack}
\textbf{Attack:} Even if the hull exists with $H_{\hat{\Sigma}} \ge 0$, IMCF requires $H > 0$ strictly.

The inverse mean curvature flow:
\begin{equation}
    \frac{\partial x}{\partial t} = \frac{\nu}{H}
\end{equation}
is undefined where $H = 0$.

If $\hat{\Sigma}$ is minimal ($H = 0$), IMCF cannot start.
\end{redattack}

\begin{bluedefense}
\textbf{Defense:}

\textbf{Case 1:} If $\hat{\Sigma}$ is minimal, apply the Riemannian Penrose inequality directly (Bray or Huisken-Ilmanen). This gives:
\begin{equation}
    M_{\ADM} \ge \sqrt{\frac{A(\hat{\Sigma})}{16\pi}}
\end{equation}

Since $A(\hat{\Sigma}) \le A(\Sigma)$, we get the desired bound.

\textbf{Case 2:} If $H_{\hat{\Sigma}} > 0$ somewhere, use Huisken-Ilmanen's \textbf{weak IMCF}, which handles $H \ge 0$ with $H = 0$ on a set of measure zero.

\textbf{Case 3:} If $H_{\hat{\Sigma}} = 0$ everywhere but $\hat{\Sigma}$ is not a connected minimal surface, there's an issue. However, by GMT, the area-minimizing hull should be a connected minimal surface or have $H > 0$.
\end{bluedefense}

\begin{verdict}
\textbf{Status: FIXED}

The IMCF starting condition is not a real gap because:
\begin{enumerate}
    \item If hull is minimal: use RPI directly
    \item If hull has $H > 0$: use weak IMCF
    \item If hull has $H \ge 0$ with $H = 0$ on measure-zero set: weak IMCF handles this
\end{enumerate}
\end{verdict}

%% ============================================================================
\section{New Attack: The Riemannian vs. Spacetime Distinction}
%% ============================================================================

\begin{newattack}
\textbf{Fundamental Problem:} The Riemannian Penrose inequality applies to \textbf{minimal surfaces}, not to \textbf{MOTS} or \textbf{trapped surfaces}.

In the spacetime setting:
\begin{itemize}
    \item A trapped surface has $\theta^+ < 0, \theta^- < 0$
    \item On a maximal slice: $H = (\theta^+ + \theta^-)/2 < 0$
    \item So trapped surfaces have $H < 0$, not $H = 0$
\end{itemize}

The hull $\hat{\Sigma}$ has $H \ge 0$, but this is a \textbf{different surface} from the trapped surface $\Sigma$.

\textbf{The key inequality} $A(\hat{\Sigma}) \le A(\Sigma)$ is what makes this work.

\textbf{But wait:} On a maximal slice, the trapped surface has $H < 0$. The outer hull has $H \ge 0$. These are geometrically very different.

\textbf{Question:} Is it always true that the area-minimizing surface enclosing $\Sigma$ has area $\le A(\Sigma)$?
\end{newattack}

\begin{redattack}
\textbf{Critical Analysis of $A(\hat{\Sigma}) \le A(\Sigma)$:}

The claim is: the outer-area minimizing hull $\hat{\Sigma}$ satisfies $A(\hat{\Sigma}) \le A(\Sigma)$.

\textbf{This is only true if $\Sigma$ is one of the surfaces considered in the minimization.}

\textbf{Problem:} The minimization is over surfaces \textbf{enclosing} $\Sigma$, not surfaces equal to $\Sigma$.

If $\Sigma$ is a trapped surface (not area-minimizing), then:
\begin{itemize}
    \item $\Sigma$ itself is in the class of surfaces enclosing $\Sigma$ (vacuously, or as the boundary of the region bounded by $\Sigma$)
    \item The infimum over this class is $\le A(\Sigma)$
    \item But the infimum might be achieved by a surface \textbf{larger than} $\Sigma$ if we're measuring area incorrectly
\end{itemize}

\textbf{Wait:} If $\Sigma$ bounds a region $\Omega$, then $\Sigma$ is a surface enclosing $\Omega$. The infimum of areas of surfaces enclosing $\Omega$ is $\le A(\Sigma)$. So $A(\hat{\Sigma}) \le A(\Sigma)$. ✓

\textbf{The inequality is correct} as long as we define ``enclosing'' properly.
\end{redattack}

\begin{verdict}
\textbf{Status: VALID}

The inequality $A(\hat{\Sigma}) \le A(\Sigma)$ is correct by definition of infimum, as long as $\Sigma$ itself is among the surfaces enclosing the region it bounds.
\end{verdict}

%% ============================================================================
\section{Synthesis: What Survives?}
%% ============================================================================

\subsection{Summary Table}

\begin{center}
\begin{tabular}{@{}llcc@{}}
\toprule
\textbf{Gap} & \textbf{Issue} & \textbf{Red} & \textbf{Blue Fix} \\
\midrule
Hull definition & Not properly defined & Valid attack & \textcolor{green}{GMT definition} \\
Hull existence & May be empty/disconnected & Valid attack & \textcolor{green}{Compactness + connectivity} \\
Maximal slice & May not exist through $\Sigma$ & \textbf{FATAL} & \textcolor{red}{No fix} \\
IMCF starting & Needs $H > 0$ & Valid attack & \textcolor{green}{Case analysis} \\
Area inequality & $A(\hat{\Sigma}) \le A(\Sigma)$? & Attempted & \textcolor{green}{Valid by definition} \\
\bottomrule
\end{tabular}
\end{center}

\subsection{The Fatal Gap}

\begin{tcolorbox}[colback=red!10!white, colframe=red!75!black, title=\textbf{FATAL GAP}]
The proof in PENROSE\_1973\_BREAKTHROUGH.tex fails because:

\textbf{Not every trapped surface lies on a maximal Cauchy surface.}

The set of trapped surfaces lying on maximal slices has \textbf{measure zero} in the space of all trapped surfaces.

Penrose's 1973 conjecture is about \textbf{all} trapped surfaces, not just those on maximal slices.
\end{tcolorbox}

%% ============================================================================
\section{Salvage Attempt: The Jang Reduction}
%% ============================================================================

\begin{bluedefense}
\textbf{Can we combine the hull idea with the Jang approach?}

\textbf{Strategy:}
\begin{enumerate}
    \item Start with general initial data $(M, g, k)$, trapped surface $\Sigma_0$
    \item Solve Jang equation blowing up at outermost MOTS $\Sigma^*$
    \item On Jang manifold: $R_{\bar{g}} \ge 0$, $\Sigma^*$ is minimal
    \item Construct outer hull of $\Sigma_0$ in Jang manifold
    \item Apply RPI to the hull
\end{enumerate}

\textbf{Problem:} The original trapped surface $\Sigma_0$ may not embed nicely in the Jang manifold (the Jang surface blows up at $\Sigma^*$, not at $\Sigma_0$).

\textbf{Key question:} Is the area of $\Sigma_0$ preserved under the Jang embedding?

On the Jang surface, area is computed with metric $\bar{g}$. For surfaces away from the blow-up, $\bar{g} \approx g$, so areas are approximately preserved.

For the trapped surface $\Sigma_0$ (which is \textbf{inside} $\Sigma^*$), the Jang surface is smooth near $\Sigma_0$, so:
\begin{equation}
    A_{\bar{g}}(\Sigma_0) \approx A_g(\Sigma_0)
\end{equation}

\textbf{Then:}
\begin{enumerate}
    \item Take outer hull $\hat{\Sigma}_0$ of $\Sigma_0$ in $(M, \bar{g})$
    \item $A_{\bar{g}}(\hat{\Sigma}_0) \le A_{\bar{g}}(\Sigma_0) \approx A_g(\Sigma_0)$
    \item Apply RPI: $M_{\ADM}(\bar{g}) \ge \sqrt{A_{\bar{g}}(\hat{\Sigma}_0)/(16\pi)}$
    \item Since $M_{\ADM}(\bar{g}) = M_{\ADM}(g)$: done!
\end{enumerate}
\end{bluedefense}

\begin{redattack}
\textbf{Attack on the salvage:}

\textbf{Problem 1:} The Jang metric $\bar{g}$ is not the same as $g$. The hull $\hat{\Sigma}_0$ in $(M, \bar{g})$ may be different from the hull in $(M, g)$.

\textbf{Problem 2:} The statement ``$A_{\bar{g}}(\Sigma_0) \approx A_g(\Sigma_0)$'' is not rigorous. We need exact equality or a one-sided bound.

\textbf{Problem 3:} Even if $A_{\bar{g}}(\hat{\Sigma}_0) \le A_{\bar{g}}(\Sigma_0)$, we need $A_{\bar{g}}(\Sigma_0) \le A_g(\Sigma_0)$ to conclude $M \ge \sqrt{A_g(\Sigma_0)/(16\pi)}$.

\textbf{The conformal relationship:} On the Jang surface, $\bar{g}_{ij} = g_{ij} + \nabla_i f \nabla_j f$ where $f$ is the Jang solution. This gives:
\begin{equation}
    A_{\bar{g}}(\Sigma) \ge A_g(\Sigma)
\end{equation}
with equality iff $\nabla f \perp \Sigma$.

\textbf{This is the WRONG direction!} We get $A_{\bar{g}} \ge A_g$, but we need $A_{\bar{g}} \le A_g$ for the Penrose inequality.
\end{redattack}

\begin{verdict}
\textbf{Status: FAILED}

The Jang + Hull approach fails because:
\begin{enumerate}
    \item Jang metric has $A_{\bar{g}}(\Sigma) \ge A_g(\Sigma)$ (wrong direction)
    \item This would give $M \ge \sqrt{A_{\bar{g}}/(16\pi)} \ge \sqrt{A_g/(16\pi)}$... \\
    \textbf{Wait, this is the RIGHT direction!}
\end{enumerate}

\textbf{Re-analysis:}
\begin{align}
    M_{\ADM}(g) &= M_{\ADM}(\bar{g}) & \text{(Jang preserves mass)} \\
    &\ge \sqrt{\frac{A_{\bar{g}}(\hat{\Sigma}_0)}{16\pi}} & \text{(RPI for hull)} \\
    &\le \sqrt{\frac{A_{\bar{g}}(\Sigma_0)}{16\pi}} & \text{(hull has smaller area)} \\
    &\le \sqrt{\frac{A_g(\Sigma_0)}{16\pi}} & \text{(NEED THIS)}
\end{align}

The last step requires $A_{\bar{g}}(\Sigma_0) \le A_g(\Sigma_0)$, but we have $A_{\bar{g}} \ge A_g$.

\textbf{The argument fails.}
\end{verdict}

%% ============================================================================
\section{Alternative: Direct Hull in Original Metric}
%% ============================================================================

\begin{bluedefense}
\textbf{New approach:} Work entirely in the original metric $(M, g)$.

\textbf{Problem:} We don't have $R_g \ge 0$ on a general slice.

\textbf{Idea:} Use the constraint equations to bound the ``deficit'' from $R_g \ge 0$.

The Hamiltonian constraint:
\begin{equation}
    R_g = |k|^2 - (\tr k)^2 + 2\mu \ge 2\mu - (\tr k)^2
\end{equation}

With DEC: $\mu \ge |J| \ge 0$. But $(\tr k)^2$ can be large and negative.

\textbf{Geroch-type monotonicity with error:}

Under IMCF in $(M, g)$:
\begin{equation}
    \frac{d m_H}{dt} = \frac{1}{2}\sqrt{\frac{A}{16\pi}} \int_\Sigma \left(\frac{R_g}{2} + \text{other terms}\right) \frac{dA}{H^2}
\end{equation}

If $R_g$ has a negative part, $m_H$ might decrease, and the monotonicity fails.

\textbf{Conclusion:} Cannot use IMCF directly on general slices.
\end{bluedefense}

%% ============================================================================
\section{Final Assessment}
%% ============================================================================

\begin{tcolorbox}[colback=yellow!10!white, colframe=orange!75!black, title=\textbf{ROUND 3 CONCLUSION}]
The ``Hull + Maximal Slice'' approach in PENROSE\_1973\_BREAKTHROUGH.tex has:

\textbf{Valid components:}
\begin{enumerate}
    \item Hull construction (with proper GMT definition)
    \item Area inequality $A(\hat{\Sigma}) \le A(\Sigma)$
    \item IMCF from hull when $H > 0$
    \item RPI application when hull is minimal
\end{enumerate}

\textbf{Fatal flaw:}
\begin{enumerate}
    \item Cannot guarantee maximal slice through arbitrary trapped surface
\end{enumerate}

\textbf{Failed salvage attempts:}
\begin{enumerate}
    \item Jang + Hull: area inequality goes wrong direction
    \item Direct hull on general slice: no $R \ge 0$
\end{enumerate}

\textbf{Status of 1973 Conjecture:} Still open. The area dominance problem persists in all approaches.
\end{tcolorbox}

%% ============================================================================
\section{New Attack Vector: Spacetime Hull}
%% ============================================================================

\begin{newattack}
\textbf{Idea:} Instead of working on a Cauchy surface, work directly in spacetime.

Define the \textbf{spacetime hull} of a trapped surface $\Sigma$ as the outermost marginally outer trapped surface (MOTS) enclosing $\Sigma$ in the spacetime.

By the barrier method (Andersson-Mars-Simon), such a MOTS $\Sigma^*$ exists.

\textbf{Key question:} Is $A(\Sigma) \le A(\Sigma^*)$?

This is precisely the area dominance problem, restated in spacetime language.

\textbf{New insight:} In spacetime, we can consider \textbf{null hypersurfaces} emanating from $\Sigma$.

Along the outgoing null hypersurface from $\Sigma$:
\begin{equation}
    \frac{dA}{d\lambda} = \int_{\Sigma_\lambda} \theta^+ dA < 0
\end{equation}
(since $\theta^+ < 0$ for trapped surface).

So the area \textbf{decreases} along the outgoing null direction. This is the wrong direction to reach the MOTS.

Along the ingoing null hypersurface:
\begin{equation}
    \frac{dA}{d\lambda} = \int_{\Sigma_\lambda} \theta^- dA < 0
\end{equation}
Also decreasing!

\textbf{Conclusion:} Null flows from trapped surfaces always decrease area. There's no natural way to reach a MOTS with larger area.

This confirms the fundamental obstruction identified in Round 2.
\end{newattack}

%% ============================================================================
\section{Remaining Viable Paths}
%% ============================================================================

After Round 3, the remaining paths are:

\begin{enumerate}
    \item \textbf{Spinor bypass:} Prove $M \ge \sqrt{A(\Sigma)/(16\pi)}$ directly without using MOTS.
    
    \item \textbf{Capacity theory:} Develop $\theta$-capacity with sharp isoperimetric constant.
    
    \item \textbf{Accept conditional:} The theorem holds under (OM) assumption.
    
    \item \textbf{Physical assumption:} Use weak cosmic censorship to justify area dominance dynamically.
\end{enumerate}

\textbf{Most promising:} Path 4 --- use the full power of WCC to show that dynamical evolution increases horizon area past $A(\Sigma)$.

\end{document}
