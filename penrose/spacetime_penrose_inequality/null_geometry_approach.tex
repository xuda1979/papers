% NOVEL APPROACH: USING THE FULL NULL STRUCTURE
%
% Key insight: We have BOTH θ⁺ ≤ 0 AND θ⁻ < 0 on a trapped surface.
% The standard Jang approach only uses θ⁺ = 0 (MOTS condition).
% Can we exploit the FULL trapped condition?

\documentclass{article}
\usepackage{amsmath,amsthm,amssymb}
\newtheorem{theorem}{Theorem}
\newtheorem{lemma}{Lemma}
\newtheorem{proposition}{Proposition}
\newtheorem{corollary}{Corollary}
\newtheorem{conjecture}{Conjecture}
\newtheorem{remark}{Remark}
\newtheorem{definition}{Definition}
\newtheorem{claim}{Claim}
\newtheorem{idea}{Idea}
\newtheorem{observation}{Observation}

\begin{document}

\title{Exploiting the Full Trapped Condition:\\A Null Geometry Approach}
\author{Mathematical Exploration}
\date{\today}
\maketitle

\section{The Underutilized Structure}

\subsection{What the Jang Approach Uses}

The Jang equation with blow-up at a MOTS $\Sigma$ uses:
\begin{itemize}
    \item $\theta^+ = H + \tr_\Sigma k = 0$ (MOTS condition)
    \item Stability: $\lambda_1(L_\Sigma) \ge 0$
\end{itemize}

For a general trapped surface $\Sigma_0$:
\begin{itemize}
    \item $\theta^+ = H + \tr_{\Sigma_0} k \le 0$ (weaker than MOTS)
    \item $\theta^- = H - \tr_{\Sigma_0} k < 0$ (NOT used at all!)
\end{itemize}

\textbf{Key observation:} The condition $\theta^- < 0$ is completely ignored in the standard approach!

\subsection{What $\theta^- < 0$ Tells Us}

$\theta^- = H - \tr_\Sigma k < 0$ means:
\begin{equation}
    H < \tr_\Sigma k
\end{equation}

Combined with $\theta^+ \le 0$: $H \le -\tr_\Sigma k$

For the unfavorable case $\tr_\Sigma k < 0$:
\begin{align}
    H &< \tr_\Sigma k < 0 \\
    \theta^+ &= H + \tr_\Sigma k < 2\tr_\Sigma k < 0 \\
    \theta^- &= H - \tr_\Sigma k < 0
\end{align}

So both null expansions are strictly negative.

\section{A New Functional}

\subsection{The Standard Hawking Mass}

The Hawking mass of a surface $\Sigma$ is:
\begin{equation}
    M_H(\Sigma) = \sqrt{\frac{A(\Sigma)}{16\pi}}\left(1 - \frac{1}{16\pi}\int_\Sigma H^2 \, dA\right)
\end{equation}

This uses only the mean curvature $H$, not the null expansions directly.

\subsection{A New "Trapped Mass" Functional}

\begin{definition}[Null Expansion Mass]
For a surface $\Sigma$ with null expansions $\theta^\pm$, define:
\begin{equation}
    M_{\text{null}}(\Sigma) = \sqrt{\frac{A(\Sigma)}{16\pi}}\left(1 - \frac{1}{16\pi}\int_\Sigma \theta^+ \theta^- \, dA\right)
\end{equation}
\end{definition}

\begin{observation}
For a trapped surface: $\theta^+ \le 0$, $\theta^- < 0$, so $\theta^+ \theta^- \ge 0$.

Thus $M_{\text{null}}(\Sigma) \le \sqrt{A(\Sigma)/(16\pi)}$.
\end{observation}

\textbf{Question:} Is there a monotonicity formula for $M_{\text{null}}$?

\subsection{Relation to Hawking Mass}

Note that:
\begin{align}
    \theta^+ \theta^- &= (H + \tr_\Sigma k)(H - \tr_\Sigma k) \\
    &= H^2 - (\tr_\Sigma k)^2
\end{align}

So:
\begin{equation}
    M_{\text{null}}(\Sigma) = M_H(\Sigma) + \sqrt{\frac{A(\Sigma)}{16\pi}} \cdot \frac{1}{16\pi}\int_\Sigma (\tr_\Sigma k)^2 \, dA
\end{equation}

The null expansion mass is the Hawking mass plus a correction term from $(\tr_\Sigma k)^2 \ge 0$.

\textbf{Consequence:} $M_{\text{null}}(\Sigma) \ge M_H(\Sigma)$ always.

\section{Monotonicity Under Null Flow}

\subsection{Outgoing Null Flow}

Consider a family of surfaces $\{\Sigma_s\}$ generated by the outgoing null flow:
\begin{equation}
    \frac{\partial x^\mu}{\partial s} = \ell^{+\mu}
\end{equation}

The area evolves as:
\begin{equation}
    \frac{dA}{ds} = \int_{\Sigma_s} \theta^+ \, dA
\end{equation}

The null expansion evolves via Raychaudhuri:
\begin{equation}
    \frac{d\theta^+}{ds} = -\frac{1}{2}(\theta^+)^2 - |\sigma^+|^2 - R_{\mu\nu}\ell^+{}^\mu\ell^+{}^\nu \le -\frac{1}{2}(\theta^+)^2
\end{equation}
using NEC.

\subsection{Computing $dM_{\text{null}}/ds$}

Let's compute how $M_{\text{null}}$ evolves under the outgoing null flow.

\begin{align}
    \frac{d}{ds}\left(\int_{\Sigma_s} \theta^+\theta^- \, dA\right) &= \int_\Sigma \left(\frac{d\theta^+}{ds}\theta^- + \theta^+\frac{d\theta^-}{ds} + \theta^+\theta^- \cdot \theta^+\right) dA
\end{align}

We need $d\theta^-/ds$ along the \emph{outgoing} null flow.

\textbf{Commutation:} The null expansions don't commute simply. Let's think more carefully.

Under the outgoing null flow (generator $\ell^+$):
\begin{itemize}
    \item $\theta^+$ satisfies Raychaudhuri along $\ell^+$
    \item $\theta^-$ is the expansion of the \emph{other} null direction $\ell^-$
\end{itemize}

The evolution of $\theta^-$ along $\ell^+$ is given by the \emph{cross-focusing equation}:
\begin{equation}
    \ell^+(\theta^-) = -\frac{1}{2}\theta^+\theta^- - |\sigma^+|^2 + \ldots
\end{equation}

(The exact form depends on the twist and other geometric quantities.)

This is getting complicated. Let me try a different approach.

\section{A Spacetime Approach: Double Null Foliation}

\subsection{Setup}

In a spacetime $(N^{3+1}, \bar{g})$, consider a double null foliation with coordinates $(u, v, x^A)$ where:
\begin{itemize}
    \item $u = \text{const}$ are outgoing null hypersurfaces
    \item $v = \text{const}$ are ingoing null hypersurfaces
    \item $x^A$ are coordinates on the 2-spheres $\Sigma_{u,v}$
\end{itemize}

The null expansions are:
\begin{align}
    \theta^+ &= \partial_u \ln \sqrt{\det \gamma} \\
    \theta^- &= \partial_v \ln \sqrt{\det \gamma}
\end{align}
where $\gamma_{AB}$ is the induced metric on $\Sigma_{u,v}$.

\subsection{Area Evolution}

The area $A(u,v) = \int_{\Sigma_{u,v}} \sqrt{\det\gamma} \, d^2x$ satisfies:
\begin{align}
    \partial_u A &= \int_{\Sigma_{u,v}} \theta^+ \sqrt{\det\gamma} \, d^2x \\
    \partial_v A &= \int_{\Sigma_{u,v}} \theta^- \sqrt{\det\gamma} \, d^2x
\end{align}

For a trapped surface: $\partial_u A \le 0$ (area non-increasing outward) and $\partial_v A < 0$ (area decreasing inward).

\subsection{The Hawking Area Theorem}

In a spacetime satisfying NEC:
\begin{itemize}
    \item If $\theta^+ \le 0$ on $\Sigma_{u_0, v_0}$, then $\theta^+ \le 0$ for all $u \ge u_0$ (on the same ingoing null hypersurface $v = v_0$).
    \item Similarly for $\theta^-$ along outgoing null hypersurfaces.
\end{itemize}

\textbf{Consequence:} On a trapped surface, both null expansions remain non-positive in the future.

\subsection{Application to Initial Data}

An initial data set $(M, g, k)$ is a slice of spacetime. The trapped surface $\Sigma_0 \subset M$ has $\theta^\pm < 0$ as measured in the spacetime.

\textbf{Question:} Can we use the spacetime evolution to prove the Penrose inequality?

\textbf{Idea:} Evolve $\Sigma_0$ to the future along a null hypersurface until it reaches the event horizon $\mathcal{H}$. By the area theorem:
\[
A(\Sigma_0) \le A(\mathcal{H} \cap M) = A(\Sigma^*)
\]

\textbf{Problem:} This requires:
\begin{enumerate}
    \item Cosmic censorship (existence of event horizon)
    \item The null hypersurface from $\Sigma_0$ reaching $\mathcal{H}$
    \item The intersection $\mathcal{H} \cap M$ being comparable to the apparent horizon
\end{enumerate}

These are all unproven assumptions!

\section{A New Idea: The Bousso Bound Approach}

\subsection{The Bousso Entropy Bound}

Bousso's covariant entropy bound states: for any null hypersurface $\mathcal{N}$ with $\theta \le 0$:
\begin{equation}
    S(\mathcal{N}) \le \frac{A_{\text{initial}}}{4}
\end{equation}

This is a bound on entropy, not mass. But the idea of using null hypersurfaces with $\theta \le 0$ is relevant.

\subsection{An Energy Bound on Null Hypersurfaces}

Consider the outgoing null hypersurface $\mathcal{N}^+$ from $\Sigma_0$.

By the Raychaudhuri equation and NEC:
\begin{equation}
    \int_{\mathcal{N}^+} R_{\mu\nu}\ell^+{}^\mu\ell^+{}^\nu \, d\lambda \, dA \ge 0
\end{equation}

This bounds the matter content on $\mathcal{N}^+$.

\textbf{Connection to mass:} The Bondi mass at null infinity satisfies:
\begin{equation}
    M_{\text{Bondi}}(u) = M_{\text{ADM}} - \int_{-\infty}^u (\text{radiated energy}) \, du'
\end{equation}

For a trapped surface, can we bound the radiated energy?

\section{A Potentially Rigorous Approach: Modified Jang with Both Expansions}

\subsection{The Generalized Jang Equation}

The standard generalized Jang equation uses one expansion:
\begin{equation}
    \bar{g}^{ij}\left(\nabla_i \nabla_j f - k_{ij}\right) = 0
\end{equation}
which blows up at MOTS where $\theta^+ = 0$.

\textbf{Idea:} Consider a "double" Jang equation that incorporates both null directions:
\begin{equation}
    \bar{g}^{ij}\left(\nabla_i \nabla_j f - k_{ij}\right) + \alpha \cdot \bar{g}^{ij}\left(\nabla_i \nabla_j (-f) + k_{ij}\right) = 0
\end{equation}

This is:
\begin{equation}
    (1 - \alpha)\bar{g}^{ij}\nabla_i\nabla_j f - (1 + \alpha)\bar{g}^{ij}k_{ij} = 0
\end{equation}

For $\alpha = 1$: this becomes $-2\tr_g k = 0$, which is only satisfied when $\tr k = 0$.

This doesn't seem useful...

\subsection{Alternative: A Coupled System}

Consider a system of two functions $(f^+, f^-)$ satisfying:
\begin{align}
    \mathcal{J}^+[f^+] &:= \bar{g}^{ij}_{f^+}(\nabla_i\nabla_j f^+ - k_{ij}) = 0 \\
    \mathcal{J}^-[f^-] &:= \bar{g}^{ij}_{f^-}(\nabla_i\nabla_j f^- + k_{ij}) = 0
\end{align}

The first equation blows up at $\theta^+ = 0$ surfaces.
The second equation blows up at $\theta^- = 0$ surfaces.

For a trapped surface with both $\theta^\pm < 0$, neither equation blows up!

\textbf{Question:} Can we combine these equations to get useful curvature bounds?

\section{Conclusion: Current State of Exploration}

I've explored several directions:

\begin{enumerate}
    \item \textbf{Null expansion mass:} A new functional $M_{\text{null}}$ using $\theta^+\theta^-$. Needs monotonicity formula.
    
    \item \textbf{Double null foliation:} Spacetime approach using area theorem. Requires cosmic censorship.
    
    \item \textbf{Bousso bound:} Energy bounds on null hypersurfaces. Connection to mass unclear.
    
    \item \textbf{Coupled Jang system:} Use both $\theta^+$ and $\theta^-$ equations. Structure unclear.
\end{enumerate}

\textbf{The most promising direction:} The null expansion mass $M_{\text{null}}$ deserves further study. If we can prove monotonicity under some geometric flow, it could provide a new approach to the Penrose inequality.

\section{Next Steps}

\begin{enumerate}
    \item Compute the evolution of $M_{\text{null}}$ under IMCF or null flow.
    \item Investigate whether the NEC gives useful bounds on $M_{\text{null}}$.
    \item Look for examples where $M_{\text{null}}$ is computable.
    \item Check if $M_{\text{null}}(\Sigma) \to M_{\text{ADM}}$ as $\Sigma \to \infty$.
\end{enumerate}

\end{document}
