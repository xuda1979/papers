% THE PENROSE SURFACE CONJECTURE
%
% Among all trapped surfaces, find the one that maximizes Hawking mass.
% This might be the key to removing the favorable jump condition.

\documentclass[12pt]{article}
\usepackage{amsmath,amsthm,amssymb}
\usepackage{mathrsfs}
\newtheorem{theorem}{Theorem}
\newtheorem{lemma}{Lemma}
\newtheorem{proposition}{Proposition}
\newtheorem{corollary}{Corollary}
\newtheorem{conjecture}{Conjecture}
\newtheorem{remark}{Remark}
\newtheorem{definition}{Definition}
\newtheorem{problem}{Problem}
\newtheorem{claim}{Claim}

\begin{document}

\title{The Penrose Surface Conjecture}
\author{Mathematical Development}
\date{\today}
\maketitle

\section{Setup and Definitions}

Let $(M, g, k)$ be asymptotically flat initial data satisfying the dominant 
energy condition (DEC). Let $\mathcal{T}$ be the trapped region with outermost 
MOTS boundary $\Sigma^* = \partial\mathcal{T}$.

\begin{definition}[Hawking Mass]
For a 2-surface $\Sigma$:
\[
M_H[\Sigma] = \sqrt{\frac{A(\Sigma)}{16\pi}}\left(1 - \frac{1}{16\pi}\int_\Sigma H^2 \, dA\right)
\]
where $H$ is the mean curvature.
\end{definition}

\begin{definition}[Penrose Mass]
The "naive" Penrose mass:
\[
M_P[\Sigma] = \sqrt{\frac{A(\Sigma)}{16\pi}}
\]
\end{definition}

Note: $M_H \le M_P$ always, with equality iff $H \equiv 0$ (minimal surface).

\section{The Penrose Surface}

\begin{definition}[Penrose Surface]
A \textbf{Penrose surface} $\Sigma_P$ is a trapped surface that locally maximizes 
the Hawking mass $M_H[\Sigma]$ among all nearby trapped surfaces.
\end{definition}

\begin{problem}[Global Penrose Surface]
Find:
\[
M_H^* = \sup\{ M_H[\Sigma] : \Sigma \subset \mathcal{T} \text{ is trapped} \}
\]
and a surface $\Sigma_P$ achieving this supremum (if it exists).
\end{problem}

\section{Variational Characterization}

\subsection{The Constraint}

A surface $\Sigma$ is trapped iff:
\begin{align}
    \theta^+ &= H + \tr_\Sigma k \le 0 \\
    \theta^- &= H - \tr_\Sigma k < 0
\end{align}

Equivalently: $H \le -\tr_\Sigma k$ and $H < \tr_\Sigma k$.

\subsection{Maximizing $M_H$ Under Constraint}

We want to maximize:
\[
M_H[\Sigma] = \sqrt{\frac{A}{16\pi}}\left(1 - \frac{1}{16\pi}\int_\Sigma H^2 \, dA\right)
\]

subject to $\theta^+ \le 0$.

\textbf{Observation}: $M_H$ increases when:
\begin{itemize}
    \item $A$ increases (larger area $\Rightarrow$ larger mass)
    \item $\int H^2$ decreases (smaller $|H|$ $\Rightarrow$ larger mass)
\end{itemize}

These can be in tension! Larger surfaces might have larger $|H|$.

\subsection{First-Order Conditions}

Vary $\Sigma$ by $\phi\nu$ (normal variation). The variation of Hawking mass is:
\[
\delta_\phi M_H = \frac{1}{2\sqrt{A/(16\pi)}}\left[\delta_\phi A - \frac{1}{16\pi}\delta_\phi\int H^2\right] + \ldots
\]

where:
\begin{align}
    \delta_\phi A &= \int_\Sigma \phi H \, dA \\
    \delta_\phi \int H^2 &= \int_\Sigma \phi\left[2H(\Delta\phi + |h|^2\phi + \mathrm{Ric}(\nu,\nu)\phi) + 2\nabla_\Sigma H \cdot \nabla_\Sigma\phi\right]
\end{align}

Actually, let me be more careful.

\subsection{Detailed Variation}

Under a normal variation $\Sigma \to \Sigma_\epsilon$ with speed $\phi$:

\textbf{Area}:
\[
\frac{d}{d\epsilon}\Big|_0 A(\Sigma_\epsilon) = \int_{\Sigma_0} \phi H \, dA
\]

\textbf{Mean curvature}:
\[
\frac{d}{d\epsilon}\Big|_0 H = \Delta_\Sigma \phi + (|h|^2 + \mathrm{Ric}(\nu,\nu))\phi
\]
where $h$ is the second fundamental form and $|h|^2 = H^2/2 + \kappa$ for some curvature term.

\textbf{$L^2$ norm of $H$}:
\begin{align}
    \frac{d}{d\epsilon}\Big|_0 \int_\Sigma H^2 \, dA &= \int_\Sigma \left[2H\frac{dH}{d\epsilon} + H^2 \cdot H\phi\right] dA \\
    &= \int_\Sigma \left[2H(\Delta\phi + |h|^2\phi + \mathrm{Ric}(\nu,\nu)\phi) + H^3\phi\right] dA
\end{align}

\subsection{The Euler-Lagrange Equation}

For an unconstrained critical point of $M_H$:

Setting $\delta_\phi M_H = 0$ for all $\phi$ gives a 4th-order elliptic PDE in general.

For a CONSTRAINED critical point (maximizing $M_H$ subject to $\theta^+ \le 0$), 
we have:
\[
\delta_\phi M_H = \lambda \cdot \delta_\phi \theta^+
\]
for some Lagrange multiplier $\lambda$ where the constraint is active ($\theta^+ = 0$).

\textbf{Conclusion}: The Penrose surface is either:
\begin{enumerate}
    \item A MOTS ($\theta^+ = 0$) where $M_H$ is maximized among MOTS
    \item An interior trapped surface where $M_H$ is locally maximal without constraint
\end{enumerate}

\section{The MOTS Case}

\subsection{Hawking Mass of a MOTS}

For a MOTS, $H = -\tr_\Sigma k$, so:
\[
M_H[\Sigma^*] = \sqrt{\frac{A(\Sigma^*)}{16\pi}}\left(1 - \frac{(\tr k)^2 A}{16\pi}\right)
\]

where we abbreviate $\tr k = \frac{1}{A}\int_\Sigma (\tr_\Sigma k) \, dA$ or consider 
pointwise bounds.

\subsection{When is MOTS the Penrose Surface?}

If $\tr_\Sigma k = 0$ on $\Sigma^*$, then $M_H[\Sigma^*] = \sqrt{A/(16\pi)} = M_P[\Sigma^*]$.

This is the maximum possible Hawking mass for surfaces of that area!

\textbf{Favorable case}: $\tr_\Sigma k \ge 0$ means $H \le 0$ on the MOTS.
This makes the MOTS a good candidate for Penrose surface.

\textbf{Unfavorable case}: $\tr_\Sigma k < 0$ means $H > 0$ on the MOTS.
The Hawking mass is REDUCED by the $H^2$ term.
There might be better surfaces inside!

\section{Searching for Better Surfaces}

\subsection{The Competition}

Consider two surfaces:
\begin{itemize}
    \item $\Sigma^*$: the outermost MOTS with area $A^*$ and $H^* = -\tr k^* < 0$ (favorable) or $H^* > 0$ (unfavorable)
    \item $\Sigma_0$: an inner trapped surface with area $A_0$ and mean curvature $H_0$
\end{itemize}

$M_H[\Sigma_0] > M_H[\Sigma^*]$ when:
\[
\sqrt{A_0}\left(1 - \frac{\langle H_0^2 \rangle A_0}{16\pi}\right) > \sqrt{A^*}\left(1 - \frac{\langle H^{*2} \rangle A^*}{16\pi}\right)
\]

\subsection{Small Mean Curvature Surfaces}

If we can find a trapped surface $\Sigma_0$ with:
\begin{itemize}
    \item Area $A_0$ comparable to $A^*$
    \item Mean curvature $|H_0| \ll |H^*|$
\end{itemize}

then $M_H[\Sigma_0]$ could exceed $M_H[\Sigma^*]$.

\subsection{Nearly Minimal Trapped Surfaces}

A surface with $H \approx 0$ but $\tr_\Sigma k < 0$ (so $\theta^+ \approx \tr_\Sigma k < 0$, 
$\theta^- \approx -\tr_\Sigma k > 0$).

Wait, $\theta^- > 0$ means NOT trapped!

So for a trapped surface ($\theta^- < 0$), we need $H < \tr_\Sigma k < 0$.

This means $H < 0$ for trapped surfaces with $\tr_\Sigma k < 0$.

\textbf{But wait}: I had the signs confused. Let me redo.

\subsection{Correct Sign Analysis}

For a surface $\Sigma$ with outward normal $\nu$ and future normal $n$:
\begin{align}
    \ell^+ &= n + \nu \quad \text{(outgoing null)} \\
    \ell^- &= n - \nu \quad \text{(ingoing null)}
\end{align}

The null expansions:
\begin{align}
    \theta^+ &= \div_\Sigma \ell^+ = H + P \\
    \theta^- &= \div_\Sigma \ell^- = -H + P
\end{align}

where $P = -\tr_\Sigma k$ (note the sign convention).

Actually, conventions vary. Let me use:
\begin{align}
    \theta^+ &= H + \tr_\Sigma k \\
    \theta^- &= H - \tr_\Sigma k
\end{align}

Trapped condition: $\theta^+ \le 0$, $\theta^- < 0$.

\textbf{Case 1}: $\tr_\Sigma k = 0$ (time-symmetric)

Then $\theta^+ = \theta^- = H$, so trapped means $H < 0$.

\textbf{Case 2}: $\tr_\Sigma k > 0$ (favorable)

$\theta^+ = H + \tr k$, so $H < -\tr k < 0$ is required for trapping.

\textbf{Case 3}: $\tr_\Sigma k < 0$ (unfavorable)

$\theta^+ = H + \tr k < 0$ means $H < -\tr k = |\tr k| > 0$ is allowed!
$\theta^- = H - \tr k = H + |\tr k| < 0$ means $H < -|\tr k| < 0$.

So in the unfavorable case, we need $H < -|\tr k| < 0$, meaning $H$ is even MORE negative.

\textbf{I had the signs backwards earlier!}

\subsection{Revised Analysis}

In the unfavorable case ($\tr k < 0$):
- $\theta^+ = H + \tr k = H - |\tr k| \le 0$ requires $H \le |\tr k|$
- $\theta^- = H - \tr k = H + |\tr k| < 0$ requires $H < -|\tr k|$

Since $|\tr k| > 0$, the second condition $H < -|\tr k|$ is stronger.
So trapped surfaces with $\tr k < 0$ have $H < -|\tr k| < 0$.

For a MOTS ($\theta^+ = 0$): $H = -\tr k = |\tr k| > 0$.

\textbf{So the MOTS has POSITIVE mean curvature in the unfavorable case!}

But inner trapped surfaces have $H < -|\tr k| < 0$ (NEGATIVE mean curvature).

\subsection{Hawking Mass Comparison}

MOTS: $H^* = |\tr k| > 0$, so $M_H[\Sigma^*] < M_P[\Sigma^*]$.

Inner trapped surface: $H < 0$, could have $|H|$ small if close to minimal.

If we find an inner surface with $H$ small (close to zero) and area close to $A^*$:
\[
M_H[\text{inner}] \approx M_P[\text{inner}] = \sqrt{A_{\text{inner}}/(16\pi)}
\]

versus
\[
M_H[\text{MOTS}] = \sqrt{A^*/(16\pi)}(1 - \epsilon) \quad \text{for some } \epsilon > 0
\]

\textbf{Key question}: Can $A_{\text{inner}}$ be larger than $(1-\epsilon)A^*$?

\section{The Area Comparison Problem}

\subsection{Formulation}

Given:
\begin{itemize}
    \item Outermost MOTS $\Sigma^*$ with area $A^*$
    \item $\tr_{\Sigma^*} k < 0$ (unfavorable)
\end{itemize}

Question: What is $\sup\{A(\Sigma) : \Sigma \subset \mathcal{T}, \text{ trapped}\}$?

Is it $A^*$, or can it be larger?

\subsection{A Heuristic}

The outermost MOTS is the boundary of the trapped region.
A surface inside the trapped region has "room to grow" in some directions.

In the unfavorable case, the MOTS has $H > 0$, meaning it's convex (curving away from the trapped region).

An inner surface could "bulge" into the concave parts and have larger area!

\subsection{A Precise Statement}

\begin{conjecture}[Area Enhancement]
In the unfavorable case ($\tr_{\Sigma^*} k < 0$), there exist trapped surfaces 
$\Sigma \subset \mathcal{T}$ with area:
\[
A(\Sigma) > A(\Sigma^*)
\]
\end{conjecture}

If true, this explains why $A(\Sigma_0) > A(\Sigma^*)$ can happen.

\section{The Main Conjecture}

\begin{conjecture}[Penrose Surface Theorem]
Let $(M, g, k)$ satisfy DEC with trapped region $\mathcal{T}$.
Define the \textbf{optimal Hawking mass}:
\[
M_H^* = \sup\{M_H[\Sigma] : \Sigma \subset \mathcal{T}, \theta^\pm \le 0\}
\]

Then:
\begin{enumerate}
    \item The supremum is achieved by some surface $\Sigma_P$ (the Penrose surface)
    \item $M_{\mathrm{ADM}} \ge M_H^*$
    \item $M_H^* \ge M_P[\Sigma_0] = \sqrt{A(\Sigma_0)/(16\pi)}$ for any trapped $\Sigma_0$
\end{enumerate}

Together, these give the Penrose inequality without sign conditions!
\end{conjecture}

\subsection{Evidence}

Part (2) is the AMO inequality: $M_{\mathrm{ADM}} \ge M_H[\Sigma^*]$ for the outermost MOTS.

If $\Sigma_P = \Sigma^*$ (MOTS is optimal), then $M_H^* = M_H[\Sigma^*]$.

Part (3) is the key new ingredient. It says the optimal Hawking mass DOMINATES 
the Penrose mass of any trapped surface.

\subsection{Why Part (3) Might Be True}

The Hawking mass satisfies:
\[
M_H[\Sigma] = M_P[\Sigma]\left(1 - \frac{\langle H^2 \rangle A}{16\pi}\right)
\]

So $M_H \ge M_P$ requires $H = 0$ (minimal).

But part (3) says: $M_H^* \ge M_P[\Sigma_0]$ for a DIFFERENT surface!

This could hold if $\Sigma_P$ has:
\begin{itemize}
    \item Area $A_P \ge A_0 (1 + \epsilon)$ for some $\epsilon > 0$
    \item Small enough $\langle H_P^2 \rangle$ to compensate
\end{itemize}

\section{Next Steps}

\begin{enumerate}
    \item Characterize the Penrose surface $\Sigma_P$ geometrically
    \item Prove existence of $\Sigma_P$ (compactness arguments)
    \item Show $M_H[\Sigma_P] \ge M_P[\Sigma_0]$ using geometric comparison
    \item Connect $M_H[\Sigma_P]$ to $M_{\mathrm{ADM}}$
\end{enumerate}

This approach avoids the Jang equation entirely and works directly with trapped 
surfaces and Hawking mass!

\end{document}
