% =========================================================================
%     EXPLICIT CALCULATIONS IN SCHWARZSCHILD
%
%     Testing the θ⁺-flow and MOTS analysis
%
%     Author: Da Xu
%     Date: December 2025
% =========================================================================

\documentclass[12pt]{article}
\usepackage{amsmath,amsthm,amssymb}
\usepackage{mathrsfs}
\usepackage{tcolorbox}

\theoremstyle{plain}
\newtheorem{theorem}{Theorem}[section]
\newtheorem{lemma}[theorem]{Lemma}
\newtheorem{proposition}[theorem]{Proposition}
\newtheorem{corollary}[theorem]{Corollary}

\theoremstyle{definition}
\newtheorem{definition}[theorem]{Definition}
\newtheorem{example}[theorem]{Example}
\newtheorem{calculation}[theorem]{Calculation}

\newcommand{\ADM}{\mathrm{ADM}}
\newcommand{\tr}{\mathrm{tr}}

\title{\textbf{Explicit Calculations in Schwarzschild Spacetime}}
\author{Da Xu}
\date{December 2025}

\begin{document}
\maketitle

\section{Schwarzschild in Various Coordinates}

\subsection{Standard Coordinates}

\[
    ds^2 = -\left(1 - \frac{2M}{r}\right)dt^2 + \left(1 - \frac{2M}{r}\right)^{-1}dr^2 + r^2 d\Omega^2
\]

The $t = $ const slice has $k = 0$ (time-symmetric).

For our purposes, this is the "favorable" case: all trapped surfaces satisfy Penrose by Riemannian methods.

\subsection{Painlevé-Gullstrand (PG) Coordinates}

\[
    ds^2 = -d\tau^2 + \left(dr + \sqrt{\frac{2M}{r}}d\tau\right)^2 + r^2 d\Omega^2
\]

Expanding:
\[
    ds^2 = -\left(1 - \frac{2M}{r}\right)d\tau^2 + 2\sqrt{\frac{2M}{r}}d\tau dr + dr^2 + r^2 d\Omega^2
\]

The $\tau = $ const slice is spatially FLAT with non-zero $k$.

\subsection{Induced Metric on $\tau = $ const}

\[
    g_{ij}dx^i dx^j = dr^2 + r^2(d\theta^2 + \sin^2\theta \, d\phi^2) = dr^2 + r^2 d\Omega^2
\]

This is the flat Euclidean metric!

\subsection{Extrinsic Curvature}

The unit normal to $\tau = $ const:
\[
    n^\mu = \frac{1}{\sqrt{1 - 2M/r}}\left(\partial_\tau - \sqrt{\frac{2M}{r}}\partial_r\right)
\]

Wait, let me be more careful. In PG coordinates, the metric is:
\[
    g_{\mu\nu} = \begin{pmatrix}
        -(1 - 2M/r) & \sqrt{2M/r} & 0 & 0 \\
        \sqrt{2M/r} & 1 & 0 & 0 \\
        0 & 0 & r^2 & 0 \\
        0 & 0 & 0 & r^2\sin^2\theta
    \end{pmatrix}
\]

The inverse:
\[
    g^{\mu\nu} = \begin{pmatrix}
        -1 & \sqrt{2M/r} & 0 & 0 \\
        \sqrt{2M/r} & 1 - 2M/r & 0 & 0 \\
        0 & 0 & 1/r^2 & 0 \\
        0 & 0 & 0 & 1/(r^2\sin^2\theta)
    \end{pmatrix}
\]

The normal to $\tau = $ const is proportional to $d\tau$, i.e., $n_\mu \propto (1, 0, 0, 0)$.

Normalized:
\[
    n_\mu = \frac{(1, 0, 0, 0)}{\sqrt{-g^{\tau\tau}}} = (1, 0, 0, 0)
\]
since $g^{\tau\tau} = -1$.

So $n^\mu = g^{\mu\nu}n_\nu = g^{\mu\tau} = (-1, \sqrt{2M/r}, 0, 0)$.

\subsection{Extrinsic Curvature Calculation}

\[
    k_{ij} = -\nabla_i n_j = -\frac{1}{2}(\partial_\tau g_{ij})|_{\text{projected}}
\]

For the PG metric, $\partial_\tau g_{ij} = 0$ (the spatial metric is independent of $\tau$).

But the extrinsic curvature is:
\[
    k_{ij} = \frac{1}{2}\mathcal{L}_n g_{ij}
\]

Using $n^\mu = (-1, \sqrt{2M/r}, 0, 0)$:

Let's compute more carefully. The extrinsic curvature is:
\[
    k_{ij} = -\frac{1}{2N}(\partial_t \gamma_{ij} - D_i N_j - D_j N_i)
\]
where $N$ is the lapse and $N_i$ is the shift.

In PG: the lapse is $N = 1$ and the shift is $N^i = (\sqrt{2M/r}, 0, 0)$.

So:
\[
    k_{ij} = -\frac{1}{2}(\partial_\tau \gamma_{ij} - D_i N_j - D_j N_i)
\]

Since $\gamma_{ij}$ is flat and independent of $\tau$:
\[
    k_{ij} = \frac{1}{2}(D_i N_j + D_j N_i)
\]

With $N_j = \gamma_{jk}N^k$. In flat coordinates: $N_r = \sqrt{2M/r}$, $N_\theta = N_\phi = 0$.

\[
    k_{rr} = D_r N_r = \partial_r\left(\sqrt{\frac{2M}{r}}\right) = \sqrt{2M} \cdot \left(-\frac{1}{2r^{3/2}}\right) = -\frac{\sqrt{2M}}{2r^{3/2}} = -\frac{1}{2}\sqrt{\frac{2M}{r^3}}
\]

For $k_{\theta\theta}$:
\[
    k_{\theta\theta} = D_\theta N_\theta + \Gamma^\lambda_{\theta\theta}N_\lambda = 0 + \Gamma^r_{\theta\theta}\sqrt{\frac{2M}{r}} = (-r)\sqrt{\frac{2M}{r}} = -\sqrt{2Mr}
\]

Similarly: $k_{\phi\phi} = -\sqrt{2Mr}\sin^2\theta$.

\subsection{Trace of k}

\[
    \tr k = g^{ij}k_{ij} = k_{rr} + \frac{k_{\theta\theta}}{r^2} + \frac{k_{\phi\phi}}{r^2\sin^2\theta}
\]
\[
    = -\frac{1}{2}\sqrt{\frac{2M}{r^3}} - \frac{\sqrt{2Mr}}{r^2} - \frac{\sqrt{2Mr}}{r^2}
\]
\[
    = -\frac{1}{2}\sqrt{\frac{2M}{r^3}} - 2\sqrt{\frac{2M}{r^3}} = -\frac{5}{2}\sqrt{\frac{2M}{r^3}}
\]

Hmm, let me double-check. Actually:
\[
    \frac{\sqrt{2Mr}}{r^2} = \sqrt{\frac{2M}{r^3}}
\]

So:
\[
    \tr k = -\frac{1}{2}\sqrt{\frac{2M}{r^3}} - 2\sqrt{\frac{2M}{r^3}} = -\frac{5}{2}\sqrt{\frac{2M}{r^3}}
\]

Wait, that doesn't look right either. Let me recalculate $k_{\theta\theta}$.

Actually, $\Gamma^r_{\theta\theta} = -r$ in flat coordinates. So:
\[
    k_{\theta\theta} = D_\theta N_\theta - \Gamma^r_{\theta\theta}N_r = 0 - (-r)\sqrt{\frac{2M}{r}} = r\sqrt{\frac{2M}{r}} = \sqrt{2Mr}
\]

Wait, I had a sign error. Let me use the formula:
\[
    k_{ij} = \frac{1}{2}(D_i N_j + D_j N_i) = \frac{1}{2}(\partial_i N_j + \partial_j N_i - 2\Gamma^k_{ij}N_k)
\]

For $k_{\theta\theta}$:
\[
    k_{\theta\theta} = \frac{1}{2}(2\partial_\theta N_\theta - 2\Gamma^k_{\theta\theta}N_k) = -\Gamma^r_{\theta\theta}N_r = -(-r)\sqrt{\frac{2M}{r}} = \sqrt{2Mr}
\]

Similarly: $k_{\phi\phi} = -\Gamma^r_{\phi\phi}N_r = -(-r\sin^2\theta)\sqrt{\frac{2M}{r}} = \sqrt{2Mr}\sin^2\theta$.

Now:
\[
    \tr k = k_{rr} + \frac{k_{\theta\theta}}{r^2} + \frac{k_{\phi\phi}}{r^2\sin^2\theta}
    = -\frac{1}{2}\sqrt{\frac{2M}{r^3}} + \frac{\sqrt{2Mr}}{r^2} + \frac{\sqrt{2Mr}}{r^2}
\]
\[
    = -\frac{1}{2}\sqrt{\frac{2M}{r^3}} + 2\sqrt{\frac{2M}{r^3}} = \frac{3}{2}\sqrt{\frac{2M}{r^3}}
\]

\textbf{So $\tr k > 0$ in PG coordinates!}

\section{Spheres in PG Slice}

\subsection{A Sphere at Radius $r = r_0$}

In the $\tau = $ const slice, consider the sphere $\{r = r_0\}$.

\textbf{Area:} $A = 4\pi r_0^2$.

\textbf{Mean curvature $H$:} This is the trace of the second fundamental form of the sphere in the flat 3-metric.
\[
    H = \frac{2}{r_0}
\]
(outward normal, positive for convex)

\textbf{Trace $\tr_\Sigma k$:} This is the restriction of $k$ to the sphere, traced.

The tangent directions to the sphere are $\theta$ and $\phi$. The induced metric on the sphere is $r_0^2 d\Omega^2$.

\[
    \tr_\Sigma k = \gamma^{ab}k_{ab}|_\Sigma = \frac{k_{\theta\theta}}{r_0^2} + \frac{k_{\phi\phi}}{r_0^2\sin^2\theta}
    = \frac{\sqrt{2Mr_0}}{r_0^2} + \frac{\sqrt{2Mr_0}}{r_0^2} = \frac{2\sqrt{2Mr_0}}{r_0^2} = 2\sqrt{\frac{2M}{r_0^3}}
\]

\subsection{Null Expansions}

\[
    \theta^+ = H + \tr_\Sigma k = \frac{2}{r_0} + 2\sqrt{\frac{2M}{r_0^3}}
\]

\[
    \theta^- = H - \tr_\Sigma k = \frac{2}{r_0} - 2\sqrt{\frac{2M}{r_0^3}}
\]

\subsection{Trapped Condition}

For trapped: $\theta^+ \leq 0$ and $\theta^- < 0$.

$\theta^- < 0$:
\[
    \frac{2}{r_0} < 2\sqrt{\frac{2M}{r_0^3}} \Rightarrow \frac{1}{r_0} < \sqrt{\frac{2M}{r_0^3}} \Rightarrow \frac{1}{r_0^2} < \frac{2M}{r_0^3} \Rightarrow r_0 < 2M
\]

$\theta^+ \leq 0$:
\[
    \frac{2}{r_0} + 2\sqrt{\frac{2M}{r_0^3}} \leq 0
\]

But both terms are positive! So $\theta^+ > 0$ always in PG slice.

\textbf{CONCLUSION:} Spheres in PG slice are NOT trapped (for the outgoing direction)!

They have $\theta^+ > 0$ but $\theta^- < 0$ for $r_0 < 2M$.

\subsection{Interpretation}

In PG coordinates, the "outgoing" direction is not the same as in standard coordinates.

The surface has $\theta^+ > 0$ (not trapped in outgoing sense).

But $\theta^- < 0$ for $r < 2M$ (trapped in ingoing sense).

\section{Eddington-Finkelstein Coordinates}

\subsection{Ingoing EF}

\[
    ds^2 = -\left(1 - \frac{2M}{r}\right)dv^2 + 2dvdr + r^2 d\Omega^2
\]

The $v = $ const slice has $k \neq 0$.

\subsection{Induced Metric}

On $v = $ const:
\[
    ds^2|_v = dr^2 + r^2 d\Omega^2
\]

This is flat! Same as PG.

\subsection{Extrinsic Curvature}

The normal to $v = $ const... need to compute.

The metric in $(v, r, \theta, \phi)$:
\[
    g_{\mu\nu} = \begin{pmatrix}
        -(1-2M/r) & 1 & 0 & 0 \\
        1 & 0 & 0 & 0 \\
        0 & 0 & r^2 & 0 \\
        0 & 0 & 0 & r^2\sin^2\theta
    \end{pmatrix}
\]

Inverse:
\[
    g^{\mu\nu} = \begin{pmatrix}
        0 & 1 & 0 & 0 \\
        1 & 1-2M/r & 0 & 0 \\
        0 & 0 & 1/r^2 & 0 \\
        0 & 0 & 0 & 1/(r^2\sin^2\theta)
    \end{pmatrix}
\]

Normal to $v = $ const: $n_\mu \propto \partial_\mu v = (1, 0, 0, 0)$.

$n^\mu = g^{\mu v}n_v = g^{\mu v} = (0, 1, 0, 0)$.

So the normal is purely in the $r$-direction: $n^\mu = (0, 1, 0, 0)$.

This is a NULL direction! The $v = $ const surfaces are null hypersurfaces.

They are NOT spacelike, so we can't use them as initial data slices.

\section{The Correct Slicing}

\subsection{Need Spacelike Slices with $\tr_\Sigma k < 0$}

PG slices have $\tr_\Sigma k > 0$ (favorable for standard methods).

To get unfavorable case, we need slices with $\tr_\Sigma k < 0$.

\subsection{Boosted Slices}

Consider a slice that is "tilted" toward the past in the exterior and toward the future in the interior.

This can give $\tr k < 0$.

\subsection{The Kerr-Schild Slicing}

In Kerr-Schild form:
\[
    g_{\mu\nu} = \eta_{\mu\nu} + \frac{2M}{r}\ell_\mu\ell_\nu
\]
where $\ell$ is a null vector.

The $t = $ const slice in these coordinates has non-trivial $k$.

\section{Testing Penrose Inequality}

\subsection{In PG Slice}

For a sphere at $r_0$:
\begin{itemize}
    \item $A = 4\pi r_0^2$
    \item $M_{\ADM} = M$
\end{itemize}

Penrose: $M \geq \sqrt{A/(16\pi)} = \sqrt{4\pi r_0^2/(16\pi)} = r_0/2$.

Since the Schwarzschild radius is $2M$, surfaces at $r_0 < 2M$ give:
\[
    M \geq \frac{r_0}{2}
\]

This is satisfied since $M = \frac{2M}{2} > \frac{r_0}{2}$ for $r_0 < 2M$. ✓

\subsection{Trapped Surfaces}

The trapped surfaces (both $\theta^\pm < 0$) in Schwarzschild occur inside the horizon.

In standard coordinates: at $r < 2M$ with $t$-slicing, we have $k = 0$ and surfaces are trapped.

In PG coordinates: spheres have $\theta^+ > 0$ (not trapped in future sense).

\textbf{To find truly trapped surfaces, need different embeddings!}

\section{Conclusion}

\begin{tcolorbox}[colback=blue!20, colframe=blue!75!black]
\textbf{KEY FINDINGS:}

1. In PG coordinates, the $\tau = $ const slice has $\tr k > 0$ (favorable).

2. Spheres in PG have $\theta^+ > 0$, so NOT trapped in the future sense.

3. To test the unfavorable case ($\tr_\Sigma k < 0$), need different slicings.

4. The Penrose inequality is satisfied in all computed cases.

\textbf{NEXT STEP:}

Construct explicit trapped surfaces with $\tr_\Sigma k < 0$ in Schwarzschild and verify the $\theta^+$-flow and Penrose inequality.
\end{tcolorbox}

\end{document}
