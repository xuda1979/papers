%% BARTNIK_MTHETA_CONNECTION.tex
%%
%% PROOF ATTEMPT: Bartnik Mass ≥ θ⁺-Weighted Mass
%%
%% Key Link: If m_B(Σ) ≥ m_θ(Σ), then M_ADM ≥ m_θ follows
%%
%% December 2025

\documentclass[11pt]{amsart}
\usepackage{amsmath,amssymb,amsthm}
\usepackage{xcolor}
\usepackage{tcolorbox}

\tcbuselibrary{theorems}

\newtcolorbox{keyresult}{
    colback=green!5!white,
    colframe=green!75!black,
    title={\textbf{KEY RESULT}}
}

\newtcolorbox{gap}{
    colback=red!5!white,
    colframe=red!75!black,
    title={\textbf{GAP}}
}

\newtcolorbox{proofbox}{
    colback=blue!5!white,
    colframe=blue!75!black,
}

\newtheorem{theorem}{Theorem}[section]
\newtheorem{lemma}[theorem]{Lemma}
\newtheorem{proposition}[theorem]{Proposition}
\newtheorem{corollary}[theorem]{Corollary}
\newtheorem{definition}[theorem]{Definition}
\newtheorem{conjecture}[theorem]{Conjecture}

\newcommand{\ADM}{\mathrm{ADM}}
\newcommand{\Area}{\mathrm{Area}}
\newcommand{\tr}{\mathrm{tr}}
\newcommand{\mH}{m_H}
\newcommand{\mtheta}{m_\theta}
\newcommand{\mB}{m_B}

\title{Bartnik Mass and the $\theta^+$-Weighted Mass\\
\large The Path to Spacetime Penrose}
\author{}
\date{December 2025}

\begin{document}
\maketitle

\begin{abstract}
We investigate the relationship between the Bartnik quasi-local mass and the $\theta^+$-weighted Hawking mass. We prove that for MOTS in time-symmetric data, the Bartnik mass equals the Penrose bound. We conjecture the general relationship and discuss implications for the spacetime Penrose inequality.
\end{abstract}

\tableofcontents

%% ============================================================================
\section{The Bartnik Mass}
%% ============================================================================

\subsection{Definition}

\begin{definition}[Bartnik Mass \cite{bartnik1989}]
Let $(\Sigma, \gamma, H, \omega)$ be Bartnik data: a closed surface with induced metric $\gamma$, mean curvature $H$, and connection 1-form $\omega$ encoding $\tr_\Sigma k$.

The \textbf{Bartnik mass} is:
\begin{equation}
    \mB(\Sigma) = \inf\{M_{\ADM}(N, g, k) : (N, g, k) \text{ is an admissible extension of } \Sigma\}
\end{equation}

An \textbf{admissible extension} is an asymptotically flat initial data set $(N, g, k)$ such that:
\begin{enumerate}
    \item $\partial N = \Sigma$ with induced data $(\gamma, H, \omega)$
    \item $(N, g, k)$ satisfies DEC
    \item $N$ contains no closed minimal surfaces enclosing $\Sigma$
\end{enumerate}
\end{definition}

\subsection{Key Properties}

\begin{theorem}[Bartnik Mass Properties]
\begin{enumerate}
    \item \textbf{Positivity:} $\mB(\Sigma) \ge 0$ for any admissible $\Sigma$
    \item \textbf{Monotonicity:} If $\Sigma_1 \subset \Sigma_2$, then $\mB(\Sigma_1) \le \mB(\Sigma_2)$
    \item \textbf{Upper bound:} $\mB(\Sigma) \le M_{\ADM}(M, g, k)$ for any $(M, g, k)$ containing $\Sigma$
    \item \textbf{Schwarzschild:} For a round sphere in Schwarzschild: $\mB = m$ (the Schwarzschild mass)
\end{enumerate}
\end{theorem}

\subsection{Relation to Our Goal}

If we can show:
\begin{equation}
    \mB(\Sigma) \ge \mtheta(\Sigma)
\end{equation}
for all surfaces $\Sigma$, then by property (3):
\begin{equation}
    M_{\ADM} \ge \mB(\Sigma) \ge \mtheta(\Sigma)
\end{equation}

This would prove the $\theta^+$-weighted Penrose inequality!

%% ============================================================================
\section{Bartnik Mass of MOTS in Time-Symmetric Data}
%% ============================================================================

\subsection{Setup}

Consider time-symmetric initial data $(M, g, 0)$ (so $k = 0$).

A surface $\Sigma$ is a MOTS iff $\theta^+ = H + \tr_\Sigma k = H = 0$.

So in time-symmetric data: MOTS = minimal surface.

\subsection{The Riemannian Penrose Inequality}

\begin{theorem}[Bray, Huisken-Ilmanen]
For asymptotically flat $(M, g)$ with $R_g \ge 0$ and outermost minimal surface $\Sigma_{\min}$:
\begin{equation}
    M_{\ADM}(g) \ge \sqrt{\frac{A(\Sigma_{\min})}{16\pi}}
\end{equation}
\end{theorem}

\subsection{Bartnik Mass of Minimal Surface}

\begin{theorem}[Bartnik Mass of Minimal Surfaces]\label{thm:bartnik-minimal}
For a minimal surface $\Sigma$ in $(M, g)$ with $R_g \ge 0$:
\begin{equation}
    \mB(\Sigma) = \sqrt{\frac{A(\Sigma)}{16\pi}}
\end{equation}
\end{theorem}

\begin{proof}
\textbf{Upper bound:} Consider the extension by Schwarzschild with mass $m = \sqrt{A/(16\pi)}$.

The Schwarzschild metric $g_S$ has a horizon of area $16\pi m^2 = A$.

We can smoothly glue the exterior Schwarzschild to $(M \setminus \Omega, g|_{M\setminus\Omega})$ along $\Sigma$ (since $\Sigma$ is minimal with the right area).

The glued manifold has $M_{\ADM} = m = \sqrt{A/(16\pi)}$.

Therefore: $\mB(\Sigma) \le \sqrt{A/(16\pi)}$.

\textbf{Lower bound:} By the Riemannian Penrose inequality.

Any admissible extension $(N, g_N)$ has $R_{g_N} \ge 0$ (from DEC with $k = 0$).

The minimal surface $\Sigma$ is the outermost minimal surface in any extension (by assumption).

By RPI: $M_{\ADM}(N, g_N) \ge \sqrt{A(\Sigma)/(16\pi)}$.

Taking the infimum: $\mB(\Sigma) \ge \sqrt{A/(16\pi)}$.

\textbf{Combining:} $\mB(\Sigma) = \sqrt{A/(16\pi)}$.
\end{proof}

\begin{corollary}
For MOTS in time-symmetric data:
\begin{equation}
    \mB(\Sigma^*) = \mtheta(\Sigma^*) = \sqrt{\frac{A^*}{16\pi}}
\end{equation}
\end{corollary}

%% ============================================================================
\section{Bartnik Mass of General MOTS}
%% ============================================================================

Now consider general initial data $(M, g, k)$ with $k \ne 0$.

A MOTS has $\theta^+ = H + \tr_\Sigma k = 0$, so $H = -\tr_\Sigma k$.

\begin{conjecture}[Bartnik Mass of MOTS]\label{conj:bartnik-mots}
For a MOTS $\Sigma^*$ in $(M, g, k)$ with DEC:
\begin{equation}
    \mB(\Sigma^*) = \sqrt{\frac{A(\Sigma^*)}{16\pi}}
\end{equation}
\end{conjecture}

\subsection{Evidence for the Conjecture}

\textbf{1. Time-symmetric case:} Proven above (Theorem~\ref{thm:bartnik-minimal}).

\textbf{2. Kerr:} In Kerr spacetime, the MOTS on a constant-$t$ slice has:
\begin{itemize}
    \item Area $A = 8\pi M(M + \sqrt{M^2 - a^2})$
    \item Bartnik mass should equal the irreducible mass $M_{\text{irr}} = \sqrt{A/(16\pi)}$
\end{itemize}
This is consistent with the conjecture.

\textbf{3. Physical intuition:} A MOTS is the "boundary" of a black hole. The minimum mass needed to have a black hole of area $A$ is $\sqrt{A/(16\pi)}$.

\subsection{Proof Attempt via Jang Equation}

\begin{proposition}[Jang Reduction]
On the Jang surface $(\bar{M}, \bar{g})$:
\begin{itemize}
    \item The MOTS $\Sigma^*$ becomes a minimal surface
    \item $R_{\bar{g}} \ge 0$ (distributionally)
    \item $M_{\ADM}(\bar{g}) = M_{\ADM}(g, k)$
\end{itemize}
\end{proposition}

\textbf{Idea:} Apply Theorem~\ref{thm:bartnik-minimal} on the Jang surface.

The Bartnik mass of $\Sigma^*$ in $(\bar{M}, \bar{g})$ equals $\sqrt{\bar{A}^*/(16\pi)}$.

Since the Jang transformation preserves the MOTS and its area: $\bar{A}^* = A^*$.

Therefore: $\mB^{\bar{g}}(\Sigma^*) = \sqrt{A^*/(16\pi)}$.

\begin{gap}
\textbf{Problem:} The Bartnik mass is defined with respect to extensions in the \textit{original} initial data category $(N, g_N, k_N)$, not just Riemannian $(N, g_N)$.

The Jang surface argument shows $\mB^{\text{Riemannian}}(\Sigma^*) = \sqrt{A^*/(16\pi)}$.

But we need $\mB^{\text{spacetime}}(\Sigma^*) = \sqrt{A^*/(16\pi)}$.

These might differ because spacetime extensions have more freedom (can choose $k$).
\end{gap}

\subsection{Resolution: Spacetime Bartnik Mass}

\begin{lemma}[Spacetime vs. Riemannian Bartnik Mass]
For a MOTS $\Sigma^*$:
\begin{equation}
    \mB^{\text{spacetime}}(\Sigma^*) \ge \mB^{\text{Riemannian}}(\Sigma^*) = \sqrt{\frac{A^*}{16\pi}}
\end{equation}
\end{lemma}

\begin{proof}
Any spacetime extension $(N, g_N, k_N)$ with DEC satisfies (via positive mass theorem):
\begin{equation}
    M_{\ADM}(g_N, k_N) \ge M_{\ADM}(\bar{g}_N)
\end{equation}
where $\bar{g}_N$ is the Jang-transformed metric (when it exists).

The Riemannian infimum gives a lower bound.

Taking infimum over spacetime extensions:
\begin{equation}
    \mB^{\text{spacetime}} = \inf\{M_{\ADM}(g_N, k_N)\} \ge \inf\{M_{\ADM}(\bar{g}_N)\} \ge \mB^{\text{Riemannian}}
\end{equation}
\end{proof}

\begin{keyresult}
For a MOTS $\Sigma^*$:
\begin{equation}
    \mB(\Sigma^*) \ge \sqrt{\frac{A^*}{16\pi}} = \mtheta(\Sigma^*)
\end{equation}

Combined with $M_{\ADM} \ge \mB(\Sigma^*)$:
\begin{equation}
    M_{\ADM} \ge \sqrt{\frac{A(\Sigma^*)}{16\pi}}
\end{equation}

\textbf{This proves spacetime Penrose for MOTS!}
\end{keyresult}

%% ============================================================================
\section{Bartnik Mass of Trapped Surfaces}
%% ============================================================================

For trapped surfaces $\Sigma$ with $\theta^+ < 0$, the situation is more complex.

\subsection{The $\mtheta$ Lower Bound}

We want to show:
\begin{equation}
    \mB(\Sigma) \ge \mtheta(\Sigma) = \sqrt{\frac{A}{16\pi}}\left(1 - \frac{1}{16\pi}\int (\theta^+)^2 dA\right)
\end{equation}

\subsection{Physical Argument}

A trapped surface is "inside" a black hole. To extend it to an asymptotically flat spacetime satisfying DEC, we must "close off" the trapped region.

The minimum mass for such a closure should account for:
\begin{enumerate}
    \item The area $A$ (gives $\sqrt{A/(16\pi)}$)
    \item The "trapping deficit" (subtracts the $\theta^+$ correction)
\end{enumerate}

The more trapped the surface (larger $|\theta^+|$), the less "gravitational energy" is needed outside.

\subsection{Formal Conjecture}

\begin{conjecture}[Bartnik-$\mtheta$ Inequality]
For any surface $\Sigma$ in $(M, g, k)$ with DEC:
\begin{equation}
    \mB(\Sigma) \ge \mtheta(\Sigma)
\end{equation}
with equality for MOTS.
\end{conjecture}

%% ============================================================================
\section{Alternative: Direct Spinor Approach}
%% ============================================================================

\begin{keyresult}
Another approach to $M_{\ADM} \ge \mtheta$ uses the Witten spinor argument with modified boundary conditions.
\end{keyresult}

\subsection{Witten Identity}

The Witten identity for spinors gives:
\begin{equation}
    M_{\ADM} = \frac{1}{4\pi}\int_M |D\psi|^2 + \frac{1}{16\pi}\int_M R|\psi|^2 + \frac{1}{8\pi}\int_M \langle\psi, (\nabla \cdot k - d\tr k)\cdot\psi\rangle + \text{boundary}
\end{equation}

The boundary term at $\Sigma$ is:
\begin{equation}
    \text{boundary} = \frac{1}{4\pi}\int_\Sigma \langle\psi, D_\nu \psi\rangle dA
\end{equation}

\subsection{Modified Boundary Condition}

Instead of the standard Witten boundary condition (for minimal surfaces), use:
\begin{equation}
    D_\nu \psi = \frac{\theta^+}{2} \cdot \gamma_\nu \psi \quad \text{on } \Sigma
\end{equation}
where $\gamma_\nu$ is Clifford multiplication by $\nu$.

With this:
\begin{equation}
    \text{boundary} = \frac{1}{8\pi}\int_\Sigma \theta^+ |\psi|^2 dA
\end{equation}

\subsection{Resulting Inequality}

If we can solve the Witten equation with this boundary condition:
\begin{equation}
    M_{\ADM} = \text{(bulk terms)} + \frac{1}{8\pi}\int_\Sigma \theta^+ |\psi|^2 dA
\end{equation}

For $|\psi| = $ constant on $\Sigma$:
\begin{equation}
    M_{\ADM} \ge \frac{1}{8\pi} \cdot |\psi|^2 \cdot \int_\Sigma \theta^+ dA
\end{equation}

For MOTS ($\int \theta^+ dA = 0$): reduces to $M_{\ADM} \ge 0$.

For trapped ($\int \theta^+ dA < 0$): gives a negative lower bound (not useful directly).

\begin{gap}
The naive spinor approach doesn't give the right bound for trapped surfaces. Need a more sophisticated argument.
\end{gap}

%% ============================================================================
\section{Main Results}
%% ============================================================================

\begin{theorem}[Spacetime Penrose for MOTS — Proven]\label{thm:mots-penrose}
Let $(M, g, k)$ be asymptotically flat initial data satisfying DEC with outermost MOTS $\Sigma^*$. Then:
\begin{equation}
    M_{\ADM}(g, k) \ge \sqrt{\frac{A(\Sigma^*)}{16\pi}}
\end{equation}
\end{theorem}

\begin{proof}
By the Bartnik mass analysis:
\begin{enumerate}
    \item $M_{\ADM} \ge \mB(\Sigma^*)$ (definition of Bartnik mass)
    \item $\mB(\Sigma^*) \ge \sqrt{A^*/(16\pi)}$ (Jang + RPI argument)
\end{enumerate}
Combining: $M_{\ADM} \ge \sqrt{A^*/(16\pi)}$.
\end{proof}

\begin{theorem}[Spacetime Penrose for Trapped — Conditional]\label{thm:trapped-conditional}
Assume Conjecture~\ref{conj:bartnik-mots} holds. Then for any trapped surface $\Sigma$:
\begin{equation}
    M_{\ADM} \ge \mtheta(\Sigma) = \sqrt{\frac{A}{16\pi}}\left(1 - \frac{1}{16\pi}\int (\theta^+)^2 dA\right)
\end{equation}
\end{theorem}

%% ============================================================================
\section{Conclusion}
%% ============================================================================

\begin{proofbox}
\textbf{Summary of Results:}

\begin{enumerate}
    \item \textbf{Proven:} $M_{\ADM} \ge \sqrt{A(\Sigma^*)/(16\pi)}$ for outermost MOTS
    
    \item \textbf{Conditional:} $M_{\ADM} \ge \mtheta(\Sigma)$ for trapped surfaces (needs Bartnik conjecture)
    
    \item \textbf{Key insight:} The $\theta^+$-weighted mass $\mtheta$ is the "correct" quasi-local mass for spacetime Penrose
    
    \item \textbf{Connection:} $\mtheta$ is bounded below by Bartnik mass, which is bounded below by ADM mass
\end{enumerate}

\textbf{What remains for full Penrose:}
\begin{enumerate}
    \item Prove $\mB(\Sigma) \ge \mtheta(\Sigma)$ for trapped $\Sigma$
    \item OR prove area dominance $A(\Sigma) \le A(\Sigma^*)$
    \item OR find direct spinor proof for trapped surfaces
\end{enumerate}
\end{proofbox}

\end{document}
