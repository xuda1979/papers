% SYMMETRIC REDUCTION APPROACH
%
% The idea: Deform the extrinsic curvature k to make tr_Σ k = 0
% while preserving DEC and the trapped condition.
% If the mass doesn't decrease, we reduce to the favorable case!

\documentclass[12pt]{article}
\usepackage{amsmath,amsthm,amssymb}
\usepackage{mathrsfs}
\newtheorem{theorem}{Theorem}
\newtheorem{lemma}{Lemma}
\newtheorem{proposition}{Proposition}
\newtheorem{corollary}{Corollary}
\newtheorem{conjecture}{Conjecture}
\newtheorem{remark}{Remark}
\newtheorem{definition}{Definition}
\newtheorem{problem}{Problem}
\newtheorem{claim}{Claim}

\begin{document}

\title{Symmetric Reduction:\\Removing the Unfavorable Jump Condition}
\author{Mathematical Development}
\date{\today}
\maketitle

\section{The Key Idea}

Given initial data $(M, g, k)$ with a trapped surface $\Sigma_0$ where $\tr_{\Sigma_0} k < 0$,
we want to construct new initial data $(M, g, \tilde{k})$ such that:

\begin{enumerate}
    \item $\tr_{\Sigma_0} \tilde{k} = 0$ (symmetric/favorable)
    \item $\Sigma_0$ is still trapped for $(g, \tilde{k})$
    \item $(M, g, \tilde{k})$ satisfies DEC
    \item $M_{\mathrm{ADM}}(g, \tilde{k}) \ge M_{\mathrm{ADM}}(g, k)$
\end{enumerate}

If we can do this, then the Penrose inequality for the original data follows from 
the (already proven) inequality for the symmetric data:
\[
M_{\mathrm{ADM}}(g, k) \le M_{\mathrm{ADM}}(g, \tilde{k}) \ge \sqrt{\frac{A(\Sigma_0)}{16\pi}}
\]

\section{The Constraint Equations}

The initial data $(g, k)$ must satisfy the Hamiltonian and momentum constraints:
\begin{align}
    R_g - |k|^2 + (\tr_g k)^2 &= 16\pi\mu \label{eq:Ham} \\
    \div_g(k - (\tr_g k)g) &= 8\pi J \label{eq:Mom}
\end{align}

The DEC requires: $\mu \ge |J|_g$.

\section{What Happens When We Modify $k$?}

\subsection{Naive Approach: Local Modification}

Let $\tilde{k} = k + \delta k$ where $\delta k$ is supported near $\Sigma_0$.

To make $\tr_{\Sigma_0} \tilde{k} = 0$, we need:
\[
\tr_{\Sigma_0} \delta k = -\tr_{\Sigma_0} k > 0
\]

\textbf{Effect on constraints}:
\begin{align}
    \delta\mu &= \frac{1}{16\pi}\left[-2\langle k, \delta k\rangle + 2(\tr k)(\tr \delta k)\right] \\
    \delta J &= \frac{1}{8\pi}\div_g(\delta k - (\tr \delta k)g)
\end{align}

\textbf{Effect on trapped condition}:

The null expansions are:
\begin{align}
    \theta^+ &= H + \tr_\Sigma k \\
    \theta^- &= H - \tr_\Sigma k
\end{align}

If we change $\tr_\Sigma k \to 0$ (from negative), we have $\delta(\tr_\Sigma k) > 0$.

Effect on null expansions:
\begin{align}
    \delta\theta^+ &= \delta(\tr_\Sigma k) > 0 \quad \text{(becomes less negative)} \\
    \delta\theta^- &= -\delta(\tr_\Sigma k) < 0 \quad \text{(becomes more negative)}
\end{align}

\textbf{Problem}: $\theta^+$ increases! If it was barely negative, it might become positive.

\subsection{Condition for Preserved Trapping}

For $\Sigma_0$ to remain trapped after modification:
\begin{align}
    \theta^+ + \delta\theta^+ &\le 0 \\
    \theta^- + \delta\theta^- &< 0 \quad \text{(automatically satisfied)}
\end{align}

The first condition requires:
\[
\delta(\tr_\Sigma k) \le -\theta^+ = -H - \tr_\Sigma k
\]

If we want $\tr_{\Sigma_0} \tilde{k} = 0$, we need $\delta(\tr_\Sigma k) = -\tr_\Sigma k$.

This requires:
\[
-\tr_\Sigma k \le -H - \tr_\Sigma k
\]
which gives:
\[
H \le 0
\]

\begin{claim}
The symmetric reduction preserves trapping if and only if $H \le 0$ on $\Sigma_0$.
\end{claim}

For a trapped surface: $\theta^+ = H + \tr_\Sigma k \le 0$ and $\theta^- = H - \tr_\Sigma k < 0$.

If $\tr_\Sigma k < 0$ (unfavorable case), then from $\theta^+ \le 0$:
\[
H \le -\tr_\Sigma k = |\tr_\Sigma k| > 0
\]

So $H$ could be positive! In that case, symmetric reduction breaks trapping.

\subsection{The Two Cases}

\textbf{Case 1}: $H \le 0$ on $\Sigma_0$.

Then symmetric reduction is possible (at the trapped condition level).

\textbf{Case 2}: $H > 0$ somewhere on $\Sigma_0$.

Then we cannot make $\tr_\Sigma k = 0$ while preserving trapping.
But we CAN make $\tr_\Sigma k = -H$ (so $\theta^+ = 0$), i.e., make $\Sigma_0$ a MOTS!

Actually, wait. If we change $\tr_\Sigma k$ to $-H$, then:
- $\tilde{\theta}^+ = H + (-H) = 0$ ✓
- $\tilde{\theta}^- = H - (-H) = 2H$

If $H > 0$, then $\tilde{\theta}^- = 2H > 0$, which violates the trapped condition!

\begin{remark}
For a trapped surface with $H > 0$ and $\tr_\Sigma k < 0$, we CANNOT deform 
to a MOTS or to $\tr_\Sigma k = 0$ while staying trapped.
\end{remark}

\section{A Different Strategy: Reduce $|H|$ as Well}

\subsection{The Idea}

Instead of just modifying $k$, modify BOTH $g$ and $k$ to get:
\begin{itemize}
    \item $\tilde{H} = 0$ (minimal surface)
    \item $\tr_{\tilde{\Sigma}} \tilde{k} = 0$ (time-symmetric embedding)
\end{itemize}

Then $\tilde{\theta}^\pm = 0$, which is the boundary of being trapped.

But this means $\tilde{\Sigma}$ is only MARGINALLY trapped, not strictly trapped.
And we've changed the GEOMETRY, not just $k$.

\subsection{Conformal Change of $g$}

Let $\tilde{g} = \phi^4 g$ for some conformal factor $\phi > 0$.

Under conformal change:
\[
\tilde{H} = \phi^{-2}(H + 4\nu(\log\phi))
\]
where $\nu$ is the unit normal.

To get $\tilde{H} = 0$:
\[
\nu(\log\phi) = -\frac{H}{4} \quad \text{on } \Sigma_0
\]

This is a Neumann condition for $\log\phi$ on $\Sigma_0$.

\subsection{Effect on $k$}

The extrinsic curvature transforms as... actually, this gets complicated.

In the Jang-inspired approach, we use $(g, k) \to \bar{g}$ where $\bar{g}$ absorbs $k$.
But here we're trying something different.

\section{A Third Strategy: The Compensated Reduction}

\subsection{The Idea}

Don't try to make $\tr_\Sigma k = 0$. Instead, COMPENSATE for the negative contribution 
of $[H]$ in the mass formula by adding a positive contribution elsewhere.

Recall the mass formula for Jang reduction:
\[
16\pi M_{\mathrm{ADM}} = \int_M R_{\tilde{g}} \, d\text{vol} + \text{boundary terms}
\]

where $R_{\tilde{g}}$ has a delta function:
\[
R_{\tilde{g}} = R_{\mathrm{smooth}} + 2[H]\delta_\Sigma
\]

When $[H] < 0$, this gives a negative contribution $\sim A(\Sigma) \cdot [H]$.

\textbf{Compensation idea}: Add matter/energy density near $\Sigma$ to offset this.

\subsection{Mathematical Formulation}

Suppose we add energy density $\Delta\mu$ supported in a tubular neighborhood 
$U_\epsilon(\Sigma_0)$ of $\Sigma_0$:
\[
\Delta\mu = \frac{|[H]|}{8\pi\epsilon} \cdot \chi_{U_\epsilon}
\]

This adds mass:
\[
\Delta M = \int \Delta\mu \, d\text{vol} \approx \frac{|[H]|}{8\pi\epsilon} \cdot A(\Sigma) \cdot \epsilon = \frac{|[H]| \cdot A(\Sigma)}{8\pi}
\]

\textbf{Problem}: This doesn't give a CLOSED inequality. We need to know $|[H]|$ 
to compensate, but $|[H]|$ depends on the Jang equation solution.

\section{Fourth Strategy: Don't Use Jang Equation at All}

\subsection{Return to AMO's Original Method}

AMO (Andersson-Mars-Outermost) uses a different approach: they work directly 
with the constraint equations and mean curvature flow, not Jang equation.

Their method gives:
\[
M_{\mathrm{ADM}} \ge M_H(\Sigma^*) \quad \text{(Hawking mass of outermost MOTS)}
\]

The Penrose inequality follows if $M_H(\Sigma^*) \ge \sqrt{A(\Sigma_0)/(16\pi)}$.

This requires $A(\Sigma^*) \ge A(\Sigma_0)$.

\subsection{When Does Area Comparison Fail?}

We know $A(\Sigma^*) \ge A(\Sigma_0)$ fails when $\tr_{\Sigma_0} k < 0$.

\textbf{Key question}: HOW MUCH smaller can $A(\Sigma^*)$ be?

Suppose:
\[
A(\Sigma^*) = A(\Sigma_0)(1 + \epsilon)
\]
for some $\epsilon$ that could be negative.

What's the LOWER BOUND on $\epsilon$?

\subsection{An Estimate}

From the analysis of trapped region geometry:

If $\tr_\Sigma k = -\kappa$ for some $\kappa > 0$, then heuristically:
\[
A(\Sigma^*) \approx A(\Sigma_0)(1 - c\kappa)
\]
for some geometric constant $c$.

This suggests:
\[
M_{\mathrm{ADM}} \ge M_H(\Sigma^*) \ge \sqrt{\frac{A(\Sigma_0)(1 - c\kappa)}{16\pi}}
\]

This is a WEAKER bound, but still non-trivial!

\section{Fifth Strategy: A Two-Step Proof}

\subsection{The Idea}

\begin{enumerate}
    \item First, prove a WEAKER inequality: $M \ge \sqrt{A/(16\pi)} \cdot (1 - C\cdot|\tr_\Sigma k|)$
    \item Then, bootstrap to the full inequality using monotonicity or iteration
\end{enumerate}

\subsection{Step 1: The Weakened Inequality}

\begin{conjecture}[Weakened Penrose Inequality]
Let $(M, g, k)$ satisfy DEC with trapped surface $\Sigma_0$. Then:
\[
M_{\mathrm{ADM}} \ge \sqrt{\frac{A(\Sigma_0)}{16\pi}} \cdot \left(1 - C \int_{\Sigma_0} |\tr_\Sigma k| \, dA\right)
\]
for some universal constant $C > 0$.
\end{conjecture}

This is easier to prove because we allow a "defect" term.

\subsection{Step 2: The Bootstrap}

\textbf{Idea}: The weakened inequality, combined with some geometric condition, 
might imply the full inequality.

For example, if the "optimal" trapped surface $\Sigma_{\mathrm{opt}}$ has small 
$|\tr_{\Sigma_{\mathrm{opt}}} k|$, then the weakened inequality gives a strong bound.

\begin{conjecture}[Optimal Trapped Surface]
Among all trapped surfaces $\Sigma$ homologous to a given $\Sigma_0$, there exists 
one minimizing $\int_\Sigma |\tr_\Sigma k| \, dA$. This optimal surface satisfies 
the full Penrose inequality.
\end{conjecture}

\section{Synthesis: The Most Promising Path}

After exploring these strategies, the most promising seems to be:

\begin{theorem}[Target Theorem - To Be Proven]
Let $(M, g, k)$ satisfy DEC with outermost MOTS $\Sigma^*$ enclosing trapped surface $\Sigma_0$.
Define the \textbf{effective area}:
\[
A_{\mathrm{eff}}(\Sigma_0) = A(\Sigma_0) \cdot \max\left(1 + 2\int_{\Sigma_0} \tr_\Sigma k \, dA / A(\Sigma_0), 0\right)
\]

Then:
\[
M_{\mathrm{ADM}} \ge \sqrt{\frac{\max(A(\Sigma^*), A_{\mathrm{eff}}(\Sigma_0))}{16\pi}}
\]
\end{theorem}

This combines:
\begin{itemize}
    \item The AMO bound via $\Sigma^*$
    \item The effective area bound from our earlier analysis
\end{itemize}

The maximum ensures we get the best possible bound.

\section{Next Steps}

\begin{enumerate}
    \item Rigorous analysis of how $A(\Sigma^*)$ relates to $A(\Sigma_0)$ and $\tr_{\Sigma_0} k$
    \item Prove the weakened Penrose inequality
    \item Study the optimal trapped surface problem
    \item Look for counterexamples to the strong inequality
\end{enumerate}

\end{document}
