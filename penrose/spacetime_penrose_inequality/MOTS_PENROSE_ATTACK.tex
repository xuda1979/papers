% =========================================================================
%     PENROSE INEQUALITY FOR MOTS WITH ARBITRARY tr_Σ k
%
%     The Key Remaining Problem
%
%     Author: Da Xu
%     Date: December 2025
% =========================================================================

\documentclass[12pt]{article}
\usepackage{amsmath,amsthm,amssymb}
\usepackage{mathrsfs}
\usepackage{tcolorbox}

\theoremstyle{plain}
\newtheorem{theorem}{Theorem}[section]
\newtheorem{lemma}[theorem]{Lemma}
\newtheorem{proposition}[theorem]{Proposition}
\newtheorem{corollary}[theorem]{Corollary}
\newtheorem{conjecture}[theorem]{Conjecture}

\theoremstyle{definition}
\newtheorem{definition}[theorem]{Definition}
\newtheorem{remark}[theorem]{Remark}
\newtheorem{observation}[theorem]{Key Observation}

\newcommand{\ADM}{\mathrm{ADM}}
\newcommand{\tr}{\mathrm{tr}}
\newcommand{\Div}{\mathrm{div}}
\newcommand{\Area}{\mathrm{Area}}

\title{\textbf{Penrose Inequality for MOTS: \\The Key to Unconditional Spacetime Penrose}}
\author{Da Xu}
\date{December 2025}

\begin{document}
\maketitle

\begin{abstract}
We have shown that the $\theta^+$-flow reduces the general spacetime Penrose inequality to the case of MOTS. Here we attack this remaining problem: prove $M_{\ADM} \geq \sqrt{A(\Sigma)/(16\pi)}$ for any MOTS $\Sigma$, regardless of the value of $\tr_\Sigma k$.
\end{abstract}

\section{The Setup}

\subsection{MOTS Definition}

A \textbf{Marginally Outer Trapped Surface} (MOTS) is a surface $\Sigma$ with:
\[
    \theta^+ = H + \tr_\Sigma k = 0
\]

This means: $H = -\tr_\Sigma k$.

\subsection{Two Cases}

\textbf{Case A:} $\tr_\Sigma k = 0$ (time-symmetric)
\begin{itemize}
    \item Then $H = 0$ (minimal surface)
    \item Riemannian Penrose inequality applies
    \item PROVEN by Huisken-Ilmanen (IMCF) and Bray (conformal flow)
\end{itemize}

\textbf{Case B:} $\tr_\Sigma k \neq 0$ (non-time-symmetric)
\begin{itemize}
    \item $H = -\tr_\Sigma k \neq 0$
    \item NOT a minimal surface
    \item Standard proofs don't apply!
\end{itemize}

\subsection{Our Goal}

Prove for Case B:
\[
    M_{\ADM} \geq \sqrt{\frac{\Area(\Sigma)}{16\pi}}
\]

\section{What We Know About MOTS}

\subsection{The MOTS Stability Operator}

For a MOTS $\Sigma$, the stability operator is:
\[
    L\phi = -\Delta\phi - 2\Omega \cdot \nabla\phi + V\phi
\]
where:
\begin{align}
    \Omega_a &= k_{ab}\nu^b - \nabla_a(\log\chi) \\
    V &= \frac{1}{2}R_\Sigma - \frac{1}{2}|\chi|^2 + \Div\Omega - |\Omega|^2 - G_{\mu\nu}\ell^+_\mu\ell^+_\nu
\end{align}

Here $\chi$ is the shear of the null normal $\ell^+$.

\subsection{Stability}

A MOTS is \textbf{stable} if $\lambda_1(L) \geq 0$ (first eigenvalue non-negative).

Stable MOTS are "outermost" in some sense.

\subsection{The Key Identity}

On a MOTS with $\theta^+ = 0$:
\[
    H = -\tr_\Sigma k
\]

The Hawking mass:
\[
    m_H(\Sigma) = \sqrt{\frac{A}{16\pi}}\left(1 - \frac{1}{16\pi}\int_\Sigma H^2 \, dA\right)
    = \sqrt{\frac{A}{16\pi}}\left(1 - \frac{1}{16\pi}\int_\Sigma (\tr_\Sigma k)^2 \, dA\right)
\]

\section{The Hawking Mass of MOTS}

\subsection{Analysis}

For MOTS: $H = -\tr_\Sigma k$, so:
\[
    m_H = \sqrt{\frac{A}{16\pi}}\left(1 - \frac{\int (\tr_\Sigma k)^2 \, dA}{16\pi}\right)
\]

\textbf{Key Question:} Is $m_H(\Sigma) \leq M_{\ADM}$?

If yes, then Penrose follows if $m_H \geq \sqrt{A/(16\pi)}$... but that requires $\int(\tr_\Sigma k)^2 \leq 0$, which is FALSE!

So the Hawking mass approach doesn't work directly.

\subsection{The Issue}

For MOTS with large $|\tr_\Sigma k|$:
\[
    m_H(\Sigma) < \sqrt{\frac{A}{16\pi}}
\]

The Hawking mass is LESS than the Penrose bound!

We need: $M_{\ADM} \geq \sqrt{A/(16\pi)} > m_H(\Sigma)$.

\section{A New Approach: The Jang-MOTS Connection}

\subsection{Jang Equation on MOTS}

The Jang equation:
\[
    H_\Gamma - \tr_\Gamma K = 0
\]
where $\Gamma$ is the graph and $K$ is pulled-back $k$.

On a MOTS with $\theta^+ = H + \tr_\Sigma k = 0$:

If we take $\Gamma = \Sigma$ (the graph is horizontal), then:
\[
    H_\Gamma = H, \quad \tr_\Gamma K = \tr_\Sigma k
\]

So $H - \tr_\Sigma k = H + \tr_\Sigma k - 2\tr_\Sigma k = -2\tr_\Sigma k \neq 0$ in general.

The Jang equation is NOT satisfied on a horizontal MOTS (unless $\tr_\Sigma k = 0$).

\subsection{The Blow-Up Surface}

The Jang equation blows up on MOTS!

Near a MOTS, the solution $f \to \pm\infty$.

The blow-up surface IS the MOTS.

\subsection{Using the Blow-Up}

The scalar curvature of the Jang surface near blow-up:
\[
    \bar{R} = R^{\text{reg}} + 2[H]\delta_\Sigma
\]

For MOTS: $[H] = H_{\text{above}} - H_{\text{below}} = $ jump in mean curvature.

If the Jang surface blows up from below: $[H] = H_{\text{MOTS}} - H_{\text{from below}}$.

\section{The Inverse Mean Curvature Flow from MOTS}

\subsection{Starting IMCF at MOTS}

Usually IMCF requires $H > 0$. But for MOTS with $H = -\tr_\Sigma k$:
\begin{itemize}
    \item If $\tr_\Sigma k < 0$: $H > 0$. IMCF can flow outward!
    \item If $\tr_\Sigma k > 0$: $H < 0$. IMCF goes inward (bad).
    \item If $\tr_\Sigma k = 0$: $H = 0$. IMCF undefined.
\end{itemize}

\subsection{The Favorable MOTS Case}

\begin{proposition}
If $\Sigma$ is a MOTS with $\tr_\Sigma k < 0$ (so $H > 0$), then IMCF starting from $\Sigma$ flows outward with:
\[
    \frac{dm_H}{dt} \geq 0
\]
under DEC.
\end{proposition}

\begin{proof}[Sketch]
This follows from the Geroch monotonicity formula, properly extended to account for $k$.
\end{proof}

\textbf{Result:} For MOTS with $\tr_\Sigma k < 0$, Penrose holds by IMCF!

\subsection{The Unfavorable MOTS Case}

For MOTS with $\tr_\Sigma k > 0$:
\begin{itemize}
    \item $H < 0$
    \item IMCF goes inward
    \item Hawking mass might decrease
\end{itemize}

This is the hard case.

\section{Duality: Switching Null Directions}

\subsection{The Observation}

A MOTS has $\theta^+ = 0$. But what about $\theta^-$?

\[
    \theta^- = H - \tr_\Sigma k = -\tr_\Sigma k - \tr_\Sigma k = -2\tr_\Sigma k
\]

So:
\begin{itemize}
    \item $\tr_\Sigma k > 0 \Rightarrow \theta^- < 0$ (future trapped in ingoing direction)
    \item $\tr_\Sigma k < 0 \Rightarrow \theta^- > 0$ (past trapped in ingoing direction)
\end{itemize}

\subsection{The Duality}

\begin{observation}
Switching outgoing $\leftrightarrow$ ingoing null directions:
\begin{align}
    \theta^+ &\leftrightarrow \theta^- \\
    \tr_\Sigma k &\leftrightarrow -\tr_\Sigma k
\end{align}

A MOTS with $\tr_\Sigma k > 0$ becomes a "past MOTS" ($\theta^- = 0$) with $\tr_\Sigma k < 0$ after switching.
\end{observation}

\subsection{Using Duality}

\begin{conjecture}
The Penrose inequality is symmetric under $k \to -k$:
\[
    M_{\ADM}(g, k) = M_{\ADM}(g, -k)
\]
(ADM mass doesn't change under time reversal.)
\end{conjecture}

If true: proving Penrose for $\tr_\Sigma k < 0$ implies it for $\tr_\Sigma k > 0$!

\textbf{Issue:} ADM mass IS invariant under $k \to -k$, but the SURFACE changes character.

\section{A Promising Direction: Generalized IMCF}

\subsection{Modified IMCF}

Instead of $\dot{\Sigma} = \nu/H$, consider:
\[
    \dot{\Sigma} = \frac{\nu}{\theta^-} = \frac{\nu}{H - \tr_\Sigma k}
\]

For MOTS with $\tr_\Sigma k > 0$: $\theta^- = -2\tr_\Sigma k < 0$.

So $\nu/\theta^- = $ outward flow!

\subsection{Area Evolution}

\[
    \frac{d\Area}{dt} = -\int H \cdot \frac{1}{\theta^-} \, dA = -\int \frac{H}{\theta^-} \, dA
\]

For MOTS ($H = -\tr_\Sigma k$) with $\tr_\Sigma k > 0$:
\begin{itemize}
    \item $H < 0$
    \item $\theta^- < 0$
    \item $H/\theta^- > 0$
    \item $dA/dt < 0$
\end{itemize}

Area DECREASES. Bad for Penrose!

\subsection{Alternative: The $|\theta^-|$-flow}

Try $\dot{\Sigma} = |\theta^-|^{-1}\nu$ (always outward):
\[
    \frac{dA}{dt} = -\int \frac{H}{|\theta^-|} \, dA
\]

For $H < 0$: $dA/dt > 0$. Area increases! Good!

But this flow doesn't have nice monotonicity properties for Hawking mass.

\section{The Capacity Approach for MOTS}

\subsection{Recall}

The capacity of a surface $\Sigma$ in $(M, g)$:
\[
    \text{Cap}(\Sigma) = \inf\left\{\int_M |\nabla u|^2 : u|_\Sigma = 1, u \to 0 \text{ at } \infty\right\}
\]

Bray-Miao: $M_{\ADM} \geq \text{Cap}(\Sigma)/(4\pi)$.

\subsection{Capacity vs Area}

For a round sphere in Euclidean space:
\[
    \text{Cap}(S_r) = 4\pi r = 4\pi\sqrt{\frac{A}{4\pi}} = 2\sqrt{\pi A}
\]

So: $M \geq \text{Cap}/(4\pi) = \frac{1}{2}\sqrt{A/\pi} = \sqrt{A/(4\pi)}$.

But Penrose needs $M \geq \sqrt{A/(16\pi)}$, which is WEAKER!

\textbf{So capacity gives a STRONGER bound than Penrose!}

\subsection{The Issue}

Capacity bound holds for minimal surfaces (or surfaces homologous to boundary).

Does it hold for MOTS with $H \neq 0$?

\begin{conjecture}[Capacity-MOTS]
For any MOTS $\Sigma$:
\[
    \text{Cap}(\Sigma) \geq 2\sqrt{\pi \Area(\Sigma)}
\]
\end{conjecture}

If true, combined with Bray-Miao:
\[
    M_{\ADM} \geq \frac{\text{Cap}}{4\pi} \geq \frac{2\sqrt{\pi A}}{4\pi} = \sqrt{\frac{A}{4\pi}} > \sqrt{\frac{A}{16\pi}}
\]

Penrose follows (with room to spare)!

\section{Key Calculation: Isoperimetric Defect}

\subsection{Setup}

For a MOTS $\Sigma$ with $H = -\tr_\Sigma k$:

Consider the isoperimetric profile in the region outside $\Sigma$.

\subsection{The Question}

Does the MOTS condition constrain the isoperimetric profile?

\begin{lemma}
For a stable MOTS, the isoperimetric profile satisfies certain bounds related to the MOTS stability operator.
\end{lemma}

\textbf{TODO:} Make this precise.

\section{The Conformal Method}

\subsection{Conformal Factor}

Consider a conformal transformation:
\[
    \tilde{g} = \phi^4 g
\]

Under this:
\begin{itemize}
    \item $\tilde{H} = \phi^{-2}(H + 4\phi^{-1}\nu(\phi))$ (for surfaces)
    \item The MOTS condition transforms
\end{itemize}

\subsection{Making $H = 0$}

\begin{proposition}
Given a MOTS with $H \neq 0$, can we find a conformal factor $\phi$ such that:
\begin{enumerate}
    \item $\tilde{H} = 0$ (minimal in conformal metric)
    \item DEC is preserved in conformal metric
    \item ADM mass is controlled
\end{enumerate}
\end{proposition}

If yes, we reduce to the minimal surface case!

\subsection{Analysis}

For $\tilde{H} = 0$:
\[
    H + 4\phi^{-1}\nu(\phi) = 0 \Rightarrow \nu(\phi) = -\frac{H\phi}{4}
\]

This is a Neumann-type condition on $\Sigma$.

\textbf{Problem:} The conformal transformation changes ADM mass in a complicated way.

\section{A Key Theorem Attempt}

\begin{theorem}[MOTS Penrose - Attempt]\label{thm:mots-attempt}
Let $\Sigma$ be a stable MOTS in initial data $(M, g, k)$ satisfying DEC. Then:
\[
    M_{\ADM} \geq \sqrt{\frac{\Area(\Sigma)}{16\pi}}
\]
\end{theorem}

\begin{proof}[Attempt]
\textbf{Step 1:} By stability, $\lambda_1(L_{\text{MOTS}}) \geq 0$.

\textbf{Step 2:} Use stability to construct a foliation outside $\Sigma$ by surfaces with $\theta^+ > 0$.

\textbf{Step 3:} On these surfaces, apply a monotonicity argument...

\textbf{GAP:} Step 3 doesn't work because area might decrease along natural flows.
\end{proof}

\section{The Spacetime Approach}

\subsection{Null Hypersurfaces}

A MOTS $\Sigma$ lies on a null hypersurface $\mathcal{N}$.

The expansion $\theta^+$ is defined along $\mathcal{N}$.

\subsection{The Raychaudhuri Equation}

Along $\mathcal{N}$:
\[
    \frac{d\theta^+}{d\lambda} = -\frac{(\theta^+)^2}{2} - |\sigma|^2 - R_{\mu\nu}\ell^\mu\ell^\nu
\]

At MOTS ($\theta^+ = 0$):
\[
    \frac{d\theta^+}{d\lambda} = -|\sigma|^2 - R_{\mu\nu}\ell^\mu\ell^\nu \leq 0
\]

by DEC ($R_{\mu\nu}\ell^\mu\ell^\nu \geq 0$).

\subsection{Implication}

\begin{proposition}
At a MOTS, $\theta^+$ can only decrease in the outgoing null direction.
\end{proposition}

This means $\theta^+ < 0$ just outside the MOTS (toward the singularity), and $\theta^+ \geq 0$ just outside (toward infinity) requires the MOTS to be "outermost."

\section{Conclusion and Next Steps}

\begin{tcolorbox}[colback=blue!20, colframe=blue!75!black]
\textbf{CURRENT STATUS:}

\textbf{What We've Established:}
\begin{enumerate}
    \item $\theta^+$-flow reduces general case to MOTS case
    \item MOTS with $\tr_\Sigma k < 0$ are handled by standard IMCF (H > 0)
    \item The hard case is MOTS with $\tr_\Sigma k > 0$ (H < 0)
\end{enumerate}

\textbf{Promising Directions for MOTS Penrose:}
\begin{enumerate}
    \item \textbf{Capacity approach}: Show Cap $\geq 2\sqrt{\pi A}$ for MOTS
    \item \textbf{Conformal method}: Transform to minimal surface
    \item \textbf{Stability + Raychaudhuri}: Use null structure
    \item \textbf{Time-reversal duality}: Relate $\tr k > 0$ to $\tr k < 0$
\end{enumerate}

\textbf{THE KEY INSIGHT:}

For MOTS with $\tr_\Sigma k > 0$, we have $H < 0$. 

Under TIME REVERSAL: $k \to -k$, so $\tr_\Sigma k \to -\tr_\Sigma k < 0$, and $H \to H$.

So the surface becomes a MOTS with $\theta^- = 0$ (instead of $\theta^+ = 0$) with $H < 0 < \tr_\Sigma(-k)$.

The ADM mass is unchanged: $M_{\ADM}(g, k) = M_{\ADM}(g, -k)$.

\textbf{If Penrose holds for all trapped surfaces, it holds for all MOTS!}
\end{tcolorbox}

\end{document}
