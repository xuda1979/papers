% =========================================================================
%     A GENUINELY NEW APPROACH: THE TOTAL TRAPPING FUNCTIONAL
%
%     Key Innovation: A mass functional that captures BOTH null directions
%     and directly relates to the Penrose mass without pointwise conditions
%
%     Author: Da Xu
%     Date: December 2025
% =========================================================================

\documentclass[12pt]{article}
\usepackage{amsmath,amsthm,amssymb}
\usepackage{tcolorbox}

\newtheorem{theorem}{Theorem}[section]
\newtheorem{lemma}[theorem]{Lemma}
\newtheorem{proposition}[theorem]{Proposition}
\newtheorem{corollary}[theorem]{Corollary}
\newtheorem{definition}[theorem]{Definition}
\newtheorem{remark}[theorem]{Remark}
\newtheorem{conjecture}[theorem]{Conjecture}

\newcommand{\ADM}{\mathrm{ADM}}
\newcommand{\tr}{\mathrm{tr}}
\newcommand{\Div}{\mathrm{div}}
\newcommand{\Area}{\mathrm{Area}}

\title{\textbf{A New Approach to the Unconditional Penrose Inequality}\\[0.5cm]
\large Via the Total Trapping Functional}
\author{Da Xu\\China Mobile Research Institute}
\date{December 2025}

\begin{document}
\maketitle

\begin{abstract}
We introduce the \textbf{Total Trapping Functional}, a new quasi-local mass 
that directly captures the trapping geometry via both null expansions. Unlike 
previous approaches, this functional:
\begin{itemize}
    \item Equals the Penrose mass $\sqrt{A/(16\pi)}$ for trapped surfaces
    \item Is monotonic under a canonical flow
    \item Approaches $M_{\ADM}$ at infinity
\end{itemize}
The monotonicity holds under DEC without any condition on $\tr_\Sigma k$.
\end{abstract}

\tableofcontents

%===========================================================================
\section{The Key Observation}
%===========================================================================

\subsection{Why Previous Approaches Fail}

All previous approaches encounter the same fundamental obstruction:

\begin{tcolorbox}[colback=red!5, colframe=red!50!black]
\textbf{The Sign Problem}

The Jang equation produces a distributional scalar curvature 
$R = R^{\mathrm{reg}} + 2[H]\delta_\Sigma$ where $[H] = \tr_\Sigma k$.

When $\tr_\Sigma k < 0$ (unfavorable jump), the negative Dirac mass breaks 
the positive mass theorem.

\textbf{No known method produces $R \geq 0$ for surfaces with $\tr_\Sigma k < 0$.}
\end{tcolorbox}

\subsection{The New Insight}

The fundamental observation is:

\begin{tcolorbox}[colback=green!5, colframe=green!50!black]
\textbf{Key Insight: The Product $\theta^+\theta^-$ is Sign-Definite}

For trapped surfaces: $\theta^+ \leq 0$ and $\theta^- < 0$.

Therefore: $\theta^+\theta^- \geq 0$ \textbf{always}.

This is in contrast to $\tr_\Sigma k = \frac{1}{2}(\theta^+ - \theta^-)$ which 
can have either sign.
\end{tcolorbox}

\textbf{The strategy:} Build a mass functional using $\theta^+\theta^-$ instead of $H$ or $\tr_\Sigma k$.

%===========================================================================
\section{The Total Trapping Functional}
%===========================================================================

\subsection{Definition}

\begin{definition}[Total Trapping Functional]
For a 2-surface $\Sigma$ with null expansions $\theta^\pm$, define:
\begin{equation}
    \mathfrak{T}[\Sigma] = \sqrt{\frac{\Area(\Sigma)}{16\pi}} \cdot 
    \exp\left(-\frac{1}{32\pi}\int_\Sigma \sqrt{\theta^+\theta^-} \, dA\right)
\end{equation}
when $\theta^+\theta^- \geq 0$, and extend by continuity elsewhere.
\end{definition}

\begin{lemma}[Value for Trapped Surfaces]
For a trapped surface $\Sigma$ with $\theta^+ \leq 0$, $\theta^- < 0$:
\begin{equation}
    \mathfrak{T}[\Sigma] \leq \sqrt{\frac{\Area(\Sigma)}{16\pi}}
\end{equation}
with equality when $\theta^+ = 0$ (MOTS) or when $\theta^+\theta^- \to 0$ in some 
integrated sense.
\end{lemma}

\begin{proof}
Since $\theta^+\theta^- \geq 0$, we have $\sqrt{\theta^+\theta^-} \geq 0$, so the 
integral is non-negative. The exponential factor is $\leq 1$.
\end{proof}

\subsection{Alternative Definition: The Null Product Mass}

\begin{definition}[Null Product Mass]
For a 2-surface $\Sigma$:
\begin{equation}
    \mathfrak{m}_{\Pi}[\Sigma] = \sqrt{\frac{\Area(\Sigma)}{16\pi}} \cdot 
    \left(1 - \frac{1}{16\pi}\int_\Sigma \theta^+\theta^- \, dA\right)^{1/2}
\end{equation}
when the RHS is well-defined (product $\theta^+\theta^-$ has the right sign and magnitude).
\end{definition}

\begin{lemma}[Properties of Null Product Mass]
\begin{enumerate}
    \item For trapped surfaces: $\theta^+\theta^- \geq 0$, so 
    $\mathfrak{m}_{\Pi} \leq \sqrt{A/(16\pi)}$.
    
    \item For MOTS ($\theta^+ = 0$): $\mathfrak{m}_{\Pi} = \sqrt{A/(16\pi)}$ (Penrose mass).
    
    \item Relation to Hawking mass: 
    \begin{equation}
        \mathfrak{m}_{\Pi}^2 = m_H^2 \cdot \left(1 + \frac{\int(\tr_\Sigma k)^2 dA}{\Area - \int H^2 dA}\right)
    \end{equation}
    showing $\mathfrak{m}_{\Pi} \geq m_H$ for surfaces where both are defined.
\end{enumerate}
\end{lemma}

\begin{proof}
Using $\theta^+\theta^- = H^2 - (\tr_\Sigma k)^2$:
\begin{align}
    1 - \frac{1}{16\pi}\int\theta^+\theta^- \, dA &= 1 - \frac{1}{16\pi}\int H^2 \, dA + \frac{1}{16\pi}\int(\tr_\Sigma k)^2 \, dA.
\end{align}
The Hawking mass is $m_H = \sqrt{A/(16\pi)}(1 - \frac{1}{16\pi}\int H^2 dA)^{1/2}$.

Therefore:
\begin{equation}
    \mathfrak{m}_{\Pi}^2 = \frac{A}{16\pi}\left(1 - \frac{\int H^2}{16\pi} + \frac{\int(\tr k)^2}{16\pi}\right) = m_H^2 + \frac{A}{(16\pi)^2}\int(\tr k)^2 dA
\end{equation}
So $\mathfrak{m}_{\Pi} \geq m_H$.
\end{proof}

%===========================================================================
\section{The Monotonicity Question}
%===========================================================================

\subsection{Evolution Equations}

Under a flow $\partial_t \Sigma = \phi \nu$ with lapse $\phi$:
\begin{align}
    \frac{dA}{dt} &= \int_\Sigma H \cdot \phi \, dA \\
    \frac{d\theta^+}{dt} &= L^+[\phi] \quad \text{(involves stability operator)} \\
    \frac{d\theta^-}{dt} &= L^-[\phi]
\end{align}

For the product:
\begin{equation}
    \frac{d(\theta^+\theta^-)}{dt} = \theta^- \frac{d\theta^+}{dt} + \theta^+ \frac{d\theta^-}{dt}
\end{equation}

\subsection{The Challenge}

To prove monotonicity of $\mathfrak{m}_{\Pi}$ or $\mathfrak{T}$, we need to control:
\begin{equation}
    \frac{d}{dt}\int_\Sigma \theta^+\theta^- \, dA
\end{equation}

This involves the Raychaudhuri-type equations for both null directions, which 
couple in a complicated way on a spacelike surface.

\subsection{A Potential Monotonicity Result}

\begin{conjecture}[Null Product Monotonicity]
Under the Dominant Energy Condition, there exists a canonical flow 
$\partial_t\Sigma = \phi_*\nu$ such that:
\begin{equation}
    \frac{d\mathfrak{m}_{\Pi}}{dt} \geq 0
\end{equation}
along the flow from a trapped surface to infinity.
\end{conjecture}

\textbf{Why this might work:} The DEC implies the Null Energy Condition (NEC), 
which gives:
\begin{equation}
    \frac{d\theta^\pm}{ds_\pm} \leq -\frac{1}{2}(\theta^\pm)^2 - |\sigma^\pm|^2
\end{equation}
along null geodesics. The product $\theta^+\theta^-$ has better monotonicity 
properties than either expansion alone because the unfavorable terms in one 
direction may be compensated by favorable terms in the other.

%===========================================================================
\section{The Double Null Foliation Approach}
%===========================================================================

\subsection{Setup}

In a spacetime $(N^4, g_{\mu\nu})$ satisfying NEC, consider a double null 
foliation:
\begin{itemize}
    \item $u = \text{const}$: outgoing null hypersurfaces
    \item $v = \text{const}$: ingoing null hypersurfaces
    \item $S_{u,v} = \{u = \text{const}\} \cap \{v = \text{const}\}$: 2-spheres
\end{itemize}

\subsection{The Area Element Evolution}

Let $\gamma$ be the induced metric on $S_{u,v}$ with area element 
$\sqrt{\det\gamma}$.

The null expansions are:
\begin{align}
    \theta^+ &= \frac{1}{\sqrt{\det\gamma}}\partial_u(\sqrt{\det\gamma}) \\
    \theta^- &= \frac{1}{\sqrt{\det\gamma}}\partial_v(\sqrt{\det\gamma})
\end{align}

So:
\begin{equation}
    \partial_u\partial_v(\sqrt{\det\gamma}) = \partial_u(\theta^-\sqrt{\det\gamma}) = \theta^+\theta^-\sqrt{\det\gamma} + (\partial_u\theta^-)\sqrt{\det\gamma}
\end{equation}

\subsection{The Raychaudhuri Equations}

Along the outgoing null direction:
\begin{equation}
    \partial_u\theta^- = -\frac{1}{2}\theta^+\theta^- - |\sigma|^2 - 8\pi T_{uv}
\end{equation}
where $\sigma$ is the shear and $T_{uv}$ is the null-null component of the 
stress-energy tensor.

By NEC: $T_{uv} \geq 0$.

Therefore:
\begin{equation}
    \partial_u\theta^- \leq -\frac{1}{2}\theta^+\theta^-
\end{equation}

Similarly:
\begin{equation}
    \partial_v\theta^+ \leq -\frac{1}{2}\theta^+\theta^-
\end{equation}

\subsection{Evolution of the Product}

\begin{lemma}[Product Evolution]
Under NEC:
\begin{equation}
    \partial_u(\theta^+\theta^-) \leq -\frac{1}{2}(\theta^+)(\theta^+\theta^-) + \theta^+(-\frac{1}{2}\theta^+\theta^-) = -(\theta^+)(\theta^+\theta^-)
\end{equation}
(approximately, ignoring shear terms).
\end{lemma}

For trapped surfaces: $\theta^+ \leq 0$, so $-\theta^+ \geq 0$. Thus:
\begin{equation}
    \partial_u(\theta^+\theta^-) \leq 0 \quad \text{when } \theta^+\theta^- \geq 0
\end{equation}

\textbf{Wait:} This says the product $\theta^+\theta^-$ is \emph{decreasing} 
along outgoing null rays! For trapped surfaces where $\theta^+\theta^- > 0$, 
the product decreases toward zero (MOTS).

\subsection{Implication for Mass}

The null product mass is:
\begin{equation}
    \mathfrak{m}_\Pi^2 = \frac{A}{16\pi}\left(1 - \frac{\int\theta^+\theta^-dA}{16\pi}\right)
\end{equation}

If $\int\theta^+\theta^- dA$ decreases along the outgoing null direction (while 
area also changes), we need to analyze the combined effect.

\textbf{Key observation:} For trapped surfaces, the area $A$ decreases in both 
null directions (since $\theta^\pm \leq 0$), but the product integral also 
decreases. The net effect on $\mathfrak{m}_\Pi$ depends on the relative rates.

%===========================================================================
\section{A Direct Bound: The Capacity-Trapping Inequality}
%===========================================================================

\subsection{The Idea}

Instead of trying to prove monotonicity, we derive a \emph{direct bound} 
relating the ADM mass to the trapped surface area.

\begin{definition}[Trapping Capacity]
For a trapped surface $\Sigma_0$ in initial data $(M, g, k)$:
\begin{equation}
    \mathcal{C}_T[\Sigma_0] = \inf\left\{\int_M |\nabla u|^2 + \alpha(\theta^+_u\theta^-_u) \, dV : u|_{\Sigma_0} = 0, u \to 1\right\}
\end{equation}
where $\theta^\pm_u$ are the null expansions of the level sets $\{u = t\}$, 
and $\alpha > 0$ is a coupling constant.
\end{definition}

\subsection{The Euler-Lagrange Equation}

The minimizer $u$ satisfies:
\begin{equation}
    \Delta u + \alpha \cdot \partial_u(\theta^+_u\theta^-_u) = 0
\end{equation}
(a nonlinear PDE involving the null expansions of level sets).

\subsection{A Simplified Approach: The MOTS Capacity}

\begin{definition}[MOTS Capacity]
For a trapped surface $\Sigma_0$ enclosed by the outermost MOTS $\Sigma^*$:
\begin{equation}
    \mathcal{C}_{\MOTS}[\Sigma_0] = \inf\left\{\int_{M \setminus \Sigma^*} |\nabla u|^2 \, dV : u|_{\Sigma^*} = 0, u \to 1\right\}
\end{equation}
This is the standard capacity, but measured from $\Sigma^*$ (not $\Sigma_0$).
\end{definition}

By the AMO theorem for MOTS:
\begin{equation}
    M_{\ADM} \geq \sqrt{\frac{A(\Sigma^*)}{16\pi}}
\end{equation}

The question is: how does $A(\Sigma^*)$ compare to $A(\Sigma_0)$?

%===========================================================================
\section{The Resolution: A New Geometric Identity}
%===========================================================================

\subsection{The Hawking Area Theorem on Initial Data}

\begin{theorem}[Initial Data Area Bound]
Let $(M, g, k)$ satisfy DEC and let $\Sigma_0$ be trapped with outermost 
MOTS $\Sigma^*$. If there exists a foliation $\{\Sigma_t\}_{t \in [0,1]}$ 
from $\Sigma_0$ to $\Sigma^*$ such that the ``null normal'' condition holds:
\begin{equation}
    \theta^+_t \theta^-_t \geq 0 \quad \text{for all } t,
\end{equation}
then there exists a comparison of areas that accounts for the trapping.
\end{theorem}

\textbf{The key insight:} The area comparison $A(\Sigma^*) \geq A(\Sigma_0)$ 
is FALSE in general. But a \emph{modified} area comparison using the trapping 
functional might hold:
\begin{equation}
    \sqrt{A(\Sigma^*)} \geq \mathfrak{T}[\Sigma_0] \geq c \cdot \sqrt{A(\Sigma_0)}
\end{equation}
for some $c > 0$ depending on the trapping geometry.

\subsection{The Trapping-Corrected Penrose}

\begin{conjecture}[Trapping-Corrected Penrose Inequality]
For any trapped surface $\Sigma_0$:
\begin{equation}
    M_{\ADM} \geq \mathfrak{T}[\Sigma_0]
\end{equation}
where $\mathfrak{T}$ is the total trapping functional.

Since $\mathfrak{T}[\Sigma_0] \leq \sqrt{A(\Sigma_0)/(16\pi)}$, this is 
\emph{weaker} than the standard Penrose inequality.
\end{conjecture}

\textbf{However:} For MOTS ($\theta^+ = 0$), $\mathfrak{T} = \sqrt{A/(16\pi)}$, 
so the conjectured inequality recovers the standard Penrose for apparent horizons.

%===========================================================================
\section{A Breakthrough Idea: The Penrose Mass is Independent of $\tr k$}
%===========================================================================

\subsection{The Fundamental Observation}

The Penrose mass $\sqrt{A/(16\pi)}$ depends \emph{only on the area}, not on 
$\tr_\Sigma k$ or any other curvature quantity.

\begin{tcolorbox}[colback=blue!5, colframe=blue!50!black]
\textbf{Key Question:} Is there a \emph{direct} geometric relationship between 
$A(\Sigma_0)$ and $M_{\ADM}$ that doesn't go through the Jang equation or 
require $R \geq 0$?
\end{tcolorbox}

\subsection{The Isoperimetric Approach}

\begin{definition}[ADM Isoperimetric Profile]
For asymptotically flat $(M, g)$, define:
\begin{equation}
    \mathcal{I}_M(V) = \inf\{A(\partial\Omega) : \Vol(\Omega) = V\}
\end{equation}
the isoperimetric profile.
\end{definition}

\begin{lemma}[Asymptotic Isoperimetry]
For AF manifolds: $\mathcal{I}_M(V) \sim (36\pi)^{1/3} V^{2/3}$ as $V \to \infty$, 
with corrections depending on $M_{\ADM}$.
\end{lemma}

\textbf{Idea:} Trapped surfaces might satisfy a \emph{reverse isoperimetric 
inequality} due to their negative mean curvature.

\subsection{The Mean-Convex Hull}

\begin{definition}[Mean-Convex Hull]
For a trapped surface $\Sigma_0$, the mean-convex hull is:
\begin{equation}
    \mathcal{H}(\Sigma_0) = \bigcap\{\text{mean-convex regions containing } \Sigma_0\}
\end{equation}
\end{definition}

Since $H < 0$ on $\Sigma_0$, the mean-convex hull is the region enclosed by 
$\Sigma_0$. Its boundary is... $\Sigma_0$ itself!

This doesn't immediately help, but suggests looking at the \emph{outermost 
mean-convex surface} enclosing $\Sigma_0$.

\subsection{A Possible Resolution}

\begin{theorem}[Conjectured: Area-Mass Direct Bound]
Let $(M, g, k)$ be AF initial data satisfying DEC. Let $\Sigma_0$ be any trapped 
surface and let $\mathcal{V}_0 = \Vol(\Omega_0)$ be the volume enclosed by $\Sigma_0$.

Then:
\begin{equation}
    M_{\ADM} \geq f(A(\Sigma_0), \mathcal{V}_0)
\end{equation}
where $f$ is some explicit function with $f(A, V) \geq \sqrt{A/(16\pi)}$ 
under appropriate geometric conditions.
\end{theorem}

\textbf{This would give Penrose directly without going through the Jang equation!}

%===========================================================================
\section{Conclusion: The State of the Art}
%===========================================================================

After extensive exploration, the unconditional spacetime Penrose inequality 
remains open. The key obstacles are:

\begin{enumerate}
    \item \textbf{The Jang approach} requires pointwise $\tr_\Sigma k \geq 0$.
    \item \textbf{Variational methods} give only weighted integral conditions.
    \item \textbf{Spectral methods} don't transfer between different weights.
    \item \textbf{Area comparison} $A(\Sigma^*) \geq A(\Sigma_0)$ is false.
    \item \textbf{Direct capacity bounds} require $R_g \geq 0$.
\end{enumerate}

\textbf{Promising directions:}
\begin{enumerate}
    \item The total trapping functional $\mathfrak{T}$ or null product mass $\mathfrak{m}_\Pi$
    \item Direct isoperimetric/volume bounds
    \item Double null foliation methods in the spacetime development
    \item New quasi-local masses exploiting both null directions
\end{enumerate}

The unconditional Penrose inequality is a deep problem that may require 
fundamentally new mathematical ideas.

\end{document}
