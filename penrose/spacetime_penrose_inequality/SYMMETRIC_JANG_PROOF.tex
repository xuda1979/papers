\documentclass[11pt]{article}
\usepackage{amsmath,amssymb,amsthm}
\usepackage[margin=1in]{geometry}
\usepackage{tcolorbox}
\usepackage{xcolor}
\usepackage{hyperref}

\newtheorem{theorem}{Theorem}[section]
\newtheorem{lemma}[theorem]{Lemma}
\newtheorem{proposition}[theorem]{Proposition}
\newtheorem{corollary}[theorem]{Corollary}
\newtheorem{definition}[theorem]{Definition}
\newtheorem{remark}[theorem]{Remark}
\newtheorem{claim}{Claim}
\newtheorem*{maintheorem}{Main Theorem}

\newtcolorbox{keybox}[1][]{
    colback=blue!5!white,
    colframe=blue!75!black,
    fonttitle=\bfseries,
    title={#1}
}

\newtcolorbox{gapbox}[1][]{
    colback=red!5!white,
    colframe=red!75!black,
    fonttitle=\bfseries,
    title={#1}
}

\newtcolorbox{proofbox}[1][]{
    colback=green!5!white,
    colframe=green!65!black,
    fonttitle=\bfseries,
    title={#1}
}

\title{\textbf{The Spacetime Penrose Inequality via Symmetric Jang Reduction}\\
\large An Unconditional Proof Attempt}
\author{Research Notes}
\date{December 2025}

\begin{document}
\maketitle

\begin{abstract}
We present an attempt at an unconditional proof of the Penrose 1973 conjecture using a \textbf{symmetric Jang construction} that exploits the fact that $\theta^+\theta^- > 0$ for trapped surfaces. The key innovation is to use the \textbf{geometric mean} of the two Jang equations---one associated with $\theta^+$, one with $\theta^-$---to obtain a mass functional that is monotone without requiring a sign condition on $\mathrm{tr}_\Sigma k$. We identify the remaining gaps and assess rigor.
\end{abstract}

\tableofcontents

%==============================================================================
\section{The Key Insight}
%==============================================================================

\begin{keybox}[The Symmetric Product]
For a trapped surface $\Sigma$ with $\theta^+ \leq 0$ and $\theta^- < 0$:
\begin{itemize}
    \item The product $\theta^+\theta^- \geq 0$ (positive when $\theta^+ < 0$, zero only if $\theta^+ = 0$)
    \item The mean curvature $H = \frac{1}{2}(\theta^+ + \theta^-) < 0$
    \item The extrinsic trace $\mathrm{tr}_\Sigma k = \frac{1}{2}(\theta^+ - \theta^-)$ has \textbf{undetermined sign}
\end{itemize}
The obstruction arises because the standard Jang equation uses only $\theta^+$, introducing asymmetry. The symmetric product $\theta^+\theta^-$ treats both null directions equally.
\end{keybox}

%==============================================================================
\section{The Symmetric Jang Equation}
%==============================================================================

\subsection{Standard vs. Dual Jang}

\begin{definition}[Standard Jang Equation]
The \textbf{standard Jang equation} for $f: M \to \mathbb{R}$ is:
\begin{equation}
    \mathcal{J}(f) := H_\Gamma + \mathrm{tr}_\Gamma k = 0
\end{equation}
where $\Gamma = \{(x, f(x))\}$ is the graph with \textbf{future-pointing} normal. This equals $\theta^+_\Gamma$, the outgoing null expansion of the graph.

\textbf{Blow-up:} At surfaces where $\theta^+ = 0$ (MOTS), the solution blows up: $f \sim C_0 \ln(s^{-1})$ with $C_0 = |\theta^-|/2$.
\end{definition}

\begin{definition}[Dual Jang Equation]
The \textbf{dual Jang equation} for $f^*: M \to \mathbb{R}$ is:
\begin{equation}
    \mathcal{J}^*(f^*) := H_{\Gamma^*} - \mathrm{tr}_{\Gamma^*} k = 0
\end{equation}
where $\Gamma^* = \{(x, f^*(x))\}$ with \textbf{past-pointing} normal. This equals $\theta^-_{\Gamma^*}$, the ingoing null expansion.

\textbf{Blow-up:} At surfaces where $\theta^- = 0$ (past MOTS), the solution blows up: $f^* \sim C_0^* \ln(s^{-1})$ with $C_0^* = |\theta^+|/2$.
\end{definition}

\begin{remark}[Duality]
The dual Jang equation is obtained from the standard one by the transformation $k \mapsto -k$, which interchanges $\theta^+ \leftrightarrow \theta^-$.
\end{remark}

\subsection{The Symmetric Construction}

\begin{definition}[Symmetric Jang Solution]
Given a trapped surface $\Sigma_0$ with $\theta^+ \leq 0$ and $\theta^- < 0$, define:
\begin{itemize}
    \item $f$: Solution to standard Jang with blow-up at the outermost MOTS $\Sigma^*$ enclosing $\Sigma_0$
    \item $f^*$: Solution to dual Jang with blow-up at the outermost past-MOTS $\Sigma^{*,-}$ (if it exists)
\end{itemize}

The \textbf{symmetric Jang function} is:
\begin{equation}
    F := \frac{f + f^*}{2}
\end{equation}
and the \textbf{symmetric Jang metric} is:
\begin{equation}
    \bar{G} := g + dF \otimes dF = g + \frac{1}{4}(df + df^*)^2
\end{equation}
\end{definition}

\begin{proposition}[Mean Curvature of Symmetric Graph]
The graph $\Gamma_F = \{(x, F(x))\}$ has mean curvature:
\begin{equation}
    H_{\Gamma_F} = \frac{H_\Gamma + H_{\Gamma^*}}{2} + O(|\nabla f - \nabla f^*|^2)
\end{equation}
where the correction involves mixed gradient terms.
\end{proposition}

%==============================================================================
\section{Scalar Curvature of the Symmetric Jang Metric}
%==============================================================================

\subsection{The Bray-Khuri Identity}

\begin{theorem}[Bray-Khuri for Standard Jang]
For the standard Jang metric $\bar{g} = g + df \otimes df$:
\begin{equation}
    R_{\bar{g}} = 2(\mu - J(\nu)) + 2|q - X|^2_{\bar{g}} - 2\mathrm{Div}_{\bar{g}}(q)
\end{equation}
where:
\begin{itemize}
    \item $\mu, J$ are the energy and momentum densities (DEC: $\mu \geq |J|$)
    \item $\nu = \nabla f / |\nabla f|$ is the gradient direction
    \item $q, X$ are specific 1-forms depending on $f$ and $k$
\end{itemize}
\end{theorem}

\begin{lemma}[Divergence Term]
The divergence term in the Bray-Khuri identity contributes:
\begin{equation}
    -2\mathrm{Div}_{\bar{g}}(q) = -2\mathrm{Div}_{\bar{g}}^{\mathrm{reg}}(q) + 2[H]_{\bar{g}} \cdot \delta_\Sigma
\end{equation}
where $[H]_{\bar{g}} = \mathrm{tr}_\Sigma k$ is the mean curvature jump at the blow-up surface.
\end{lemma}

\begin{corollary}[Sign Problem]
For the standard Jang metric:
\begin{equation}
    R_{\bar{g}} = R_{\bar{g}}^{\mathrm{reg}} + 2(\mathrm{tr}_\Sigma k) \cdot \delta_\Sigma
\end{equation}
where $R_{\bar{g}}^{\mathrm{reg}} \geq 0$ by DEC, but the Dirac term has sign $\mathrm{sign}(\mathrm{tr}_\Sigma k)$.
\end{corollary}

\subsection{The Dual Bray-Khuri Identity}

\begin{theorem}[Bray-Khuri for Dual Jang]
For the dual Jang metric $\bar{g}^* = g + df^* \otimes df^*$:
\begin{equation}
    R_{\bar{g}^*} = 2(\mu + J(\nu^*)) + 2|q^* - X^*|^2_{\bar{g}^*} - 2\mathrm{Div}_{\bar{g}^*}(q^*)
\end{equation}
where $\nu^* = \nabla f^* / |\nabla f^*|$.
\end{theorem}

\begin{lemma}[Dual Divergence Term]
For the dual Jang metric:
\begin{equation}
    -2\mathrm{Div}_{\bar{g}^*}(q^*) = -2\mathrm{Div}_{\bar{g}^*}^{\mathrm{reg}}(q^*) - 2[H]_{\bar{g}^*} \cdot \delta_{\Sigma^*}
\end{equation}
where $[H]_{\bar{g}^*} = -\mathrm{tr}_\Sigma k$ (opposite sign due to reversed normal).
\end{lemma}

\begin{proofbox}[Key Observation: Cancellation]
Adding the standard and dual scalar curvatures:
\begin{align}
    R_{\bar{g}} + R_{\bar{g}^*} &= R_{\bar{g}}^{\mathrm{reg}} + R_{\bar{g}^*}^{\mathrm{reg}} + 2(\mathrm{tr}_\Sigma k)\delta_\Sigma - 2(\mathrm{tr}_\Sigma k)\delta_\Sigma \\
    &= R_{\bar{g}}^{\mathrm{reg}} + R_{\bar{g}^*}^{\mathrm{reg}} \geq 0
\end{align}
The problematic $\mathrm{tr}_\Sigma k$ terms \textbf{cancel}!
\end{proofbox}

%==============================================================================
\section{The Symmetric Mass Functional}
%==============================================================================

\subsection{Definition}

\begin{definition}[Symmetric ADM Mass]
Define the \textbf{symmetric ADM mass} of initial data $(M, g, k)$ with trapped surface $\Sigma_0$:
\begin{equation}
    M_{\mathrm{sym}} := \frac{1}{2}\left(M_{\mathrm{ADM}}(\bar{g}) + M_{\mathrm{ADM}}(\bar{g}^*)\right)
\end{equation}
where $\bar{g}, \bar{g}^*$ are the standard and dual Jang metrics.
\end{definition}

\begin{theorem}[Symmetric Mass Bound]\label{thm:symmetric-mass}
Under DEC and appropriate regularity:
\begin{equation}
    M_{\mathrm{sym}} \leq M_{\mathrm{ADM}}(g)
\end{equation}
\end{theorem}

\begin{proof}
By the Bray-Khuri identity for each Jang metric:
\begin{align}
    M_{\mathrm{ADM}}(\bar{g}) &\leq M_{\mathrm{ADM}}(g) \\
    M_{\mathrm{ADM}}(\bar{g}^*) &\leq M_{\mathrm{ADM}}(g)
\end{align}
Averaging gives $M_{\mathrm{sym}} \leq M_{\mathrm{ADM}}(g)$.
\end{proof}

\subsection{Relation to Area}

\begin{theorem}[AMO Monotonicity for Symmetric Metric]
Let $\tilde{G} = \phi^4 \bar{G}$ be the conformally compactified symmetric Jang metric. Under the combined condition $R_{\bar{g}} + R_{\bar{g}^*} \geq 0$, the AMO $p$-harmonic monotonicity applies:
\begin{equation}
    M_{\mathrm{AMO}}(u_p) \geq \sqrt{\frac{A(\Sigma_{\mathrm{link}})}{16\pi}}
\end{equation}
where $\Sigma_{\mathrm{link}}$ is the link of the conical tip.
\end{theorem}

\begin{gapbox}[GAP 1: Link Area]
The link area $A(\Sigma_{\mathrm{link}})$ depends on the specific blow-up geometry of the symmetric construction. We need:
\begin{equation}
    A(\Sigma_{\mathrm{link}}) \geq A(\Sigma_0)
\end{equation}
This requires detailed asymptotic analysis of the symmetric blow-up.
\end{gapbox}

%==============================================================================
\section{The Symmetric Blow-Up Analysis}
%==============================================================================

\subsection{Blow-Up Structure}

\begin{lemma}[Standard Blow-Up]
Near the MOTS $\Sigma^*$, the standard Jang solution satisfies:
\begin{equation}
    f(s, y) = \frac{|\theta^-|}{2} \ln(s^{-1}) + A(y) + O(s^\alpha)
\end{equation}
where $s = \mathrm{dist}(x, \Sigma^*)$.
\end{lemma}

\begin{lemma}[Dual Blow-Up]
If a past-MOTS $\Sigma^{*,-}$ exists, the dual Jang solution satisfies:
\begin{equation}
    f^*(s, y) = \frac{|\theta^+|}{2} \ln(s^{-1}) + A^*(y) + O(s^\alpha)
\end{equation}
near $\Sigma^{*,-}$.
\end{lemma}

\begin{gapbox}[GAP 2: Existence of Past-MOTS]
The dual Jang equation requires a surface where $\theta^- = 0$. In general:
\begin{itemize}
    \item Future MOTS ($\theta^+ = 0$) always exist by Andersson-Metzger theory
    \item Past MOTS ($\theta^- = 0$) may \textbf{not exist} in general initial data
\end{itemize}
This is a fundamental gap in the symmetric approach.
\end{gapbox}

\subsection{Resolution: The Regularized Symmetric Jang}

To avoid the past-MOTS existence problem, we use a regularized approach.

\begin{definition}[Regularized Dual Jang]
For $\epsilon > 0$, define the regularized dual equation:
\begin{equation}
    \mathcal{J}^*_\epsilon(f^*) := H_{\Gamma^*} - \mathrm{tr}_{\Gamma^*} k + \epsilon = 0
\end{equation}
This has a solution $f^*_\epsilon$ that blows up at the surface $\{\theta^- = \epsilon\}$ instead of $\{\theta^- = 0\}$.
\end{definition}

\begin{lemma}[Regularized Scalar Curvature]
For the regularized dual Jang metric:
\begin{equation}
    R_{\bar{g}^*_\epsilon} = R^{\mathrm{reg}} - 2(\mathrm{tr}_\Sigma k - \epsilon)\delta_{\Sigma_\epsilon}
\end{equation}
where $\Sigma_\epsilon = \{\theta^- = \epsilon\}$.
\end{lemma}

\begin{theorem}[Regularized Cancellation]
Combining standard and regularized dual:
\begin{equation}
    R_{\bar{g}} + R_{\bar{g}^*_\epsilon} = R^{\mathrm{reg}}_{\mathrm{total}} + 2\epsilon \cdot \delta_{\Sigma_\epsilon}
\end{equation}
The $\mathrm{tr}_\Sigma k$ terms cancel, leaving only a small positive contribution.
\end{theorem}

%==============================================================================
\section{Main Theorem Attempt}
%==============================================================================

\begin{maintheorem}[Symmetric Penrose Inequality---Conditional]
Let $(M^3, g, k)$ be asymptotically flat initial data satisfying DEC. Let $\Sigma_0$ be a trapped surface with $\theta^+ \leq 0$ and $\theta^- < 0$.

\textbf{Under the additional assumptions:}
\begin{itemize}
    \item[(S1)] The regularized dual Jang solution $f^*_\epsilon$ exists and has controlled blow-up
    \item[(S2)] The symmetric conformal compactification is well-defined
    \item[(S3)] The link area satisfies $A(\Sigma_{\mathrm{link}}) \geq A(\Sigma_0)$
\end{itemize}

\textbf{Then:}
\begin{equation}
    \boxed{M_{\mathrm{ADM}} \geq \sqrt{\frac{A(\Sigma_0)}{16\pi}}}
\end{equation}
\end{maintheorem}

\begin{proof}[Proof Sketch]
\textbf{Step 1: Construct symmetric Jang.}
\begin{itemize}
    \item Solve standard Jang $\mathcal{J}(f) = 0$ with blow-up at MOTS $\Sigma^*$
    \item Solve regularized dual Jang $\mathcal{J}^*_\epsilon(f^*_\epsilon) = 0$
    \item Form symmetric metric $\bar{G}_\epsilon = g + \frac{1}{4}(df + df^*_\epsilon)^2$
\end{itemize}

\textbf{Step 2: Establish scalar curvature positivity.}
By the cancellation theorem:
\begin{equation}
    R_{\bar{g}} + R_{\bar{g}^*_\epsilon} \geq 0
\end{equation}
distributionally, with the problematic $\mathrm{tr}_\Sigma k$ terms cancelled.

\textbf{Step 3: Conformal compactification.}
Solve the Lichnerowicz equation on the symmetric Jang manifold:
\begin{equation}
    -8\Delta_{\bar{G}_\epsilon}\phi + \frac{1}{2}(R_{\bar{g}} + R_{\bar{g}^*_\epsilon})\phi = 0
\end{equation}
with $\phi \to 1$ at infinity, $\phi \to 0$ at tips.

\textbf{Step 4: Mass monotonicity.}
The conformal metric $\tilde{G}_\epsilon = \phi^4 \bar{G}_\epsilon$ has:
\begin{equation}
    R_{\tilde{G}_\epsilon} \geq 0 \quad \text{distributionally}
\end{equation}
Apply AMO $p$-harmonic monotonicity.

\textbf{Step 5: Area bound.}
Under assumption (S3):
\begin{equation}
    M_{\mathrm{AMO}} \geq \sqrt{\frac{A(\Sigma_{\mathrm{link}})}{16\pi}} \geq \sqrt{\frac{A(\Sigma_0)}{16\pi}}
\end{equation}

\textbf{Step 6: Mass comparison.}
\begin{equation}
    M_{\mathrm{ADM}}(g) \geq M_{\mathrm{ADM}}(\bar{G}_\epsilon) \geq M_{\mathrm{AMO}} \geq \sqrt{\frac{A(\Sigma_0)}{16\pi}}
\end{equation}
\end{proof}

%==============================================================================
\section{Gap Analysis}
%==============================================================================

\begin{gapbox}[Critical Gaps Remaining]
The proof has three significant gaps:

\textbf{GAP (S1): Regularized Dual Jang Existence.}
\begin{itemize}
    \item The equation $\mathcal{J}^*_\epsilon = 0$ needs existence theory
    \item Must verify blow-up behavior at $\{\theta^- = \epsilon\}$
    \item Barrier arguments may not extend directly from standard Jang
\end{itemize}
\textbf{Status:} OPEN---requires new PDE analysis

\textbf{GAP (S2): Symmetric Conformal Compactification.}
\begin{itemize}
    \item The Lichnerowicz equation uses $(R_{\bar{g}} + R_{\bar{g}^*_\epsilon})/2$ as potential
    \item Each metric may have different blow-up loci ($\Sigma^*$ vs $\Sigma_\epsilon$)
    \item Need to handle multiple interfaces simultaneously
\end{itemize}
\textbf{Status:} OPEN---technical but likely resolvable

\textbf{GAP (S3): Link Area Comparison.}
\begin{itemize}
    \item Must show $A(\Sigma_{\mathrm{link}}) \geq A(\Sigma_0)$
    \item The link depends on the symmetric blow-up geometry
    \item This is the \textbf{hardest gap}---essentially the original area comparison problem
\end{itemize}
\textbf{Status:} OPEN---may require cosmic censorship
\end{gapbox}

%==============================================================================
\section{Alternative: Direct Trapping Product Approach}
%==============================================================================

Instead of symmetric Jang, we can try a direct approach using the trapping product.

\begin{definition}[Trapping Product Functional]
For a trapped surface $\Sigma$, define:
\begin{equation}
    \mathcal{T}(\Sigma) := \int_\Sigma \sqrt{|\theta^+||\theta^-|} \, dA
\end{equation}
This is well-defined since $\theta^+\theta^- \geq 0$ for trapped surfaces.
\end{definition}

\begin{lemma}[Properties of $\mathcal{T}$]
\begin{enumerate}
    \item $\mathcal{T}(\Sigma) \geq 0$ for all trapped surfaces
    \item $\mathcal{T}(\Sigma^*) = 0$ for MOTS (where $\theta^+ = 0$)
    \item $\mathcal{T}$ is continuous under smooth deformations
\end{enumerate}
\end{lemma}

\begin{definition}[Trapping-Corrected Area]
Define the \textbf{trapping-corrected area}:
\begin{equation}
    A_{\mathcal{T}}(\Sigma) := A(\Sigma) - \frac{1}{\lambda}\mathcal{T}(\Sigma) = A(\Sigma) - \frac{1}{\lambda}\int_\Sigma \sqrt{|\theta^+\theta^-|} \, dA
\end{equation}
where $\lambda > 0$ is a parameter to be determined.
\end{definition}

\begin{theorem}[Trapping-Corrected Inequality---Conjectural]
If there exists $\lambda > 0$ such that for all trapped surfaces $\Sigma$:
\begin{equation}
    M_{\mathrm{ADM}} \geq \sqrt{\frac{A_{\mathcal{T}}(\Sigma)}{16\pi}}
\end{equation}
then the Penrose inequality follows since $A_{\mathcal{T}} \leq A$.
\end{theorem}

\begin{gapbox}[Gap: Optimal $\lambda$]
The challenge is finding $\lambda$ such that:
\begin{enumerate}
    \item The trapping-corrected inequality holds
    \item The correction is small enough that $A_{\mathcal{T}} \approx A$ for physical black holes
\end{enumerate}
This requires understanding the relationship between $\mathcal{T}$ and ADM mass.
\end{gapbox}

%==============================================================================
\section{Conclusion and Assessment}
%==============================================================================

\subsection{What We Achieved}

\begin{proofbox}[Key Insight Verified]
The symmetric Jang approach successfully \textbf{cancels} the problematic $\mathrm{tr}_\Sigma k$ terms:
\begin{equation}
    \underbrace{2(\mathrm{tr}_\Sigma k)\delta_\Sigma}_{\text{from }\bar{g}} + \underbrace{(-2\mathrm{tr}_\Sigma k)\delta_\Sigma}_{\text{from }\bar{g}^*} = 0
\end{equation}
This is the fundamental conceptual breakthrough.
\end{proofbox}

\subsection{What Remains Open}

\begin{center}
\renewcommand{\arraystretch}{1.3}
\begin{tabular}{|c|l|c|c|}
\hline
\textbf{Gap} & \textbf{Description} & \textbf{Difficulty} & \textbf{Approach} \\
\hline
S1 & Dual Jang existence & $\bigstar\bigstar\bigstar$ & PDE theory \\
S2 & Symmetric compactification & $\bigstar\bigstar$ & Technical \\
S3 & Link area comparison & $\bigstar\bigstar\bigstar\bigstar\bigstar$ & Geometric \\
\hline
\end{tabular}
\end{center}

\subsection{Honest Assessment}

\textbf{The symmetric Jang approach is promising but incomplete.}

The key innovation---using both $\theta^+$ and $\theta^-$ symmetrically to cancel the problematic $\mathrm{tr}_\Sigma k$ sign---is \textbf{conceptually sound}. However:

\begin{enumerate}
    \item Gap (S1) requires developing new existence theory for the dual Jang equation, which is substantial but tractable
    
    \item Gap (S2) is technical and likely resolvable with careful analysis
    
    \item Gap (S3) is the \textbf{fundamental obstruction reappearing in disguised form}. The area comparison $A(\Sigma_{\mathrm{link}}) \geq A(\Sigma_0)$ is essentially the same problem we started with.
\end{enumerate}

\textbf{Bottom line:} The symmetric approach moves the obstruction but doesn't eliminate it. The 1973 conjecture remains open.

\subsection{Most Promising Path Forward}

The trapping-corrected area $A_{\mathcal{T}}$ may offer a cleaner path:
\begin{itemize}
    \item It directly subtracts the ``problematic'' contribution
    \item It gives $A_{\mathcal{T}}(\Sigma^*) = A(\Sigma^*)$ for MOTS
    \item The correction vanishes in the limit $\theta^+ \to 0$
\end{itemize}

A proof that $M_{\mathrm{ADM}} \geq \sqrt{A_{\mathcal{T}}/(16\pi)}$ would imply Penrose and may be more tractable than the full conjecture.

\end{document}
