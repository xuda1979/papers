% =========================================================================
%     PARABOLIC REGULARIZATION APPROACH TO THE PENROSE INEQUALITY
%
%     Using heat flow and mean curvature flow to regularize the sign problem
%
%     Author: Da Xu
%     Date: December 2025
% =========================================================================

\documentclass[12pt]{article}
\usepackage{amsmath,amsthm,amssymb}
\usepackage{mathrsfs}
\usepackage{tcolorbox}

\theoremstyle{plain}
\newtheorem{theorem}{Theorem}[section]
\newtheorem{lemma}[theorem]{Lemma}
\newtheorem{proposition}[theorem]{Proposition}
\newtheorem{corollary}[theorem]{Corollary}

\theoremstyle{definition}
\newtheorem{definition}[theorem]{Definition}
\newtheorem{remark}[theorem]{Remark}

\newcommand{\ADM}{\mathrm{ADM}}
\newcommand{\tr}{\mathrm{tr}}
\newcommand{\Div}{\mathrm{div}}
\newcommand{\Area}{\mathrm{Area}}

\title{\textbf{Parabolic Regularization for the Penrose Inequality}}
\author{Da Xu}
\date{December 2025}

\begin{document}
\maketitle

\section{The Regularization Strategy}

\subsection{Core Idea}

The sign obstruction is that $\tr_\Sigma k$ can be negative.

\textbf{Strategy:} Use parabolic evolution to:
\begin{enumerate}
    \item Smoothly deform the initial data $(g, k) \to (g_t, k_t)$
    \item Make $\tr_\Sigma k_t \geq 0$ for $t > 0$
    \item Track the Penrose functional through the flow
\end{enumerate}

\section{Approach 1: Heat Flow on k}

\subsection{The Evolution}

Consider the heat equation for the tensor $k$:
\[
    \partial_t k = \Delta_g k
\]
with $k|_{t=0} = k_0$ the original extrinsic curvature.

\begin{lemma}
Under heat flow, $k$ becomes smoother and extrema are attenuated:
\[
    \min_\Sigma (\tr_\Sigma k_t) \to \bar{k} \quad \text{as } t \to \infty
\]
where $\bar{k}$ is the average.
\end{lemma}

\subsection{The Problem}

\textbf{Issue 1:} Heat flow on $k$ alone violates the constraint equations:
\[
    R_g + (\tr k)^2 - |k|^2 = 16\pi\mu \quad \text{(Hamiltonian constraint)}
\]
\[
    \Div(k - (\tr k)g) = 8\pi J \quad \text{(Momentum constraint)}
\]

Heat flow on $k$ changes $|k|^2$ and $\tr k$, breaking constraints.

\textbf{Issue 2:} Even if we project back to constraint surface, the ADM mass changes:
\[
    M_{\ADM}(g_t, k_t) \neq M_{\ADM}(g_0, k_0)
\]

\section{Approach 2: Constraint-Preserving Flow}

\subsection{The Coupled System}

To preserve constraints, we need a coupled flow on $(g, k)$.

\textbf{Einstein flow} (vacuum):
\begin{align}
    \partial_t g_{ij} &= -2N k_{ij} + \mathcal{L}_X g_{ij} \\
    \partial_t k_{ij} &= -\nabla_i \nabla_j N + N(R_{ij} + (\tr k)k_{ij} - 2k_i^l k_{lj}) + \mathcal{L}_X k_{ij}
\end{align}

This is the ADM evolution with lapse $N$ and shift $X$.

\begin{proposition}
If $(g_0, k_0)$ satisfies constraints, then $(g_t, k_t)$ satisfies constraints for all $t$.
\end{proposition}

\subsection{Choosing Lapse to Improve Sign}

\textbf{Idea:} Choose $N$ such that $\tr_\Sigma k$ increases in the negative region.

From the evolution of $\tr k$ on a surface:
\[
    \partial_t(\tr_\Sigma k) = -\Delta_\Sigma N + N(\cdots) + \text{(lower order)}
\]

To make $\tr_\Sigma k$ increase where it's negative, we need $N$ large in that region.

\textbf{Problem:} This is a \textbf{parabolic constraint} on $N$, not freely specifiable.

The evolution becomes:
\[
    \partial_t(\tr_\Sigma k) \geq C \cdot (\text{minimum of } \tr_\Sigma k) \quad \text{???}
\]

No maximum principle applies here.

\section{Approach 3: Mean Curvature Flow of $\Sigma$}

\subsection{The Evolution}

Instead of evolving $(g, k)$, evolve the surface $\Sigma$ by mean curvature flow:
\[
    \partial_t X = H \cdot \nu
\]

For trapped surfaces, $H < 0$, so the flow goes \textbf{inward}.

\begin{lemma}
Under MCF:
\[
    \frac{d\Area}{dt} = -\int_\Sigma H^2 \, dA < 0
\]
Area strictly decreases.
\end{lemma}

\subsection{The Inverse Problem}

Run MCF \textbf{backward}:
\[
    \partial_t X = -H \cdot \nu
\]

This expands the surface.

\textbf{Problem:} Backward MCF is ill-posed (unstable).

Small perturbations grow exponentially, and solutions may not exist.

\subsection{Regularized Backward MCF}

Add a regularizing term:
\[
    \partial_t X = -H \cdot \nu + \epsilon \Delta_\Sigma X
\]

This stabilizes the flow but changes the geometry.

\textbf{Unclear:} How does this affect the Penrose inequality?

\section{Approach 4: Ricci Flow on the Ambient Space}

\subsection{The Flow}

Hamilton's Ricci flow:
\[
    \partial_t g = -2\text{Ric}_g
\]

Under Ricci flow, scalar curvature satisfies:
\[
    \partial_t R = \Delta R + 2|\text{Ric}|^2
\]

\begin{theorem}[Hamilton]
If $R_g \geq 0$ initially and $M$ is compact, then $R_{g_t} \geq 0$ for all $t$.
\end{theorem}

\subsection{Application to Initial Data}

For initial data $(M^3, g, k)$, apply Ricci flow to $g$:
\[
    \partial_t g = -2\text{Ric}_g
\]
keeping $k$ fixed (but measuring it with respect to evolving $g$).

\textbf{Problem 1:} Constraints are violated (as before).

\textbf{Problem 2:} Even for pure $g$, the trapped condition 
$\theta^+ = H + \tr_\Sigma k$ changes unpredictably.

\textbf{Problem 3:} ADM mass may change or become ill-defined.

\section{Approach 5: Heat Kernel Regularization}

\subsection{Smoothing the Jump}

The problematic term is $[H]\delta_\Sigma$ in the scalar curvature.

\textbf{Idea:} Replace $\delta_\Sigma$ by the heat kernel $p_t(x, \Sigma)$:
\[
    R_\epsilon = R^{\text{reg}} + 2[H] \cdot p_\epsilon(x, \Sigma)
\]
where $p_\epsilon \to \delta_\Sigma$ as $\epsilon \to 0$.

\subsection{Analysis}

For small $\epsilon$, $p_\epsilon$ is approximately:
\[
    p_\epsilon(x, \Sigma) \approx \frac{1}{(4\pi\epsilon)^{1/2}} e^{-d(x,\Sigma)^2/(4\epsilon)}
\]

The regularized scalar curvature is:
\[
    R_\epsilon(x) = R^{\text{reg}}(x) + \frac{2[H]}{(4\pi\epsilon)^{1/2}} e^{-d(x,\Sigma)^2/(4\epsilon)}
\]

When $[H] < 0$, this creates a \textbf{negative well} near $\Sigma$.

\begin{lemma}
The total ``negative mass'' is preserved:
\[
    \int_M (R_\epsilon)_- \, dV \to [H] \cdot \Area(\Sigma) \quad \text{as } \epsilon \to 0
\]
\end{lemma}

\textbf{Conclusion:} Smoothing doesn't eliminate the negative contribution;
it just spreads it out.

\section{Approach 6: Parabolic Positive Mass Theorem}

\subsection{Time-Dependent PMT}

\begin{theorem}[Schoen-Yau, Parabolic Version?]
Let $(g_t)$ be a family of metrics with $R_{g_t} \geq 0$ and $M_{\ADM}(g_t)$ 
well-defined for all $t \in [0, T]$.

Is $M_{\ADM}(g_t)$ monotonic in $t$?
\end{theorem}

For Ricci flow with $R \geq 0$, the ADM mass satisfies:
\[
    \frac{d M_{\ADM}}{dt} \leq 0
\]
(mass decreases).

\subsection{Application}

If we could find a flow $(g_t, k_t)$ such that:
\begin{enumerate}
    \item Constraints are preserved
    \item $M_{\ADM}(g_t)$ is non-increasing
    \item $\tr_\Sigma k_t \to \tr_\Sigma k_t \geq 0$ as $t \to T$
\end{enumerate}

Then we could prove the Penrose inequality by flowing to the favorable case.

\textbf{The problem:} No such flow is known to exist!

Requirements (1) and (3) are in tension:
\begin{itemize}
    \item Preserving constraints tightly couples $g$ and $k$
    \item Changing the sign of $\tr_\Sigma k$ requires specific, non-generic evolution
\end{itemize}

\begin{tcolorbox}[colback=red!10, colframe=red!75!black, title=\textbf{Conclusion: Parabolic Methods}]
\textbf{Summary:} All parabolic regularization approaches face fundamental issues:

\begin{enumerate}
    \item \textbf{Constraint violation:} Evolving $k$ alone breaks constraints
    \item \textbf{Coupled evolution:} Preserving constraints doesn't improve sign
    \item \textbf{Mass change:} Most flows change ADM mass unpredictably
    \item \textbf{Smoothing doesn't help:} Negative mass spreads out but doesn't vanish
\end{enumerate}

\textbf{Status:} No parabolic method resolves the unconditional case.
\end{tcolorbox}

\end{document}
