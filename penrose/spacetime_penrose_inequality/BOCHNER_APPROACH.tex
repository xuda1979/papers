% =========================================================================
%     BOCHNER TECHNIQUES AND INTEGRAL IDENTITIES
%
%     Using Bochner-Weitzenböck formulas for the Penrose inequality
%
%     Author: Da Xu
%     Date: December 2025
% =========================================================================

\documentclass[12pt]{article}
\usepackage{amsmath,amsthm,amssymb}
\usepackage{mathrsfs}
\usepackage{tcolorbox}

\theoremstyle{plain}
\newtheorem{theorem}{Theorem}[section]
\newtheorem{lemma}[theorem]{Lemma}
\newtheorem{proposition}[theorem]{Proposition}
\newtheorem{corollary}[theorem]{Corollary}

\theoremstyle{definition}
\newtheorem{definition}[theorem]{Definition}
\newtheorem{remark}[theorem]{Remark}

\newcommand{\ADM}{\mathrm{ADM}}
\newcommand{\tr}{\mathrm{tr}}
\newcommand{\Div}{\mathrm{div}}
\newcommand{\Area}{\mathrm{Area}}
\newcommand{\Ric}{\mathrm{Ric}}

\title{\textbf{Bochner Techniques for the Penrose Inequality}}
\author{Da Xu}
\date{December 2025}

\begin{document}
\maketitle

\section{Bochner-Weitzenböck Identities}

\subsection{The Basic Identity}

For a function $f$ on a Riemannian manifold:
\[
    \frac{1}{2}\Delta|\nabla f|^2 = |\nabla^2 f|^2 + \langle \nabla f, \nabla \Delta f \rangle + \Ric(\nabla f, \nabla f)
\]

Integrating and using $R \geq 0$:
\[
    \int_M |\nabla^2 f|^2 \, dV \leq \int_M \langle \nabla f, \nabla \Delta f \rangle \, dV
\]

\subsection{Application to Harmonic Functions}

For $\Delta f = 0$:
\[
    \int_M |\nabla^2 f|^2 \, dV \leq 0
\]

With $R \geq 0$ and $\Ric \geq 0$, this implies $\nabla^2 f = 0$ (f is affine).

\section{Application to Initial Data}

\subsection{The Green's Function Approach}

Let $G(x, \Sigma)$ be the Green's function with pole on $\Sigma$:
\[
    \Delta G = 0 \quad \text{on } M \setminus \Sigma
\]
with $G \to 0$ at infinity and appropriate blow-up at $\Sigma$.

\subsection{The Energy Identity}

\begin{lemma}
For the Green's function $G$:
\[
    \int_M |\nabla G|^2 \, dV = \text{(boundary contribution from } \Sigma\text{)}
\]
\end{lemma}

The ADM mass can be expressed as:
\[
    M_{\ADM} = \lim_{r \to \infty} \frac{r^2}{4\pi} \int_{S_r} \partial_\nu G \, dA + \text{(correction terms)}
\]

\section{Bochner on the Jang Surface}

\subsection{Setup}

On the Jang surface $(\bar{M}, \bar{g})$:
\[
    R_{\bar{g}} = R^{\text{reg}} + 2[H]\delta_\Sigma
\]

Let $u$ be a harmonic function on $\bar{M} \setminus \Sigma$.

\subsection{The Bochner Identity with Boundary}

\[
    \int_{\bar{M}} (|\nabla^2 u|^2 + \Ric_{\bar{g}}(\nabla u, \nabla u)) \, dV = 
    \int_{\partial\bar{M}} \langle \nabla u, \nabla_\nu \nabla u \rangle \, dA
\]

Near $\Sigma$, the Ricci curvature has a delta contribution:
\[
    \Ric_{\bar{g}} = \Ric^{\text{reg}} + [H] \cdot (\text{surface terms}) \cdot \delta_\Sigma
\]

\subsection{The Signed Integral}

\[
    \int_{\bar{M}} \Ric_{\bar{g}}(\nabla u, \nabla u) \, dV = 
    \int_{\bar{M}} \Ric^{\text{reg}}(\nabla u, \nabla u) \, dV + [H] \int_\Sigma |\nabla u|^2 \, dA
\]

When $[H] < 0$:
\[
    [H] \int_\Sigma |\nabla u|^2 \, dA < 0
\]

This negative contribution cannot be controlled by the regular part!

\section{Attempt: Weighted Bochner}

\subsection{Idea}

Use a weight function $w > 0$ to balance the negative contribution:
\[
    \int_{\bar{M}} w \cdot \Ric_{\bar{g}}(\nabla u, \nabla u) \, dV
\]

If $w$ is small near $\Sigma$, the negative contribution is suppressed.

\subsection{The Modified Identity}

With weight $w$:
\[
    \int_{\bar{M}} w(|\nabla^2 u|^2 + \Ric(\nabla u, \nabla u)) \, dV + \text{(gradient terms)} = \cdots
\]

The gradient terms involve $\nabla w$, creating cross-terms.

\textbf{Problem:} Making $w$ small near $\Sigma$ requires $|\nabla w|$ large near $\Sigma$,
which creates other bad terms.

\section{The Stress-Energy Approach}

\subsection{Definition}

The stress-energy tensor for a function $f$:
\[
    T_{ij}[f] = \partial_i f \partial_j f - \frac{1}{2}|\nabla f|^2 g_{ij}
\]

\subsection{Divergence Identity}

\[
    \nabla^j T_{ij}[f] = (\Delta f) \partial_i f
\]

For harmonic $f$ ($\Delta f = 0$):
\[
    \Div T[f] = 0
\]

\subsection{Application to Mass}

The ADM mass can be written as:
\[
    M_{\ADM} = \lim_{r \to \infty} \int_{S_r} (T_{ij}[f] - \text{(background)})(\nu^i e^j) \, dA
\]
for appropriate choice of $f$ (related to coordinate functions).

\subsection{With Trapped Surface}

In the presence of $\Sigma$ with $[H] < 0$:
\[
    \int_M \nabla^j T_{ij}[f] \, dV = \int_\Sigma [T_{ij}]\nu^j \, dA
\]

The jump in $T_{ij}$ depends on how $f$ behaves across $\Sigma$.

\textbf{Problem:} For smooth $f$, the jump is zero. But constructing $f$ with
controlled jump doesn't give the Penrose bound.

\section{Integral Identity from DEC}

\subsection{The DEC Integral}

From DEC: $\mu \geq |J|$, i.e.,
\[
    R + (\tr k)^2 - |k|^2 \geq 16\pi|J|
\]

Integrating:
\[
    \int_M R \, dV + \int_M ((\tr k)^2 - |k|^2) \, dV \geq 16\pi \int_M |J| \, dV
\]

\subsection{The Schoen-Yau Strategy}

Schoen-Yau's proof uses:
\[
    16\pi M_{\ADM} = \lim_{r \to \infty} \int_{S_r} (h_{ij,j} - h_{jj,i}) \nu^i \, dA
\]
where $h_{ij} = g_{ij} - \delta_{ij}$.

Combined with DEC integral:
\[
    M_{\ADM} \geq C \cdot (\text{topological term})
\]

\subsection{For Penrose}

We need:
\[
    M_{\ADM} \geq \sqrt{\frac{\Area(\Sigma_0)}{16\pi}}
\]

The DEC integral gives:
\[
    M_{\ADM} \geq C \int_M (R + \text{k-terms}) \, dV
\]

\textbf{Problem:} The integral $\int R \, dV$ doesn't directly give $\sqrt{\Area}$.

\section{The Generalized Pohozaev Identity}

\subsection{Definition}

For a PDE $Lu = f$ on a domain $\Omega$:
\[
    \int_\Omega (x \cdot \nabla u) Lu \, dV = \text{(boundary terms)} + \text{(bulk terms)}
\]

\subsection{Application}

For the constraint equations:
\[
    R + (\tr k)^2 - |k|^2 = 16\pi\mu
\]

Multiply by a test function $\phi$ and integrate:
\[
    \int_M (R + (\tr k)^2 - |k|^2)\phi \, dV = 16\pi \int_M \mu\phi \, dV
\]

\textbf{Problem:} This doesn't isolate the area of $\Sigma_0$ in a useful way.

\section{The Twisted Bochner Approach}

\subsection{Idea}

Use a ``twisted'' connection that incorporates $k$:
\[
    \tilde{\nabla}_i X^j = \nabla_i X^j + k_i^j X + \cdots
\]

\subsection{Twisted Bochner Identity}

For the twisted connection:
\[
    \tilde{\Delta}|\nabla f|^2 = 2|\tilde{\nabla}^2 f|^2 + 2\tilde{\Ric}(\nabla f, \nabla f) + \cdots
\]

The ``twisted Ricci'' $\tilde{\Ric}$ might incorporate DEC in a useful way.

\textbf{Problem:} The twisted connection doesn't have good geometric properties,
and the resulting identities are complicated without clear improvement.

\begin{tcolorbox}[colback=yellow!10, colframe=orange!75!black, title=\textbf{Key Observation}]
\textbf{Observation:} All Bochner-type identities produce integral inequalities
of the form:
\[
    \int_M (\text{positive terms}) \, dV + \int_\Sigma (\text{signed terms}) \, dA \geq 0
\]

For the Penrose inequality, we need:
\[
    M_{\ADM} - \sqrt{\frac{\Area(\Sigma_0)}{16\pi}} \geq 0
\]

But Bochner identities give:
\[
    M_{\ADM} - \text{(integral over } M\text{)} \geq 0
\]

The mismatch between ``integral over $M$'' and ``square root of area of $\Sigma$''
is fundamental and not resolved by any known identity.
\end{tcolorbox}

\begin{tcolorbox}[colback=red!10, colframe=red!75!black, title=\textbf{Conclusion: Bochner Methods}]
\textbf{Summary:} Bochner techniques fail because:

\begin{enumerate}
    \item \textbf{Delta contribution:} The $[H]\delta_\Sigma$ term creates uncontrolled
    negative contributions
    \item \textbf{Weighting:} Suppressing the bad term creates other bad terms
    \item \textbf{Wrong functional form:} Bochner gives integrals, Penrose needs
    square roots of area
    \item \textbf{No direct $\sqrt{\Area}$ identity:} No known integral identity
    has $\sqrt{\Area(\Sigma)}$ as a natural term
\end{enumerate}

\textbf{Status:} No Bochner method resolves the unconditional case.
\end{tcolorbox}

\end{document}
