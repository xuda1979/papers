% =========================================================================
%     NULL HYPERSURFACE AND RAYCHAUDHURI APPROACH
%
%     Direct methods using null geometry and focusing theorems
%
%     Author: Da Xu
%     Date: December 2025
% =========================================================================

\documentclass[12pt]{article}
\usepackage{amsmath,amsthm,amssymb}
\usepackage{mathrsfs}
\usepackage{tcolorbox}

\theoremstyle{plain}
\newtheorem{theorem}{Theorem}[section]
\newtheorem{lemma}[theorem]{Lemma}
\newtheorem{proposition}[theorem]{Proposition}
\newtheorem{corollary}[theorem]{Corollary}

\theoremstyle{definition}
\newtheorem{definition}[theorem]{Definition}
\newtheorem{remark}[theorem]{Remark}

\newcommand{\ADM}{\mathrm{ADM}}
\newcommand{\tr}{\mathrm{tr}}
\newcommand{\Div}{\mathrm{div}}
\newcommand{\Area}{\mathrm{Area}}
\newcommand{\NEC}{\mathrm{NEC}}
\newcommand{\DEC}{\mathrm{DEC}}

\title{\textbf{Null Hypersurface Approach to the Penrose Inequality}}
\author{Da Xu}
\date{December 2025}

\begin{document}
\maketitle

\section{Null Geometry Preliminaries}

\subsection{Null Expansions}

For a surface $\Sigma$ in spacetime with null normals $\ell$ (outgoing) and $n$ (ingoing):
\begin{align}
    \theta^+ &= \nabla_\mu \ell^\mu |_\Sigma = H + \tr_\Sigma k \quad \text{(outgoing expansion)} \\
    \theta^- &= \nabla_\mu n^\mu |_\Sigma = H - \tr_\Sigma k \quad \text{(ingoing expansion)}
\end{align}

\subsection{The Raychaudhuri Equation}

Along a null geodesic congruence with tangent $\ell$:
\[
    \frac{d\theta^+}{d\lambda} = -\frac{(\theta^+)^2}{2} - \sigma_+^2 - R_{\mu\nu}\ell^\mu \ell^\nu
\]
where:
\begin{itemize}
    \item $\sigma_+$ is the shear of the null congruence
    \item $R_{\mu\nu}\ell^\mu \ell^\nu \geq 0$ under NEC (null energy condition)
\end{itemize}

\begin{theorem}[Focusing Theorem]
Under NEC, if $\theta^+|_{\Sigma_0} < 0$, then $\theta^+$ decreases along outgoing
null geodesics and reaches $-\infty$ in finite affine parameter.
\end{theorem}

\section{The Null Penrose Inequality Strategy}

\subsection{Hawking's Original Argument}

Penrose's heuristic for the inequality:
\begin{enumerate}
    \item Start with trapped surface $\Sigma_0$ of area $A_0$
    \item Under cosmic censorship, an event horizon $\mathcal{H}^+$ forms
    \item The area of horizon cross-sections is non-decreasing (Hawking area theorem)
    \item At late times, the black hole settles to Kerr with $M_{\text{Bondi}} \geq \sqrt{A_\infty/(16\pi)}$
    \item Since $A_\infty \geq A_0$, we get $M \geq \sqrt{A_0/(16\pi)}$
\end{enumerate}

\textbf{Gap:} This requires cosmic censorship and global spacetime evolution.

\subsection{Initial Data Alone}

Can we use null geometry on initial data alone?

\textbf{Key Observation:} The null expansion $\theta^+$ is computable from initial data:
\[
    \theta^+ = H + \tr_\Sigma k
\]

But the \textbf{evolution} of $\theta^+$ requires the full spacetime!

\section{Approach 1: Null Cone Construction}

\subsection{Future Light Cone}

From $\Sigma_0$, construct the future light cone $\mathcal{C}^+$ by shooting out
null geodesics in the $\ell$ direction.

The area of cross-sections satisfies:
\[
    \frac{d(\log A)}{d\lambda} = \int_{\Sigma_\lambda} \theta^+ \, dA
\]

For trapped surfaces, $\theta^+ < 0$, so $\frac{dA}{d\lambda} < 0$.

\textbf{Area decreases along the light cone!}

\subsection{Past Light Cone}

Similarly, the past light cone $\mathcal{C}^-$ has area evolution:
\[
    \frac{d(\log A)}{d\lambda} = \int_{\Sigma_\lambda} \theta^- \, dA
\]

For trapped surfaces, $\theta^- < 0$, so area also decreases going into the past!

\textbf{Conclusion:} Neither null direction increases area from trapped surfaces.

\section{Approach 2: The Penrose Null Cone Mass}

\subsection{Definition}

Define a mass on null cones:
\[
    m_{\text{null}}(\mathcal{C}) = \frac{1}{4\pi} \oint_{\Sigma_\lambda} |\sigma_+|^2 \, dA \cdot (\text{area factor})
\]

This is related to the Bondi mass at null infinity.

\subsection{Monotonicity}

Under Einstein's equations with NEC:
\[
    \frac{d m_{\text{null}}}{d\lambda} = -\frac{1}{4\pi} \oint R_{\mu\nu}\ell^\mu\ell^\nu \, dA \leq 0
\]

The null mass is \textbf{non-increasing} along outgoing null geodesics.

\textbf{Problem:} This gives $m_{\text{null}}(\Sigma_\infty) \leq m_{\text{null}}(\Sigma_0)$,
which is the wrong direction!

We need mass to \textbf{increase} from trapped surface to infinity.

\section{Approach 3: Ingoing Null Hypersurface}

\subsection{Idea}

Instead of going outward, consider the \textbf{ingoing} null hypersurface from $\Sigma_0$.

\begin{lemma}
On the ingoing null hypersurface $\mathcal{N}^-$:
\[
    \frac{d\theta^-}{d\mu} = -\frac{(\theta^-)^2}{2} - \sigma_-^2 - R_{\mu\nu}n^\mu n^\nu
\]
where $\mu$ is affine parameter along $n$.
\end{lemma}

Since $\theta^-|_{\Sigma_0} < 0$ and RHS is negative, $\theta^-$ decreases further.

\textbf{The surface focuses and collapses inward.}

\subsection{Connecting to Initial Data}

The ingoing null hypersurface intersects the initial data slice at some surface $\Sigma_1$.

\textbf{Question:} Is $\Area(\Sigma_1) \geq \Area(\Sigma_0)$?

\textbf{Analysis:} Moving along $\mathcal{N}^-$:
\[
    \frac{d\Area}{d\mu} = \int \theta^- \, dA < 0
\]

Area decreases! So $\Area(\Sigma_1) < \Area(\Sigma_0)$.

\section{Approach 4: Double Null Foliation}

\subsection{Setup}

Foliate a neighborhood of $\Sigma_0$ by double null coordinates $(u, v)$:
\begin{itemize}
    \item $u = $ const are outgoing null hypersurfaces
    \item $v = $ const are ingoing null hypersurfaces
    \item $\Sigma_{u,v} = \{u\} \cap \{v\}$ are 2-surfaces
\end{itemize}

\subsection{The Hawking Mass in Double Null}

The Hawking mass of $\Sigma_{u,v}$:
\[
    m_H(u,v) = \sqrt{\frac{A(u,v)}{16\pi}} \left(1 + \frac{1}{4\pi} \int_{\Sigma_{u,v}} \theta^+ \theta^- \, dA \right)
\]

Note: $1 - \frac{1}{16\pi}\int H^2 = 1 + \frac{1}{4\pi}\int \theta^+\theta^- - \frac{1}{4\pi}\int (\tr k)^2$.

\subsection{Evolution of Hawking Mass}

\begin{lemma}
In double null coordinates:
\[
    \partial_u m_H = \text{(matter terms)} + \text{(shear terms)} + \text{(curvature terms)}
\]
\end{lemma}

The detailed calculation involves:
\[
    \partial_u \theta^+ = -\frac{(\theta^+)^2}{2} - \sigma_+^2 - 8\pi T_{\ell\ell}
\]
\[
    \partial_v \theta^- = -\frac{(\theta^-)^2}{2} - \sigma_-^2 - 8\pi T_{nn}
\]

Under DEC, these terms have definite signs, but...

\textbf{Problem:} The mass doesn't have simple monotonicity in both $u$ and $v$.

\section{Approach 5: The Bray-Hayward Mass}

\subsection{Definition}

A quasi-local mass for surfaces in spacetime:
\[
    m_{BH}(\Sigma) = \sqrt{\frac{A}{16\pi}} + \frac{1}{8\pi}\oint_\Sigma \theta^+ \theta^- \, dA \cdot (\text{factor})
\]

\subsection{Properties}

For trapped surfaces: $\theta^+ < 0$, $\theta^- < 0$, so $\theta^+ \theta^- > 0$.

This \textbf{adds} to the mass, not subtracts!

\begin{proposition}
For trapped surfaces:
\[
    m_{BH}(\Sigma_0) \geq \sqrt{\frac{\Area(\Sigma_0)}{16\pi}}
\]
\end{proposition}

\textbf{This is promising!}

But we need to show $m_{BH}$ equals $M_{\ADM}$ at infinity...

\subsection{Limit at Infinity}

\begin{lemma}
As $\Sigma_r \to \infty$ (large spheres):
\[
    \theta^+ \sim \frac{2}{r} - \frac{2M}{r^2} + O(r^{-3})
\]
\[
    \theta^- \sim -\frac{2}{r} - \frac{2M}{r^2} + O(r^{-3})
\]
\[
    \theta^+ \theta^- \sim -\frac{4}{r^2} + O(r^{-3})
\]
\end{lemma}

Computing $m_{BH}$ at infinity:
\[
    m_{BH}(S_r) = \sqrt{\frac{4\pi r^2}{16\pi}} + \frac{1}{8\pi} \cdot 4\pi r^2 \cdot \left(-\frac{4}{r^2}\right) + \cdots
\]

The calculation doesn't immediately give $M_{\ADM}$.

\section{Approach 6: Raychaudhuri-Based Monotonicity}

\subsection{The Integrated Raychaudhuri}

Integrate the Raychaudhuri equation along the light cone:
\[
    \theta^+(\lambda) - \theta^+(0) = -\int_0^\lambda \left[\frac{(\theta^+)^2}{2} + \sigma_+^2 + R_{\ell\ell}\right] d\lambda'
\]

All terms on RHS are non-positive, so $\theta^+$ is non-increasing.

\subsection{Area Bound}

From $\frac{d\log A}{d\lambda} = \theta^+$:
\[
    \log A(\lambda) - \log A(0) = \int_0^\lambda \theta^+(s) \, ds
\]

If $\theta^+(0) < 0$ and $\theta^+$ decreases, then:
\[
    A(\lambda) < A(0) \cdot e^{\theta^+(0) \cdot \lambda}
\]

Area decays exponentially!

\textbf{Key Issue:} There's no mechanism to convert this to a mass bound.

\begin{tcolorbox}[colback=red!10, colframe=red!75!black, title=\textbf{Conclusion: Null Methods}]
\textbf{Summary:} Null hypersurface methods face the fundamental issue that:

\begin{enumerate}
    \item Outgoing null expansions: Area decreases from trapped surfaces
    \item Ingoing null expansions: Also decreasing (both null directions are ``bad'')
    \item Hawking/Bondi mass: Monotonic in wrong direction
    \item Quasi-local masses: Don't have clean limit to ADM mass
\end{enumerate}

\textbf{Core Problem:} The trapped condition means BOTH null directions
lead to area decrease. This is the definition of a trapped surface!

\textbf{Status:} Pure null methods cannot prove the unconditional Penrose inequality
without additional global structure (cosmic censorship).
\end{tcolorbox}

\end{document}
