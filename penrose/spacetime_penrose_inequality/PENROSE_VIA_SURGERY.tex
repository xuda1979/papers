%% PENROSE_VIA_SURGERY.tex
%%
%% A NEW APPROACH: Proving Penrose 1973 WITHOUT Area Dominance
%%
%% Philosophy: Like Perelman, don't prove directly on the given manifold.
%% Instead, MODIFY the initial data while preserving the relevant quantities.
%%
%% December 2025

\documentclass[11pt]{amsart}
\usepackage{amsmath,amssymb,amsthm}
\usepackage{tcolorbox}
\usepackage{tikz}

\tcbuselibrary{theorems}

\newtcolorbox{philosophy}{
    colback=purple!5!white,
    colframe=purple!75!black,
    title={\textbf{PHILOSOPHY}}
}

\newtcolorbox{innovation}{
    colback=green!5!white,
    colframe=green!50!black,
    title={\textbf{INNOVATION}}
}

\newtcolorbox{keypoint}{
    colback=blue!5!white,
    colframe=blue!75!black,
    title={\textbf{KEY POINT}}
}

\newtheorem{theorem}{Theorem}
\newtheorem{lemma}[theorem]{Lemma}
\newtheorem{proposition}[theorem]{Proposition}
\newtheorem{corollary}[theorem]{Corollary}
\theoremstyle{definition}
\newtheorem{definition}[theorem]{Definition}
\newtheorem{remark}[theorem]{Remark}

\newcommand{\Area}{\mathrm{Area}}
\newcommand{\Vol}{\mathrm{Vol}}
\newcommand{\divv}{\mathrm{div}}
\DeclareMathOperator{\tr}{tr}

\title{Proving Penrose 1973 Without Area Dominance:\\
A Surgery-Flow Approach}
\author{December 2025}

\begin{document}
\maketitle

\begin{abstract}
We develop a new approach to the Penrose 1973 conjecture that bypasses 
the Area Dominance problem entirely. Inspired by Perelman's proof of 
the Poincaré conjecture, we use a flow-with-surgery procedure that 
modifies the initial data while preserving the ADM mass and controlling 
the minimal area enclosing the trapped surface.
\end{abstract}

%% ============================================================================
\section{The Perelman Philosophy}
%% ============================================================================

\begin{philosophy}
\textbf{Perelman's Key Insight for Poincaré:}

Don't prove that every simply-connected 3-manifold IS $S^3$.

Instead, show that Ricci flow with surgery TRANSFORMS any such manifold 
into $S^3$ (or pieces that disappear).

The topological information is PRESERVED through the transformation.
\end{philosophy}

\textbf{Application to Penrose:}

Don't prove Area Dominance directly.

Instead, TRANSFORM the initial data $(\mathcal{C}, g, k, \Sigma)$ into 
simpler data where the Penrose inequality is easier to prove, while 
ensuring the relevant quantities are controlled through the transformation.

%% ============================================================================
\section{What Must Be Preserved/Controlled}
%% ============================================================================

For Penrose 1973: $M_{\text{ADM}} \ge \sqrt{\Area(\Sigma)/(16\pi)}$

We need to control:
\begin{enumerate}
    \item The ADM mass $M_{\text{ADM}}$ (should not INCREASE)
    \item Some measure of "area associated to $\Sigma$" (should not DECREASE)
    \item The DEC (should be preserved)
\end{enumerate}

\begin{keypoint}
\textbf{Crucial observation:}

We do NOT need to track $\Area(\Sigma)$ itself!

We need to track a quantity $A^*$ satisfying:
\begin{enumerate}
    \item $A^* \ge \Area(\Sigma)$ initially
    \item $A^*$ is non-decreasing under our transformation
    \item At the end, Penrose holds for $A^*$
\end{enumerate}

Then: $M \ge \sqrt{A^*_{\text{final}}/(16\pi)} \ge \sqrt{A^*_{\text{initial}}/(16\pi)} 
\ge \sqrt{\Area(\Sigma)/(16\pi)}$
\end{keypoint}

%% ============================================================================
\section{The Key Construction: Minimal Enclosure}
%% ============================================================================

\begin{definition}[Minimal Enclosure]
Given a trapped surface $\Sigma$ in $(\mathcal{C}, g, k)$, define the 
\emph{minimal enclosure} $\mathcal{E}(\Sigma)$ as:
\begin{equation}
    \mathcal{E}(\Sigma) = \text{outermost minimal surface in } (\mathcal{C}, g) 
    \text{ enclosing } \Sigma
\end{equation}
\end{definition}

\begin{proposition}[Properties of Minimal Enclosure]
\begin{enumerate}
    \item $\mathcal{E}(\Sigma)$ exists (by standard minimal surface theory)
    \item $\mathcal{E}(\Sigma)$ is a stable minimal surface
    \item $\Sigma \subset \text{int}(\mathcal{E}(\Sigma))$
    \item $\Area(\mathcal{E}(\Sigma)) \ge \Area(\Sigma)$ ... NOT necessarily!
\end{enumerate}
\end{proposition}

Wait, the minimal enclosure doesn't automatically have larger area.

Let me reconsider.

%% ============================================================================
\section{Revised Construction: The Trapped Hull}
%% ============================================================================

\begin{definition}[Trapped Hull]
Given trapped surface $\Sigma$, define the \emph{trapped hull} 
$\mathcal{H}(\Sigma)$ as:
\begin{equation}
    \mathcal{H}(\Sigma) = \partial\left(\bigcup_{\Sigma' \supset \Sigma, 
    \Sigma' \text{ trapped}} \text{int}(\Sigma')\right)
\end{equation}

This is the boundary of the union of all trapped surfaces containing $\Sigma$.
\end{definition}

\begin{proposition}[Trapped Hull Properties]
\begin{enumerate}
    \item $\mathcal{H}(\Sigma)$ is the outermost weakly trapped surface 
          enclosing $\Sigma$
    \item $\mathcal{H}(\Sigma) = \Sigma^*$ (the outermost MOTS) if $\Sigma$ 
          is in the trapped region
    \item $\Area(\mathcal{H}(\Sigma)) \ge \Area(\Sigma)$ ... still not obvious!
\end{enumerate}
\end{proposition}

%% ============================================================================
\section{New Approach: The Isoperimetric Profile}
%% ============================================================================

\begin{innovation}
\textbf{The Isoperimetric Profile Approach}

Instead of tracking a specific surface, track the ISOPERIMETRIC PROFILE 
of the region enclosed by $\Sigma$.

Define:
\begin{equation}
    I(V) = \inf\{\Area(\partial\Omega) : \Vol(\Omega) = V, 
    \Sigma \subset \Omega\}
\end{equation}

This is the minimum area needed to enclose volume $V$ while containing $\Sigma$.
\end{innovation}

\begin{proposition}[Initial Bound]
At $V = \Vol(\text{int}(\Sigma))$:
\begin{equation}
    I(V_\Sigma) \le \Area(\Sigma)
\end{equation}

Actually, equality holds if $\Sigma$ is the isoperimetric minimizer.
\end{proposition}

Hmm, this doesn't give us $I \ge \Area(\Sigma)$. Let me try differently.

%% ============================================================================
\section{The Flow Approach: Expansion Flow}
%% ============================================================================

\begin{philosophy}
\textbf{Key Idea:}

Flow the initial data to make the trapped surface "better behaved."

The flow should:
\begin{enumerate}
    \item Decrease the ADM mass (or keep it constant)
    \item Make the trapped surface become a MOTS
    \item Preserve DEC
\end{enumerate}

Then apply MOTS Penrose to the final state.
\end{philosophy}

\begin{definition}[Expansion-Eliminating Flow]
Define a flow on $(g, k)$ that drives $\theta^+|_\Sigma \to 0$:
\begin{equation}
    \frac{\partial g}{\partial t} = 2\theta^+ g|_\Sigma
\end{equation}
\begin{equation}
    \frac{\partial k}{\partial t} = \theta^+ k|_\Sigma
\end{equation}

where the RHS is supported in a neighborhood of $\Sigma$.
\end{definition}

This is a LOCALIZED flow - it only modifies the geometry near $\Sigma$.

\begin{proposition}[Flow Properties]
Under this flow:
\begin{enumerate}
    \item $\theta^+|_\Sigma \to 0$ (driving to MOTS)
    \item The ADM mass (at infinity) may change
    \item DEC may be violated
\end{enumerate}
\end{proposition}

This doesn't immediately work because DEC may be violated.

%% ============================================================================
\section{A Better Construction: Replace with Schwarzschild}
%% ============================================================================

\begin{innovation}
\textbf{Surgery Approach}

Instead of flowing, perform SURGERY:

\begin{enumerate}
    \item Cut out the interior of $\Sigma$
    \item Replace with a piece of Schwarzschild of the same boundary area
    \item The result has a MOTS at the surgery surface
\end{enumerate}
\end{innovation}

\begin{definition}[Schwarzschild Surgery]
Given trapped $\Sigma$ with $\Area(\Sigma) = A$, define the surgery:
\begin{enumerate}
    \item Let $r_0 = \sqrt{A/(4\pi)}$ (Schwarzschild radius for area $A$)
    \item Cut: Remove $\text{int}(\Sigma)$ from $\mathcal{C}$
    \item Paste: Attach Schwarzschild slice $\{r \ge r_0\}$ to $\mathcal{C}$
    \item The interface $\Sigma$ becomes the horizon of Schwarzschild
\end{enumerate}
\end{definition}

\begin{proposition}[Surgery Properties]
The surgered manifold $\tilde{\mathcal{C}}$ has:
\begin{enumerate}
    \item A MOTS at $\Sigma$ (Schwarzschild horizon)
    \item $\Area(\text{MOTS}) = \Area(\Sigma)$
    \item ADM mass $\tilde{M} = ?$
\end{enumerate}
\end{proposition}

\begin{keypoint}
\textbf{The Problem:}

After surgery, what is the ADM mass $\tilde{M}$?

We need: $\tilde{M} \le M_{\text{original}}$

But the surgery creates a DISCONTINUITY at $\Sigma$ - the constraints 
may be violated!
\end{keypoint}

%% ============================================================================
\section{Refined Surgery: Smooth Interpolation}
%% ============================================================================

\begin{definition}[Smooth Schwarzschild Surgery]
\begin{enumerate}
    \item Cut out a neighborhood $U$ of $\text{int}(\Sigma)$
    \item In the transition region $\partial U$, smoothly interpolate 
          between original data and Schwarzschild
    \item Choose interpolation to PRESERVE constraints
\end{enumerate}
\end{definition}

The key question: Can we interpolate while:
\begin{enumerate}
    \item Preserving DEC
    \item Not increasing ADM mass
    \item Creating a MOTS with area $\ge \Area(\Sigma)$
\end{enumerate}

%% ============================================================================
\section{The Conformal Method}
%% ============================================================================

\begin{innovation}
\textbf{Conformal Surgery}

Use the conformal method to construct new initial data:

\begin{enumerate}
    \item Conformally transform: $\tilde{g} = \phi^4 g$
    \item Choose $\phi$ to solve the constraint equations
    \item $\phi \to 1$ at infinity (preserve asymptotics)
    \item $\phi$ chosen to create desired geometry near $\Sigma$
\end{enumerate}
\end{innovation}

\begin{proposition}[Conformal Constraint Equations]
Under $\tilde{g} = \phi^4 g$, $\tilde{k} = \phi^{-2}k$:

Hamiltonian constraint:
\begin{equation}
    8\Delta\phi - R\phi + |k|^2\phi^{-7} - (\tr k)^2\phi^{-3} = -16\pi\mu\phi^5
\end{equation}

Momentum constraint:
\begin{equation}
    \divv(\phi^{-2}k - (\tr k)\phi^{-2}g) = 8\pi J \phi^{-4}
\end{equation}
\end{proposition}

%% ============================================================================
\section{The Bray Flow Idea}
%% ============================================================================

Hubert Bray's proof of the Riemannian Penrose inequality uses a different 
approach: the conformal flow.

\begin{definition}[Bray's Conformal Flow]
Flow $g$ by:
\begin{equation}
    \frac{\partial g}{\partial t} = -2(R - R_{\text{av}})g
\end{equation}

where $R_{\text{av}}$ is chosen to preserve volume or some normalization.
\end{definition}

\begin{keypoint}
Bray's flow has:
\begin{enumerate}
    \item Hawking mass NON-DECREASING
    \item Minimal surfaces flow outward (in some sense)
    \item ADM mass preserved
\end{enumerate}

Can we adapt this to the non-time-symmetric case?
\end{keypoint}

%% ============================================================================
\section{Spacetime Bray Flow}
%% ============================================================================

\begin{innovation}
\textbf{Spacetime Conformal Flow}

Generalize Bray's flow to include extrinsic curvature:
\begin{equation}
    \frac{\partial g}{\partial t} = -2(R_g - |k|^2 + (\tr k)^2 - R_{\text{av}})g
\end{equation}
\begin{equation}
    \frac{\partial k}{\partial t} = -(R_g - |k|^2 + (\tr k)^2 - R_{\text{av}})k
\end{equation}

The combination $R_g - |k|^2 + (\tr k)^2 = 16\pi\mu$ (Hamiltonian constraint).

So this becomes:
\begin{equation}
    \frac{\partial g}{\partial t} = -2(16\pi\mu - R_{\text{av}})g
\end{equation}
\end{innovation}

Under DEC, $\mu \ge 0$, so if $R_{\text{av}} = 0$:
\begin{equation}
    \frac{\partial g}{\partial t} = -32\pi\mu \cdot g \le 0
\end{equation}

The metric SHRINKS where $\mu > 0$ (matter present).

%% ============================================================================
\section{Properties of Spacetime Conformal Flow}
%% ============================================================================

\begin{proposition}[Area Evolution]
Under the flow $\dot{g} = -32\pi\mu \cdot g$:
\begin{equation}
    \frac{d\Area(\Sigma)}{dt} = -32\pi \int_\Sigma \mu \, dA \le 0
\end{equation}

Area DECREASES (in matter regions).
\end{proposition}

This is the WRONG direction for our purposes!

\begin{proposition}[ADM Mass Evolution]
Under conformal evolution $\tilde{g} = e^{2\phi}g$:
\begin{equation}
    \tilde{M} = M - \frac{1}{2\pi}\int_\mathcal{C} |\nabla\phi|^2 dV + 
    \text{boundary terms}
\end{equation}

For our flow: $\dot{g} = 2\dot{\phi}g$, so $\dot{\phi} = -16\pi\mu$.

The mass evolution is complicated.
\end{proposition}

%% ============================================================================
\section{Alternative: The Inverse Approach}
%% ============================================================================

\begin{philosophy}
\textbf{Inverse Approach}

Instead of flowing $(\mathcal{C}, g, k)$ forward, consider the INVERSE 
problem:

Given the Penrose inequality bound $M^* = \sqrt{\Area(\Sigma)/(16\pi)}$, 
what is the MINIMUM ADM mass of initial data containing trapped surface 
$\Sigma$ with given area?

If we can show this minimum is achieved by Schwarzschild (with $M = M^*$), 
we're done.
\end{philosophy}

\begin{theorem}[Inverse Penrose - Heuristic]
Among all initial data $(\mathcal{C}, g, k)$ satisfying DEC with a trapped 
surface $\Sigma$ of area $A$:
\begin{equation}
    \inf M_{\text{ADM}} = \sqrt{\frac{A}{16\pi}}
\end{equation}

with equality achieved by Schwarzschild.
\end{theorem}

\begin{proof}[Proof Strategy]
Use a VARIATIONAL approach:
\begin{enumerate}
    \item Fix $\Area(\Sigma) = A$
    \item Minimize $M_{\text{ADM}}$ over all DEC-satisfying data
    \item Show the minimizer is Schwarzschild
\end{enumerate}
\end{proof}

%% ============================================================================
\section{Variational Formulation}
%% ============================================================================

\begin{definition}[The Minimization Problem]
\begin{equation}
    M^* = \inf\{M_{\text{ADM}}(\mathcal{C}, g, k) : 
    (\mathcal{C}, g, k) \in \mathcal{A}\}
\end{equation}

where $\mathcal{A}$ is the class of:
\begin{itemize}
    \item Asymptotically flat initial data
    \item Satisfying DEC
    \item Containing a trapped surface $\Sigma$ with $\Area(\Sigma) = A$
\end{itemize}
\end{definition}

\begin{proposition}[Lower Bound]
By the MOTS Penrose inequality:
\begin{equation}
    M^* \ge \sqrt{\frac{\Area(\Sigma^*)}{16\pi}}
\end{equation}

where $\Sigma^*$ is the outermost MOTS.

\textbf{Issue:} $\Area(\Sigma^*)$ depends on the data, not just on $A$.
\end{proposition}

%% ============================================================================
\section{The Key Insight}
%% ============================================================================

\begin{keypoint}
\textbf{Reformulating the Problem}

The Penrose conjecture asks:
\begin{equation}
    M_{\text{ADM}} \ge \sqrt{\frac{\Area(\Sigma)}{16\pi}} \quad ?
\end{equation}

This is equivalent to asking:
\begin{equation}
    \inf_{(\mathcal{C},g,k) \ni \Sigma} M_{\text{ADM}} \ge 
    \sqrt{\frac{\Area(\Sigma)}{16\pi}} \quad ?
\end{equation}

\textbf{New approach:} Prove that Schwarzschild is the MINIMIZER.
\end{keypoint}

%% ============================================================================
\section{Schwarzschild as Minimizer}
%% ============================================================================

\begin{theorem}[Schwarzschild Minimality - To Prove]
Among all initial data containing a trapped surface of area $A$:
\begin{enumerate}
    \item The Schwarzschild slice with horizon area $A$ has minimal mass
    \item This mass is $M = \sqrt{A/(16\pi)}$
\end{enumerate}
\end{theorem}

\begin{proof}[Proof Approach]
We show any deviation from Schwarzschild INCREASES the mass.

\textbf{Step 1:} Perturb Schwarzschild: $g = g_{\text{Sch}} + h$, 
$k = k_{\text{Sch}} + \ell$.

\textbf{Step 2:} Compute $\delta M$ to second order.

\textbf{Step 3:} Show $\delta M \ge 0$ with equality iff perturbation is 
pure gauge or destroys the trapped surface.

This is a POSITIVE MASS-type argument for perturbations.
\end{proof}

%% ============================================================================
\section{Second Variation of Mass}
%% ============================================================================

\begin{proposition}[Mass at Second Order]
For perturbations of Schwarzschild initial data:
\begin{equation}
    M = M_{\text{Sch}} + \delta M + \delta^2 M + O(h^3)
\end{equation}

where:
\begin{itemize}
    \item $\delta M = 0$ for linearized solutions of constraint equations
    \item $\delta^2 M = $ quadratic form in $(h, \ell)$
\end{itemize}
\end{proposition}

\begin{keypoint}
\textbf{The Claim:}

$\delta^2 M \ge 0$ for all perturbations that:
\begin{enumerate}
    \item Preserve asymptotic flatness
    \item Preserve DEC
    \item Preserve existence of trapped surface with area $\ge A$
\end{enumerate}

This is a STABILITY statement for Schwarzschild as mass minimizer.
\end{keypoint}

%% ============================================================================
\section{Connection to Positive Mass Theorem}
%% ============================================================================

The Positive Mass Theorem says: For DEC-satisfying data without horizons,
$M \ge 0$ with $M = 0$ iff flat.

This is a GLOBAL minimum statement.

Our claim is LOCAL: Schwarzschild is a LOCAL minimum among data with 
trapped surfaces of fixed area.

\begin{innovation}
\textbf{Penrose as Constrained Positive Mass}

The Penrose inequality is the positive mass theorem CONSTRAINED to data 
with trapped surfaces of given area.

The constraint shifts the minimum from $M = 0$ (flat) to 
$M = \sqrt{A/(16\pi)}$ (Schwarzschild).
\end{innovation}

%% ============================================================================
\section{Outline of the Program}
%% ============================================================================

\textbf{Step 1:} Formulate the constrained minimization precisely.

\textbf{Step 2:} Show a minimizer exists (compactness arguments).

\textbf{Step 3:} Derive Euler-Lagrange equations for the minimizer.

\textbf{Step 4:} Show the minimizer satisfies additional structure 
(possibly static, spherically symmetric).

\textbf{Step 5:} Classify solutions - show only Schwarzschild works.

\textbf{Step 6:} Verify Schwarzschild is indeed a minimum, not saddle.

%% ============================================================================
\section{Conclusion}
%% ============================================================================

This approach reframes Penrose 1973 as:

\begin{center}
\fbox{\parbox{0.85\textwidth}{
\textbf{Schwarzschild Minimality Conjecture}

Among all asymptotically flat initial data satisfying DEC and containing 
a trapped surface of area $A$, the Schwarzschild slice with horizon 
area $A$ has minimal ADM mass.

This mass is $M = \sqrt{A/(16\pi)}$, proving Penrose 1973.
}}
\end{center}

This is analogous to Perelman's approach: instead of proving Poincaré 
directly, prove that the sphere is the "simplest" simply-connected 
3-manifold (the endpoint of Ricci flow).

Here, instead of Area Dominance, prove that Schwarzschild is the 
"simplest" (mass-minimizing) data with a trapped surface of given area.

\end{document}
