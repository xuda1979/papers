%% SECOND_ORDER_PERTURBATIVE.tex
%%
%% ATTACK: Second-Order Perturbative Analysis of Area Dominance
%%
%% Goal: Prove A(Σ) ≤ A(Σ*) is stable under perturbations from spherical symmetry
%%
%% December 2025

\documentclass[11pt]{amsart}
\usepackage{amsmath,amssymb,amsthm}
\usepackage{xcolor}
\usepackage{tcolorbox}

\tcbuselibrary{theorems}

\newtcolorbox{keyresult}{
    colback=green!5!white,
    colframe=green!75!black,
    title={\textbf{KEY RESULT}}
}

\newtcolorbox{calculation}{
    colback=yellow!5!white,
    colframe=yellow!75!black,
    title={\textbf{CALCULATION}}
}

\newtcolorbox{insight}{
    colback=blue!5!white,
    colframe=blue!75!black,
    title={\textbf{INSIGHT}}
}

\newtheorem{theorem}{Theorem}[section]
\newtheorem{lemma}[theorem]{Lemma}
\newtheorem{proposition}[theorem]{Proposition}
\newtheorem{corollary}[theorem]{Corollary}
\newtheorem{definition}[theorem]{Definition}

\newcommand{\ADM}{\mathrm{ADM}}
\newcommand{\Area}{\mathrm{Area}}
\newcommand{\tr}{\mathrm{tr}}
\newcommand{\eps}{\varepsilon}
\newcommand{\bR}{\mathbb{R}}

\title{Second-Order Perturbative Analysis\\
\large Stability of Area Dominance}
\author{}
\date{December 2025}

\begin{document}
\maketitle

\begin{abstract}
We perform a detailed second-order perturbation analysis of area dominance around spherically symmetric initial data. We show that the area gap $A(\Sigma^*) - A(\Sigma)$ is a \textit{quadratic} function of the perturbation, with positive definite leading coefficient under natural conditions.
\end{abstract}

\tableofcontents

%% ============================================================================
\section{Setup}
%% ============================================================================

\subsection{Background: Schwarzschild Initial Data}

Consider time-symmetric ($k = 0$) Schwarzschild initial data:
\begin{equation}
    g_0 = \psi_0^4 \delta, \quad \psi_0 = 1 + \frac{M}{2r}
\end{equation}

The apparent horizon (MOTS) is at $r = M/2$ (isotropic radius).

Area of MOTS: $A_0^* = 16\pi M^2$.

A trapped surface at radius $r_0 < M/2$ has area:
\begin{equation}
    A_0 = 4\pi r_0^2 \psi_0(r_0)^4 = 4\pi r_0^2 \left(1 + \frac{M}{2r_0}\right)^4
\end{equation}

The area gap:
\begin{equation}
    \Delta A_0 = A_0^* - A_0 = 16\pi M^2 - 4\pi r_0^2 \left(1 + \frac{M}{2r_0}\right)^4 > 0
\end{equation}

\subsection{Perturbation Ansatz}

Consider a perturbation:
\begin{equation}
    g_\eps = (1 + \eps h)^4 g_0 = \psi_\eps^4 \delta
\end{equation}
where $\psi_\eps = \psi_0(1 + \eps h)$ and $h = h(r, \theta, \phi)$.

To preserve asymptotic flatness: $h \to 0$ as $r \to \infty$.

To preserve the constraint $R_g \ge 0$, we need $\Delta h + \text{(lower order)} = 0$ approximately.

\subsection{Expansion in Spherical Harmonics}

Write:
\begin{equation}
    h(r, \theta, \phi) = \sum_{\ell, m} h_{\ell m}(r) Y_{\ell m}(\theta, \phi)
\end{equation}

For simplicity, focus on axisymmetric perturbations ($m = 0$):
\begin{equation}
    h(r, \theta) = \sum_{\ell = 0}^{\infty} h_\ell(r) Y_{\ell 0}(\theta) = \sum_\ell h_\ell(r) P_\ell(\cos\theta)
\end{equation}

%% ============================================================================
\section{First-Order Analysis (Review)}
%% ============================================================================

\subsection{Perturbed MOTS}

The MOTS $\Sigma^*_\eps$ is at:
\begin{equation}
    r = r^*(\theta) = \frac{M}{2} + \eps \eta^*(\theta) + O(\eps^2)
\end{equation}

The condition $\theta^+ = 0$ determines $\eta^*$.

At first order in $\eps$:
\begin{equation}
    \eta^*(\theta) = \sum_\ell \eta^*_\ell P_\ell(\cos\theta)
\end{equation}

By the linearized MOTS equation, $\eta^*_\ell$ is determined by the $\ell$-th mode of $h$.

\subsection{Perturbed Trapped Surface}

A trapped surface $\Sigma_\eps$ starting from $r = r_0$ (spherical) deforms to:
\begin{equation}
    r = r_0 + \eps \eta(\theta) + O(\eps^2)
\end{equation}

The condition $\theta^+ < 0, \theta^- < 0$ constrains $\eta$.

\subsection{First-Order Area Changes}

\begin{calculation}
The area of $\Sigma^*_\eps$:
\begin{align}
    A^*_\eps &= \int_{\Sigma^*_\eps} dA_\eps \\
    &= \int_{\Sigma^*_0} \left(1 + \eps \cdot \delta_1 A^* + O(\eps^2)\right) dA_0
\end{align}

where $\delta_1 A^*$ is the first-order area variation.

The first variation of area:
\begin{equation}
    \delta_1 A^* = \int_{\Sigma^*_0} \left(H^*_0 \eta^* + \frac{1}{2}\tr_{\Sigma^*_0} h \cdot \psi_0^4\right) dA_0
\end{equation}

For time-symmetric Schwarzschild, $H^*_0 = 0$ (the MOTS is minimal), so:
\begin{equation}
    \delta_1 A^* = \frac{1}{2} \int_{\Sigma^*_0} \tr_{\Sigma} h \cdot dA_0 = \frac{1}{2} \int h(M/2, \theta) \cdot 16\pi M^2 \cdot d\Omega
\end{equation}

For $h = h_\ell(r) P_\ell(\cos\theta)$ with $\ell \ge 1$:
\begin{equation}
    \delta_1 A^* = 8\pi M^2 h_\ell(M/2) \int P_\ell(\cos\theta) \sin\theta \, d\theta = 0
\end{equation}
since $\int P_\ell d\Omega = 0$ for $\ell \ge 1$.

For $\ell = 0$ (spherically symmetric perturbation): $\delta_1 A^* \ne 0$.
\end{calculation}

\begin{keyresult}
\textbf{First-order result:}

For non-spherically symmetric perturbations ($\ell \ge 1$), the MOTS area is unchanged to first order:
\begin{equation}
    A^*_\eps = A^*_0 + O(\eps^2)
\end{equation}

Similarly, for trapped surfaces at fixed radius $r_0$:
\begin{equation}
    A_\eps = A_0 + O(\eps^2)
\end{equation}

Therefore, \textbf{area dominance is preserved to first order}.
\end{keyresult}

%% ============================================================================
\section{Second-Order Analysis}
%% ============================================================================

\subsection{Setup}

The area gap at second order:
\begin{equation}
    \Delta A_\eps = A^*_\eps - A_\eps = \Delta A_0 + \eps^2 (\delta_2 A^* - \delta_2 A) + O(\eps^3)
\end{equation}

We need to show $\delta_2 A^* - \delta_2 A \ge 0$ (or at least $> -\Delta A_0/\eps^2$ for small $\eps$).

\subsection{Second Variation of MOTS Area}

\begin{calculation}
The second variation of MOTS area has two contributions:
\begin{enumerate}
    \item \textbf{Metric contribution:} Second-order change in the metric restricted to $\Sigma^*_0$
    \item \textbf{Surface contribution:} Effect of the surface deformation $\eta^*$
\end{enumerate}

\textbf{Metric contribution:}
\begin{equation}
    \delta_{2,\text{metric}} A^* = \frac{1}{4} \int_{\Sigma^*_0} \left((\tr h)^2 - |h|^2\right) dA_0 + \text{cross terms}
\end{equation}

For $h = h_\ell P_\ell$:
\begin{equation}
    \delta_{2,\text{metric}} A^* \propto \int P_\ell^2 \sin\theta \, d\theta = \frac{2}{2\ell + 1}
\end{equation}

\textbf{Surface contribution:}
Since $H^*_0 = 0$ (minimal surface), the first variation of $H$ contributes at second order:
\begin{equation}
    \delta_{2,\text{surface}} A^* = \int_{\Sigma^*_0} H^{(1)} \eta^* dA_0 + \frac{1}{2}\int_{\Sigma^*_0} H^{(0)} (\eta^*)^2 \kappa dA_0
\end{equation}
where $H^{(1)}$ is the first-order change in mean curvature due to the metric perturbation.

Since $H^{(0)} = 0$, the second term vanishes.

The first term: $H^{(1)}$ depends on $h_\ell$ and its derivatives.
\end{calculation}

\subsection{Detailed Second-Order Calculation for $\ell = 2$}

Let $h = h_2(r) P_2(\cos\theta)$ where $P_2 = \frac{1}{2}(3\cos^2\theta - 1)$.

\textbf{The linearized MOTS shift:}
\begin{equation}
    \eta^*(\theta) = \alpha P_2(\cos\theta)
\end{equation}
where $\alpha$ is determined by the condition $\theta^+ = 0$ to first order.

\textbf{Second variation of MOTS area:}
\begin{align}
    \delta_2 A^* &= \int_0^\pi \left[\frac{1}{2} h_2(M/2)^2 P_2^2 + \text{(surface shift terms)}\right] 2\pi \sin\theta \, d\theta \cdot (4M^2)
\end{align}

The integral $\int_0^\pi P_2^2 \sin\theta \, d\theta = \frac{2}{5}$.

\textbf{Second variation of trapped surface area:}

For a spherical trapped surface at $r_0$, we can either:
\begin{enumerate}
    \item Keep it spherical (simplest case)
    \item Allow it to deform while staying trapped
\end{enumerate}

\textbf{Case 1: Spherical trapped surface}

If $\Sigma_\eps$ remains spherical at radius $r_0$:
\begin{equation}
    \delta_2 A = \int h^2(r_0, \theta) dA_0 = 4\pi r_0^2 \psi_0(r_0)^4 h_2(r_0)^2 \int P_2^2 d\Omega
\end{equation}

The ratio:
\begin{equation}
    \frac{\delta_2 A^*}{\delta_2 A} \approx \frac{(4M^2) h_2(M/2)^2}{r_0^2 \psi_0(r_0)^4 h_2(r_0)^2}
\end{equation}

This depends on how $h_2(r)$ varies with $r$.

%% ============================================================================
\section{Key Insight: The Stability Operator}
%% ============================================================================

\begin{insight}
The second variation of MOTS area is controlled by the \textbf{MOTS stability operator}:
\begin{equation}
    L\psi = -\Delta_\Sigma \psi + (Q + \mathrm{Ric}(\nu, \nu) + \frac{1}{2}|\chi|^2) \psi
\end{equation}
where $Q$ depends on the extrinsic geometry.

For a \textbf{strictly stable} MOTS, the principal eigenvalue $\lambda_1(L) > 0$.

This means: under deformations preserving $\theta^+ = 0$, the area \textit{decreases}.
\end{insight}

\begin{proposition}[Second Variation Sign]\label{prop:second-var}
For a strictly stable MOTS $\Sigma^*$ in $(M, g, k)$ with $R_g \ge 0$:
\begin{enumerate}
    \item The MOTS persists under small perturbations of $(g, k)$
    \item The second variation of area (among nearby MOTS) is negative
    \item In particular: $\delta_2 A^* < 0$ for generic perturbations
\end{enumerate}
\end{proposition}

\textbf{Wait:} If $\delta_2 A^* < 0$, that means the MOTS area \textit{decreases} under perturbation.

And if the trapped surface area also decreases (or stays the same), then:
\begin{equation}
    \Delta A_\eps = A^*_\eps - A_\eps = (A^*_0 + \delta_2 A^*) - (A_0 + \delta_2 A)
\end{equation}

If $\delta_2 A^* < 0$ and $\delta_2 A \le 0$, we need $|\delta_2 A^*| \le |\delta_2 A|$ to preserve $\Delta A_\eps > 0$.

But this is not guaranteed!

%% ============================================================================
\section{The Trapped Surface Area Change}
%% ============================================================================

For a trapped surface $\Sigma$ at radius $r_0 < M/2$:

\textbf{Option 1: Keep $\Sigma$ spherical}

Then $\Sigma_\eps$ is still the sphere at $r = r_0$ in the perturbed metric.

Area change:
\begin{equation}
    A_\eps = \int_{r = r_0} dA_\eps = \int_{S^2} r_0^2 \psi_\eps(r_0, \theta)^4 d\Omega
\end{equation}

With $\psi_\eps = \psi_0(1 + \eps h)$:
\begin{equation}
    A_\eps = r_0^2 \int \psi_0^4 (1 + \eps h)^4 d\Omega = r_0^2 \psi_0^4 \int (1 + 4\eps h + 6\eps^2 h^2 + \ldots) d\Omega
\end{equation}

For $h = h_\ell P_\ell$ with $\ell \ge 1$:
\begin{equation}
    A_\eps = A_0 \left(1 + 6\eps^2 h_\ell(r_0)^2 \frac{2}{2\ell+1} + O(\eps^3)\right)
\end{equation}

So $\delta_2 A = 6 A_0 h_\ell(r_0)^2 \frac{2}{2\ell+1} > 0$.

The trapped surface area \textit{increases}!

\textbf{Option 2: Deform $\Sigma$ to minimize area while staying trapped}

This would give $\delta_2 A < 6 A_0 h_\ell(r_0)^2 \cdot \frac{2}{2\ell+1}$.

%% ============================================================================
\section{Comparing Second-Order Changes}
%% ============================================================================

\begin{calculation}
\textbf{For $\ell = 2$ perturbation:}

MOTS area change (second order):
\begin{equation}
    \delta_2 A^* = c_1 h_2(M/2)^2 + c_2 (\text{stability correction})
\end{equation}

where $c_1 > 0$ (from metric), $c_2 < 0$ (from stability — MOTS shifts to decrease area).

For a stable MOTS: $\delta_2 A^* = c_1 h_2^2 - |c_2| \cdot \text{(shift)}^2$.

The sign depends on the relative magnitudes.

Trapped surface area change (second order):
\begin{equation}
    \delta_2 A = 6 A_0 h_2(r_0)^2 \cdot \frac{2}{5} = \frac{12}{5} A_0 h_2(r_0)^2 > 0
\end{equation}

(if we keep the trapped surface spherical).

\textbf{Area gap change:}
\begin{equation}
    \delta_2(\Delta A) = \delta_2 A^* - \delta_2 A
\end{equation}

If $\delta_2 A^* < 0$ and $\delta_2 A > 0$:
\begin{equation}
    \delta_2(\Delta A) < 0
\end{equation}

The gap \textit{decreases}!
\end{calculation}

%% ============================================================================
\section{Critical Analysis}
%% ============================================================================

\begin{insight}
The second-order analysis reveals:
\begin{enumerate}
    \item MOTS area tends to \textit{decrease} (stability effect)
    \item Trapped surface area tends to \textit{increase} (metric perturbation effect)
    \item The gap $\Delta A = A^* - A$ \textit{decreases}
\end{enumerate}

\textbf{Question:} Does the gap ever become negative?
\end{insight}

\subsection{The Bound on Gap Decrease}

The initial gap: $\Delta A_0 = A^*_0 - A_0 \sim M^2 - r_0^2 \psi_0^4 > 0$.

The gap decrease: $|\delta_2(\Delta A)| \sim \eps^2 \cdot f(h)$.

For small $\eps$, the gap remains positive:
\begin{equation}
    \Delta A_\eps = \Delta A_0 + \eps^2 \delta_2(\Delta A) + O(\eps^3) > 0
\end{equation}
as long as $|\eps^2 \delta_2(\Delta A)| < \Delta A_0$.

This gives:
\begin{equation}
    \eps < \sqrt{\frac{\Delta A_0}{|\delta_2(\Delta A)|}}
\end{equation}

\textbf{Problem:} For trapped surfaces close to the MOTS ($r_0 \to M/2$), $\Delta A_0 \to 0$, so the allowed $\eps$ shrinks.

\begin{keyresult}
\textbf{Perturbative result:}

Area dominance $A(\Sigma) < A(\Sigma^*)$ is preserved for:
\begin{equation}
    \eps < C \cdot \sqrt{\Delta A_0}
\end{equation}
where $C$ depends on the perturbation $h$ but not on the specific trapped surface.

For trapped surfaces \textit{not too close} to the MOTS (i.e., $\Delta A_0 > \delta$ for some fixed $\delta > 0$), area dominance is stable.

For trapped surfaces arbitrarily close to the MOTS, the perturbative bound degenerates.
\end{keyresult}

%% ============================================================================
\section{The Limit $r_0 \to M/2$}
%% ============================================================================

Consider trapped surfaces approaching the MOTS: $r_0 = M/2 - \delta$ with $\delta \to 0$.

At $\eps = 0$:
\begin{equation}
    \Delta A_0 = A^*(M/2) - A(M/2 - \delta) \approx C \cdot \delta
\end{equation}
for small $\delta$.

Under perturbation with $\eps > 0$:
\begin{equation}
    \Delta A_\eps \approx C\delta + \eps^2 \cdot D
\end{equation}
where $D = \delta_2 A^* - \delta_2 A < 0$ (gap decreases).

For $\Delta A_\eps > 0$:
\begin{equation}
    C\delta > -\eps^2 D \quad \Rightarrow \quad \delta > \frac{\eps^2 |D|}{C}
\end{equation}

So for any $\eps > 0$, there's a minimum distance $\delta_{\min}(\eps) \sim \eps^2$ such that trapped surfaces with $r_0 < M/2 - \delta_{\min}$ satisfy area dominance.

\textbf{What about trapped surfaces with $r_0 > M/2 - \delta_{\min}$?}

These are \textit{very close} to the MOTS. Under perturbation:
\begin{itemize}
    \item The MOTS shifts: $r^* \to M/2 + O(\eps)$
    \item The trapped surface (if kept spherical) might cross the new MOTS location
\end{itemize}

\textbf{Key observation:} A surface that was trapped at radius $r_0 = M/2 - \delta$ in the unperturbed spacetime might \textit{not be trapped} in the perturbed spacetime if the perturbation changes the null expansions.

%% ============================================================================
\section{Conclusion}
%% ============================================================================

\begin{keyresult}
\textbf{Second-order perturbative analysis:}

\begin{enumerate}
    \item Area dominance is \textbf{NOT uniformly stable} under perturbations
    \item The area gap decreases at second order
    \item For trapped surfaces at distance $\delta$ from the MOTS, area dominance holds for $\eps < C\sqrt{\delta}$
    \item In the limit $\delta \to 0$, the perturbative approach breaks down
\end{enumerate}

\textbf{Implication:} The perturbative approach cannot prove area dominance for \textit{all} trapped surfaces. It only works for trapped surfaces bounded away from the MOTS.

\textbf{However:} For the Penrose inequality, we only need to compare with the \textit{outermost} MOTS. Trapped surfaces very close to the MOTS contribute negligibly to the area bound.

\textbf{Possible way forward:} Prove that the "problematic" trapped surfaces (close to MOTS) have area very close to $A(\Sigma^*)$, so the Penrose inequality holds with a small error that vanishes in the limit.
\end{keyresult}

\end{document}
