% =========================================================================
%     NOVEL APPROACH: INTRINSIC TRAPPED SURFACE GEOMETRY
%     
%     Using the Codazzi Structure of Doubly-Trapped Surfaces
%
%     Author: Da Xu
%     Date: December 13, 2025
% =========================================================================

\documentclass[12pt]{article}
\usepackage{amsmath,amsthm,amssymb}
\usepackage{tcolorbox}

\newtheorem{theorem}{Theorem}[section]
\newtheorem{lemma}[theorem]{Lemma}
\newtheorem{proposition}[theorem]{Proposition}
\newtheorem{corollary}[theorem]{Corollary}
\newtheorem{definition}[theorem]{Definition}
\newtheorem{conjecture}[theorem]{Conjecture}
\newtheorem{remark}[theorem]{Remark}

\newcommand{\ADM}{\mathrm{ADM}}
\newcommand{\tr}{\mathrm{tr}}

\title{\textbf{Novel Approach}\\[0.3cm]
\large Using Intrinsic Geometry of Trapped Surfaces}
\author{Da Xu}
\date{December 13, 2025}

\begin{document}
\maketitle

\begin{abstract}
We explore a genuinely novel approach to the Penrose inequality based on the 
intrinsic geometry of trapped surfaces. The key observation is that trapped 
surfaces satisfy strong constraints from both null directions simultaneously, 
which may provide enough structure to bypass the unfavorable jump problem.
\end{abstract}

%===========================================================================
\section{The Core Observation}
%===========================================================================

For a trapped surface $\Sigma$ in initial data $(M^3, g, k)$:

\textbf{Null expansions:}
\begin{align}
    \theta^+ &= H + \tr_\Sigma k \leq 0 \label{eq:thetaplus}\\
    \theta^- &= H - \tr_\Sigma k < 0 \label{eq:thetaminus}
\end{align}

\textbf{Universal constraint:} Adding \eqref{eq:thetaplus} and \eqref{eq:thetaminus}:
\begin{equation}
    H = \frac{\theta^+ + \theta^-}{2} < 0
    \label{eq:universal}
\end{equation}

\textbf{This is key:} Every trapped surface has strictly negative mean curvature 
in $(M, g)$, regardless of the sign of $\tr_\Sigma k$.

%===========================================================================
\section{The Inverse Mean Curvature Flow Perspective}
%===========================================================================

\subsection{IMCF in $(M, g)$}

The inverse mean curvature flow:
\begin{equation}
    \partial_t F = -\frac{\nu}{H}
\end{equation}
evolves surfaces in the direction of decreasing $|H|$ (outward when $H < 0$).

\textbf{Key property:} For trapped surfaces with $H < 0$, IMCF flows \emph{outward} 
(in the direction of increasing area).

The Hawking mass under IMCF:
\begin{equation}
    m_H[\Sigma] = \sqrt{\frac{A}{16\pi}}\left(1 - \frac{1}{16\pi}\int_\Sigma H^2 \, dA\right)
\end{equation}

\subsection{IMCF Monotonicity Formula}

Under IMCF:
\begin{equation}
    \frac{d m_H}{dt} = \sqrt{\frac{A}{16\pi}} \cdot \frac{1}{16\pi}\int_\Sigma \left(
        R_g + 2|\mathring{A}|^2 + \frac{(H')^2}{H^2} - H^2
    \right) \frac{dA}{|H|}
\end{equation}
where $H' = \Delta H + |A|^2 H + \text{Ric}(\nu,\nu)H$.

\textbf{Problem:} This requires $R_g \geq 0$, which the DEC does not give.

%===========================================================================
\section{New Idea: The Trapped Mass}
%===========================================================================

\subsection{Motivation}

The issue with standard Hawking mass is it only uses $H$. For trapped surfaces, 
we have \emph{two} pieces of information: $\theta^+ \leq 0$ AND $\theta^- < 0$.

\subsection{Definition of Trapped Mass}

\begin{definition}[Trapped Mass]
For a surface $\Sigma$ in $(M, g, k)$, define:
\begin{equation}
    m_T[\Sigma] = \sqrt{\frac{A}{16\pi}}\left(1 - \frac{1}{16\pi}\int_\Sigma 
        \theta^+ \theta^- \, dA \right)^{1/2}
\end{equation}
when the integrand makes the expression well-defined.
\end{definition}

\textbf{Properties:}
\begin{enumerate}
    \item For trapped surfaces: $\theta^+\theta^- \geq 0$, so the integral is 
    non-negative.
    \item For MOTS ($\theta^+ = 0$): $m_T = \sqrt{A/(16\pi)}$.
    \item For marginally inner trapped ($\theta^- = 0$, if exists): $m_T = \sqrt{A/(16\pi)}$.
    \item The correction term $\int \theta^+\theta^- dA$ captures the ``degree 
    of trapping.''
\end{enumerate}

\subsection{Relating to $H$ and $\tr_\Sigma k$}

We have:
\begin{equation}
    \theta^+\theta^- = (H + \tr_\Sigma k)(H - \tr_\Sigma k) = H^2 - (\tr_\Sigma k)^2
\end{equation}

So:
\begin{equation}
    m_T[\Sigma] = \sqrt{\frac{A}{16\pi}}\left(1 - \frac{1}{16\pi}\int_\Sigma 
        \left(H^2 - (\tr_\Sigma k)^2\right) dA \right)^{1/2}
\end{equation}

Compare to Hawking mass:
\begin{equation}
    m_H[\Sigma] = \sqrt{\frac{A}{16\pi}}\left(1 - \frac{1}{16\pi}\int_\Sigma H^2 \, dA\right)
\end{equation}

\textbf{Crucial observation:}
\begin{equation}
    m_T[\Sigma] \geq m_H[\Sigma]
\end{equation}
since we subtract the non-negative $(\tr_\Sigma k)^2$ from $H^2$.

%===========================================================================
\section{Evolution of Trapped Mass}
%===========================================================================

\subsection{General Formula}

Under any flow $\partial_t F = f\nu$ (with $\nu$ outward unit normal):

\begin{align}
    \frac{dA}{dt} &= -\int fH \, dA \\
    \frac{d}{dt}\int \theta^+\theta^- \, dA &= \int \left(
        \partial_t(\theta^+\theta^-) + fH\cdot\theta^+\theta^-
    \right) dA
\end{align}

The evolution of $\theta^\pm$ involves the constraint equations:

\begin{align}
    \partial_t \theta^+ &= -f\left(|A^+|^2 + G(\ell^+,\ell^+)\right) + \Delta_\Sigma f + \ldots \\
    \partial_t \theta^- &= -f\left(|A^-|^2 + G(\ell^-,\ell^-)\right) + \Delta_\Sigma f + \ldots
\end{align}

where $\ell^\pm$ are the null normals and $G = \text{Ric} - \frac{1}{2}Rg$ is the 
Einstein tensor.

Under DEC: $G(\ell^\pm, \ell^\pm) \geq 0$.

\subsection{IMCF Evolution}

For IMCF with $f = -1/H$ (outward when $H < 0$):

\begin{equation}
    \frac{d}{dt}(\theta^+\theta^-) = \frac{1}{|H|}\left(|A^+|^2 + G(\ell^+,\ell^+)\right)\theta^-
        + \frac{1}{|H|}\left(|A^-|^2 + G(\ell^-,\ell^-)\right)\theta^+ + \ldots
\end{equation}

For trapped surfaces with $\theta^+ \leq 0$ and $\theta^- < 0$:
\begin{itemize}
    \item First term: $\geq 0$ times $\theta^- < 0$, so $\leq 0$.
    \item Second term: $\geq 0$ times $\theta^+ \leq 0$, so $\leq 0$.
\end{itemize}

\textbf{Conclusion:} $\frac{d}{dt}(\theta^+\theta^-) \leq 0$ along IMCF!

\subsection{Implication}

If IMCF preserves the trapped condition, then:
\begin{equation}
    \int_{\Sigma_t} \theta^+\theta^- \, dA \leq \int_{\Sigma_0} \theta^+\theta^- \, dA
\end{equation}

Combined with area increase along IMCF, this suggests:
\begin{equation}
    m_T[\Sigma_t] \text{ is increasing along IMCF}
\end{equation}

%===========================================================================
\section{Critical Analysis: What Needs to be Verified}
%===========================================================================

\subsection{Does IMCF Stay in Trapped Region?}

\textbf{Issue:} We need the flow to stay trapped ($\theta^\pm \leq 0$) until 
reaching a MOTS or the asymptotic region.

This is NOT guaranteed. The flow could:
\begin{itemize}
    \item Exit the trapped region (surfaces become untrapped)
    \item Develop singularities before reaching useful boundary
\end{itemize}

\subsection{Weak IMCF}

Huisken-Ilmanen's weak IMCF exists but:
\begin{itemize}
    \item Requires $R_g \geq 0$ for the monotonicity formula.
    \item May jump over trapped regions.
\end{itemize}

\subsection{The Full Trapped Mass Evolution}

The complete formula:
\begin{equation}
    \frac{dm_T}{dt} = \text{(complicated expression)}
\end{equation}

requires careful analysis of:
\begin{itemize}
    \item All second fundamental form terms
    \item Einstein tensor contributions
    \item Laplacian terms from $\theta^\pm$ evolution
    \item Area scaling factors
\end{itemize}

\textbf{Key question:} Does DEC alone guarantee $\frac{dm_T}{dt} \geq 0$?

%===========================================================================
\section{Alternative: Static Comparison}
%===========================================================================

Instead of evolution, consider direct comparison.

\begin{proposition}
For a trapped surface $\Sigma_0$ in $(M, g, k)$, if there exists a MOTS 
$\Sigma^*$ with:
\begin{enumerate}
    \item $\Sigma_0$ is interior to $\Sigma^*$
    \item $m_T[\Sigma^*] \leq m_T[\Sigma_0]$ (trapped mass doesn't increase 
    outward along ANY path connecting them)
\end{enumerate}
then:
\begin{equation}
    M_{\ADM} \geq \sqrt{\frac{A(\Sigma^*)}{16\pi}} = m_T[\Sigma^*] \geq m_T[\Sigma_0]
\end{equation}
\end{proposition}

\textbf{Issue:} Condition (2) is what we need to prove, not assume.

%===========================================================================
\section{Conclusion}
%===========================================================================

The trapped mass $m_T[\Sigma] = \sqrt{A/(16\pi)}(1 - \frac{1}{16\pi}\int\theta^+\theta^-)^{1/2}$ 
is a natural quantity that:

\begin{enumerate}
    \item Uses both null expansions (the full trapped condition).
    \item Equals $\sqrt{A/(16\pi)}$ for MOTS.
    \item Has promising monotonicity properties under IMCF.
    \item Satisfies $m_T \geq m_H$ (improves on Hawking mass).
\end{enumerate}

\textbf{However:} A complete proof requires:
\begin{itemize}
    \item Rigorous analysis of $m_T$ evolution under weak IMCF.
    \item Showing the flow stays trapped or appropriately handles exit.
    \item Proving $m_T \to M_{\ADM}$ at infinity.
\end{itemize}

This remains an \textbf{incomplete but promising direction}.

%===========================================================================
\section{Final Remarks on the State of the Problem}
%===========================================================================

After extensive exploration, we conclude:

\begin{tcolorbox}[colback=red!5, colframe=red!50!black]
\textbf{The unconditional spacetime Penrose inequality for trapped surfaces 
with $\tr_\Sigma k < 0$ is a GENUINELY OPEN PROBLEM.}

No rigorous proof exists. All proposed approaches have identified gaps.

The fundamental obstruction is: \emph{Converting a trapped surface condition 
($\theta^\pm \leq 0$) into a scalar curvature condition ($R \geq 0$) without 
losing information about the surface's area.}

New mathematical ideas are needed.
\end{tcolorbox}

\end{document}
