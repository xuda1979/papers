%% SYMMETRIZATION_PROOF_ATTEMPT.tex
%%
%% A RIGOROUS ATTEMPT AT THE SYMMETRIZATION THEOREM
%%
%% This document attempts to prove the key lemma:
%% Spherical symmetrization decreases ADM mass while preserving
%% the trapped surface constraint.
%%
%% December 2025

\documentclass[11pt]{amsart}
\usepackage{amsmath,amssymb,amsthm}
\usepackage{tcolorbox}
\usepackage{mathrsfs}

\tcbuselibrary{theorems}

\newtcolorbox{maintheorem}{
    colback=green!5!white,
    colframe=green!50!black,
    title={\textbf{MAIN THEOREM}}
}

\newtcolorbox{keylemma}{
    colback=blue!5!white,
    colframe=blue!75!black,
    title={\textbf{KEY LEMMA}}
}

\newtcolorbox{proofstep}{
    colback=gray!5!white,
    colframe=gray!50!black,
    title={\textbf{PROOF}}
}

\newtcolorbox{gap}{
    colback=red!5!white,
    colframe=red!75!black,
    title={\textbf{GAP}}
}

\newtheorem{theorem}{Theorem}[section]
\newtheorem{lemma}[theorem]{Lemma}
\newtheorem{proposition}[theorem]{Proposition}
\newtheorem{corollary}[theorem]{Corollary}
\theoremstyle{definition}
\newtheorem{definition}[theorem]{Definition}
\newtheorem{remark}[theorem]{Remark}
\newtheorem{claim}{Claim}

\newcommand{\Area}{\mathrm{Area}}
\newcommand{\Vol}{\mathrm{Vol}}
\newcommand{\divv}{\mathrm{div}}
\DeclareMathOperator{\tr}{tr}
\newcommand{\Sch}{\mathrm{Sch}}

\title{The Symmetrization Theorem:\\
A Rigorous Proof Attempt}
\author{December 2025}

\begin{document}
\maketitle

\begin{abstract}
We attempt a rigorous proof of the symmetrization theorem for general 
relativistic initial data. The goal is to show that any DEC data with 
a trapped surface can be "symmetrized" to spherically symmetric data 
with smaller mass and preserved trapped surface area.
\end{abstract}

%% ============================================================================
\section{Statement of the Theorem}
%% ============================================================================

\begin{maintheorem}
\textbf{Symmetrization Theorem}

Let $(M, g, k)$ be asymptotically flat initial data satisfying DEC, 
containing a trapped surface $\Sigma$ of area $A$.

Then there exists spherically symmetric initial data $(M^*, g^*, k^*)$ such that:
\begin{enumerate}
    \item $(M^*, g^*, k^*)$ is asymptotically flat
    \item $(M^*, g^*, k^*)$ satisfies DEC
    \item $(M^*, g^*, k^*)$ contains a trapped surface $\Sigma^*$ with 
          $\Area(\Sigma^*) \ge A$
    \item $M_{\text{ADM}}(g^*, k^*) \le M_{\text{ADM}}(g, k)$
\end{enumerate}

\textbf{Equality in (4) iff $(g, k)$ is already spherically symmetric.}
\end{maintheorem}

%% ============================================================================
\section{Proof Strategy}
%% ============================================================================

We prove this in stages:

\textbf{Stage A:} Prove for time-symmetric data ($k = 0$)

\textbf{Stage B:} Extend to maximal data ($\tr k = 0$)  

\textbf{Stage C:} General case

%% ============================================================================
\section{Stage A: Time-Symmetric Case}
%% ============================================================================

For $k = 0$:
\begin{itemize}
    \item DEC becomes: $R \ge 0$
    \item Trapped becomes: minimal surface ($H = 0$)
    \item This is the Riemannian case
\end{itemize}

\begin{keylemma}
\textbf{Riemannian Symmetrization}

Let $(M, g)$ be a complete Riemannian 3-manifold with:
\begin{itemize}
    \item Asymptotically flat
    \item $R \ge 0$
    \item Minimal surface boundary $\Sigma$ with area $A$
\end{itemize}

Then there exists a spherically symmetric $(M^*, g^*)$ with:
\begin{itemize}
    \item Asymptotically flat
    \item $R^* \ge 0$
    \item Minimal sphere boundary of area $A$
    \item $M_{\text{ADM}}(g^*) \le M_{\text{ADM}}(g)$
\end{itemize}
\end{keylemma}

\begin{proofstep}
\textbf{Proof of Riemannian Symmetrization}

\textbf{Step 1: Construct the model.}

Take the comparison space to be Schwarzschild:
\begin{equation}
    g^* = \left(1 - \frac{2m}{r}\right)^{-1} dr^2 + r^2 d\Omega^2
\end{equation}

with $m$ chosen so the horizon area equals $A$:
\begin{equation}
    m = \sqrt{\frac{A}{16\pi}}
\end{equation}

\textbf{Step 2: Apply Riemannian Penrose.}

By Huisken-Ilmanen or Bray:
\begin{equation}
    M_{\text{ADM}}(g) \ge \sqrt{\frac{A}{16\pi}} = m = M_{\text{ADM}}(g^*)
\end{equation}

\textbf{Step 3: Verify properties.}

$(M^*, g^*) = $ Schwarzschild satisfies:
\begin{itemize}
    \item Asymptotically flat: Yes
    \item $R^* = 0 \ge 0$: Yes (Ricci flat)
    \item Minimal boundary of area $A$: Yes (horizon is minimal)
    \item Mass $\le M_{\text{ADM}}(g)$: Yes (by Penrose inequality)
\end{itemize}

\textbf{Conclusion:} Schwarzschild IS the symmetrization!
\end{proofstep}

\begin{remark}
The Riemannian case is "trivial" because the Penrose inequality is already 
proven. The symmetrization is not a construction - it's the assertion that 
Schwarzschild achieves the minimum mass.
\end{remark}

%% ============================================================================
\section{Stage B: Maximal Case}
%% ============================================================================

For $\tr k = 0$ (maximal slicing):
\begin{itemize}
    \item DEC: $R \ge |k|^2 \ge 0$
    \item The constraint simplifies but $k \neq 0$
\end{itemize}

\begin{keylemma}
\textbf{Maximal Data Symmetrization}

Let $(M, g, k)$ be maximal ($\tr k = 0$) initial data with DEC and 
trapped surface $\Sigma$ of area $A$.

Then there exists spherically symmetric maximal data $(M^*, g^*, k^*)$ with:
\begin{itemize}
    \item DEC satisfied
    \item Trapped surface of area $\ge A$
    \item $M_{\text{ADM}}(g^*, k^*) \le M_{\text{ADM}}(g, k)$
\end{itemize}
\end{keylemma}

\begin{proofstep}
\textbf{Approach: Reduction to Riemannian}

\textbf{Step 1: Conformal decomposition.}

For maximal data, write:
\begin{equation}
    g = \phi^4 \bar{g}
\end{equation}

where $\bar{g}$ is a "background" metric.

The constraint equations become:
\begin{equation}
    -8\Delta_{\bar{g}}\phi + R_{\bar{g}}\phi = -|k|^2_{\bar{g}} \phi^{-7}
\end{equation}

\textbf{Step 2: Control via DEC.}

DEC with $\tr k = 0$: $\mu = \frac{1}{16\pi}(R - |k|^2) \ge 0$.

So $R \ge |k|^2 \ge 0$.

\textbf{Step 3: Compare to time-symmetric.}

Consider the "associated" time-symmetric data $(\bar{M}, \bar{g}, 0)$.

This has $R_{\bar{g}} \ge 0$ (need to verify this follows from DEC).

Apply Riemannian symmetrization to $(\bar{M}, \bar{g})$.

\textbf{Step 4: Reconstruct $(g^*, k^*)$.}

From the symmetrized $\bar{g}^*$, construct $k^*$ satisfying:
\begin{itemize}
    \item Spherical symmetry
    \item Maximal: $\tr k^* = 0$
    \item Constraint equations satisfied
\end{itemize}
\end{proofstep}

\begin{gap}
\textbf{Gap in Stage B}

The reduction to Riemannian is not clean:
\begin{enumerate}
    \item The conformal factor $\phi$ depends on $k$
    \item The background $\bar{g}$ may not have $R \ge 0$ even if $(g, k)$ 
          satisfies DEC
    \item Reconstructing $k^*$ from symmetrized data is not unique
\end{enumerate}

\textbf{Alternative approach needed.}
\end{gap}

%% ============================================================================
\section{Alternative: Direct Averaging}
%% ============================================================================

\begin{definition}[Angular Average]
For a function $f$ on $(M, g)$ with spherical-like coordinates $(r, \theta, \phi)$:
\begin{equation}
    \langle f \rangle(r) = \frac{1}{4\pi} \int_{S^2} f(r, \omega) \, d\omega
\end{equation}

For a tensor $T$, average each component:
\begin{equation}
    \langle T \rangle_{ij}(r) = \frac{1}{4\pi} \int_{S^2} T_{ij}(r, \omega) \, d\omega
\end{equation}
\end{definition}

\begin{definition}[Averaged Data]
Given $(M, g, k)$, define:
\begin{align}
    g^{(0)} &= \langle g \rangle\\
    k^{(0)} &= \langle k \rangle
\end{align}

These are spherically symmetric by construction.
\end{definition}

\begin{proposition}[Constraint Violation]
The averaged data $(g^{(0)}, k^{(0)})$ generally does NOT satisfy the 
constraint equations.

The violation is:
\begin{equation}
    \mathcal{H}^{(0)} = R^{(0)} - |k^{(0)}|^2 + (\tr k^{(0)})^2 - 16\pi\mu \neq 0
\end{equation}

in general.
\end{proposition}

\begin{proofstep}
\textbf{Correction via Projection}

\textbf{Step 1: Measure the violation.}
\begin{align}
    \mathcal{H}^{(0)} &= R^{(0)} - |k^{(0)}|^2 + (\tr k^{(0)})^2 - 16\pi\langle\mu\rangle\\
    \mathcal{M}^{(0)}_i &= \nabla^{(0)}_j(k^{(0)ij} - (\tr k^{(0)})g^{(0)ij}) - 8\pi\langle J\rangle_i
\end{align}

\textbf{Step 2: Project to constraint surface.}

Find $(g^*, k^*)$ solving:
\begin{align}
    R^* - |k^*|^2 + (\tr k^*)^2 &= 16\pi\mu^*\\
    \nabla^*_j(k^{*ij} - (\tr k^*)g^{*ij}) &= 8\pi J^{*i}
\end{align}

with $(g^*, k^*)$ close to $(g^{(0)}, k^{(0)})$ and spherically symmetric.

\textbf{Step 3: Show projection preserves properties.}

Need:
\begin{itemize}
    \item Mass: $M_{\text{ADM}}(g^*, k^*) \le M_{\text{ADM}}(g, k)$
    \item Trapped: preserved or improved
    \item DEC: preserved
\end{itemize}
\end{proofstep}

%% ============================================================================
\section{The Projection Lemma}
%% ============================================================================

\begin{keylemma}
\textbf{Constraint Projection}

Let $(g^{(0)}, k^{(0)})$ be spherically symmetric data with small constraint 
violation:
\begin{equation}
    \|\mathcal{H}^{(0)}\|_{L^2} + \|\mathcal{M}^{(0)}\|_{L^2} = \epsilon \ll 1
\end{equation}

Then there exists spherically symmetric $(g^*, k^*)$ satisfying the constraints 
with:
\begin{equation}
    \|g^* - g^{(0)}\|_{H^2} + \|k^* - k^{(0)}\|_{H^1} = O(\epsilon)
\end{equation}

Moreover:
\begin{equation}
    |M_{\text{ADM}}(g^*, k^*) - M_{\text{ADM}}(g^{(0)}, k^{(0)})| = O(\epsilon^2)
\end{equation}
\end{keylemma}

\begin{proofstep}
\textbf{Proof Sketch}

The constraint equations are elliptic. The linearized constraints have 
kernel only from diffeomorphisms (gauge).

On spherically symmetric data, the gauge freedom is reduced.

By implicit function theorem, small violations can be corrected with small 
perturbations.

Mass is a smooth functional, so changes are quadratic in the perturbation.
\end{proofstep}

%% ============================================================================
\section{Mass Under Averaging}
%% ============================================================================

\begin{claim}
\textbf{Key Claim:} Averaging decreases ADM mass:
\begin{equation}
    M_{\text{ADM}}(g^{(0)}, k^{(0)}) \le M_{\text{ADM}}(g, k)
\end{equation}
\end{claim}

\begin{proofstep}
\textbf{Attempt at Proof}

The ADM mass is:
\begin{equation}
    M_{\text{ADM}} = \frac{1}{16\pi}\lim_{r\to\infty} \int_{S_r} 
    (g_{ij,i} - g_{ii,j})\nu^j \, dA
\end{equation}

Under averaging:
\begin{equation}
    g^{(0)}_{ij,i} = \langle g_{ij,i} \rangle
\end{equation}

The mass integral becomes:
\begin{equation}
    M^{(0)} = \frac{1}{16\pi}\lim_{r\to\infty} \int_{S_r} 
    (\langle g_{ij,i}\rangle - \langle g_{ii,j}\rangle)\nu^j \, dA
\end{equation}

Since integration commutes with averaging:
\begin{equation}
    M^{(0)} = \frac{1}{16\pi}\lim_{r\to\infty} 
    \left\langle \int_{S_r} (g_{ij,i} - g_{ii,j})\nu^j \, dA \right\rangle
\end{equation}

But this is just:
\begin{equation}
    M^{(0)} = M_{\text{ADM}} \quad \text{(averaging doesn't change mass!)}
\end{equation}
\end{proofstep}

\begin{gap}
\textbf{Critical Issue}

Simple angular averaging does NOT decrease mass. The mass integral 
at infinity is already angle-averaged!

\textbf{We need a different symmetrization that actually decreases mass.}
\end{gap}

%% ============================================================================
\section{Correct Approach: Isoperimetric Symmetrization}
%% ============================================================================

\begin{definition}[Isoperimetric Symmetrization]
Instead of averaging at fixed coordinate radius, rearrange based on 
isoperimetric profile.

\textbf{Step 1:} For $(M, g)$, compute the isoperimetric profile:
\begin{equation}
    I_g(V) = \min\{\Area(\partial\Omega) : \Vol(\Omega) = V\}
\end{equation}

\textbf{Step 2:} Construct spherically symmetric $g^*$ with the same profile:
\begin{equation}
    I_{g^*}(V) = I_g(V) \quad \forall V
\end{equation}

This determines $g^*$ uniquely (up to gauge).
\end{definition}

\begin{proposition}[Mass Comparison for Isoperimetric Symmetrization]
For $g$ with $R \ge 0$:
\begin{equation}
    M_{\text{ADM}}(g^*) \le M_{\text{ADM}}(g)
\end{equation}

\textbf{Reason:} Isoperimetric inequality + Geroch monotonicity.
\end{proposition}

\begin{proof}[Proof for Riemannian case]
By Geroch's result, for $R \ge 0$:
\begin{equation}
    m_H(\Sigma) \le M_{\text{ADM}}
\end{equation}
for any surface $\Sigma$.

The isoperimetric surfaces achieve minimum $\int H^2$ for fixed area, 
giving maximum Hawking mass.

For spherically symmetric $g^*$, all level sets are isoperimetric.

The Hawking mass at infinity equals $M_{\text{ADM}}(g^*)$.

Therefore:
\begin{equation}
    M_{\text{ADM}}(g^*) = \lim_{A\to\infty} m_H^*(\Sigma_A) 
    \le \lim_{A\to\infty} m_H(\Sigma_A) \le M_{\text{ADM}}(g)
\end{equation}

Wait, this inequality goes the wrong way for the middle step...
\end{proof}

%% ============================================================================
\section{The Fundamental Difficulty}
%% ============================================================================

\begin{gap}
\textbf{Why Mass Comparison is Hard}

For Riemannian data with $R \ge 0$:
\begin{itemize}
    \item Hawking mass $m_H$ increases along IMCF
    \item $m_H \to M_{\text{ADM}}$ at infinity
    \item This proves $M \ge m_H(\Sigma_0)$ (Penrose!)
\end{itemize}

But this doesn't directly give mass comparison under symmetrization.

The issue: Hawking mass of a sphere depends on how it sits in the geometry, 
not just the isoperimetric profile.

\textbf{Isoperimetric profile does NOT determine ADM mass uniquely!}

Different geometries with the same isoperimetric profile can have different 
ADM masses.
\end{gap}

%% ============================================================================
\section{Revised Strategy}
%% ============================================================================

\begin{claim}
\textbf{New Approach}

Instead of proving:
\begin{center}
"Symmetrization decreases mass"
\end{center}

Prove directly:
\begin{center}
"Schwarzschild minimizes mass among all DEC data with given trapped area"
\end{center}

This is a variational statement, not a symmetrization statement.
\end{claim}

\begin{proofstep}
\textbf{Direct Variational Proof}

\textbf{Setup:} Fix $A > 0$. Consider:
\begin{equation}
    \mathcal{D}_A = \{(g, k) : \text{AF, DEC, } \exists \text{ trapped } \Sigma \text{ with } \Area(\Sigma) \ge A\}
\end{equation}

\textbf{Goal:} Prove $\inf_{(g,k) \in \mathcal{D}_A} M_{\text{ADM}}(g, k) = \sqrt{A/(16\pi)}$.

\textbf{Approach:} Use calculus of variations.

\textbf{Step 1:} Write Euler-Lagrange equations for the constrained problem.

\textbf{Step 2:} Show the only solution is Schwarzschild.

\textbf{Step 3:} Show it's a minimum (second variation $\ge 0$).
\end{proofstep}

%% ============================================================================
\section{Euler-Lagrange for Constrained Mass Minimization}
%% ============================================================================

\begin{definition}[The Lagrangian]
\begin{equation}
    \mathcal{L}[g, k, \Sigma, \lambda, \Lambda] = M_{\text{ADM}}[g, k] 
    + \lambda \cdot (\Area(\Sigma) - A) 
    + \int_M \Lambda^i \mathcal{M}_i \, dV
    + \int_M \Lambda_0 \mathcal{H} \, dV
\end{equation}

where:
\begin{itemize}
    \item $\lambda$ enforces area constraint
    \item $\Lambda^i$ enforce momentum constraint
    \item $\Lambda_0$ enforces Hamiltonian constraint
\end{itemize}
\end{definition}

\begin{proposition}[Stationarity Conditions]
At a critical point:
\begin{align}
    \frac{\delta M}{\delta g} + \lambda \frac{\delta \Area}{\delta g} 
    + \Lambda_0 \frac{\delta\mathcal{H}}{\delta g} 
    + \Lambda^i \frac{\delta\mathcal{M}_i}{\delta g} &= 0\\
    \frac{\delta M}{\delta k} 
    + \Lambda_0 \frac{\delta\mathcal{H}}{\delta k} 
    + \Lambda^i \frac{\delta\mathcal{M}_i}{\delta k} &= 0\\
    \theta^+[\Sigma] &\le 0 \quad \text{(trapped constraint)}
\end{align}

plus boundary conditions.
\end{proposition}

%% ============================================================================
\section{Checking Schwarzschild}
%% ============================================================================

\begin{proposition}[Schwarzschild Satisfies Euler-Lagrange]
Schwarzschild data $(g_{\Sch}, k_{\Sch} = 0)$ with horizon area $A$ 
satisfies the Euler-Lagrange equations with appropriate $\lambda$.
\end{proposition}

\begin{proof}
For Schwarzschild:
\begin{itemize}
    \item $\mathcal{H} = \mathcal{M} = 0$ (vacuum, constraints satisfied)
    \item Horizon is MOTS: $\theta^+ = 0$
    \item $M_{\text{ADM}} = \sqrt{A/(16\pi)}$
\end{itemize}

The variations vanish because Schwarzschild is a critical point of 
$M_{\text{ADM}}$ on the constraint surface (it's an isolated point 
among spherically symmetric vacuum solutions with given horizon area).
\end{proof}

%% ============================================================================
\section{Uniqueness}
%% ============================================================================

\begin{claim}
\textbf{Schwarzschild is the UNIQUE minimizer.}

Any other critical point either:
\begin{itemize}
    \item Has larger mass (not a minimizer)
    \item Can be perturbed to decrease mass (saddle point)
\end{itemize}
\end{claim}

This requires second variation analysis and is the technical heart of 
the proof.

%% ============================================================================
\section{Conclusion}
%% ============================================================================

\begin{maintheorem}
\textbf{Summary}

The symmetrization theorem, as stated, is difficult to prove directly 
because:
\begin{enumerate}
    \item Angular averaging doesn't decrease mass
    \item Isoperimetric symmetrization doesn't uniquely determine mass
    \item The $(g, k)$ coupling through constraints complicates everything
\end{enumerate}

\textbf{The correct approach is variational:}

Instead of constructing a symmetrization map, prove that:
\begin{center}
\fbox{Schwarzschild is the unique minimizer of $M_{\text{ADM}}$ on $\mathcal{D}_A$}
\end{center}

This requires:
\begin{enumerate}
    \item Compactness/existence of minimizer
    \item Euler-Lagrange analysis
    \item Second variation (stability)
\end{enumerate}

Each of these is a concrete mathematical problem.
\end{maintheorem}

\end{document}
