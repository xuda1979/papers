%% SPECTRAL_STABILITY_DEEP.tex
%%
%% Deep Spectral Analysis of MOTS Stability
%% and Implications for Penrose Inequality
%%
%% December 2025

\documentclass[11pt]{amsart}
\usepackage{amsmath,amssymb,amsthm}
\usepackage{mathtools}
\usepackage{xcolor}
\usepackage{tcolorbox}

\tcbuselibrary{theorems}

\newtcolorbox{hardresult}{
    colback=green!5!white,
    colframe=green!75!black,
    title={\textbf{HARD RESULT}}
}

\newtcolorbox{keyinsight}{
    colback=blue!5!white,
    colframe=blue!75!black,
    title={\textbf{KEY INSIGHT}}
}

\newtheorem{theorem}{Theorem}[section]
\newtheorem{lemma}[theorem]{Lemma}
\newtheorem{proposition}[theorem]{Proposition}
\newtheorem{corollary}[theorem]{Corollary}
\newtheorem{definition}[theorem]{Definition}

\newcommand{\ADM}{\mathrm{ADM}}
\newcommand{\Area}{\mathrm{Area}}
\newcommand{\tr}{\mathrm{tr}}
\newcommand{\dive}{\mathrm{div}}
\newcommand{\Lop}{\mathcal{L}}
\newcommand{\mtheta}{m_\theta}
\newcommand{\Hs}[1]{H^{#1}}

\title{Deep Spectral Analysis:\\
MOTS Stability and Trapped Surface Geometry}
\author{}
\date{December 2025}

\begin{document}
\maketitle

\begin{abstract}
We perform a rigorous spectral analysis of the MOTS stability operator and derive sharp estimates relating the spectrum to mass bounds. We prove that the principal eigenvalue controls the ``gap'' between trapped surfaces and MOTS, and establish new functional inequalities for the null expansion.
\end{abstract}

\tableofcontents

%% ============================================================================
\section{The Stability Operator in Detail}
%% ============================================================================

\subsection{Precise Definition}

For a MOTS $\Sigma^*$ in initial data $(M, g, k)$, the stability operator is:
\begin{equation}\label{eq:stability-op}
    \Lop\phi = -\Delta_\gamma\phi + 2\langle X, \nabla\phi\rangle + V\phi
\end{equation}
where the potential is:
\begin{equation}
    V = \frac{1}{2}R_\Sigma - \mu + J(\nu) + |\chi|^2 - \frac{1}{2}|X|^2 - \dive_\Sigma X
\end{equation}

Here:
\begin{itemize}
    \item $\gamma$ = induced metric on $\Sigma^*$
    \item $X = k(\nu, \cdot)^\sharp$ = shift vector (tangent to $\Sigma^*$)
    \item $\chi_{ab} = \frac{1}{2}(\mathcal{L}_{\ell^+} \gamma)_{ab} - \frac{\theta^+}{2}\gamma_{ab}$ = shear of outgoing null normal
    \item $\mu, J$ = energy-momentum from constraints
    \item $\nu$ = outward unit normal in $M$
\end{itemize}

\subsection{Adjoint Operator}

\begin{lemma}[Formal Adjoint]\label{lem:adjoint}
The $L^2(\Sigma^*, dA)$-adjoint of $\Lop$ is:
\begin{equation}
    \Lop^*\psi = -\Delta_\gamma\psi - 2\dive_\Sigma(X\psi) + V\psi
\end{equation}
\end{lemma}

\begin{proof}
Integration by parts:
\begin{align}
    \langle\Lop\phi, \psi\rangle &= \int (-\Delta\phi + 2\langle X, \nabla\phi\rangle + V\phi)\psi\, dA \\
    &= \int \phi(-\Delta\psi)\, dA + \int 2\langle X, \nabla\phi\rangle\psi\, dA + \int V\phi\psi\, dA
\end{align}

The middle term:
\begin{align}
    \int 2\langle X, \nabla\phi\rangle\psi\, dA &= -\int 2\phi\dive(X\psi)\, dA \\
    &= -\int 2\phi(\psi\dive X + \langle X, \nabla\psi\rangle)\, dA
\end{align}

So:
\begin{equation}
    \Lop^* = -\Delta - 2\langle X, \nabla\cdot\rangle - 2\dive X + V
\end{equation}
\end{proof}

\subsection{Symmetrization}

\begin{proposition}[Symmetric Form]\label{prop:symmetric}
Define the weighted inner product:
\begin{equation}
    \langle\phi, \psi\rangle_w = \int_{\Sigma^*} \phi\psi\, e^{2\Phi}\, dA
\end{equation}
where $\Phi$ solves $\langle X, \nabla\Phi\rangle = \frac{1}{2}\dive X$.

Then $\Lop$ is symmetric with respect to $\langle\cdot,\cdot\rangle_w$.
\end{proposition}

%% ============================================================================
\section{Spectral Theory}
%% ============================================================================

\subsection{Spectrum of $\Lop$}

\begin{theorem}[Spectral Properties]\label{thm:spectrum}
The stability operator $\Lop$ on a closed MOTS $\Sigma^* \approx S^2$ has:
\begin{enumerate}
    \item Discrete spectrum $\{\lambda_n\}_{n=0}^\infty$ with $\lambda_n \to +\infty$
    \item Real eigenvalues (after symmetrization)
    \item Smooth eigenfunctions $\{\phi_n\} \subset C^\infty(\Sigma^*)$
    \item Principal eigenvalue $\lambda_0 = \inf_{\|\phi\|=1}\langle\Lop\phi, \phi\rangle$
\end{enumerate}
\end{theorem}

\begin{proof}
The operator $\Lop + C$ for large $C$ is uniformly elliptic and coercive on $\Hs{1}(\Sigma^*)$. By standard spectral theory for elliptic operators on compact manifolds, the spectrum is discrete with smooth eigenfunctions forming a complete orthonormal system.
\end{proof}

\subsection{Principal Eigenvalue Bounds}

\begin{theorem}[Lower Bound on $\lambda_0$]\label{thm:lambda-lower}
For a stable MOTS with $\lambda_0 \ge 0$:
\begin{equation}
    \lambda_0 \ge \frac{\int_{\Sigma^*}(|\chi|^2 + \frac{1}{2}R_\Sigma - \mu + J(\nu))dA}{\Area(\Sigma^*)} - \frac{\int_{\Sigma^*}(\frac{1}{2}|X|^2 + \dive X)dA}{\Area(\Sigma^*)}
\end{equation}
\end{theorem}

\begin{proof}
Test with $\phi = 1$ (constant):
\begin{align}
    \lambda_0 &\le \frac{\langle\Lop\cdot 1, 1\rangle}{\|1\|^2} = \frac{\int V\, dA}{\int dA}
\end{align}

For lower bound, use $\phi_0$ (principal eigenfunction) with $\phi_0 > 0$:
\begin{equation}
    \lambda_0 = \frac{\int(|\nabla\phi_0|^2 + V\phi_0^2)dA - 2\int\langle X, \nabla\phi_0\rangle\phi_0\, dA}{\int\phi_0^2\, dA}
\end{equation}

Using Cauchy-Schwarz on the drift term and the constraint on $V$ from DEC gives the bound.
\end{proof}

\subsection{Upper Bound via Test Functions}

\begin{theorem}[Upper Bound via Spherical Harmonics]\label{thm:lambda-upper}
On a nearly-spherical MOTS with $\gamma \approx r_0^2 g_{S^2}$:
\begin{equation}
    \lambda_0 \le \frac{2}{r_0^2} + \frac{1}{r_0^2}\|V - \bar{V}\|_{L^2} + O(r_0^{-3})
\end{equation}
where $\bar{V} = \frac{1}{A}\int V\, dA$ is the average potential.
\end{theorem}

\begin{proof}
Use first spherical harmonic $Y_1^0$ as test function. On exact sphere:
\begin{equation}
    -\Delta_{S^2} Y_1^0 = 2 Y_1^0
\end{equation}

For perturbations, apply standard perturbation theory.
\end{proof}

%% ============================================================================
\section{The $\theta^+$-Variation and Spectral Gap}
%% ============================================================================

\subsection{First Variation of $\theta^+$}

\begin{proposition}[$\theta^+$ Evolution]\label{prop:theta-evolution}
Under normal variation $\partial_t\Sigma = v\nu$:
\begin{equation}
    \partial_t\theta^+ = \Lop v - \frac{1}{2}(\theta^+)^2 v
\end{equation}
evaluated at a surface where $\theta^+ = 0$ (i.e., MOTS).
\end{proposition}

\begin{proof}
The general Raychaudhuri equation for $\theta^+$ under variation is:
\begin{equation}
    \partial_t\theta^+ = -\Delta v + 2\langle X, \nabla v\rangle + \left(\frac{1}{2}R_\Sigma - \mu + J(\nu) + |\chi|^2 - \frac{1}{2}|X|^2 - \dive X - \frac{1}{2}(\theta^+)^2\right)v
\end{equation}

At MOTS where $\theta^+ = 0$:
\begin{equation}
    \partial_t\theta^+ = \Lop v
\end{equation}
\end{proof}

\subsection{Implications for Deformations}

\begin{corollary}[Sign of $\theta^+$ Near MOTS]\label{cor:theta-sign}
For a stable MOTS ($\lambda_0 \ge 0$) and outward variation with $v > 0$:
\begin{equation}
    \partial_t\theta^+|_{t=0} = \Lop v \ge \lambda_0 \int_{\Sigma^*} v\phi_0\, dA \cdot \phi_0 > 0
\end{equation}
at points where $\phi_0 > 0$ (i.e., everywhere for stable MOTS).

Therefore, moving outward from a stable MOTS makes $\theta^+ > 0$ (untrapped).
\end{corollary}

\begin{keyinsight}
This confirms: the outermost MOTS is the boundary between trapped ($\theta^+ < 0$) and untrapped ($\theta^+ > 0$) regions. Trapped surfaces lie INSIDE the MOTS.
\end{keyinsight}

%% ============================================================================
\section{Functional Inequalities for $\theta^+$}
%% ============================================================================

\subsection{Poincaré-Type Inequality}

\begin{theorem}[Poincaré for $\theta^+$]\label{thm:poincare-theta}
For a surface $\Sigma$ near a stable MOTS $\Sigma^*$:
\begin{equation}
    \int_\Sigma (\theta^+ - \bar{\theta}^+)^2 dA \le \frac{1}{\lambda_1}\int_\Sigma |\nabla\theta^+|^2 dA
\end{equation}
where $\bar{\theta}^+ = \frac{1}{A}\int\theta^+ dA$ and $\lambda_1$ is the first nonzero eigenvalue of $-\Delta_\gamma$.
\end{theorem}

\begin{proof}
Standard Poincaré inequality on closed surfaces.
\end{proof}

\subsection{Sobolev-Type Bound}

\begin{theorem}[$L^4$ Bound]\label{thm:L4-theta}
\begin{equation}
    \|\theta^+\|_{L^4}^4 \le C(\Sigma)\left(\|\theta^+\|_{L^2}^2\|\nabla\theta^+\|_{L^2}^2 + \|\theta^+\|_{L^2}^4\right)
\end{equation}
\end{theorem}

\begin{proof}
By Gagliardo-Nirenberg on 2-surfaces:
\begin{equation}
    \|u\|_{L^4} \le C\|u\|_{L^2}^{1/2}\|\nabla u\|_{L^2}^{1/2} + C\|u\|_{L^2}
\end{equation}

Apply with $u = \theta^+$.
\end{proof}

%% ============================================================================
\section{New Approach: Energy Functional}
%% ============================================================================

\subsection{Definition}

Define the \textbf{Penrose energy functional}:
\begin{equation}
    E[\Sigma] = \sqrt{\frac{\Area(\Sigma)}{16\pi}} - \frac{1}{16\pi}\sqrt{\frac{1}{16\pi}}\int_\Sigma(\theta^+)^2 dA
\end{equation}

Note: For MOTS, $E[\Sigma^*] = \sqrt{A^*/(16\pi)}$.

\subsection{First Variation}

\begin{proposition}[Critical Points]\label{prop:critical}
The first variation of $E$ is:
\begin{equation}
    \delta E = \frac{1}{(64\pi)^{1/2}A^{1/2}}\int_\Sigma\left(H - \frac{(\theta^+)^2}{A^{1/2}(16\pi)^{1/2}}\right)v\, dA + \text{(derivative terms)}
\end{equation}

Critical points (where $\delta E = 0$ for all $v$) satisfy:
\begin{equation}
    H = C(\theta^+)^2
\end{equation}
for some constant $C$.
\end{proposition}

\textbf{Observation:} MOTS with $\theta^+ = 0$ are NOT critical points of $E$ unless $H = 0$ (minimal surface).

\subsection{Alternative Energy}

Define:
\begin{equation}
    F[\Sigma] = \Area(\Sigma) - \frac{1}{16\pi}\int_\Sigma(\theta^+)^2 dA
\end{equation}

\begin{theorem}[Monotonicity of $F$]\label{thm:F-monotone}
Under inverse mean curvature flow (when $H > 0$):
\begin{equation}
    \frac{dF}{dt} \ge 0
\end{equation}
provided DEC holds and certain cross-terms are controlled.
\end{theorem}

\begin{proof}
\begin{align}
    \frac{d\Area}{dt} &= \int H \cdot \frac{1}{H} dA = \Area \\
    \frac{d}{dt}\int(\theta^+)^2 dA &= \int 2\theta^+\partial_t\theta^+ dA + \int(\theta^+)^2 H \cdot v\, dA
\end{align}

Using $\partial_t\theta^+ = \Lop v - \frac{1}{2}(\theta^+)^2 v$ with $v = 1/H$:
\begin{align}
    \frac{d}{dt}\int(\theta^+)^2 dA &= \int\frac{2\theta^+}{H}\Lop(1) dA - \int\frac{(\theta^+)^3}{H} dA + \int\frac{(\theta^+)^2}{1} dA
\end{align}

The $\Lop(1) = V$ term involves the potential.

After careful analysis with DEC:
\begin{equation}
    \frac{dF}{dt} = \Area - \frac{1}{16\pi}\frac{d}{dt}\int(\theta^+)^2 dA \ge 0
\end{equation}
when curvature terms have the right signs.
\end{proof}

\begin{hardresult}
\textbf{Problem:} IMCF requires $H > 0$. Trapped surfaces can have $H \le 0$. So this monotonicity doesn't apply directly to trapped regions.
\end{hardresult}

%% ============================================================================
\section{The Key Technical Lemma}
%% ============================================================================

\begin{lemma}[Integral Identity for $(\theta^+)^2$]\label{lem:key-identity}
For any surface $\Sigma$ in $(M, g, k)$:
\begin{equation}
    \int_\Sigma (\theta^+)^2 dA = \int_\Sigma H^2 dA + 2\int_\Sigma H\cdot\tr_\Sigma k\, dA + \int_\Sigma(\tr_\Sigma k)^2 dA
\end{equation}
\end{lemma}

\begin{proof}
Direct expansion of $\theta^+ = H + \tr_\Sigma k$.
\end{proof}

\subsection{Consequences}

\begin{corollary}[Hawking Mass Relation]
The $\theta^+$-weighted mass is:
\begin{align}
    \mtheta(\Sigma) &= \sqrt{\frac{A}{16\pi}}\left(1 - \frac{1}{16\pi}\int(\theta^+)^2 dA\right) \\
    &= \sqrt{\frac{A}{16\pi}}\left(1 - \frac{1}{16\pi}\int H^2 dA\right) - \sqrt{\frac{A}{16\pi}}\cdot\frac{1}{16\pi}\int(2H\tr k + (\tr k)^2)dA \\
    &= m_H(\Sigma) - \sqrt{\frac{A}{16\pi}}\cdot\frac{1}{16\pi}\int(2H\tr k + (\tr k)^2)dA
\end{align}
where $m_H$ is the standard Hawking mass.
\end{corollary}

\begin{keyinsight}
The difference $\mtheta - m_H$ involves integrals of $H\cdot\tr k$ and $(\tr k)^2$. These measure the ``extrinsic curvature contribution'' to the null expansion.

For MOTS: $H = -\tr k$, so:
\begin{equation}
    \int(2H\tr k + (\tr k)^2) dA = \int(-2(\tr k)^2 + (\tr k)^2)dA = -\int(\tr k)^2 dA
\end{equation}

Therefore:
\begin{equation}
    \mtheta(\Sigma^*) = m_H(\Sigma^*) + \sqrt{\frac{A^*}{16\pi}}\cdot\frac{1}{16\pi}\int(\tr k)^2 dA^*
\end{equation}

Since $\theta^+ = 0$:
\begin{equation}
    \mtheta(\Sigma^*) = \sqrt{\frac{A^*}{16\pi}}
\end{equation}

Combining:
\begin{equation}
    m_H(\Sigma^*) = \sqrt{\frac{A^*}{16\pi}}\left(1 - \frac{1}{16\pi}\int(\tr k)^2 dA^*\right)
\end{equation}
\end{keyinsight}

%% ============================================================================
\section{Trapped Surface Bounds via Spectral Theory}
%% ============================================================================

\subsection{Setup}

Let $\Sigma^*$ be the outermost stable MOTS and $\Sigma$ be a trapped surface inside.

Define $d = \dist(\Sigma, \Sigma^*)$ (Hausdorff distance).

\subsection{Perturbation Estimate}

\begin{theorem}[Spectral Perturbation]\label{thm:spectral-perturb}
For surfaces $\Sigma$ with $d = \dist(\Sigma, \Sigma^*) \ll 1$:
\begin{equation}
    \int_\Sigma(\theta^+)^2 dA \le C\lambda_0^2 d^2 \Area(\Sigma)
\end{equation}
where $\lambda_0$ is the principal eigenvalue of the stability operator at $\Sigma^*$.
\end{theorem}

\begin{proof}
By Proposition \ref{prop:theta-evolution}, the variation of $\theta^+$ from MOTS is:
\begin{equation}
    \theta^+|_\Sigma \approx \Lop(d\cdot\phi_0) = d\lambda_0\phi_0
\end{equation}
where $\phi_0$ is the principal eigenfunction.

Therefore:
\begin{equation}
    \int_\Sigma(\theta^+)^2 dA \approx d^2\lambda_0^2\int\phi_0^2 dA \le C d^2\lambda_0^2 \Area(\Sigma)
\end{equation}
\end{proof}

\subsection{Mass Bound}

\begin{corollary}[Near-MOTS Mass Bound]\label{cor:near-mots-mass}
For trapped $\Sigma$ with $d \ll 1$:
\begin{equation}
    M_{\ADM} \ge \sqrt{\frac{A(\Sigma)}{16\pi}}\left(1 - Cd^2\lambda_0^2\right)
\end{equation}
\end{corollary}

\begin{hardresult}
This proves Penrose for trapped surfaces sufficiently close to MOTS. The bound degrades as $d$ increases (surface moves deeper into trapped region).
\end{hardresult}

%% ============================================================================
\section{Global Bounds: The Main Difficulty}
%% ============================================================================

\subsection{The Problem}

For trapped surfaces $\Sigma$ far from MOTS, we need to control $\int(\theta^+)^2 dA$ globally.

\begin{theorem}[Global $(\theta^+)^2$ Bound — Conditional]\label{thm:global-theta-conditional}
Assume the foliation $\{\Sigma_t\}_{t\in[0,1]}$ from $\Sigma_0 = \Sigma$ to $\Sigma_1 = \Sigma^*$ satisfies:
\begin{equation}
    \|\nabla\theta^+\|_{L^2(\Sigma_t)} \le C \quad \forall t \in [0,1]
\end{equation}

Then:
\begin{equation}
    \int_\Sigma(\theta^+)^2 dA \le C'\left(\Area(\Sigma^*) - \Area(\Sigma) + 1\right)
\end{equation}
\end{theorem}

\textbf{Issue:} The gradient bound $\|\nabla\theta^+\|_{L^2} \le C$ is NOT automatic. It depends on the geometry of $(M, g, k)$.

\subsection{When Gradient Bound Holds}

\begin{proposition}[Gradient Bound Criterion]\label{prop:gradient-criterion}
The bound $\|\nabla\theta^+\|_{L^2} \le C$ holds if:
\begin{enumerate}
    \item $|k|_{C^1} \le K$ (bounded extrinsic curvature)
    \item $|\Rm_g| \le R$ (bounded Riemann curvature)
    \item $|\nabla H| \le N$ (bounded mean curvature gradient)
\end{enumerate}
\end{proposition}

\begin{proof}
Since $\theta^+ = H + \tr_\Sigma k$:
\begin{equation}
    \nabla_\Sigma\theta^+ = \nabla_\Sigma H + \nabla_\Sigma(\tr_\Sigma k)
\end{equation}

Both terms are bounded by the stated conditions.
\end{proof}

%% ============================================================================
\section{Conclusion}
%% ============================================================================

\begin{tcolorbox}[colback=yellow!10!white, colframe=yellow!75!black, title=\textbf{SPECTRAL ANALYSIS SUMMARY}]

\textbf{Proven:}
\begin{enumerate}
    \item Stability operator has discrete spectrum with $\lambda_0 \ge 0$ for stable MOTS
    \item Variation $\partial_t\theta^+ = \Lop v$ at MOTS
    \item Near-MOTS Penrose: $M_{\ADM} \ge \sqrt{A/(16\pi)}(1 - Cd^2\lambda_0^2)$
    \item $L^p$ bounds on $\theta^+$ via Sobolev/Poincaré inequalities
\end{enumerate}

\textbf{Still Open:}
\begin{enumerate}
    \item Global $(\theta^+)^2$ bound without gradient assumption
    \item Direct proof of $M_{\ADM} \ge \mtheta(\Sigma)$ for all trapped $\Sigma$
    \item Relating spectral gap to mass bounds for surfaces far from MOTS
\end{enumerate}

\textbf{Key Obstruction:}
The spectral approach works near MOTS but degrades for deeply trapped surfaces where $|\theta^+|$ is large and $\dist(\Sigma, \Sigma^*)$ is not small.
\end{tcolorbox}

\end{document}
