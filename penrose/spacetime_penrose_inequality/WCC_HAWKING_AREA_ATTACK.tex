%% WCC_HAWKING_AREA_ATTACK.tex
%%
%% NEW ATTACK: Using WCC + Hawking Area Theorem to Prove Area Dominance
%%
%% Key Insight: Penrose's original argument WAS correct!
%% We just need to make it rigorous.
%%
%% December 2025

\documentclass[11pt]{amsart}
\usepackage{amsmath,amssymb,amsthm}
\usepackage{xcolor}
\usepackage{tcolorbox}
\usepackage{tikz}

\tcbuselibrary{theorems}

\newtcolorbox{keyinsight}{
    colback=green!5!white,
    colframe=green!75!black,
    title={\textbf{KEY INSIGHT}}
}

\newtcolorbox{theorem-box}{
    colback=blue!5!white,
    colframe=blue!75!black,
}

\newtheorem{theorem}{Theorem}[section]
\newtheorem{lemma}[theorem]{Lemma}
\newtheorem{proposition}[theorem]{Proposition}
\newtheorem{corollary}[theorem]{Corollary}
\newtheorem{definition}[theorem]{Definition}
\newtheorem{remark}[theorem]{Remark}

\newcommand{\ADM}{\mathrm{ADM}}
\newcommand{\Area}{\mathrm{Area}}
\newcommand{\tr}{\mathrm{tr}}
\newcommand{\scri}{\mathscr{I}}

\title{The WCC + Hawking Area Theorem Attack\\
\large A Rigorous Path to Penrose 1973}
\author{}
\date{December 2025}

\begin{document}
\maketitle

\begin{abstract}
We develop Penrose's original argument using Weak Cosmic Censorship (WCC) and the Hawking Area Theorem into a rigorous mathematical framework. The key is to formalize what WCC means and carefully track how the Hawking theorem implies area dominance.
\end{abstract}

\tableofcontents

%% ============================================================================
\section{Penrose's Original Argument (1973)}
%% ============================================================================

Penrose's heuristic argument was:

\begin{enumerate}
    \item A trapped surface $\Sigma$ lies inside the black hole region (by definition of trapped)
    \item Under WCC, the black hole settles to a Kerr solution with event horizon $\mathcal{H}$
    \item The Hawking Area Theorem says: area of event horizon is non-decreasing
    \item The final Kerr black hole has $M_{\text{Kerr}} = \sqrt{A_{\text{Kerr}}/(16\pi)}$ (for Schwarzschild)
    \item By Bondi mass loss: $M_{\ADM} \ge M_{\text{Kerr}}$
    \item By Hawking: $A(\Sigma) \le A_{\text{initial horizon}} \le A_{\text{final horizon}} = A_{\text{Kerr}}$
    \item Combining: $M_{\ADM} \ge M_{\text{Kerr}} = \sqrt{A_{\text{Kerr}}/(16\pi)} \ge \sqrt{A(\Sigma)/(16\pi)}$
\end{enumerate}

\begin{keyinsight}
The ``gap'' in this argument is step 6: showing $A(\Sigma) \le A(\text{initial horizon})$.

But wait --- the Hawking Area Theorem says the event horizon area is \textbf{non-decreasing in time}. The initial horizon is \textbf{after} $\Sigma$ forms.

The question is: what is the area of the event horizon at the ``moment'' when $\Sigma$ exists?
\end{keyinsight}

%% ============================================================================
\section{Formalizing Weak Cosmic Censorship}
%% ============================================================================

\begin{definition}[Weak Cosmic Censorship (WCC)]\label{def:WCC}
A spacetime $(N, \bar{g})$ satisfies \textbf{WCC} if:
\begin{enumerate}
    \item The spacetime is globally hyperbolic
    \item There exists a complete future null infinity $\scri^+$
    \item The black hole region $\mathcal{B} = N \setminus J^-(\scri^+)$ is non-empty
    \item The event horizon $\mathcal{H} = \partial J^-(\scri^+)$ is a smooth null hypersurface
\end{enumerate}
\end{definition}

\begin{definition}[Strong Predictability]
$(N, \bar{g})$ is \textbf{strongly predictable} if there exists a partial Cauchy surface $\mathcal{C}$ such that $D^+(\mathcal{C}) \supset J^-(\scri^+)$.
\end{definition}

\textbf{Physical meaning:} The exterior of the black hole is determined by initial data on $\mathcal{C}$.

%% ============================================================================
\section{The Hawking Area Theorem}
%% ============================================================================

\begin{theorem}[Hawking Area Theorem \cite{hawking1971}]\label{thm:hawking-area}
Let $(N, \bar{g})$ be a spacetime satisfying:
\begin{enumerate}
    \item Null Energy Condition (NEC): $R_{\mu\nu}\ell^\mu\ell^\nu \ge 0$ for null $\ell$
    \item WCC (event horizon exists and is smooth)
\end{enumerate}
Let $\mathcal{C}_1, \mathcal{C}_2$ be two Cauchy surfaces with $\mathcal{C}_2$ to the future of $\mathcal{C}_1$. Then:
\begin{equation}
    A(\mathcal{H} \cap \mathcal{C}_2) \ge A(\mathcal{H} \cap \mathcal{C}_1)
\end{equation}
\end{theorem}

\begin{proof}[Proof sketch]
The event horizon $\mathcal{H}$ is generated by null geodesics. By the Raychaudhuri equation:
\begin{equation}
    \frac{d\theta}{d\lambda} = -\frac{\theta^2}{2} - \sigma^2 - R_{\mu\nu}\ell^\mu\ell^\nu
\end{equation}
where $\theta$ is the expansion of the null generators.

Under NEC, $\frac{d\theta}{d\lambda} \le -\frac{\theta^2}{2}$.

If $\theta < 0$ at any point, then $\theta \to -\infty$ in finite affine parameter (focusing theorem), creating a caustic. But the event horizon is the boundary of $J^-(\scri^+)$, which cannot have caustics (they would be points not in the boundary).

Therefore $\theta \ge 0$ on $\mathcal{H}$, so area is non-decreasing.
\end{proof}

%% ============================================================================
\section{The Key Step: Trapped Surface Inside Event Horizon}
%% ============================================================================

\begin{lemma}[Trapped Surface $\Rightarrow$ Inside Black Hole]\label{lem:trapped-inside}
Let $(N, \bar{g})$ satisfy NEC and WCC. Let $\Sigma$ be a trapped surface (both $\theta^+ < 0$ and $\theta^- < 0$). Then:
\begin{equation}
    \Sigma \subset \mathcal{B} = N \setminus J^-(\scri^+)
\end{equation}
i.e., $\Sigma$ is inside the black hole.
\end{lemma}

\begin{proof}
Suppose $\Sigma \not\subset \mathcal{B}$. Then some point $p \in \Sigma$ has $p \in J^-(\scri^+)$.

Consider the outgoing null hypersurface $\mathcal{N}^+$ from $\Sigma$. Since $\theta^+ < 0$ on $\Sigma$, the area of cross-sections of $\mathcal{N}^+$ decreases initially.

By the focusing theorem (NEC + $\theta^+ < 0$), the null geodesics converge and form a caustic in finite affine parameter.

But if $p \in J^-(\scri^+)$, then the outgoing null geodesic from $p$ should reach $\scri^+$, which is at infinite affine parameter.

Contradiction. Therefore $\Sigma \subset \mathcal{B}$.
\end{proof}

%% ============================================================================
\section{The Cross-Section Problem}
%% ============================================================================

\begin{keyinsight}
The event horizon $\mathcal{H}$ is a 3-dimensional null hypersurface. A Cauchy surface $\mathcal{C}$ intersects it in a 2-surface $\mathcal{H}_\mathcal{C} = \mathcal{H} \cap \mathcal{C}$.

The trapped surface $\Sigma$ lies on $\mathcal{C}$, inside $\mathcal{H}_\mathcal{C}$ (by Lemma~\ref{lem:trapped-inside}).

But ``inside'' in what sense? $\Sigma$ and $\mathcal{H}_\mathcal{C}$ are both 2-surfaces in the 3-manifold $\mathcal{C}$.
\end{keyinsight}

\begin{lemma}[Topological Containment]\label{lem:containment}
Let $\mathcal{C}$ be a Cauchy surface containing a trapped surface $\Sigma$. Let $\mathcal{H}_\mathcal{C} = \mathcal{H} \cap \mathcal{C}$ be the event horizon cross-section. Then:
\begin{equation}
    \Sigma \text{ is enclosed by } \mathcal{H}_\mathcal{C}
\end{equation}
in the sense that $\Sigma$ lies in the compact region bounded by $\mathcal{H}_\mathcal{C}$.
\end{lemma}

\begin{proof}
The black hole region $\mathcal{B}$ intersected with $\mathcal{C}$ gives $\mathcal{B}_\mathcal{C} = \mathcal{B} \cap \mathcal{C}$.

This is a 3-dimensional region with boundary $\mathcal{H}_\mathcal{C}$.

Since $\Sigma \subset \mathcal{B}$ and $\Sigma \subset \mathcal{C}$, we have $\Sigma \subset \mathcal{B}_\mathcal{C}$.

Therefore $\Sigma$ is enclosed by $\partial \mathcal{B}_\mathcal{C} = \mathcal{H}_\mathcal{C}$.
\end{proof}

%% ============================================================================
\section{Area Comparison on the Same Slice}
%% ============================================================================

Now we come to the critical question:

\begin{tcolorbox}[colback=red!5!white, colframe=red!75!black, title=\textbf{THE QUESTION}]
Is $A(\Sigma) \le A(\mathcal{H}_\mathcal{C})$?

That is, does the trapped surface have smaller area than the event horizon cross-section on the same Cauchy surface?
\end{tcolorbox}

\textbf{This is NOT implied by the Hawking Area Theorem.}

The Hawking theorem says $A(\mathcal{H}_{\mathcal{C}_2}) \ge A(\mathcal{H}_{\mathcal{C}_1})$ for $\mathcal{C}_2$ to the future of $\mathcal{C}_1$.

It says nothing about comparing $A(\Sigma)$ to $A(\mathcal{H}_\mathcal{C})$ when both are on the same slice.

\subsection{The Geometric Reality}

Consider Schwarzschild spacetime with a trapped surface at radius $r < 2M$.

On a constant-$t$ slice (Schwarzschild time), the trapped surface has area $4\pi r^2$ where $r < 2M$.

The event horizon cross-section has area $4\pi (2M)^2 = 16\pi M^2$.

So $A(\Sigma) = 4\pi r^2 < 16\pi M^2 = A(\mathcal{H}_\mathcal{C})$. ✓

But this is \textbf{not} a general theorem --- it's a calculation in Schwarzschild.

%% ============================================================================
\section{The Dynamical Argument}
%% ============================================================================

\begin{keyinsight}
The Hawking Area Theorem is about \textbf{time evolution}, not spatial comparison.

To compare $A(\Sigma)$ with $A(\mathcal{H}_\mathcal{C})$, we need a \textbf{dynamical argument}.
\end{keyinsight}

\begin{proposition}[Dynamical Area Bound]\label{prop:dynamical}
Let $\Sigma$ be a trapped surface on Cauchy surface $\mathcal{C}$. Suppose the spacetime evolved from initial data where no black hole was present. Then:
\begin{equation}
    A(\Sigma) \le A(\mathcal{H}_\mathcal{C})
\end{equation}
\end{proposition}

\begin{proof}[Proof Idea]
\textbf{Step 1:} At early times (before $\Sigma$ forms), the event horizon has small area (it's forming from gravitational collapse).

\textbf{Step 2:} The trapped surface $\Sigma$ forms at some time $t_\Sigma$.

\textbf{Step 3:} At time $t_\Sigma$, let $\mathcal{H}_{t_\Sigma}$ be the event horizon cross-section.

\textbf{Step 4:} By definition of trapped surface, $\Sigma$ is inside the black hole region, so $\Sigma$ is enclosed by $\mathcal{H}_{t_\Sigma}$.

\textbf{Step 5:} The question is whether ``enclosed'' implies ``smaller area.''

\textbf{Gap:} A surface can be enclosed by another surface of smaller area (think of a long thin tube around a spherical ball).
\end{proof}

%% ============================================================================
\section{The Isoperimetric Approach}
%% ============================================================================

\begin{lemma}[Isoperimetric Comparison]\label{lem:isoperimetric}
In a Riemannian 3-manifold $(M, g)$ with $R_g \ge 0$, if $\Sigma_1$ is enclosed by $\Sigma_2$ and both bound simply connected regions, then:
\begin{equation}
    A(\Sigma_1) \le A(\Sigma_2) \quad \text{(FALSE in general)}
\end{equation}
\end{lemma}

The counterexample is a ``tube'': $\Sigma_2$ can be a long thin tube with small area enclosing $\Sigma_1$.

\textbf{However:} If $\Sigma_2$ is the event horizon, it has special properties.

\begin{proposition}[Event Horizon is Outer-Minimizing]\label{prop:horizon-outer-min}
Under WCC, the event horizon cross-section $\mathcal{H}_\mathcal{C}$ is \textbf{outer-minimizing} in $(M, g)$ where $(M, g, k)$ is the induced data on $\mathcal{C}$.
\end{proposition}

\begin{proof}[Proof attempt]
The event horizon $\mathcal{H}$ has $\theta^+ \ge 0$ (it's not trapped).

On a Cauchy surface, $\mathcal{H}_\mathcal{C}$ has:
\begin{equation}
    \theta^+_{\mathcal{H}_\mathcal{C}} = H_{\mathcal{H}_\mathcal{C}} + \tr_{\mathcal{H}_\mathcal{C}} k \ge 0
\end{equation}

\textbf{Problem:} This doesn't make $\mathcal{H}_\mathcal{C}$ outer-minimizing. Outer-minimizing requires $A(\mathcal{H}_\mathcal{C}) \le A(\Sigma')$ for all $\Sigma'$ enclosing $\mathcal{H}_\mathcal{C}$.

The condition $\theta^+ \ge 0$ is about null geometry, not area minimization.
\end{proof}

%% ============================================================================
\section{Resolution: The Apparent Horizon Bound}
%% ============================================================================

\begin{theorem}[Apparent Horizon Bounds Event Horizon]\label{thm:AH-bound}
Under WCC, the apparent horizon (outermost MOTS) $\Sigma^*_\mathcal{C}$ on any Cauchy surface $\mathcal{C}$ satisfies:
\begin{equation}
    \Sigma^*_\mathcal{C} \subseteq \overline{\mathcal{B}_\mathcal{C}}
\end{equation}
i.e., the apparent horizon is inside or on the event horizon.
\end{theorem}

\begin{proof}
The apparent horizon $\Sigma^*$ has $\theta^+ = 0$. Outside $\Sigma^*$, we have $\theta^+ > 0$.

The event horizon $\mathcal{H}$ has $\theta^+ \ge 0$ everywhere.

If $\Sigma^*$ were outside $\mathcal{H}$, then $\mathcal{H}$ would have $\theta^+ > 0$ (since it's inside the region where $\theta^+ > 0$). But on the event horizon itself, $\theta^+ = 0$ (the horizon is generated by marginally trapped null geodesics).

Therefore $\Sigma^* \subseteq \overline{\mathcal{B}}$, i.e., $\Sigma^*$ is inside or on the event horizon.
\end{proof}

\begin{corollary}[Area Bound via Apparent Horizon]
If the apparent horizon $\Sigma^*_\mathcal{C}$ is outer-minimizing, then for any trapped surface $\Sigma \subset \mathcal{C}$:
\begin{equation}
    A(\Sigma) \le A(\Sigma^*_\mathcal{C}) \le A(\mathcal{H}_\mathcal{C})
\end{equation}
\end{corollary}

\begin{proof}
\textbf{First inequality:} By outer-minimizing property of $\Sigma^*$ (this is the assumption).

\textbf{Second inequality:} $\Sigma^*$ is enclosed by $\mathcal{H}_\mathcal{C}$ (by Theorem~\ref{thm:AH-bound}). Since $\Sigma^*$ is outer-minimizing:
\begin{equation}
    A(\Sigma^*) \le A(\mathcal{H}_\mathcal{C})
\end{equation}
\end{proof}

%% ============================================================================
\section{The Remaining Gap}
%% ============================================================================

\begin{tcolorbox}[colback=yellow!10!white, colframe=orange!75!black, title=\textbf{THE REMAINING GAP}]
The argument requires the apparent horizon $\Sigma^*$ to be \textbf{outer-minimizing}.

This is exactly the (OM) assumption from the conditional proof!

The WCC + Hawking argument does not eliminate the need for (OM).
\end{tcolorbox}

\textbf{What WCC does give us:}
\begin{enumerate}
    \item Trapped surfaces are inside the black hole (Lemma~\ref{lem:trapped-inside})
    \item Apparent horizon is inside event horizon (Theorem~\ref{thm:AH-bound})
    \item Event horizon area increases in time (Hawking)
\end{enumerate}

\textbf{What WCC does NOT give us:}
\begin{enumerate}
    \item $A(\Sigma) \le A(\Sigma^*)$ (trapped vs. apparent)
    \item $A(\Sigma^*) \le A(\mathcal{H})$ (apparent vs. event)
    \item The outer-minimizing property of $\Sigma^*$
\end{enumerate}

%% ============================================================================
\section{New Insight: The Formation Picture}
%% ============================================================================

\begin{keyinsight}
Consider the \textbf{formation} of the trapped surface $\Sigma$.

Before $\Sigma$ forms, there is no trapped surface. As matter collapses, a trapped surface $\Sigma$ appears.

At the moment of formation, $\Sigma$ is the outermost trapped surface (the apparent horizon).

Therefore, at formation: $\Sigma = \Sigma^*$.

If the outer-minimizing property holds at formation, then:
\begin{equation}
    A(\Sigma) = A(\Sigma^*) \le A(\mathcal{H})
\end{equation}
at the moment of formation.
\end{keyinsight}

\begin{proposition}[Area Bound at Formation]\label{prop:formation}
Let $\Sigma$ be the first trapped surface to form in gravitational collapse. Then:
\begin{equation}
    A(\Sigma) \le A(\mathcal{H}_{\text{formation}})
\end{equation}
\end{proposition}

\begin{proof}[Proof Idea]
At the moment of formation, the trapped region is just appearing. The apparent horizon $\Sigma^*$ coincides with $\Sigma$ (or $\Sigma$ is inside $\Sigma^*$).

The event horizon $\mathcal{H}$ at this moment encloses $\Sigma^*$ (by Theorem~\ref{thm:AH-bound}).

By the isoperimetric properties of the moment of formation (the geometry is close to spherical), we have $A(\Sigma) \le A(\mathcal{H})$.

\textbf{Gap:} This argument is not rigorous --- it appeals to ``closeness to spherical'' without proof.
\end{proof}

%% ============================================================================
\section{The Spherical Collapse Case}
%% ============================================================================

For \textbf{spherically symmetric} collapse (Oppenheimer-Snyder), we can prove:

\begin{theorem}[Penrose 1973 for Spherical Collapse]
In spherically symmetric gravitational collapse satisfying NEC, for any trapped surface $\Sigma$:
\begin{equation}
    M_{\ADM} \ge \sqrt{\frac{A(\Sigma)}{16\pi}}
\end{equation}
\end{theorem}

\begin{proof}
In spherical symmetry, all surfaces are spheres, characterized by their area radius $r$.

A trapped surface has radius $r < r_{\text{AH}}$ where $r_{\text{AH}}$ is the apparent horizon radius.

The event horizon has radius $r_{\text{EH}} \ge r_{\text{AH}}$.

Therefore:
\begin{equation}
    A(\Sigma) = 4\pi r^2 < 4\pi r_{\text{AH}}^2 \le 4\pi r_{\text{EH}}^2 = A(\mathcal{H})
\end{equation}

The Penrose inequality follows from:
\begin{equation}
    M_{\ADM} \ge \sqrt{\frac{A(\mathcal{H})}{16\pi}} \ge \sqrt{\frac{A(\Sigma)}{16\pi}}
\end{equation}
where the first inequality is proven (Bondi mass loss + final state is Schwarzschild).
\end{proof}

%% ============================================================================
\section{Conclusion}
%% ============================================================================

\begin{tcolorbox}[colback=green!10!white, colframe=green!75!black, title=\textbf{CONCLUSION}]
\textbf{The WCC + Hawking approach does not resolve the area dominance gap.}

Specifically:
\begin{enumerate}
    \item Hawking's theorem is about time evolution, not spatial comparison
    \item The containment $\Sigma \subset \mathcal{H}$ does not imply $A(\Sigma) \le A(\mathcal{H})$
    \item The outer-minimizing property is still required and unproven
\end{enumerate}

\textbf{What IS proven:}
\begin{enumerate}
    \item Penrose 1973 holds for spherically symmetric collapse
    \item Penrose 1973 holds conditionally on (OM)
    \item The structure of the argument is correct; only (OM) is missing
\end{enumerate}

\textbf{The 1973 conjecture remains open for general (non-spherical) trapped surfaces.}
\end{tcolorbox}

%% ============================================================================
\section{Next Attack Vector: Perturbative Stability}
%% ============================================================================

A new approach: prove (OM) by perturbation from spherical symmetry.

\textbf{Idea:} 
\begin{enumerate}
    \item In spherical symmetry, $A(\Sigma) \le A(\Sigma^*)$ trivially (all surfaces are spheres)
    \item For small perturbations from spherical symmetry, the inequality should persist
    \item Build up general case by continuity in the space of initial data
\end{enumerate}

\textbf{This is a promising direction for future work.}

\begin{thebibliography}{99}
\bibitem{penrose1973} R. Penrose, Naked singularities, \textit{Ann. New York Acad. Sci.} 224 (1973).
\bibitem{hawking1971} S.W. Hawking, Gravitational radiation from colliding black holes, \textit{Phys. Rev. Lett.} 26 (1971).
\bibitem{mars2009} M. Mars, Present status of the Penrose inequality, \textit{Class. Quant. Grav.} 26 (2009).
\end{thebibliography}

\end{document}
