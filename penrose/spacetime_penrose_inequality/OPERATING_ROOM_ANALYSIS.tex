\documentclass[11pt,a4paper]{article}
\usepackage[utf8]{inputenc}
\usepackage{amsmath,amssymb,amsthm}
\usepackage{geometry}
\usepackage{booktabs}
\usepackage{array}
\usepackage{xcolor}
\usepackage{tcolorbox}
\usepackage{tikz}
\usetikzlibrary{shapes,arrows,positioning}

\geometry{margin=2cm}

\newtheorem{theorem}{Theorem}
\newtheorem{lemma}[theorem]{Lemma}
\newtheorem{proposition}[theorem]{Proposition}
\newtheorem{definition}[theorem]{Definition}
\newtheorem{remark}[theorem]{Remark}
\newtheorem{conjecture}[theorem]{Conjecture}

% Surgery color coding
\newcommand{\blade}[1]{\textcolor{red!80!black}{\textbf{[BLADE]} #1}}
\newcommand{\suture}[1]{\textcolor{blue!70!black}{\textbf{[SUTURE]} #1}}
\newcommand{\clamp}[1]{\textcolor{orange!80!black}{\textbf{[CLAMP]} #1}}
\newcommand{\wound}[1]{\textcolor{purple!80!black}{\textbf{[OPEN WOUND]} #1}}
\newcommand{\healed}[1]{\textcolor{green!60!black}{\textbf{[HEALED]} #1}}

\title{\textbf{Operating Room Analysis:\\Surgical Annotation of the Penrose 1973 Program}}
\author{Spacetime Penrose Inequality --- Surgical Roadmap}
\date{December 2025}

\begin{document}
\maketitle

\begin{abstract}
This document provides a line-by-line surgical annotation of the boost-invariant quasi-local mass approach to the spacetime Penrose inequality. Each formula is classified as:
\begin{itemize}
\item \blade{BLADE}: The critical cutting tool that eliminates bad terms
\item \suture{SUTURE}: Technical steps that close the argument
\item \clamp{CLAMP}: Control mechanisms that prevent bleeding (divergence)
\item \wound{OPEN WOUND}: Gaps that remain unhealed
\item \healed{HEALED}: Rigorously proven components
\end{itemize}
\end{abstract}

\tableofcontents

\section{The Patient: Hawking Mass Variation}

\subsection{The Disease: Indefinite Signs in Null Variation}

The Hawking mass variation along outgoing null is:
\begin{equation}
\boxed{\frac{dm_H}{ds} = \frac{\sqrt{|\Sigma|/16\pi}}{16\pi} \int_\Sigma \left[(\mu - |J|)\Psi - 2\sigma^+:\sigma^- - |\zeta|^2 + \cdots\right] dA}
\end{equation}

\begin{tcolorbox}[colback=red!5!white, colframe=red!75!black, title=Diagnosis: Three Malignant Terms]
\begin{enumerate}
\item $\bad{-2\sigma^+:\sigma^-}$ \hfill \textbf{[TUMOR \#1: Shear coupling]}
\begin{itemize}
\item Indefinite sign: $\sigma^+:\sigma^- = \mathrm{tr}(\sigma^+_{ab}\sigma^{-ab})$ can be $\gtrless 0$
\item \textbf{Prognosis}: Fatal without surgery
\end{itemize}

\item $\bad{-|\zeta|^2}$ \hfill \textbf{[TUMOR \#2: Twist]}
\begin{itemize}
\item Always negative: directly destroys monotonicity
\item \textbf{Prognosis}: Must be absorbed into definition
\end{itemize}

\item $\bad{\theta^+ \to 0 \text{ or } -\infty}$ \hfill \textbf{[TUMOR \#3: Geometric degeneration]}
\begin{itemize}
\item MOTS: $\theta^+ = 0$ causes $\sigma^+/\theta^+$ to diverge
\item Caustic: $\theta^+ \to -\infty$ causes flow breakdown
\item \textbf{Prognosis}: Cannot be cut; must be bypassed (jump surgery)
\end{itemize}
\end{enumerate}
\end{tcolorbox}

\section{Surgery \#1: The Completing-Square Blade}

\subsection{The Blade}

\begin{tcolorbox}[colback=red!10!white, colframe=red!80!black, title=\blade{PRIMARY BLADE: Completing the Square}]
\begin{equation}
\boxed{-\sigma^+:\sigma^- = -\frac{1}{4}|\sigma^+ + \sigma^-|^2 + \frac{1}{4}|\sigma^+ - \sigma^-|^2}
\end{equation}

\textbf{Surgical action}: Converts indefinite bilinear term into difference of two definite squares.

\textbf{Status}: \healed{PROVEN} --- Pure algebra, no assumptions needed.
\end{tcolorbox}

\subsection{Why This Cut Works}

\blade{The blade exploits the algebraic structure of symmetric trace-free tensors:}

For any $\sigma^+, \sigma^- \in S^2_0(T^*\Sigma)$ (symmetric, trace-free):
\begin{align}
|\sigma^+ + \sigma^-|^2 &= |\sigma^+|^2 + 2\sigma^+:\sigma^- + |\sigma^-|^2 \\
|\sigma^+ - \sigma^-|^2 &= |\sigma^+|^2 - 2\sigma^+:\sigma^- + |\sigma^-|^2
\end{align}

Subtracting:
\begin{equation}
|\sigma^+ + \sigma^-|^2 - |\sigma^+ - \sigma^-|^2 = 4\sigma^+:\sigma^-
\end{equation}

Therefore:
\begin{equation}
-\sigma^+:\sigma^- = \frac{1}{4}\left(|\sigma^+ - \sigma^-|^2 - |\sigma^+ + \sigma^-|^2\right)
\end{equation}

\healed{This is the exact analog of Wang Hong's multilinear decomposition: converting overlap into orthogonal pieces.}

\section{Surgery \#2: The Boost-Invariance Blade}

\subsection{The Blade}

\begin{tcolorbox}[colback=red!10!white, colframe=red!80!black, title=\blade{SECONDARY BLADE: Boost-Invariant Normalization}]
\begin{equation}
\boxed{\Delta_{\text{inv}} := \frac{\sigma^+}{\theta^+} - \frac{\sigma^-}{\theta^-}}
\end{equation}

\textbf{Surgical action}: Creates a gauge-invariant combination that:
\begin{enumerate}
\item Has boost weight 0 (invariant under $\ell \to \lambda\ell$, $n \to \lambda^{-1}n$)
\item Can be squared to give a definite-sign term
\item Exposes the true degeneracy at MOTS/caustic
\end{enumerate}

\textbf{Status}: \healed{WELL-DEFINED} where $\theta^+\theta^- \neq 0$
\end{tcolorbox}

\subsection{Boost Weight Analysis}

\suture{Technical verification of boost invariance:}

\begin{center}
\begin{tabular}{c|c|c|c}
\toprule
Quantity & Definition & Boost $\ell \to \lambda\ell$ & Weight \\
\midrule
$\theta^+$ & $q^{ab}\nabla_a \ell_b$ & $\to \lambda\theta^+$ & $+1$ \\
$\sigma^+$ & $(\nabla_a\ell_b)_{\text{TF}}$ & $\to \lambda\sigma^+$ & $+1$ \\
$\theta^-$ & $q^{ab}\nabla_a n_b$ & $\to \lambda^{-1}\theta^-$ & $-1$ \\
$\sigma^-$ & $(\nabla_a n_b)_{\text{TF}}$ & $\to \lambda^{-1}\sigma^-$ & $-1$ \\
\midrule
$\sigma^+/\theta^+$ & ratio & $\to \sigma^+/\theta^+$ & $0$ \\
$\sigma^-/\theta^-$ & ratio & $\to \sigma^-/\theta^-$ & $0$ \\
$\Delta_{\text{inv}}$ & difference & $\to \Delta_{\text{inv}}$ & $0$ \\
\bottomrule
\end{tabular}
\end{center}

\healed{The combination $\Delta_{\text{inv}} = \sigma^+/\theta^+ - \sigma^-/\theta^-$ is boost-invariant.}

\subsection{The Price: MOTS Singularity}

\begin{tcolorbox}[colback=orange!10!white, colframe=orange!80!black, title=\clamp{BLEEDING POINT: Division by $\theta^+$}]
When $\theta^+ \to 0$ (approaching MOTS):
\begin{equation}
\frac{\sigma^+}{\theta^+} \to \begin{cases} +\infty & \text{if } \sigma^+ > 0 \\ -\infty & \text{if } \sigma^+ < 0 \\ \text{indeterminate} & \text{if } \sigma^+ = 0 \end{cases}
\end{equation}

\textbf{This is NOT a removable singularity.}

The algebraic identity shows this is \textbf{forced by boost invariance}:
\begin{itemize}
\item Dimension: $[\sigma] = [\theta] = L^{-1}$
\item Only dimensionless ratios can be boost-invariant
\item Division by $\theta^\pm$ is \textbf{unavoidable}
\end{itemize}
\end{tcolorbox}

\section{The Corrected Mass Functional}

\subsection{Definition of $\mathcal{Q}$}

\begin{tcolorbox}[colback=green!5!white, colframe=green!60!black, title=\healed{SUTURED DEFINITION}]
\begin{equation}
\boxed{\mathcal{Q}(\Sigma) = \sqrt{\frac{|\Sigma|}{16\pi}}\left(1 - \frac{1}{16\pi}\int_\Sigma \left[\theta^+\theta^- + |\zeta|^2 + \frac{1}{4}|\Delta_{\text{inv}}|^2\theta^+\theta^-\right] dA\right)}
\end{equation}

where $\Delta_{\text{inv}} = \frac{\sigma^+}{\theta^+} - \frac{\sigma^-}{\theta^-}$.

\textbf{Domain}: Surfaces with $\theta^+\theta^- \neq 0$ (no MOTS).

\textbf{Status}: \healed{WELL-DEFINED} on domain.
\end{tcolorbox}

\subsection{What Each Term Does}

\begin{center}
\renewcommand{\arraystretch}{1.5}
\begin{tabular}{|c|c|c|c|}
\hline
\textbf{Term} & \textbf{Origin} & \textbf{Surgery} & \textbf{Status} \\
\hline
$\theta^+\theta^-$ & Hawking mass & None needed & \healed{Inherited} \\
\hline
$|\zeta|^2$ & Twist (bad: $-|\zeta|^2$) & \blade{Absorb with $+$ sign} & \healed{Cured} \\
\hline
$\frac{1}{4}|\Delta_{\text{inv}}|^2\theta^+\theta^-$ & Shear (bad: $-\sigma^+:\sigma^-$) & \blade{Complete square + boost inv.} & \healed{Cured} \\
\hline
\end{tabular}
\end{center}

\section{The Variation Formula: Line-by-Line Annotation}

\subsection{Main Theorem}

\begin{theorem}[$\mathcal{Q}$-Monotonicity on Smooth Flows]
Let $\{\Sigma_s\}$ be a smooth null flow with $|\theta^\pm| \geq \delta > 0$. Under DEC:
\begin{equation}
\frac{d\mathcal{Q}}{ds} \geq 0
\end{equation}
\end{theorem}

\subsection{Proof Dissection}

\textbf{Step 1: Area evolution} \suture{Standard}
\begin{equation}
\frac{d|\Sigma|}{ds} = \int_\Sigma \theta^+ \, dA
\end{equation}
\healed{Proven: First variation of area along null.}

\textbf{Step 2: $\theta^+\theta^-$ evolution} \suture{Raychaudhuri}
\begin{equation}
\frac{d(\theta^+\theta^-)}{ds} = \theta^-\left(-\frac{(\theta^+)^2}{2} - |\sigma^+|^2 - 8\pi T_{\ell\ell}\right) + \text{transport terms}
\end{equation}
\healed{Proven: Direct from Raychaudhuri equation.}

\textbf{Step 3: DEC contribution} \blade{The good term}
\begin{equation}
-8\pi\theta^- T_{\ell\ell} = -8\pi\theta^-(\mu - J\cdot\ell) \geq 0 \quad \text{when } \theta^- < 0
\end{equation}
\healed{DEC gives $\mu \geq |J|$, so $\mu - J\cdot\ell \geq 0$.}

\textbf{Step 4: Shear terms after completing square} \blade{Critical surgery}
\begin{align}
&-\theta^-|\sigma^+|^2 - \theta^+|\sigma^-|^2 - 2\sigma^+:\sigma^- \\
&= -\theta^-|\sigma^+|^2 - \theta^+|\sigma^-|^2 + \frac{1}{2}|\sigma^+ - \sigma^-|^2 - \frac{1}{2}|\sigma^+ + \sigma^-|^2
\end{align}

\clamp{Need to verify:} The combination with boost-invariant $|\Delta_{\text{inv}}|^2\theta^+\theta^-$ gives:
\begin{equation}
\frac{d}{ds}\left(\frac{1}{4}|\Delta_{\text{inv}}|^2\theta^+\theta^-\right) = \text{???}
\end{equation}

\wound{GAP: The full evolution of the correction term requires careful computation. Currently assumed but not fully verified.}

\textbf{Step 5: Twist term} \blade{Absorbed}
\begin{equation}
\frac{d|\zeta|^2}{ds} = \text{transport} + \text{curvature coupling}
\end{equation}
\clamp{Need to verify:} The $|\zeta|^2$ term in $\mathcal{Q}$ exactly cancels the $-|\zeta|^2$ from Hawking mass variation.

\wound{GAP: Full evolution of $|\zeta|^2$ along null flow needs verification.}

\section{Surgery \#3: Weak Null Flow with Jumps}

\subsection{The Problem}

\begin{tcolorbox}[colback=purple!5!white, colframe=purple!80!black, title=\wound{OPEN WOUND: Flow Breakdown}]
The smooth flow breaks down at:
\begin{enumerate}
\item \textbf{Caustics}: $\theta^+ \to -\infty$ (focusing singularity)
\item \textbf{MOTS approach}: $\theta^+ \to 0$ (our $\mathcal{Q}$ diverges)
\end{enumerate}

\textbf{Current status}: No rigorous weak flow theory for null hypersurfaces.
\end{tcolorbox}

\subsection{Proposed Surgery: Jump to Outer Hull}

\begin{definition}[MOTS-Avoiding Weak Null Flow]
A flow $\{\Sigma_s\}$ that:
\begin{enumerate}
\item[(WA1)] Starts at trapped surface $\Sigma_0$ with $\theta^+ < 0$
\item[(WA2)] Evolves smoothly where $|\theta^\pm| \geq \delta > 0$
\item[(WA3)] \clamp{Jumps to outer hull when $\theta^+ \to -\infty$ (caustic)}
\item[(WA4)] \clamp{Jumps to outer hull when $|\theta^+| < \delta$ (near MOTS)}
\item[(WA5)] Reaches $\mathscr{I}^+$
\end{enumerate}
\end{definition}

\subsection{What ``Outer Hull'' Means}

\wound{CRITICAL GAP: Need Lorentzian geometric measure theory}

In Riemannian (Huisken-Ilmanen):
\begin{itemize}
\item Outer hull = outward minimizing hull in sense of currents
\item Area is lower semicontinuous at jumps
\item Hawking mass is monotonic across jumps
\end{itemize}

In Lorentzian/null:
\begin{itemize}
\item \wound{No established theory of null currents}
\item \wound{``Outward minimizing'' unclear on null hypersurface}
\item \wound{$\mathcal{Q}$-monotonicity at jumps unproven}
\end{itemize}

\subsection{Jump Monotonicity Conjecture}

\begin{tcolorbox}[colback=purple!10!white, colframe=purple!80!black, title=\wound{CORE OPEN WOUND}]
\begin{conjecture}[Jump Monotonicity]
At each jump $\Sigma^- \to \Sigma^+$ (outer hull):
\begin{equation}
\mathcal{Q}(\Sigma^+) \geq \mathcal{Q}(\Sigma^-)
\end{equation}
\end{conjecture}

\textbf{Difficulty 1}: $\mathcal{Q}(\Sigma^-)$ may be $-\infty$ if approaching MOTS with $\sigma^+ \neq 0$

\textbf{Difficulty 2}: Outer hull definition in null geometry

\textbf{Difficulty 3}: Comparison principle for $\mathcal{Q}$

\textbf{Status}: \wound{COMPLETELY OPEN}
\end{tcolorbox}

\section{Asymptotic Analysis: Convergence to Bondi Mass}

\subsection{The Claim}

\begin{equation}
\lim_{s \to \infty} \mathcal{Q}(\Sigma_s) = M_B
\end{equation}

\subsection{Verification}

\suture{Asymptotic expansion in Bondi coordinates:}

Near $\mathscr{I}^+$ at large areal radius $r$:
\begin{align}
|\Sigma_r| &= 4\pi r^2 + O(r) \\
\theta^+ &= \frac{2}{r} + O(r^{-2}) \\
\theta^- &= -\frac{1}{r} + O(r^{-2}) \\
\sigma^\pm &= O(r^{-2}) \quad \text{(news function decay)} \\
\zeta &= O(r^{-2})
\end{align}

Therefore:
\begin{align}
\theta^+\theta^- &= -\frac{2}{r^2} + O(r^{-3}) \\
|\Delta_{\text{inv}}|^2\theta^+\theta^- &= O(r^{-2}) \cdot O(r^{-2}) = O(r^{-4}) \\
|\zeta|^2 &= O(r^{-4})
\end{align}

The Hawking mass contribution:
\begin{equation}
m_H(\Sigma_r) = \frac{r}{2}\left(1 + \frac{2}{r^2} \cdot \frac{4\pi r^2}{16\pi} + O(r^{-1})\right) = M_B + O(r^{-1})
\end{equation}

Correction terms contribute $O(r^{-1})$, so:
\begin{equation}
\mathcal{Q}(\Sigma_r) = M_B + O(r^{-1})
\end{equation}

\healed{The asymptotic analysis is rigorous given standard Bondi falloff.}

\section{Complete Surgical Status Report}

\begin{center}
\renewcommand{\arraystretch}{1.8}
\begin{tabular}{|>{\raggedright}p{4.5cm}|c|>{\raggedright\arraybackslash}p{6cm}|}
\hline
\textbf{Component} & \textbf{Status} & \textbf{Notes} \\
\hline
\hline
Completing-square identity & \healed{HEALED} & Pure algebra \\
\hline
Boost invariance of $\Delta_{\text{inv}}$ & \healed{HEALED} & Direct verification \\
\hline
Definition of $\mathcal{Q}$ & \healed{HEALED} & Well-defined where $\theta^\pm \neq 0$ \\
\hline
Smooth-flow monotonicity & \clamp{CLAMPED} & Conditional on full variation formula \\
\hline
Full variation formula for $\mathcal{Q}$ & \wound{WOUND} & Needs detailed computation \\
\hline
Weak null flow existence & \wound{WOUND} & No Lorentzian GMT \\
\hline
Outer hull definition (null) & \wound{WOUND} & Foundational gap \\
\hline
Jump monotonicity & \wound{WOUND} & Core open problem \\
\hline
Asymptotic $\mathcal{Q} \to M_B$ & \healed{HEALED} & Standard Bondi analysis \\
\hline
Initial value for outermost trapped & \clamp{CLAMPED} & Needs careful analysis \\
\hline
Rigidity (equality case) & \wound{WOUND} & Conditional on everything above \\
\hline
\end{tabular}
\end{center}

\section{Comparison: Wang Hong vs. Penrose 1973}

\begin{center}
\renewcommand{\arraystretch}{1.5}
\begin{tabular}{|>{\raggedright}p{3cm}|>{\raggedright}p{5.5cm}|>{\raggedright\arraybackslash}p{5.5cm}|}
\hline
\textbf{Surgical Element} & \textbf{Kakeya (Wang Hong)} & \textbf{Penrose 1973 (This Program)} \\
\hline
\hline
\textbf{Primary Blade} & Multilinear Kakeya decomposition & Completing-the-square identity \\
\hline
\textbf{Secondary Blade} & Wave packet orthogonality & Boost-invariant normalization \\
\hline
\textbf{Gauge Freedom} & Translation/rotation of tubes & Boost $\ell \to \lambda\ell$, $n \to \lambda^{-1}n$ \\
\hline
\textbf{Bad Structure} & Tube overlap $\to$ measure blowup & Shear coupling $\to$ sign indefiniteness \\
\hline
\textbf{Singularity Type} & Kakeya set (measure zero) & MOTS ($\theta^+=0$), Caustic ($\theta^+=-\infty$) \\
\hline
\textbf{Surgery for Singularity} & Induction on scales & Jump to outer hull \\
\hline
\textbf{Telescoping} & Multilinear expansion & Weak flow with controlled jumps \\
\hline
\textbf{Sharp Constant} & Achieved via orthogonality & DEC + square completion \\
\hline
\textbf{Status} & \healed{SOLVED} & \wound{OPEN} (jumps unproven) \\
\hline
\end{tabular}
\end{center}

\section{What Would Close the Proof}

\begin{tcolorbox}[colback=yellow!5!white, colframe=yellow!75!black, title=\textbf{Three Paths to Closure}]

\textbf{Path A: Full Weak Flow Theory}
\begin{enumerate}
\item Develop Lorentzian geometric measure theory for null hypersurfaces
\item Define outer hull via variational principle
\item Prove jump monotonicity for $\mathcal{Q}$
\item \textbf{Difficulty}: Foundational; likely years of work
\end{enumerate}

\textbf{Path B: Restricted Class (Perturbative)}
\begin{enumerate}
\item Restrict to perturbations of Schwarzschild/Kerr
\item No jumps needed (smooth flow exists)
\item Monotonicity follows from smooth analysis
\item \textbf{Difficulty}: Not the full conjecture; publishable but limited
\end{enumerate}

\textbf{Path C: Alternative Monotonic Quantity}
\begin{enumerate}
\item Find a different $\tilde{\mathcal{Q}}$ that avoids $\theta^\pm$ in denominators
\item Must still be boost-invariant and satisfy DEC monotonicity
\item Gap 2 analysis suggests this may be impossible
\item \textbf{Difficulty}: Algebraic obstruction (see gap2\_analysis.tex)
\end{enumerate}

\end{tcolorbox}

\section{Conclusion: The Operating Room Summary}

\begin{tcolorbox}[colback=blue!5!white, colframe=blue!75!black, title=\textbf{Final Surgical Report}]

\textbf{Patient}: Spacetime Penrose Inequality (1973)

\textbf{Primary Surgery}: Boost-invariant quasi-local mass with completing-the-square

\textbf{Surgical Blades Used}:
\begin{enumerate}
\item $-\sigma^+:\sigma^- = -\frac{1}{4}|\sigma^++\sigma^-|^2 + \frac{1}{4}|\sigma^+-\sigma^-|^2$ \hfill \healed{SUCCESS}
\item $\Delta_{\text{inv}} = \frac{\sigma^+}{\theta^+} - \frac{\sigma^-}{\theta^-}$ (boost invariant) \hfill \healed{SUCCESS}
\end{enumerate}

\textbf{Sutures Applied}:
\begin{itemize}
\item Definition of $\mathcal{Q}$ \hfill \healed{CLOSED}
\item Asymptotic convergence to $M_B$ \hfill \healed{CLOSED}
\item Model verification (Schwarzschild, Kerr) \hfill \healed{CLOSED}
\end{itemize}

\textbf{Clamps Holding}:
\begin{itemize}
\item Smooth-flow monotonicity \hfill \clamp{CONDITIONAL}
\item Initial value analysis \hfill \clamp{CONDITIONAL}
\end{itemize}

\textbf{Open Wounds}:
\begin{itemize}
\item Weak null flow existence \hfill \wound{CRITICAL}
\item Jump monotonicity \hfill \wound{CRITICAL}
\item Outer hull definition (null GMT) \hfill \wound{FOUNDATIONAL}
\end{itemize}

\textbf{Prognosis}: \textbf{Surgery 60\% complete}. Core blades are sharp and effective. Closure requires either (A) foundational GMT work, (B) restriction to perturbative regime, or (C) discovery of alternative quantity (unlikely per algebraic analysis).

\end{tcolorbox}

\end{document}
