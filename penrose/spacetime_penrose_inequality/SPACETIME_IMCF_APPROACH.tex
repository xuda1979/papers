%% SPACETIME_IMCF_APPROACH.tex
%%
%% INVERSE MEAN CURVATURE FLOW IN SPACETIME
%%
%% Key insight: Generalize Huisken-Ilmanen's IMCF to the spacetime setting
%% using null expansions instead of mean curvature.
%%
%% The flow evolves surfaces outward with speed 1/θ⁺, maintaining
%% control on both null expansions.
%%
%% December 2025

\documentclass[11pt]{amsart}
\usepackage{amsmath,amssymb,amsthm}
\usepackage{tcolorbox}

\tcbuselibrary{theorems}

\newtcolorbox{maintheorem}{
    colback=green!5!white,
    colframe=green!50!black,
    title={\textbf{MAIN THEOREM}}
}

\newtcolorbox{keylemma}{
    colback=blue!5!white,
    colframe=blue!75!black,
    title={\textbf{KEY LEMMA}}
}

\newtcolorbox{proofstep}{
    colback=gray!5!white,
    colframe=gray!50!black,
    title={\textbf{PROOF STEP}}
}

\newtcolorbox{insight}{
    colback=purple!5!white,
    colframe=purple!75!black,
    title={\textbf{INSIGHT}}
}

\newtcolorbox{gap}{
    colback=red!5!white,
    colframe=red!75!black,
    title={\textbf{GAP/ISSUE}}
}

\newtheorem{theorem}{Theorem}[section]
\newtheorem{lemma}[theorem]{Lemma}
\newtheorem{proposition}[theorem]{Proposition}
\newtheorem{corollary}[theorem]{Corollary}
\theoremstyle{definition}
\newtheorem{definition}[theorem]{Definition}
\newtheorem{remark}[theorem]{Remark}

\newcommand{\Area}{\mathrm{Area}}
\newcommand{\Vol}{\mathrm{Vol}}
\newcommand{\divv}{\mathrm{div}}
\DeclareMathOperator{\tr}{tr}

\title{Spacetime Inverse Mean Curvature Flow:\\
A Path to Penrose 1973}
\author{December 2025}

\begin{document}
\maketitle

\begin{abstract}
We develop a spacetime generalization of inverse mean curvature flow 
suitable for proving the Penrose inequality with general initial data. 
The key innovation is using the outgoing null expansion $\theta^+$ instead 
of mean curvature $H$, which naturally incorporates the extrinsic curvature 
$k$ and responds correctly to the dominant energy condition.
\end{abstract}

%% ============================================================================
\section{Motivation: Why Spacetime IMCF?}
%% ============================================================================

\begin{insight}
\textbf{Huisken-Ilmanen for Riemannian Penrose}

For time-symmetric data ($k = 0$), Huisken-Ilmanen proved Penrose via IMCF:
\begin{equation}
    \frac{\partial \Sigma_t}{\partial t} = \frac{\nu}{H}
\end{equation}

Key properties:
\begin{enumerate}
    \item Area increases: $\frac{d}{dt}\Area(\Sigma_t) = \Area(\Sigma_t)$
    \item Hawking mass monotone: $\frac{d}{dt}m_H(\Sigma_t) \ge 0$ under $R \ge 0$
    \item Limit gives: $m_H(\Sigma_\infty) = M_{\text{ADM}}$
\end{enumerate}

For general data: $H = \theta^+ - P$ where $P = \tr_\Sigma k$.

The DEC controls $\theta^+$ via Raychaudhuri, NOT $H$ directly.
\end{insight}

\begin{insight}
\textbf{The Key Observation}

Replace IMCF by \textbf{Inverse Expansion Flow}:
\begin{equation}
    \frac{\partial \Sigma_t}{\partial t} = \frac{\nu}{\theta^+}
\end{equation}

This flow:
\begin{itemize}
    \item Uses $\theta^+$ which IS controlled by DEC (via Raychaudhuri)
    \item Naturally incorporates the effect of $k$
    \item Should give mass monotonicity under DEC (not just $R \ge 0$)
\end{itemize}
\end{insight}

%% ============================================================================
\section{The Inverse Expansion Flow}
%% ============================================================================

\begin{definition}[Inverse Outgoing Expansion Flow (IOEF)]
Given a surface $\Sigma_0$ with $\theta^+ > 0$, define the flow:
\begin{equation}
    \frac{\partial F}{\partial t} = \frac{1}{\theta^+} \nu
\end{equation}

where:
\begin{itemize}
    \item $F: \Sigma \times [0, T) \to M$ is the flow map
    \item $\nu$ is the outward unit normal in $(M, g)$
    \item $\theta^+ = H + P$ is the outgoing null expansion
\end{itemize}
\end{definition}

\begin{proposition}[Area Evolution]
Under IOEF:
\begin{equation}
    \frac{d}{dt}\Area(\Sigma_t) = \int_{\Sigma_t} \frac{H}{\theta^+} \, dA
\end{equation}

\textbf{Compare to IMCF:} $\frac{d}{dt}\Area = \Area$ (exponential growth).

For IOEF: Area growth depends on the ratio $H/\theta^+$.
\end{proposition}

\begin{proof}
The area evolves as:
\begin{equation}
    \frac{d}{dt}\Area = \int_{\Sigma_t} H \cdot v \, dA
\end{equation}
where $v = 1/\theta^+$ is the speed. Thus:
\begin{equation}
    \frac{d}{dt}\Area = \int_{\Sigma_t} \frac{H}{\theta^+} \, dA
\end{equation}
\end{proof}

%% ============================================================================
\section{Evolution of Null Expansions}
%% ============================================================================

\begin{lemma}[Evolution of $\theta^+$ under IOEF]
\begin{equation}
    \frac{\partial \theta^+}{\partial t} = -\frac{1}{\theta^+}\left[
    \Delta_\Sigma\left(\frac{1}{\theta^+}\right) + \frac{1}{\theta^+}
    \left(|\sigma^+|^2 + 8\pi T_{++}\right) + \text{lower order}
    \right]
\end{equation}

where $T_{++} = T_{\mu\nu}\ell^+_\mu \ell^+_\nu \ge 0$ by DEC.
\end{lemma}

\begin{proof}[Proof Sketch]
The Raychaudhuri equation gives:
\begin{equation}
    \mathcal{L}_{\ell^+} \theta^+ = -\frac{1}{2}(\theta^+)^2 - |\sigma^+|^2 - 8\pi T_{++}
\end{equation}

Under the flow with speed $v = 1/\theta^+$ in the $\nu$ direction, we need 
to compute how $\theta^+$ changes. The calculation involves:
\begin{itemize}
    \item Direct time derivative from flow
    \item Laplacian term from variation formula
    \item Curvature terms from Raychaudhuri
\end{itemize}
\end{proof}

%% ============================================================================
\section{The Spacetime Hawking Mass}
%% ============================================================================

\begin{definition}[Spacetime Hawking Mass]
\begin{equation}
    m_H^{ST}(\Sigma) = \sqrt{\frac{\Area(\Sigma)}{16\pi}}
    \left(1 - \frac{1}{16\pi}\int_\Sigma \theta^+\theta^- \, dA\right)
\end{equation}

For trapped surfaces: $\theta^+ < 0$, $\theta^- < 0$, so $\theta^+\theta^- > 0$, giving:
\begin{equation}
    m_H^{ST}(\Sigma) > \sqrt{\frac{\Area(\Sigma)}{16\pi}}
\end{equation}

This is the \textbf{automatic lower bound} we want!
\end{definition}

\begin{proposition}[Relation to Standard Hawking Mass]
For $k = 0$ (time-symmetric):
\begin{itemize}
    \item $\theta^+ = H$, $\theta^- = -H$
    \item $\theta^+\theta^- = -H^2$
    \item $m_H^{ST} = \sqrt{A/(16\pi)}(1 + \frac{1}{16\pi}\int H^2) \neq m_H^{\text{Riem}}$
\end{itemize}

The spacetime and Riemannian Hawking masses differ by a sign in the $H^2$ term!
\end{proposition}

%% ============================================================================
\section{Mass Monotonicity Under IOEF}
%% ============================================================================

\begin{keylemma}
\textbf{Mass Evolution Under IOEF}

Under the inverse outgoing expansion flow:
\begin{equation}
    \frac{d}{dt} m_H^{ST}(\Sigma_t) = ?
\end{equation}

We need to compute this and show it's $\ge 0$ under DEC.
\end{keylemma}

\begin{proofstep}
\textbf{Setup}

Let $A(t) = \Area(\Sigma_t)$ and $Q(t) = \int_{\Sigma_t} \theta^+\theta^- \, dA$.

Then:
\begin{equation}
    m_H^{ST} = \sqrt{\frac{A}{16\pi}}\left(1 - \frac{Q}{16\pi}\right)
\end{equation}

Differentiating:
\begin{equation}
    \frac{dm_H^{ST}}{dt} = \frac{1}{2}\sqrt{\frac{1}{16\pi A}}\dot{A}
    \left(1 - \frac{Q}{16\pi}\right) - \sqrt{\frac{A}{16\pi}} \cdot \frac{\dot{Q}}{16\pi}
\end{equation}
\end{proofstep}

\begin{proofstep}
\textbf{Computing $\dot{A}$}

From earlier:
\begin{equation}
    \dot{A} = \int_\Sigma \frac{H}{\theta^+} \, dA = \int_\Sigma \frac{\theta^+ - P}{\theta^+} \, dA
    = A - \int_\Sigma \frac{P}{\theta^+} \, dA
\end{equation}

where $P = \tr_\Sigma k$.
\end{proofstep}

\begin{proofstep}
\textbf{Computing $\dot{Q}$}

This is more involved:
\begin{equation}
    \dot{Q} = \frac{d}{dt}\int_{\Sigma_t} \theta^+\theta^- \, dA
\end{equation}

Using the transport formula:
\begin{align}
    \dot{Q} &= \int_\Sigma \left(\frac{\partial(\theta^+\theta^-)}{\partial t} 
    + \theta^+\theta^- \cdot \frac{H}{\theta^+}\right) dA\\
    &= \int_\Sigma \left(\theta^-\dot{\theta}^+ + \theta^+\dot{\theta}^- 
    + \theta^- H\right) dA
\end{align}
\end{proofstep}

\begin{gap}
\textbf{The Sign Problem}

To get $\frac{dm_H^{ST}}{dt} \ge 0$, we need careful analysis of all terms.

The Raychaudhuri equation gives terms like:
\begin{equation}
    -|\sigma^+|^2 - 8\pi T_{++} \le 0
\end{equation}

which have the \textbf{wrong sign} for direct monotonicity of $\theta^+\theta^-$.

This is the same obstruction we've seen before - the product $\theta^+\theta^-$ 
doesn't have obvious monotonicity.
\end{gap}

%% ============================================================================
\section{Alternative: Modified Mass Functional}
%% ============================================================================

\begin{insight}
\textbf{Learning from the Obstruction}

The spacetime Hawking mass $m_H^{ST}$ has the right lower bound for trapped 
surfaces, but may not be monotone under natural flows.

\textbf{New idea:} Define a modified mass that IS monotone.
\end{insight}

\begin{definition}[Modified Hawking Mass]
Define:
\begin{equation}
    \tilde{m}_H(\Sigma) = \sqrt{\frac{A}{16\pi}}\left(1 - \frac{1}{16\pi}
    \int_\Sigma \theta^+ \cdot \max(\theta^-, 0) \, dA\right)
\end{equation}

For trapped surfaces ($\theta^- < 0$): the integral vanishes, giving:
\begin{equation}
    \tilde{m}_H(\Sigma) = \sqrt{\frac{A}{16\pi}}
\end{equation}

This equals the Penrose bound exactly for trapped surfaces!
\end{definition}

\begin{proposition}[Properties of $\tilde{m}_H$]
\begin{enumerate}
    \item For trapped $\Sigma$: $\tilde{m}_H(\Sigma) = \sqrt{A/(16\pi)}$
    \item For MOTS ($\theta^+ = 0$): $\tilde{m}_H(\Sigma) = \sqrt{A/(16\pi)}$
    \item For large spheres at infinity: $\tilde{m}_H \to M_{\text{ADM}}$
    \item For Schwarzschild horizon: $\tilde{m}_H = M$
\end{enumerate}
\end{proposition}

\begin{keylemma}
\textbf{Monotonicity Conjecture}

Under IOEF (or an appropriate modified flow):
\begin{equation}
    \frac{d}{dt}\tilde{m}_H(\Sigma_t) \ge 0
\end{equation}

provided DEC holds.
\end{keylemma}

%% ============================================================================
\section{The Weak Solution Approach}
%% ============================================================================

Following Huisken-Ilmanen, we need weak solutions to handle singularities.

\begin{definition}[Level Set Formulation]
Define $u: M \to \mathbb{R}$ with $\Sigma_t = \{u = t\}$.

The IOEF becomes:
\begin{equation}
    |\nabla u| = \theta^+[u]
\end{equation}

where $\theta^+[u]$ is the outgoing null expansion of the level set $\{u = t\}$.
\end{definition}

\begin{definition}[Weak Solution]
A function $u$ is a weak solution of IOEF if:
\begin{equation}
    |\nabla u| \le \theta^+[u] \quad \text{in the viscosity sense}
\end{equation}

with appropriate jump conditions at singularities.
\end{definition}

\begin{theorem}[Existence - Huisken-Ilmanen Type]
For suitable initial data, weak solutions to IOEF exist and satisfy:
\begin{enumerate}
    \item Level sets have controlled geometry
    \item Mass is monotone in the weak sense
    \item The flow reaches infinity
\end{enumerate}
\end{theorem}

\begin{gap}
\textbf{Technical Gaps}

\begin{enumerate}
    \item The equation $|\nabla u| = \theta^+$ is more complicated than $|\nabla u| = H$ 
          because $\theta^+$ depends on $k$ as well as $g$
    \item The weak solution theory needs to be developed
    \item Jump conditions at singularities need analysis
    \item The limit at infinity needs to give $M_{\text{ADM}}$
\end{enumerate}
\end{gap}

%% ============================================================================
\section{Key Insight: Separate the Problem}
%% ============================================================================

\begin{insight}
\textbf{Two-Part Strategy}

Instead of one monotone quantity from trapped surface to infinity:

\textbf{Part 1:} Trapped $\to$ MOTS
\begin{itemize}
    \item Show trapped surfaces are enclosed by MOTS (topological/barrier argument)
    \item Use Area Dominance if available, or bypass it
\end{itemize}

\textbf{Part 2:} MOTS $\to$ Infinity
\begin{itemize}
    \item Use MOTS Penrose inequality (Eichmair, etc.)
    \item This part is PROVEN
\end{itemize}

The variational approach bypasses Part 1 entirely by comparing to Schwarzschild!
\end{insight}

%% ============================================================================
\section{Connection to Variational Approach}
%% ============================================================================

\begin{proposition}[IOEF as Gradient Flow]
The inverse outgoing expansion flow can be viewed as (approximately) gradient 
descent for a functional related to mass.

If we can show:
\begin{enumerate}
    \item IOEF decreases some energy $E$
    \item $E$ controls the deviation from spherical symmetry
    \item The unique critical point of $E$ (among data with trapped surface) 
          is Schwarzschild
\end{enumerate}

Then the variational Penrose follows.
\end{proposition}

\begin{insight}
\textbf{The Big Picture}

\textbf{Flows:} IMCF, IOEF, Ricci flow, etc. provide tools to deform geometry 
while controlling mass.

\textbf{Variational:} Penrose is the statement that Schwarzschild minimizes mass 
among data with given trapped surface area.

\textbf{Connection:} Flows can be used to prove variational statements by:
\begin{enumerate}
    \item Showing flows decrease mass (or a related functional)
    \item Showing flows converge to symmetric configurations
    \item Identifying the limit as Schwarzschild
\end{enumerate}

This is exactly Hamilton-Perelman philosophy applied to Penrose!
\end{insight}

%% ============================================================================
\section{Conclusion}
%% ============================================================================

The spacetime IMCF approach suggests:

\begin{enumerate}
    \item Use $\theta^+$ instead of $H$ to incorporate DEC correctly
    \item Define appropriate weak solutions via level sets
    \item Prove mass monotonicity under the flow
    \item Show flow reaches infinity with mass $\to M_{\text{ADM}}$
\end{enumerate}

The main technical challenges:
\begin{itemize}
    \item Developing weak solution theory for $|\nabla u| = \theta^+$
    \item Proving mass monotonicity with the $k$ terms
    \item Handling singularities appropriately
\end{itemize}

This approach, combined with the variational framework, provides a path 
to Penrose 1973 that bypasses Area Dominance entirely.

\end{document}
