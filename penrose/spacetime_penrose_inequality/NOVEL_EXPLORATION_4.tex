% =========================================================================
%     NOVEL EXPLORATION 4: OPTIMAL TRANSPORT AND MASS
%
%     Using Kantorovich theory in a new way
%
%     Author: Da Xu
%     Date: December 2025
% =========================================================================

\documentclass[12pt]{article}
\usepackage{amsmath,amsthm,amssymb}
\usepackage{mathrsfs}
\usepackage{tcolorbox}

\theoremstyle{plain}
\newtheorem{theorem}{Theorem}[section]
\newtheorem{lemma}[theorem]{Lemma}
\newtheorem{proposition}[theorem]{Proposition}
\newtheorem{corollary}[theorem]{Corollary}
\newtheorem{conjecture}[theorem]{Conjecture}

\theoremstyle{definition}
\newtheorem{definition}[theorem]{Definition}
\newtheorem{remark}[theorem]{Remark}
\newtheorem{observation}[theorem]{Key Observation}

\newcommand{\ADM}{\mathrm{ADM}}
\newcommand{\tr}{\mathrm{tr}}
\newcommand{\Div}{\mathrm{div}}
\newcommand{\Area}{\mathrm{Area}}

\title{\textbf{Novel Exploration 4: Optimal Transport Perspective}}
\author{Da Xu}
\date{December 2025}

\begin{document}
\maketitle

\section{The Idea}

\textbf{Optimal transport} has revolutionized analysis and geometry:
\begin{itemize}
    \item Wasserstein distances
    \item Ricci curvature lower bounds
    \item Stability of geometric inequalities
\end{itemize}

\textbf{Question:} Can optimal transport tools apply to Penrose inequality?

\section{Mass as a Measure}

\subsection{The Energy Density}

The DEC provides:
\[
    \mu = \frac{R + (\tr k)^2 - |k|^2}{16\pi} \geq |J|
\]

The ADM mass:
\[
    M_{\ADM} = \frac{1}{16\pi} \lim_{r\to\infty} \int_{S_r} (g_{ij,i} - g_{ii,j}) \nu^j \, dA
\]

\subsection{Mass as Total Energy}

Under certain decay conditions:
\[
    M_{\ADM} = \int_M \mu \, dV + \text{boundary terms}
\]

Actually, the precise relation involves the constraint equations.

\subsection{The Mass Measure}

\begin{definition}
The \textbf{mass measure} on $(M, g)$ is:
\[
    d\mathfrak{m} = \mu \, dV
\]
where $\mu \geq 0$ by DEC and $dV$ is the volume form.
\end{definition}

Total mass (in some sense): $\mathfrak{m}(M) = \int_M \mu \, dV$.

\section{The Transport Problem}

\subsection{Formulation}

Consider two measures:
\begin{enumerate}
    \item \textbf{Source:} Mass measure $\mathfrak{m}$ on $M$
    \item \textbf{Target:} Dirac mass $M_{\ADM} \cdot \delta_\infty$ at infinity
\end{enumerate}

The ADM mass ``collects'' all the distributed mass at infinity.

\subsection{Optimal Transport Cost}

\begin{definition}
The \textbf{transport cost} from mass to infinity is:
\[
    \mathcal{T}(\mathfrak{m}) = \inf_\gamma \int_{M \times \{\infty\}} c(x, \infty) \, d\gamma(x, \infty)
\]
where $c$ is a cost function and $\gamma$ is a transport plan.
\end{definition}

If $c(x, \infty) = d(x, \infty)$ (distance to infinity):
\[
    \mathcal{T}(\mathfrak{m}) = \int_M d(x, \infty) \, \mu \, dV
\]

But $d(x, \infty) = \infty$... so this is not immediately useful.

\subsection{Modified Cost}

Use a ``gravitational potential'' cost:
\[
    c(x, y) = -\frac{1}{|x - y|} \quad \text{(in flat space)}
\]

In curved space: use Green's function.

\section{The Capacity Connection}

\subsection{Capacity via Transport}

The capacity of a compact set $K$ relates to optimal transport:
\[
    \text{Cap}(K) = \sup_{\mu: \text{supp}(\mu) \subset K} \mathfrak{m}(K)
\]
subject to the potential being bounded.

Actually, the capacity has the representation:
\[
    \text{Cap}(K) = \inf\left\{\int |\nabla u|^2 : u = 1 \text{ on } K, u \to 0 \text{ at } \infty\right\}
\]

\subsection{Benamou-Brenier Interpretation}

The Wasserstein-2 distance has a dynamical formulation (Benamou-Brenier):
\[
    W_2^2(\mu, \nu) = \inf \int_0^1 \int |\dot{\gamma}_t|^2 \, d\rho_t \, dt
\]
over paths of measures from $\mu$ to $\nu$.

\textbf{Idea:} The capacity might relate to a transport problem!

\section{A New Functional}

\subsection{The Wasserstein-Capacity Functional}

\begin{definition}
For a surface $\Sigma$, define:
\[
    \mathcal{WC}(\Sigma) = W_2^2(\sigma_\Sigma, \delta_\infty)
\]
where $\sigma_\Sigma$ is the uniform measure on $\Sigma$ (normalized to mass 1).
\end{definition}

In asymptotically flat space:
\[
    \mathcal{WC}(\Sigma) \approx \int_\Sigma d(x, \infty)^2 \, d\sigma
\]

For large $r$: $d(x, \infty) \approx$ const, so this doesn't distinguish well.

\subsection{Relative Transport}

\begin{definition}
The \textbf{relative transport cost} from $\Sigma_0$ to $\Sigma_\infty$ (sphere at infinity) is:
\[
    \mathcal{RC}(\Sigma_0) = W_2^2(\sigma_{\Sigma_0}, \sigma_{\Sigma_\infty})
\]
\end{definition}

This measures how much ``transport'' is needed to move from $\Sigma_0$ to infinity.

\section{Optimal Transport on Surfaces}

\subsection{A Different Perspective}

Instead of transport in spacetime, consider transport ON surfaces.

The Hawking mass involves $\int_\Sigma H^2 \, dA$.

Can we interpret this via transport?

\subsection{The Mean Curvature and Displacement}

Mean curvature $H$ measures how the surface ``wants to move'':
\begin{itemize}
    \item $H > 0$: locally convex, wants to shrink
    \item $H < 0$: locally concave, wants to expand
\end{itemize}

The squared integral:
\[
    \int_\Sigma H^2 \, dA = \int_\Sigma |\text{displacement velocity}|^2 \, dA
\]

This is like a kinetic energy!

\subsection{The Displacement Interpretation}

\begin{proposition}[Informal]
Under mean curvature flow:
\[
    \partial_t x = -H\nu
\]
the ``kinetic energy'' is:
\[
    K(t) = \int_{\Sigma_t} |\partial_t x|^2 \, dA = \int_{\Sigma_t} H^2 \, dA
\]
\end{proposition}

The Hawking mass contains: $1 - \frac{1}{16\pi}\int H^2 \, dA$.

\section{A Speculative Connection}

\subsection{Transport Cost and Hawking Mass}

\begin{conjecture}[Very Speculative]
The Hawking mass has an optimal transport interpretation:
\[
    m_H(\Sigma) = \sqrt{\frac{\Area(\Sigma)}{16\pi}}\left(1 - \frac{\mathcal{T}(\Sigma)}{C}\right)
\]
where $\mathcal{T}(\Sigma)$ is some transport cost.
\end{conjecture}

\subsection{Why This Might Work}

In Riemannian geometry:
\begin{itemize}
    \item Ricci $\geq 0$ implies transport contraction
    \item Positive mass theorem needs ``positive Ricci in some sense''
    \item Transport perspective might give monotonicity
\end{itemize}

\section{A Concrete New Direction}

\subsection{The Mass Transport Bound}

\begin{conjecture}[Mass Transport Bound]
Let $\Sigma_0$ be trapped and $\Sigma_\infty$ be a large sphere. Then:
\[
    M_{\ADM} \geq \frac{W_2(\sigma_{\Sigma_0}, \sigma_{\Sigma_\infty})}{C \cdot r_\infty}
\]
for some universal constant $C$.
\end{conjecture}

\textbf{Intuition:} More mass means more ``distortion'' of the transport map.

\subsection{Comparison with Penrose}

Penrose: $M \geq \sqrt{A/(16\pi)}$.

If $W_2 \sim \sqrt{A}$ (dimensional analysis), the bound becomes:
\[
    M \geq \frac{\sqrt{A}}{C \cdot r_\infty}
\]

For $r_\infty \to \infty$, this becomes trivial!

Need a RELATIVE measure that doesn't go to infinity.

\section{The Capacity-Transport Duality}

\subsection{Observation}

In Euclidean space:
\[
    \text{Cap}(\bar{B}_r) = 4\pi r
\]

The Wasserstein distance:
\[
    W_2(\sigma_{S_r}, \delta_0) = r
\]

So: $\text{Cap} = 4\pi \cdot W_2$!

\subsection{Generalization}

\begin{conjecture}
In asymptotically flat $(M, g)$:
\[
    \text{Cap}(\Sigma) \approx 4\pi \cdot W_2(\sigma_\Sigma, \delta_\text{center})
\]
where ``center'' is some appropriate reference point.
\end{conjecture}

If true, capacity bounds become transport bounds!

\section{The Brenier Map Approach}

\subsection{The Optimal Map}

For optimal transport between nice measures, there exists an \textbf{optimal map} $T$:
\[
    W_2^2(\mu, \nu) = \int |x - T(x)|^2 \, d\mu(x)
\]

Brenier's theorem: $T = \nabla \phi$ for some convex $\phi$.

\subsection{Application to Surfaces}

From trapped $\Sigma_0$ to MOTS $\Sigma^*$:
\begin{itemize}
    \item Let $\sigma_0, \sigma^*$ be uniform measures
    \item Optimal map $T: \Sigma_0 \to \Sigma^*$
    \item $T = \nabla \phi$ for some function $\phi$
\end{itemize}

The transport cost:
\[
    W_2^2(\sigma_0, \sigma^*) = \int_{\Sigma_0} |x - T(x)|^2 \, d\sigma_0
\]

\subsection{Relation to Area}

The Jacobian of $T$ relates to area change:
\[
    \det(DT) = \frac{d\sigma^*}{d\sigma_0}
\]

If $\Area(\Sigma^*) < \Area(\Sigma_0)$, then $\det(DT) < 1$ on average.

This means $T$ contracts!

\begin{observation}
The area decrease $\Area(\Sigma^*) < \Area(\Sigma_0)$ corresponds to the optimal map being contractive on average.
\end{observation}

\section{A Promising Direction}

\subsection{Transport and Trapping}

\begin{conjecture}[Transport-Trapping]
The trapped condition creates a ``transport barrier'':
\[
    W_2(\sigma_{\Sigma_0}, \sigma_{\text{infinity}}) \geq C \cdot \sqrt{\Area(\Sigma_0)}
\]
\end{conjecture}

\textbf{Intuition:} Being trapped means you're ``far from infinity'' in transport sense.

\subsection{Combined with Mass Bound}

If we could prove:
\[
    M_{\ADM} \geq \frac{W_2(\sigma_{\Sigma_0}, \sigma_\infty)}{16\pi r_\infty}
\]
and
\[
    W_2(\sigma_{\Sigma_0}, \sigma_\infty) \geq r_\infty \cdot \sqrt{\Area(\Sigma_0)/(16\pi)}
\]
then Penrose follows!

\section{Conclusion}

\begin{tcolorbox}[colback=blue!10, colframe=blue!75!black]
\textbf{Key Insights:}

\begin{enumerate}
    \item Mass can be viewed as a measure; transport theory applies
    \item Hawking mass might have transport interpretation via $\int H^2$
    \item Capacity might equal $4\pi \times$ Wasserstein distance
    \item Area decrease corresponds to contractive transport map
\end{enumerate}

\textbf{Main Challenge:}

Making these ideas QUANTITATIVE with the right scaling.

\textbf{Most Promising:}

The \textbf{transport-trapping conjecture}: being trapped creates lower bound on transport cost, which might translate to mass bound.
\end{tcolorbox}

\end{document}
