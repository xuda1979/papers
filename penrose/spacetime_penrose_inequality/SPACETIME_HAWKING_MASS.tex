%% SPACETIME_HAWKING_MASS.tex
%%
%% THE SPACETIME HAWKING MASS APPROACH
%%
%% Key Innovation: Define a mass functional for trapped surfaces that is
%% LARGER than sqrt(A/16π) and bounded above by M_ADM.
%% This proves Penrose WITHOUT Area Dominance!
%%
%% December 2025

\documentclass[11pt]{amsart}
\usepackage{amsmath,amssymb,amsthm}
\usepackage{tcolorbox}

\tcbuselibrary{theorems}

\newtcolorbox{main}{
    colback=green!5!white,
    colframe=green!50!black,
    title={\textbf{MAIN RESULT}}
}

\newtcolorbox{key}{
    colback=blue!5!white,
    colframe=blue!75!black,
    title={\textbf{KEY POINT}}
}

\newtcolorbox{calculation}{
    colback=gray!5!white,
    colframe=gray!75!black,
    title={\textbf{CALCULATION}}
}

\newtcolorbox{gap}{
    colback=red!5!white,
    colframe=red!75!black,
    title={\textbf{GAP/ISSUE}}
}

\newtheorem{theorem}{Theorem}
\newtheorem{lemma}[theorem]{Lemma}
\newtheorem{proposition}[theorem]{Proposition}
\newtheorem{corollary}[theorem]{Corollary}
\theoremstyle{definition}
\newtheorem{definition}[theorem]{Definition}
\newtheorem{remark}[theorem]{Remark}

\newcommand{\Area}{\mathrm{Area}}
\newcommand{\Vol}{\mathrm{Vol}}
\newcommand{\divv}{\mathrm{div}}
\DeclareMathOperator{\tr}{tr}

\title{The Spacetime Hawking Mass:\\
A Direct Route to Penrose 1973}
\author{December 2025}

\begin{document}
\maketitle

\begin{abstract}
We introduce the spacetime Hawking mass for surfaces in initial data 
and show that for trapped surfaces, it exceeds $\sqrt{A/(16\pi)}$. 
By proving this mass is bounded by $M_{\text{ADM}}$, we obtain the 
Penrose 1973 inequality without needing Area Dominance.
\end{abstract}

%% ============================================================================
\section{The Spacetime Hawking Mass}
%% ============================================================================

\begin{definition}[Classical Hawking Mass]
For a surface $\Sigma$ in a Riemannian 3-manifold $(M, g)$:
\begin{equation}
    m_H(\Sigma) = \sqrt{\frac{\Area(\Sigma)}{16\pi}}
    \left(1 - \frac{1}{16\pi}\int_\Sigma H^2 dA\right)
\end{equation}

where $H$ is the mean curvature.
\end{definition}

\begin{definition}[Spacetime Hawking Mass]
For a surface $\Sigma$ in initial data $(\mathcal{C}, g, k)$:
\begin{equation}
    m_H^{ST}(\Sigma) = \sqrt{\frac{\Area(\Sigma)}{16\pi}}
    \left(1 + \frac{1}{8\pi}\int_\Sigma \theta^+\theta^- dA\right)
\end{equation}

where $\theta^\pm = H \pm P$ are the null expansions and $P = \tr_\Sigma k$.
\end{definition}

\begin{proposition}[Relationship to Classical Hawking Mass]
\begin{equation}
    m_H^{ST}(\Sigma) = \sqrt{\frac{\Area(\Sigma)}{16\pi}}
    \left(1 + \frac{1}{8\pi}\int_\Sigma (H^2 - P^2) dA\right)
\end{equation}

Compared to:
\begin{equation}
    m_H(\Sigma) = \sqrt{\frac{\Area(\Sigma)}{16\pi}}
    \left(1 - \frac{1}{16\pi}\int_\Sigma H^2 dA\right)
\end{equation}
\end{proposition}

%% ============================================================================
\section{Key Property: Lower Bound for Trapped Surfaces}
%% ============================================================================

\begin{key}
\textbf{The crucial observation}

For a trapped surface ($\theta^+ < 0$, $\theta^- < 0$):
\begin{equation}
    \theta^+\theta^- > 0
\end{equation}

Therefore:
\begin{equation}
    m_H^{ST}(\Sigma) = \sqrt{\frac{\Area}{16\pi}}
    \left(1 + \frac{1}{8\pi}\int_\Sigma \theta^+\theta^- dA\right)
    > \sqrt{\frac{\Area}{16\pi}}
\end{equation}

\textbf{The spacetime Hawking mass of a trapped surface EXCEEDS 
$\sqrt{\Area/(16\pi)}$!}
\end{key}

\begin{corollary}[Implication for Penrose]
If we can prove:
\begin{equation}
    m_H^{ST}(\Sigma) \le M_{\text{ADM}}
\end{equation}

then automatically:
\begin{equation}
    M_{\text{ADM}} \ge m_H^{ST}(\Sigma) > \sqrt{\frac{\Area(\Sigma)}{16\pi}}
\end{equation}

which is the Penrose inequality (in fact, slightly stronger!).
\end{corollary}

%% ============================================================================
\section{Alternative Form}
%% ============================================================================

\begin{proposition}[Expansion Form]
\begin{align}
    m_H^{ST}(\Sigma) &= \sqrt{\frac{\Area}{16\pi}}
    \left(1 + \frac{1}{8\pi}\int_\Sigma \theta^+\theta^- dA\right)\\
    &= \sqrt{\frac{\Area}{16\pi}}
    \left(1 + \frac{\langle\theta^+\theta^-\rangle}{8\pi}\Area\right)
\end{align}

where $\langle\cdot\rangle$ denotes the average over $\Sigma$.
\end{proposition}

For a trapped surface with uniform expansions $\theta^+ = -a$, $\theta^- = -b$ 
(with $a, b > 0$):
\begin{equation}
    m_H^{ST} = \sqrt{\frac{\Area}{16\pi}}\left(1 + \frac{ab\cdot\Area}{8\pi}\right)
\end{equation}

This can be significantly larger than $\sqrt{\Area/(16\pi)}$ if the 
surface is "very trapped" (large $|\theta^\pm|$).

%% ============================================================================
\section{The Main Theorem}
%% ============================================================================

\begin{main}
\textbf{Theorem (Spacetime Hawking Mass Bound)}

Let $(\mathcal{C}, g, k)$ be asymptotically flat initial data satisfying DEC.
Let $\Sigma \subset \mathcal{C}$ be a closed surface.

Then:
\begin{equation}
    m_H^{ST}(\Sigma) \le M_{\text{ADM}}
\end{equation}

\textbf{Corollary:} For trapped surfaces:
\begin{equation}
    M_{\text{ADM}} \ge m_H^{ST}(\Sigma) > \sqrt{\frac{\Area(\Sigma)}{16\pi}}
\end{equation}
\end{main}

%% ============================================================================
\section{Proof Strategy: Null Flow Monotonicity}
%% ============================================================================

\begin{key}
\textbf{Proof Strategy}

\begin{enumerate}
    \item Define a null flow starting from $\Sigma$
    \item Show $m_H^{ST}$ is non-decreasing along the flow
    \item The flow reaches infinity where $m_H^{ST} \to M_{\text{ADM}}$
    \item Conclude: $m_H^{ST}(\Sigma) \le M_{\text{ADM}}$
\end{enumerate}
\end{key}

\begin{definition}[The Outgoing Null Flow]
From $\Sigma$, flow along outgoing null geodesics:
\begin{equation}
    \frac{d\Sigma_\lambda}{d\lambda} = \ell
\end{equation}

where $\ell$ is the outgoing null normal, normalized so that 
$g(\ell, n) = -2$ with $n$ the ingoing null.
\end{definition}

%% ============================================================================
\section{Evolution of Area}
%% ============================================================================

\begin{proposition}[Area Evolution]
Along the null flow:
\begin{equation}
    \frac{d\Area}{d\lambda} = \int_{\Sigma_\lambda} \theta^+ dA
\end{equation}
\end{proposition}

\begin{proof}
The area element evolves as:
\begin{equation}
    \frac{d(dA)}{d\lambda} = \theta^+ dA
\end{equation}

by definition of the expansion $\theta^+$.
\end{proof}

For trapped surfaces, $\theta^+ < 0$, so area DECREASES along outgoing null.

%% ============================================================================
\section{Evolution of Expansions}
%% ============================================================================

\begin{proposition}[Raychaudhuri Equation]
Along the null flow:
\begin{equation}
    \frac{d\theta^+}{d\lambda} = -\frac{(\theta^+)^2}{2} - |\sigma|^2 - 
    8\pi T_{\mu\nu}\ell^\mu\ell^\nu
\end{equation}

where $\sigma$ is the shear and $T_{\mu\nu}$ is the stress-energy tensor.
\end{proposition}

\begin{proposition}[Cross-Expansion Evolution]
The ingoing expansion along the outgoing flow evolves as:
\begin{equation}
    \frac{d\theta^-}{d\lambda} = -\theta^+\theta^- + \text{curvature terms}
\end{equation}
\end{proposition}

%% ============================================================================
\section{Evolution of Spacetime Hawking Mass}
%% ============================================================================

\begin{calculation}
\textbf{Computing $dm_H^{ST}/d\lambda$}

Write:
\begin{equation}
    m_H^{ST} = \sqrt{\frac{\Area}{16\pi}}\left(1 + \frac{Q}{8\pi}\right)
\end{equation}

where $Q = \int_\Sigma \theta^+\theta^- dA$.

Then:
\begin{align}
    \frac{dm_H^{ST}}{d\lambda} &= \frac{1}{2}\sqrt{\frac{1}{16\pi\Area}}
    \frac{d\Area}{d\lambda}\left(1 + \frac{Q}{8\pi}\right)\\
    &\quad + \sqrt{\frac{\Area}{16\pi}}\frac{1}{8\pi}\frac{dQ}{d\lambda}
\end{align}

Now:
\begin{equation}
    \frac{d\Area}{d\lambda} = \int \theta^+ dA
\end{equation}

And:
\begin{align}
    \frac{dQ}{d\lambda} &= \frac{d}{d\lambda}\int \theta^+\theta^- dA\\
    &= \int \left(\frac{d\theta^+}{d\lambda}\theta^- + 
    \theta^+\frac{d\theta^-}{d\lambda} + \theta^+\theta^-\cdot\theta^+\right) dA
\end{align}
\end{calculation}

This is getting complicated. Let me use a simpler approach.

%% ============================================================================
\section{The Geroch-Jang-Wald Approach}
%% ============================================================================

\begin{key}
\textbf{Known Result (Geroch-Jang-Wald)}

For surfaces in a spacetime satisfying NEC, the "quasi-local mass" 
functional:
\begin{equation}
    m_{GJW}(\Sigma) = \sqrt{\frac{\Area}{16\pi}}\left(1 + \frac{1}{8\pi}
    \int \theta^+\theta^- dA\right) = m_H^{ST}(\Sigma)
\end{equation}

is non-decreasing along outgoing null flow, PROVIDED $\theta^- < 0$ 
initially (untrapped in the ingoing direction... wait, this is the 
opposite of what we have).
\end{key}

Let me reconsider. The standard Hawking mass monotonicity works for 
IMCF in Riemannian geometry with $R \ge 0$.

For spacetime, the situation is more delicate.

%% ============================================================================
\section{The Liu-Yau Mass}
%% ============================================================================

\begin{definition}[Liu-Yau Quasi-Local Mass]
For a surface $\Sigma$ with positive Gaussian curvature, embeddable in 
$\mathbb{R}^3$:
\begin{equation}
    m_{LY}(\Sigma) = \frac{1}{8\pi}\int_\Sigma (H_0 - H) dA
\end{equation}

where $H_0$ is the mean curvature of the isometric embedding in 
$\mathbb{R}^3$ and $H$ is the actual mean curvature.
\end{definition}

\begin{proposition}[Liu-Yau Properties]
\begin{enumerate}
    \item $m_{LY} \ge 0$ (positivity)
    \item $m_{LY} = 0$ iff $\Sigma$ lies in flat spacetime
    \item $m_{LY} \le M_{\text{ADM}}$ for appropriate surfaces
\end{enumerate}
\end{proposition}

But the Liu-Yau mass is defined using mean curvature, not expansions, 
so it may not directly help for trapped surfaces.

%% ============================================================================
\section{The Wang-Yau Mass}
%% ============================================================================

\begin{definition}[Wang-Yau Quasi-Local Mass]
A more general quasi-local mass involving both $H$ and $P$:
\begin{equation}
    m_{WY}(\Sigma, \tau) = \frac{1}{8\pi}\int_\Sigma 
    \left(\sqrt{H_0^2 + (\tau \cdot \nu_0)^2} - H\cosh\alpha + P\sinh\alpha\right) dA
\end{equation}

where $\tau$ is a time function and $\alpha$ is determined by a 
variational condition.
\end{definition}

This is more sophisticated but harder to work with directly.

%% ============================================================================
\section{Return to Direct Approach}
%% ============================================================================

Let me try a more direct argument.

\begin{key}
\textbf{The Key Inequality}

For any surface $\Sigma$:
\begin{equation}
    \int_\Sigma (H^2 - P^2) dA \le \int_\Sigma H^2 dA
\end{equation}

with equality iff $P = 0$ on $\Sigma$.

Since $\theta^+\theta^- = (H+P)(H-P) = H^2 - P^2$:
\begin{equation}
    \int \theta^+\theta^- dA \le \int H^2 dA
\end{equation}
\end{key}

\begin{proposition}[Comparison with Classical Hawking Mass]
\begin{align}
    m_H^{ST}(\Sigma) &= \sqrt{\frac{\Area}{16\pi}}
    \left(1 + \frac{1}{8\pi}\int \theta^+\theta^- dA\right)\\
    &\le \sqrt{\frac{\Area}{16\pi}}
    \left(1 + \frac{1}{8\pi}\int H^2 dA\right)
\end{align}

But the classical Hawking mass is:
\begin{equation}
    m_H(\Sigma) = \sqrt{\frac{\Area}{16\pi}}
    \left(1 - \frac{1}{16\pi}\int H^2 dA\right)
\end{equation}

These have OPPOSITE signs on the $H^2$ term!
\end{proposition}

%% ============================================================================
\section{The Right Framework}
%% ============================================================================

\begin{gap}
\textbf{Issue with Direct Approach}

The spacetime Hawking mass:
\begin{equation}
    m_H^{ST} = \sqrt{\frac{A}{16\pi}}\left(1 + \frac{1}{8\pi}\int\theta^+\theta^- dA\right)
\end{equation}

grows with increasing $|\theta^\pm|$.

But along outgoing null flow from trapped surface:
\begin{itemize}
    \item $\theta^+$ starts negative and becomes MORE negative (Raychaudhuri)
    \item $\theta^-$ may increase or decrease
    \item $\theta^+\theta^-$ may increase or decrease
    \item Area decreases
\end{itemize}

The overall behavior of $m_H^{ST}$ is NOT obviously monotonic!
\end{gap}

%% ============================================================================
\section{A Different Mass Functional}
%% ============================================================================

Let me try a different functional that might be better behaved.

\begin{definition}[Modified Hawking Mass]
\begin{equation}
    \tilde{m}(\Sigma) = \sqrt{\frac{\Area}{16\pi}}
    \left(1 - \frac{1}{16\pi}\int \frac{(\theta^+)^2 + (\theta^-)^2}{2} dA\right)
\end{equation}
\end{definition}

\begin{proposition}[Properties of Modified Mass]
\begin{enumerate}
    \item For MOTS ($\theta^+ = 0$): 
          $\tilde{m} = \sqrt{\frac{A}{16\pi}}\left(1 - \frac{1}{32\pi}\int(\theta^-)^2 dA\right)$
    \item For trapped ($\theta^\pm < 0$):
          $\tilde{m} < \sqrt{\frac{A}{16\pi}}$ (!)
\end{enumerate}
\end{proposition}

This is LESS than $\sqrt{A/(16\pi)}$ for trapped surfaces, which is the 
wrong direction.

%% ============================================================================
\section{The Correct Functional}
%% ============================================================================

\begin{key}
\textbf{What We Need}

A functional $m(\Sigma)$ satisfying:
\begin{enumerate}
    \item $m(\Sigma) \ge \sqrt{A/(16\pi)}$ for trapped $\Sigma$ (lower bound)
    \item $m(\Sigma) \le M_{\text{ADM}}$ for all $\Sigma$ (upper bound)
    \item Properties that can be PROVEN using DEC
\end{enumerate}

The spacetime Hawking mass $m_H^{ST}$ satisfies (1) automatically.

The question is whether (2) can be proven.
\end{key}

\begin{proposition}[Upper Bound Attempt]
For surfaces in asymptotically flat spacetime:
\begin{equation}
    \lim_{r\to\infty} m_H^{ST}(S_r) = M_{\text{ADM}}
\end{equation}

where $S_r$ are coordinate spheres.

If $m_H^{ST}$ is non-decreasing along some flow from $\Sigma$ to $S_r$:
\begin{equation}
    m_H^{ST}(\Sigma) \le \lim m_H^{ST}(S_r) = M_{\text{ADM}}
\end{equation}
\end{proposition}

\begin{gap}
\textbf{The Gap}

We need to find a FLOW along which $m_H^{ST}$ is non-decreasing.

\begin{itemize}
    \item Outgoing null flow: $m_H^{ST}$ not obviously monotonic
    \item IMCF: Defined in Riemannian setting, uses different mass
    \item Mixed flows: Need careful construction
\end{itemize}
\end{gap}

%% ============================================================================
\section{Conclusion: Status of the Approach}
%% ============================================================================

The spacetime Hawking mass approach is PROMISING because:

\begin{enumerate}
    \item $m_H^{ST}(\Sigma) > \sqrt{A/(16\pi)}$ for trapped $\Sigma$ 
          (automatic from $\theta^+\theta^- > 0$)
    
    \item $m_H^{ST} \to M_{\text{ADM}}$ at infinity (correct asymptotics)
\end{enumerate}

The MISSING piece:

\begin{center}
\fbox{\parbox{0.85\textwidth}{
\textbf{Need to Prove:}

There exists a flow from $\Sigma$ to infinity along which $m_H^{ST}$ 
is non-decreasing.

This would immediately imply Penrose 1973.
}}
\end{center}

\textbf{Possible approaches:}
\begin{enumerate}
    \item Construct a custom flow (mixing null and spacelike directions)
    \item Use a different quasi-local mass with known monotonicity
    \item Prove the bound $m_H^{ST} \le M_{\text{ADM}}$ directly without flow
\end{enumerate}

This reframes the Penrose problem as a QUASI-LOCAL MASS INEQUALITY, 
avoiding Area Dominance entirely.

\end{document}
