% ================================================================
% PAPER ROADMAP ANALYSIS: Spacetime Penrose Inequality
% Generated: December 2025
% ================================================================

\documentclass[11pt]{article}
\usepackage{amsmath,amssymb,amsthm}
\usepackage{geometry}
\usepackage{booktabs}
\usepackage{xcolor}
\usepackage{tcolorbox}
\usepackage{enumitem}

\geometry{margin=2.5cm}

\newcommand{\status}[1]{\textcolor{#1}{\textbf{#1}}}
\newcommand{\proven}{\textcolor{green!60!black}{\textbf{PROVEN}}}
\newcommand{\conditional}{\textcolor{orange!80!black}{\textbf{CONDITIONAL}}}
\newcommand{\open}{\textcolor{red!70!black}{\textbf{OPEN}}}

\title{\textbf{Paper Roadmap Analysis}\\
\large Spacetime Penrose Inequality: Logical Structure and Gap Analysis}
\date{December 2025}

\begin{document}
\maketitle

\tableofcontents
\newpage

\section{Executive Summary}

The paper \texttt{paper.tex} (28,054 lines) presents multiple approaches to the Spacetime Penrose Inequality. The logical structure follows two parallel tracks:

\begin{tcolorbox}[colback=blue!5!white, colframe=blue!60!black, title=\textbf{Track A: Initial Data Approach (Jang + AMO)}]
\textbf{Status: RIGOROUS for outermost MOTS}

Logical chain:
\[
(M^3, g, k) \xrightarrow{\text{Jang}} (\bar{M}, \bar{g}) \xrightarrow{\text{Conformal}} (\tilde{M}, \tilde{g}) \xrightarrow{\text{AMO}} M_{\text{ADM}} \geq \sqrt{A/16\pi}
\]
\end{tcolorbox}

\begin{tcolorbox}[colback=yellow!5!white, colframe=yellow!60!black, title=\textbf{Track B: Spacetime Approach (Boost-Invariant $\mathcal{Q}$)}]
\textbf{Status: PROGRAMMATIC FRAMEWORK (Gap 1 open)}

Logical chain:
\[
\Sigma_{\text{trapped}} \xrightarrow{\text{Weak null flow}} \Sigma_s \xrightarrow{\mathcal{Q}\text{ monotone}} \mathscr{I}^+ \Rightarrow M_B \geq \sqrt{A/16\pi}
\]
\end{tcolorbox}

\section{Six-Stage Proof Architecture}

Based on the roadmap provided, here is the complete logical flow:

\subsection{Stage 1: Geometric Setup and Preliminaries}

\begin{tabular}{|p{4cm}|p{8cm}|p{2.5cm}|}
\hline
\textbf{Component} & \textbf{Content in paper.tex} & \textbf{Status} \\
\hline
Initial data $(M^3, g, k)$ & Section 1.1: Conventions, DEC & \proven \\
Asymptotic flatness $\tau > 1/2$ & Remark 1.2, Section 4.4 (borderline) & \proven \\
MOTS definition & Definition 2.1, Section 2.1 & \proven \\
Trapped surface conventions & Remark 1.6 & \proven \\
\hline
\end{tabular}

\subsection{Stage 2: Metric Deformation (Jang Equation)}

\begin{tabular}{|p{4cm}|p{8cm}|p{2.5cm}|}
\hline
\textbf{Component} & \textbf{Content in paper.tex} & \textbf{Status} \\
\hline
Generalized Jang equation & Definition in Section 3 & \proven \\
Han-Khuri existence & Theorem (Han-Khuri 2013 cited) & External \\
Blow-up asymptotics at MOTS & Lemma (Sharp Asymptotics) & \proven \\
Mass reduction $M_{\text{ADM}}(\bar{g}) \leq M_{\text{ADM}}(g)$ & Theorem (Jang Reduction) & \proven \\
\hline
\end{tabular}

\subsection{Stage 3: Conformal Transformation}

\begin{tabular}{|p{4cm}|p{8cm}|p{2.5cm}|}
\hline
\textbf{Component} & \textbf{Content in paper.tex} & \textbf{Status} \\
\hline
Lichnerowicz equation & Section 4 (Analysis) & \proven \\
Conformal bound $\phi \leq 1$ & Theorem (PhiBound) via Bray-Khuri identity & \proven \\
Mass reduction & $M_{\text{ADM}}(\tilde{g}) \leq M_{\text{ADM}}(\bar{g})$ & \proven \\
Mean curvature jump $[H] \geq 0$ & Theorem (CompleteMeanCurvatureJump) & \proven \\
\hline
\end{tabular}

\subsection{Stage 4: $p$-Harmonic Level Set Analysis (AMO Method)}

\begin{tabular}{|p{4cm}|p{8cm}|p{2.5cm}|}
\hline
\textbf{Component} & \textbf{Content in paper.tex} & \textbf{Status} \\
\hline
$p$-harmonic equation $\Delta_p u = 0$ & Section 3.2 (AMO review) & \proven \\
Level set geometry & Section 3 & \proven \\
Critical point handling & Section 3 (capacity removability) & \proven \\
Lipschitz metric extension & Theorem (AMOHypothesisVerification) & \proven \\
\hline
\end{tabular}

\subsection{Stage 5: Monotonicity Formula}

\begin{tabular}{|p{4cm}|p{8cm}|p{2.5cm}|}
\hline
\textbf{Component} & \textbf{Content in paper.tex} & \textbf{Status} \\
\hline
Geroch-type monotonicity & Theorem (AMOMonotonicity) & \proven \\
$\mathcal{M}_p(t)$ is non-decreasing & Section 3 & \proven \\
Double limit $p \to 1^+$, $\epsilon \to 0$ & Theorem (CompleteDblLimit) & \proven \\
Mosco convergence & Section 5 & \proven \\
\hline
\end{tabular}

\subsection{Stage 6: Asymptotic Analysis and Synthesis}

\begin{tabular}{|p{4cm}|p{8cm}|p{2.5cm}|}
\hline
\textbf{Component} & \textbf{Content in paper.tex} & \textbf{Status} \\
\hline
$\lim_{t \to \infty} Q(t) = M_{\text{ADM}}$ & Section 5 (Synthesis) & \proven \\
$\lim_{t \to 0} Q(t) = \sqrt{A/16\pi}$ & Hawking mass at horizon & \proven \\
Inequality derivation & Section 5 & \proven \\
Rigidity (equality case) & Section 6 (Schwarzschild) & \conditional \\
\hline
\end{tabular}

\section{Gap Analysis for Spacetime Approach (Section 9)}

The boost-invariant quasi-local mass approach (Section 9, ``$\mathcal{Q}$-method'') has three identified gaps:

\subsection{Gap 1: Weak Null Flow Existence}

\begin{tcolorbox}[colback=red!5!white, colframe=red!70!black]
\textbf{Status: OPEN}

\textbf{Problem:} Null geodesics develop caustics where the foliation degenerates.

\textbf{Required:}
\begin{enumerate}[label=(\alph*)]
\item Variational selection principle (outward-minimizing hull analog)
\item Jump monotonicity: $\mathcal{Q}^*$ non-decreasing across jumps
\item Global existence from $\Sigma$ to $\mathscr{I}^+$
\end{enumerate}

\textbf{Analogy:} This is the Lorentzian analog of Huisken-Ilmanen's weak IMCF.

\textbf{What would close it:}
\begin{itemize}
\item Lorentzian geometric measure theory for null hypersurfaces
\item Restricted class: perturbations of Schwarzschild
\item Analytic spacetimes (unique continuation)
\end{itemize}
\end{tcolorbox}

\subsection{Gap 2: MOTS Degeneracy (SUBSUMED)}

\begin{tcolorbox}[colback=green!5!white, colframe=green!60!black]
\textbf{Status: RESOLVED (absorbed into Gap 1)}

\textbf{Original problem:} $\sigma^+/\theta^+ \to \infty$ at MOTS.

\textbf{Resolution:} The MOTS-avoiding weak flow framework (Definition in paper.tex) jumps \textbf{before} reaching MOTS, not through it. Gap 2 and Gap 1 are unified.
\end{tcolorbox}

\subsection{Gap 3: Rigidity}

\begin{tcolorbox}[colback=yellow!5!white, colframe=yellow!60!black]
\textbf{Status: CONDITIONAL}

\textbf{Problem:} Prove equality $\Rightarrow$ Schwarzschild.

\textbf{Current status:}
\begin{itemize}
\item Complete five-step outline (Proposition in paper.tex)
\item Unconditional in spherical symmetry (Theorem: spherical anchor)
\item General case needs unique continuation / Carleman estimates
\end{itemize}
\end{tcolorbox}

\section{Reference Coverage Checklist}

\subsection{Essential References (Must Cite)}

\begin{tabular}{|p{5cm}|p{6cm}|p{3cm}|}
\hline
\textbf{Author(s)} & \textbf{Contribution} & \textbf{Cited?} \\
\hline
Penrose (1973) & Original conjecture & Yes (\texttt{penrose1973}) \\
Huisken-Ilmanen (2001) & IMCF, weak solutions & Yes (\texttt{huiskenilmanen2001}) \\
Bray (2001) & Conformal flow & Yes (\texttt{bray2001}) \\
Han-Khuri (2013) & Generalized Jang & Yes (\texttt{hankhuri2013}) \\
Andersson-Metzger (2009) & MOTS existence & Yes \\
Agostiniani-Mazzieri-Oronzio (2022) & $p$-harmonic method & Yes (\texttt{amo2022}) \\
Bray-Khuri & Divergence identity & Yes \\
\hline
\end{tabular}

\subsection{Should Also Cite}

\begin{tabular}{|p{5cm}|p{6cm}|p{3cm}|}
\hline
\textbf{Author(s)} & \textbf{Contribution} & \textbf{Check} \\
\hline
Hawking (1971) & Area theorem & Check \\
Hayward (1994) & Quasi-local mass & Check \\
Mars-Senovilla & Trapped surface theory & Check \\
Christodoulou-Klainerman & Global stability & Check \\
Schoen-Yau (1979) & Positive mass theorem & Check \\
\hline
\end{tabular}

\section{Suggested ``Organization of the Paper'' Text}

The following is a polished paragraph for insertion into the Introduction:

\begin{tcolorbox}[colback=white, colframe=black]
\textbf{Organization of the paper.} 
Section~2 establishes the geometric framework: asymptotically flat initial data $(M^3, g, k)$, the dominant energy condition, and the definition of trapped surfaces and MOTS. 
Section~3 reviews the $p$-harmonic level set method of Agostiniani--Mazzieri--Oronzio, including the monotonicity formula and its extension to metrics with distributional curvature. 
Section~4 develops the generalized Jang equation following Han--Khuri, establishing existence, blow-up asymptotics at MOTS, and the mass reduction inequality $M_{\text{ADM}}(\bar{g}) \leq M_{\text{ADM}}(g)$. 
Section~5 constitutes the analytic core: we prove the conformal bound $\phi \leq 1$ via the Bray--Khuri divergence identity, establish the mean curvature jump condition $[H] \geq 0$ at stable MOTS, and verify the hypotheses of the AMO framework for Lipschitz metrics with conical singularities at bubble tips. 
Section~6 synthesizes these components, combining Jang reduction, conformal deformation, and $p$-harmonic level sets to derive the spacetime Penrose inequality. 
Section~7 addresses the equality case and rigidity to Schwarzschild. 
Section~8 develops the $\theta^+$-flow method for the ``unfavorable'' case $\text{tr}_\Sigma k < 0$, establishing area monotonicity and the slice independence theorem. 
Section~9 presents a boost-invariant quasi-local mass program for the full spacetime setting, identifying three gaps (weak flow existence, MOTS degeneracy, rigidity) and proving conditional results including spherical symmetry verification. 
Appendix~A collects technical lemmas on weighted Sobolev spaces and capacity estimates.
\end{tcolorbox}

\section{Critical Path Summary}

\begin{center}
\begin{tikzpicture}[node distance=2cm, auto,
  block/.style={rectangle, draw, fill=blue!10, text width=5cm, text centered, rounded corners, minimum height=1cm},
  decision/.style={diamond, draw, fill=yellow!20, text width=2.5cm, text centered, inner sep=0pt, minimum height=1cm},
  line/.style={draw, -latex'}]
  
% This would be a TikZ diagram showing the proof flow
% Simplified text representation:
\end{tikzpicture}
\end{center}

\textbf{For MOTS Penrose (Track A):}
\[
\boxed{\text{DEC}} \to \boxed{\text{Jang}} \to \boxed{\text{Conformal}} \to \boxed{\text{AMO}} \to \boxed{M \geq \sqrt{A/16\pi}}
\]
\textbf{Status: RIGOROUS}

\textbf{For Arbitrary Trapped (Track B):}
\[
\boxed{\text{DEC}} \to \boxed{\text{Weak null flow}} \to \boxed{\mathcal{Q}\text{-monotone}} \to \boxed{M_B \geq \sqrt{A/16\pi}}
\]
\textbf{Status: Gap 1 OPEN}

\end{document}
