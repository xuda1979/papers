% THE NULL EXPANSION FLOW
%
% A novel geometric flow that uses null expansions instead of mean curvature.
% This might give a proof of Penrose without the favorable jump condition.

\documentclass[12pt]{article}
\usepackage{amsmath,amsthm,amssymb}
\usepackage{mathrsfs}
\newtheorem{theorem}{Theorem}
\newtheorem{lemma}{Lemma}
\newtheorem{proposition}{Proposition}
\newtheorem{corollary}{Corollary}
\newtheorem{conjecture}{Conjecture}
\newtheorem{remark}{Remark}
\newtheorem{definition}{Definition}
\newtheorem{problem}{Problem}
\newtheorem{claim}{Claim}

\begin{document}

\title{The Null Expansion Flow:\\A Novel Approach to Penrose}
\author{Mathematical Development}
\date{\today}
\maketitle

\section{Motivation: Why Mean Curvature is Wrong}

The standard IMCF uses $\dot{x} = \nu/H$ where $H$ is mean curvature.
For trapped surfaces:
\begin{itemize}
    \item $H = \theta^+ - \tr_\Sigma k$ can have either sign
    \item When $\tr_\Sigma k < 0$ and large, $H$ can be positive even though $\theta^+ < 0$
    \item A positive $H$ means IMCF moves outward too fast
\end{itemize}

The PHYSICAL quantity is the null expansion $\theta^+$, not the mean curvature $H$.

\section{The Null Expansion Flow (NEF)}

\subsection{Definition}

\begin{definition}[Null Expansion Flow]
Given a family of surfaces $\Sigma_t$ with outer null expansion $\theta^+_t$, 
the \textbf{Null Expansion Flow} is:
\[
\frac{\partial x}{\partial t} = -\frac{\ell^+}{\theta^+}
\]
where $\ell^+ = \nu + n$ is the outgoing null normal and $\theta^+ < 0$.
\end{definition}

\textbf{Interpretation}: The surface expands in the null direction at rate $1/|\theta^+|$.
When $|\theta^+|$ is small (marginally trapped), expansion is fast.
When $|\theta^+|$ is large (strongly trapped), expansion is slow.

\subsection{Comparison with IMCF}

\begin{center}
\begin{tabular}{|c|c|c|}
\hline
& IMCF & NEF \\
\hline
Speed & $1/H$ & $1/|\theta^+|$ \\
Direction & Spacelike $\nu$ & Null $\ell^+$ \\
Well-defined when & $H > 0$ & $\theta^+ < 0$ \\
Natural for & Minimal surfaces & Trapped surfaces \\
\hline
\end{tabular}
\end{center}

\section{Evolution Equations}

\subsection{Area Evolution}

Under a general flow $\dot{x} = V$:
\[
\frac{dA}{dt} = \int_\Sigma \langle V, H\nu \rangle \, dA
\]

For NEF with $V = -\ell^+/\theta^+ = -(\nu + n)/\theta^+$:
\[
\frac{dA}{dt} = -\frac{1}{\theta^+} \int_\Sigma H \, dA
\]

Wait, this assumes the flow is in the $\nu$ direction. Let me redo this.

\subsection{Careful Derivation}

The null vector $\ell^+ = \nu + n$ where:
\begin{itemize}
    \item $\nu$ = unit outward spacelike normal to $\Sigma$ in $M$
    \item $n$ = future unit timelike normal to $M$ in spacetime
\end{itemize}

For a surface moving in spacetime, the relevant quantity is how the area changes.

On the initial data surface $M$, if we project the null flow, we get:
\[
V_{\mathrm{spatial}} = -\frac{\nu}{\theta^+}
\]

This is just IMCF with $H$ replaced by $-\theta^+$!

\subsection{A True Null Flow}

For a genuine null flow, we need to leave the initial data surface and move 
into spacetime.

Consider: $\Sigma_0 \subset M$ flows to $\Sigma_t \subset \text{spacetime}$.

The area evolution along a null geodesic congruence is governed by:
\[
\frac{d(\ln A)}{d\lambda} = \theta
\]

where $\theta$ is the expansion of the congruence.

For the outgoing null congruence from $\Sigma_0$:
\[
\frac{dA}{d\lambda} = \theta^+ \cdot A
\]

Since $\theta^+ < 0$ for trapped surfaces, the area DECREASES!

\textbf{This is the wrong direction for Penrose.}

\section{Reparametrization: The Inverse Null Flow}

\subsection{The Idea}

Instead of following null geodesics with $d/d\lambda$, reparametrize so that 
area increases:
\[
\frac{dA}{dt} = \text{const} > 0
\]

Define parameter $t$ by:
\[
\frac{dt}{d\lambda} = \frac{c}{|\theta^+| A}
\]
for some constant $c > 0$.

Then:
\[
\frac{dA}{dt} = \frac{dA}{d\lambda} \cdot \frac{d\lambda}{dt} = \theta^+ A \cdot \frac{1}{c/(|\theta^+ |A)} = c
\]

Wait, this gives $dA/dt = -c$ since $\theta^+ < 0$.

\subsection{Reversing Direction?}

If we follow the INCOMING null direction $\ell^- = \nu - n$:
\[
\frac{dA}{d\lambda} = \theta^- \cdot A
\]

For trapped surfaces, $\theta^- < 0$ also, so area still decreases.

\textbf{Fundamental issue}: For trapped surfaces, area decreases in BOTH null directions!

\section{A Different Approach: The Quasi-Local Mass Flow}

\subsection{The Hawking Mass}

The Hawking mass is:
\[
M_H = \sqrt{\frac{A}{16\pi}}\left(1 - \frac{1}{16\pi}\int_\Sigma H^2 \, dA\right)
\]

\subsection{Evolution Under Null Flow}

Under the outgoing null flow:
\[
\frac{d}{d\lambda}M_H = ?
\]

This requires knowing how $H$ evolves. In spacetime:
\[
H = \frac{1}{2}(\theta^+ + \theta^-)
\]

The Raychaudhuri equation gives:
\begin{align}
    \frac{d\theta^+}{d\lambda} &= -\frac{1}{2}(\theta^+)^2 - |\sigma^+|^2 - R_{\mu\nu}\ell^{+\mu}\ell^{+\nu}
\end{align}

where NEC implies $R_{\mu\nu}\ell^{+\mu}\ell^{+\nu} \ge 0$.

\subsection{The Monotonicity?}

For vacuum ($R_{\mu\nu} = 0$) and shear-free ($\sigma^+ = 0$):
\[
\frac{d\theta^+}{d\lambda} = -\frac{1}{2}(\theta^+)^2
\]

This has solution $\theta^+(\lambda) = \frac{\theta^+_0}{1 + \frac{1}{2}\theta^+_0 \lambda}$.

Since $\theta^+_0 < 0$, this becomes more negative, then blows up at $\lambda = -2/\theta^+_0$.

\textbf{This is a focusing singularity!}

For trapped surfaces, following null geodesics leads to focusing, not expansion.

\section{The Fundamental Tension}

There's a fundamental tension:

\begin{enumerate}
    \item The Penrose inequality says area should be BOUNDED from below
    \item The trapped condition says area DECREASES along null directions
    \item This seems contradictory!
\end{enumerate}

\subsection{Resolution}

The resolution is that the Penrose inequality is about DIFFERENT surfaces:
- $\Sigma_0$: the given trapped surface (initial data)
- Schwarzschild horizon: the final state (formed after collapse)

The ADM mass measures the TOTAL energy, which includes:
- Rest mass of the black hole
- Gravitational binding energy
- Kinetic energy of matter

The inequality says the total energy is at least the rest mass of the hole.

\section{A New Functional: The Penrose Energy}

\subsection{Definition}

\begin{definition}[Penrose Energy]
For a trapped surface $\Sigma$ in initial data $(M, g, k)$, define:
\[
E_P[\Sigma] = M_{\mathrm{ADM}} - \sqrt{\frac{A(\Sigma)}{16\pi}}
\]
\end{definition}

The Penrose inequality is equivalent to: $E_P[\Sigma] \ge 0$ for all trapped $\Sigma$.

\subsection{Variation of Penrose Energy}

How does $E_P$ vary as we deform $\Sigma$?

\[
\frac{\delta E_P}{\delta \Sigma} = -\frac{1}{\sqrt{16\pi A}} \cdot \frac{\delta A}{\delta \Sigma} = -\frac{H}{2\sqrt{A/(16\pi)}}
\]

Critical points ($\delta E_P = 0$) are surfaces with $H = 0$ (minimal surfaces)!

But minimal surfaces are NOT trapped in general.

\subsection{Constrained Variation}

Vary $\Sigma$ among TRAPPED surfaces only.

The constraint is $\theta^+ \le 0$, i.e., $H \le -\tr_\Sigma k$.

The Lagrangian is:
\[
\mathcal{L} = E_P - \lambda(\theta^+ - 0) = E_P - \lambda(H + \tr_\Sigma k)
\]

Critical points satisfy:
\[
-\frac{H}{2\sqrt{A/(16\pi)}} = \lambda \cdot \frac{\delta(H + \tr k)}{\delta \Sigma}
\]

This is complicated, but the point is: MOTS ($\theta^+ = 0$) are likely critical.

\section{A Flow that Increases $E_P$}

\subsection{The Ideal Flow}

We want a flow of trapped surfaces $\Sigma_t$ such that:
\[
\frac{d E_P}{dt} \ge 0
\]

Since $M_{\mathrm{ADM}}$ is fixed (it's a property of the initial data, not the surface):
\[
\frac{d E_P}{dt} = -\frac{1}{2\sqrt{A/(16\pi)}} \cdot \frac{dA}{dt}
\]

So we want $dA/dt \le 0$: area should DECREASE!

\subsection{But Wait...}

If area decreases, then $\sqrt{A/(16\pi)}$ decreases, so $E_P$ INCREASES.

This means: flowing to SMALLER trapped surfaces makes $E_P$ closer to zero!

The SMALLEST trapped surface (if it exists) would give the LARGEST value of $E_P$.

\subsection{The Minimum Area Trapped Surface}

\begin{conjecture}
Among all trapped surfaces homologous to $\Sigma_0$, there exists one with 
minimum area. Call it $\Sigma_{\min}$.

Then: $E_P[\Sigma_{\min}] \ge E_P[\Sigma_0]$.

If we can prove $E_P[\Sigma_{\min}] \ge 0$ independently, Penrose follows.
\end{conjecture}

\subsection{What is $\Sigma_{\min}$?}

The minimum area trapped surface satisfies:
- $\theta^+ \le 0$ (trapped)
- $H = $ minimal among trapped (critical point of area)

At a critical point of area within the trapped class, we either have:
- $\theta^+ = 0$ (saturating the constraint) $\Rightarrow$ MOTS
- $H = 0$ (unconstrained critical) $\Rightarrow$ minimal surface

For a MOTS: $H = -\tr_\Sigma k$

The MOTS is minimal among trapped surfaces when $\tr_\Sigma k \ge 0$ (favorable case)!

When $\tr_\Sigma k < 0$: $H = -\tr_\Sigma k > 0$, so MOTS is NOT a minimum.

\section{Conclusion: The True Insight}

\begin{claim}[Core Insight]
In the favorable case ($\tr_\Sigma k \ge 0$), the outermost MOTS $\Sigma^*$ is a 
local MINIMUM of area among trapped surfaces nearby.

In the unfavorable case ($\tr_\Sigma k < 0$), the MOTS is a SADDLE POINT of area.
There exist trapped surfaces with SMALLER area than the MOTS!
\end{claim}

\subsection{Implications}

The Penrose inequality via MOTS is:
\[
M \ge M_H(\Sigma^*) = \sqrt{\frac{A(\Sigma^*)}{16\pi}}\left(1 - \frac{(\tr k)^2 A}{16\pi}\right)
\]

For unfavorable $\tr k < 0$, this gives a WEAKER bound than $\sqrt{A(\Sigma^*)/(16\pi)}$.

But there might be SMALLER trapped surfaces with BETTER Hawking mass!

\section{Next Step: Find the Optimal Surface}

\begin{problem}[Optimal Penrose Surface]
Among all trapped surfaces in the trapped region, find the one that MAXIMIZES 
the Hawking mass $M_H[\Sigma]$.

Call this the \textbf{Penrose surface} $\Sigma_P$.
\end{problem}

\begin{conjecture}[Penrose Surface Inequality]
\[
M_{\mathrm{ADM}} \ge M_H[\Sigma_P]
\]
and
\[
M_H[\Sigma_P] \ge \sqrt{\frac{A(\Sigma_0)}{16\pi}}
\]
for any trapped surface $\Sigma_0$ in the region.
\end{conjecture}

This would prove the full Penrose inequality without any sign conditions!

\end{document}
