%% AREA_DOMINANCE_ATTACK.tex
%%
%% Blue/Red Team Attack on Area Dominance Proof
%% December 2025

\documentclass[11pt]{amsart}
\usepackage{amsmath,amssymb,amsthm}
\usepackage{xcolor}
\usepackage{tcolorbox}

\tcbuselibrary{theorems,skins}

\newtcolorbox{redteam}{
    colback=red!5!white,
    colframe=red!75!black,
    title={\textbf{RED TEAM ATTACK}}
}

\newtcolorbox{blueteam}{
    colback=blue!5!white,
    colframe=blue!75!black,
    title={\textbf{BLUE TEAM DEFENSE}}
}

\newtcolorbox{fatal}{
    colback=red!20!white,
    colframe=red!90!black,
    title={\textbf{FATAL?}}
}

\newtcolorbox{verdict}{
    colback=yellow!10!white,
    colframe=orange!75!black,
    title={\textbf{VERDICT}}
}

\newtheorem{theorem}{Theorem}
\newcommand{\Area}{\mathrm{Area}}

\title{Blue/Red Attack: Area Dominance Proof}
\date{December 2025}

\begin{document}
\maketitle

%% ============================================================================
\section{The Claimed Proof}
%% ============================================================================

\textbf{Claim:} For trapped $\Sigma$ inside MOTS $\Sigma^*$: $\Area(\Sigma) \le \Area(\Sigma^*)$.

\textbf{Proof idea:} Foliate by $\theta^+$ level sets. Area increases as $\theta^+ \to 0$.

%% ============================================================================
\section{Attack Round 1: Does the Foliation Exist?}
%% ============================================================================

\begin{redteam}
\textbf{Attack 1.1:} You claim the region between $\Sigma$ and $\Sigma^*$ is foliated by $\theta^+$ level sets.

But $\theta^+$ is defined on SURFACES, not on points of $\mathcal{C}$!

How can you have ``level sets of $\theta^+$'' when $\theta^+$ is not a function on the 3-manifold?
\end{redteam}

\begin{blueteam}
\textbf{Defense 1.1:}

You're absolutely right. $\theta^+[S]$ is a functional on surfaces, not a function on $\mathcal{C}$.

\textbf{Correction:} We don't foliate by level sets of $\theta^+$.

Instead, we consider a 1-parameter family of surfaces $\{\Sigma_t\}_{t \in [0,1]}$ connecting $\Sigma = \Sigma_0$ to $\Sigma^* = \Sigma_1$.

Along this family, $\theta^+(\Sigma_t)$ varies from $\theta^+(\Sigma) < 0$ to $\theta^+(\Sigma^*) = 0$.

The claim is that area increases along this family.
\end{blueteam}

\begin{redteam}
\textbf{Attack 1.2:} But there are MANY families connecting $\Sigma$ to $\Sigma^*$!

For some families, area might increase. For others, it might decrease.

Which family do you use, and why does it give the right answer?
\end{redteam}

\begin{blueteam}
\textbf{Defense 1.2:}

We use the ``canonical'' family defined by the normal exponential map or a smooth interpolation.

\textbf{However}, the proof doesn't depend on the specific family!

The key is: what happens to $\Area$ as a function of $\theta^+$ along ANY family?

\textbf{New argument:}

Define $f(s) = \min\{\Area(S) : \theta^+(S) = s, S \text{ encloses } \Sigma\}$.

We want to show $f(s)$ is increasing in $s$ (for $s \le 0$).

This would give $\Area(\Sigma) \le f(0) = \Area(\Sigma^*)$.
\end{blueteam}

\begin{verdict}
\textbf{ISSUE IDENTIFIED:} The original ``foliation by $\theta^+$ level sets'' is ill-defined.

\textbf{Status:} Need a cleaner argument.
\end{verdict}

%% ============================================================================
\section{Attack Round 2: The Variation Formula}
%% ============================================================================

\begin{redteam}
\textbf{Attack 2.1:} You wrote:
\[
\frac{d\Area(\Sigma_s)}{ds} = \int_{\Sigma_s} \frac{1}{|\nabla\theta^+|} dA
\]

This formula is WRONG. Where did it come from?

The first variation of area is $\frac{dA}{dt} = \int H \phi \, dA$ where $\phi$ is normal speed.

Where does $|\nabla\theta^+|$ appear?
\end{redteam}

\begin{blueteam}
\textbf{Defense 2.1:}

You're right, that formula was incorrect.

Let me redo the variation properly.

\textbf{Correct first variation:}

For a family $\Sigma_t$ with normal velocity $V = \phi \nu$:
\begin{equation}
    \frac{d}{dt}\Area(\Sigma_t) = \int_{\Sigma_t} H \phi \, dA
\end{equation}

where $H$ = mean curvature (trace of second fundamental form in $\mathcal{C}$).

\textbf{Variation of $\theta^+$:}

\begin{equation}
    \frac{d}{dt}\theta^+(\Sigma_t) = L(\phi)
\end{equation}

where $L$ is the stability operator:
\begin{equation}
    L(\phi) = -\Delta\phi + 2\omega \cdot \nabla\phi + Q\phi
\end{equation}

with $Q$ depending on curvatures.

\textbf{Chain rule:}

If we parameterize by $s = \theta^+$, then:
\begin{equation}
    \frac{d\Area}{ds} = \frac{d\Area/dt}{d\theta^+/dt} = \frac{\int H\phi \, dA}{\int L(\phi) dA}
\end{equation}

This is NOT obviously positive!
\end{blueteam}

\begin{fatal}
\textbf{The variation formula does NOT obviously give area increase!}

$\frac{d\Area}{ds}$ depends on $H$, $\phi$, and the stability operator $L$.

The sign is not determined by the trapped/MOTS structure alone.
\end{fatal}

%% ============================================================================
\section{Attack Round 3: Schwarzschild Verification}
%% ============================================================================

\begin{redteam}
\textbf{Attack 3.1:} You claimed Schwarzschild confirms area dominance.

VERIFY THIS EXPLICITLY.

Compute $\Area(S_r)$ for coordinate spheres inside the horizon and check $\Area(S_r) \le \Area(S_{2M})$.
\end{redteam}

\begin{blueteam}
\textbf{Defense 3.1:}

\textbf{Schwarzschild metric (outside horizon):}
\[
ds^2 = -\left(1 - \frac{2M}{r}\right)dt^2 + \left(1 - \frac{2M}{r}\right)^{-1}dr^2 + r^2 d\Omega^2
\]

\textbf{Coordinate spheres:} $S_r = \{t = const, r = const\}$ have $\Area(S_r) = 4\pi r^2$.

\textbf{Horizon:} $r = 2M$, $\Area = 16\pi M^2$.

\textbf{Inside horizon ($r < 2M$):}

The coordinates swap roles: $r$ becomes timelike, $t$ becomes spacelike.

On a ``constant time'' slice (which is now $r = const$), we have 2-spheres with area $4\pi r^2$.

For $r < 2M$: $\Area = 4\pi r^2 < 4\pi (2M)^2 = 16\pi M^2$.

\textbf{Verification:} $\Area(S_r) < \Area(S_{2M})$ for $r < 2M$. ✓

\textbf{Trapped check:}

For $r < 2M$, the spheres $S_r$ are trapped: both null expansions $\theta^\pm < 0$.

The MOTS is at $r = 2M$: $\theta^+ = 0$.

\textbf{Conclusion:} Area dominance holds in Schwarzschild. ✓
\end{blueteam}

\begin{verdict}
\textbf{VERIFIED in Schwarzschild.}

But Schwarzschild is spherically symmetric — doesn't prove the general case.
\end{verdict}

%% ============================================================================
\section{Attack Round 4: Non-Spherical Surfaces}
%% ============================================================================

\begin{redteam}
\textbf{Attack 4.1:} In Schwarzschild, you only checked ROUND spheres.

What about a NON-SPHERICAL trapped surface inside the horizon?

Can't you crinkle a surface to make it have larger area while keeping $\theta^+ < 0$?
\end{redteam}

\begin{blueteam}
\textbf{Defense 4.1:}

This is the key question. Let me analyze.

In Schwarzschild, consider a non-spherical surface $\Sigma$ at some average radius $r_0 < 2M$.

$\Sigma$ is a deformation of $S_{r_0}$: $\Sigma = \{r = r_0 + f(\theta, \phi)\}$.

\textbf{Area:}
\[
\Area(\Sigma) = \int (r_0 + f)^2 \sqrt{1 + |\nabla f|^2/(r_0+f)^2} \sin\theta \, d\theta d\phi
\]

For large $|\nabla f|$, area can be much larger than $4\pi r_0^2$.

\textbf{Expansion:}

$\theta^+ = H + P$ where $H$ is mean curvature and $P = \tr_\Sigma k$.

For a crinkled surface, $H$ oscillates. To maintain $\theta^+ < 0$ everywhere, we need $H < -P$ at every point.

\textbf{The constraint:}

In Schwarzschild, $P$ depends on the surface shape. For highly crinkled surfaces, $P$ can vary significantly.

The question: can we have $\theta^+ < 0$ everywhere while $\Area(\Sigma) > 16\pi M^2$?

\textbf{Analysis:}

For a surface to have large area, it needs large $|\nabla f|$ (high gradients).

Large gradients create regions of high positive mean curvature $H > 0$ (in the ``valleys'').

To keep $\theta^+ = H + P < 0$ in these regions, we need $P < -H < 0$, i.e., $P$ very negative.

But $P$ is constrained by the ambient geometry (Schwarzschild).

\textbf{Claim:} The Schwarzschild geometry constrains $P$ such that highly crinkled surfaces cannot maintain $\theta^+ < 0$ everywhere.

\textbf{This needs rigorous proof.}
\end{blueteam}

\begin{redteam}
\textbf{Attack 4.2:} You claim $P$ is constrained, but you haven't proven it.

Give me a rigorous bound: $\Area(\Sigma) \le C$ for trapped $\Sigma$ in Schwarzschild.
\end{redteam}

\begin{blueteam}
\textbf{Defense 4.2 (Schwarzschild case):}

In Schwarzschild, the extrinsic curvature of the $t = const$ slice is $k = 0$ (time-symmetric).

So $P = \tr_\Sigma k = 0$.

Therefore $\theta^+ = H$.

Trapped means $H < 0$ everywhere!

\textbf{Gauss-Bonnet constraint:}

For a surface of genus 0:
\[
\int_\Sigma K_\Sigma \, dA = 4\pi
\]

By Gauss equation in the curved ambient space:
\[
K_\Sigma = \bar{K} + \frac{H^2 - |\mathrm{I\!I}|^2}{2}
\]

where $\bar{K}$ involves ambient curvature.

Since $H < 0$ and $H^2 \le |\mathrm{I\!I}|^2$ (by Cauchy-Schwarz):
\[
K_\Sigma \le \bar{K}
\]

Integrating:
\[
4\pi = \int K_\Sigma \, dA \le \int \bar{K} \, dA
\]

In Schwarzschild at radius $r$, $\bar{K} \sim 1/r^2$, so:
\[
4\pi \le \frac{C}{r^2} \Area(\Sigma)
\]

giving $\Area(\Sigma) \ge 4\pi r^2 / C$.

This is a LOWER bound, not upper!

\textbf{Different approach:}

For $H < 0$ everywhere on $\Sigma$:
\[
\int_\Sigma H \, dA < 0
\]

The total mean curvature is negative.

\textbf{Minkowski-type inequality:}

In flat space, for convex surfaces: $\int H \, dA \ge c \sqrt{\Area}$.

For $H < 0$: this reverses to $\int |H| \, dA \ge c\sqrt{\Area}$, i.e., $-\int H \, dA \ge c\sqrt{\Area}$.

Since $\int H \, dA < 0$, we have $|\int H \, dA| \ge c\sqrt{\Area}$.

\textbf{But we need an upper bound on Area, not a lower bound!}

The Minkowski inequality bounds area from below in terms of mean curvature, not above.

\textbf{Hmm, this is tricky.}
\end{blueteam}

\begin{verdict}
\textbf{STATUS: OPEN}

Even in Schwarzschild, a rigorous area upper bound for non-spherical trapped surfaces is non-trivial.

The spherical case works, but general trapped surfaces need more analysis.
\end{verdict}

%% ============================================================================
\section{Attack Round 5: The Real Issue}
%% ============================================================================

\begin{redteam}
\textbf{Attack 5.1 (FUNDAMENTAL):}

Let me construct a potential counterexample.

Consider a ``tube'' shaped trapped surface in Schwarzschild:
\begin{itemize}
    \item Take two small spheres $S_1, S_2$ at $r = r_0 < 2M$, separated in the $t$ direction
    \item Connect them by a thin tube
    \item The total area = $2 \times 4\pi r_0^2 + \text{tube area}$
\end{itemize}

The tube area can be made arbitrarily large.

Is this surface trapped?
\end{redteam}

\begin{blueteam}
\textbf{Defense 5.1:}

\textbf{Checking if the tube is trapped:}

On the thin tube connecting the spheres:
\begin{itemize}
    \item The tube has $H > 0$ (curving inward, like a cylinder)
    \item For $\theta^+ = H + P < 0$, we need $P < -H < 0$
\end{itemize}

In Schwarzschild time-symmetric slice: $P = \tr_\Sigma k = 0$.

So $\theta^+ = H > 0$ on the tube — NOT trapped!

\textbf{Conclusion:} The tube construction fails. The tube portion has $\theta^+ > 0$.

\textbf{To be trapped, the surface must have $H < 0$ everywhere (in time-symmetric case).}

A thin tube has $H > 0$, so it cannot be part of a trapped surface.
\end{blueteam}

\begin{redteam}
\textbf{Attack 5.2:} OK, what about a ``bumpy sphere'' — small bumps on a sphere at $r_0 < 2M$?

The bumps add area but the surface is still mostly at radius $r_0$.
\end{redteam}

\begin{blueteam}
\textbf{Defense 5.2:}

For a bumpy sphere $\Sigma$ near $r = r_0$:

At the ``peaks'' of bumps (pointing outward): $H < 0$ (like the ambient sphere). ✓

At the ``valleys'' between bumps: $H$ could be $> 0$ (local minima curve inward).

For $\theta^+ = H < 0$ everywhere (time-symmetric), we need $H < 0$ in the valleys too.

\textbf{Constraint:} Valleys can't be too deep/sharp, or $H > 0$ there.

This limits how bumpy the surface can be.

\textbf{Quantitative bound:}

If $\Sigma$ is a graph over $S_{r_0}$: $r = r_0 + f(\theta, \phi)$ with $|f| \le \epsilon$, $|\nabla f| \le C$.

The mean curvature perturbation:
\[
H = H_0 + \delta H
\]

where $H_0 = H(S_{r_0}) < 0$ (since $r_0 < 2M$), and $\delta H \sim \Delta f + O(|\nabla f|^2)$.

For $H < 0$: $H_0 + \delta H < 0$, i.e., $\delta H < -H_0 = |H_0|$.

Since $|\delta H| \lesssim |\Delta f| + |\nabla f|^2$, we need $|\Delta f| + |\nabla f|^2 \lesssim |H_0|$.

This bounds $|\nabla f|$, hence bounds the area perturbation:
\[
\Area(\Sigma) - \Area(S_{r_0}) \sim \int |\nabla f|^2 \lesssim |H_0|^2 \Area(S_{r_0})
\]

So the area can only increase by a bounded factor!
\end{blueteam}

\begin{verdict}
\textbf{KEY INSIGHT:}

The trapped condition $H < 0$ (time-symmetric case) DOES constrain area.

Bumps/wrinkles create regions where $H$ wants to be positive.

To maintain $H < 0$ everywhere, wrinkles must be limited.

\textbf{This suggests area dominance IS TRUE, with a rigorous proof possible.}
\end{verdict}

%% ============================================================================
\section{Rigorous Proof Sketch}
%% ============================================================================

\begin{theorem}[Area Dominance — Time Symmetric Case]
For time-symmetric initial data ($k = 0$), if $\Sigma$ is trapped ($H < 0$) inside MOTS $\Sigma^*$ ($H = 0$), then:
\[
\Area(\Sigma) \le \Area(\Sigma^*)
\]
\end{theorem}

\begin{proof}
\textbf{Step 1: Comparison with minimal surfaces.}

$\Sigma^*$ is a MOTS, which in time-symmetric case means $H = 0$ (minimal surface).

$\Sigma$ has $H < 0$ (mean-convex inward).

\textbf{Step 2: Foliation by $H$-level sets.}

Consider the level sets of the ``arrival time'' function from $\Sigma^*$.

Actually, let's use a different approach.

\textbf{Step 3: Maximum principle argument.}

Consider the region $\Omega$ bounded by $\Sigma$ and $\Sigma^*$.

Let $u$ solve:
\[
\Delta u = 0 \text{ in } \Omega, \quad u|_\Sigma = 0, \quad u|_{\Sigma^*} = 1
\]

By maximum principle, $0 < u < 1$ in $\Omega$, and $|\nabla u| > 0$.

\textbf{Step 4: Co-area formula.}

\[
\Vol(\Omega) = \int_0^1 \left(\int_{\{u=t\}} \frac{1}{|\nabla u|} dA\right) dt
\]

\textbf{Step 5: Relate to areas.}

Level sets $\{u = t\}$ interpolate between $\Sigma$ and $\Sigma^*$.

We need: $\Area(\{u=t\})$ is monotonic in $t$.

\textbf{Step 6: Mean curvature of level sets.}

The mean curvature of $\{u = t\}$ is:
\[
H_t = -\divv\left(\frac{\nabla u}{|\nabla u|}\right) = -\frac{\Delta u}{|\nabla u|} + \frac{\nabla^2 u(\nabla u, \nabla u)}{|\nabla u|^3} = \frac{\nabla^2 u(\nabla u, \nabla u)}{|\nabla u|^3}
\]

using $\Delta u = 0$.

\textbf{Step 7: Sign of $H_t$.}

For harmonic $u$ in a region between $H < 0$ and $H = 0$ surfaces:

By comparison principle, the level sets transition from $H < 0$ to $H = 0$.

This suggests $H_t$ is increasing in $t$.

\textbf{Step 8: Area monotonicity.}

By the first variation of area for level sets:
\[
\frac{d}{dt}\Area(\{u=t\}) = \int_{\{u=t\}} \frac{H_t}{|\nabla u|} dA
\]

If $H_t < 0$ for $t$ near 0 and $H_t = 0$ for $t = 1$:

Area is increasing as $t$ increases (since we're moving against the mean curvature direction).

Wait, the sign needs careful checking...

\textbf{Alternative Step 8:}

Use the monotonicity formula for harmonic functions:
\[
\frac{d}{dt}\int_{\{u=t\}} |\nabla u| dA = \int_{\{u=t\}} \left(|\nabla^2 u|^2 / |\nabla u| - H_t |\nabla u|\right) dA
\]

This is getting complicated. Let me try a cleaner approach.

\textbf{CLEAN PROOF:}

\textbf{Claim:} In the time-symmetric case, $\Sigma^*$ is the unique area-minimizing surface in its homology class (among surfaces enclosing $\Sigma$).

\textbf{Proof of claim:} $\Sigma^*$ is a minimal surface ($H = 0$). By standard results, minimal surfaces are area-minimizing in their homology class (at least locally, and globally under suitable conditions).

Since $\Sigma$ is homologous to $\Sigma^*$ and is inside the region:
\[
\Area(\Sigma) \ge \Area(\Sigma^*) \text{ ???}
\]

Wait, this is the WRONG direction!

\textbf{Re-examination:}

If $\Sigma^*$ is area-MINIMIZING, then $\Area(\Sigma) \ge \Area(\Sigma^*)$.

But we want $\Area(\Sigma) \le \Area(\Sigma^*)$.

\textbf{Resolution:}

$\Sigma^*$ being minimal doesn't mean it minimizes area among all surfaces!

It's a critical point, which could be a saddle.

For OUTERMOST MOTS, it's actually area-MAXIMIZING among nearby trapped surfaces.

\textbf{Correct statement:}

The outermost MOTS $\Sigma^*$ is stable, meaning:

For deformations moving OUTWARD: area increases (second variation $\ge 0$).

For deformations moving INWARD: area decreases.

So $\Sigma^*$ has LARGER area than nearby surfaces moving inward (toward trapped region).

This gives $\Area(\Sigma) \le \Area(\Sigma^*)$ for $\Sigma$ ``close'' to $\Sigma^*$.

\textbf{For general $\Sigma$:} Use a flow from $\Sigma$ to $\Sigma^*$ and show area increases.
\end{proof}

%% ============================================================================
\section{Final Status}
%% ============================================================================

\begin{verdict}
\textbf{AREA DOMINANCE STATUS:}

\textbf{Time-symmetric case:} 
\begin{itemize}
    \item Strongly supported by Schwarzschild
    \item Constraint $H < 0$ limits area (bumps create $H > 0$)
    \item Full proof requires careful analysis of $H < 0$ constraint
    \item \textbf{VERY LIKELY TRUE}
\end{itemize}

\textbf{General case:}
\begin{itemize}
    \item More complex due to $P \ne 0$
    \item $\theta^+ = H + P < 0$ is weaker constraint
    \item Still constrained by DEC
    \item \textbf{LIKELY TRUE but harder to prove}
\end{itemize}

\textbf{Remaining work:}
\begin{enumerate}
    \item Rigorous proof that $H < 0$ bounds area (time-symmetric)
    \item Extension to general $k \ne 0$ case
    \item Verify no counterexamples exist
\end{enumerate}
\end{verdict}

\end{document}
