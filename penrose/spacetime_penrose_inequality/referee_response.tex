% Response to Referee Report
% The Spacetime Penrose Inequality (Da Xu)
% December 2025

\documentclass[11pt]{amsart}
\usepackage{amsmath,amssymb,amsthm}
\usepackage{enumitem}
\usepackage{xcolor}
\usepackage{mdframed}
\usepackage[margin=1in]{geometry}

\newcommand{\tr}{\mathrm{tr}}
\newcommand{\ADM}{\mathrm{ADM}}

\theoremstyle{plain}
\newtheorem{theorem}{Theorem}
\newtheorem{lemma}[theorem]{Lemma}
\theoremstyle{remark}
\newtheorem*{response}{Response}
\newtheorem*{correction}{Correction}

\definecolor{criticalcolor}{RGB}{180,0,0}
\definecolor{importantcolor}{RGB}{0,0,150}
\definecolor{fixcolor}{RGB}{0,120,0}

\newenvironment{critical}{\begin{mdframed}[linecolor=criticalcolor,linewidth=2pt]\color{criticalcolor}\textbf{CRITICAL ISSUE:}}{\end{mdframed}}
\newenvironment{important}{\begin{mdframed}[linecolor=importantcolor,linewidth=1pt]\color{importantcolor}\textbf{IMPORTANT:}}{\end{mdframed}}
\newenvironment{fix}{\begin{mdframed}[linecolor=fixcolor,linewidth=1pt,backgroundcolor=green!5]\color{fixcolor}\textbf{FIX:}}{\end{mdframed}}

\title{Response to Referee Report\\``The Spacetime Penrose Inequality''}
\author{Da Xu}
\date{December 2025}

\begin{document}
\maketitle

We thank the referee for an extraordinarily thorough and penetrating review. The referee has identified a \textbf{critical sign error} in the manuscript that invalidates the central argument as currently written. We address each major concern below with detailed corrections.

\tableofcontents

%==============================================================================
\section{Critical Issue 1: The Sign/Logic Error in the Jump Condition}
%==============================================================================

\begin{critical}
The referee correctly identifies that the ``sign analysis'' in Lemma~\ref{lem:TrappedMeanCurvatureJump} is \textbf{logically invalid}. The claim that $\theta^+ \le 0$ and $\theta^- < 0$ implies $\tr_{\Sigma_0} k \ge 0$ is \textbf{FALSE}.
\end{critical}

\subsection{The Error in Detail}

The manuscript states (around line 7557--7559):
\begin{quote}
``Subtracting the second from the first: $2\tr_{\Sigma_0} k \le 0 - \theta^- = -\theta^- > 0$, which gives $\tr_{\Sigma_0} k > -\theta^-/2 > 0$.''
\end{quote}

This is incorrect. The inequality $2\tr_{\Sigma_0} k \le -\theta^- > 0$ provides an \textbf{upper bound}, not a \textbf{lower bound}. From this we can only conclude:
\[
    \tr_{\Sigma_0} k < \frac{-\theta^-}{2} = \frac{|H - \tr_{\Sigma_0} k|}{2}
\]
which says nothing about the sign of $\tr_{\Sigma_0} k$.

\subsection{Explicit Counterexample}

The referee's counterexample is devastating:
\begin{itemize}
    \item Take $H_{\Sigma_0} = -3$ and $\tr_{\Sigma_0} k = -1$.
    \item Then $\theta^+ = H + \tr_{\Sigma_0} k = -3 + (-1) = -4 \le 0$ \quad (\checkmark future trapped)
    \item And $\theta^- = H - \tr_{\Sigma_0} k = -3 - (-1) = -2 < 0$ \quad (\checkmark inner trapped)
    \item But $\tr_{\Sigma_0} k = -1 < 0$ \quad (contradiction to the claimed $[H] = \tr_{\Sigma_0} k > 0$)
\end{itemize}

This shows the trapped conditions do \textbf{NOT} imply $\tr_{\Sigma_0} k \ge 0$.

\subsection{Consequences}

This error affects:
\begin{enumerate}
    \item \textbf{Lemma~\ref{lem:TrappedMeanCurvatureJump}}: The main claim is false as stated.
    \item \textbf{Theorem~\ref{thm:DirectTrappedJang}(iv)}: The ``favorable jump'' conclusion is unjustified.
    \item \textbf{Corner smoothing}: Miao-type smoothing requires $[H] \ge 0$ for $R_{\hat{g}_\epsilon} \ge 0$.
    \item \textbf{The entire proof chain}: The Penrose inequality for arbitrary trapped surfaces fails.
\end{enumerate}

\begin{fix}
\textbf{Resolution Options:}

\textbf{Option A: Strengthen the hypothesis.} The theorem should require an \textbf{additional condition} on $\Sigma_0$:
\begin{equation}\label{eq:ExtraCondition}
    \tr_{\Sigma_0} k \ge 0.
\end{equation}
This is \emph{not} automatic from the trapped conditions but is satisfied by many physically relevant surfaces.

\textbf{Option B: Restrict to outerminimizing/stable MOTS.} For stable MOTS ($\theta^+ = 0$, $\lambda_1(L_\Sigma) \ge 0$), the jump positivity follows from the stability formula, not from the trapped condition directly. This is the content of Theorem~\ref{thm:CompleteMeanCurvatureJump}, which remains valid.

\textbf{Option C: Use a different jump formula.} The correct relationship is:
\[
    [H]_{\bar{g}} = \tr_{\Sigma_0} k,
\]
which can be positive, negative, or zero depending on the surface. For $[H] < 0$, the distributional scalar curvature has a \emph{negative} singular part, and Miao smoothing does \emph{not} preserve $R \ge 0$.

\textbf{Recommended fix:} The paper should:
\begin{enumerate}
    \item Remove the claim that arbitrary trapped surfaces satisfy $[H] \ge 0$.
    \item Restrict the main theorem to surfaces satisfying $\tr_{\Sigma_0} k \ge 0$ (or equivalently $\theta^+ \le \theta^-$).
    \item Alternatively, prove the Penrose inequality only for the \emph{outermost MOTS} (where stability provides $[H] \ge 0$), acknowledging that the ``any trapped surface'' claim was premature.
\end{enumerate}
\end{fix}

\subsection{Corrected Statement}

The corrected lemma should read:

\begin{lemma}[Mean Curvature Jump for Future Trapped Surfaces --- Corrected]
Let $\Sigma_0$ be a closed future trapped surface satisfying $\theta^+ \le 0$ and $\theta^- < 0$. Then the mean curvature jump in the Jang metric satisfies:
\[
    [H]_{\bar{g}} = \tr_{\Sigma_0} k.
\]
The sign of $[H]$ is \textbf{not determined} by the trapped conditions alone. We have:
\begin{itemize}
    \item $[H] > 0$ iff $\tr_{\Sigma_0} k > 0$ iff $\theta^+ < \theta^-$,
    \item $[H] = 0$ iff $\tr_{\Sigma_0} k = 0$ iff $\theta^+ = \theta^-$,
    \item $[H] < 0$ iff $\tr_{\Sigma_0} k < 0$ iff $\theta^+ > \theta^-$.
\end{itemize}
For corner smoothing to preserve $R \ge 0$, we require the additional hypothesis $\tr_{\Sigma_0} k \ge 0$.
\end{lemma}

%==============================================================================
\section{Critical Issue 2: The Definition of $\theta^-$}
%==============================================================================

\begin{important}
The referee correctly notes that the conventions for $\theta^\pm$ need explicit clarification and consistency checking.
\end{important}

\subsection{Current Conventions (Stated in Remark~\ref{rem:SignConventionsSummary})}

The paper uses:
\begin{align}
    \theta^+ &= H_\Sigma + \tr_\Sigma k \quad \text{(outward/future null expansion)}, \\
    \theta^- &= H_\Sigma - \tr_\Sigma k \quad \text{(inward/past null expansion)}.
\end{align}

\subsection{Standard Conventions in the Literature}

With $u$ the future-directed unit normal to the Cauchy slice $M$ and $\nu$ the outward spacelike unit normal to $\Sigma$ within $M$, the null normals are:
\begin{align}
    \ell^+ &= u + \nu \quad \text{(future-outgoing)}, \\
    \ell^- &= u - \nu \quad \text{(future-ingoing)}.
\end{align}
The null expansions are $\theta^\pm = \text{div}_\Sigma(\ell^\pm)$. Under the standard convention where $H = \text{div}_\Sigma(\nu)$ (positive for outward-convex surfaces), we have:
\[
    \theta^\pm = H \pm \tr_\Sigma k.
\]
This matches the paper's convention.

\begin{fix}
\textbf{Action Required:}
\begin{enumerate}
    \item Add an explicit ``Conventions'' subsection (Section 1.5 or earlier) defining:
    \begin{itemize}
        \item The unit normals $u$ (future timelike) and $\nu$ (outward spacelike).
        \item The null normals $\ell^\pm = u \pm \nu$.
        \item The mean curvature $H = \text{div}_\Sigma \nu$ with sign (sphere has $H > 0$).
        \item The extrinsic curvature $k_{ij}$ with sign convention.
        \item The null expansions $\theta^\pm = H \pm \tr_\Sigma k$.
    \end{itemize}
    \item Verify every equation involving $\theta^\pm$ against these definitions.
    \item State explicitly: ``A surface is \emph{future trapped} if $\theta^+ \le 0$ and $\theta^- \le 0$; it is \emph{strictly future trapped} if both inequalities are strict.''
\end{enumerate}
\end{fix}

%==============================================================================
\section{Critical Issue 3: The ``Any Trapped Surface'' Claim}
%==============================================================================

\begin{critical}
The claim that the Penrose inequality holds for \textbf{any} closed trapped surface is extraordinary and, given the sign error in Issue 1, is \textbf{not established} by the current proof.
\end{critical}

\subsection{What is Actually Proved}

With the sign error corrected, the paper proves the Penrose inequality for trapped surfaces satisfying the \emph{additional} condition $\tr_{\Sigma_0} k \ge 0$. This includes:
\begin{itemize}
    \item \textbf{Stable outermost MOTS}: These have $\theta^+ = 0$, and stability implies $[H] \ge 0$ via a different mechanism (spectral positivity of $L_\Sigma$).
    \item \textbf{Surfaces with $\theta^+ \le \theta^-$}: This is equivalent to $\tr_\Sigma k \ge 0$.
\end{itemize}

However, it does \emph{not} include:
\begin{itemize}
    \item \textbf{Interior trapped surfaces with $\tr_\Sigma k < 0$}: These can have $[H] < 0$, breaking the smoothing argument.
    \item \textbf{``Past-dominant'' surfaces}: Surfaces where $\theta^-$ is more negative than $\theta^+$.
\end{itemize}

\subsection{Physical Interpretation}

The condition $\tr_\Sigma k \ge 0$ has a physical meaning: it says the surface is not ``past-dominated'' in the sense that the inward contraction (related to $\theta^-$) is not too strong relative to the outward expansion (related to $\theta^+$).

In many physical situations (e.g., quasistationary black holes, surfaces near apparent horizons), we expect $\tr_\Sigma k \ge 0$. But in highly dynamical situations (e.g., late stages of binary mergers), inner surfaces can have $\tr_\Sigma k < 0$.

\begin{fix}
\textbf{Options for the Revised Paper:}

\textbf{Option A (Recommended):} State the theorem with the additional hypothesis:
\begin{theorem}[Spacetime Penrose Inequality --- Corrected]
Let $(M^3, g, k)$ be AF initial data satisfying DEC. Let $\Sigma_0$ be a closed future trapped surface with $\theta^+ \le 0$, $\theta^- < 0$, \textbf{and} $\tr_{\Sigma_0} k \ge 0$. Then:
\[
    M_{\ADM}(g) \ge \sqrt{\frac{A(\Sigma_0)}{16\pi}}.
\]
\end{theorem}
This is still a significant result but does not claim ``any'' trapped surface.

\textbf{Option B:} Prove the inequality only for the \emph{outermost stable MOTS} (apparent horizon), where stability provides the required sign condition. This matches the scope of Bray--Khuri and Han--Khuri.

\textbf{Option C:} Investigate whether a different approach (not relying on $[H] \ge 0$) can handle arbitrary trapped surfaces. This would require new ideas beyond the current framework.
\end{fix}

%==============================================================================
\section{Issue 4: The $C_0$ Coefficient---Constant vs.\ Function}
%==============================================================================

\begin{important}
The referee correctly identifies ambiguity in whether $C_0 = |\theta^-|/2$ is constant or varies over $\Sigma_0$.
\end{important}

\subsection{The Issue}

The paper states (Theorem~\ref{thm:DirectTrappedJang}):
\[
    f(s, y) = C_0 \ln(s^{-1}) + A(y) + O(s^\alpha),
\]
with $C_0 = |\theta^-|/2 > 0$. But $\theta^-(y) = H_{\Sigma_0}(y) - \tr_{\Sigma_0} k(y)$ depends on $y \in \Sigma_0$, so $C_0$ should be $C_0(y)$.

\subsection{Correct Statement}

The asymptotic expansion should read:
\[
    f(s, y) = C_0(y) \ln(s^{-1}) + A(y) + O(s^\alpha),
\]
where:
\[
    C_0(y) = \frac{|\theta^-(y)|}{2} = \frac{|H_{\Sigma_0}(y) - \tr_{\Sigma_0} k(y)|}{2} > 0 \quad \text{for all } y \in \Sigma_0.
\]
The condition $\theta^- < 0$ on all of $\Sigma_0$ ensures $C_0^{\min} := \inf_{y} C_0(y) > 0$ (since $\Sigma_0$ is compact and $\theta^-$ is continuous).

\begin{fix}
\textbf{Actions:}
\begin{enumerate}
    \item Replace all instances of ``$C_0 = |\theta^-|/2$'' with ``$C_0(y) = |\theta^-(y)|/2$'' or clarify when we mean $C_0^{\min}$, $C_0^{\max}$, or a representative value.
    \item In the barrier construction, use $C_0^{\min} = \inf_{\Sigma_0} C_0(y)$ for the subsolution.
    \item In asymptotic analysis, note that the $y$-dependence of $C_0(y)$ creates lower-order corrections in the cylindrical metric.
\end{enumerate}
\end{fix}

%==============================================================================
\section{Issue 5: Decay Rate Inconsistencies}
%==============================================================================

\begin{important}
The referee notes inconsistencies between $\tau > 1/2$, $\tau > 1$, and borderline $\tau \in (1/2, 1]$.
\end{important}

\subsection{Current State}

The paper mentions:
\begin{itemize}
    \item Abstract: $\tau > 1/2$ (broadest claim).
    \item Main theorem: $\tau > 1$ with full derivative bounds.
    \item Borderline extension: $\tau \in (1/2, 1]$ with additional assumptions.
\end{itemize}

\begin{fix}
\textbf{Recommended Organization:}

\textbf{Primary theorem (Theorem~\ref{thm:MainTheorem}):} State with $\tau > 1$, the standard regime where all classical estimates apply directly.

\textbf{Corollary/Extension (Theorem~\ref{thm:PenroseBorderline}):} State the borderline case $\tau \in (1/2, 1]$ as a separate result with explicit additional hypotheses:
\begin{itemize}
    \item Harmonic coordinate gauge existence.
    \item ADM mass defined via the coefficient in $g_{ij} = \delta_{ij} + \frac{2M}{r}\delta_{ij} + O(r^{-1-\epsilon})$.
    \item Possibly: RT coordinates or Bartnik--Chru\'sciel mass definition.
\end{itemize}

\textbf{Abstract:} Change ``$\tau > 1/2$'' to ``$\tau > 1$ (with extension to $\tau > 1/2$ under additional assumptions)''.
\end{fix}

%==============================================================================
\section{Issue 6: Double Limit and Low-Regularity Analysis}
%==============================================================================

The referee acknowledges the difficulty of the $(p, \epsilon) \to (1^+, 0)$ limit but requests clearer organization.

\begin{fix}
\textbf{Actions:}
\begin{enumerate}
    \item Create a standalone section (or appendix) titled ``Analytic Prerequisites'' listing:
    \begin{itemize}
        \item Tolksdorf regularity for $p$-harmonic functions.
        \item Mosco convergence framework.
        \item Distributional curvature for Lipschitz metrics.
        \item Capacity estimates for conical singularities.
    \end{itemize}
    \item For each lemma, state: (a) precise hypotheses, (b) conclusion, (c) reference or proof location.
    \item Reduce reliance on ``well-known'' claims; provide explicit references or self-contained arguments.
\end{enumerate}
\end{fix}

%==============================================================================
\section{Structural Recommendations}
%==============================================================================

\subsection{Add a Conventions Section}

Insert after the Introduction (Section 1.1 or 1.2):

\begin{quote}
\textbf{Section 1.X: Conventions}

\textbf{Geometric setup:} Let $(N^{3+1}, \bar{g})$ be a spacetime with Cauchy hypersurface $(M^3, g)$. Let $u$ be the future-directed unit normal to $M$. The extrinsic curvature of $M$ in $N$ is $k_{ij} = \bar{g}(\bar{\nabla}_{\partial_i} u, \partial_j)$.

\textbf{Null normals:} For a 2-surface $\Sigma \subset M$ with outward unit normal $\nu$, define $\ell^\pm = u \pm \nu$.

\textbf{Mean curvature:} $H_\Sigma = \text{div}_\Sigma \nu = \sum_a \langle \nabla_{e_a} \nu, e_a \rangle$ where $\{e_a\}$ is an orthonormal frame on $\Sigma$. A round sphere in $\mathbb{R}^3$ has $H > 0$.

\textbf{Null expansions:} $\theta^\pm = H_\Sigma \pm \tr_\Sigma k$.

\textbf{Trapped surfaces:} A surface is \emph{future trapped} if $\theta^+ \le 0$ and $\theta^- \le 0$. It is \emph{strictly future trapped} if both are negative. A \emph{MOTS} has $\theta^+ = 0$.

\textbf{Mean curvature jump:} At an interface $\Sigma$ with regions $\Omega^\pm$, $[H] = H^+ - H^-$ where $H^\pm$ are computed with normals pointing into $\Omega^\pm$.
\end{quote}

\subsection{Clarify the Scope of the Main Theorem}

The Introduction should clearly state:
\begin{enumerate}
    \item The theorem applies to trapped surfaces satisfying $\theta^+ \le 0$, $\theta^- < 0$, \textbf{and} $\tr_{\Sigma_0} k \ge 0$.
    \item The condition $\tr_\Sigma k \ge 0$ is \emph{not} automatic from trapping.
    \item For stable MOTS, the condition is automatic via spectral positivity.
    \item The ``any trapped surface'' claim in the original abstract was overstated.
\end{enumerate}

\subsection{Rigidity Statement Clarification}

Add to the rigidity theorem (Theorem~\ref{thm:MainB}):
\begin{quote}
``The equality case $M_{\ADM} = \sqrt{A(\Sigma)/(16\pi)}$ forces $\Sigma$ to be the \emph{unique} horizon component. If the original trapped surface $\Sigma_0$ used in the construction was not the outermost MOTS, then equality cannot hold unless $\Sigma_0$ coincides with the apparent horizon.''
\end{quote}

%==============================================================================
\section{Summary of Required Changes}
%==============================================================================

\begin{enumerate}
    \item \textbf{[CRITICAL]} Correct the sign analysis in Lemma~\ref{lem:TrappedMeanCurvatureJump}. Add hypothesis $\tr_{\Sigma_0} k \ge 0$ or restrict to stable MOTS.
    
    \item \textbf{[CRITICAL]} Update abstract, introduction, and all instances of ``any trapped surface'' to reflect the corrected scope.
    
    \item \textbf{[IMPORTANT]} Add explicit Conventions section defining $H$, $\nu$, $\theta^\pm$, trapped surfaces.
    
    \item \textbf{[IMPORTANT]} Fix $C_0$ to be $C_0(y)$ throughout, clarifying when we use $C_0^{\min}$ or $C_0^{\max}$.
    
    \item \textbf{[IMPORTANT]} Consolidate decay rate assumptions: primary theorem with $\tau > 1$, extension with $\tau > 1/2$.
    
    \item \textbf{[RECOMMENDED]} Create ``Analytic Prerequisites'' section/appendix with clear statement of all technical lemmas.
    
    \item \textbf{[RECOMMENDED]} Clarify rigidity statement regarding which surface achieves equality.
    
    \item \textbf{[RECOMMENDED]} Consider splitting the paper into Core Result + Analytic Machinery.
\end{enumerate}

%==============================================================================
\section{Changes Implemented}
%==============================================================================

The following corrections have been made to \texttt{paper.tex}:

\begin{enumerate}
    \item \textbf{Sign/Logic Error (CRITICAL):}
    \begin{itemize}
        \item Rewrote Lemma~\ref{lem:TrappedMeanCurvatureJump} to correctly state that $[H] = \tr_{\Sigma_0} k$ without claiming positivity from trapped conditions.
        \item Added explicit counterexample: $H = -3$, $\tr k = -1$ gives $\theta^+ = -4$, $\theta^- = -2$, but $\tr k = -1 < 0$.
        \item For stable MOTS, referenced Theorem~\ref{thm:CompleteMeanCurvatureJump} for the favorable jump via spectral positivity.
    \end{itemize}
    
    \item \textbf{Main Theorem (Theorem~\ref{thm:MainTheorem}):}
    \begin{itemize}
        \item Added hypothesis: ``$\Sigma$ satisfies the \textbf{favorable jump condition} $\tr_\Sigma k \ge 0$''.
        \item Clarified this is automatic for stable MOTS and for surfaces with $\theta^+ \le \theta^-$.
    \end{itemize}
    
    \item \textbf{Abstract:}
    \begin{itemize}
        \item Changed scope from ``any closed trapped surface'' to ``closed trapped surface $\Sigma$ satisfying the favorable jump condition $\tr_\Sigma k \ge 0$''.
        \item Added explanation that this is automatic for stable MOTS.
    \end{itemize}
    
    \item \textbf{Conventions Section:}
    \begin{itemize}
        \item Added new \S1.X ``Sign conventions and definitions'' after Introduction.
        \item Explicit definitions of $H$, $\nu$, $\ell^\pm$, $\theta^\pm$, trapped surfaces, favorable jump.
        \item Includes the counterexample showing trapping alone doesn't imply $\tr k \ge 0$.
    \end{itemize}
    
    \item \textbf{Theorem~\ref{thm:DirectTrappedJang}:}
    \begin{itemize}
        \item Added favorable jump condition $\tr_{\Sigma_0} k \ge 0$ to hypotheses.
        \item Changed part (iv) to state $[H] = \tr_{\Sigma_0} k \ge 0$ is \emph{assumed}, not derived.
    \end{itemize}
    
    \item \textbf{$C_0$ Notation:}
    \begin{itemize}
        \item Changed $C_0 = |\theta^-|/2$ to $C_0(y) = |\theta^-(y)|/2$ in key locations.
        \item Added note clarifying $C_0^{\min}$ is used for barrier arguments.
    \end{itemize}
    
    \item \textbf{Decay Assumptions:}
    \begin{itemize}
        \item Comparison table updated to show $\tau > 1$ with footnote for extension to $\tau > 1/2$.
        \item Abstract already stated ``$\tau > 1$ (with extension to $\tau > 1/2$ under additional assumptions)''.
    \end{itemize}
    
    \item \textbf{``Directly Implies'' Claims:}
    \begin{itemize}
        \item Fixed approximately 15 locations where ``trapped condition directly implies $[H] \ge 0$'' was incorrectly stated.
        \item Now correctly state that favorable jump is either a hypothesis or follows from stability for MOTS.
    \end{itemize}
    
    \item \textbf{Rigidity Clarification (Theorem~\ref{thm:MainB}):}
    \begin{itemize}
        \item Clarified that equality forces the original trapped surface $\Sigma_0$ to coincide with the unique outermost MOTS.
        \item Added explicit statement: interior trapped surfaces must satisfy strict inequality.
    \end{itemize}
    
    \item \textbf{Core Proof Summary (\S\ref{subsec:CoreProofSummary}):}
    \begin{itemize}
        \item Added new self-contained 3-page summary of the complete proof.
        \item All seven steps numbered with explicit theorem/section references.
        \item Boxed format for easy reference by specialists.
    \end{itemize}
    
    \item \textbf{Removed ``well-known'' phrasing:}
    \begin{itemize}
        \item Replaced ``well-known formula'' with explicit references (Wald, Chandrasekhar) for the Kerr horizon area formula.
    \end{itemize}
\end{enumerate}

%==============================================================================
\section{Conclusion}
%==============================================================================

We are grateful to the referee for identifying the critical sign error, which would have invalidated the main claims. The corrected theorem, while more restrictive than originally claimed, still represents a significant advance:

\begin{itemize}
    \item For \textbf{stable MOTS} (apparent horizons), the Penrose inequality is proved unconditionally---this was the main open problem.
    \item For \textbf{trapped surfaces with $\tr_\Sigma k \ge 0$}, the inequality is established.
    \item The ``any trapped surface'' claim was overstated and has been withdrawn.
\end{itemize}

The revised paper will clearly delineate what is proved and under which hypotheses.

\end{document}
