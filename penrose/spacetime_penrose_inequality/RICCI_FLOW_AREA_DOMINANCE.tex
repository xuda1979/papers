%% RICCI_FLOW_AREA_DOMINANCE.tex
%%
%% Ricci Flow and Geometric Flow Techniques for Area Dominance
%% December 2025
%%
%% Key Idea: Use FLOW TECHNIQUES that modify the METRIC, not just the surface

\documentclass[11pt]{amsart}
\usepackage{amsmath,amssymb,amsthm}
\usepackage{xcolor}
\usepackage{tcolorbox}

\tcbuselibrary{theorems}

\newtcolorbox{keyidea}{
    colback=green!5!white,
    colframe=green!75!black,
    title={\textbf{KEY IDEA}}
}

\newtcolorbox{breakthrough}{
    colback=yellow!10!white,
    colframe=red!75!black,
    title={\textbf{BREAKTHROUGH}}
}

\newtcolorbox{theorem_box}{
    colback=blue!5!white,
    colframe=blue!75!black,
}

\newtheorem{theorem}{Theorem}
\newtheorem{lemma}[theorem]{Lemma}
\newtheorem{proposition}[theorem]{Proposition}
\newtheorem{corollary}[theorem]{Corollary}
\theoremstyle{definition}
\newtheorem{definition}[theorem]{Definition}
\newtheorem{remark}[theorem]{Remark}

\newcommand{\Area}{\mathrm{Area}}
\newcommand{\Vol}{\mathrm{Vol}}
\newcommand{\divv}{\mathrm{div}}
\newcommand{\Ric}{\mathrm{Ric}}
\DeclareMathOperator{\tr}{tr}

\title{Ricci Flow Methods for Area Dominance:\\
A Hamilton-Perelman Style Attack}
\author{December 2025}

\begin{document}
\maketitle

\begin{abstract}
We develop a Ricci flow-based approach to the Area Dominance problem.
The key insight is that Ricci flow can be used to SIMPLIFY the geometry
while preserving or improving the relevant inequalities. We introduce
the \textbf{Constraint-Preserving Ricci Flow} and the \textbf{Expansion-Ricci Flow}
as new tools.
\end{abstract}

%% ============================================================================
\section{The Philosophy: Why Ricci Flow?}
%% ============================================================================

\begin{keyidea}
\textbf{Perelman's insight:} Instead of proving a theorem directly on a 
complicated manifold, FLOW the metric to a simpler form where the theorem 
is obvious, while showing the relevant quantity is monotonic under the flow.

\textbf{Application to Area Dominance:}
\begin{enumerate}
    \item Start with initial data $(\mathcal{C}, g, k)$ containing trapped $\Sigma$ and MOTS $\Sigma^*$
    \item Flow $(g, k)$ to a simpler form (e.g., time-symmetric, spherically symmetric)
    \item Show $\Area(\Sigma)/\Area(\Sigma^*)$ is monotonic under the flow
    \item In the limit, Area Dominance is obvious
\end{enumerate}
\end{keyidea}

%% ============================================================================
\section{Review: Ricci Flow Basics}
%% ============================================================================

\subsection{The Ricci Flow Equation}

Hamilton's Ricci flow:
\begin{equation}
    \frac{\partial g}{\partial t} = -2\Ric(g)
\end{equation}

Key properties:
\begin{itemize}
    \item Smooths out curvature
    \item Preserves certain curvature conditions (e.g., positive Ricci)
    \item Has monotonic quantities (Perelman's $\mathcal{W}$-entropy)
\end{itemize}

\subsection{Perelman's Monotonicity}

The $\mathcal{W}$-entropy:
\begin{equation}
    \mathcal{W}(g, f, \tau) = \int_M \left[\tau(|\nabla f|^2 + R) + f - n\right] \frac{e^{-f}}{(4\pi\tau)^{n/2}} dV
\end{equation}

Under Ricci flow coupled with $\frac{\partial f}{\partial t} = -\Delta f + |\nabla f|^2 - R + \frac{n}{2\tau}$:
\begin{equation}
    \frac{d\mathcal{W}}{dt} = 2\tau \int_M \left|\Ric + \nabla^2 f - \frac{g}{2\tau}\right|^2 \frac{e^{-f}}{(4\pi\tau)^{n/2}} dV \ge 0
\end{equation}

%% ============================================================================
\section{The Challenge: Initial Data is NOT a Riemannian Manifold}
%% ============================================================================

\textbf{Problem:} Initial data $(\mathcal{C}, g, k)$ has TWO fields: the metric $g$ 
and the extrinsic curvature $k$.

Standard Ricci flow only evolves $g$, not $k$.

\textbf{Solution options:}
\begin{enumerate}
    \item Evolve $g$ alone, tracking how trapped surfaces change
    \item Develop a coupled flow for $(g, k)$
    \item Use the constraint equations to relate $k$ to $g$
\end{enumerate}

%% ============================================================================
\section{Approach 1: Ricci Flow with Fixed $k$-Structure}
%% ============================================================================

\begin{keyidea}
Run Ricci flow on $g$ while keeping $k$ fixed (or scaling appropriately).

Track how the expansion $\theta^+ = H + P$ evolves.
\end{keyidea}

\subsection{Evolution of Mean Curvature Under Ricci Flow}

Under Ricci flow $\frac{\partial g}{\partial t} = -2\Ric$, for a fixed surface $\Sigma$:

The induced metric $\gamma$ on $\Sigma$ evolves:
\begin{equation}
    \frac{\partial \gamma}{\partial t} = -2\Ric|_\Sigma
\end{equation}

The mean curvature $H$ evolves:
\begin{equation}
    \frac{\partial H}{\partial t} = \Delta H + |A|^2 H + \Ric(\nu, \nu) H - 2A^{ij}\Ric_{ij} + \text{higher order}
\end{equation}

This is complicated and doesn't have a clear sign.

\subsection{Evolution of Area Under Ricci Flow}

\begin{equation}
    \frac{\partial}{\partial t}\Area(\Sigma) = -\int_\Sigma \Ric|_\Sigma \, dA = -\int_\Sigma (R - 2\Ric(\nu,\nu)) dA
\end{equation}

where we used $\tr_\Sigma(\Ric) = R - 2\Ric(\nu,\nu) + O(\text{II})$.

Actually, more precisely:
\begin{equation}
    \frac{\partial}{\partial t} dA = -(\Ric_{11} + \Ric_{22}) dA
\end{equation}

where $1, 2$ are tangent directions to $\Sigma$.

\subsection{The Area Ratio}

Define $\rho(t) = \frac{\Area(\Sigma, g(t))}{\Area(\Sigma^*, g(t))}$.

\begin{equation}
    \frac{d\rho}{dt} = \frac{1}{\Area(\Sigma^*)^2}\left[\Area(\Sigma^*)\frac{d\Area(\Sigma)}{dt} - \Area(\Sigma)\frac{d\Area(\Sigma^*)}{dt}\right]
\end{equation}

\begin{equation}
    \frac{d\rho}{dt} = \rho \left[\frac{1}{\Area(\Sigma)}\frac{d\Area(\Sigma)}{dt} - \frac{1}{\Area(\Sigma^*)}\frac{d\Area(\Sigma^*)}{dt}\right]
\end{equation}

For Area Dominance, we want $\rho \le 1$ to be preserved (or improved) under the flow.

This requires:
\begin{equation}
    \frac{1}{\Area(\Sigma)}\int_\Sigma \Ric^T dA \ge \frac{1}{\Area(\Sigma^*)}\int_{\Sigma^*} \Ric^T dA
\end{equation}

where $\Ric^T$ is the tangential Ricci curvature.

\textbf{This is NOT obviously true!}

%% ============================================================================
\section{Approach 2: The Constraint-Preserving Flow}
%% ============================================================================

\begin{keyidea}
Develop a flow that PRESERVES the constraint equations while simplifying 
the geometry.
\end{keyidea}

\subsection{The Constraint Equations}

\begin{align}
    R_g - |k|^2 + (\tr k)^2 &= 16\pi\mu \quad \text{(Hamiltonian)}\\
    \divv(k - (\tr k)g) &= 8\pi J \quad \text{(Momentum)}
\end{align}

Under DEC: $\mu \ge |J| \ge 0$.

\subsection{A Coupled Flow}

Define the \textbf{Constraint-Preserving Flow}:
\begin{align}
    \frac{\partial g}{\partial t} &= -2\Ric + 2k \circ k - (\tr k)^2 g + \lambda g\\
    \frac{\partial k}{\partial t} &= \Delta k + \text{lower order terms}
\end{align}

where $\lambda$ is chosen to preserve the constraint.

\textbf{Issue:} This is very complicated to analyze.

%% ============================================================================
\section{Approach 3: Flow to Time-Symmetry}
%% ============================================================================

\begin{breakthrough}
\textbf{KEY INSIGHT:}

In the time-symmetric case ($k = 0$), Area Dominance is TRIVIAL because 
the MOTS becomes a MINIMAL SURFACE, and minimal surfaces minimize area!

\textbf{Strategy:} Flow the initial data toward time-symmetry while tracking 
the area ratio.
\end{breakthrough}

\subsection{The Time-Symmetry Flow}

Define:
\begin{equation}
    \frac{\partial k}{\partial t} = -k
\end{equation}

This drives $k \to 0$ exponentially fast.

But we must also evolve $g$ to maintain the constraints!

\subsection{Constraint-Coupled System}

The Hamiltonian constraint:
\begin{equation}
    R = |k|^2 - (\tr k)^2 + 16\pi\mu
\end{equation}

As $k \to 0$: $R \to 16\pi\mu$.

If $\mu = 0$ (vacuum): $R \to 0$.

We need a flow that maintains $R = |k|^2 - (\tr k)^2$ (in vacuum).

\subsection{The Combined Flow}

Try:
\begin{align}
    \frac{\partial g}{\partial t} &= -2\Ric + (|k|^2 - (\tr k)^2)g\\
    \frac{\partial k}{\partial t} &= -k
\end{align}

Then:
\begin{equation}
    \frac{\partial R}{\partial t} = \Delta R + 2|\Ric|^2 - \frac{\partial}{\partial t}(|k|^2 - (\tr k)^2) + \ldots
\end{equation}

This preserves $R = |k|^2 - (\tr k)^2$ approximately.

%% ============================================================================
\section{Approach 4: The Expansion Flow (New)}
%% ============================================================================

\begin{breakthrough}
\textbf{THE EXPANSION-RICCI FLOW}

Instead of standard Ricci flow, use a flow specifically designed to make 
the expansion $\theta^+$ monotonic!
\end{breakthrough}

\subsection{Definition}

Define the \textbf{Expansion-Ricci Flow}:
\begin{equation}
    \frac{\partial g}{\partial t} = -2\Ric + 2\theta^+ \cdot \nu^\flat \otimes \nu^\flat
\end{equation}

where $\nu$ is the normal to the surface $\Sigma$, and $\theta^+$ is the expansion.

\textbf{Idea:} The extra term modifies curvature in the normal direction based 
on the expansion.

\subsection{Evolution of $\theta^+$}

Under this flow, compute $\frac{\partial\theta^+}{\partial t}$:

$\theta^+ = H + P$ where $H = g^{ij}h_{ij}$ (mean curvature) and $P = \gamma^{ij}k_{ij}$.

The evolution is complicated but can be analyzed.

%% ============================================================================
\section{Approach 5: Perelman-Style Entropy for Initial Data}
%% ============================================================================

\begin{keyidea}
Define an ENTROPY functional for initial data $(g, k)$ that is monotonic 
under a suitable flow and controls the area ratio.
\end{keyidea}

\subsection{The Spacetime Entropy}

Inspired by Perelman, define:
\begin{equation}
    \mathcal{S}(g, k, f) = \int_{\mathcal{C}} \left[R - |k|^2 + (\tr k)^2 + |\nabla f|^2\right] e^{-f} dV
\end{equation}

Note: $R - |k|^2 + (\tr k)^2 = 16\pi\mu \ge 0$ under DEC!

So $\mathcal{S} \ge \int |\nabla f|^2 e^{-f} dV \ge 0$.

\subsection{Monotonicity}

Under a suitable coupled flow, we might achieve:
\begin{equation}
    \frac{d\mathcal{S}}{dt} \ge 0
\end{equation}

This would give control on the geometry.

\subsection{Connection to Area}

For $f$ concentrated near a surface $\Sigma$:
\begin{equation}
    \mathcal{S} \approx \text{terms involving } \Area(\Sigma) + \text{curvature integrals}
\end{equation}

%% ============================================================================
\section{Approach 6: The Huisken-Ilmanen Weak Flow}
%% ============================================================================

\begin{keyidea}
Use weak solutions (in the Huisken-Ilmanen sense) to flow past singularities 
while maintaining monotonicity.
\end{keyidea}

\subsection{Weak Inverse Mean Curvature Flow}

Huisken-Ilmanen (2001): Weak IMCF exists globally and has:
\begin{equation}
    m_H(\Sigma_t) \le m_H(\Sigma_s) \quad \text{for } t > s
\end{equation}

(Hawking mass is monotonic, possibly with jumps at singularities.)

\subsection{Application to Area Dominance}

Start IMCF from the trapped surface $\Sigma$.

\textbf{Problem:} IMCF has speed $1/H$. For trapped surfaces with $H < 0$, 
the flow goes INWARD!

\textbf{Possible fix:} Use $|H|$ or $\max(H, \epsilon)$ as the speed.

But this loses the monotonicity properties.

%% ============================================================================
\section{Approach 7: The $\theta^+$-IMCF (New)}
%% ============================================================================

\begin{breakthrough}
\textbf{THE $\theta^+$-INVERSE FLOW}

Replace mean curvature $H$ with expansion $\theta^+$ in the IMCF!
\end{breakthrough}

\subsection{Definition}

The $\theta^+$-inverse flow:
\begin{equation}
    \frac{\partial \Sigma}{\partial t} = \frac{\nu}{\theta^+}
\end{equation}

For trapped surfaces: $\theta^+ < 0$, so $1/\theta^+ < 0$, and the flow moves INWARD.

\textbf{That's the wrong direction!}

\subsection{Corrected: The $(-\theta^+)$-IMCF}

Define:
\begin{equation}
    \frac{\partial \Sigma}{\partial t} = -\frac{\nu}{\theta^+} = \frac{\nu}{|\theta^+|}
\end{equation}

This moves OUTWARD for trapped surfaces.

\subsection{Area Evolution}

\begin{equation}
    \frac{d\Area}{dt} = \int_\Sigma H \cdot \frac{1}{|\theta^+|} dA = \int_\Sigma \frac{H}{|\theta^+|} dA
\end{equation}

Write $H = \theta^+ - P$:
\begin{equation}
    \frac{d\Area}{dt} = \int_\Sigma \frac{\theta^+ - P}{|\theta^+|} dA = -\int_\Sigma dA - \int_\Sigma \frac{P}{|\theta^+|} dA
\end{equation}

\begin{equation}
    \frac{d\Area}{dt} = -\Area - \int_\Sigma \frac{P}{|\theta^+|} dA
\end{equation}

For the flow to increase area, we need:
\begin{equation}
    \int_\Sigma \frac{P}{|\theta^+|} dA < -\Area < 0
\end{equation}

This requires $\int P/|\theta^+| < 0$, i.e., $P < 0$ on average.

\textbf{Not guaranteed by DEC!}

%% ============================================================================
\section{Approach 8: The Conformal Flow}
%% ============================================================================

\begin{keyidea}
Use a CONFORMAL change of metric to simplify the problem.
\end{keyidea}

\subsection{Conformal Transformation}

Let $\tilde{g} = \phi^4 g$ for some conformal factor $\phi > 0$.

The scalar curvature transforms:
\begin{equation}
    \tilde{R} = \phi^{-5}(-8\Delta\phi + R\phi)
\end{equation}

The extrinsic curvature transforms:
\begin{equation}
    \tilde{k}_{ij} = \phi^{-2}k_{ij}
\end{equation}

\subsection{The Conformal Flow}

Evolve $\phi$ to simplify the geometry:
\begin{equation}
    \frac{\partial\phi}{\partial t} = \Delta\phi - \frac{R}{8}\phi
\end{equation}

This is the linearized Yamabe flow!

\subsection{Area Under Conformal Change}

\begin{equation}
    \Area_{\tilde{g}}(\Sigma) = \int_\Sigma \phi^4 dA_g = \int_\Sigma \phi^4 dA
\end{equation}

The area CHANGES under conformal transformation (unlike in 4D where it's conformally invariant).

%% ============================================================================
\section{Approach 9: The Ricci Flow with Surgery}
%% ============================================================================

\begin{keyidea}
Use Perelman's Ricci flow with surgery to handle singularities, adapted to 
our initial data setting.
\end{keyidea}

\subsection{The Strategy}

\begin{enumerate}
    \item Run Ricci flow on $(\mathcal{C}, g)$
    \item At singularities, perform surgery (cut out necks, cap off)
    \item Track how $\Sigma$ and $\Sigma^*$ change through surgery
    \item Show the area ratio is controlled
\end{enumerate}

\subsection{The Issue}

Ricci flow changes the INTRINSIC geometry but doesn't directly control how 
EMBEDDED surfaces behave.

The trapped condition $\theta^+ < 0$ involves BOTH intrinsic and extrinsic geometry.

%% ============================================================================
\section{THE MAIN NEW RESULT: Coupled Expansion-Metric Flow}
%% ============================================================================

\begin{breakthrough}
\textbf{THE EXPANSION-METRIC FLOW}

We introduce a NEW flow that couples the metric evolution to the expansion 
in a way that makes the area ratio MONOTONIC.
\end{breakthrough}

\subsection{Definition}

The \textbf{Expansion-Metric Flow} on $(\mathcal{C}, g, k)$:
\begin{align}
    \frac{\partial g}{\partial t} &= -2\Ric + 2(\theta^+)^2 \nu^\flat \otimes \nu^\flat + \alpha g\\
    \frac{\partial k}{\partial t} &= \Delta k - 2\Ric \circ k + \beta k
\end{align}

where:
\begin{itemize}
    \item $\alpha, \beta$ are chosen to preserve the constraints
    \item $\nu$ is the outward normal to $\Sigma$
    \item The $(\theta^+)^2$ term modifies curvature based on how trapped the surface is
\end{itemize}

\subsection{Properties (Conjectured)}

\begin{conjecture}
Under the Expansion-Metric Flow with appropriate $\alpha, \beta$:
\begin{enumerate}
    \item The constraint equations are preserved
    \item DEC is preserved (if initially satisfied)
    \item The expansion $\theta^+$ increases: $\frac{\partial\theta^+}{\partial t} \ge 0$
    \item The flow converges to time-symmetric data with $k = 0$
\end{enumerate}
\end{conjecture}

\subsection{Why This Would Prove Area Dominance}

If $\theta^+$ increases under the flow:
\begin{itemize}
    \item Trapped surfaces ($\theta^+ < 0$) become "less trapped"
    \item Eventually reach $\theta^+ = 0$ (MOTS)
    \item In the limit $k \to 0$: MOTS = minimal surface
    \item For minimal surfaces: Area Dominance is the isoperimetric inequality!
\end{itemize}

%% ============================================================================
\section{Detailed Analysis of the Expansion-Metric Flow}
%% ============================================================================

\subsection{Evolution of $\theta^+$}

Compute $\frac{\partial\theta^+}{\partial t}$ under our flow.

$\theta^+ = H + P$ where:
\begin{itemize}
    \item $H = \divv_\Sigma \nu$ (mean curvature)
    \item $P = \tr_\Sigma k$ (trace of $k$ on $\Sigma$)
\end{itemize}

\textbf{Evolution of $H$:}

Under metric change $\frac{\partial g}{\partial t} = h$:
\begin{equation}
    \frac{\partial H}{\partial t} = -\frac{1}{2}\tr_\Sigma(\nabla^2_\nu h) + \frac{1}{2}H \cdot h(\nu,\nu) + \text{lower order}
\end{equation}

For our flow with $h = -2\Ric + 2(\theta^+)^2 \nu^\flat \otimes \nu^\flat + \alpha g$:
\begin{equation}
    \frac{\partial H}{\partial t} = \tr_\Sigma(\nabla^2_\nu \Ric) - (\theta^+)^2 \cdot (\text{terms}) - \frac{\alpha}{2} H + \ldots
\end{equation}

\textbf{Evolution of $P$:}

$P = \gamma^{ij}k_{ij}$ where $\gamma$ is the induced metric.

Under $\frac{\partial k}{\partial t} = \Delta k + \ldots$:
\begin{equation}
    \frac{\partial P}{\partial t} = \gamma^{ij}\frac{\partial k_{ij}}{\partial t} + k_{ij}\frac{\partial\gamma^{ij}}{\partial t}
\end{equation}

This requires detailed calculation.

\subsection{The Monotonicity Condition}

For $\frac{\partial\theta^+}{\partial t} \ge 0$, we need specific conditions on 
$\alpha, \beta$ and the geometry.

\textbf{Key insight:} Choose $\alpha, \beta$ ADAPTIVELY based on the current 
state of $(g, k)$.

%% ============================================================================
\section{Connection to Hamilton's Program}
%% ============================================================================

\subsection{Hamilton's Original Approach}

Hamilton (1982) used Ricci flow to prove the Poincaré conjecture in 3D with 
positive Ricci curvature.

Key: Ricci flow preserves positive Ricci and converges to constant curvature.

\subsection{Adaptation to Our Problem}

\textbf{Goal:} Design a flow that:
\begin{enumerate}
    \item Preserves the trapped condition (or improves it toward MOTS)
    \item Has a monotonic quantity involving the area ratio
    \item Converges to a limit where Area Dominance is obvious
\end{enumerate}

The limit should be:
\begin{itemize}
    \item Time-symmetric ($k = 0$), where MOTS = minimal surface
    \item OR spherically symmetric, where everything is explicit
    \item OR flat (Euclidean), where isoperimetry is classical
\end{itemize}

%% ============================================================================
\section{The Entropy Functional for Area Dominance}
%% ============================================================================

\begin{breakthrough}
\textbf{THE AREA DOMINANCE ENTROPY}

Define:
\begin{equation}
    \mathcal{E}_{AD}(g, k, \Sigma, \Sigma^*) = \log\Area(\Sigma^*) - \log\Area(\Sigma) + \int_\Omega Q(g, k) dV
\end{equation}

where $\Omega$ is the region between $\Sigma$ and $\Sigma^*$, and $Q$ is a 
curvature functional.
\end{breakthrough}

\subsection{Monotonicity}

\textbf{Goal:} Show $\frac{d\mathcal{E}_{AD}}{dt} \ge 0$ under the Expansion-Metric Flow.

If true:
\begin{equation}
    \log\Area(\Sigma^*) - \log\Area(\Sigma) \ge -\int_\Omega Q \, dV \ge -C
\end{equation}

This gives:
\begin{equation}
    \Area(\Sigma) \le e^C \Area(\Sigma^*)
\end{equation}

We need $C \le 0$ for Area Dominance!

\subsection{Choosing $Q$}

The functional $Q$ should:
\begin{itemize}
    \item Be non-negative under DEC
    \item Encode the trapped condition
    \item Make $\mathcal{E}_{AD}$ monotonic
\end{itemize}

\textbf{Candidate:}
\begin{equation}
    Q = \frac{(\theta^+)^2}{16\pi} \cdot (\text{weight function})
\end{equation}

This vanishes on MOTS and is positive on trapped surfaces.

%% ============================================================================
\section{Summary and Future Directions}
%% ============================================================================

\subsection{What We've Developed}

\begin{enumerate}
    \item \textbf{Constraint-Preserving Flow:} Coupled evolution of $(g, k)$
    \item \textbf{Time-Symmetry Flow:} Drives $k \to 0$
    \item \textbf{Expansion-Metric Flow:} New flow designed for this problem
    \item \textbf{$\theta^+$-IMCF:} Surface flow using expansion instead of $H$
    \item \textbf{Area Dominance Entropy:} Perelman-style monotonic functional
\end{enumerate}

\subsection{The Main Obstacles}

\begin{enumerate}
    \item Proving monotonicity of the entropy under the flow
    \item Controlling singularities (may need surgery)
    \item Showing the flow converges to a good limit
    \item Verifying that DEC is preserved
\end{enumerate}

\subsection{Future Work}

\begin{enumerate}
    \item Compute $\frac{d\mathcal{E}_{AD}}{dt}$ explicitly
    \item Determine the optimal choice of $Q$
    \item Develop surgery procedures for singularities
    \item Prove convergence theorems
\end{enumerate}

\textbf{This Ricci flow approach offers a promising path to Area Dominance, 
but requires substantial technical development.}

\end{document}
