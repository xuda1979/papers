% THE DYNAMICAL HORIZON PROOF
%
% Using the area increase law on dynamical horizons
% to prove Penrose without the favorable jump condition.

\documentclass[12pt]{article}
\usepackage{amsmath,amsthm,amssymb}
\usepackage{mathrsfs}
\newtheorem{theorem}{Theorem}
\newtheorem{lemma}{Lemma}
\newtheorem{proposition}{Proposition}
\newtheorem{corollary}{Corollary}
\newtheorem{conjecture}{Conjecture}
\newtheorem{remark}{Remark}
\newtheorem{definition}{Definition}
\newtheorem{problem}{Problem}
\newtheorem{claim}{Claim}
\newtheorem{principle}{Principle}
\newtheorem{insight}{Key Insight}

\begin{document}

\title{The Dynamical Horizon Proof\\of the Spacetime Penrose Inequality}
\author{Mathematical Development}
\date{\today}
\maketitle

\section{Overview}

We present an approach to the Penrose inequality using dynamical horizons 
that DOES NOT require the favorable jump condition.

\textbf{Key ingredients}:
\begin{enumerate}
    \item Dynamical horizon area increase law (proven under NEC)
    \item Bondi mass decay (proven)
    \item Comparison between ADM and Bondi mass (standard)
    \item A "weak cosmic censorship" assumption
\end{enumerate}

\section{Dynamical Horizons}

\subsection{Definition}

\begin{definition}[Dynamical Horizon]
A smooth 3-dimensional submanifold $\mathcal{H}$ of spacetime is a 
\textbf{future outer trapping horizon (FOTH)} if:
\begin{enumerate}
    \item $\mathcal{H}$ is foliated by 2-spheres $S_v$ (labeled by parameter $v$)
    \item Each $S_v$ is a MOTS: $\theta^+[S_v] = 0$
    \item $\theta^-[S_v] < 0$ (trapped in the ingoing direction)
    \item $\mathcal{L}_n \theta^+ < 0$ (the horizon is "moving outward")
\end{enumerate}
\end{definition}

Condition 4 means: if you go slightly inward (in the $-n$ direction), 
$\theta^+$ becomes negative (trapped).

\subsection{The Area Increase Law}

\begin{theorem}[Ashtekar-Krishnan Area Law]
On a future outer trapping horizon $\mathcal{H}$ in a spacetime satisfying NEC:
\[
\frac{dA}{dv} \ge 0
\]
where $A = A(S_v)$ is the area of the MOTS leaves.

More precisely:
\[
\frac{dA}{dv} = \int_{S_v} |\sigma|^2 + 8\pi T_{\mu\nu}\ell^\mu n^\nu \ge 0
\]
where $\sigma$ is the shear and NEC gives $T_{\mu\nu}\ell^\mu n^\nu \ge 0$.
\end{theorem}

\textbf{This is the key!} The area of the dynamical horizon INCREASES toward the future.

\section{The Setup}

Let $(M, g, k)$ be asymptotically flat initial data satisfying DEC with a 
trapped surface $\Sigma_0$.

Let $(V, g_{\mu\nu})$ be the maximal globally hyperbolic development.

\begin{assumption}[Weak Cosmic Censorship]
The development $(V, g_{\mu\nu})$ contains a future outer trapping horizon 
$\mathcal{H}$ such that:
\begin{enumerate}
    \item $\mathcal{H}$ intersects $M$ in a MOTS $\Sigma^*$ enclosing $\Sigma_0$
    \item $\mathcal{H}$ extends to future null infinity $\mathscr{I}^+$
\end{enumerate}
\end{assumption}

This is weaker than full cosmic censorship—we only need the horizon to exist 
and connect to infinity, not that ALL singularities are hidden.

\section{The Chain of Inequalities}

\subsection{Step 1: Inner Surface to MOTS}

Since $\Sigma_0$ is trapped and $\Sigma^* = \mathcal{H} \cap M$ is the outermost 
MOTS enclosing it:
\[
A(\Sigma^*) \ge A(\Sigma_0) \quad \text{???}
\]

\textbf{Wait!} This is exactly what FAILS in the unfavorable case!

Let me reconsider...

\subsection{The Actual Relationship}

We DON'T need $A(\Sigma^*) \ge A(\Sigma_0)$.

What we have:
\begin{itemize}
    \item $\Sigma_0$ is trapped, inside the trapped region
    \item $\Sigma^*$ is the OUTERMOST MOTS on the initial slice
    \item The dynamical horizon $\mathcal{H}$ starts at $\Sigma^*$
\end{itemize}

The dynamical horizon area law gives:
\[
A(S_\infty) \ge A(\Sigma^*) \quad \text{(area at late times)}
\]

But we want to relate to $A(\Sigma_0)$, not $A(\Sigma^*)$.

\subsection{The Missing Link}

\begin{problem}
Show that $A(\Sigma^*) \ge A(\Sigma_0)$ or find an alternative bound.
\end{problem}

In the favorable case ($\tr_{\Sigma_0} k \ge 0$), this holds by the area comparison 
theorem.

In the unfavorable case, this can FAIL!

\section{A Different Approach: Trace Back the Horizon}

\subsection{The Idea}

Instead of starting at $\Sigma_0$ and going forward, start at the EVENT HORIZON 
at late times and trace BACK to the initial slice.

\begin{definition}[Event Horizon]
The event horizon $\mathcal{E}$ is the boundary of the past of future null infinity:
\[
\mathcal{E} = \partial J^-(\mathscr{I}^+)
\]
\end{definition}

The event horizon is always a null hypersurface (when it exists).

\subsection{The Area Theorem}

\begin{theorem}[Hawking's Area Theorem]
Under NEC, the area of the event horizon is non-decreasing toward the future:
\[
A(\mathcal{E}_{t_2}) \ge A(\mathcal{E}_{t_1}) \quad \text{for } t_2 > t_1
\]
\end{theorem}

\subsection{Applying to Initial Data}

Let $\mathcal{E}_0 = \mathcal{E} \cap M$ be the event horizon's intersection 
with the initial data slice.

Then:
\[
A(\mathcal{E}_\infty) \ge A(\mathcal{E}_0)
\]

\textbf{Question}: How does $\mathcal{E}_0$ relate to $\Sigma_0$?

\subsection{Event Horizon Encloses Trapped Surfaces}

\begin{theorem}[Trapped Surfaces Inside Event Horizon]
Any trapped surface $\Sigma$ lies inside the event horizon:
\[
\Sigma \subset J^-(\mathcal{E})
\]
\end{theorem}

\textbf{Proof sketch}: If $\Sigma$ were outside $\mathcal{E}$, outgoing null 
geodesics from $\Sigma$ would reach $\mathscr{I}^+$. But the trapped condition 
means these geodesics focus, contradiction (by singularity theorem).

\subsection{The Area Relationship}

Since $\Sigma_0$ lies inside $\mathcal{E}_0$:
\[
A(\mathcal{E}_0) \ge A(\Sigma_0) \quad \text{???}
\]

This is NOT automatic! A surface inside another can have larger or smaller area.

\textbf{Example}: A very "wrinkled" inner surface could have huge area.

\section{The Correct Statement}

\subsection{What We Can Prove}

\begin{enumerate}
    \item Event horizon exists (weak cosmic censorship assumption)
    \item $A(\mathcal{E}_\infty) \ge A(\mathcal{E}_0)$ (Hawking area theorem)
    \item $\Sigma_0 \subset J^-(\mathcal{E}_0)$ (trapped surfaces inside horizon)
    \item $M_{\mathrm{ADM}} \ge M_B(\infty) \ge M_{\mathrm{final}}$ (mass hierarchy)
    \item If final state is Schwarzschild: $M_{\mathrm{final}} = \sqrt{A(\mathcal{E}_\infty)/(16\pi)}$
\end{enumerate}

\subsection{The Missing Step}

We need: $A(\mathcal{E}_0) \ge A(\Sigma_0)$ or some substitute.

This is GEOMETRICALLY the statement that the event horizon cross-section is 
"at least as large" as any trapped surface inside.

\subsection{A Weaker Statement}

Instead of area comparison, maybe we can use a MASS comparison:

\begin{conjecture}[Hawking Mass Comparison]
For any trapped surface $\Sigma_0$ inside the event horizon $\mathcal{E}_0$:
\[
M_H[\mathcal{E}_0] \ge M_P[\Sigma_0] = \sqrt{\frac{A(\Sigma_0)}{16\pi}}
\]
\end{conjecture}

Combined with the mass hierarchy, this gives Penrose!

\section{Proving the Hawking Mass Comparison}

\subsection{The Event Horizon is a MOTS (Approximately)}

At late times, the event horizon approaches the apparent horizon (MOTS).
On the initial slice, they might differ.

For the event horizon: $\theta^+[\mathcal{E}_0]$ is typically very small 
(approaching 0 at late times).

\subsection{The Hawking Mass of the Event Horizon}

\[
M_H[\mathcal{E}_0] = \sqrt{\frac{A(\mathcal{E}_0)}{16\pi}}\left(1 - \frac{1}{16\pi}\int_{\mathcal{E}_0} H^2\right)
\]

If $\mathcal{E}_0$ is close to a MOTS ($H \approx -\tr k$), then:
\[
M_H[\mathcal{E}_0] \approx \sqrt{\frac{A(\mathcal{E}_0)}{16\pi}}\left(1 - \frac{(\tr k)^2 A}{16\pi}\right)
\]

\subsection{Comparison}

We need:
\[
\sqrt{A_{\mathcal{E}_0}}\left(1 - \frac{(\tr k)^2 A_{\mathcal{E}_0}}{16\pi}\right) \ge \sqrt{A_{\Sigma_0}}
\]

This requires:
\[
A_{\mathcal{E}_0}\left(1 - \frac{(\tr k)^2 A_{\mathcal{E}_0}}{16\pi}\right)^2 \ge A_{\Sigma_0}
\]

If $A_{\mathcal{E}_0} >> A_{\Sigma_0}$, this is satisfied even with a large $(\tr k)^2$ penalty.

If $A_{\mathcal{E}_0} \approx A_{\Sigma_0}$, we need the penalty term to be small.

\section{A New Idea: The Intrinsic vs. Extrinsic Area}

\subsection{The Observation}

The area $A(\Sigma)$ is an INTRINSIC property of the 2-surface in spacetime.

The "effective" black hole size might be measured differently!

\subsection{The Kodama Vector}

In spherically symmetric spacetimes, the Kodama vector $K^\mu$ is a canonical 
timelike-outside/spacelike-inside vector field.

The Misner-Sharp mass:
\[
M_{\mathrm{MS}} = \frac{r}{2}\left(1 - g^{\mu\nu}\partial_\mu r \partial_\nu r\right)
\]

where $r = \sqrt{A/(4\pi)}$ is the area radius.

\subsection{The Kodama-Inspired Mass}

\begin{definition}[Kodama Mass]
For a 2-surface $\Sigma$ with area $A$ and null expansions $\theta^\pm$:
\[
M_K[\Sigma] = \sqrt{\frac{A}{16\pi}}\left(1 + \frac{A\theta^+\theta^-}{16\pi}\right)
\]
\end{definition}

\textbf{Properties}:
\begin{itemize}
    \item For round spheres in Schwarzschild: $M_K = M$ ✓
    \item For MOTS ($\theta^+ = 0$): $M_K = \sqrt{A/(16\pi)}$ ✓
    \item For trapped ($\theta^+\theta^- > 0$): $M_K > \sqrt{A/(16\pi)}$ !
\end{itemize}

\textbf{Wait, this goes the WRONG direction for trapped surfaces!}

Let me reconsider the formula...

\subsection{Correct Misner-Sharp Formula}

The actual Misner-Sharp mass in spherical symmetry:
\[
M_{\mathrm{MS}} = \frac{r}{2}\left(1 - \frac{r^2 \theta^+\theta^-}{4}\right)
\]

For trapped surfaces: $\theta^+\theta^- > 0$, so the term is SUBTRACTED.

This gives:
\[
M_{\mathrm{MS}} = \frac{r}{2}\left(1 - \frac{r^2 \theta^+\theta^-}{4}\right) < \frac{r}{2} = M_P
\]

So $M_{\mathrm{MS}} < M_P$ for trapped surfaces. The Misner-Sharp mass is LESS 
than the Penrose mass!

\section{Re-examining the Strategy}

\subsection{What's Really Needed}

We need an inequality of the form:
\[
\text{(Mass at infinity)} \ge \text{(Mass-like quantity at } \Sigma_0 \text{)}
\]

where the "mass-like quantity" equals $\sqrt{A(\Sigma_0)/(16\pi)}$ for the Penrose inequality.

\subsection{The Hawking Mass is Too Weak}

$M_H[\Sigma_0] \le M_P[\Sigma_0]$ always (with equality only for minimal surfaces).

So bounding $M$ by $M_H[\Sigma_0]$ doesn't give Penrose.

\subsection{The Kodama/Misner-Sharp Mass is Also Too Weak}

$M_{\mathrm{MS}}[\Sigma_0] < M_P[\Sigma_0]$ for trapped surfaces.

\subsection{The Right Mass?}

We need a quasi-local mass $\mathcal{M}$ with:
\[
\mathcal{M}[\Sigma] = \sqrt{\frac{A(\Sigma)}{16\pi}} \quad \text{for trapped } \Sigma
\]

This is NOT the Hawking mass, not the Misner-Sharp mass...

\textbf{The Penrose mass} $M_P[\Sigma] = \sqrt{A/(16\pi)}$ has this property by 
definition! But it's not known to satisfy any good monotonicity.

\section{The Penrose Mass Monotonicity}

\subsection{The Dream}

\begin{conjecture}[Penrose Mass Monotonicity]
There exists a foliation of spacetime such that $M_P[\Sigma_t]$ is monotone 
increasing toward infinity and $\lim M_P = M_{\mathrm{ADM}}$.
\end{conjecture}

This would immediately give the Penrose inequality!

\subsection{Why This is Hard}

Under a general evolution:
\[
\frac{dM_P}{dt} = \frac{d}{dt}\sqrt{\frac{A}{16\pi}} = \frac{1}{2\sqrt{16\pi A}}\frac{dA}{dt}
\]

So $M_P$ is monotone iff $A$ is monotone.

For trapped surfaces, area DECREASES along outgoing null directions (focusing)!

\subsection{The Resolution}

We need a foliation that is NOT the natural outgoing null direction.

\textbf{Possibility}: Move along the EVENT HORIZON, which has increasing area by Hawking's theorem.

But the event horizon is defined teleologically (depends on the future), 
not locally.

\section{The Final Synthesis}

\subsection{What We Know}

\begin{enumerate}
    \item Penrose inequality holds for $\tr_\Sigma k \ge 0$ (proven)
    \item For $\tr_\Sigma k < 0$, the Jang equation method fails
    \item The dynamical horizon area law gives $A(S_t) \ge A(S_0)$ for MOTS slices
    \item The event horizon area theorem gives $A(\mathcal{E}_t) \ge A(\mathcal{E}_0)$
\end{enumerate}

\subsection{What We Need}

A connection between $A(\Sigma_0)$ (arbitrary trapped surface) and 
$A(\mathcal{E}_0)$ or $A(S_0)$ (horizon cross-section).

\subsection{The Conjecture}

\begin{conjecture}[Horizon Area Dominance]
For any trapped surface $\Sigma_0$ in the domain of dependence of the event 
horizon $\mathcal{E}$:
\[
A(\mathcal{E}_0) \ge A(\Sigma_0)
\]
where $\mathcal{E}_0$ is the event horizon on the same slice.
\end{conjecture}

\textbf{If true}: Combined with $M \ge M_{\mathrm{final}} = \sqrt{A(\mathcal{E}_\infty)/(16\pi)} 
\ge \sqrt{A(\mathcal{E}_0)/(16\pi)} \ge \sqrt{A(\Sigma_0)/(16\pi)}$, the Penrose 
inequality follows!

\section{Conclusion}

The spacetime approach is promising but incomplete. The key missing piece is 
the \textbf{Horizon Area Dominance Conjecture}.

\textbf{Alternative}: If this conjecture is FALSE, then the Penrose inequality 
itself might be false for $\tr_\Sigma k < 0$, and the modified inequality is 
the correct statement.

\end{document}
