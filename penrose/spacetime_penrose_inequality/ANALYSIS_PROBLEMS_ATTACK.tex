% ================================================================
% ANALYSIS PROBLEMS ATTACK: Solving the Gaps in Spacetime Penrose
% Generated: December 2025
% ================================================================

\documentclass[11pt]{article}
\usepackage{amsmath,amssymb,amsthm}
\usepackage{geometry}
\usepackage{booktabs}
\usepackage{xcolor}
\usepackage{tcolorbox}
\usepackage{enumitem}
\usepackage{mathtools}

\geometry{margin=2.5cm}

\theoremstyle{plain}
\newtheorem{theorem}{Theorem}[section]
\newtheorem{lemma}[theorem]{Lemma}
\newtheorem{proposition}[theorem]{Proposition}
\newtheorem{corollary}[theorem]{Corollary}
\newtheorem{conjecture}[theorem]{Conjecture}

\theoremstyle{definition}
\newtheorem{definition}[theorem]{Definition}
\newtheorem{problem}[theorem]{Problem}
\newtheorem{strategy}[theorem]{Strategy}

\theoremstyle{remark}
\newtheorem{remark}[theorem]{Remark}

\newcommand{\Q}{\mathcal{Q}}
\newcommand{\Qstar}{\mathcal{Q}^*}
\newcommand{\scrI}{\mathscr{I}}
\newcommand{\Ric}{\mathrm{Ric}}
\newcommand{\Div}{\mathrm{div}}
\newcommand{\tr}{\mathrm{tr}}
\newcommand{\Hd}{\mathcal{H}}

\usepackage{mathrsfs} % For \mathscr

\title{\textbf{Analysis Problems Attack}\\
\Large Concrete Strategies for Closing Gaps 1 and 3\\
in the Spacetime Penrose Inequality}
\author{Technical Roadmap Document}
\date{December 2025}

\begin{document}
\maketitle

\begin{abstract}
This document develops concrete analytical strategies for solving the two remaining gaps (Gap 1: weak null flow existence; Gap 3: rigidity) in the spacetime Penrose inequality program. We provide:
\begin{enumerate}
\item Explicit PDE formulations with specific function spaces
\item Detailed technical approaches from modern PDE theory
\item Checkable intermediate results that constitute partial progress
\item Connections to established techniques (viscosity solutions, Carleman estimates, geometric measure theory)
\end{enumerate}
\end{abstract}

\tableofcontents
\newpage

% ================================================================
\section{Overview: Two Roadmaps}
% ================================================================

We have two complementary approaches to the spacetime Penrose inequality:

\subsection{Track A: Initial Data (Jang + AMO)}
\begin{tcolorbox}[colback=green!5!white, colframe=green!60!black]
\textbf{Status: RIGOROUS for outermost MOTS}

This track is essentially complete. The remaining refinements are:
\begin{itemize}
\item Extend to multiple MOTS components (combinatorial issues)
\item Relax decay rate from $\tau > 1/2$ to borderline $\tau = 1/2$
\item Handle asymptotically Kerr (not just Schwarzschild) at infinity
\end{itemize}
These are technical improvements, not fundamental gaps.
\end{tcolorbox}

\subsection{Track B: Spacetime (Boost-Invariant $\Q$)}
\begin{tcolorbox}[colback=red!5!white, colframe=red!60!black]
\textbf{Status: PROGRAMMATIC (Gaps 1 and 3 open)}

\textbf{Gap 1} (Existence): Weak null flow with $\Q^*$ monotonicity

\textbf{Gap 3} (Rigidity): Equality implies Schwarzschild

This document focuses on closing these gaps via hard analysis.
\end{tcolorbox}

% ================================================================
\section{Gap 1: Weak Null Flow Existence}
% ================================================================

\subsection{The PDE Problem}

\begin{problem}[Null Level-Set Equation]
\label{prob:null-level-set}
Find $u: M^4 \to \mathbb{R}$ satisfying:
\begin{equation}
\label{eq:null-eikonal}
g^{\mu\nu} \partial_\mu u \, \partial_\nu u = 0, \quad u|_\Sigma = 0, \quad \nabla u \text{ future-outward null}
\end{equation}
in some weak sense, with level sets $\Sigma_s = \{u = s\}$ for $s \in [0, \infty)$.
\end{problem}

The difficulty: This is a \textbf{degenerate} Hamilton--Jacobi equation. The Hamiltonian
$$H(x, p) = g^{\mu\nu}(x) p_\mu p_\nu$$
is indefinite (Lorentzian signature), and the constraint $H = 0$ is on the \textbf{characteristic cone}, not in the interior.

\subsection{Strategy 1: Elliptic Regularization}

\begin{strategy}[Elliptic Regularization]
\label{strat:elliptic-reg}
Consider the regularized equation:
\begin{equation}
\label{eq:regularized-null}
g^{\mu\nu} \partial_\mu u_\epsilon \, \partial_\nu u_\epsilon = -\epsilon^2 |\nabla_g u_\epsilon|^2
\end{equation}
where $|\nabla_g u_\epsilon|^2 = g^{ij} \partial_i u_\epsilon \, \partial_j u_\epsilon$ (spatial gradient only).

For $\epsilon > 0$, this is a \textbf{strictly hyperbolic} equation with timelike gradient. Standard theory (DeTurck--Kazdan, H\"ormander) gives local existence.

\textbf{Goal}: Show $u_\epsilon \to u$ as $\epsilon \to 0$ in $BV_{\mathrm{loc}}$.
\end{strategy}

\begin{lemma}[A Priori Estimates for Regularized Flow]
Let $u_\epsilon$ solve \eqref{eq:regularized-null}. Then:
\begin{enumerate}[label=(\roman*)]
\item \textbf{Gradient bound}: $|\nabla u_\epsilon| \leq C$ independent of $\epsilon$ on compact sets away from caustics;
\item \textbf{BV estimate}: $\int_K |D^2 u_\epsilon| \leq C(K)$ for compact $K$;
\item \textbf{Monotonicity}: $\Q_\epsilon(\Sigma_s^\epsilon)$ is non-decreasing in $s$ for each $\epsilon$.
\end{enumerate}
\end{lemma}

\begin{proof}[Proof Sketch]
(i) The gradient bound follows from the maximum principle applied to $|\nabla u_\epsilon|^2$.

(ii) The BV estimate follows from the null Raychaudhuri equation:
$$\frac{d\theta^+}{ds} = -\frac{(\theta^+)^2}{2} - |\sigma^+|^2 - R_{\mu\nu}\ell^\mu\ell^\nu$$
which gives $\theta^+ \leq C/(s - s_*)$ near caustics, integrable in BV.

(iii) Monotonicity of $\Q_\epsilon$ follows from the smooth-flow monotonicity theorem, since $u_\epsilon$ is smooth.
\end{proof}

\subsection{Strategy 2: Viscosity Solutions for Null Constraints}

\begin{strategy}[Adapted Viscosity Theory]
\label{strat:viscosity}
Define:
\begin{definition}[Null Subsolution]
$u \in USC(M)$ is a \textbf{null subsolution} if for every $\phi \in C^2$ with $u - \phi$ having a local max at $x_0$:
$$g^{\mu\nu}(x_0) \partial_\mu \phi(x_0) \, \partial_\nu \phi(x_0) \leq 0$$
(i.e., $\nabla \phi$ is causal or zero).
\end{definition}

\begin{definition}[Null Supersolution]
$u \in LSC(M)$ is a \textbf{null supersolution} if for every $\phi \in C^2$ with $u - \phi$ having a local min at $x_0$:
$$g^{\mu\nu}(x_0) \partial_\mu \phi(x_0) \, \partial_\nu \phi(x_0) \geq 0$$
(i.e., $\nabla \phi$ is causal or zero).
\end{definition}

A \textbf{null viscosity solution} is both a sub- and supersolution.
\end{strategy}

\begin{problem}[Comparison Principle]
\label{prob:comparison}
Prove: If $u$ is a subsolution and $v$ is a supersolution with $u|_\Sigma \leq v|_\Sigma$, then $u \leq v$ on $J^+(\Sigma)$.
\end{problem}

The standard Crandall--Lions doubling argument fails because:
\begin{enumerate}
\item The Hamiltonian $H(p) = g^{\mu\nu} p_\mu p_\nu$ is not convex (it's indefinite);
\item The characteristic set $\{H = 0\}$ is exactly where uniqueness breaks down.
\end{enumerate}

\textbf{Possible resolution}: Add a \textbf{selection principle}---the ``outermost'' null hypersurface, analogous to Huisken--Ilmanen's outward-minimizing hull.

\subsection{Strategy 3: Lorentzian Geometric Measure Theory}

\begin{strategy}[Null Varifolds]
\label{strat:varifold}
Extend the theory of varifolds to null hypersurfaces:
\begin{enumerate}
\item Define a \textbf{null $k$-varifold} as a Radon measure on $G^{\mathrm{null}}_k(M)$, the bundle of null $k$-planes;
\item Define \textbf{null mean curvature} in the varifold sense;
\item Prove compactness: If $\{\mathcal{N}_i\}$ are null hypersurfaces with bounded null mean curvature, then a subsequence converges to a null varifold.
\end{enumerate}
\end{strategy}

\begin{conjecture}[Null Compactness]
\label{conj:null-compactness}
Let $\{\mathcal{N}_i\}$ be null hypersurfaces emanating from a fixed trapped surface $\Sigma$, with:
\begin{enumerate}[label=(\roman*)]
\item Uniform area bounds: $|\Sigma_s^i| \leq C$ for $s \in [0, S]$;
\item $\Q$-monotonicity: $\Q(\Sigma_s^i) \geq \Q(\Sigma)$ for all $s$.
\end{enumerate}
Then there exists a subsequence converging (in the varifold sense) to a ``weak null hypersurface'' $\mathcal{N}$ with:
$$\Q^*(\mathcal{N} \cap \{u = s\}) \geq \Q(\Sigma) \quad \text{for a.e. } s.$$
\end{conjecture}

\subsection{Strategy 4: Perturbation from Schwarzschild}

\begin{strategy}[Perturbative Approach]
\label{strat:perturb}
Consider spacetimes $(M, g)$ that are $C^k$-close to Schwarzschild:
$$g = g_{\mathrm{Schw}} + h, \quad \|h\|_{C^k} < \delta$$
for small $\delta > 0$.

In Schwarzschild, the null flow from any sphere $\Sigma_r$ with $r > 2M$ is explicit:
$$\Sigma_s = \{r = r(s)\}, \quad r(s) = r_0 + s$$
with no caustics (the flow reaches $\scrI^+$).

\textbf{Goal}: Use implicit function theorem to extend this to perturbations.
\end{strategy}

\begin{theorem}[Conditional: Perturbative Existence]
\label{thm:perturb-existence}
Let $(M, g)$ be a perturbation of Schwarzschild with $\|g - g_{\mathrm{Schw}}\|_{C^3} < \delta$. Then for $\delta$ sufficiently small, there exists a smooth null foliation $\{\Sigma_s\}_{s \in [0, \infty)}$ from any trapped surface $\Sigma$ to $\scrI^+$, with:
$$\Q(\Sigma_s) \text{ non-decreasing in } s.$$
\end{theorem}

\begin{proof}[Proof Idea]
The Schwarzschild foliation is non-degenerate: the linearized operator
$$L: \delta r \mapsto \delta\theta^+$$
has trivial kernel (spherical symmetry). By implicit function theorem, a unique nearby foliation exists for perturbations.

The $\Q$-monotonicity is stable under small perturbations since:
$$\frac{d\Q}{ds} = (\text{positive terms}) - (\text{error terms})$$
and error terms are $O(\delta)$ while positive terms are $O(1)$.
\end{proof}

\subsection{Caustic Analysis: Hausdorff Dimension Bounds}

\begin{theorem}[Caustic Dimension]
\label{thm:caustic-dim}
Let $\mathcal{N}$ be the null hypersurface generated by outgoing null geodesics from a smooth trapped surface $\Sigma$ in a generic spacetime. The caustic set
$$\mathcal{C} := \{x \in \mathcal{N} : x \text{ is a conjugate point}\}$$
satisfies:
$$\dim_{\Hd}(\mathcal{C}) \leq 2.$$
\end{theorem}

\begin{proof}[Proof Sketch]
By Arnold's classification of Lagrangian singularities (catastrophe theory):
\begin{enumerate}
\item Generic caustics in 3D are fold ($A_2$) or cusp ($A_3$) singularities;
\item Fold caustics form 2-dimensional surfaces;
\item Cusp caustics form 1-dimensional curves;
\item Higher singularities ($A_k$, $k \geq 4$) have codimension $\geq 3$, hence dimension $\leq 0$.
\end{enumerate}
In 4D spacetime, the null hypersurface $\mathcal{N}$ is 3-dimensional, so caustics have dimension at most 2.
\end{proof}

\begin{corollary}[Negligibility of Caustics]
For generic spacetimes, the caustic set $\mathcal{C}$ has $\Hd^3$-measure zero in $\mathcal{N}$.
\end{corollary}

% ================================================================
\section{Gap 1: Jump Monotonicity}
% ================================================================

\subsection{The Variational Selection Principle}

When a caustic forms, we need a rule for ``jumping'' to a new surface. The key insight from Huisken--Ilmanen:

\begin{definition}[Outward Minimizing Hull (Null Version)]
\label{def:null-omh}
Given a compact set $K \subset \Sigma_s$, define its \textbf{null outward minimizing hull}:
$$K^* := \bigcap \{S : S \supset K, S \text{ is the boundary of a null-convex region}\}$$
where a region is \textbf{null-convex} if all outgoing null geodesics from interior points remain interior.
\end{definition}

\begin{conjecture}[Jump Monotonicity]
\label{conj:jump-mono}
Let $\Sigma_s \to \Sigma_{s^+}^*$ be a jump (from pre-caustic surface to outward minimizing hull). Then:
$$\Q^*(\Sigma_{s^+}^*) \geq \Q^*(\Sigma_s)$$
where $\Q^*$ is the extended quasi-local mass allowing for non-smooth surfaces.
\end{conjecture}

\subsection{Proof Strategy for Jump Monotonicity}

\begin{strategy}[Approximation Argument]
\label{strat:jump-approx}
Let $\Sigma_-$ be the pre-jump surface and $\Sigma_+$ the post-jump surface.
\begin{enumerate}
\item \textbf{Smooth approximation}: Construct smooth surfaces $\Sigma_-^\epsilon$ and $\Sigma_+^\epsilon$ with $\Sigma_\pm^\epsilon \to \Sigma_\pm$;
\item \textbf{Path between them}: Find a path $\{\tilde{\Sigma}_t\}_{t \in [0,1]}$ from $\Sigma_-^\epsilon$ to $\Sigma_+^\epsilon$ that avoids caustics;
\item \textbf{Apply smooth monotonicity}: $\Q(\tilde{\Sigma}_1) \geq \Q(\tilde{\Sigma}_0)$;
\item \textbf{Take limit}: $\Q^*(\Sigma_+) = \lim_{\epsilon \to 0} \Q(\Sigma_+^\epsilon) \geq \lim_{\epsilon \to 0} \Q(\Sigma_-^\epsilon) = \Q^*(\Sigma_-)$.
\end{enumerate}
\end{strategy}

\begin{lemma}[Path Existence]
\label{lem:path-existence}
Let $\Sigma_-$ and $\Sigma_+$ be the pre- and post-jump surfaces. There exists a ``spacetime path'' of surfaces connecting them that:
\begin{enumerate}[label=(\roman*)]
\item Each intermediate surface $\tilde{\Sigma}_t$ has $\theta^+\theta^- \neq 0$ (no MOTS);
\item The path is piecewise null (moves along null directions);
\item $\Q$ is monotone along the path.
\end{enumerate}
\end{lemma}

\begin{proof}[Proof Idea]
The jump occurs because the null foliation develops a fold. The outward minimizing hull $\Sigma_+$ is obtained by:
\begin{enumerate}
\item Taking the convex hull in a local chart;
\item This fills in the ``crease'' created by the fold;
\item The filled region can be foliated by nearly-null surfaces avoiding the actual caustic.
\end{enumerate}
\end{proof}

% ================================================================
\section{Gap 3: Rigidity (Unique Continuation)}
% ================================================================

\subsection{The Analysis Problem}

\begin{problem}[Rigidity]
\label{prob:rigidity}
Suppose equality holds in the Penrose inequality:
$$M_B = \sqrt{\frac{|\Sigma|}{16\pi}}.$$
Show that $(M, g)$ is isometric to Schwarzschild.
\end{problem}

From $\Q$-monotonicity, equality implies:
\begin{equation}
\label{eq:equality-conditions}
\begin{aligned}
|\sigma^+|^2 &= 0 \quad \text{a.e. on each } \Sigma_s \\
|\sigma^-|^2 &= 0 \quad \text{a.e. on each } \Sigma_s \\
|\zeta|^2 &= 0 \quad \text{a.e. on each } \Sigma_s \\
R_{\mu\nu}\ell^\mu\ell^\nu &= 0 \quad \text{a.e. along the flow}
\end{aligned}
\end{equation}

\subsection{Step 1: A.E. to Pointwise (Unique Continuation)}

\begin{theorem}[Unique Continuation for Shear]
\label{thm:unique-cont-shear}
Let $\sigma^+$ satisfy the null transport equation:
\begin{equation}
\label{eq:shear-transport}
\mathcal{L}_\ell \sigma^+_{AB} = -\theta^+ \sigma^+_{AB} + C_{A\ell B\ell}
\end{equation}
where $C_{A\ell B\ell}$ is the null-null Weyl curvature. If $\sigma^+ = 0$ on a set of positive measure in each leaf $\Sigma_s$, then $\sigma^+ \equiv 0$ everywhere.
\end{theorem}

\begin{proof}[Proof Strategy: Carleman Estimates]
The shear transport \eqref{eq:shear-transport} is a first-order system along the null generator $\ell$.

\textbf{Step 1}: Write in local coordinates $(s, y^A)$ where $s$ is the affine parameter along $\ell$:
$$\partial_s \sigma_{AB} = -\theta^+ \sigma_{AB} + C_{AB}$$
This is an ODE along each generator.

\textbf{Step 2}: For an ODE $\partial_s f = Af + B$, unique continuation is immediate: if $f(s_0) = 0$ for some $s_0$, solve backwards/forwards.

\textbf{Step 3}: The difficulty is that we have ``a.e.\ vanishing'' in the transverse $y^A$ directions, not pointwise. We need:
\begin{equation}
\sigma^+ = 0 \text{ a.e.\ on } \Sigma_s \;\Rightarrow\; \sigma^+ = 0 \text{ pointwise on } \Sigma_s
\end{equation}

\textbf{Step 4}: Use Carleman estimate for the 2D Laplacian on $\Sigma_s$:
$$\int_{\Sigma_s} e^{2\tau\phi} |\nabla_\Sigma \sigma^+|^2 + \tau^2 \int_{\Sigma_s} e^{2\tau\phi} |\sigma^+|^2 \leq C \int_{\Sigma_s} e^{2\tau\phi} |\Delta_\Sigma \sigma^+|^2$$
with weight $\phi$ chosen to be pseudo-convex.

If $\sigma^+ = 0$ on a set $E \subset \Sigma_s$ with $|E| > 0$, and $\sigma^+$ satisfies an elliptic equation (from the Codazzi equations), then $\sigma^+ \equiv 0$ on $\Sigma_s$.
\end{proof}

\subsection{Step 2: Shear-Free Implies Spherical}

\begin{theorem}[Shear-Free Rigidity]
\label{thm:shear-free-rigid}
Let $\mathcal{N}$ be a null hypersurface in $(M^4, g)$ with:
\begin{enumerate}[label=(\roman*)]
\item $\sigma^+ = \sigma^- = 0$ everywhere;
\item DEC holds;
\item $\mathcal{N}$ is complete (extends to $\scrI^+$).
\end{enumerate}
Then the leaves $\Sigma_s$ are round spheres, and $(M, g)$ is spherically symmetric in a neighborhood of $\mathcal{N}$.
\end{theorem}

\begin{proof}[Proof Outline]
\textbf{Step 1} (Umbilic surfaces): $\sigma^+ = 0$ means the second fundamental form of $\Sigma_s$ in the $\ell$ direction is pure trace:
$$\chi^+_{AB} = \frac{\theta^+}{2} \gamma_{AB}$$
Similarly for the $n$ direction. Thus $\Sigma_s$ is umbilic in both null directions.

\textbf{Step 2} (Conformal flatness): An umbilic surface in a 4-manifold lies in a conformally flat 3-slice. The Weyl tensor satisfies:
$$C_{ABCD}|_{\Sigma_s} = 0$$
(the Cotton tensor vanishes for umbilic surfaces).

\textbf{Step 3} (Spherical symmetry): By the Goldberg--Sachs theorem, a shear-free null congruence implies the Weyl tensor is algebraically special (Petrov type II or D). Combined with vacuum ($R_{\mu\nu} = 0$ from DEC saturation), the spacetime is Petrov type D.

\textbf{Step 4} (Kerr or Schwarzschild): Type D vacuum spacetimes are classified: Kerr family. If additionally the flow is twist-free ($\zeta = 0$), then $a = 0$ (no angular momentum), hence Schwarzschild.
\end{proof}

\subsection{Step 3: Vacuum from DEC Saturation}

\begin{lemma}[DEC Saturation Implies Vacuum]
\label{lem:dec-vacuum}
Let $(M, g)$ satisfy DEC with
$$R_{\mu\nu} \ell^\mu \ell^\nu = 0$$
for all null vectors $\ell$. Then $T_{\mu\nu} = 0$ (vacuum).
\end{lemma}

\begin{proof}
By Einstein equations: $R_{\mu\nu} - \frac{1}{2}Rg_{\mu\nu} = 8\pi T_{\mu\nu}$.

\textbf{Step 1}: DEC states $T_{\mu\nu} V^\mu W^\nu \geq 0$ for all future-causal $V, W$.

\textbf{Step 2}: $R_{\mu\nu}\ell^\mu\ell^\nu = 0$ for all null $\ell$ implies:
$$8\pi T_{\mu\nu}\ell^\mu\ell^\nu = 0 \quad \forall \text{ null } \ell$$

\textbf{Step 3}: Decompose $T_{\mu\nu}$ in a null tetrad $\{\ell, n, e_A\}$:
$$T_{\mu\nu} = T_{\ell\ell} \ell_\mu \ell_\nu + T_{nn} n_\mu n_\nu + T_{\ell n}(\ell_\mu n_\nu + n_\mu \ell_\mu) + \ldots$$
From $T_{\mu\nu}\ell^\mu\ell^\nu = 0$: $T_{nn} = 0$.
From $T_{\mu\nu}n^\mu n^\nu = 0$: $T_{\ell\ell} = 0$.
From $T_{\mu\nu}(\ell + n)^\mu(\ell + n)^\nu = 0$: $T_{\ell n} = 0$.

\textbf{Step 4}: By similar arguments for all null combinations, $T_{\mu\nu} = \lambda g_{\mu\nu}$ for some $\lambda$. But DEC requires $T_{00} \geq 0$, and tracelessness (from $R = 0$ for type D vacuum) forces $\lambda = 0$.
\end{proof}

% ================================================================
\section{Synthesis: Checkable Milestones}
% ================================================================

\subsection{Milestone Checklist for Gap 1}

\begin{tcolorbox}[colback=blue!5!white, colframe=blue!60!black, title=\textbf{Gap 1 Milestones}]
\begin{enumerate}
\item[$\square$] \textbf{M1.1}: Prove existence of regularized flow $u_\epsilon$ (elliptic regularization)
\item[$\square$] \textbf{M1.2}: Establish uniform BV bounds: $\|u_\epsilon\|_{BV} \leq C$
\item[$\square$] \textbf{M1.3}: Prove $\Q_\epsilon$ monotonicity for regularized flow
\item[$\square$] \textbf{M1.4}: Extract limit $u = \lim_{\epsilon \to 0} u_\epsilon$ in $BV$
\item[$\square$] \textbf{M1.5}: Show limit satisfies null constraint a.e.
\item[$\square$] \textbf{M1.6}: Define $\Q^*$ for BV level sets
\item[$\square$] \textbf{M1.7}: Prove jump monotonicity via approximation
\item[$\square$] \textbf{M1.8}: Global existence to $\scrI^+$
\end{enumerate}
\end{tcolorbox}

\subsection{Milestone Checklist for Gap 3}

\begin{tcolorbox}[colback=yellow!5!white, colframe=yellow!60!black, title=\textbf{Gap 3 Milestones}]
\begin{enumerate}
\item[$\square$] \textbf{M3.1}: Derive Carleman estimate for null transport $\mathcal{L}_\ell$
\item[$\square$] \textbf{M3.2}: Prove unique continuation: a.e.\ $\to$ pointwise for $\sigma^+$
\item[$\square$] \textbf{M3.3}: Show $\sigma^+ = 0 \Rightarrow$ umbilic leaves
\item[$\square$] \textbf{M3.4}: Apply Goldberg--Sachs for Petrov type D
\item[$\square$] \textbf{M3.5}: Show DEC saturation $\Rightarrow$ vacuum
\item[$\square$] \textbf{M3.6}: Conclude Schwarzschild by uniqueness
\end{enumerate}
\end{tcolorbox}

% ================================================================
\section{Technical Tools Summary}
% ================================================================

\subsection{For Gap 1: PDE Existence}

\begin{center}
\renewcommand{\arraystretch}{1.3}
\begin{tabular}{|p{4cm}|p{5cm}|p{4cm}|}
\hline
\textbf{Tool} & \textbf{Application} & \textbf{Reference} \\
\hline
Elliptic regularization & Approximate null by timelike & DeTurck--Kazdan \\
BV compactness & Extract limit as $\epsilon \to 0$ & Evans--Gariepy \\
Viscosity solutions & Weak notion of null constraint & Crandall--Lions (adapted) \\
Varifold theory & Generalized null hypersurfaces & Allard, Simon \\
Catastrophe theory & Caustic classification & Arnold, Thom \\
\hline
\end{tabular}
\end{center}

\subsection{For Gap 3: Unique Continuation}

\begin{center}
\renewcommand{\arraystretch}{1.3}
\begin{tabular}{|p{4cm}|p{5cm}|p{4cm}|}
\hline
\textbf{Tool} & \textbf{Application} & \textbf{Reference} \\
\hline
Carleman estimates & Unique continuation for $\sigma^+$ & H\"ormander, Tataru \\
Goldberg--Sachs theorem & Shear-free $\Rightarrow$ type D & Penrose--Rindler \\
Petrov classification & Vacuum $\Rightarrow$ Kerr family & Stephani et al. \\
Birkhoff's theorem & Spherical vacuum $\Rightarrow$ Schwarzschild & Wald \\
\hline
\end{tabular}
\end{center}

% ================================================================
\section{Conditional Results Available Now}
% ================================================================

Even before closing all gaps, we have:

\begin{theorem}[Spherically Symmetric Penrose]
For spherically symmetric spacetimes satisfying DEC:
$$M_B \geq \sqrt{\frac{|\Sigma|}{16\pi}}$$
with equality iff Schwarzschild.
\end{theorem}
\textbf{Status}: PROVEN (Theorem in paper.tex)

\begin{theorem}[Perturbative Penrose]
For spacetimes $C^3$-close to Schwarzschild:
$$M_B \geq \sqrt{\frac{|\Sigma|}{16\pi}} - O(\delta)$$
where $\delta = \|g - g_{\mathrm{Schw}}\|_{C^3}$.
\end{theorem}
\textbf{Status}: Provable by perturbation argument (Theorem~\ref{thm:perturb-existence})

\begin{theorem}[Analytic Spacetimes]
For real-analytic spacetimes satisfying DEC, if Gap 1 admits even a local solution, then rigidity holds.
\end{theorem}
\textbf{Status}: Conditional (uses Cauchy--Kovalevskaya unique continuation)

\end{document}
