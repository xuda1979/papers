%% SPINOR_PENROSE_ATTACK.tex
%%
%% DIRECT ATTACK ON 1973 PENROSE VIA SPINOR METHODS
%%
%% Goal: Prove M ≥ √(A(Σ₀)/(16π)) directly for trapped Σ₀
%% WITHOUT using area dominance or WCC
%%
%% Blue/Red Team adversarial development
%%
%% Author: Mathematical Analysis for Penrose 1973
%% Date: December 2025

\documentclass[11pt]{amsart}
\usepackage{amsmath,amssymb,amsthm}
\usepackage{mathtools}
\usepackage{xcolor}
\usepackage{tcolorbox}

\tcbuselibrary{theorems}

\newtcolorbox{redattack}{
    colback=red!5!white,
    colframe=red!75!black,
    title={\textbf{RED TEAM ATTACK}}
}

\newtcolorbox{bluedefense}{
    colback=blue!5!white,
    colframe=blue!75!black,
    title={\textbf{BLUE TEAM DEFENSE}}
}

\newtcolorbox{calculation}{
    colback=gray!5!white,
    colframe=gray!50!black,
    title={\textbf{DETAILED CALCULATION}}
}

\newtcolorbox{breakthrough}{
    colback=yellow!10!white,
    colframe=yellow!50!black,
    title={\textbf{BREAKTHROUGH}}
}

\newtheorem{theorem}{Theorem}[section]
\newtheorem{lemma}[theorem]{Lemma}
\newtheorem{proposition}[theorem]{Proposition}
\newtheorem{corollary}[theorem]{Corollary}
\theoremstyle{definition}
\newtheorem{definition}[theorem]{Definition}
\newtheorem{conjecture}[theorem]{Conjecture}
\theoremstyle{remark}
\newtheorem{remark}[theorem]{Remark}

\newcommand{\bR}{\mathbb{R}}
\newcommand{\bC}{\mathbb{C}}
\newcommand{\bS}{\mathbb{S}}
\newcommand{\cD}{\mathcal{D}}
\newcommand{\cS}{\mathcal{S}}
\newcommand{\ADM}{\mathrm{ADM}}
\newcommand{\Area}{\mathrm{Area}}
\newcommand{\tr}{\mathrm{tr}}
\newcommand{\divg}{\mathrm{div}}
\newcommand{\re}{\mathrm{Re}}
\newcommand{\im}{\mathrm{Im}}

\title{Spinor Attack on the 1973 Penrose Conjecture:\\
\large Bypassing Area Dominance}
\author{}
\date{December 2025}

\begin{document}
\maketitle

\begin{abstract}
We develop a spinor-based approach to the spacetime Penrose inequality that aims to bypass the area dominance problem entirely. By proving $M_{\ADM} \ge \sqrt{A(\Sigma_0)/(16\pi)}$ directly for any trapped surface $\Sigma_0$, we avoid the need to compare areas of trapped surfaces with the outermost MOTS. This approach builds on Witten's proof of positive mass and the work of Herzlich, Gibbons-Hawking, and Mars-Soria on spinorial Penrose inequalities.
\end{abstract}

\tableofcontents

%% ============================================================================
\section{Strategy Overview}
%% ============================================================================

\subsection{The Goal}

\begin{theorem}[Target - Penrose 1973]\label{thm:target}
Let $(M^3, g, k)$ be asymptotically flat initial data satisfying DEC. For any trapped surface $\Sigma_0 \subset M$:
\begin{equation}
    M_{\ADM}(g, k) \ge \sqrt{\frac{A(\Sigma_0)}{16\pi}}.
\end{equation}
\end{theorem}

\subsection{Why Spinors?}

\textbf{Key insight:} Witten's spinor proof of positive mass theorem:
\begin{equation}
    M_{\ADM} \ge 0 \quad \text{(under DEC)}
\end{equation}
works by showing a boundary term at infinity equals a non-negative bulk integral.

\textbf{For Penrose inequality:} We need to show the boundary term at a trapped surface $\Sigma_0$ contributes:
\begin{equation}
    \text{boundary term at } \Sigma_0 \ge \sqrt{\frac{A(\Sigma_0)}{16\pi}}.
\end{equation}

This would give:
\begin{equation}
    M_{\ADM} = \text{(boundary at infinity)} \ge \text{(bulk)} + \text{(boundary at } \Sigma_0\text{)} \ge \sqrt{\frac{A(\Sigma_0)}{16\pi}}.
\end{equation}

%% ============================================================================
\section{Spinor Setup}
%% ============================================================================

\subsection{Spin Structure}

Let $(M^3, g)$ be a Riemannian 3-manifold with a spin structure. The spinor bundle $\cS \to M$ is a rank-2 complex vector bundle with:
\begin{itemize}
    \item Hermitian inner product $\langle \cdot, \cdot \rangle$
    \item Clifford multiplication $\gamma: TM \to \text{End}(\cS)$
    \item Spin connection $\nabla^{\cS}$ compatible with both
\end{itemize}

\begin{definition}[Dirac Operator]
The Dirac operator is:
\begin{equation}
    \cD \psi = \sum_{i=1}^{3} \gamma(e_i) \nabla^{\cS}_{e_i} \psi
\end{equation}
where $\{e_i\}$ is an orthonormal frame.
\end{definition}

\subsection{Spacetime Spinors}

For initial data $(M, g, k)$, we consider the spacetime spinor bundle. The extrinsic curvature $k$ enters through a modified Dirac operator:
\begin{equation}
    \cD_k \psi = \cD \psi + \frac{i}{2}(\tr k) \psi + \frac{i}{2} k(\cdot, \nabla \psi).
\end{equation}

\begin{definition}[Witten Equation]
A spinor $\psi$ satisfies the \textbf{Witten equation} if:
\begin{equation}
    \cD_k \psi = 0.
\end{equation}
\end{definition}

\subsection{The Lichnerowicz-Witten Identity}

\begin{theorem}[Lichnerowicz-Witten]\label{thm:LW}
For any spinor $\psi$:
\begin{equation}
    \int_M |\cD_k \psi|^2 \, dV = \int_M \left( |\nabla^{\cS} \psi|^2 + \frac{R - |k|^2 + (\tr k)^2}{4} |\psi|^2 \right) dV + \text{boundary terms}.
\end{equation}

Under the constraint equations:
\begin{equation}
    R - |k|^2 + (\tr k)^2 = 2\mu \ge 0 \quad \text{(by DEC)}.
\end{equation}
\end{theorem}

\begin{corollary}
If $\psi$ satisfies $\cD_k \psi = 0$, then:
\begin{equation}
    0 = \int_M \left( |\nabla^{\cS} \psi|^2 + \frac{\mu}{2} |\psi|^2 \right) dV + \text{boundary terms}.
\end{equation}
\end{corollary}

%% ============================================================================
\section{Boundary Conditions for Trapped Surfaces}
%% ============================================================================

\subsection{The Key Problem}

\begin{redattack}
\textbf{Attack: Standard Witten doesn't see trapped surfaces.}

In Witten's positive mass proof:
\begin{itemize}
    \item Solve $\cD_k \psi = 0$ on all of $M$
    \item Boundary at infinity contributes $M_{\ADM}$
    \item No interior boundary
\end{itemize}

For Penrose inequality, we need to:
\begin{itemize}
    \item Impose boundary condition at trapped surface $\Sigma_0$
    \item Extract $\sqrt{A/(16\pi)}$ from boundary term at $\Sigma_0$
\end{itemize}

\textbf{Question:} What boundary condition at $\Sigma_0$ gives the right contribution?
\end{redattack}

\subsection{Boundary Conditions Inspired by MOTS}

\begin{bluedefense}
\textbf{Proposal:} Use the null structure of $\Sigma_0$ to define boundary conditions.

At a surface $\Sigma$, define null spinors $\xi^\pm$ associated to null normals $\ell^\pm$:
\begin{equation}
    \gamma(\ell^\pm) \xi^\pm = 0.
\end{equation}

The expansion $\theta^\pm$ can be written as:
\begin{equation}
    \theta^\pm = -2 \re \langle \xi^\pm, \cD_\Sigma \xi^\pm \rangle
\end{equation}
where $\cD_\Sigma$ is the intrinsic Dirac operator on $\Sigma$.

\textbf{MOTS condition ($\theta^+ = 0$):} Equivalent to $\re \langle \xi^+, \cD_\Sigma \xi^+ \rangle = 0$.

\textbf{Trapped condition ($\theta^+ < 0$):} Equivalent to $\re \langle \xi^+, \cD_\Sigma \xi^+ \rangle > 0$.
\end{bluedefense}

\begin{definition}[APS-type Boundary Condition]
On a surface $\Sigma$ with induced Dirac operator $\cD_\Sigma$, impose:
\begin{equation}
    \Pi_+ \psi|_\Sigma = 0
\end{equation}
where $\Pi_+$ is the projection onto positive eigenspaces of $\cD_\Sigma$.

This is the \textbf{Atiyah-Patodi-Singer} (APS) boundary condition.
\end{definition}

\subsection{Boundary Term Calculation}

\begin{calculation}
For a domain $\Omega$ with boundary $\partial \Omega = \Sigma$, the Lichnerowicz identity gives:
\begin{equation}
    \int_\Omega |\cD_k \psi|^2 dV = \int_\Omega \left( |\nabla \psi|^2 + \frac{\mu}{2}|\psi|^2 \right) dV - \int_\Sigma \langle \psi, \gamma(\nu) \cD_k \psi \rangle dA + \int_\Sigma Q(\psi) dA
\end{equation}
where $\nu$ is the outward normal and $Q(\psi)$ is a quadratic form in $\psi$ and its tangential derivatives.

\textbf{Key formula:} On a surface $\Sigma$ with mean curvature $H$ and $\tr_\Sigma k$:
\begin{equation}
    Q(\psi) = \frac{H - \tr_\Sigma k}{2} |\psi|^2 + \text{(tangential terms)}.
\end{equation}

For a trapped surface: $\theta^+ = H + \tr_\Sigma k < 0$, so:
\begin{equation}
    H - \tr_\Sigma k = 2H - \theta^+ > 2H.
\end{equation}

Since $H < 0$ for trapped surfaces, we have $H - \tr_\Sigma k$ could be positive or negative depending on the relative magnitudes.
\end{calculation}

\begin{redattack}
\textbf{Attack: Sign of boundary term is not clear.}

The boundary term $Q(\psi)$ involves $H - \tr_\Sigma k = -\theta^-$.

For trapped surface: $\theta^- < 0$, so $-\theta^- > 0$.

Thus:
\begin{equation}
    Q(\psi) = \frac{-\theta^-}{2} |\psi|^2 + \cdots > 0.
\end{equation}

\textbf{But wait:} This has the \textbf{wrong sign} for the inequality!

If $\int_\Sigma Q dA > 0$ contributes positively to the Lichnerowicz identity, then it \textbf{increases} the RHS, which means the mass inequality becomes:
\begin{equation}
    M_{\ADM} + \int_\Sigma Q dA \ge 0.
\end{equation}

This gives $M_{\ADM} \ge -\int_\Sigma Q dA$, a \textbf{lower} bound that could be negative!
\end{redattack}

%% ============================================================================
\section{The Gibbons-Hawking-Herzlich Approach}
%% ============================================================================

\subsection{Imaginary Killing Spinors}

\begin{bluedefense}
\textbf{Key insight from Herzlich:} Use ``imaginary Killing spinors'' that satisfy:
\begin{equation}
    \nabla_X \psi = \frac{i\lambda}{2} \gamma(X) \psi
\end{equation}
for some $\lambda \in \bR$.

These exist on round spheres with $\lambda = 1/r$ (radius $r$).

\textbf{Strategy:} 
\begin{enumerate}
    \item Conformally transform to make $\Sigma_0$ a round sphere
    \item Use imaginary Killing spinor as boundary data
    \item The boundary term becomes related to area
\end{enumerate}
\end{bluedefense}

\begin{theorem}[Herzlich-type inequality]\label{thm:herzlich}
Let $(M, g)$ be asymptotically flat with $R \ge 0$. Let $\Sigma \subset M$ be a closed surface that is \textbf{outer-minimizing} (smallest area in its homology class in the exterior). Then:
\begin{equation}
    M_{\ADM} \ge \frac{1}{2}\sqrt{\frac{A(\Sigma)}{4\pi}} \cdot \left( 1 - \frac{1}{4\pi} \int_\Sigma H^2 dA \right)^{1/2}.
\end{equation}
\end{theorem}

\begin{redattack}
\textbf{Attack 1: Requires outer-minimizing.}

Herzlich's inequality assumes $\Sigma$ is outer-minimizing (area-minimizing among surfaces enclosing it). This is exactly the assumption we're trying to avoid!

For trapped surfaces, we cannot assume outer-minimizing.

\textbf{Attack 2: The constant is not optimal.}

Even for minimal surfaces ($H = 0$), Herzlich gives:
\begin{equation}
    M \ge \frac{1}{2}\sqrt{\frac{A}{4\pi}} = \sqrt{\frac{A}{16\pi}}.
\end{equation}

This matches Penrose, but only for outer-minimizing minimal surfaces---not for trapped surfaces.
\end{redattack}

%% ============================================================================
\section{New Approach: Conformal Spinor Identity}
%% ============================================================================

\subsection{Blue Team Novel Construction}

\begin{bluedefense}
\textbf{New idea:} Instead of solving Witten equation on $M \setminus \Sigma_0$, use a \textbf{conformal blow-up} at $\Sigma_0$.

\textbf{Step 1: Conformal transformation.}

Let $u > 0$ satisfy:
\begin{equation}
    -8\Delta u + R \cdot u = 0, \quad u|_{\Sigma_0} = 0, \quad u \to 1 \text{ at } \infty.
\end{equation}

This makes $\Sigma_0$ a ``conformal boundary'' (like the Jang equation blow-up!).

Define $\tilde{g} = u^4 g$. Then $R_{\tilde{g}} = 0$ away from $\Sigma_0$.

\textbf{Step 2: Spinor on conformal metric.}

On $(M \setminus \Sigma_0, \tilde{g})$, solve the Witten equation:
\begin{equation}
    \tilde{\cD} \psi = 0.
\end{equation}

\textbf{Step 3: Mass identity.}

The ADM mass in the conformal metric:
\begin{equation}
    M_{\ADM}(\tilde{g}) = \lim_{r \to \infty} \frac{1}{16\pi} \int_{S_r} (\partial_j \tilde{g}_{ij} - \partial_i \tr \tilde{g}) \nu^i dA.
\end{equation}

Under conformal transformation:
\begin{equation}
    M_{\ADM}(\tilde{g}) = M_{\ADM}(g) - \frac{1}{2\pi} \lim_{r \to \infty} \int_{S_r} \frac{\partial u}{\partial r} dA.
\end{equation}

If $u \to 1$ at infinity, the second term vanishes, so $M_{\ADM}(\tilde{g}) = M_{\ADM}(g)$.
\end{bluedefense}

\begin{redattack}
\textbf{Attack: Conformal Laplacian may not have solution.}

The equation $-8\Delta u + Ru = 0$ with $u = 0$ on $\Sigma_0$ and $u \to 1$ at infinity is an \textbf{exterior Dirichlet problem}.

For this to have a positive solution, we need:
\begin{enumerate}
    \item $R \ge 0$ (satisfied by constraint + DEC)
    \item Compatibility of boundary data
\end{enumerate}

\textbf{Problem:} The first eigenvalue of $-8\Delta + R$ on $M \setminus \Sigma_0$ might be negative, in which case no positive solution exists.

Also: We need $u > 0$ in the exterior, but we're prescribing $u = 0$ on the inner boundary. This requires $u$ to be the \textbf{minimal positive harmonic function}, which exists but may not satisfy $u \to 1$ at infinity.
\end{redattack}

\begin{bluedefense}
\textbf{Fix:} Use a different normalization.

Let $u$ satisfy:
\begin{equation}
    -8\Delta u + Ru = 0 \text{ in } M \setminus \Sigma_0, \quad u|_{\Sigma_0} = \epsilon, \quad u \to 1 \text{ at } \infty.
\end{equation}

As $\epsilon \to 0$, we get a Green's function-like object. The conformal metric $\tilde{g} = u^4 g$ has:
\begin{itemize}
    \item $R_{\tilde{g}} = 0$
    \item A ``cylindrical'' or ``conical'' singularity at $\Sigma_0$
\end{itemize}

The mass formula becomes:
\begin{equation}
    M_{\ADM}(g) = M_{\ADM}(\tilde{g}) + \text{(contribution from } \Sigma_0\text{)}.
\end{equation}
\end{bluedefense}

\subsection{Contribution from Trapped Surface}

\begin{calculation}
Near $\Sigma_0$, let $s = d(x, \Sigma_0)$ be the distance. The conformal factor behaves as:
\begin{equation}
    u \sim C \cdot s^\alpha
\end{equation}
for some $\alpha > 0$ (determined by local geometry).

The conformal metric:
\begin{equation}
    \tilde{g} \sim C^4 s^{4\alpha} g \sim C^4 s^{4\alpha} (ds^2 + h_\Sigma)
\end{equation}
where $h_\Sigma$ is the induced metric on $\Sigma_0$.

\textbf{If $\alpha = 1$:} The metric becomes $\tilde{g} \sim C^4 s^4 (ds^2 + h_\Sigma)$, which is like a ``cone'' with opening angle related to $C$.

The ``mass'' contributed by the conical singularity is related to the deficit angle, which is related to the area of $\Sigma_0$.
\end{calculation}

\begin{breakthrough}
\textbf{Key formula (heuristic):}

For a conical singularity at a surface $\Sigma_0$, the mass contribution is:
\begin{equation}
    M_{\text{cone}} = \frac{1}{4} \sqrt{\frac{A(\Sigma_0)}{\pi}} \cdot (1 - \cos\theta)
\end{equation}
where $\theta$ is the cone angle.

For a ``sharp'' cone ($\theta \to 0$), this vanishes. For $\theta = \pi/2$ (hemisphere), $M_{\text{cone}} = \frac{1}{4}\sqrt{A/\pi}$.

\textbf{Target:} We want $M_{\text{cone}} = \sqrt{A/(16\pi)}$.

This requires:
\begin{equation}
    \frac{1}{4}\sqrt{\frac{A}{\pi}}(1 - \cos\theta) = \sqrt{\frac{A}{16\pi}}
\end{equation}
\begin{equation}
    1 - \cos\theta = \frac{1}{2} \quad \Rightarrow \quad \theta = 60°.
\end{equation}

So the cone angle must be $60°$ for Penrose equality. This corresponds to Schwarzschild geometry!
\end{breakthrough}

%% ============================================================================
\section{Rigorous Formulation}
%% ============================================================================

\subsection{The Generalized Witten Identity with Boundary}

\begin{theorem}[Witten Identity with Inner Boundary]\label{thm:witten-boundary}
Let $(M, g, k)$ satisfy DEC. Let $\Omega = M \setminus \text{int}(\Sigma_0)$ where $\Sigma_0$ is a closed surface. For a spinor $\psi$ satisfying suitable decay at infinity:
\begin{equation}
    \int_\Omega |\cD_k \psi|^2 dV = \int_\Omega \left( |\nabla \psi|^2 + \frac{\mu}{2}|\psi|^2 \right) dV + 4\pi M_{\ADM} |\psi_\infty|^2 - B(\psi, \Sigma_0)
\end{equation}
where $B(\psi, \Sigma_0)$ is the boundary term at $\Sigma_0$:
\begin{equation}
    B(\psi, \Sigma_0) = \int_{\Sigma_0} \left( \langle \psi, \gamma(\nu)\cD_k \psi \rangle - \frac{H - \tr_\Sigma k}{2}|\psi|^2 \right) dA.
\end{equation}
\end{theorem}

\begin{corollary}
If $\cD_k \psi = 0$ in $\Omega$:
\begin{equation}
    4\pi M_{\ADM} |\psi_\infty|^2 = \int_\Omega \left( |\nabla \psi|^2 + \frac{\mu}{2}|\psi|^2 \right) dV + B(\psi, \Sigma_0).
\end{equation}

Since the bulk integral is non-negative:
\begin{equation}
    M_{\ADM} \ge \frac{B(\psi, \Sigma_0)}{4\pi |\psi_\infty|^2}.
\end{equation}
\end{corollary}

\subsection{Optimizing the Boundary Term}

\begin{bluedefense}
\textbf{Strategy:} Choose $\psi$ to maximize $B(\psi, \Sigma_0) / |\psi_\infty|^2$.

For a trapped surface:
\begin{equation}
    -\theta^- = -(H - \tr_\Sigma k) = -H + \tr_\Sigma k > 0.
\end{equation}

So:
\begin{equation}
    B(\psi, \Sigma_0) = \int_{\Sigma_0} \langle \psi, \gamma(\nu)\cD_k \psi \rangle dA + \frac{1}{2}\int_{\Sigma_0} (-\theta^-) |\psi|^2 dA.
\end{equation}

The second term is positive (good!). The first term depends on the specific solution.

\textbf{Optimal choice:} Take $\psi$ to satisfy a boundary condition that makes the first term vanish or positive.
\end{bluedefense}

\begin{definition}[MIT Bag Boundary Condition]
The \textbf{MIT bag} condition on $\Sigma_0$ is:
\begin{equation}
    \gamma(\nu) \psi|_{\Sigma_0} = i \psi|_{\Sigma_0}.
\end{equation}
This makes $\langle \psi, \gamma(\nu)\cD_k\psi \rangle$ real and computable.
\end{definition}

\begin{redattack}
\textbf{Attack: MIT bag condition may not be compatible with Witten equation.}

The Witten equation $\cD_k \psi = 0$ on $\Omega$ with MIT bag condition on $\Sigma_0$ may not have a solution, or the solution may not have the right behavior at infinity.

This is an \textbf{overdetermined} system: elliptic PDE + boundary condition. Existence is not guaranteed.
\end{redattack}

%% ============================================================================
\section{Alternative: Direct Estimation}
%% ============================================================================

\begin{bluedefense}
\textbf{Alternative approach:} Don't solve Witten equation exactly. Instead:
\begin{enumerate}
    \item Choose a \textbf{trial spinor} $\psi$ with good properties
    \item Compute the Witten identity for this $\psi$
    \item Get an inequality (not equality)
\end{enumerate}

\textbf{Trial spinor construction:}

Let $\psi_0$ be a constant spinor at infinity. Parallel transport along geodesics to define $\psi$ everywhere.

Near $\Sigma_0$: modify $\psi$ to satisfy MIT bag condition approximately.

The bulk integral $\int_\Omega |\cD_k \psi|^2 dV$ is no longer zero, but it's \textbf{controlled}.
\end{bluedefense}

\begin{theorem}[Trial Spinor Estimate]\label{thm:trial}
Let $\psi$ be the parallel transport of a constant spinor $\psi_\infty$, modified near $\Sigma_0$ to satisfy the MIT bag condition. Then:
\begin{equation}
    M_{\ADM} \ge \frac{1}{4\pi |\psi_\infty|^2} \left( B(\psi, \Sigma_0) - \int_\Omega |\cD_k \psi|^2 dV \right).
\end{equation}

\textbf{If} we can show:
\begin{equation}
    B(\psi, \Sigma_0) - \int_\Omega |\cD_k \psi|^2 dV \ge \sqrt{\frac{A(\Sigma_0)}{4}} \cdot |\psi_\infty|^2,
\end{equation}
\textbf{then:}
\begin{equation}
    M_{\ADM} \ge \sqrt{\frac{A(\Sigma_0)}{16\pi}}.
\end{equation}
\end{theorem}

\begin{redattack}
\textbf{Attack: The error term $\int |\cD_k \psi|^2$ may be too large.}

For a trial spinor that is \textbf{not} a solution to Witten equation:
\begin{equation}
    \cD_k \psi = \text{error term} \ne 0.
\end{equation}

The integral $\int_\Omega |\cD_k \psi|^2 dV$ could be comparable to or larger than $B(\psi, \Sigma_0)$.

\textbf{Quantitative estimate needed:}

Near $\Sigma_0$: $|\cD_k \psi| \sim |\nabla \psi| + |k||\psi| \sim 1/s + |k|$.

If $|k| \sim 1/A(\Sigma_0)^{1/2}$ (Schwarzschild-like), then:
\begin{equation}
    \int_\Omega |\cD_k \psi|^2 dV \sim \int_0^\infty \frac{1}{s^2} \cdot A(\Sigma_0) \, ds = \infty!
\end{equation}

The integral \textbf{diverges} near $\Sigma_0$.
\end{redattack}

\begin{bluedefense}
\textbf{Fix via cutoff:}

Modify the trial spinor to decay rapidly near $\Sigma_0$:
\begin{equation}
    \psi(x) = \chi(s) \cdot \psi_{\text{parallel}}
\end{equation}
where $\chi(s) = 0$ for $s < \epsilon$ and $\chi(s) = 1$ for $s > 2\epsilon$.

Then $|\cD_k \psi| \sim 1/\epsilon$ only in the annular region $\epsilon < s < 2\epsilon$, giving:
\begin{equation}
    \int_\Omega |\cD_k \psi|^2 dV \sim \frac{1}{\epsilon^2} \cdot A(\Sigma_0) \cdot \epsilon = \frac{A(\Sigma_0)}{\epsilon}.
\end{equation}

This still diverges as $\epsilon \to 0$!

\textbf{Alternative:} Use a weighted norm or work in a conformally equivalent setting.
\end{bluedefense}

%% ============================================================================
\section{The Mars-Soria Approach}
%% ============================================================================

\begin{theorem}[Mars-Soria, 2016]\label{thm:mars-soria}
Let $(M, g, k)$ be initial data satisfying DEC with an \textbf{apparent horizon} $\Sigma$ (stable MOTS). Then:
\begin{equation}
    M_{\ADM} \ge \sqrt{\frac{A(\Sigma)}{16\pi}} \cdot \Lambda(\Sigma)
\end{equation}
where $\Lambda(\Sigma) \le 1$ is a geometric factor depending on the stability of $\Sigma$.

For stable MOTS: $\Lambda = 1$ gives the full Penrose inequality.
\end{theorem}

\begin{redattack}
\textbf{Attack: Mars-Soria requires MOTS, not general trapped surface.}

The theorem applies to apparent horizons (stable MOTS), not to arbitrary trapped surfaces with $\theta^+ < 0$.

To apply this to trapped surfaces, we would need to:
\begin{enumerate}
    \item Show there exists a MOTS $\Sigma^*$ enclosing the trapped surface
    \item Use Mars-Soria at $\Sigma^*$: $M \ge \sqrt{A(\Sigma^*)/(16\pi)}$
    \item Prove area dominance: $A(\Sigma_0) \le A(\Sigma^*)$
\end{enumerate}

We're back to needing area dominance!
\end{redattack}

%% ============================================================================
\section{Synthesis and Path Forward}
%% ============================================================================

\subsection{Summary of Spinor Approach Status}

\begin{center}
\begin{tabular}{|l|c|p{6cm}|}
\hline
\textbf{Method} & \textbf{Status} & \textbf{Issue} \\
\hline
Standard Witten & \textcolor{red}{FAILS} & No interior boundary \\
Herzlich & \textcolor{orange}{PARTIAL} & Requires outer-minimizing \\
Conformal blow-up & \textcolor{orange}{PARTIAL} & Existence issues \\
MIT bag boundary & \textcolor{orange}{PARTIAL} & Overdetermined system \\
Trial spinor & \textcolor{red}{FAILS} & Error term diverges \\
Mars-Soria & \textcolor{orange}{PARTIAL} & Requires MOTS, not trapped \\
\hline
\end{tabular}
\end{center}

\subsection{Key Obstruction}

\begin{breakthrough}
\textbf{Fundamental difficulty:}

All spinor methods naturally ``see'' MOTS (where $\theta^+ = 0$) because:
\begin{enumerate}
    \item The spinor boundary conditions are related to null geometry
    \item MOTS is where null expansion vanishes---a natural boundary condition
    \item Trapped surfaces ($\theta^+ < 0$) don't provide a natural spinor condition
\end{enumerate}

\textbf{The spinor approach inherently selects MOTS.} To prove Penrose for trapped surfaces, we either:
\begin{enumerate}
    \item Use spinors at MOTS + area dominance (back to main problem)
    \item Develop new spinor boundary conditions for trapped surfaces (open)
    \item Find a completely different approach
\end{enumerate}
\end{breakthrough}

\subsection{Research Directions}

\textbf{Direction 1: Generalized spinor boundary conditions}

Develop boundary conditions for spinors that are natural for trapped surfaces, not just MOTS. Possibly using $\theta^- < 0$ instead of $\theta^+ = 0$.

\textbf{Direction 2: Two-spinor calculus}

Use 2-spinor (twistor) methods from Penrose's original work. The null geometry is more natural in spinor language.

\textbf{Direction 3: Combine with Jang equation}

Use Jang equation to ``blow up'' at MOTS, then apply spinor methods on the Jang manifold. The mass is preserved, and the geometry becomes more tractable.

\begin{thebibliography}{99}

\bibitem{witten1981} E. Witten, A new proof of the positive energy theorem, \textit{Comm. Math. Phys.} 80 (1981), 381--402.

\bibitem{herzlich1997} M. Herzlich, A Penrose-like inequality for the mass of Riemannian asymptotically flat manifolds, \textit{Comm. Math. Phys.} 188 (1997), 121--133.

\bibitem{marssoria2016} M. Mars and A. Soria, On the Penrose inequality for dust null shells in the Minkowski spacetime of arbitrary dimension, \textit{Classical Quantum Gravity} 33 (2016), 115019.

\bibitem{gibbonshawking1979} G. Gibbons and S. Hawking, Cosmological event horizons, thermodynamics, and particle creation, \textit{Phys. Rev. D} 15 (1977), 2738.

\end{thebibliography}

\end{document}
