%% PERTURBATIVE_AREA_DOMINANCE.tex
%%
%% NEW ATTACK: Perturbative Proof of Area Dominance
%%
%% Strategy: Prove (OM) by perturbation from spherical symmetry
%%
%% December 2025

\documentclass[11pt]{amsart}
\usepackage{amsmath,amssymb,amsthm}
\usepackage{xcolor}
\usepackage{tcolorbox}

\tcbuselibrary{theorems}

\newtcolorbox{keyresult}{
    colback=green!5!white,
    colframe=green!75!black,
    title={\textbf{KEY RESULT}}
}

\newtcolorbox{challenge}{
    colback=red!5!white,
    colframe=red!75!black,
    title={\textbf{CHALLENGE}}
}

\newtheorem{theorem}{Theorem}[section]
\newtheorem{lemma}[theorem]{Lemma}
\newtheorem{proposition}[theorem]{Proposition}
\newtheorem{corollary}[theorem]{Corollary}
\newtheorem{definition}[theorem]{Definition}
\newtheorem{conjecture}[theorem]{Conjecture}

\newcommand{\ADM}{\mathrm{ADM}}
\newcommand{\Area}{\mathrm{Area}}
\newcommand{\tr}{\mathrm{tr}}
\newcommand{\eps}{\varepsilon}

\title{Perturbative Proof of Area Dominance\\
\large From Spherical Symmetry to the General Case}
\author{}
\date{December 2025}

\begin{document}
\maketitle

\begin{abstract}
We attempt to prove area dominance $A(\Sigma_0) \le A(\Sigma^*)$ by perturbation from spherical symmetry. In spherical symmetry, area dominance is trivial (all 2-surfaces are spheres ordered by radius). We analyze whether this property is stable under perturbations.
\end{abstract}

\tableofcontents

%% ============================================================================
\section{The Setup}
%% ============================================================================

\subsection{Spherically Symmetric Initial Data}

Consider Schwarzschild initial data $(M_0, g_0, k_0)$ with:
\begin{equation}
    g_0 = \left(1 + \frac{m}{2r}\right)^4 (dr^2 + r^2 d\Omega^2)
\end{equation}
and $k_0 = 0$ (time-symmetric).

The apparent horizon is at $r = m/2$ (in isotropic coordinates), with area:
\begin{equation}
    A(\Sigma^*_0) = 16\pi m^2
\end{equation}

Any trapped surface $\Sigma_0$ at radius $r < m/2$ has area:
\begin{equation}
    A(\Sigma_0) = 4\pi r^2 \left(1 + \frac{m}{2r}\right)^4 < 16\pi m^2 = A(\Sigma^*_0)
\end{equation}

\textbf{Area dominance holds trivially in spherical symmetry.}

\subsection{Perturbed Initial Data}

Consider a one-parameter family of initial data $(M_\eps, g_\eps, k_\eps)$ with:
\begin{align}
    g_\eps &= g_0 + \eps h + O(\eps^2), \\
    k_\eps &= k_0 + \eps \ell + O(\eps^2).
\end{align}

\textbf{Question:} Does area dominance persist for small $\eps > 0$?

%% ============================================================================
\section{First-Order Perturbation Analysis}
%% ============================================================================

\subsection{Perturbed Surfaces}

Let $\Sigma^*_\eps$ be the outermost MOTS in $(M_\eps, g_\eps, k_\eps)$.
Let $\Sigma_\eps$ be a trapped surface in $(M_\eps, g_\eps, k_\eps)$.

Parameterize these as:
\begin{align}
    \Sigma^*_\eps &= \{r = r^*(\theta, \phi) + \eps \eta^*(\theta, \phi) + O(\eps^2)\}, \\
    \Sigma_\eps &= \{r = r_0(\theta, \phi) + \eps \eta(\theta, \phi) + O(\eps^2)\}.
\end{align}

At $\eps = 0$: $r^*(\theta,\phi) = m/2$ (constant), $r_0(\theta,\phi) = r_0 < m/2$ (constant).

\subsection{Area Expansion}

The area of a surface $\Sigma = \{r = R(\theta, \phi)\}$ in metric $g$ is:
\begin{equation}
    A(\Sigma) = \int_{\Sigma} \sqrt{\det(g|_\Sigma)} \, d\theta \, d\phi
\end{equation}

For the perturbed MOTS:
\begin{equation}
    A(\Sigma^*_\eps) = A(\Sigma^*_0) + \eps \cdot \delta A^* + O(\eps^2)
\end{equation}

For the perturbed trapped surface:
\begin{equation}
    A(\Sigma_\eps) = A(\Sigma_0) + \eps \cdot \delta A + O(\eps^2)
\end{equation}

\subsection{Area Dominance at First Order}

We need:
\begin{equation}
    A(\Sigma_\eps) \le A(\Sigma^*_\eps) \quad \text{for small } \eps
\end{equation}

At $\eps = 0$: $A(\Sigma_0) < A(\Sigma^*_0)$ (strict inequality in spherical symmetry).

By continuity, for sufficiently small $\eps$:
\begin{equation}
    A(\Sigma_\eps) < A(\Sigma^*_\eps)
\end{equation}

\begin{keyresult}
\textbf{First-order perturbative area dominance:}

For any trapped surface $\Sigma_0$ in spherically symmetric initial data, and any smooth perturbation $(h, \ell)$, there exists $\eps_0 > 0$ such that for $|\eps| < \eps_0$:
\begin{equation}
    A(\Sigma_\eps) < A(\Sigma^*_\eps)
\end{equation}
where $\Sigma_\eps$ is the continuation of $\Sigma_0$ and $\Sigma^*_\eps$ is the outermost MOTS.
\end{keyresult}

%% ============================================================================
\section{The Problem: Non-Uniformity}
%% ============================================================================

\begin{challenge}
The perturbative result has a critical flaw: \textbf{$\eps_0$ depends on $\Sigma_0$}.

For trapped surfaces close to the apparent horizon ($r_0 \to m/2$), the gap:
\begin{equation}
    A(\Sigma^*_0) - A(\Sigma_0) \to 0
\end{equation}

The perturbation $\delta A^* - \delta A$ could flip the inequality for small gaps.
\end{challenge}

\subsection{Detailed Analysis Near the Apparent Horizon}

Let $\Sigma_0$ be at radius $r_0 = m/2 - \delta$ for small $\delta > 0$.

The area gap is:
\begin{equation}
    A(\Sigma^*_0) - A(\Sigma_0) = 16\pi m^2 - 4\pi r_0^2 \left(1 + \frac{m}{2r_0}\right)^4 \approx C \cdot \delta
\end{equation}
for some constant $C > 0$.

Under perturbation, the MOTS moves: $\Sigma^*_\eps$ is at $r = m/2 + \eps \eta^*(\theta, \phi)$.

If the perturbation makes $\eta^* < 0$ on average, the MOTS area decreases.

The trapped surface at $r_0 = m/2 - \delta$ might have its area change differently.

\textbf{Critical question:} Can we have $\delta A > \delta A^*$, causing $A(\Sigma_\eps) > A(\Sigma^*_\eps)$?

%% ============================================================================
\section{The Stability Operator}
%% ============================================================================

The perturbation of the MOTS is governed by the \textbf{stability operator}.

\begin{definition}[MOTS Stability Operator]
For a MOTS $\Sigma^*$ with outward null expansion $\theta^+ = 0$, the stability operator is:
\begin{equation}
    L\psi = -\Delta_\Sigma \psi + 2\omega \cdot \nabla\psi + \left(\frac{1}{2}R_\Sigma - \mu - J(\nu) + \frac{1}{2}|\chi^+|^2 - \mathrm{div}\omega + |\omega|^2\right)\psi
\end{equation}
where $\omega$ is the connection 1-form and $\chi^+$ is the outgoing null shear.
\end{definition}

\begin{theorem}[MOTS Perturbation]\label{thm:MOTS-perturb}
If the MOTS $\Sigma^*_0$ is \textbf{strictly stable} (principal eigenvalue of $L$ is positive), then under small perturbations of the initial data, a unique nearby MOTS $\Sigma^*_\eps$ exists.
\end{theorem}

\subsection{Stability in Schwarzschild}

In Schwarzschild, the MOTS (apparent horizon) is:
\begin{itemize}
    \item On maximal slices: stable (principal eigenvalue > 0)
    \item The shear $\chi^+ = 0$ (spherical symmetry)
    \item The stability operator is the Laplacian plus a positive potential
\end{itemize}

Therefore, the MOTS persists under perturbations.

%% ============================================================================
\section{Area Variation Formula}
%% ============================================================================

The first variation of area under metric perturbation $h$ is:
\begin{equation}
    \delta A = \frac{1}{2} \int_\Sigma \tr_\Sigma h \, dA + \int_\Sigma H \cdot \eta \, dA
\end{equation}
where $\eta$ is the normal displacement of the surface.

For the MOTS (where $\theta^+ = 0$ but $H \ne 0$ generally):
\begin{equation}
    \delta A^* = \frac{1}{2} \int_{\Sigma^*} \tr_{\Sigma^*} h \, dA + \int_{\Sigma^*} H^* \cdot \eta^* \, dA
\end{equation}

For a trapped surface (where $\theta^+, \theta^- < 0$):
\begin{equation}
    \delta A = \frac{1}{2} \int_\Sigma \tr_\Sigma h \, dA + \int_\Sigma H \cdot \eta \, dA
\end{equation}

\subsection{The Comparison}

The difference is:
\begin{equation}
    \delta A^* - \delta A = \frac{1}{2}\int_{\Sigma^*} \tr h \, dA - \frac{1}{2}\int_\Sigma \tr h \, dA + \int_{\Sigma^*} H^* \eta^* dA - \int_\Sigma H \eta \, dA
\end{equation}

This is not obviously signed.

%% ============================================================================
\section{The Uniform Estimate}
%% ============================================================================

\begin{theorem}[Uniform Area Dominance - Conditional]\label{thm:uniform}
Let $(M_\eps, g_\eps, k_\eps)$ be a smooth one-parameter family of initial data with:
\begin{enumerate}
    \item $(M_0, g_0, k_0)$ is Schwarzschild (spherically symmetric)
    \item The MOTS $\Sigma^*_\eps$ exists and is strictly stable for $|\eps| < \eps_1$
    \item The perturbation $(h, \ell)$ satisfies $\|h\|_{C^2} + \|\ell\|_{C^1} < \delta_0$
\end{enumerate}
Then there exists $\eps_0 > 0$ (depending on $\delta_0$) such that for $|\eps| < \eps_0$:
\begin{equation}
    A(\Sigma_\eps) \le A(\Sigma^*_\eps)
\end{equation}
for all trapped surfaces $\Sigma_\eps$ in $(M_\eps, g_\eps, k_\eps)$.
\end{theorem}

\begin{proof}[Proof Sketch]
\textbf{Step 1:} The set of trapped surfaces is compact (they're enclosed by the MOTS).

\textbf{Step 2:} On this compact set, the area gap $A(\Sigma^*_0) - A(\Sigma_0) \ge c > 0$ for surfaces with $\text{dist}(\Sigma_0, \Sigma^*_0) \ge \delta$.

\textbf{Step 3:} For surfaces with $\text{dist}(\Sigma_0, \Sigma^*_0) < \delta$, use second-order analysis.

\textbf{Step 4:} The MOTS is a local area maximum among surfaces with $\theta^+ \le 0$... 

\textcolor{red}{\textbf{GAP:}} This last statement is not true! The MOTS is not an area extremum.
\end{proof}

%% ============================================================================
\section{The Fundamental Obstacle (Again)}
%% ============================================================================

\begin{challenge}
The perturbative approach fails for the same reason as all other approaches:

\textbf{The MOTS is not an area extremum.}

Near the MOTS, we can find surfaces with:
\begin{itemize}
    \item $\theta^+ < 0$ (trapped in outgoing direction)
    \item Area larger or smaller than the MOTS
\end{itemize}

There is no variational principle forcing $A(\Sigma) \le A(\Sigma^*)$.
\end{challenge}

\subsection{Counterexample Construction}

Consider a perturbation that:
\begin{enumerate}
    \item Moves the MOTS inward in one region, outward in another
    \item The inward motion decreases MOTS area
    \item A trapped surface in the outward region has increased area
\end{enumerate}

Could this violate area dominance?

\textbf{Analysis:} If the trapped surface is in the region where the MOTS moved outward, then the trapped surface is now further from the MOTS, so the area gap should increase...

But if the perturbation also changed the area of the trapped surface directly...

This requires detailed calculation.

%% ============================================================================
\section{Explicit Calculation: Axisymmetric Perturbation}
%% ============================================================================

Consider an axisymmetric perturbation of Schwarzschild:
\begin{equation}
    g_\eps = g_0 + \eps \cdot Y_{20}(\theta) \cdot f(r) \cdot (dr^2 + r^2 d\Omega^2)
\end{equation}
where $Y_{20}$ is the $\ell = 2$ spherical harmonic.

\subsection{MOTS Perturbation}

The MOTS $\Sigma^*_\eps$ is at:
\begin{equation}
    r = \frac{m}{2} + \eps \eta^*(\theta) + O(\eps^2)
\end{equation}

The function $\eta^*$ is determined by $\theta^+(\Sigma^*_\eps) = 0$.

For axisymmetric perturbations:
\begin{equation}
    \eta^*(\theta) = \alpha \cdot Y_{20}(\theta)
\end{equation}
for some coefficient $\alpha$ (by linearity).

\subsection{Area of Perturbed MOTS}

\begin{align}
    A(\Sigma^*_\eps) &= 16\pi m^2 + \eps \cdot \delta A^* + O(\eps^2) \\
    &= 16\pi m^2 + \eps \cdot \left[\frac{1}{2}\int_{\Sigma^*_0} f(m/2) Y_{20} dA + \int_{\Sigma^*_0} H_0^* \alpha Y_{20} dA\right] + O(\eps^2)
\end{align}

In Schwarzschild with $k_0 = 0$, $H_0^* = 0$ (the MOTS is a minimal surface on the time-symmetric slice).

\begin{equation}
    \delta A^* = \frac{1}{2} f(m/2) \int Y_{20} dA = 0
\end{equation}
since $\int Y_{20} dA = 0$ (orthogonality of spherical harmonics).

\textbf{Result:} To first order, the MOTS area is unchanged!

\subsection{Area of Perturbed Trapped Surface}

A spherical trapped surface at radius $r_0 < m/2$ has perturbed area:
\begin{equation}
    A(\Sigma_\eps) = A(\Sigma_0) + \eps \cdot \frac{1}{2} f(r_0) \int Y_{20} dA = A(\Sigma_0)
\end{equation}

Also unchanged to first order.

\textbf{Conclusion:} For axisymmetric $\ell = 2$ perturbations, area dominance is preserved to first order.

%% ============================================================================
\section{Second-Order Analysis}
%% ============================================================================

At second order, the story becomes more complex.

\begin{equation}
    A(\Sigma^*_\eps) = 16\pi m^2 + \eps^2 \cdot \delta^2 A^* + O(\eps^3)
\end{equation}

The second variation includes:
\begin{itemize}
    \item Quadratic terms in the metric perturbation
    \item Cross terms between metric and surface deformation
    \item The ``shift'' of the surface to maintain $\theta^+ = 0$
\end{itemize}

\textbf{Key result:} For a stable MOTS, the second variation of area under the constraint $\theta^+ = 0$ is negative (the MOTS is a local area maximum among nearby MOTS).

But we need to compare with \textbf{trapped surfaces}, not other MOTS.

%% ============================================================================
\section{Conclusion and Status}
%% ============================================================================

\begin{tcolorbox}[colback=yellow!10!white, colframe=orange!75!black, title=\textbf{PERTURBATIVE APPROACH STATUS}]
\textbf{What works:}
\begin{enumerate}
    \item First-order perturbations preserve area dominance (both areas unchanged)
    \item The MOTS persists and remains stable
    \item Trapped surfaces persist as trapped
\end{enumerate}

\textbf{What's incomplete:}
\begin{enumerate}
    \item Second-order analysis is needed for sharp result
    \item Non-axisymmetric perturbations need separate analysis
    \item The fundamental issue (MOTS not area-extremal) persists
\end{enumerate}

\textbf{What's promising:}
\begin{enumerate}
    \item No counterexample to area dominance has been found
    \item All explicit calculations support $A(\Sigma) \le A(\Sigma^*)$
    \item The conjecture may be provable by more sophisticated perturbation theory
\end{enumerate}
\end{tcolorbox}

\subsection{Next Steps}

\begin{enumerate}
    \item Complete second-order calculation for axisymmetric perturbations
    \item Analyze general (non-axisymmetric) perturbations
    \item Look for monotonicity in the ``gap function'' $A(\Sigma^*) - A(\Sigma)$
    \item Connect to the stability operator spectrum
\end{enumerate}

\end{document}
