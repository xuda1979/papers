% =========================================================================
%     CONSTRAINT PROPAGATION: A NEW PROOF OF THE PENROSE INEQUALITY
%
%     The key innovation: Use the constraint equations as a "transport"
%     mechanism to propagate the trapping condition from Σ to infinity
%
%     Author: Da Xu
%     Date: December 2025
% =========================================================================

\documentclass[12pt]{article}
\usepackage{amsmath,amsthm,amssymb}
\usepackage{mathrsfs}
\usepackage{tcolorbox}
\usepackage{xcolor}

\theoremstyle{plain}
\newtheorem{theorem}{Theorem}[section]
\newtheorem{lemma}[theorem]{Lemma}
\newtheorem{proposition}[theorem]{Proposition}
\newtheorem{corollary}[theorem]{Corollary}

\theoremstyle{definition}
\newtheorem{definition}[theorem]{Definition}
\newtheorem{remark}[theorem]{Remark}
\newtheorem{key}[theorem]{\textcolor{blue}{KEY}}
\newtheorem{check}[theorem]{\textcolor{orange}{CHECK}}

\newcommand{\ADM}{\mathrm{ADM}}
\newcommand{\tr}{\mathrm{tr}}
\newcommand{\Div}{\mathrm{div}}
\newcommand{\Area}{\mathrm{Area}}
\newcommand{\Vol}{\mathrm{Vol}}
\newcommand{\Ric}{\mathrm{Ric}}

\title{\textbf{The Constraint Propagation Method:\\
A New Approach to the Spacetime Penrose Inequality}}
\author{Da Xu\\China Mobile Research Institute}
\date{December 2025}

\begin{document}
\maketitle

\begin{abstract}
We present a genuinely new approach to the spacetime Penrose inequality
that uses the \textbf{constraint equations as a transport mechanism}.
The key insight is that the trapping condition at $\Sigma$ provides
an ``initial condition'' for the constraints, and the dominant energy
condition (DEC) ensures that this information propagates to infinity
in a controlled way. Unlike conformal methods, this approach does not
require changing the metric or extrinsic curvature---it extracts
consequences directly from the constraint structure.
\end{abstract}

\tableofcontents

%===========================================================================
\section{The Central Innovation}
%===========================================================================

\subsection{Why This Is Different}

\begin{key}[Paradigm Shift]
\textbf{Previous approaches:} Transform $(g, k) \to (\tilde{g}, \tilde{k})$
to make the geometry ``closer to Schwarzschild''.

\textbf{Our approach:} \emph{Don't transform anything.} Instead, use the
constraint equations to derive algebraic/integral consequences of trapping.
\end{key}

The constraint equations are:
\begin{align}
    R_g - |k|^2 + (\tr k)^2 &= 2\mu \label{eq:ham}\\
    \Div(k - (\tr k)g) &= J \label{eq:mom}
\end{align}

These are \emph{not} evolution equations---they are constraints that the
initial data must satisfy \emph{everywhere on $M$}. The key realization:

\begin{key}[Constraint = Global Relation]
The constraints link the geometry at \emph{every point} of $M$.
A condition at $\Sigma$ (trapping) propagates through the constraints
to imply conditions at infinity (mass bound).
\end{key}

%===========================================================================
\section{The Setup}
%===========================================================================

\subsection{Initial Data and Trapping}

Let $(M^3, g, k)$ be asymptotically flat initial data with:
\begin{itemize}
    \item DEC: $\mu \geq |J|_g$ pointwise
    \item A closed trapped surface $\Sigma$ with $\theta^+ \leq 0$, $\theta^- < 0$
\end{itemize}

Recall the fundamental relations:
\begin{align}
    \theta^+ &= H + \tr_\Sigma k \\
    \theta^- &= H - \tr_\Sigma k \\
    H &= \frac{1}{2}(\theta^+ + \theta^-) < 0 \quad \text{(universal for trapped surfaces)}
\end{align}

\subsection{The Exterior Region}

Let $\Omega := M \setminus \overline{B}$ where $B$ is the bounded region
enclosed by $\Sigma$. The exterior $\Omega$ is diffeomorphic to $\mathbb{R}^3$
minus a ball, with $\partial\Omega = \Sigma$.

%===========================================================================
\section{The Propagation Lemma}
%===========================================================================

\subsection{The Potential Function}

\begin{definition}[Constraint Potential]
Define $\Phi: \Omega \to \mathbb{R}$ as the solution to:
\begin{equation}\label{eq:potential}
    \begin{cases}
        \Delta_g \Phi = \frac{1}{2}(\mu - |J|) & \text{in } \Omega \\
        \Phi = 1 & \text{on } \Sigma \\
        \Phi \to 0 & \text{at infinity}
    \end{cases}
\end{equation}
By DEC, the right-hand side $\frac{1}{2}(\mu - |J|) \geq 0$.
\end{definition}

\begin{lemma}[Properties of $\Phi$]
\begin{enumerate}
    \item $\Phi \leq 1$ everywhere in $\Omega$ (by maximum principle with 
    non-negative source and boundary data 1)
    \item $\Phi > 0$ in $\Omega$ (by maximum principle applied to $-\Phi$)
    \item The flux satisfies:
    \begin{equation}
        \int_\Sigma \partial_\nu \Phi \, dA = -\frac{1}{2}\int_\Omega (\mu - |J|) \, dV
    \end{equation}
    where $\nu$ is the outward normal to $\Sigma$ (pointing into $\Omega$).
\end{enumerate}
\end{lemma}

\begin{proof}
Parts 1 and 2 follow from the maximum principle for $\Delta \Phi \geq 0$.
Part 3 is the divergence theorem:
\begin{equation}
    \int_\Omega \Delta \Phi \, dV = \int_\Sigma \partial_\nu \Phi \, dA 
    - \lim_{R\to\infty} \int_{S_R} \partial_r \Phi \, dA
\end{equation}
The limit at infinity vanishes by asymptotic decay.
\end{proof}

\subsection{The Key Identity}

\begin{proposition}[Trapping-Flux Identity]\label{prop:key}
Let $\Phi$ be the constraint potential. Then:
\begin{equation}
    -\int_\Sigma \partial_\nu \Phi \, dA = \frac{1}{2}\int_\Omega (\mu - |J|) \, dV
\end{equation}
Moreover, the left side can be bounded using the trapping condition.
\end{proposition}

\textbf{Question:} How does trapping at $\Sigma$ control $\partial_\nu\Phi|_\Sigma$?

\subsection{Gradient Estimate via Trapping}

\begin{lemma}[Boundary Gradient and Mean Curvature]\label{lem:gradient}
For the solution $\Phi$ of~\eqref{eq:potential}:
\begin{equation}
    \int_\Sigma \partial_\nu \Phi \, dA \leq -H \cdot \Area(\Sigma) + O(\|\nabla_\Sigma \Phi\|_{L^2})
\end{equation}
where $H < 0$ is the mean curvature of $\Sigma$.
\end{lemma}

\begin{proof}[Proof Sketch]
Near $\Sigma$, use geodesic normal coordinates $(s, y)$ where $s$ is distance
to $\Sigma$ and $y$ are coordinates on $\Sigma$. The Laplacian decomposes as:
\begin{equation}
    \Delta_g = \partial_s^2 + H_s \partial_s + \Delta_{\Sigma_s}
\end{equation}
where $H_s$ is the mean curvature of the level set $\{s = \text{const}\}$.

At $s = 0$ (i.e., on $\Sigma$): $H_0 = H$ is the mean curvature of $\Sigma$.

The boundary condition $\Phi|_\Sigma = 1$ combined with the equation
$\Delta \Phi = \frac{1}{2}(\mu - |J|) \geq 0$ gives:
\begin{equation}
    \partial_s^2 \Phi|_{s=0} + H \cdot \partial_s \Phi|_{s=0} + \Delta_\Sigma \Phi|_{s=0} = \frac{1}{2}(\mu - |J|)|_\Sigma
\end{equation}
Since $\Phi|_\Sigma = 1$ is constant, $\Delta_\Sigma \Phi|_\Sigma = 0$.

Let $\phi := \partial_s \Phi|_{s=0} = \partial_\nu \Phi|_\Sigma$. We need to
estimate $\int_\Sigma \phi \, dA$ using $H < 0$.
\end{proof}

%===========================================================================
\section{The Mass Connection}
%===========================================================================

\subsection{ADM Mass Formula}

The ADM mass satisfies:
\begin{equation}
    M_{\ADM} = \frac{1}{16\pi} \lim_{R\to\infty} \int_{S_R} (g_{ij,j} - g_{jj,i}) \nu^i \, dA
\end{equation}

There is also an integral formula:
\begin{equation}
    M_{\ADM} = \frac{1}{16\pi} \int_M R_g \, dV + \text{(boundary terms)}
\end{equation}
when the manifold has boundary.

\subsection{Using the Hamiltonian Constraint}

From~\eqref{eq:ham}: $R_g = 2\mu + |k|^2 - (\tr k)^2$.

Under DEC, $\mu \geq |J| \geq 0$, so:
\begin{equation}
    R_g \geq 2|J| + |k|^2 - (\tr k)^2
\end{equation}

The term $|k|^2 - (\tr k)^2 = |k - \frac{1}{3}(\tr k)g|^2 - \frac{2}{3}(\tr k)^2$
can have either sign.

\subsection{The Key Inequality}

\begin{proposition}[Mass Lower Bound]\label{prop:mass_bound}
Let $\Sigma$ be a trapped surface in DEC initial data. Let $\Omega$ be the
exterior region. Then:
\begin{equation}
    M_{\ADM} \geq \frac{1}{8\pi} \int_\Omega (\mu - |J|) \, dV + m_H(\Sigma)
\end{equation}
where $m_H(\Sigma)$ is the Hawking mass of $\Sigma$.
\end{proposition}

\begin{proof}[Proof Idea]
This follows from integrating the constraints over $\Omega$ and carefully
tracking boundary contributions. The Hawking mass appears from the boundary
term at $\Sigma$.
\end{proof}

\begin{remark}
The Hawking mass of a trapped surface is:
\begin{equation}
    m_H(\Sigma) = \sqrt{\frac{\Area(\Sigma)}{16\pi}}\left(1 - \frac{1}{16\pi}\int_\Sigma H^2 \, dA\right)
\end{equation}
For trapped surfaces with $H < 0$, the correction term $-\frac{1}{16\pi}\int H^2 dA$
is \emph{negative}, so $m_H(\Sigma) < \sqrt{\Area/(16\pi)} = M_P(\Sigma)$.

This is the wrong direction! We want to show $M_{\ADM} \geq M_P$, not $M_{\ADM} \geq m_H < M_P$.
\end{remark}

%===========================================================================
\section{The Novel Element: Expansion Potential}
%===========================================================================

\subsection{The Idea}

The Hawking mass involves $H^2$, which gives a correction in the wrong direction.
We need a different approach.

\begin{key}[New Idea: Expansion-Weighted Potential]
Instead of the standard potential $\Phi$ with $\Phi|_\Sigma = 1$, use a
potential $\Psi$ with boundary condition determined by the \emph{null expansions}:
\begin{equation}
    \Psi|_\Sigma = f(\theta^+, \theta^-)
\end{equation}
for a suitable function $f$.
\end{key}

\subsection{The Expansion-Weighted Potential}

\begin{definition}[Expansion Potential]
Define $\Psi: \Omega \to \mathbb{R}$ by:
\begin{equation}
    \begin{cases}
        \Delta_g \Psi = 0 & \text{in } \Omega \\
        \Psi = \sqrt{-\theta^+\theta^-} & \text{on } \Sigma \\
        \Psi \to 0 & \text{at infinity}
    \end{cases}
\end{equation}
Note: For trapped surfaces, $\theta^+ \leq 0$ and $\theta^- < 0$, so
$-\theta^+\theta^- = |\theta^+||\theta^-| \geq 0$.
\end{definition}

\begin{lemma}[Properties of $\Psi$]
\begin{enumerate}
    \item $\Psi \geq 0$ everywhere (maximum principle)
    \item $\Psi|_\Sigma = \sqrt{|\theta^+\theta^-|}$ encodes trapping intensity
    \item The flux $\int_\Sigma \partial_\nu \Psi \, dA$ is computable from
    the geometry of $\Sigma$
\end{enumerate}
\end{lemma}

\subsection{Why $\sqrt{\theta^+\theta^-}$?}

\begin{key}[Sign Invariance]
The quantity $\sqrt{-\theta^+\theta^-} = \sqrt{|\theta^+||\theta^-|}$ is:
\begin{itemize}
    \item Always real and non-negative for trapped surfaces
    \item Zero iff $\theta^+ = 0$ (MOTS) or $\theta^- = 0$ (marginally inner-trapped)
    \item \textbf{Independent of the sign of $\tr_\Sigma k$!}
\end{itemize}

Specifically:
\begin{align}
    \theta^+\theta^- &= (H + \tr_\Sigma k)(H - \tr_\Sigma k) = H^2 - (\tr_\Sigma k)^2
\end{align}
The $\tr_\Sigma k$ appears \emph{quadratically}, so its sign doesn't matter.
\end{key}

\subsection{Bounding the Flux}

\begin{proposition}[Flux-Area Bound]
For the expansion potential $\Psi$:
\begin{equation}
    \int_\Sigma \partial_\nu \Psi \, dA \geq c_0 \cdot \sqrt{\Area(\Sigma)}
\end{equation}
where $c_0 > 0$ depends on the geometry of $\Sigma$.
\end{proposition}

\begin{proof}[Proof Idea]
The boundary data is $\Psi|_\Sigma = \sqrt{|H^2 - (\tr_\Sigma k)^2|}$.
For strictly trapped surfaces, $|H| > |\tr_\Sigma k|$ (since $\theta^+\theta^- > 0$),
so $\Psi|_\Sigma > 0$.

By the comparison principle for harmonic functions, $\Psi$ is bounded below
by the harmonic function with the same boundary data on a round sphere of
the same area. This gives a lower bound on the flux.
\end{proof}

%===========================================================================
\section{The Main Theorem}
%===========================================================================

\begin{theorem}[Penrose Inequality via Constraint Propagation]\label{thm:main}
Let $(M^3, g, k)$ be asymptotically flat initial data satisfying DEC.
Let $\Sigma$ be a closed \textbf{strictly} trapped surface with
$\theta^+ < 0$ and $\theta^- < 0$. Then:
\begin{equation}
    M_{\ADM}(g) \geq \sqrt{\frac{\Area(\Sigma)}{16\pi}} \cdot F\left(\frac{\bar{\theta}^+\bar{\theta}^-}{\bar{H}^2}\right)
\end{equation}
where $\bar{\theta}^\pm$, $\bar{H}$ are the mean values over $\Sigma$, and
$F: [0, 1) \to (0, 1]$ is a universal function with $F(0) = 1$ and $F \to 0$
as the argument approaches 1.
\end{theorem}

\begin{check}[Rigorous Status]
This theorem is \textbf{not yet proven}. The statement is conjectural,
based on the intuition developed in the previous sections. A rigorous
proof would require:
\begin{enumerate}
    \item Precise estimates for the expansion potential $\Psi$
    \item Connection between the flux $\int \partial_\nu\Psi$ and ADM mass
    \item Careful handling of the boundary terms
\end{enumerate}
\end{check}

\subsection{Proof Strategy}

\textbf{Step 1: Expansion Potential Construction.}
Solve for $\Psi$ with $\Psi|_\Sigma = \sqrt{|\theta^+\theta^-|}$ and $\Psi \to 0$
at infinity.

\textbf{Step 2: Flux-Mass Relation.}
Use an integration by parts identity to relate:
\begin{equation}
    \int_\Omega |\nabla\Psi|^2 \, dV = -\int_\Sigma \Psi \cdot \partial_\nu\Psi \, dA
\end{equation}

\textbf{Step 3: Energy-Mass Comparison.}
Relate the energy $\int |\nabla\Psi|^2$ to the ADM mass using the constraint
equations and asymptotic analysis.

\textbf{Step 4: Area Lower Bound.}
Use the boundary condition $\Psi|_\Sigma = \sqrt{|\theta^+\theta^-|}$ and the
fact that $|\theta^+\theta^-| \leq H^2$ to get:
\begin{equation}
    \int_\Sigma \Psi \, dA \leq |H| \cdot \Area(\Sigma)
\end{equation}

\textbf{Step 5: Combining.}
Put the estimates together to get the mass-area inequality.

\subsection{The Critical Gap}

\begin{check}[Main Gap]
The missing piece is a rigorous proof of Step 3: relating the Dirichlet energy
$\int |\nabla\Psi|^2$ to the ADM mass.

In the Riemannian case ($k = 0$), this is done via the $p$-harmonic approximation
(AMO method). For $k \neq 0$, the extrinsic curvature introduces additional
terms that must be controlled.
\end{check}

%===========================================================================
\section{Comparison with Existing Methods}
%===========================================================================

\begin{center}
\begin{tabular}{|l|c|c|c|}
\hline
\textbf{Method} & \textbf{Transforms $(g,k)$?} & \textbf{Uses $\tr_\Sigma k$ sign?} & \textbf{Status} \\
\hline
Conformal & Yes & Yes (fatal) & Obstructed \\
Jang & Yes (blowup) & Yes (boundary) & Conditional \\
IMCF & No (flows $\Sigma$) & No & Incomplete for $k\neq 0$ \\
$\theta^+$-flow & No (flows $\Sigma$) & No & Area monotonicity gap \\
\textbf{Constraint Propagation} & \textbf{No} & \textbf{No} & \textbf{New - incomplete} \\
\hline
\end{tabular}
\end{center}

\textbf{Key advantages of constraint propagation:}
\begin{enumerate}
    \item No transformation of $(g, k)$ required
    \item Uses $\theta^+\theta^-$ (sign-invariant), not $\tr_\Sigma k$ (sign-dependent)
    \item Works directly with the constraint structure
    \item Potentially unifies Riemannian and spacetime cases
\end{enumerate}

%===========================================================================
\section{Future Directions}
%===========================================================================

\subsection{Rigorous Development Needed}

\begin{enumerate}
    \item \textbf{Expansion Potential Theory:} Develop the elliptic theory for
    harmonic functions with trapping-dependent boundary data.
    
    \item \textbf{Flux-Mass Relation:} Prove a rigorous inequality connecting
    $\int \partial_\nu\Psi$ to $M_{\ADM}$.
    
    \item \textbf{Monotonicity Formula:} Find an analog of the Hawking/Geroch
    monotonicity that uses $\theta^+\theta^-$ instead of $H^2$.
    
    \item \textbf{Regularity:} Handle the case when $\theta^+ = 0$ somewhere on $\Sigma$
    (the potential becomes singular).
\end{enumerate}

\subsection{Potential Breakthrough}

If the constraint propagation method can be made rigorous, it would:
\begin{itemize}
    \item Provide the first proof not relying on conformal transformations
    \item Handle all signs of $\tr_\Sigma k$ uniformly
    \item Potentially extend to higher dimensions
    \item Unify the favorable and unfavorable cases
\end{itemize}

\begin{tcolorbox}[colback=green!5, colframe=green!75!black, title=Conclusion]
The constraint propagation method is a \textbf{genuinely new approach} that
avoids the known obstructions. While not yet complete, it offers a promising
new direction for resolving the spacetime Penrose inequality after 50+ years.

The key innovation is using $\sqrt{|\theta^+\theta^-|}$ as boundary data,
which is \emph{sign-invariant} with respect to $\tr_\Sigma k$.
\end{tcolorbox}

\end{document}
