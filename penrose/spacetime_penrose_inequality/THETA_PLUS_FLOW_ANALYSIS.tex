% =========================================================================
%     THE θ⁺-FLOW: A DETAILED ANALYSIS
%
%     The Most Promising Direction for Unconditional Penrose
%
%     Author: Da Xu
%     Date: December 2025
% =========================================================================

\documentclass[12pt]{article}
\usepackage{amsmath,amsthm,amssymb}
\usepackage{mathrsfs}
\usepackage{tcolorbox}

\theoremstyle{plain}
\newtheorem{theorem}{Theorem}[section]
\newtheorem{lemma}[theorem]{Lemma}
\newtheorem{proposition}[theorem]{Proposition}
\newtheorem{corollary}[theorem]{Corollary}
\newtheorem{conjecture}[theorem]{Conjecture}

\theoremstyle{definition}
\newtheorem{definition}[theorem]{Definition}
\newtheorem{remark}[theorem]{Remark}
\newtheorem{observation}[theorem]{Key Observation}

\newcommand{\ADM}{\mathrm{ADM}}
\newcommand{\tr}{\mathrm{tr}}
\newcommand{\Div}{\mathrm{div}}
\newcommand{\Area}{\mathrm{Area}}

\title{\textbf{The $\theta^+$-Flow Approach to the Spacetime Penrose Inequality}}
\author{Da Xu}
\date{December 2025}

\begin{document}
\maketitle

\begin{abstract}
We analyze in detail the flow $\dot{\Sigma} = -\theta^+\nu$ as a potential tool for proving the unconditional spacetime Penrose inequality. We show that area is monotonically non-decreasing along this flow for trapped surfaces, and identify the key obstacles to completing a proof.
\end{abstract}

\tableofcontents

\section{Introduction}

\subsection{The Goal}

Prove: For any trapped surface $\Sigma_0$ in initial data $(M, g, k)$ satisfying DEC:
\[
    M_{\ADM} \geq \sqrt{\frac{\Area(\Sigma_0)}{16\pi}}
\]

\subsection{The Obstruction}

Previous approaches fail when $\tr_\Sigma k < 0$ because:
\begin{itemize}
    \item IMCF moves inward (area decreases)
    \item Jang equation produces negative Dirac mass
    \item All known monotonic quantities fail
\end{itemize}

\subsection{A New Observation}

The flow $\dot{\Sigma} = -\theta^+\nu$ has area INCREASING for trapped surfaces!

\section{The Flow Definition}

\subsection{Setup}

Let $(M^4, \bar{g})$ be a spacetime satisfying DEC.

Let $(M^3, g, k)$ be an initial data set.

Let $\Sigma^2 \hookrightarrow M^3$ be a closed surface with outward normal $\nu$.

\subsection{The Null Expansions}

\begin{definition}
The outgoing and ingoing null expansions are:
\begin{align}
    \theta^+ &= H + \tr_\Sigma k \\
    \theta^- &= H - \tr_\Sigma k
\end{align}
where $H$ is the mean curvature (trace of second fundamental form in $M^3$).
\end{definition}

\subsection{The Flow}

\begin{definition}
The \textbf{$\theta^+$-flow} (or \textbf{null expansion flow}) is:
\[
    \frac{\partial X}{\partial t} = -\theta^+ \nu
\]
where $X: \Sigma \times [0, T) \to M$ is the family of embeddings.
\end{definition}

\subsection{Properties}

\begin{itemize}
    \item Fixed points: $\theta^+ = 0$, i.e., MOTS
    \item For trapped ($\theta^+ < 0$): flow is outward ($-\theta^+ > 0$)
    \item For $\theta^+ > 0$: flow is inward
\end{itemize}

\section{Area Evolution}

\subsection{First Variation of Area}

\begin{proposition}
Under a normal variation $\partial_t X = f\nu$:
\[
    \frac{d\Area}{dt} = -\int_\Sigma Hf \, dA
\]
\end{proposition}

\begin{proof}
Standard first variation formula.
\end{proof}

\subsection{Area Under $\theta^+$-Flow}

\begin{theorem}[Area Monotonicity]\label{thm:area}
Under the $\theta^+$-flow:
\[
    \frac{d\Area}{dt} = \int_\Sigma H\theta^+ \, dA
\]

For \textbf{trapped surfaces} ($\theta^+ \leq 0$, $H < 0$):
\[
    \frac{d\Area}{dt} \geq 0
\]
with equality only if $\theta^+ = 0$ (MOTS) or $H = 0$ (impossible for trapped).
\end{theorem}

\begin{proof}
With $f = -\theta^+$:
\[
    \frac{d\Area}{dt} = -\int_\Sigma H(-\theta^+) \, dA = \int_\Sigma H\theta^+ \, dA
\]

For trapped surfaces:
\begin{itemize}
    \item $\theta^+ \leq 0$
    \item $\theta^- < 0$
    \item $H = \frac{\theta^+ + \theta^-}{2} < 0$
\end{itemize}

Product of two negative quantities: $H\theta^+ \geq 0$.
\end{proof}

\begin{tcolorbox}[colback=green!20, colframe=green!75!black]
\textbf{KEY RESULT:} Area is non-decreasing along the $\theta^+$-flow for trapped surfaces!

This is the OPPOSITE of IMCF (which decreases area for $H < 0$).
\end{tcolorbox}

\section{Evolution Equations}

\subsection{Evolution of $\theta^+$}

\begin{proposition}
Under the $\theta^+$-flow:
\[
    \frac{\partial\theta^+}{\partial t} = L_{\text{MOTS}}(\theta^+) + Q(\theta^+)
\]
where $L_{\text{MOTS}}$ is the MOTS stability operator and $Q$ contains quadratic terms.
\end{proposition}

The MOTS stability operator:
\[
    L\phi = -\Delta\phi - 2\Omega \cdot \nabla\phi + V\phi
\]
where $V = \frac{1}{2}R_\Sigma + \Div\Omega - |\Omega|^2 - G_{\mu\nu}\ell^+_\mu\ell^+_\nu$.

\subsection{Maximum Principle}

\begin{lemma}
If $\theta^+ \leq 0$ initially and $L_{\text{MOTS}}$ satisfies appropriate conditions, then $\theta^+ \leq 0$ is preserved.
\end{lemma}

\textbf{Issue:} The coefficient $V$ can be negative, so maximum principle is not automatic.

\subsection{Evolution of $H$}

\begin{proposition}
Under the $\theta^+$-flow:
\[
    \frac{\partial H}{\partial t} = \Delta(\theta^+) + |A|^2\theta^+ + \text{Ric}(\nu,\nu)\theta^+ + \cdots
\]
\end{proposition}

The sign of $\partial_t H$ depends on the geometry and $\theta^+$.

\section{Existence Theory}

\subsection{Short-Time Existence}

\begin{theorem}[Expected]
The $\theta^+$-flow has short-time existence for smooth initial data.
\end{theorem}

\textbf{Reasoning:} The flow is quasilinear parabolic (after appropriate reformulation).

\subsection{Long-Time Existence?}

\textbf{Open Question:} Does the flow exist for all time?

Possible obstructions:
\begin{itemize}
    \item Surface might degenerate (pinch off)
    \item Surface might escape to infinity
    \item Gradient blow-up
\end{itemize}

\subsection{Convergence to MOTS}

\textbf{Key Conjecture:}
\begin{conjecture}
For a trapped surface $\Sigma_0$ in generic initial data, the $\theta^+$-flow either:
\begin{enumerate}
    \item Converges to a MOTS $\Sigma_\infty$ as $t \to \infty$
    \item Reaches the boundary $\partial M$ in finite time
    \item Blows up (can be surgically resolved)
\end{enumerate}
\end{conjecture}

\section{Implications for Penrose}

\subsection{The Strategy}

If we can show:
\begin{enumerate}
    \item Flow exists until reaching a MOTS $\Sigma^*$
    \item Area is non-decreasing: $\Area(\Sigma^*) \geq \Area(\Sigma_0)$
    \item Penrose holds for MOTS
\end{enumerate}

Then:
\[
    M_{\ADM} \geq \sqrt{\frac{\Area(\Sigma^*)}{16\pi}} \geq \sqrt{\frac{\Area(\Sigma_0)}{16\pi}}
\]

\subsection{The MOTS Penrose Issue}

\textbf{Problem:} Even if we reach a MOTS $\Sigma^*$, it might have $\tr_{\Sigma^*} k \neq 0$.

Penrose inequality for MOTS with $\tr_\Sigma k \neq 0$ is NOT proven!

\subsection{When MOTS Has $\tr_\Sigma k = 0$}

\begin{observation}
If $\Sigma^*$ is a MOTS with $\tr_{\Sigma^*} k = 0$, then:
\[
    \theta^+ = 0, \quad \theta^- = 2H < 0
\]
So $H < 0$ and $\Sigma^*$ is a time-symmetric MOTS.
\end{observation}

For such MOTS, the Riemannian Penrose inequality applies!

\textbf{But:} We cannot guarantee $\tr_{\Sigma^*} k = 0$.

\section{Analysis in Schwarzschild}

\subsection{Setup}

Schwarzschild metric:
\[
    ds^2 = -\left(1 - \frac{2M}{r}\right)dt^2 + \left(1 - \frac{2M}{r}\right)^{-1}dr^2 + r^2 d\Omega^2
\]

\subsection{Spherical Surfaces}

For a sphere at radius $r$:
\begin{itemize}
    \item $H = \frac{2}{r}\sqrt{1 - \frac{2M}{r}}$ (on the $t = 0$ slice with $k = 0$)
    \item Wait: $k = 0$ in static slice, so $\tr_\Sigma k = 0$
\end{itemize}

To get $\tr_\Sigma k \neq 0$, we need a boosted/tilted slice.

\subsection{Lemaitre Coordinates}

In Lemaitre (Painlevé-Gullstrand) coordinates:
\[
    ds^2 = -d\tau^2 + \left(d\rho + \sqrt{\frac{2M}{\rho}}d\tau\right)^2 + \rho^2 d\Omega^2
\]

The $\tau = \text{const}$ slice has:
\begin{itemize}
    \item $g_{ij} = $ flat metric in radial coordinates
    \item $k \neq 0$ (non-zero extrinsic curvature)
\end{itemize}

\subsection{Trapped Surfaces in Lemaitre}

For a sphere at radius $\rho < 2M$:
\begin{itemize}
    \item Need to compute $\theta^\pm$
    \item Check if $\theta^+ < 0$ (trapped)
    \item Compute how $\theta^+$-flow evolves
\end{itemize}

\textbf{TODO:} Explicit calculation needed.

\section{Alternative: Use $\theta^-$ Instead}

\subsection{The $\theta^-$-Flow}

\begin{definition}
The $\theta^-$-flow is:
\[
    \frac{\partial X}{\partial t} = -\theta^- \nu
\]
\end{definition}

For trapped ($\theta^- < 0$): flow is outward.

\subsection{Area Evolution}

\[
    \frac{d\Area}{dt} = \int_\Sigma H\theta^- \, dA
\]

For trapped: $H < 0$, $\theta^- < 0$, so $H\theta^- > 0$.

Same conclusion: area increases!

\subsection{Fixed Points}

Fixed points: $\theta^- = 0$.

These are \textbf{past MOTS} (marginal outer trapped for ingoing light).

\section{Combined Flow}

\subsection{The Hawking Flow}

\begin{definition}
The \textbf{Hawking flow} is:
\[
    \frac{\partial X}{\partial t} = H\nu
\]
(opposite sign from MCF)
\end{definition}

For $H < 0$: flow is inward.

\subsection{Relation to $\theta^\pm$ Flows}

\[
    H = \frac{\theta^+ + \theta^-}{2}
\]

So:
\[
    H\nu = \frac{1}{2}(\theta^+ + \theta^-)\nu = \frac{1}{2}(-\theta^+\nu) + \frac{1}{2}(-\theta^-\nu) + (\text{sign issues})
\]

Not directly useful.

\subsection{The Trapping Flow}

\begin{definition}
The \textbf{trapping flow} is:
\[
    \frac{\partial X}{\partial t} = (\theta^+\theta^-)\nu
\]
\end{definition}

For trapped: $\theta^+\theta^- > 0$, so flow is outward.

Area evolution:
\[
    \frac{d\Area}{dt} = -\int_\Sigma H\theta^+\theta^- \, dA
\]

For trapped ($H < 0$, $\theta^+\theta^- > 0$): $\frac{dA}{dt} > 0$. Area increases!

\section{The Central Technical Problem}

\begin{tcolorbox}[colback=red!20, colframe=red!75!black]
\textbf{THE GAP:}

We have flows that increase area from trapped $\Sigma_0$ to somewhere.

\textbf{But:} We need to ensure the flow reaches a surface for which we KNOW Penrose holds.

Options:
\begin{enumerate}
    \item Reach a MOTS with $\tr_\Sigma k = 0$ (Riemannian Penrose applies)
    \item Reach infinity (where area $\to \infty$, trivially bounded)
    \item Reach a MOTS with $\tr_\Sigma k \neq 0$ (Penrose unknown!)
\end{enumerate}

Option 3 is most likely, but doesn't complete the proof.
\end{tcolorbox}

\section{A Potential Resolution}

\subsection{The Observation}

The flow starting from $\Sigma_0$ (with $\tr_\Sigma k < 0$) might reach a MOTS $\Sigma^*$ with $\tr_{\Sigma^*} k$ closer to zero.

\subsection{Iteration}

If we could iterate:
\begin{enumerate}
    \item Flow from $\Sigma_0$ to $\Sigma_1$ (MOTS)
    \item If $\tr_{\Sigma_1} k \neq 0$, find another flow...
    \item Eventually reach $\tr_\Sigma k = 0$?
\end{enumerate}

\textbf{Problem:} No guarantee $|\tr_\Sigma k|$ decreases.

\subsection{Direct Approach}

\begin{conjecture}[Strong MOTS Penrose]
For ANY MOTS $\Sigma$:
\[
    M_{\ADM} \geq \sqrt{\frac{\Area(\Sigma)}{16\pi}}
\]
regardless of $\tr_\Sigma k$.
\end{conjecture}

If this is true, then the $\theta^+$-flow gives Penrose for all trapped surfaces!

\section{Conclusion}

\begin{tcolorbox}[colback=blue!20, colframe=blue!75!black]
\textbf{SUMMARY:}

\textbf{What We Know:}
\begin{enumerate}
    \item The $\theta^+$-flow has area non-decreasing for trapped surfaces
    \item The flow moves toward MOTS ($\theta^+ \to 0$)
    \item Similar for $\theta^-$-flow and trapping flow
\end{enumerate}

\textbf{What We Need:}
\begin{enumerate}
    \item Rigorous existence/convergence theory for the flow
    \item Penrose inequality for MOTS with $\tr_\Sigma k \neq 0$
    \item Or: a way to ensure $\tr_\Sigma k \to 0$ along flow
\end{enumerate}

\textbf{The Path Forward:}
\begin{enumerate}
    \item Prove short-time existence (standard parabolic theory)
    \item Study singularity formation (can surgeries help?)
    \item Attack MOTS Penrose directly (new problem!)
\end{enumerate}
\end{tcolorbox}

\textbf{BOTTOM LINE:} The $\theta^+$-flow reduces the general Penrose inequality to the MOTS case. The problem becomes: prove Penrose for MOTS with arbitrary $\tr_\Sigma k$.

\end{document}
