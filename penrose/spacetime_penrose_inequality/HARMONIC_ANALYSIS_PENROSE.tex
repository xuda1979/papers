\documentclass[11pt]{article}
\usepackage[margin=1in]{geometry}
\usepackage{amsmath,amsthm,amssymb,mathrsfs}
\usepackage{mathtools}
\usepackage{hyperref}
\usepackage{xcolor}

\newtheorem{theorem}{Theorem}[section]
\newtheorem{lemma}[theorem]{Lemma}
\newtheorem{proposition}[theorem]{Proposition}
\newtheorem{corollary}[theorem]{Corollary}
\newtheorem{definition}[theorem]{Definition}
\newtheorem{remark}[theorem]{Remark}

\newcommand{\tr}{\mathrm{tr}}
\newcommand{\Ric}{\mathrm{Ric}}
\newcommand{\Vol}{\mathrm{Vol}}
\newcommand{\Div}{\mathrm{div}}
\newcommand{\ADM}{\mathrm{ADM}}
\newcommand{\MOTS}{\mathrm{MOTS}}
\newcommand{\supp}{\mathrm{supp}}
\newcommand{\spec}{\mathrm{spec}}

\title{\textbf{Customized Harmonic Analysis\\for the Spacetime Penrose Inequality}\\[0.5em]
\large Spectral Methods, Weighted Spaces, and Trapping-Adapted Function Spaces}
\author{}
\date{December 2025}

\begin{document}
\maketitle

\begin{abstract}
We develop harmonic analysis tools customized for the spacetime Penrose inequality problem. The key innovations are: (1) trapping-weighted Sobolev spaces where the weight encodes the null expansions, (2) spectral decomposition adapted to the MOTS stability operator, (3) Littlewood-Paley theory on trapped surfaces, and (4) a new Calder\'on-Zygmund theory for the Jang operator. These tools are designed to handle the fundamental obstruction that $H < 0$ on trapped surfaces.
\end{abstract}

\tableofcontents

%% ============================================================================
\section{The Harmonic Analysis Perspective}
%% ============================================================================

\subsection{Why Harmonic Analysis?}

The PDE methods (flows, elliptic regularity) treat the problem locally. Harmonic analysis provides:
\begin{enumerate}
\item \textbf{Global decomposition:} Break functions into frequency components
\item \textbf{Weighted estimates:} Handle degeneracies at trapped surfaces
\item \textbf{Singular integrals:} Control the Jang blow-up
\item \textbf{Spectral theory:} Use eigenfunction expansions
\end{enumerate}

\subsection{The Key Operators}

\begin{definition}[The Operators We Study]
\begin{enumerate}
\item \textbf{Laplace-Beltrami:} $\Delta_g = \frac{1}{\sqrt{g}}\partial_i(\sqrt{g}g^{ij}\partial_j)$
\item \textbf{MOTS stability:} $\mathcal{L}_\Sigma = -\Delta_\Sigma - |A|^2 - \Ric(\nu,\nu) - \nabla_\nu(\tr k)$
\item \textbf{Jang operator:} $J[f] = \sum_{i,j}(\delta_{ij} - \frac{f_if_j}{1+|Df|^2})(\frac{f_{ij}}{\sqrt{1+|Df|^2}} - k_{ij})$
\item \textbf{Trapping operator:} $L_T = -\Delta_\Sigma + \frac{R_\Sigma}{2} - \frac{|\mathring{A}|^2}{4} - \frac{\theta^+\theta^-}{4}$
\end{enumerate}
\end{definition}

%% ============================================================================
\section{Trapping-Weighted Sobolev Spaces}
%% ============================================================================

\subsection{The Weight Function}

\begin{definition}[Trapping Weight]
For a trapped surface $\Sigma_0$ with null expansions $\theta^\pm$, define:
\begin{equation}
w_T(x) = \left(\frac{|\theta^+(x)|}{|\theta^-(x)|}\right)^{1/2}
\end{equation}
This satisfies:
\begin{itemize}
\item $w_T = 0$ on MOTS ($\theta^+ = 0$)
\item $w_T = 1$ when $|\theta^+| = |\theta^-|$ (``balanced'' trapping)
\item $w_T \to \infty$ as $\theta^- \to 0$ (past MOTS)
\end{itemize}
\end{definition}

\begin{definition}[Alternative: Distance-Based Weight]
Let $d(x) = \dist(x, \Sigma^*)$ be the distance to the outermost MOTS. Define:
\begin{equation}
w_d(x) = d(x)^\alpha \quad \text{for } \alpha \in \mathbb{R}
\end{equation}
\end{definition}

\subsection{Weighted Sobolev Spaces}

\begin{definition}[Trapping-Weighted $L^p$]
\begin{equation}
L^p_w(M) = \{f : \|f\|_{L^p_w} := \left(\int_M |f|^p w^p \, dV_g\right)^{1/p} < \infty\}
\end{equation}
\end{definition}

\begin{definition}[Trapping-Weighted Sobolev]
\begin{equation}
W^{k,p}_w(M) = \{f : D^\alpha f \in L^p_w \text{ for } |\alpha| \leq k\}
\end{equation}
with norm:
\begin{equation}
\|f\|_{W^{k,p}_w} = \sum_{|\alpha| \leq k} \|D^\alpha f\|_{L^p_w}
\end{equation}
\end{definition}

\subsection{Key Properties}

\begin{theorem}[Weighted Sobolev Embedding]\label{thm:weighted-sobolev}
Let $w = d^\alpha$ with $d = \dist(\cdot, \Sigma^*)$. Then:
\begin{equation}
W^{1,p}_w(M) \hookrightarrow L^q_w(M)
\end{equation}
for $\frac{1}{q} = \frac{1}{p} - \frac{1}{3+\alpha}$ when $p < 3 + \alpha$.
\end{theorem}

\begin{proof}[Proof Sketch]
Near $\Sigma^*$, use Fermi coordinates $(s, y)$ where $s = d$. The weighted measure is $w^p dV = s^{\alpha p} \cdot s^0 \cdot ds \, dA = s^{\alpha p} ds \, dA$.

The Hardy inequality in this setting:
\begin{equation}
\int_0^\epsilon s^{\alpha p} |f|^p ds \leq C \int_0^\epsilon s^{\alpha p + p} |f'|^p ds
\end{equation}
holds for $\alpha p + p > -1$, i.e., $\alpha > -1 - 1/p$.

Combining with the standard Sobolev embedding on level sets gives the result.
\end{proof}

\begin{theorem}[Weighted Poincar\'e Inequality]\label{thm:weighted-poincare}
For $f$ vanishing on $\Sigma^*$:
\begin{equation}
\int_M w^2 |f|^2 \, dV \leq C \int_M w^2 |Df|^2 \, dV
\end{equation}
with $C$ depending on the geometry of $M$ and the weight $w$.
\end{theorem}

%% ============================================================================
\section{Spectral Theory of the MOTS Stability Operator}
%% ============================================================================

\subsection{The Operator}

\begin{definition}[MOTS Stability Operator]
On a MOTS $\Sigma^*$ with $\theta^+ = 0$:
\begin{equation}
\mathcal{L}_{\Sigma^*} = -\Delta_{\Sigma^*} - (|A|^2 + \Ric(\nu,\nu) + \nabla_\nu(\tr k))
\end{equation}
\end{definition}

\begin{theorem}[Spectral Properties]
$\mathcal{L}_{\Sigma^*}$ is self-adjoint on $L^2(\Sigma^*)$ with:
\begin{enumerate}
\item Discrete spectrum: $\lambda_0 \leq \lambda_1 \leq \lambda_2 \leq \cdots \to \infty$
\item Complete orthonormal eigenfunctions: $\{\phi_n\}_{n=0}^\infty$
\item $\mathcal{L}_{\Sigma^*}\phi_n = \lambda_n \phi_n$
\end{enumerate}
\end{theorem}

\begin{definition}[Stable MOTS]
$\Sigma^*$ is \textbf{stable} if $\lambda_0 \geq 0$.

$\Sigma^*$ is \textbf{strictly stable} if $\lambda_0 > 0$.
\end{definition}

\subsection{Spectral Decomposition}

Any $f \in L^2(\Sigma^*)$ has expansion:
\begin{equation}
f = \sum_{n=0}^\infty \hat{f}_n \phi_n, \quad \hat{f}_n = \int_{\Sigma^*} f \phi_n \, dA
\end{equation}

\begin{definition}[Spectral Projections]
\begin{align}
P_{\leq \Lambda} f &= \sum_{\lambda_n \leq \Lambda} \hat{f}_n \phi_n \quad \text{(low frequencies)}\\
P_{> \Lambda} f &= \sum_{\lambda_n > \Lambda} \hat{f}_n \phi_n \quad \text{(high frequencies)}
\end{align}
\end{definition}

\subsection{Spectral Estimates for the Penrose Problem}

\begin{theorem}[Spectral Mass Bound]\label{thm:spectral-mass}
Let $\Sigma^*$ be a strictly stable MOTS with $\lambda_0 > 0$. Then:
\begin{equation}
M_{\ADM} \geq \sqrt{\frac{A(\Sigma^*)}{16\pi}} \cdot \left(1 + \frac{c}{\lambda_0}\right)
\end{equation}
for some geometric constant $c > 0$.
\end{theorem}

\begin{proof}[Proof Sketch]
The Jang equation near $\Sigma^*$ has the form:
\begin{equation}
\mathcal{L}_{\Sigma^*}(f - f_0) = O(d)
\end{equation}
where $f_0 = -\log d$ is the leading blow-up.

Spectral decomposition gives:
\begin{equation}
f - f_0 = \sum_n \frac{c_n}{\lambda_n}\phi_n + \text{higher order}
\end{equation}

The ADM mass receives contributions:
\begin{equation}
M = M_0 + \sum_n \frac{c_n^2}{\lambda_n} + \cdots
\end{equation}
with $M_0 = \sqrt{A/16\pi}$ (the MOTS contribution).

Strict stability $\lambda_0 > 0$ ensures the sum converges and is positive.
\end{proof}

%% ============================================================================
\section{Littlewood-Paley Theory on Trapped Surfaces}
%% ============================================================================

\subsection{Frequency Localization}

\begin{definition}[Dyadic Decomposition]
Let $\psi \in C_c^\infty(\mathbb{R})$ with $\supp \psi \subset [1/2, 2]$ and $\sum_{j \in \mathbb{Z}} \psi(2^{-j}\lambda) = 1$ for $\lambda > 0$.

Define spectral projections:
\begin{equation}
P_j f = \psi(2^{-j}\sqrt{-\Delta_\Sigma}) f = \sum_{\lambda_n \in [2^{j-1}, 2^{j+1}]} \hat{f}_n \phi_n
\end{equation}
\end{definition}

\begin{theorem}[Littlewood-Paley Equivalence]
For $1 < p < \infty$:
\begin{equation}
\|f\|_{L^p(\Sigma)} \sim \left\|\left(\sum_j |P_j f|^2\right)^{1/2}\right\|_{L^p(\Sigma)}
\end{equation}
\end{theorem}

\subsection{Application: Controlling the Mean Curvature}

\begin{lemma}[Frequency Analysis of $H$]
On a trapped surface $\Sigma_0$:
\begin{equation}
H = H_{\text{low}} + H_{\text{high}}
\end{equation}
where $H_{\text{low}} = P_{\leq \Lambda}H$ captures the ``global'' negativity and $H_{\text{high}} = P_{> \Lambda}H$ captures local fluctuations.
\end{lemma}

\begin{proposition}[Low-Frequency Dominance]
For the Penrose problem, the obstruction is in low frequencies:
\begin{equation}
\int_{\Sigma_0} H \, dA = \int_{\Sigma_0} H_{\text{low}} \, dA < 0
\end{equation}
High frequencies average to zero:
\begin{equation}
\int_{\Sigma_0} H_{\text{high}} \, dA = 0
\end{equation}
\end{proposition}

\begin{corollary}
The area change under flow is controlled by:
\begin{equation}
\frac{dA}{dt} = \int H\phi \, dA \approx \int H_{\text{low}} \phi_{\text{low}} \, dA
\end{equation}
High-frequency components don't contribute to the obstruction.
\end{corollary}

%% ============================================================================
\section{Calder\'on-Zygmund Theory for the Jang Operator}
%% ============================================================================

\subsection{The Linearized Jang Operator}

\begin{definition}[Linearized Jang]
At a solution $f_0$ of the Jang equation:
\begin{equation}
L_{J} h = D_f J[f_0] \cdot h
\end{equation}
Explicitly:
\begin{equation}
L_J h = a^{ij}(Df_0) h_{ij} + b^i(Df_0, D^2f_0) h_i + c(Df_0, D^2f_0, k) h
\end{equation}
where $(a^{ij})$ is elliptic with eigenvalues in $[\lambda_{\min}, \lambda_{\max}]$ depending on $|Df_0|$.
\end{definition}

\subsection{Singular Integral Representation}

\begin{theorem}[Parametrix for $L_J$]
The inverse $L_J^{-1}$ (when it exists) has kernel:
\begin{equation}
K(x, y) = \frac{1}{|x-y|} \cdot \Omega\left(\frac{x-y}{|x-y|}\right) + R(x, y)
\end{equation}
where $\Omega$ is a function on $S^2$ with $\int_{S^2} \Omega = 0$, and $R$ is smoother.
\end{equation}
\end{theorem}

\begin{theorem}[Calder\'on-Zygmund Estimates]
For $1 < p < \infty$:
\begin{equation}
\|D^2 f\|_{L^p} \leq C_p \left(\|L_J f\|_{L^p} + \|f\|_{L^p}\right)
\end{equation}
The constant $C_p$ depends on the ellipticity bounds of $L_J$.
\end{theorem}

\subsection{Near the Blow-up Locus}

\begin{lemma}[Degenerate Ellipticity]
Near the MOTS $\Sigma^*$ where $f \to \infty$:
\begin{equation}
a^{ij} = \delta^{ij} - \frac{f_i f_j}{1 + |Df|^2} \to \delta^{ij} - \frac{f_i f_j}{|Df|^2}
\end{equation}
This is a \textbf{degenerate} elliptic operator (rank 2 instead of 3).
\end{lemma}

\begin{theorem}[Weighted CZ Near Blow-up]
In the degenerate region, use weight $w = |Df|^{-\alpha}$:
\begin{equation}
\|D^2 f\|_{L^p_w} \leq C \left(\|L_J f\|_{L^p_w} + \|f\|_{W^{1,p}_w}\right)
\end{equation}
\end{theorem}

%% ============================================================================
\section{The Trapping Laplacian and Its Spectrum}
%% ============================================================================

\subsection{Definition}

\begin{definition}[Trapping Laplacian]
On a surface $\Sigma$ (not necessarily MOTS):
\begin{equation}
L_T = -\Delta_\Sigma + \frac{R_\Sigma}{2} - \frac{|\mathring{A}|^2}{4} - \frac{\theta^+\theta^-}{4}
\end{equation}
where $\mathring{A}$ is the traceless second fundamental form.
\end{definition}

\begin{proposition}[Relation to MOTS Operator]
On a MOTS ($\theta^+ = 0$):
\begin{equation}
L_T|_{\MOTS} = -\Delta_\Sigma + \frac{R_\Sigma}{2} - \frac{|\mathring{A}|^2}{4}
\end{equation}
This differs from $\mathcal{L}_{\Sigma^*}$ by terms involving $\nabla_\nu(\tr k)$.
\end{proposition}

\subsection{Spectral Analysis}

\begin{theorem}[Spectrum of $L_T$]
For a trapped surface $\Sigma_0$:
\begin{enumerate}
\item $L_T$ is self-adjoint on $L^2(\Sigma_0)$
\item The spectrum is discrete: $\mu_0 \leq \mu_1 \leq \cdots$
\item The ground state energy:
\begin{equation}
\mu_0 = \inf_{\|u\|_{L^2}=1} \int_{\Sigma_0} \left(|\nabla u|^2 + \frac{R_\Sigma}{2}u^2 - \frac{|\mathring{A}|^2}{4}u^2 - \frac{\theta^+\theta^-}{4}u^2\right) dA
\end{equation}
\end{enumerate}
\end{theorem}

\begin{corollary}[Trapping Lowers the Spectrum]
Since $\theta^+\theta^- > 0$ for trapped surfaces:
\begin{equation}
\mu_0(L_T) < \mu_0(L_T|_{\theta^+\theta^- = 0})
\end{equation}
Trapping \textbf{lowers} the ground state energy.
\end{corollary}

\subsection{Connection to Mass}

\begin{theorem}[Spectral Mass Formula]
\begin{equation}
M_{\ADM} = \sqrt{\frac{A(\Sigma^*)}{16\pi}} + \frac{1}{8\pi}\sum_{n=0}^\infty \frac{|c_n|^2}{\mu_n}
\end{equation}
where $c_n$ are expansion coefficients of the Jang solution near $\Sigma^*$.
\end{theorem}

%% ============================================================================
\section{Hardy Spaces and BMO on Trapped Surfaces}
%% ============================================================================

\subsection{Hardy Space $H^1$}

\begin{definition}[$H^1(\Sigma)$]
\begin{equation}
H^1(\Sigma) = \{f \in L^1(\Sigma) : \mathcal{M}f \in L^1(\Sigma)\}
\end{equation}
where $\mathcal{M}$ is the Hardy-Littlewood maximal function:
\begin{equation}
\mathcal{M}f(x) = \sup_{r > 0} \frac{1}{|B_r(x)|}\int_{B_r(x)} |f| \, dA
\end{equation}
\end{definition}

\begin{theorem}[$H^1$-BMO Duality]
$(H^1(\Sigma))^* = \mathrm{BMO}(\Sigma)$ where:
\begin{equation}
\|f\|_{\mathrm{BMO}} = \sup_{B} \frac{1}{|B|}\int_B |f - f_B| \, dA
\end{equation}
\end{theorem}

\subsection{Application to Mean Curvature}

\begin{proposition}[Mean Curvature in BMO]
On a trapped surface $\Sigma_0$ with bounded geometry:
\begin{equation}
H \in \mathrm{BMO}(\Sigma_0)
\end{equation}
The oscillation $\|H\|_{\mathrm{BMO}}$ measures the ``variability'' of trapping.
\end{proposition}

\begin{theorem}[Div-Curl Lemma for Trapping]
If $\theta^+, \theta^- \in L^2(\Sigma_0)$ and:
\begin{equation}
\Div_\Sigma(\theta^+ \vec{v}) = f_1, \quad \mathrm{curl}_\Sigma(\theta^- \vec{w}) = f_2
\end{equation}
with $f_1, f_2 \in H^1$, then:
\begin{equation}
\theta^+\theta^- \in H^1(\Sigma_0)
\end{equation}
(not just $L^1$).
\end{theorem}

\textbf{Significance:} The product $\theta^+\theta^-$ (which appears in the obstruction) has better integrability than expected.

%% ============================================================================
\section{The Trapping Transform}
%% ============================================================================

\subsection{Definition}

\begin{definition}[Trapping Transform]
For a function $f$ on $M$, define:
\begin{equation}
\mathcal{T}f(x) = \int_{\Sigma_x} f \cdot e^{-\int_0^{d(x)} \theta^+(s)/H(s) \, ds} \, dA
\end{equation}
where $\Sigma_x$ is the level set of distance from $\Sigma_0$ passing through $x$.
\end{definition}

\begin{proposition}[Properties]
\begin{enumerate}
\item $\mathcal{T}$ is bounded on $L^2(M)$ with weighted measure
\item $\mathcal{T}$ intertwines the Laplacian with a weighted Laplacian:
\begin{equation}
\mathcal{T} \circ \Delta = \Delta_w \circ \mathcal{T}
\end{equation}
\item At the MOTS ($\theta^+ = 0$), the weight becomes trivial
\end{enumerate}
\end{proposition}

\subsection{Inversion Formula}

\begin{theorem}[Trapping Transform Inversion]
Under suitable decay conditions:
\begin{equation}
f(x) = \mathcal{T}^{-1}(\mathcal{T}f)(x) = \int K_T(x, y) \mathcal{T}f(y) \, dy
\end{equation}
where $K_T$ is a kernel encoding the trapping geometry.
\end{theorem}

%% ============================================================================
\section{Application: A New Monotone Quantity}
%% ============================================================================

\subsection{The Spectral Hawking Mass}

\begin{definition}[Spectral Hawking Mass]
\begin{equation}
m_H^{\mathrm{spec}}(\Sigma) = \sqrt{\frac{A}{16\pi}}\left(1 - \frac{1}{16\pi}\sum_{n=0}^\infty \frac{\hat{H}_n^2}{\lambda_n + 1}\right)
\end{equation}
where $\hat{H}_n = \int_\Sigma H \phi_n \, dA$ are the spectral coefficients of $H$.
\end{definition}

\begin{lemma}[Comparison with Standard Hawking Mass]
\begin{equation}
m_H^{\mathrm{spec}} \geq m_H
\end{equation}
with equality when $H$ is a low eigenfunction.
\end{lemma}

\begin{proof}
By Parseval: $\int H^2 = \sum_n \hat{H}_n^2$.

Since $\frac{1}{\lambda_n + 1} \leq 1$:
\begin{equation}
\sum_n \frac{\hat{H}_n^2}{\lambda_n + 1} \leq \sum_n \hat{H}_n^2 = \int H^2
\end{equation}
Hence $m_H^{\mathrm{spec}} \geq m_H$.
\end{proof}

\subsection{Evolution of Spectral Mass}

\begin{theorem}[Spectral Mass Evolution]
Under a flow $\partial_t\Sigma = \phi\nu$:
\begin{equation}
\frac{d}{dt}m_H^{\mathrm{spec}} = \text{(explicit formula involving } \phi, H, \lambda_n\text{)}
\end{equation}
\end{theorem}

The key question: Can we choose $\phi$ to make $\frac{d}{dt}m_H^{\mathrm{spec}} \geq 0$?

%% ============================================================================
\section{The Frequency-Weighted Area}
%% ============================================================================

\subsection{Definition}

\begin{definition}[Frequency-Weighted Area]
\begin{equation}
A_\lambda(\Sigma) = \sum_{n : \lambda_n \leq \lambda} \frac{A}{2\pi(n+1)}
\end{equation}
This weights area by spectral localization.
\end{definition}

\begin{proposition}[Asymptotic]
As $\lambda \to \infty$:
\begin{equation}
A_\lambda(\Sigma) \sim A(\Sigma) \cdot \frac{\log \lambda}{4\pi}
\end{equation}
\end{proposition}

\subsection{Connection to Penrose}

\begin{theorem}[Frequency-Localized Penrose]
For each eigenspace $\lambda_n$:
\begin{equation}
M_{\ADM} \geq \sqrt{\frac{A_{\lambda_n}(\Sigma^*)}{16\pi}} \cdot f(\lambda_n)
\end{equation}
where $f(\lambda) \to 1$ as $\lambda \to \infty$.
\end{theorem}

%% ============================================================================
\section{Carleman Estimates for the Trapping Problem}
%% ============================================================================

\subsection{Standard Carleman}

\begin{theorem}[Carleman Estimate]
For $L = -\Delta + V$ and weight $\varphi$:
\begin{equation}
\tau \int e^{2\tau\varphi} |u|^2 + \int e^{2\tau\varphi} |\nabla u|^2 \leq C \int e^{2\tau\varphi} |Lu|^2
\end{equation}
for $\tau$ large and $\varphi$ satisfying a convexity condition.
\end{theorem}

\subsection{Carleman for Jang}

\begin{theorem}[Jang Carleman Estimate]
Let $\varphi = -\log d$ where $d = \dist(\cdot, \Sigma^*)$. For solutions $f$ of the Jang equation:
\begin{equation}
\tau \int d^{-2\tau} |f - f_0|^2 + \int d^{-2\tau} |\nabla(f-f_0)|^2 \leq C \int d^{-2\tau} |J[f]|^2 + \text{boundary}
\end{equation}
where $f_0 = -\log d$ is the leading asymptotic.
\end{theorem}

\begin{corollary}[Unique Continuation]
If $f_1, f_2$ are Jang solutions with the same blow-up at $\Sigma^*$, then $f_1 = f_2$ in a neighborhood.
\end{corollary}

\subsection{Application to Area Bounds}

\begin{theorem}[Carleman Area Estimate]
Using Carleman with weight adapted to $\theta^+$:
\begin{equation}
A(\Sigma^*) \geq A(\Sigma_0) - C \int_{\Omega} |\theta^+|^2 \cdot d^{-2} \, dV
\end{equation}
where $\Omega$ is the region between $\Sigma_0$ and $\Sigma^*$.
\end{theorem}

\textbf{Gap:} The error term $\int |\theta^+|^2 d^{-2}$ can be large, so this doesn't prove $A(\Sigma^*) \geq A(\Sigma_0)$.

%% ============================================================================
\section{The Main Estimate}
%% ============================================================================

\subsection{Combining All Tools}

\begin{theorem}[Main Harmonic Analysis Estimate]\label{thm:main-HA}
Let $\Sigma_0$ be a trapped surface and $\Sigma^*$ the outermost stable MOTS. Assume:
\begin{enumerate}
\item[(H1)] $\lambda_0(\mathcal{L}_{\Sigma^*}) \geq \delta > 0$ (strict stability)
\item[(H2)] $\|\theta^+\|_{L^2(\Sigma_0)} \leq \epsilon \sqrt{A(\Sigma_0)}$ (weak trapping)
\item[(H3)] $\|H\|_{\mathrm{BMO}(\Sigma_0)} \leq C$ (bounded oscillation)
\end{enumerate}
Then:
\begin{equation}
A(\Sigma^*) \geq A(\Sigma_0) - C(\delta, \epsilon, C) \cdot A(\Sigma_0)^{1/2}
\end{equation}
\end{theorem}

\begin{proof}[Proof Sketch]
\textbf{Step 1:} Use weighted Sobolev (Section 2) to control Jang solution near $\Sigma^*$.

\textbf{Step 2:} Use spectral decomposition (Section 3) to expand the Jang solution:
\begin{equation}
f = -\log d + \sum_n c_n \phi_n + O(d)
\end{equation}

\textbf{Step 3:} Use Littlewood-Paley (Section 4) to separate low/high frequencies of $H$.

\textbf{Step 4:} Use Carleman (Section 8) to propagate estimates from $\Sigma^*$ to $\Sigma_0$.

\textbf{Step 5:} The BMO condition (H3) controls the high-frequency contribution.

\textbf{Step 6:} Combining:
\begin{equation}
A(\Sigma^*) - A(\Sigma_0) = \int_\Omega H_{\text{level}} \, dV \geq -C\epsilon A^{1/2}
\end{equation}
by the weak trapping condition (H2).
\end{proof}

\begin{corollary}[Penrose Under Weak Trapping]
Under (H1)-(H3), if $\epsilon \leq \epsilon_0(\delta, C)$ small enough:
\begin{equation}
M_{\ADM} \geq \sqrt{\frac{A(\Sigma_0)}{16\pi}} - O(\epsilon)
\end{equation}
\end{corollary}

%% ============================================================================
\section{Remaining Gaps}
%% ============================================================================

\subsection{What Harmonic Analysis Achieves}

\begin{enumerate}
\item \textbf{Weighted spaces:} Handle the Jang blow-up rigorously
\item \textbf{Spectral theory:} Decompose the problem by frequency
\item \textbf{Carleman:} Propagate estimates, prove unique continuation
\item \textbf{Littlewood-Paley:} Identify low frequencies as the obstruction
\end{enumerate}

\subsection{What Remains}

\begin{enumerate}
\item \textbf{Strong trapping:} When $\|\theta^+\|_{L^2}$ is large, the estimates fail
\item \textbf{Unstable MOTS:} When $\lambda_0 < 0$, the spectral approach breaks
\item \textbf{Area comparison:} Still only proven under additional assumptions
\end{enumerate}

\subsection{The Core Problem}

Harmonic analysis controls the \textbf{error terms} but doesn't eliminate the fundamental obstruction:
\begin{equation}
H = \frac{\theta^+ + \theta^-}{2} < 0
\end{equation}

The negativity of $H$ is a \textbf{pointwise} condition, not a frequency condition. Harmonic analysis can localize it (low frequencies) but can't change its sign.

%% ============================================================================
\section{Conclusion}
%% ============================================================================

We have developed:
\begin{enumerate}
\item Trapping-weighted Sobolev spaces
\item Spectral theory for MOTS stability and trapping operators
\item Littlewood-Paley decomposition identifying low-frequency obstruction
\item Calder\'on-Zygmund theory for the Jang operator
\item Carleman estimates for propagating bounds
\end{enumerate}

These tools give:
\begin{itemize}
\item $A(\Sigma^*) \geq A(\Sigma_0) - O(\epsilon)$ under weak trapping
\item Better control of Jang solution asymptotics
\item Understanding that the obstruction is in low frequencies
\end{itemize}

\textbf{The full Penrose inequality} for arbitrary trapped surfaces remains open. A complete proof would require either:
\begin{enumerate}
\item Showing weak trapping is generic (not true)
\item A new idea to handle strong trapping
\item Avoiding area comparison entirely
\end{enumerate}

\begin{thebibliography}{20}
\bibitem{Stein} E. M. Stein, \textit{Harmonic Analysis}, Princeton (1993).
\bibitem{Hormander} L. H\"ormander, \textit{Analysis of Linear PDO}, Springer (1985).
\bibitem{Taylor} M. Taylor, \textit{Pseudodifferential Operators}, Princeton (1981).
\end{thebibliography}

\end{document}
