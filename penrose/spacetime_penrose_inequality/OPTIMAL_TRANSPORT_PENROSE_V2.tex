%% OPTIMAL_TRANSPORT_PENROSE_V2.tex
%%
%% OPTIMAL TRANSPORT APPROACH TO PENROSE 1973
%%
%% Key idea: Use optimal transport to define a canonical map from
%% any initial data to Schwarzschild, and prove mass decreases.
%%
%% This is inspired by the Brenier theorem and McCann's interpolation.
%%
%% December 2025

\documentclass[11pt]{amsart}
\usepackage{amsmath,amssymb,amsthm}
\usepackage{tcolorbox}
\usepackage{mathrsfs}

\tcbuselibrary{theorems}

\newtcolorbox{maintheorem}{
    colback=green!5!white,
    colframe=green!50!black,
    title={\textbf{MAIN THEOREM}}
}

\newtcolorbox{keylemma}{
    colback=blue!5!white,
    colframe=blue!75!black,
    title={\textbf{KEY LEMMA}}
}

\newtcolorbox{proofstep}{
    colback=gray!5!white,
    colframe=gray!50!black,
    title={\textbf{PROOF STEP}}
}

\newtcolorbox{insight}{
    colback=purple!5!white,
    colframe=purple!75!black,
    title={\textbf{INSIGHT}}
}

\newtcolorbox{attack}{
    colback=red!5!white,
    colframe=red!75!black,
    title={\textbf{RED TEAM ATTACK}}
}

\newtheorem{theorem}{Theorem}
\newtheorem{lemma}[theorem]{Lemma}
\newtheorem{proposition}[theorem]{Proposition}
\newtheorem{corollary}[theorem]{Corollary}
\theoremstyle{definition}
\newtheorem{definition}[theorem]{Definition}
\newtheorem{remark}[theorem]{Remark}

\newcommand{\Area}{\mathrm{Area}}
\newcommand{\Vol}{\mathrm{Vol}}
\newcommand{\divv}{\mathrm{div}}
\DeclareMathOperator{\tr}{tr}
\newcommand{\Sch}{\mathrm{Sch}}

\title{Optimal Transport Approach to Penrose 1973:\\
Mass Comparison via Brenier Maps}
\author{December 2025}

\begin{document}
\maketitle

\begin{abstract}
We develop an optimal transport approach to the Penrose 1973 conjecture. 
The key insight is that mass, like certain transport costs, may satisfy 
comparison inequalities under push-forward by optimal maps. This gives 
a canonical way to compare any initial data to Schwarzschild.
\end{abstract}

%% ============================================================================
\section{Optimal Transport Background}
%% ============================================================================

\begin{definition}[Wasserstein-2 Distance]
For probability measures $\mu, \nu$ on $\mathbb{R}^n$:
\begin{equation}
    W_2(\mu, \nu)^2 = \inf_{\gamma \in \Gamma(\mu, \nu)} 
    \int |x - y|^2 \, d\gamma(x, y)
\end{equation}

where $\Gamma(\mu, \nu)$ is the set of couplings with marginals $\mu, \nu$.
\end{definition}

\begin{theorem}[Brenier]
If $\mu$ is absolutely continuous, the optimal coupling is induced by 
the gradient of a convex function:
\begin{equation}
    \gamma = (\text{id}, \nabla\phi)_\# \mu
\end{equation}

where $\nabla\phi$ pushes $\mu$ to $\nu$.
\end{theorem}

\begin{theorem}[McCann Interpolation]
The geodesic in Wasserstein space is:
\begin{equation}
    \mu_t = ((1-t)\text{id} + t\nabla\phi)_\# \mu_0
\end{equation}

This defines a path from $\mu_0$ to $\mu_1 = \nu$.
\end{theorem}

%% ============================================================================
\section{Mass as a Transport Cost}
%% ============================================================================

\begin{insight}
\textbf{The Core Idea}

ADM mass has the asymptotic form:
\begin{equation}
    M_{\text{ADM}} = \lim_{r \to \infty} \frac{1}{16\pi} \int_{S_r}
    (g_{ij,i} - g_{ii,j}) \nu^j \, dA
\end{equation}

This is determined by the metric's deviation from flat at infinity.

\textbf{Question:} Can we express $M_{\text{ADM}}$ as a transport cost?
\end{insight}

\begin{definition}[Energy Density Measure]
For initial data $(M, g, k)$, define the energy density:
\begin{equation}
    d\rho = \mu \, dV_g + \frac{1}{16\pi}(R_g - |k|^2 + (\tr k)^2) \, dV_g
\end{equation}

For vacuum DEC: $d\rho = \frac{1}{16\pi}R_g \, dV_g$ (Riemannian case).
\end{definition}

\begin{proposition}[Mass as Total Energy]
By the positive mass theorem philosophy:
\begin{equation}
    M_{\text{ADM}} = \int_M d\rho + \text{boundary terms}
\end{equation}

The mass is the "total integrated energy density."
\end{proposition}

%% ============================================================================
\section{Optimal Transport Map to Schwarzschild}
%% ============================================================================

\begin{definition}[Target Schwarzschild]
Given trapped surface area $A$, let:
\begin{equation}
    m_* = \sqrt{\frac{A}{16\pi}}
\end{equation}

Define the target: Schwarzschild initial data with mass $m_*$.

In isotropic coordinates:
\begin{equation}
    g_{\Sch} = \left(1 + \frac{m_*}{2r}\right)^4 \delta
\end{equation}
\end{definition}

\begin{definition}[Optimal Map to Schwarzschild]
For initial data $(M, g, k)$ with trapped surface $\Sigma$ of area $A$, 
define:
\begin{equation}
    T: (M, g) \to (\mathbb{R}^3 \setminus B_{r_*}, g_{\Sch})
\end{equation}

as the optimal transport map with respect to some cost function.

\textbf{Choices for cost:}
\begin{itemize}
    \item $c(x, y) = d_g(x, y)^2$ (geodesic cost)
    \item $c(x, y) = |x - y|^2$ (Euclidean cost in coordinates)
    \item $c(x, y)$ = some geometric functional
\end{itemize}
\end{definition}

%% ============================================================================
\section{Mass Comparison via Optimal Transport}
%% ============================================================================

\begin{maintheorem}
\textbf{Transport Mass Inequality (Conjecture)}

Let $T: (M, g, k) \to (M_{\Sch}, g_{\Sch}, k_{\Sch})$ be the optimal 
transport map with respect to an appropriate cost.

If the initial data $(g, k)$ satisfies DEC and contains a trapped surface 
of area $A$, then:
\begin{equation}
    M_{\text{ADM}}(g, k) \ge M_{\text{ADM}}(g_{\Sch}, k_{\Sch}) = \sqrt{\frac{A}{16\pi}}
\end{equation}

\textbf{Equality holds iff $(g, k) = (g_{\Sch}, k_{\Sch})$.}
\end{maintheorem}

\begin{proofstep}
\textbf{Why Optimal Transport Might Help}

Optimal transport provides:
\begin{enumerate}
    \item A \textbf{canonical map} from any data to Schwarzschild
    \item \textbf{Convexity} properties along transport geodesics
    \item \textbf{Isoperimetric} connections (transport and isoperimetry are related)
\end{enumerate}

The hope: ADM mass satisfies a comparison principle under optimal transport.
\end{proofstep}

%% ============================================================================
\section{The Key Technical Ingredient}
%% ============================================================================

\begin{keylemma}
\textbf{Mass Convexity Along Transport}

Let $\mu_t$ be the geodesic in Wasserstein space from $(g, k)$ to 
$(g_{\Sch}, k_{\Sch})$.

If ADM mass is \textbf{convex} along $\mu_t$:
\begin{equation}
    M_{\text{ADM}}(\mu_t) \le (1-t) M_{\text{ADM}}(\mu_0) + t M_{\text{ADM}}(\mu_1)
\end{equation}

Then:
\begin{equation}
    M_{\text{ADM}}(g, k) = M_{\text{ADM}}(\mu_0) \ge M_{\text{ADM}}(\mu_1) 
    = M_{\text{ADM}}(g_{\Sch}, k_{\Sch})
\end{equation}

if the target has minimal mass.
\end{keylemma}

\begin{attack}
\textbf{Problem: Mass Convexity}

Mass convexity along transport geodesics is NOT obvious.

In fact, for general transport, mass need not be monotone at all!

Need to choose the cost function carefully.
\end{attack}

%% ============================================================================
\section{Special Cost Function: Gravitational Work}
%% ============================================================================

\begin{insight}
\textbf{Geometric Cost}

Instead of squared distance, use a geometrically meaningful cost:
\begin{equation}
    c(x, y) = \text{(mass contribution of transporting $x$ to $y$)}
\end{equation}

If the cost is chosen so that total cost equals the mass difference, 
then optimality gives:
\begin{equation}
    \text{cost}(T) = M_{\text{ADM}}(g, k) - M_{\text{ADM}}(g_{\Sch})
\end{equation}

and cost $\ge 0$ gives Penrose!
\end{insight}

\begin{definition}[Mass-Based Cost]
Define:
\begin{equation}
    c(x, y) = \left|\frac{1}{|x|} - \frac{1}{|y|}\right|
    \cdot (\text{local energy at } x)
\end{equation}

This measures how much "mass" is displaced by the transport.
\end{definition}

\begin{proposition}[Total Cost Interpretation]
If $T$ transports matter from $(M, g)$ to $(M_{\Sch}, g_{\Sch})$, and 
the cost measures "work against gravity," then:
\begin{equation}
    \int c(x, T(x)) \, d\rho(x) \approx M_{\text{ADM}}(g) - M_{\text{ADM}}(g_{\Sch})
\end{equation}

This is because work done against gravity equals energy gained.
\end{proposition}

%% ============================================================================
\section{Rigorous Construction: The Mass Functional}
%% ============================================================================

Let me develop this more rigorously.

\begin{definition}[Mass as Variational Problem]
Define:
\begin{equation}
    E[g] = M_{\text{ADM}}[g] = \lim_{r\to\infty} \frac{1}{16\pi}
    \int_{S_r} (g_{ij,i} - g_{ii,j})\nu^j \, dA
\end{equation}

For $g$ close to Euclidean, in harmonic coordinates:
\begin{equation}
    E[g] = \frac{1}{32\pi} \int_M \left(|\partial g|^2 + O(|h|^2|\partial h|)\right) dV
\end{equation}

where $g = \delta + h$.
\end{definition}

\begin{proposition}[Dirichlet Energy Structure]
For weak-field limit:
\begin{equation}
    M_{\text{ADM}} \approx \frac{1}{32\pi} \int |\nabla h|^2 \, dV
\end{equation}

This has the structure of a Dirichlet energy!
\end{proposition}

\begin{theorem}[Pólya-Szegő for Mass]
If $h$ is replaced by its spherically symmetric rearrangement $h^*$:
\begin{equation}
    \int |\nabla h^*|^2 \le \int |\nabla h|^2
\end{equation}

Therefore, symmetrization decreases the weak-field ADM mass.
\end{theorem}

\begin{attack}
\textbf{Problem: This Only Works for Weak Field}

The Dirichlet approximation $M_{\text{ADM}} \approx \frac{1}{32\pi}\int|\nabla h|^2$ 
is only valid for small perturbations of flat space.

For strong gravity (like near a black hole), this approximation fails.

Need a different approach for the full nonlinear case.
\end{attack}

%% ============================================================================
\section{Nonlinear Extension: Conformal Mass}
%% ============================================================================

\begin{definition}[Conformal Formulation]
Write $g = u^4 \bar{g}$ where $\bar{g}$ is a background metric.

The scalar curvature:
\begin{equation}
    R_g = u^{-5}(-8\Delta_{\bar{g}} u + R_{\bar{g}} u)
\end{equation}

The mass:
\begin{equation}
    M_{\text{ADM}} = -\frac{1}{2\pi} \lim_{r\to\infty} \int_{S_r} 
    \partial_r u \, dA
\end{equation}

(for appropriate normalization)
\end{definition}

\begin{keylemma}
\textbf{Conformal Laplacian and Mass}

For scalar-flat data ($R_g = 0$):
\begin{equation}
    \Delta_{\bar{g}} u = \frac{R_{\bar{g}}}{8} u
\end{equation}

The solution $u$ determines the mass via its asymptotic decay.

\textbf{Key point:} This is an elliptic PDE! The solution is controlled 
by the geometry of $\bar{g}$.
\end{keylemma}

\begin{proposition}[Comparison via Maximum Principle]
If $\bar{g}_1$ and $\bar{g}_2$ satisfy:
\begin{itemize}
    \item Same boundary condition at a trapped surface
    \item $R_{\bar{g}_1} \ge R_{\bar{g}_2}$ everywhere
\end{itemize}

Then by comparison principles:
\begin{equation}
    u_1(r) \ge u_2(r) \quad \text{outside trapped surface}
\end{equation}

This implies:
\begin{equation}
    M_1 \le M_2 \quad (\text{mass comparison})
\end{equation}

\textbf{Wait - the sign!} Need to be careful about signs here.
\end{proposition}

%% ============================================================================
\section{The Correct Statement}
%% ============================================================================

\begin{maintheorem}
\textbf{Mass Comparison via Scalar Curvature}

Let $(M, g)$ and $(M_{\Sch}, g_{\Sch})$ both have:
\begin{itemize}
    \item Same trapped surface boundary condition (area $A$)
    \item $R_g \ge 0$ (from DEC for time-symmetric)
    \item $R_{g_{\Sch}} = 0$ (Schwarzschild is Ricci flat)
\end{itemize}

The conformal factors satisfy:
\begin{equation}
    -8\Delta u + R_{\bar{g}} u = 0 \quad \text{vs} \quad 
    -8\Delta u_{\Sch} = 0
\end{equation}

By the maximum principle applied appropriately:
\begin{equation}
    u(r) \to 1 + \frac{M}{2r} \quad \text{vs} \quad
    u_{\Sch}(r) \to 1 + \frac{M_{\Sch}}{2r}
\end{equation}

with $M \ge M_{\Sch}$ if the comparison works correctly.
\end{maintheorem}

\begin{insight}
\textbf{This is Essentially the Riemannian Penrose Proof!}

The comparison argument above is the heart of existing proofs 
(Huisken-Ilmanen, Bray).

The challenge is extending this to $k \neq 0$.
\end{insight}

%% ============================================================================
\section{Extension to $k \neq 0$: The Full Problem}
%% ============================================================================

\begin{proofstep}
\textbf{The $k \neq 0$ Obstacle}

For general initial data:
\begin{equation}
    16\pi\mu = R - |k|^2 + (\tr k)^2
\end{equation}

Even with $\mu \ge 0$ (DEC), we can have $R < 0$ if $|k|^2 - (\tr k)^2 > 0$.

This breaks the comparison argument.
\end{proofstep}

\begin{insight}
\textbf{The Key Challenge}

For Riemannian Penrose ($k = 0$): $R \ge 0$ from DEC, comparison works.

For general Penrose ($k \neq 0$): $R$ can be negative, comparison fails.

\textbf{We need a different functional that "sees" the DEC correctly.}
\end{insight}

%% ============================================================================
\section{New Functional: Energy-Momentum Tensor}
%% ============================================================================

\begin{definition}[Spacetime Energy Functional]
Instead of comparing scalar curvatures, compare the full energy-momentum:
\begin{equation}
    E[\Sigma] = \int_\Sigma (\mu - J \cdot \nu) \, dA
\end{equation}

where $\nu$ is the outward normal.

By DEC: $E[\Sigma] \ge 0$ for all surfaces.
\end{definition}

\begin{proposition}[Flux Formula]
The ADM mass satisfies:
\begin{equation}
    M_{\text{ADM}} = E[\Sigma_\infty] = \lim_{r\to\infty} E[S_r]
\end{equation}

where the limit is taken along large spheres.
\end{proposition}

\begin{proposition}[Schwarzschild as Ground State]
For Schwarzschild:
\begin{equation}
    E[\Sigma] = 0 \quad \forall \text{ surfaces outside horizon}
\end{equation}

because Schwarzschild is vacuum.

The mass is "stored at infinity" in the asymptotics.
\end{proposition}

%% ============================================================================
\section{Conclusion}
%% ============================================================================

The optimal transport perspective suggests:

\begin{center}
\fbox{\parbox{0.85\textwidth}{
\textbf{Penrose as Comparison Problem}

\begin{enumerate}
    \item Fix trapped surface area $A$
    \item Compare any DEC data to Schwarzschild with same $A$
    \item Show: $M_{\text{ADM}} \ge M_{\Sch} = \sqrt{A/(16\pi)}$
\end{enumerate}

The comparison can be implemented via:
\begin{itemize}
    \item Optimal transport (canonical map)
    \item Variational principles (minimize over path)
    \item PDE comparison (elliptic/parabolic)
\end{itemize}
}}
\end{center}

The key remaining challenge: finding the right functional that:
\begin{itemize}
    \item Incorporates both $g$ and $k$
    \item Responds to DEC (not just scalar curvature)
    \item Admits comparison with Schwarzschild
\end{itemize}

This is the frontier of the problem.

\end{document}
