% =========================================================================
%     DOUBLING AND REFLECTION METHODS FOR THE PENROSE INEQUALITY
%
%     Constructing auxiliary manifolds to cancel the negative mass contribution
%
%     Author: Da Xu
%     Date: December 2025
% =========================================================================

\documentclass[12pt]{article}
\usepackage{amsmath,amsthm,amssymb}
\usepackage{mathrsfs}
\usepackage{tcolorbox}

\theoremstyle{plain}
\newtheorem{theorem}{Theorem}[section]
\newtheorem{lemma}[theorem]{Lemma}
\newtheorem{proposition}[theorem]{Proposition}
\newtheorem{corollary}[theorem]{Corollary}

\theoremstyle{definition}
\newtheorem{definition}[theorem]{Definition}
\newtheorem{remark}[theorem]{Remark}

\newcommand{\ADM}{\mathrm{ADM}}
\newcommand{\tr}{\mathrm{tr}}
\newcommand{\Div}{\mathrm{div}}
\newcommand{\Area}{\mathrm{Area}}

\title{\textbf{Doubling Methods for the Penrose Inequality}}
\author{Da Xu}
\date{December 2025}

\begin{document}
\maketitle

\section{The Doubling Principle}

\subsection{Basic Idea}

The problematic term in the Jang approach is:
\[
    R_{\bar{g}} = R^{\text{reg}} + 2[H]\delta_\Sigma
\]
where $[H] = \tr_\Sigma k < 0$.

\textbf{Strategy:} Construct a doubled manifold where the negative contributions
cancel.

\subsection{Standard Doubling}

For a manifold with boundary $(M, g, \partial M = \Sigma)$:

\begin{definition}
The \textbf{double} of $M$ is:
\[
    \hat{M} = M \cup_\Sigma M' \quad \text{(two copies glued along $\Sigma$)}
\]
with metric $\hat{g}$ extending $g$ by reflection.
\end{definition}

For smooth gluing, we need compatibility conditions on $\Sigma$.

\section{Doubling with Extrinsic Curvature}

\subsection{The Setup}

Let $(M, g, k)$ be initial data with trapped surface $\Sigma_0$.

Consider two copies: $(M, g, k)$ and $(M', g', k')$.

\subsection{Choice of $k'$}

\textbf{Option 1:} $k' = k$ (same extrinsic curvature)
\begin{itemize}
    \item The jump $[H]$ doubles: $[H]_{\text{total}} = 2[H] < 0$
    \item Worse, not better!
\end{itemize}

\textbf{Option 2:} $k' = -k$ (opposite extrinsic curvature)
\begin{itemize}
    \item The jumps cancel: $[H]_{+} + [H]_{-} = \tr_\Sigma k + (-\tr_\Sigma k) = 0$
    \item Promising!
\end{itemize}

\subsection{Analysis of Option 2}

With $k' = -k$:
\begin{itemize}
    \item On $M$: $R^{\text{reg}}_M + 2(\tr_\Sigma k)\delta_\Sigma$
    \item On $M'$: $R^{\text{reg}}_{M'} + 2(-\tr_\Sigma k)\delta_\Sigma$
    \item Total: $R^{\text{reg}}_M + R^{\text{reg}}_{M'} + 0 \cdot \delta_\Sigma$
\end{itemize}

\textbf{The delta-function terms cancel!}

But wait...

\subsection{The Constraint Problem}

The DEC requires:
\[
    \mu = \frac{1}{16\pi}(R + (\tr k)^2 - |k|^2) \geq |J|
\]

On $M$: $\mu \geq 0$ (assumed)

On $M'$: With $k' = -k$:
\begin{align}
    \mu' &= \frac{1}{16\pi}(R + (\tr(-k))^2 - |-k|^2) \\
    &= \frac{1}{16\pi}(R + (\tr k)^2 - |k|^2) \\
    &= \mu \geq 0 \quad \checkmark
\end{align}

The Hamiltonian constraint is preserved!

But the momentum constraint:
\[
    J = \frac{1}{8\pi}\Div(k - (\tr k)g)
\]

On $M'$ with $k' = -k$:
\[
    J' = \frac{1}{8\pi}\Div(-k - (-\tr k)g) = -J
\]

The momentum density flips sign!

\subsection{Consequence}

The doubled manifold $(\hat{M}, \hat{g}, \hat{k})$ has:
\begin{itemize}
    \item Hamiltonian constraint satisfied
    \item \textbf{Discontinuous momentum} at $\Sigma$: $J^+ = J$, $J^- = -J$
\end{itemize}

This creates a \textbf{delta-function in $J$} at $\Sigma$!

The DEC becomes: $\mu \geq |J + [J]\delta_\Sigma|$, which is violated at $\Sigma$.

\section{Attempt: Matching Conditions}

\subsection{Smooth Doubling}

For a smooth doubled manifold, we need:
\begin{align}
    g|_M &= g|_{M'} \quad \text{at } \Sigma \\
    k|_M &= k|_{M'} \quad \text{at } \Sigma
\end{align}

This gives $k' = k$, which doesn't help with the sign.

\subsection{Singular Doubling with Controlled Jump}

Allow a jump in $k$ but control it:

\begin{definition}
A \textbf{k-reflected double} has $k'|_\Sigma = k|_\Sigma - 2(\tr_\Sigma k)g_\Sigma$.
\end{definition}

This reflects the trace while keeping the traceless part.

\begin{lemma}
For k-reflected doubling:
\[
    \tr_{\Sigma} k' = \tr_\Sigma k - 2\tr_\Sigma k \cdot \tr(g_\Sigma^{-1}g_\Sigma) = \tr_\Sigma k - 2\tr_\Sigma k \cdot 2 = -3\tr_\Sigma k
\]
\end{lemma}

Wait, this calculation is wrong. Let me reconsider...

Actually, if $k' = k - 2(\tr_\Sigma k) \cdot (\text{projection})$, then:
\[
    \tr_\Sigma k' = \tr_\Sigma k - 2(\tr_\Sigma k) \cdot \tr(\text{projection})
\]

The projection onto the surface has trace 2, so:
\[
    \tr_\Sigma k' = \tr_\Sigma k - 4\tr_\Sigma k = -3\tr_\Sigma k
\]

This doesn't give $-\tr_\Sigma k$. The algebra doesn't work out for simple reflection.

\section{Alternative: Asymmetric Doubling}

\subsection{Idea}

Instead of gluing $M$ to a copy of itself, glue to a \textbf{different} manifold
$(M', g', k')$ chosen to cancel the bad terms.

\subsection{Requirements}

We need $(M', g', k')$ such that:
\begin{enumerate}
    \item $\partial M' = \Sigma$ (same boundary)
    \item $g'|_\Sigma = g|_\Sigma$ (metrics match)
    \item $\tr_\Sigma k' = -\tr_\Sigma k$ (jumps cancel)
    \item $(M', g', k')$ satisfies DEC
    \item $M_{\ADM}(M') = 0$ or is small
\end{enumerate}

\subsection{Construction Attempt}

Take $M' = \Sigma \times [0, \infty)$ with:
\begin{align}
    g' &= dr^2 + g_\Sigma + O(r) \\
    k' &= -(\tr_\Sigma k) \cdot g_\Sigma / 2 + O(r)
\end{align}

This is asymptotically cylindrical.

\textbf{Problem:} Such $M'$ is not asymptotically flat, so it doesn't have
well-defined ADM mass.

\subsection{Making $M'$ Asymptotically Flat}

Modify $M'$ to be asymptotically flat at the other end:
\[
    g' \sim dr^2 + r^2 d\omega^2 \quad \text{as } r \to \infty
\]

But this requires the total matter content to be zero:
\[
    M_{\ADM}(M') = \frac{1}{16\pi}\int_{M'} (R + \cdots) dV
\]

For this to be zero or positive while having $\tr_\Sigma k' = -\tr_\Sigma k$
at the boundary is very restrictive.

\section{The Bartnik Mass Approach}

\subsection{Bartnik Mass}

\begin{definition}
The \textbf{Bartnik mass} of a bounded region $\Omega$ with boundary data
$(\gamma, H)$ on $\Sigma = \partial\Omega$ is:
\[
    m_B(\Sigma, \gamma, H) = \inf\{M_{\ADM}(M', g') : M' \text{ extends } \Sigma \text{ with DEC}\}
\]
\end{definition}

\subsection{Application}

For a trapped surface $\Sigma_0$ with data $(\gamma_0, H_0)$:
\begin{itemize}
    \item $H_0 < 0$ (trapped condition)
    \item The Bartnik mass $m_B(\Sigma_0, \gamma_0, H_0)$ is the minimum mass
    of any extension
\end{itemize}

\textbf{Conjecture (Bartnik):}
\[
    m_B(\Sigma_0, \gamma_0, H_0) = \sqrt{\frac{\Area(\Sigma_0)}{16\pi}}
\]
for surfaces with $H_0 = 0$ (minimal).

For trapped surfaces, this is unknown.

\subsection{The Fill-In Problem}

\textbf{Question:} Can we fill in $\Sigma_0$ with an interior region having
zero ADM mass contribution?

If yes, then the exterior would carry all the mass, and we might prove Penrose.

\textbf{Problem:} The fill-in problem for trapped surfaces is extremely delicate.
No solution is known.

\section{Reflection Across MOTS}

\subsection{Idea}

The outermost MOTS $\Sigma^*$ is special. Can we reflect across it?

\begin{definition}
The \textbf{MOTS reflection} of $(M, g, k)$ is constructed by:
\begin{enumerate}
    \item Identify the outermost MOTS $\Sigma^*$
    \item Remove the trapped region inside $\Sigma^*$
    \item Glue a reflected copy of the exterior across $\Sigma^*$
\end{enumerate}
\end{definition}

\subsection{Analysis}

At $\Sigma^*$: $\theta^+ = H + \tr_\Sigma k = 0$, so $H = -\tr_\Sigma k$.

After reflection, we have two copies of the exterior, each with $\tr_\Sigma k$
of the same sign.

The jumps add rather than cancel!

\textbf{Conclusion:} Reflection across MOTS doesn't help.

\begin{tcolorbox}[colback=red!10, colframe=red!75!black, title=\textbf{Conclusion: Doubling Methods}]
\textbf{Summary:} Doubling approaches fail because:

\begin{enumerate}
    \item \textbf{Simple doubling:} Doesn't change the sign problem
    \item \textbf{$k \to -k$ reflection:} Creates discontinuity in momentum (violates DEC)
    \item \textbf{Asymmetric doubling:} No known construction with right properties
    \item \textbf{Bartnik fill-in:} Unsolved for trapped surfaces
    \item \textbf{MOTS reflection:} Jumps add, not cancel
\end{enumerate}

\textbf{Status:} No doubling method resolves the unconditional case.
\end{tcolorbox}

\end{document}
