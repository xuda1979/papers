% THE OPTIMAL TRAPPED SURFACE APPROACH
%
% Key insight: Among all trapped surfaces, find one that optimizes
% some functional that makes the Penrose inequality provable.

\documentclass[12pt]{article}
\usepackage{amsmath,amsthm,amssymb}
\usepackage{mathrsfs}
\newtheorem{theorem}{Theorem}
\newtheorem{lemma}{Lemma}
\newtheorem{proposition}{Proposition}
\newtheorem{corollary}{Corollary}
\newtheorem{conjecture}{Conjecture}
\newtheorem{remark}{Remark}
\newtheorem{definition}{Definition}
\newtheorem{problem}{Problem}
\newtheorem{claim}{Claim}

\begin{document}

\title{The Optimal Trapped Surface Approach}
\author{Mathematical Development}
\date{\today}
\maketitle

\section{The Core Observation}

The Penrose inequality involves a SPECIFIC trapped surface $\Sigma_0$.
But the physical content is about BLACK HOLES, which are characterized by 
their ENTIRE trapped region, not a single surface.

\begin{definition}[Trapped Region]
The trapped region $\mathcal{T}$ is the set of all points $p \in M$ such that 
$p$ lies inside some trapped surface.
\end{definition}

\begin{definition}[Outermost MOTS]
The boundary $\partial\mathcal{T} = \Sigma^*$ is called the outermost MOTS 
(marginally outer trapped surface).
\end{definition}

The Penrose inequality should really be stated in terms of $\Sigma^*$, not $\Sigma_0$:

\begin{conjecture}[MOTS Penrose Inequality]
\[
M_{\mathrm{ADM}} \ge \sqrt{\frac{A(\Sigma^*)}{16\pi}}
\]
\end{conjecture}

This is KNOWN to be true (AMO's work). The question is: when does 
$A(\Sigma^*) \ge A(\Sigma_0)$ hold?

\section{The Optimization Perspective}

\subsection{Minimizing the Obstruction}

The obstruction to proving Penrose for general $\Sigma_0$ is the term $\tr_{\Sigma_0} k$.

\begin{definition}[Obstruction Functional]
For a trapped surface $\Sigma$, define:
\[
\mathcal{O}[\Sigma] = -\int_\Sigma \tr_\Sigma k \, dA
\]
\end{definition}

When $\mathcal{O}[\Sigma] \le 0$, we have the "favorable" case.
When $\mathcal{O}[\Sigma] > 0$, we have the "unfavorable" case.

\begin{problem}[Optimal Trapped Surface]
Given a trapped region $\mathcal{T}$, find a trapped surface $\Sigma \subset \mathcal{T}$ 
minimizing $\mathcal{O}[\Sigma]$.
\end{problem}

\subsection{The Variational Problem}

Consider surfaces $\Sigma$ that are:
\begin{itemize}
    \item Trapped: $\theta^+ \le 0$, $\theta^- < 0$
    \item Homologous to $\Sigma_0$ (same topology, bounds same region)
\end{itemize}

We want to minimize:
\[
\mathcal{O}[\Sigma] = -\int_\Sigma \tr_\Sigma k \, dA
\]

\subsection{Euler-Lagrange Equation}

Let $\Sigma_t$ be a variation with velocity $\phi\nu$ where $\nu$ is the outward normal.

\textbf{Variation of $\tr_\Sigma k$}:

Under a normal variation:
\[
\delta_\phi(\tr_\Sigma k) = \phi(\nabla^2\tr k - H \cdot \nu(\tr k) - k^{ij}h_{ij}) + \ldots
\]

This is getting complicated. Let's think differently.

\subsection{A Simpler Functional}

Consider instead:
\[
\mathcal{F}[\Sigma] = A(\Sigma) + \lambda \int_\Sigma \tr_\Sigma k \, dA
\]

for some Lagrange multiplier $\lambda$.

Minimizing $A$ subject to $\int \tr_\Sigma k = c$ is equivalent to finding critical 
points of $\mathcal{F}$.

The first variation:
\[
\delta_\phi \mathcal{F} = \int_\Sigma \phi(H + \lambda \cdot [\text{variation of } \tr k]) \, dA
\]

For a MOTS, $H = -\tr_\Sigma k$ (when $\theta^+ = 0$). So:
\[
H + \lambda \tr_\Sigma k = (1-\lambda)\tr_\Sigma k - \theta^+
\]

If $\lambda = 1$ and $\theta^+ = 0$, this vanishes! So MOTS are critical points of $\mathcal{F}$.

\section{A Key Insight: MOTS Stability}

\subsection{The Stability Operator}

For a MOTS $\Sigma^*$ with $\theta^+ = 0$, the stability is determined by:
\[
L\phi = \delta_\phi \theta^+
\]

A MOTS is \textbf{stable} if $L\phi \ge 0$ for all outward variations $\phi \ge 0$.

\subsection{Connection to Penrose Inequality}

\begin{lemma}[AMO Key Lemma]
If $\Sigma^*$ is a stable MOTS and $\Sigma_0 \subset \mathrm{interior}(\Sigma^*)$ is 
any surface, then $\Sigma_0$ being trapped implies $A(\Sigma^*) \ge A(\Sigma_0)$... 

\textbf{NO, this is FALSE in general!}
\end{lemma}

The issue is that stability of $\Sigma^*$ doesn't constrain the geometry INSIDE 
the trapped region.

\section{A New Functional: The Penrose Mass}

\subsection{Definition}

For a 2-surface $\Sigma$, define the \textbf{Penrose mass}:
\[
M_P[\Sigma] = \sqrt{\frac{A(\Sigma)}{16\pi}}
\]

This is the mass the surface would have if it were the event horizon of a 
Schwarzschild black hole.

\subsection{A New Optimization Problem}

\begin{problem}[Maximum Penrose Mass]
Among all trapped surfaces in the trapped region $\mathcal{T}$, find:
\[
M_P^* = \sup_{\Sigma \subset \mathcal{T}, \text{ trapped}} M_P[\Sigma]
\]
\end{problem}

The Penrose inequality is: $M_{\mathrm{ADM}} \ge M_P[\Sigma_0]$ for any trapped $\Sigma_0$.

This is equivalent to: $M_{\mathrm{ADM}} \ge M_P^*$.

\subsection{What Surface Achieves $M_P^*$?}

The supremum is achieved by the LARGEST trapped surface.

\begin{claim}
The outermost MOTS $\Sigma^*$ achieves $M_P^* = M_P[\Sigma^*]$... 

\textbf{NO, also FALSE!}
\end{claim}

A trapped surface $\Sigma_0$ inside $\Sigma^*$ could have LARGER area than $\Sigma^*$!

\begin{remark}
The trapped region $\mathcal{T}$ is not necessarily convex. A "tentacle" region 
could have a wider cross-section in the interior than at the boundary $\Sigma^*$.
\end{remark}

\section{The Real Question}

The fundamental question is:

\begin{problem}[Area Maximization in Trapped Region]
Given a trapped region $\mathcal{T}$ with boundary MOTS $\Sigma^*$, what is:
\[
A^* = \sup \{ A(\Sigma) : \Sigma \subset \mathcal{T} \text{ is trapped} \}
\]
and how does it relate to $A(\Sigma^*)$?
\end{problem}

\subsection{Case Analysis}

\textbf{Case 1}: $A^* = A(\Sigma^*)$ (MOTS is largest)

Then: $M_P^* = M_P[\Sigma^*]$, and the MOTS Penrose inequality 
$M \ge M_P[\Sigma^*]$ gives the full result.

\textbf{Case 2}: $A^* > A(\Sigma^*)$ (some interior surface is larger)

Then: $M_P^* = \sqrt{A^*/(16\pi)} > M_P[\Sigma^*]$.

The MOTS Penrose inequality gives $M \ge M_P[\Sigma^*] < M_P^*$.
This is NOT sufficient to prove Penrose for the larger surface!

\section{When is Case 2 Possible?}

\subsection{Geometric Condition}

A trapped surface $\Sigma_0$ inside $\Sigma^*$ has larger area when:
\[
A(\Sigma_0) > A(\Sigma^*)
\]

This happens when $\Sigma_0$ is "wider" than $\Sigma^*$ despite being inside.

\subsection{Relation to $\tr_\Sigma k$}

\begin{claim}[Heuristic]
If $\tr_{\Sigma_0} k < 0$ (unfavorable), then the surface is "contracting" 
in the time direction, allowing it to be geometrically larger than the MOTS 
"above" it.
\end{claim}

This is not rigorous, but suggests the unfavorable case is exactly when 
$A(\Sigma_0) > A(\Sigma^*)$ can occur.

\section{A New Conjecture}

\begin{conjecture}[Penrose Area Bound]
Let $\mathcal{T}$ be a trapped region with MOTS boundary $\Sigma^*$. For any 
trapped surface $\Sigma_0 \subset \mathcal{T}$:
\[
M_P[\Sigma_0] \le M_P[\Sigma^*] + \frac{1}{8\pi}\int_{\Sigma_0} |\tr_{\Sigma_0} k| \, dA
\]
\end{conjecture}

This bounds how much larger $\Sigma_0$ can be compared to $\Sigma^*$, in terms 
of the "badness" $|\tr_\Sigma k|$.

If true, combined with $M \ge M_P[\Sigma^*]$, we get:
\[
M + \frac{1}{8\pi}\int_{\Sigma_0} |\tr_{\Sigma_0} k| \, dA \ge M_P[\Sigma_0]
\]

This is a WEAKENED Penrose inequality with an additive correction.

\section{Another Direction: The Inverse Mean Curvature Flow}

\subsection{IMCF Approach}

The Huisken-Ilmanen proof uses Inverse Mean Curvature Flow (IMCF):
\[
\frac{\partial\Sigma_t}{\partial t} = \frac{\nu}{H}
\]

where $\nu$ is the outward normal.

Along IMCF:
\[
\frac{d}{dt} M_H(\Sigma_t) \ge 0
\]

where $M_H$ is the Hawking mass.

\subsection{Problem with Trapped Surfaces}

For trapped surfaces, $H = \theta^+ - \tr_\Sigma k$ could have either sign!

If $H < 0$, then $1/H < 0$, and IMCF moves INWARD, which is wrong.

\subsection{A Modified Flow}

Consider instead:
\[
\frac{\partial\Sigma_t}{\partial t} = \frac{\nu}{|\theta^+|}
\]

This uses the outer null expansion instead of mean curvature.

Along this flow, what happens to Hawking mass?

\begin{align}
    M_H &= \sqrt{\frac{A}{16\pi}}\left(1 - \frac{1}{16\pi}\int H^2\right) \\
    &= \sqrt{\frac{A}{16\pi}}\left(1 - \frac{1}{16\pi}\int (\theta^+ - \tr k)^2\right)
\end{align}

The evolution is complicated because $\tr k$ also evolves along the flow.

\section{The Ultimate Goal: A Flow-Based Proof}

\begin{conjecture}[Penrose Flow Theorem]
There exists a geometric flow starting from any trapped surface $\Sigma_0$ that:
\begin{enumerate}
    \item Maintains or increases the "Penrose mass" $M_P[\Sigma_t]$
    \item Terminates at infinity or a "canonical" surface
    \item Allows computation of $M_{\mathrm{ADM}}$ as a limit
\end{enumerate}
\end{conjecture}

Such a flow would prove: $M_{\mathrm{ADM}} = \lim_{t\to\infty} M_P[\Sigma_t] \ge M_P[\Sigma_0]$.

\section{Summary of Approaches}

\begin{enumerate}
    \item \textbf{Optimal surface}: Find a trapped surface minimizing the obstruction.
    
    \item \textbf{Area comparison}: Bound $A(\Sigma_0)$ vs $A(\Sigma^*)$ with corrections.
    
    \item \textbf{Modified flow}: Design a flow with monotone Penrose mass.
    
    \item \textbf{Weakened inequality}: Prove $M \ge M_P - \text{error}$ and bootstrap.
\end{enumerate}

The most promising for new mathematics seems to be the \textbf{modified flow approach}, 
as it could give a completely new proof structure.

\end{document}
