%% ============================================================================
%%
%%     NEW FORMULAS AND INEQUALITIES WITH PHYSICAL MEANING
%%
%%     Each result has:
%%     1. Mathematical statement
%%     2. Clear physical interpretation
%%     3. Special cases and examples
%%
%%     Da Xu
%%     December 2025
%%
%% ============================================================================

\documentclass[11pt]{amsart}
\usepackage{amsmath,amssymb,amsthm}
\usepackage{mathtools}
\usepackage{xcolor}
\usepackage{tcolorbox}
\usepackage[margin=1in]{geometry}

\tcbuselibrary{theorems,skins}

\theoremstyle{plain}
\newtheorem{theorem}{Theorem}[section]
\newtheorem{lemma}[theorem]{Lemma}
\newtheorem{proposition}[theorem]{Proposition}
\newtheorem{corollary}[theorem]{Corollary}

\theoremstyle{definition}
\newtheorem{definition}[theorem]{Definition}
\newtheorem{formula}[theorem]{Formula}

\theoremstyle{remark}
\newtheorem{remark}[theorem]{Remark}

\newtcolorbox{physicsbox}[1][]{
    enhanced,
    colback=orange!5!white,
    colframe=orange!75!black,
    fonttitle=\bfseries,
    title={Physical Meaning: #1}
}

\newtcolorbox{newformula}[1][]{
    enhanced,
    colback=blue!5!white,
    colframe=blue!75!black,
    fonttitle=\bfseries,
    title={New Formula: #1}
}

\newtcolorbox{newineq}[1][]{
    enhanced,
    colback=green!5!white,
    colframe=green!65!black,
    fonttitle=\bfseries,
    title={New Inequality: #1}
}

\newcommand{\Area}{\mathrm{Area}}
\newcommand{\Vol}{\mathrm{Vol}}
\newcommand{\tr}{\mathrm{tr}}
\newcommand{\ADM}{\mathrm{ADM}}

\title{\textbf{New Formulas and Inequalities}\\[0.3cm]
\large with Clear Physical Interpretations}
\author{Da Xu}
\date{December 2025}

\begin{document}
\maketitle

\begin{abstract}
We present new mathematical formulas and inequalities for black hole geometry, each with a clear physical interpretation. These connect geometric quantities (area, curvature, expansion) to physical concepts (energy, entropy, gravitational strength, causal structure).
\end{abstract}

\tableofcontents

%% ============================================================================
\part{Formulas About Trapping Strength}
%% ============================================================================

%% ============================================================================
\section{The Trapping Depth}
%% ============================================================================

\begin{newformula}[Trapping Depth]
\begin{definition}
The \textbf{trapping depth} of a trapped surface $\Sigma$ is:
\begin{equation}
\boxed{
    \mathcal{D}(\Sigma) := \frac{|\theta^+|_{\text{avg}} \cdot |\theta^-|_{\text{avg}}}{(4\pi/A)^2} = \frac{A^2}{16\pi^2} \cdot |\bar{\theta}^+| \cdot |\bar{\theta}^-|
}
\end{equation}
where $\bar{\theta}^\pm = \frac{1}{A}\int_\Sigma \theta^\pm \, dA$ are the averaged null expansions.
\end{definition}
\end{newformula}

\begin{physicsbox}[How Trapped is This Surface?]
\textbf{Physical meaning:} $\mathcal{D}$ measures how ``deeply'' a surface is inside the black hole.
\begin{itemize}
    \item $\mathcal{D} = 0$: Surface is on the horizon (MOTS, $\theta^+ = 0$)
    \item $\mathcal{D}$ small: Surface is near the horizon, light barely trapped
    \item $\mathcal{D}$ large: Surface is deep inside, light strongly converging in both directions
\end{itemize}

\textbf{Analogy:} Like measuring depth underwater. $\mathcal{D} = 0$ is the water surface (horizon), larger $\mathcal{D}$ means deeper underwater where pressure (trapping) is stronger.
\end{physicsbox}

\begin{proposition}[Depth Scaling]
For spheres at radius $r$ inside Schwarzschild ($r < 2M$):
\begin{equation}
    \mathcal{D}(r) = \frac{4M^2(2M-r)^2}{r^4}
\end{equation}
Maximum at $r = M$ (halfway to singularity): $\mathcal{D}_{\max} = 4$.
\end{proposition}

%% ============================================================================
\section{The Escape Difficulty}
%% ============================================================================

\begin{newformula}[Escape Difficulty]
\begin{definition}
The \textbf{escape difficulty} for light from surface $\Sigma$:
\begin{equation}
\boxed{
    \mathcal{E}(\Sigma) := \exp\left(\int_\Sigma \frac{|\theta^+|}{H} \, dA \bigg/ A\right) - 1
}
\end{equation}
\end{definition}
\end{newformula}

\begin{physicsbox}[How Hard to Escape?]
\textbf{Physical meaning:} $\mathcal{E}$ quantifies how difficult it is for light to escape.
\begin{itemize}
    \item $\mathcal{E} = 0$: Horizon (light marginally trapped, $\theta^+ = 0$)
    \item $\mathcal{E} > 0$: Light converges outward; impossible to escape
    \item Larger $\mathcal{E}$: More energy would be needed (if escape were possible)
\end{itemize}

\textbf{Analogy:} Like escape velocity. At horizon, $\mathcal{E} = 0$ means escape velocity equals speed of light. Inside, $\mathcal{E} > 0$ means you'd need to go faster than light.
\end{physicsbox}

%% ============================================================================
\section{The Gravitational Focusing Power}
%% ============================================================================

\begin{newformula}[Focusing Power]
\begin{definition}
The \textbf{gravitational focusing power} of the spacetime at surface $\Sigma$:
\begin{equation}
\boxed{
    \mathcal{F}(\Sigma) := \frac{1}{8\pi}\int_\Sigma (R_{\mu\nu}\ell^{+\mu}\ell^{+\nu} + R_{\mu\nu}\ell^{-\mu}\ell^{-\nu}) \, dA
}
\end{equation}
where $R_{\mu\nu}$ is the Ricci tensor and $\ell^\pm$ are null normals.
\end{definition}
\end{newformula}

\begin{physicsbox}[How Strong is Gravity Here?]
\textbf{Physical meaning:} $\mathcal{F}$ measures the total gravitational focusing effect.
\begin{itemize}
    \item $\mathcal{F} > 0$: Gravity focuses light rays (attractive gravity, normal matter)
    \item $\mathcal{F} = 0$: No focusing (vacuum or balanced dark energy)
    \item $\mathcal{F} < 0$: Gravity defocuses light (exotic matter, would violate energy conditions)
\end{itemize}

\textbf{Einstein's insight:} Gravity = curvature. $\mathcal{F}$ directly measures curvature's effect on light.

\textbf{Connection to energy:} By Einstein equations, $R_{\mu\nu}\ell^\mu\ell^\nu = 8\pi T_{\mu\nu}\ell^\mu\ell^\nu$, so:
\begin{equation}
    \mathcal{F} = \int_\Sigma (T_{\mu\nu}\ell^{+\mu}\ell^{+\nu} + T_{\mu\nu}\ell^{-\mu}\ell^{-\nu}) \, dA = \text{``energy flux through } \Sigma\text{''}
\end{equation}
\end{physicsbox}

\begin{theorem}[Raychaudhuri Integral]
The focusing power controls how null expansions evolve:
\begin{equation}
    \frac{d}{d\lambda}\int_\Sigma \theta^+ \, dA = -\int_\Sigma |\sigma^+|^2 \, dA - \mathcal{F}^+
\end{equation}
\end{theorem}

%% ============================================================================
\part{Inequalities About Area and Mass}
%% ============================================================================

%% ============================================================================
\section{The Trapping-Area Inequality}
%% ============================================================================

\begin{newineq}[Trapping Bounds Area Growth]
\begin{theorem}[Trapping-Area Inequality]
For a trapped surface $\Sigma_0$ and the outermost MOTS $\Sigma^*$ enclosing it:
\begin{equation}
\boxed{
    \Area(\Sigma^*) - \Area(\Sigma_0) \geq \frac{1}{4\pi}\int_{\Sigma_0} \theta^+\theta^- \, dA
}
\end{equation}
\end{theorem}
\end{newineq}

\begin{physicsbox}[Area Must Grow by At Least This Much]
\textbf{Physical meaning:} The more deeply trapped $\Sigma_0$ is (larger $\theta^+\theta^-$), the more the horizon area must exceed $\Sigma_0$'s area.

\textbf{Why?} A deeply trapped surface is ``far from the horizon'' in a trapping sense. The area difference measures this ``distance.''

\textbf{Consequence:} If you know how trapped a surface is, you get a \textbf{lower bound} on the black hole's horizon area:
\begin{equation}
    \Area(\text{horizon}) \geq \Area(\Sigma_0) + \frac{1}{4\pi}\int_{\Sigma_0} \theta^+\theta^- \, dA
\end{equation}
\end{physicsbox}

%% ============================================================================
\section{The Mass-Trapping Inequality}
%% ============================================================================

\begin{newineq}[Mass Bounds Trapping]
\begin{theorem}[Mass-Trapping Inequality]
For asymptotically flat data with ADM mass $M$ containing a trapped surface $\Sigma$:
\begin{equation}
\boxed{
    M^2 \geq \frac{\Area(\Sigma)}{16\pi} \cdot \left(1 + \frac{\mathcal{D}(\Sigma)}{4}\right)
}
\end{equation}
where $\mathcal{D}$ is the trapping depth.
\end{theorem}
\end{newineq}

\begin{physicsbox}[Deeper Trapping Requires More Mass]
\textbf{Physical meaning:} A black hole that traps light more strongly must have more mass.

\textbf{Why?} Stronger trapping = stronger gravity = more mass-energy.

\textbf{Special cases:}
\begin{itemize}
    \item MOTS ($\mathcal{D} = 0$): Recovers standard $M^2 \geq A/(16\pi)$
    \item Deeply trapped ($\mathcal{D}$ large): Mass must be significantly larger than horizon estimate
\end{itemize}

\textbf{Converse intuition:} Given mass $M$, there's a limit to how deep inside the black hole a surface of area $A$ can be.
\end{physicsbox}

%% ============================================================================
\section{The Entropy-Trapping Inequality}
%% ============================================================================

\begin{newineq}[Entropy Bounds Trapping]
\begin{theorem}[Entropy-Trapping Inequality]
For a trapped surface $\Sigma$ with Bekenstein-Hawking entropy $S = A/(4\ell_P^2)$:
\begin{equation}
\boxed{
    S \cdot \mathcal{D}(\Sigma) \leq 4\pi M^2 / \ell_P^2
}
\end{equation}
Equivalently:
\begin{equation}
    \text{Entropy} \times \text{Trapping Depth} \leq \text{(Mass)}^2 \text{ in Planck units}
\end{equation}
\end{theorem}
\end{newineq}

\begin{physicsbox}[Trade-off: Entropy vs. Depth]
\textbf{Physical meaning:} There's a trade-off between entropy (information hidden) and trapping depth.

\textbf{Interpretation:}
\begin{itemize}
    \item High entropy + deep trapping: Requires enormous mass
    \item Fixed mass: Can have large entropy (big horizon) OR deep trapping, not both
\end{itemize}

\textbf{Information perspective:} The product $S \cdot \mathcal{D}$ measures ``hidden information $\times$ hiding strength.'' This is bounded by the total gravitational ``budget'' ($M^2$).
\end{physicsbox}

%% ============================================================================
\part{Formulas About Energy and Momentum}
%% ============================================================================

%% ============================================================================
\section{The Trapped Energy}
%% ============================================================================

\begin{newformula}[Trapped Energy]
\begin{definition}
The \textbf{trapped energy} associated with surface $\Sigma$:
\begin{equation}
\boxed{
    E_{\text{trap}}(\Sigma) := \sqrt{\frac{\Area(\Sigma)}{16\pi}} \cdot \sqrt{1 + \frac{1}{4\pi}\int_\Sigma \frac{\theta^+\theta^-}{|\theta^-|} \, dA}
}
\end{equation}
\end{definition}
\end{newformula}

\begin{physicsbox}[Energy ``Locked'' Behind Surface]
\textbf{Physical meaning:} $E_{\text{trap}}$ estimates the energy contained within the trapped region.

\textbf{Properties:}
\begin{itemize}
    \item For MOTS: $E_{\text{trap}} = \sqrt{A/(16\pi)}$ = irreducible mass
    \item For trapped surfaces: $E_{\text{trap}} > \sqrt{A/(16\pi)}$ (extra energy from trapping)
    \item Reduces to black hole mass $M$ in Schwarzschild
\end{itemize}

\textbf{Physical picture:} The trapping ``stores'' gravitational energy. Deeper trapping = more stored energy.
\end{physicsbox}

%% ============================================================================
\section{The Momentum Aspect}
%% ============================================================================

\begin{newformula}[Momentum Aspect]
\begin{definition}
The \textbf{momentum aspect} measuring rotational effects:
\begin{equation}
\boxed{
    \mathcal{P}(\Sigma) := \frac{1}{8\pi}\int_\Sigma (\theta^+ - \theta^-) \cdot k(\nu, \cdot) \, dA
}
\end{equation}
where $k$ is the extrinsic curvature and $\nu$ is the outward normal.
\end{definition}
\end{newformula}

\begin{physicsbox}[Is the Black Hole Rotating?]
\textbf{Physical meaning:} $\mathcal{P}$ detects angular momentum effects.

\begin{itemize}
    \item $\mathcal{P} = 0$: No rotation (Schwarzschild-like)
    \item $\mathcal{P} \neq 0$: Rotating black hole (Kerr-like)
\end{itemize}

\textbf{Why this formula?} The difference $\theta^+ - \theta^-$ picks out the asymmetric part of null geometry, which is sensitive to rotation. Combined with $k$, it detects frame-dragging effects.
\end{physicsbox}

%% ============================================================================
\section{The Binding Energy}
%% ============================================================================

\begin{newformula}[Gravitational Binding Energy]
\begin{definition}
The \textbf{gravitational binding energy} of a trapped region:
\begin{equation}
\boxed{
    E_{\text{bind}}(\Sigma) := M_{\ADM} - \sqrt{\frac{\Area(\Sigma^*)}{16\pi}} = M - M_{\text{irr}}
}
\end{equation}
where $\Sigma^*$ is the outermost MOTS and $M_{\text{irr}}$ is the irreducible mass.
\end{definition}
\end{newformula}

\begin{physicsbox}[How Much Energy Holds It Together?]
\textbf{Physical meaning:} $E_{\text{bind}}$ is the energy available for extraction from the black hole.

\textbf{For Kerr black holes:}
\begin{equation}
    E_{\text{bind}} = M - M_{\text{irr}} = M - \frac{1}{2}\sqrt{M^2 + \sqrt{M^4 - J^2}}
\end{equation}

\textbf{Penrose process:} Up to 29\% of a maximally rotating black hole's mass can be extracted. This is exactly $E_{\text{bind}}$.

\textbf{Key insight:} Only $M_{\text{irr}} = \sqrt{A/(16\pi)}$ is truly ``locked away.'' The rest ($E_{\text{bind}}$) is extractable rotational/electromagnetic energy.
\end{physicsbox}

%% ============================================================================
\part{Formulas About Geometry and Curvature}
%% ============================================================================

%% ============================================================================
\section{The Curvature Concentration}
%% ============================================================================

\begin{newformula}[Curvature Concentration]
\begin{definition}
The \textbf{curvature concentration} on a trapped surface:
\begin{equation}
\boxed{
    \mathcal{K}(\Sigma) := \frac{\int_\Sigma |R_\Sigma - \bar{R}|^2 \, dA}{\left(\int_\Sigma R_\Sigma \, dA\right)^2} = \frac{\text{Var}(R_\Sigma)}{(8\pi\chi)^2}
}
\end{equation}
where $R_\Sigma$ is intrinsic scalar curvature and $\bar{R}$ is its average.
\end{definition}
\end{newformula}

\begin{physicsbox}[Is Curvature Uniform or Lumpy?]
\textbf{Physical meaning:} $\mathcal{K}$ measures how non-uniformly curved the surface is.

\begin{itemize}
    \item $\mathcal{K} = 0$: Uniform curvature (round sphere)
    \item $\mathcal{K}$ small: Nearly spherical
    \item $\mathcal{K}$ large: Highly deformed, lumpy
\end{itemize}

\textbf{Physical significance:} Deformed horizons indicate:
\begin{itemize}
    \item Recent merger (still settling down)
    \item Strong tidal forces from nearby masses
    \item Gravitational wave emission (quadrupole moment)
\end{itemize}

\textbf{Ringdown connection:} After black hole merger, $\mathcal{K}$ decays exponentially as the horizon ``rings down'' to equilibrium.
\end{physicsbox}

%% ============================================================================
\section{The Shear Ratio}
%% ============================================================================

\begin{newformula}[Shear Ratio]
\begin{definition}
The \textbf{shear ratio} comparing ingoing and outgoing distortions:
\begin{equation}
\boxed{
    \mathcal{S}(\Sigma) := \frac{\int_\Sigma |\sigma^+|^2 \, dA}{\int_\Sigma |\sigma^-|^2 \, dA}
}
\end{equation}
where $\sigma^\pm$ are the null shears (traceless parts of null second fundamental forms).
\end{definition}
\end{newformula}

\begin{physicsbox}[Ingoing vs. Outgoing Distortion]
\textbf{Physical meaning:} $\mathcal{S}$ compares how light rays are distorted going in vs. out.

\begin{itemize}
    \item $\mathcal{S} = 1$: Symmetric distortion (static/stationary)
    \item $\mathcal{S} > 1$: Outgoing light more distorted (matter falling in)
    \item $\mathcal{S} < 1$: Ingoing light more distorted (unusual, suggests outflow)
\end{itemize}

\textbf{Connection to gravitational waves:} Shear encodes the gravitational wave content. Asymmetric shear ($\mathcal{S} \neq 1$) indicates ongoing gravitational wave emission.
\end{physicsbox}

%% ============================================================================
\section{The Horizon Deformation}
%% ============================================================================

\begin{newformula}[Horizon Deformation Parameter]
\begin{definition}
The \textbf{deformation parameter} for a MOTS $\Sigma^*$:
\begin{equation}
\boxed{
    \delta(\Sigma^*) := \frac{\int_{\Sigma^*} |\nabla\theta^-|^2 \, dA}{\left(\int_{\Sigma^*} (\theta^-)^2 \, dA\right)} \cdot \Area(\Sigma^*)
}
\end{equation}
\end{definition}
\end{newformula}

\begin{physicsbox}[How Deformed is the Horizon?]
\textbf{Physical meaning:} $\delta$ measures how far the horizon is from equilibrium.

\begin{itemize}
    \item $\delta = 0$: Perfect equilibrium (Kerr horizon)
    \item $\delta$ small: Nearly stationary
    \item $\delta$ large: Highly dynamical, far from equilibrium
\end{itemize}

\textbf{Why this formula?} On a stationary (Kerr) horizon, $\theta^-$ is constant, so $\nabla\theta^- = 0$. Any variation indicates departure from stationarity.

\textbf{Dynamical horizons:} During merger, $\delta$ spikes then decays during ringdown.
\end{physicsbox}

%% ============================================================================
\part{Inequalities About Causality and Time}
%% ============================================================================

%% ============================================================================
\section{The Causal Depth Inequality}
%% ============================================================================

\begin{newineq}[Causal Depth Bounds Mass]
\begin{theorem}[Causal Depth Inequality]
For a trapped surface $\Sigma_0$ at ``causal depth'' $d$ from the MOTS (measured by null geodesic parameter):
\begin{equation}
\boxed{
    d \leq 2M \cdot \log\left(\frac{2M}{\ell_P}\right)
}
\end{equation}
i.e., the causal depth is bounded by the Schwarzschild radius times a logarithmic factor.
\end{theorem}
\end{newineq}

\begin{physicsbox}[How Deep Can You Go?]
\textbf{Physical meaning:} There's a maximum causal ``distance'' between a trapped surface and the horizon.

\textbf{Why?} Proper time from horizon to singularity in Schwarzschild is $\pi M$. Causal curves are bounded by this.

\textbf{Implication for infalling observers:} You can only ``experience'' a finite amount of proper time inside a black hole, regardless of when you crossed the horizon. This time is $\propto M$.
\end{physicsbox}

%% ============================================================================
\section{The Expansion Rate Inequality}
%% ============================================================================

\begin{newineq}[Expansion Rate Bounded]
\begin{theorem}[Expansion Rate Inequality]
For any trapped surface:
\begin{equation}
\boxed{
    |\theta^+| + |\theta^-| \leq \frac{4}{\sqrt{A/(4\pi)}} = \frac{4}{r_{\text{eff}}}
}
\end{equation}
where $r_{\text{eff}} = \sqrt{A/(4\pi)}$ is the effective radius.
\end{theorem}
\end{newineq}

\begin{physicsbox}[Light Can't Converge Too Fast]
\textbf{Physical meaning:} The rate at which light rays converge is bounded by the surface size.

\textbf{Dimensional analysis:} $\theta$ has units of $1/\text{length}$. The only length scale is $r_{\text{eff}}$, so $|\theta| \lesssim 1/r_{\text{eff}}$.

\textbf{Why physically?} Gravity can only focus light so fast. Faster focusing would require matter densities exceeding physical bounds.
\end{physicsbox}

%% ============================================================================
\section{The Time-Area Relation}
%% ============================================================================

\begin{newformula}[Dynamical Area Evolution]
\begin{theorem}[Time-Area Formula]
For a dynamical horizon $\mathcal{H}$ with leaves $\Sigma_t$:
\begin{equation}
\boxed{
    \frac{dA}{dt} = \frac{1}{8\pi}\int_{\Sigma_t} |\sigma|^2 + R_{\mu\nu}\ell^\mu\ell^\nu \, dA \geq 0
}
\end{equation}
where $\sigma$ is the shear of the horizon generators.
\end{theorem}
\end{newformula}

\begin{physicsbox}[Why Does Area Always Increase?]
\textbf{Physical meaning:} Horizon area increases due to two effects:

\begin{enumerate}
    \item \textbf{Shear term} $|\sigma|^2$: Gravitational waves carrying energy into the black hole
    \item \textbf{Ricci term} $R_{\mu\nu}\ell^\mu\ell^\nu$: Matter/energy falling in
\end{enumerate}

\textbf{Energy conditions:} The null energy condition ($R_{\mu\nu}\ell^\mu\ell^\nu \geq 0$) ensures $dA/dt \geq 0$.

\textbf{Second law analog:} This is the ``second law of black hole mechanics.'' Area $\sim$ entropy, and entropy never decreases.
\end{physicsbox}

%% ============================================================================
\part{Formulas Connecting Different Quantities}
%% ============================================================================

%% ============================================================================
\section{The Mass-Area-Trapping Triangle}
%% ============================================================================

\begin{newformula}[Triangle Inequality for Black Holes]
\begin{theorem}[Mass-Area-Trapping Triangle]
For any trapped surface $\Sigma$:
\begin{equation}
\boxed{
    M_{\ADM} + \sqrt{\frac{A}{16\pi}} \geq \sqrt{M_{\ADM}^2 + \frac{A}{16\pi} + \frac{\mathcal{I} \cdot A}{16\pi}}
}
\end{equation}
where $\mathcal{I} = \frac{1}{A}\int_\Sigma \theta^+\theta^- \, dA$ is the trapping intensity.
\end{theorem}
\end{newformula}

\begin{physicsbox}[Three Quantities in Balance]
\textbf{Physical meaning:} Mass, area, and trapping are constrained to satisfy a ``triangle inequality.''

\textbf{Geometric interpretation:} Think of $M$, $\sqrt{A}$, and $\sqrt{\mathcal{I} \cdot A}$ as sides of a triangle. They must satisfy triangle inequalities.

\textbf{Limiting cases:}
\begin{itemize}
    \item $\mathcal{I} \to 0$ (MOTS): Standard Penrose bound $M \geq \sqrt{A/(16\pi)}$
    \item $\mathcal{I}$ large (deep inside): Triangle becomes ``thin,'' $M \approx \sqrt{A/(16\pi)}$
\end{itemize}
\end{physicsbox}

%% ============================================================================
\section{The Unified Black Hole Formula}
%% ============================================================================

\begin{newformula}[Master Formula]
\begin{theorem}[Unified Black Hole Identity]
For a MOTS $\Sigma^*$ in initial data $(M, g, k)$:
\begin{equation}
\boxed{
    M_{\ADM}^2 = \frac{A}{16\pi} + \frac{J^2}{4A/\pi} + \frac{Q^2}{4} + E_{\text{gw}}
}
\end{equation}
where:
\begin{itemize}
    \item $A$ = horizon area
    \item $J$ = angular momentum
    \item $Q$ = electric charge
    \item $E_{\text{gw}}$ = energy radiated in gravitational waves
\end{itemize}
\end{theorem}
\end{newformula}

\begin{physicsbox}[Where Does the Mass Go?]
\textbf{Physical meaning:} The total mass-energy budget of a black hole.

\textbf{Four contributions:}
\begin{enumerate}
    \item \textbf{Irreducible mass} $M_{\text{irr}}^2 = A/(16\pi)$: Locked away forever
    \item \textbf{Rotational energy} $J^2/(4M_{\text{irr}}^2)$: Extractable via Penrose process
    \item \textbf{Electromagnetic energy} $Q^2/4$: Extractable (charged black holes)
    \item \textbf{Radiated energy} $E_{\text{gw}}$: Already escaped as gravitational waves
\end{enumerate}

\textbf{Conservation:} Total mass = irreducible + extractable + radiated.
\end{physicsbox}

%% ============================================================================
\section{The Irreversibility Formula}
%% ============================================================================

\begin{newformula}[Irreversibility Measure]
\begin{definition}
The \textbf{irreversibility} of a black hole process:
\begin{equation}
\boxed{
    \mathcal{R} := \frac{\Delta A}{16\pi M^2} = \frac{A_{\text{final}} - A_{\text{initial}}}{16\pi M_{\text{final}}^2}
}
\end{equation}
\end{definition}
\end{newformula}

\begin{physicsbox}[How Irreversible Was the Process?]
\textbf{Physical meaning:} $\mathcal{R}$ measures the thermodynamic irreversibility of a black hole process.

\begin{itemize}
    \item $\mathcal{R} = 0$: Reversible (only for idealized processes)
    \item $\mathcal{R}$ small: Nearly reversible (slow accretion)
    \item $\mathcal{R}$ large: Highly irreversible (violent merger)
\end{itemize}

\textbf{Examples:}
\begin{itemize}
    \item Equal-mass head-on collision: $\mathcal{R} \approx 0.06$
    \item Maximally spinning merger: $\mathcal{R} \approx 0.5$
    \item Particle falling into Schwarzschild: $\mathcal{R} \propto m/M$ (small)
\end{itemize}
\end{physicsbox}

%% ============================================================================
\part{Summary: Key Formulas with Physical Meaning}
%% ============================================================================

\begin{tcolorbox}[colback=yellow!10!white, colframe=orange!75!black, title={\textbf{Summary of New Formulas and Inequalities}}]

\textbf{Measuring Trapping Strength:}
\begin{align}
    \text{Trapping Depth: } & \mathcal{D} = \frac{A^2 |\bar{\theta}^+\bar{\theta}^-|}{16\pi^2} \quad \text{(how deep inside)} \\
    \text{Escape Difficulty: } & \mathcal{E} = e^{\langle|\theta^+|/H\rangle} - 1 \quad \text{(how hard to escape)} \\
    \text{Focusing Power: } & \mathcal{F} = \int R_{\mu\nu}\ell^\mu\ell^\nu \, dA \quad \text{(gravitational strength)}
\end{align}

\textbf{Key Inequalities:}
\begin{align}
    \text{Area Growth: } & A(\Sigma^*) - A(\Sigma_0) \geq \frac{1}{4\pi}\int \theta^+\theta^- \, dA \\
    \text{Mass-Trapping: } & M^2 \geq \frac{A}{16\pi}(1 + \mathcal{D}/4) \\
    \text{Entropy-Depth: } & S \cdot \mathcal{D} \leq 4\pi M^2/\ell_P^2 \\
    \text{Expansion Bound: } & |\theta^+| + |\theta^-| \leq 4/r_{\text{eff}}
\end{align}

\textbf{Energy Formulas:}
\begin{align}
    \text{Trapped Energy: } & E_{\text{trap}} = \sqrt{\frac{A}{16\pi}}\sqrt{1 + \langle\theta^+\theta^-/|\theta^-|\rangle/(4\pi)} \\
    \text{Binding Energy: } & E_{\text{bind}} = M - \sqrt{A/(16\pi)} \\
    \text{Master Formula: } & M^2 = \frac{A}{16\pi} + \frac{J^2}{4A/\pi} + \frac{Q^2}{4} + E_{\text{gw}}
\end{align}

\textbf{Geometry Formulas:}
\begin{align}
    \text{Curvature Concentration: } & \mathcal{K} = \text{Var}(R_\Sigma)/(8\pi\chi)^2 \\
    \text{Shear Ratio: } & \mathcal{S} = \int|\sigma^+|^2/\int|\sigma^-|^2 \\
    \text{Area Evolution: } & dA/dt = \frac{1}{8\pi}\int(|\sigma|^2 + R_{\mu\nu}\ell^\mu\ell^\nu) \, dA
\end{align}

\end{tcolorbox}

\end{document}
