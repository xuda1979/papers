%% OPTIMAL_TRANSPORT_PENROSE.tex
%% A New Mathematical Framework: Optimal Transport for the Penrose Inequality
%% 
%% This develops a genuinely novel approach using Wasserstein geometry.

\documentclass[11pt]{amsart}
\usepackage{amsmath,amssymb,amsthm}
\usepackage{mathtools}
\usepackage{xcolor}

\newtheorem{theorem}{Theorem}[section]
\newtheorem{lemma}[theorem]{Lemma}
\newtheorem{proposition}[theorem]{Proposition}
\newtheorem{corollary}[theorem]{Corollary}
\newtheorem{definition}[theorem]{Definition}
\newtheorem{remark}[theorem]{Remark}
\newtheorem*{maintheorem}{Main Theorem}

\newcommand{\ADM}{\mathrm{ADM}}
\newcommand{\Was}{\mathcal{W}}
\newcommand{\Area}{\mathrm{Area}}

\title{Optimal Transport and the Penrose Inequality:\\A New Framework}
\author{}
\date{December 2025}

\begin{document}
\maketitle

\begin{abstract}
We develop a new approach to the Penrose inequality using optimal transport theory. The key idea is to view the area comparison $A(\Sigma) \le A(\mathcal{H}_\mathcal{C})$ as a consequence of a transport inequality between trapped surfaces and the event horizon. This approach naturally incorporates the causal structure of spacetime.
\end{abstract}

%% ============================================================================
\section{Introduction: The Optimal Transport Viewpoint}
%% ============================================================================

\subsection{Setup}

Consider a globally hyperbolic spacetime $(M^4, g)$ satisfying DEC. Let:
\begin{itemize}
    \item $\Sigma_0 \subset M$ be a trapped surface (compact, spacelike, $\theta^+ < 0$, $\theta^- < 0$)
    \item $\mathcal{H}^+ = \partial J^-(\mathscr{I}^+)$ be the event horizon
    \item $\mathcal{H}_\mathcal{C} = \mathcal{H}^+ \cap \mathcal{C}$ be the horizon cross-section on a Cauchy surface
\end{itemize}

The 1973 Penrose conjecture asserts: $A(\Sigma_0) \le A(\mathcal{H}_\mathcal{C})$.

\subsection{Optimal Transport Interpretation}

Think of:
\begin{itemize}
    \item $\Sigma_0$ as a "source" distribution
    \item $\mathcal{H}_\mathcal{C}$ as a "target" distribution
    \item The causal future $J^+(\Sigma_0)$ as the "transport domain"
\end{itemize}

The trapped condition ensures that light rays from $\Sigma_0$ converge, so the "mass" on $\Sigma_0$ gets compressed as it flows to $\mathcal{H}_\mathcal{C}$. This should give an area inequality.

%% ============================================================================
\section{Causal Optimal Transport}
%% ============================================================================

\subsection{The Lorentzian Cost Function}

\begin{definition}[Lorentzian Cost]\label{def:lorentz-cost}
For points $x, y \in M$ with $y \in J^+(x)$, define:
\begin{equation}
    c(x, y) := \tau(x, y)^2
\end{equation}
where $\tau(x, y)$ is the Lorentzian distance (supremum of proper time over causal curves from $x$ to $y$).

If $y \notin J^+(x)$, set $c(x, y) = +\infty$.
\end{definition}

\begin{remark}
The cost $c(x,y) = \tau^2$ is natural because:
\begin{enumerate}
    \item $\tau$ is semicontinuous and respects causality
    \item $\tau^2$ is quadratic, analogous to the Euclidean $|x-y|^2$
    \item The Lorentzian McCann interpolation uses this cost
\end{enumerate}
\end{remark}

\subsection{Transport Plans}

\begin{definition}[Causal Transport Plan]\label{def:transport-plan}
Let $\mu_0$ be a probability measure supported on $\Sigma_0$, and $\mu_1$ be a probability measure supported on $\mathcal{H}_\mathcal{C}$.

A \textbf{causal transport plan} is a measure $\pi$ on $M \times M$ such that:
\begin{enumerate}
    \item $\pi$ has marginals $\mu_0$ and $\mu_1$
    \item $\pi$ is supported on $\{(x,y) : y \in J^+(x)\}$ (causality constraint)
\end{enumerate}
\end{definition}

\begin{definition}[Causal Wasserstein Distance]\label{def:causal-was}
\begin{equation}
    \Was_2^2(\mu_0, \mu_1) := \inf_\pi \int_{M \times M} \tau(x,y)^2 \, d\pi(x,y)
\end{equation}
where the infimum is over causal transport plans.
\end{definition}

\subsection{Existence and Optimality}

\begin{theorem}[Existence of Optimal Transport]\label{thm:ot-exists}
Assume $\mathcal{H}_\mathcal{C} \subset J^+(\Sigma_0)$ (cosmic censorship). Then:
\begin{enumerate}
    \item There exists an optimal transport plan $\pi^*$ achieving the infimum.
    \item The optimal transport is supported on timelike or null geodesics.
\end{enumerate}
\end{theorem}

\begin{proof}
This follows from the general theory of optimal transport with lower semicontinuous cost. The key point is that $\tau^2$ is upper semicontinuous, so $c = \tau^2$ is lower semicontinuous with respect to the inverse.

The support on geodesics follows from the $c$-cyclical monotonicity of optimal plans.
\end{proof}

%% ============================================================================
\section{The Area Comparison via Jacobian Bounds}
%% ============================================================================

\subsection{Monge Transport}

Under regularity conditions, the optimal transport plan is induced by a \textbf{transport map} $T: \Sigma_0 \to \mathcal{H}_\mathcal{C}$.

\begin{theorem}[Monge Structure]\label{thm:monge}
If $\mu_0 = \frac{1}{A(\Sigma_0)} dA_{\Sigma_0}$ (uniform on $\Sigma_0$) and $\mu_1$ is absolutely continuous with respect to $dA_{\mathcal{H}_\mathcal{C}}$, then the optimal plan is induced by a map $T$:
\begin{equation}
    \pi^* = (\text{Id} \times T)_\# \mu_0
\end{equation}
\end{theorem}

\subsection{Jacobian and Area}

\begin{definition}[Transport Jacobian]\label{def:jacobian}
The Jacobian of the transport map is:
\begin{equation}
    J_T(x) := \frac{d\mu_1}{d\mu_0}(x) = \frac{dT_\# \mu_0}{d\mu_1}(T(x))^{-1}
\end{equation}
\end{definition}

By the change of variables formula:
\begin{equation}
    \int_{\Sigma_0} J_T \, d\mu_0 = \int_{\mathcal{H}_\mathcal{C}} d\mu_1 = 1
\end{equation}

\subsection{Raychaudhuri and the Jacobian Bound}

The key observation is that the Raychaudhuri equation controls the Jacobian:

\begin{theorem}[Jacobian Bound from Raychaudhuri]\label{thm:jacobian-bound}
Let $T$ be the optimal transport map along null geodesics. Under DEC:
\begin{equation}
    J_T(x) \le 1 \quad \text{for all } x \in \Sigma_0
\end{equation}
with equality only if $\theta^+ = 0$ on $\Sigma_0$.
\end{theorem}

\begin{proof}
The Jacobian along a null geodesic congruence satisfies:
\begin{equation}
    \frac{d}{d\lambda} \log J = \theta^+
\end{equation}
where $\theta^+$ is the expansion (trace of the second fundamental form of the null hypersurface).

For a trapped surface, $\theta^+|_{\Sigma_0} < 0$. By the Raychaudhuri equation:
\begin{equation}
    \frac{d\theta^+}{d\lambda} = -\frac{1}{2}(\theta^+)^2 - |\sigma|^2 - R_{\mu\nu} k^\mu k^\nu \le -\frac{1}{2}(\theta^+)^2
\end{equation}
using DEC ($R_{\mu\nu} k^\mu k^\nu \ge 0$).

This means $\theta^+$ becomes more negative as we flow. However, on the horizon, $\theta^+ = 0$.

\textcolor{red}{\textbf{Issue:}} If $\theta^+$ becomes more negative, the Jacobian $J = e^{\int \theta^+ d\lambda}$ decreases along the flow. This seems to give $J_T < 1$, not $J_T > 1$.

But we're flowing from $\Sigma_0$ (trapped) to $\mathcal{H}_\mathcal{C}$ (horizon). Let me reconsider the direction.

\textbf{Corrected analysis:}
At $\Sigma_0$: $\theta^+ < 0$ (trapped).
At $\mathcal{H}_\mathcal{C}$: $\theta^+ = 0$ (horizon is MOTS).

Along the outgoing null flow from $\Sigma_0$ to $\mathcal{H}_\mathcal{C}$:
- Initially $\theta^+ < 0$
- Raychaudhuri: $\theta^+$ decreases further (becomes more negative)
- Contradiction: we can't reach $\theta^+ = 0$ on the horizon!

\textcolor{red}{\textbf{This is the fundamental sign problem.}} The outgoing null flow from a trapped surface doesn't reach the event horizon; it refocuses and hits a singularity.

\end{proof}

%% ============================================================================
\section{Resolution: Time-Reversed Transport}
%% ============================================================================

\subsection{Past-Directed Transport}

The key insight is to consider \textbf{past-directed} transport: from the horizon $\mathcal{H}_\mathcal{C}$ back to the trapped surface $\Sigma_0$.

\begin{definition}[Past Transport]\label{def:past-transport}
Define the cost:
\begin{equation}
    c^-(y, x) := \tau(x, y)^2 \quad \text{for } x \in J^-(y)
\end{equation}
and the past-directed Wasserstein distance:
\begin{equation}
    \Was_2^{-}(\mu_1, \mu_0)^2 := \inf_\pi \int \tau(x,y)^2 \, d\pi(y, x)
\end{equation}
over plans with $x \in J^-(y)$.
\end{definition}

\begin{theorem}[Past Jacobian Bound]\label{thm:past-jacobian}
Let $S: \mathcal{H}_\mathcal{C} \to \Sigma_0$ be the optimal past-directed transport map. Under DEC:
\begin{equation}
    J_S(y) \ge 1 \quad \text{for all } y \in \mathcal{H}_\mathcal{C}
\end{equation}
\end{theorem}

\begin{proof}
For past-directed null geodesics starting at $\mathcal{H}_\mathcal{C}$ with $\theta^+ = 0$ and ending at $\Sigma_0$ with $\theta^+ < 0$:

The expansion must \textit{increase} from 0 to something less negative (wait, that's also wrong).

\textbf{Correct setup:} Consider \textit{past-directed} null geodesics, i.e., parametrized by decreasing affine parameter. The past expansion $\theta^-$ satisfies its own Raychaudhuri equation.

Actually, for past-directed transport along the \textit{ingoing} null direction:
- At $\mathcal{H}_\mathcal{C}$: we're on the horizon, $\theta^+ = 0$
- Moving past-inward from the horizon...

This is getting confusing. Let me try a different approach.
\end{proof}

%% ============================================================================
\section{Alternative: Hawking Mass Transport}
%% ============================================================================

\subsection{The Hawking Mass as a Transport Cost}

Instead of $\tau^2$, use a cost based on the Hawking mass:

\begin{definition}[Hawking Cost]\label{def:hawking-cost}
For surfaces $\Sigma_1, \Sigma_2$ with $\Sigma_1 \subset J^-(\Sigma_2)$:
\begin{equation}
    c_H(\Sigma_1, \Sigma_2) := M_H(\Sigma_2) - M_H(\Sigma_1)
\end{equation}
where $M_H(\Sigma) = \sqrt{A/(16\pi)}(1 - \frac{1}{16\pi}\oint \theta^+ \theta^- dA)$.
\end{definition}

\begin{theorem}[Hawking Mass Monotonicity]\label{thm:hawking-mono}
Under DEC, along smooth null flows:
\begin{equation}
    c_H(\Sigma_1, \Sigma_2) \ge 0
\end{equation}
\end{theorem}

\begin{proof}
This is the Hawking mass monotonicity theorem (proven by Hawking for spherical symmetry, extended by others for general surfaces).
\end{proof}

\subsection{Area from Hawking Mass}

\begin{corollary}[Area Comparison]\label{cor:hawking-area}
If $\Sigma$ is trapped and $\mathcal{H}_\mathcal{C}$ is the horizon:
\begin{equation}
    M_H(\Sigma) \le M_H(\mathcal{H}_\mathcal{C})
\end{equation}

For the horizon $\mathcal{H}_\mathcal{C}$ (a MOTS with $\theta^+ = 0$):
\begin{equation}
    M_H(\mathcal{H}_\mathcal{C}) = \sqrt{\frac{A(\mathcal{H}_\mathcal{C})}{16\pi}}
\end{equation}

\textcolor{red}{\textbf{Gap:}} This gives $M_H(\Sigma) \le \sqrt{A(\mathcal{H}_\mathcal{C})/(16\pi)}$, but $M_H(\Sigma)$ is not simply related to $A(\Sigma)$ for trapped surfaces.
\end{corollary}

%% ============================================================================
\section{The Entropic Approach}
%% ============================================================================

\subsection{Boltzmann Entropy of Surfaces}

\begin{definition}[Surface Entropy]\label{def:entropy}
For a surface $\Sigma$ with area measure $dA$:
\begin{equation}
    S(\Sigma) := \frac{A(\Sigma)}{4G\hbar}
\end{equation}
(the Bekenstein-Hawking entropy).
\end{definition}

\begin{theorem}[Second Law (Generalized)]\label{thm:second-law}
Under cosmic censorship and DEC:
\begin{equation}
    S(\mathcal{H}_\mathcal{C}) \ge S(\Sigma)
\end{equation}
for any trapped surface $\Sigma$ in the past of $\mathcal{H}_\mathcal{C}$.
\end{theorem}

\begin{proof}
This is equivalent to the Penrose conjecture: $A(\mathcal{H}_\mathcal{C}) \ge A(\Sigma)$.

The physical intuition is the generalized second law: the horizon entropy dominates all earlier entropies.

\textcolor{red}{\textbf{Gap:}} This is the statement we want to prove, not a proof!
\end{proof}

%% ============================================================================
\section{Honest Assessment}
%% ============================================================================

\textbf{What the optimal transport approach achieves:}
\begin{enumerate}
    \item Provides a new geometric framework for thinking about the problem
    \item Connects to the Raychaudhuri equation via Jacobian bounds
    \item Highlights the fundamental sign problem in null flows
\end{enumerate}

\textbf{What remains unproven:}
\begin{enumerate}
    \item The Jacobian bound goes the wrong way for future-directed flow
    \item Past-directed flow leaves the Cauchy surface
    \item The Hawking mass doesn't directly give area comparison
    \item The entropic "second law" is the conjecture itself, not a proof
\end{enumerate}

\textbf{Fundamental obstruction:}
The optimal transport approach, like all others, runs into the basic problem:
\begin{quote}
\textit{There is no smooth flow that stays within a Cauchy surface and connects a trapped surface to the event horizon while preserving a monotonic area functional.}
\end{quote}

The event horizon is defined globally (as a boundary of a causal set), while trapped surfaces are local. No local flow can capture this global structure.

%% ============================================================================
\section{Conclusion}
%% ============================================================================

The optimal transport framework provides a elegant geometric language for the Penrose inequality but does not immediately yield a proof. The fundamental issue—the global nature of the event horizon versus the local nature of trapping—persists in this framework as well.

The most promising direction may be:
\begin{enumerate}
    \item Combining optimal transport with the Jang equation (which handles the extrinsic curvature)
    \item Using entropic formulations that may have different regularity requirements
    \item Developing a truly Lorentzian optimal transport theory (currently in its infancy)
\end{enumerate}

\end{document}
