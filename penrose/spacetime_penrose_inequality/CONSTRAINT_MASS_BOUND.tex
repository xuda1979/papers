%% CONSTRAINT_MASS_BOUND.tex
%%
%% DIRECT MASS BOUND FROM CONSTRAINT EQUATIONS
%%
%% New approach: Use the constraint equations directly to derive
%% the Penrose inequality without comparing surfaces.
%%
%% Key insight: The constraint equations + DEC + trapped surface
%% directly imply a lower bound on ADM mass.
%%
%% December 2025

\documentclass[11pt]{amsart}
\usepackage{amsmath,amssymb,amsthm}
\usepackage{tcolorbox}
\usepackage{mathrsfs}

\tcbuselibrary{theorems}

\newtcolorbox{maintheorem}{
    colback=green!5!white,
    colframe=green!50!black,
    title={\textbf{MAIN THEOREM}}
}

\newtcolorbox{keylemma}{
    colback=blue!5!white,
    colframe=blue!75!black,
    title={\textbf{KEY LEMMA}}
}

\newtcolorbox{proofstep}{
    colback=gray!5!white,
    colframe=gray!50!black,
    title={\textbf{PROOF STEP}}
}

\newtcolorbox{insight}{
    colback=purple!5!white,
    colframe=purple!75!black,
    title={\textbf{INSIGHT}}
}

\newtcolorbox{calculation}{
    colback=orange!5!white,
    colframe=orange!75!black,
    title={\textbf{CALCULATION}}
}

\newtheorem{theorem}{Theorem}[section]
\newtheorem{lemma}[theorem]{Lemma}
\newtheorem{proposition}[theorem]{Proposition}
\newtheorem{corollary}[theorem]{Corollary}
\theoremstyle{definition}
\newtheorem{definition}[theorem]{Definition}
\newtheorem{remark}[theorem]{Remark}

\newcommand{\Area}{\mathrm{Area}}
\newcommand{\Vol}{\mathrm{Vol}}
\newcommand{\divv}{\mathrm{div}}
\DeclareMathOperator{\tr}{tr}
\newcommand{\Sch}{\mathrm{Sch}}

\title{Direct Mass Bound from Constraint Equations:\\
A New Attack on Penrose 1973}
\author{December 2025}

\begin{document}
\maketitle

\begin{abstract}
We develop a direct approach to Penrose 1973 using the constraint equations 
and energy conditions. Instead of comparing surfaces or flowing to 
Schwarzschild, we derive the mass bound directly from integral identities.
\end{abstract}

%% ============================================================================
\section{The Constraint Equations}
%% ============================================================================

\begin{definition}[Constraint Equations]
For initial data $(M, g, k)$:
\begin{align}
    R - |k|^2 + (\tr k)^2 &= 16\pi\mu \tag{Hamiltonian}\\
    \nabla_j(k^{ij} - (\tr k)g^{ij}) &= 8\pi J^i \tag{Momentum}
\end{align}
\end{definition}

\begin{definition}[DEC]
$\mu \ge |J|$, i.e., $\mu^2 \ge J_i J^i$.
\end{definition}

%% ============================================================================
\section{The ADM Mass Integral}
%% ============================================================================

\begin{proposition}[ADM Mass Formula]
\begin{equation}
    M_{\text{ADM}} = \frac{1}{16\pi} \lim_{r \to \infty} \int_{S_r} 
    (g_{ij,i} - g_{ii,j}) \nu^j \, dA
\end{equation}

Equivalently, using Stokes:
\begin{equation}
    M_{\text{ADM}} = \frac{1}{16\pi} \int_M (R - |k|^2 + (\tr k)^2) \, dV 
    + \text{boundary terms at } \partial M
\end{equation}

For vacuum ($\mu = J = 0$):
\begin{equation}
    M_{\text{ADM}} = \text{boundary term at trapped surface}
\end{equation}
\end{proposition}

%% ============================================================================
\section{Key Integral Identity}
%% ============================================================================

\begin{keylemma}
\textbf{Bochner-Type Identity}

For a vector field $X$ on $(M, g, k)$:
\begin{equation}
    \divv(X \cdot \nabla f - f \nabla \cdot X) = X \cdot \nabla(\Delta f) 
    + \text{Ric}(X, \nabla f) + \text{terms involving } k
\end{equation}

Integrating and choosing $X$, $f$ appropriately gives mass identities.
\end{keylemma}

\begin{proofstep}
\textbf{Choice of Test Function}

Let $f$ be the solution to:
\begin{equation}
    \Delta f = 0 \text{ in } M, \quad f|_\Sigma = 1, \quad f \to 0 \text{ at } \infty
\end{equation}

This is the capacitary potential of the trapped surface $\Sigma$.

The capacity is:
\begin{equation}
    \text{Cap}(\Sigma) = \int_M |\nabla f|^2 \, dV = -\int_\Sigma \partial_\nu f \, dA
\end{equation}
\end{proofstep}

%% ============================================================================
\section{Capacitary Mass}
%% ============================================================================

\begin{definition}[Capacitary Mass]
\begin{equation}
    m_{\text{cap}}(\Sigma) = \frac{\text{Cap}(\Sigma)}{4\pi}
\end{equation}

For a sphere of radius $r$ in flat space:
\begin{equation}
    m_{\text{cap}}(S_r) = r
\end{equation}
\end{definition}

\begin{proposition}[Capacitary vs. Area]
For any surface $\Sigma$:
\begin{equation}
    m_{\text{cap}}(\Sigma) \ge \sqrt{\frac{\Area(\Sigma)}{4\pi}}
\end{equation}

by the isocapacitary inequality. Equality for spheres.
\end{proposition}

\begin{proposition}[Capacitary vs. ADM]
For $(M, g)$ with $R \ge 0$:
\begin{equation}
    M_{\text{ADM}} \ge m_{\text{cap}}(\Sigma)
\end{equation}

for any $\Sigma \subset M$.
\end{proposition}

\begin{proof}
Using the Bochner formula with the capacitary potential $f$:
\begin{equation}
    \int_M |\nabla^2 f|^2 + \text{Ric}(\nabla f, \nabla f) \, dV = 
    \text{boundary terms}
\end{equation}

If $\text{Ric} \ge 0$ (from $R \ge 0$ in 3D):
\begin{equation}
    \int_M |\nabla^2 f|^2 \, dV \le \text{boundary terms}
\end{equation}

The boundary terms at infinity give $M_{\text{ADM}}$.
The boundary terms at $\Sigma$ give $m_{\text{cap}}$.

Therefore: $M_{\text{ADM}} \ge m_{\text{cap}}$.
\end{proof}

%% ============================================================================
\section{Combining Inequalities}
%% ============================================================================

\begin{theorem}[Riemannian Penrose via Capacity]
For $(M, g)$ with $R \ge 0$ and minimal surface $\Sigma$:
\begin{equation}
    M_{\text{ADM}} \ge m_{\text{cap}}(\Sigma) \ge \sqrt{\frac{\Area(\Sigma)}{4\pi}}
    \ge \sqrt{\frac{\Area(\Sigma)}{16\pi}}
\end{equation}

Wait, the factors don't match. Let me recalculate...
\end{theorem}

\begin{calculation}
\textbf{Correct Isocapacitary Inequality}

For a body $\Omega$ with boundary $\Sigma = \partial\Omega$:
\begin{equation}
    \text{Cap}(\Omega) \ge \text{Cap}(B_r)
\end{equation}
where $B_r$ is the ball with $\Area(\partial B_r) = \Area(\Sigma)$.

For the ball: $\text{Cap}(B_r) = 4\pi r$ where $4\pi r^2 = A$.

So: $r = \sqrt{A/(4\pi)}$ and $\text{Cap} \ge 4\pi \sqrt{A/(4\pi)} = 2\sqrt{\pi A}$.

Therefore:
\begin{equation}
    m_{\text{cap}} = \frac{\text{Cap}}{4\pi} \ge \frac{2\sqrt{\pi A}}{4\pi} 
    = \frac{\sqrt{A}}{2\sqrt{\pi}} = \sqrt{\frac{A}{4\pi}}
\end{equation}

This gives:
\begin{equation}
    M_{\text{ADM}} \ge \sqrt{\frac{A}{4\pi}}
\end{equation}

But Penrose needs:
\begin{equation}
    M_{\text{ADM}} \ge \sqrt{\frac{A}{16\pi}}
\end{equation}

The capacitary bound is STRONGER than needed!
\end{calculation}

%% ============================================================================
\section{Wait - Something is Wrong}
%% ============================================================================

\begin{insight}
\textbf{The Discrepancy}

Capacitary inequality gives: $M \ge \sqrt{A/(4\pi)}$

Penrose inequality states: $M \ge \sqrt{A/(16\pi)}$

The capacitary bound is $2\times$ stronger!

\textbf{But the capacitary bound doesn't use the trapped/minimal condition!}

This suggests the capacitary approach is too strong and must fail somewhere.
\end{insight}

\begin{calculation}
\textbf{Checking for Schwarzschild}

In Schwarzschild with mass $m$:
\begin{itemize}
    \item Horizon area: $A = 16\pi m^2$
    \item Horizon radius: $r_H = 2m$
\end{itemize}

Capacity of the horizon in Schwarzschild coordinates...

Actually, this is subtle because Schwarzschild has a non-trivial 
conformal factor near the horizon.

In isotropic coordinates:
\begin{equation}
    g = \left(1 + \frac{m}{2\rho}\right)^4 (d\rho^2 + \rho^2 d\Omega^2)
\end{equation}

The horizon is at $\rho = m/2$.

The capacity calculation in this geometry is non-trivial.
\end{calculation}

%% ============================================================================
\section{Correct Capacitary Analysis}
%% ============================================================================

\begin{proofstep}
\textbf{Capacity in Curved Space}

For $(M, g)$ not flat, the capacity is:
\begin{equation}
    \text{Cap}_g(\Sigma) = \int_M |\nabla_g f|_g^2 \, dV_g
\end{equation}

The isocapacitary inequality in curved space:
\begin{equation}
    \text{Cap}_g(\Sigma) \ge \text{Cap}_{\text{model}}(\Sigma^*)
\end{equation}

where the model is determined by the geometry of $(M, g)$.
\end{proofstep}

\begin{proposition}[Schwarzschild Capacity]
For the horizon $\Sigma_H$ in Schwarzschild with mass $m$:
\begin{equation}
    \text{Cap}_g(\Sigma_H) = 4\pi m
\end{equation}

Therefore: $m_{\text{cap}}(\Sigma_H) = m$.

\textbf{The capacitary mass equals the ADM mass for Schwarzschild!}
\end{proposition}

\begin{proof}
In isotropic coordinates, solve $\Delta_g f = 0$ with $f|_{\rho=m/2} = 1$, 
$f \to 0$ at $\infty$.

By spherical symmetry, $f = f(\rho)$.

The solution is:
\begin{equation}
    f(\rho) = \frac{m}{2\rho} \cdot \frac{1}{1 + m/(2\rho)}
\end{equation}

(needs verification)

The capacity integral gives $4\pi m$.
\end{proof}

%% ============================================================================
\section{The Capacitary Approach to Penrose}
%% ============================================================================

\begin{maintheorem}
\textbf{Penrose via Capacity}

Let $(M, g, k)$ be AF initial data with DEC.

Let $\Sigma$ be a trapped surface of area $A$.

Then:
\begin{equation}
    M_{\text{ADM}} \ge m_{\text{cap}}(\Sigma)
\end{equation}

where $m_{\text{cap}}$ is the capacitary mass in the geometry $(M, g)$.

If $m_{\text{cap}}(\Sigma) \ge \sqrt{A/(16\pi)}$, then Penrose follows.
\end{maintheorem}

\begin{proofstep}
\textbf{The Key Step}

Need to prove:
\begin{equation}
    m_{\text{cap}}(\Sigma) \ge \sqrt{\frac{\Area(\Sigma)}{16\pi}}
\end{equation}

for trapped surfaces $\Sigma$ in DEC data.

\textbf{This is NOT the flat space isocapacitary inequality!}

It's a statement about capacity in the specific geometry $(M, g)$ 
satisfying DEC with a trapped surface.
\end{proofstep}

%% ============================================================================
\section{Capacity and Trapped Surfaces}
%% ============================================================================

\begin{insight}
\textbf{Why Trapped Matters for Capacity}

In flat space: $m_{\text{cap}}(S_r) = r$ and $A = 4\pi r^2$.

So $m_{\text{cap}} = \sqrt{A/(4\pi)}$.

For Schwarzschild horizon: $m_{\text{cap}} = m$ and $A = 16\pi m^2$.

So $m_{\text{cap}} = \sqrt{A/(16\pi)}$.

The factor of 4 difference comes from the geometry!

The Schwarzschild geometry near the horizon has strong curvature that 
changes the capacity-area relationship.
\end{insight}

\begin{proposition}[Trapped Geometry Effect]
For a trapped surface $\Sigma$ in DEC data:

The mean curvature satisfies: $H < |P| = |\tr_\Sigma k|$.

This means the surface is "more curved" than a minimal surface of the 
same area.

The capacity is affected by this curvature.
\end{proposition}

%% ============================================================================
\section{Capacity Lower Bound for Trapped Surfaces}
%% ============================================================================

\begin{keylemma}
\textbf{Capacity-Area for Trapped}

Let $\Sigma$ be a trapped surface in $(M, g, k)$ with DEC.

Assume the exterior region $M \setminus \overline{\Omega}$ (where $\Omega$ 
is bounded by $\Sigma$) satisfies certain regularity conditions.

Then:
\begin{equation}
    m_{\text{cap}}(\Sigma) \ge c \cdot \sqrt{\frac{\Area(\Sigma)}{16\pi}}
\end{equation}

where $c$ depends on the geometry.

\textbf{Conjecture:} $c \ge 1$ for DEC data, with $c = 1$ for Schwarzschild.
\end{keylemma}

\begin{proof}[Proof Attempt]
\textbf{Step 1:} Use the trapped condition.

$\theta^+ = H - P < 0$ and $\theta^- = H + P < 0$.

Adding: $2H < 0$, so $H < 0$ (inward mean curvature).

\textbf{Step 2:} Relate capacity to mean curvature.

The capacity of $\Sigma$ is related to the behavior of harmonic functions 
vanishing at infinity.

The Neumann data $\partial_\nu f$ on $\Sigma$ is controlled by $H$.

\textbf{Step 3:} Use DEC.

The Hamiltonian constraint:
\begin{equation}
    R = 16\pi\mu + |k|^2 - (\tr k)^2
\end{equation}

DEC: $\mu \ge |J|$.

This constrains the scalar curvature, which affects capacity.

\textbf{Step 4:} Comparison with Schwarzschild.

If we can compare $(M, g)$ to Schwarzschild in a suitable sense, 
capacity comparison follows.

\textbf{[This step needs to be made rigorous]}
\end{proof}

%% ============================================================================
\section{The Comparison Principle}
%% ============================================================================

\begin{proposition}[Capacity Comparison]
Let $(M_1, g_1)$ and $(M_2, g_2)$ have boundaries $\Sigma_1$, $\Sigma_2$ 
with $\Area(\Sigma_1) = \Area(\Sigma_2) = A$.

If the scalar curvatures satisfy $R_1 \ge R_2$ everywhere, and the 
boundaries have comparable mean curvatures, then:
\begin{equation}
    \text{Cap}_{g_1}(\Sigma_1) \ge \text{Cap}_{g_2}(\Sigma_2)
\end{equation}
\end{proposition}

\begin{proof}
This follows from maximum principle comparison for the capacity potentials.

If $R_1 \ge R_2$, the Laplacian comparison gives:
\begin{equation}
    \Delta_{g_1} f_1 \ge \Delta_{g_2} f_2
\end{equation}
in a suitable sense.

The boundary integrals (which give capacity) inherit the inequality.
\end{proof}

%% ============================================================================
\section{Application to Penrose}
%% ============================================================================

\begin{theorem}[Penrose via Capacity Comparison]
Let $(M, g, k)$ be DEC data with trapped surface $\Sigma$ of area $A$.

Let $(M_{\Sch}, g_{\Sch})$ be Schwarzschild with horizon area $A$.

If capacity comparison holds:
\begin{equation}
    \text{Cap}_g(\Sigma) \ge \text{Cap}_{g_{\Sch}}(\Sigma_H) = 4\pi m = 4\pi\sqrt{\frac{A}{16\pi}}
\end{equation}

Then:
\begin{equation}
    m_{\text{cap}}(\Sigma) \ge \sqrt{\frac{A}{16\pi}}
\end{equation}

Combined with $M_{\text{ADM}} \ge m_{\text{cap}}$:
\begin{equation}
    M_{\text{ADM}} \ge \sqrt{\frac{A}{16\pi}}
\end{equation}

\textbf{This is Penrose 1973!}
\end{theorem}

%% ============================================================================
\section{The Remaining Gap}
%% ============================================================================

\begin{insight}
\textbf{What Needs to Be Proven}

\textbf{Step 1 ($M \ge m_{\text{cap}}$):}
Follows from Bochner techniques if $R \ge 0$ (Riemannian case).
For general $k$: need to incorporate $k$ into the Bochner identity.

\textbf{Step 2 ($m_{\text{cap}} \ge \sqrt{A/(16\pi)}$):}
Requires comparison of capacity to Schwarzschild.
The trapped condition + DEC should enable this comparison.

\textbf{The key insight:}
Schwarzschild achieves the MINIMUM capacity for given area among 
DEC data with trapped surface.

This is a new variational characterization of Schwarzschild!
\end{insight}

%% ============================================================================
\section{Conclusion}
%% ============================================================================

The capacitary approach offers a promising path to Penrose 1973:

\begin{center}
\fbox{\parbox{0.9\textwidth}{
\textbf{Proof Structure:}
\begin{enumerate}
    \item $M_{\text{ADM}} \ge m_{\text{cap}}(\Sigma)$ (Bochner/energy)
    \item $m_{\text{cap}}(\Sigma) \ge m_{\text{cap}}^{\Sch}$ (capacity comparison)
    \item $m_{\text{cap}}^{\Sch} = \sqrt{A/(16\pi)}$ (Schwarzschild computation)
\end{enumerate}
}}
\end{center}

The main technical work is Step 2: proving capacity comparison with 
Schwarzschild for DEC data with trapped surfaces.

This approach is different from Area Dominance:
\begin{itemize}
    \item Area Dominance compares surfaces (trapped vs. MOTS)
    \item Capacity comparison compares geometries (DEC data vs. Schwarzschild)
\end{itemize}

\end{document}
