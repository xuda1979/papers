%% OM_PROOF_IMCF.tex  
%% Rigorous proof of (OM) using Inverse Mean Curvature Flow from the horizon
%% Key idea: IMCF from horizon cross-section toward trapped surface

\documentclass[11pt]{amsart}
\usepackage{amsmath,amssymb,amsthm}
\usepackage{mathtools}

\newtheorem{theorem}{Theorem}[section]
\newtheorem{lemma}[theorem]{Lemma}
\newtheorem{proposition}[theorem]{Proposition}
\newtheorem{corollary}[theorem]{Corollary}
\newtheorem{definition}[theorem]{Definition}
\newtheorem{remark}[theorem]{Remark}

\newcommand{\ADM}{\mathrm{ADM}}
\newcommand{\Area}{\mathrm{Area}}

\title{Rigorous Proof of (OM) via Inward IMCF}
\author{}
\date{December 2025}

\begin{document}
\maketitle

\section{The Key Idea}

Standard IMCF flows outward with area increasing. For trapped surfaces (which have $H < 0$), the direction reverses.

\textbf{Key Observation:} The event horizon cross-section $\mathcal{H}_\mathcal{C}$ has $H_{\mathcal{H}_\mathcal{C}} \ge 0$ (it's either minimal or has positive mean curvature pointing toward the black hole exterior).

If we flow INWARD from $\mathcal{H}_\mathcal{C}$ using inverse mean curvature flow (with sign adjusted), the area DECREASES.

\section{Setup}

Let $(M^3, g, k)$ be the initial data on Cauchy surface $\mathcal{C}$. Under cosmic censorship:
\begin{itemize}
    \item $\mathcal{H}_\mathcal{C} = \mathcal{H} \cap \mathcal{C}$ is the horizon cross-section
    \item $\Sigma$ is a trapped surface with $\theta^+, \theta^- < 0$
    \item $\Sigma \subset \text{Int}(\mathcal{H}_\mathcal{C})$ (topologically inside)
\end{itemize}

\section{The Flow Construction}

\begin{definition}[Inward Mean Curvature Flow]
Starting from $\mathcal{H}_\mathcal{C}$, define the inward flow:
\begin{equation}
    \frac{\partial \Sigma_t}{\partial t} = -\frac{\nu}{|H|}
\end{equation}
where $\nu$ is the \textbf{inward} unit normal and $H$ is the mean curvature.
\end{definition}

\begin{remark}
Near $\mathcal{H}_\mathcal{C}$: If $\mathcal{H}_\mathcal{C}$ is a MOTS with $\theta^+ = 0$, then $H = -\tr_{\mathcal{H}_\mathcal{C}} k$. The flow is well-defined when $H \ne 0$.
\end{remark}

\section{Area Evolution}

\begin{lemma}[Area Decrease Under Inward Flow]
Under the inward MCF from $\mathcal{H}_\mathcal{C}$:
\begin{equation}
    \frac{dA}{dt} = -\int_{\Sigma_t} |H| \cdot \frac{1}{|H|} \, dA = -A(\Sigma_t) < 0
\end{equation}
\end{lemma}

Wait, this gives exponential decay $A(t) = A(0) e^{-t}$, which doesn't directly give us what we need...

\section{Alternative: Null Comparison}

Let me try a different approach using the null structure more directly.

\begin{theorem}[Null Comparison Principle]
Let $\Sigma$ be a trapped surface and $\mathcal{H}_\mathcal{C}$ be the horizon cross-section. Let $\mathcal{N}$ be any null hypersurface connecting them. Then under NEC:
\begin{equation}
    A(\Sigma) \le A(\mathcal{H}_\mathcal{C})
\end{equation}
if $\mathcal{N}$ is the \textbf{past ingoing} null hypersurface from $\mathcal{H}_\mathcal{C}$.
\end{theorem}

\begin{proof}
\textbf{Step 1:} The event horizon $\mathcal{H}$ has $\theta^+ = 0$ (marginally trapped in the outgoing direction).

\textbf{Step 2:} Consider the past-directed ingoing null hypersurface $\mathcal{N}^-_{\text{in}}$ from $\mathcal{H}_\mathcal{C}$.

The ingoing null expansion on $\mathcal{H}$ satisfies $\theta^-_{\mathcal{H}} < 0$ (the horizon is strictly ingoing-trapped).

\textbf{Step 3:} Along $\mathcal{N}^-_{\text{in}}$ going to the past, by the Raychaudhuri equation:
\begin{equation}
    \frac{d\theta^-}{d\lambda} = -\frac{1}{2}(\theta^-)^2 - |\sigma|^2 - R_{\mu\nu}\ell^-{}^\mu \ell^-{}^\nu \le 0
\end{equation}
under NEC. So $\theta^-$ becomes more negative (or stays negative) going to the past.

\textbf{Step 4:} The area evolution along $\mathcal{N}^-_{\text{in}}$ is:
\begin{equation}
    \frac{dA}{d\lambda} = \int_{\Sigma_\lambda} \theta^- \, dA < 0
\end{equation}
So area \textbf{decreases} going to the past along the ingoing null direction.

\textbf{Step 5:} Now here's the key geometric fact: Under the collapse assumption (COL), the past-directed ingoing null hypersurface from $\mathcal{H}_\mathcal{C}$ eventually reaches the trapped surface $\Sigma$ (or passes through the region containing it).

Actually, wait - this isn't quite right. The ingoing null hypersurface from $\mathcal{H}_\mathcal{C}$ goes into the black hole interior, not toward $\Sigma$ on the same Cauchy surface...

\textbf{Revision:} Let me reconsider the geometry.
\end{proof}

\section{Correct Geometric Setup}

On the Cauchy surface $\mathcal{C}$:
\begin{itemize}
    \item $\mathcal{H}_\mathcal{C}$ is a 2-surface (the horizon cross-section)
    \item $\Sigma$ is a trapped 2-surface inside the black hole region
    \item Both lie on the same 3-dimensional slice $\mathcal{C}$
\end{itemize}

The comparison must be done \textbf{within the Cauchy surface}, not along null hypersurfaces (which leave the slice).

\section{Spacelike Approach: Maximum Principle}

\begin{theorem}[Area Comparison via Maximum Principle]\label{thm:max-principle}
Let $(M^3, g)$ be a Riemannian manifold with $R_g \ge 0$. Let $\Sigma_{\text{out}}$ be an outer-minimizing surface and $\Sigma_{\text{in}}$ be any surface enclosed by $\Sigma_{\text{out}}$. Then:
\begin{equation}
    A(\Sigma_{\text{in}}) \le A(\Sigma_{\text{out}})
\end{equation}
\end{theorem}

\begin{proof}
By definition of outer-minimizing, $\Sigma_{\text{out}}$ minimizes area among all surfaces enclosing $\Sigma_{\text{in}}$. Thus $A(\Sigma_{\text{out}}) \le A(\Sigma')$ for any $\Sigma'$ enclosing $\Sigma_{\text{in}}$.

Taking $\Sigma' = \Sigma_{\text{in}}$... wait, this doesn't work because $\Sigma_{\text{in}}$ doesn't enclose $\Sigma_{\text{in}}$ in the required sense.
\end{proof}

\section{The Correct Statement}

Actually, the problem is more subtle. Let me state what we actually know and need:

\textbf{What we know:}
\begin{enumerate}
    \item $\Sigma$ is trapped: $\theta^\pm < 0$, which implies $H = \frac{1}{2}(\theta^+ + \theta^-) < 0$
    \item $\mathcal{H}_\mathcal{C}$ is the apparent horizon (outermost MOTS) with $\theta^+ = 0$
    \item $\Sigma$ lies inside the trapped region bounded by $\mathcal{H}_\mathcal{C}$... 
    
    Wait, this is wrong! The event horizon $\mathcal{H}_\mathcal{C}$ is NOT the same as the apparent horizon $\Sigma^*$.
\end{enumerate}

\textbf{Key distinction:}
\begin{itemize}
    \item \textbf{Apparent horizon} $\Sigma^*$: Outermost MOTS on the initial data slice
    \item \textbf{Event horizon cross-section} $\mathcal{H}_\mathcal{C}$: Where the global event horizon intersects the slice
\end{itemize}

Under cosmic censorship: $\Sigma^* \subseteq \mathcal{H}_\mathcal{C}$ (apparent horizon inside or on event horizon).

\section{The Two-Step Approach}

To prove $A(\Sigma) \le A(\mathcal{H}_\mathcal{C})$:

\textbf{Step A:} Prove $A(\Sigma) \le A(\Sigma^*)$ (trapped surface vs apparent horizon)

This is the problematic step that can fail in binary mergers!

\textbf{Step B:} Prove $A(\Sigma^*) \le A(\mathcal{H}_\mathcal{C})$ (apparent vs event horizon)

This should follow from $\Sigma^* \subseteq \mathcal{H}_\mathcal{C}$... but containment doesn't imply area comparison!

\section{Resolution: Use Spacetime Structure}

The key is that we're not just looking at the slice - we have the full spacetime.

\begin{theorem}[Event Horizon Dominates Apparent Horizon Area]
Under cosmic censorship (WCC) and null energy condition (NEC):
\begin{equation}
    A(\Sigma^*) \le A(\mathcal{H}_\mathcal{C})
\end{equation}
\end{theorem}

\begin{proof}
\textbf{Step 1:} By cosmic censorship, the apparent horizon $\Sigma^*$ lies on or inside the event horizon: there exists a point $p \in \Sigma^*$ that lies on $\mathcal{H}$ or in the black hole interior.

\textbf{Step 2:} If $\Sigma^* \subset \mathcal{H}$, then $\Sigma^* = \mathcal{H}_\mathcal{C}$ (both are MOTS on the slice) and $A(\Sigma^*) = A(\mathcal{H}_\mathcal{C})$.

\textbf{Step 3:} If $\Sigma^* \subset \text{Int}(\mathcal{B})$ (strictly inside black hole), consider:

The event horizon $\mathcal{H}$ is a null hypersurface with $\theta^+_{\mathcal{H}} = 0$. 

Consider a spacelike slice $\mathcal{C}'$ slightly to the future of $\mathcal{C}$ such that $\Sigma^*$ evolves to $\Sigma^*{}'$ and $\mathcal{H}_\mathcal{C}$ evolves to $\mathcal{H}_{\mathcal{C}'}$.

By the Hawking area theorem on $\mathcal{H}$:
\begin{equation}
    A(\mathcal{H}_{\mathcal{C}'}) \ge A(\mathcal{H}_\mathcal{C})
\end{equation}

For the apparent horizon evolution, under appropriate conditions (dynamical horizon), we have:
\begin{equation}
    A(\Sigma^*{}') \le A(\mathcal{H}_{\mathcal{C}'})
\end{equation}

By continuity in time and taking appropriate limits... 

\textbf{Actually, this approach has gaps too.}
\end{proof}

\section{Conclusion}

The proof of (OM) remains challenging. The fundamental difficulty is:
\begin{enumerate}
    \item Topological containment $\Sigma \subset \text{Int}(\mathcal{H}_\mathcal{C})$ does NOT imply $A(\Sigma) \le A(\mathcal{H}_\mathcal{C})$
    \item The natural flows (IMCF, MCF) have the wrong monotonicity direction in the trapped region
    \item Null hypersurface arguments leave the Cauchy surface
\end{enumerate}

A rigorous proof likely requires new mathematical tools combining:
\begin{itemize}
    \item The causal structure (not just the 3-geometry)
    \item The constraint equations (linking $g, k$ on the slice)
    \item The evolution equations (showing how surfaces evolve)
\end{itemize}

\end{document}
