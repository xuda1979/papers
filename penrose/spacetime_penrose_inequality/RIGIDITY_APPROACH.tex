% =========================================================================
%     THE RIGIDITY APPROACH TO SPACETIME PENROSE INEQUALITY
%
%     Strategy: Prove by contradiction that M < √(A/16π) + trapped surface
%     leads to geometric impossibility
%
%     Author: Da Xu
%     Date: December 2025
% =========================================================================

\documentclass[12pt]{article}
\usepackage{amsmath,amsthm,amssymb}
\usepackage{mathrsfs}
\usepackage{tcolorbox}
\usepackage{xcolor}

\theoremstyle{plain}
\newtheorem{theorem}{Theorem}[section]
\newtheorem{lemma}[theorem]{Lemma}
\newtheorem{proposition}[theorem]{Proposition}
\newtheorem{corollary}[theorem]{Corollary}

\theoremstyle{definition}
\newtheorem{definition}[theorem]{Definition}
\newtheorem{remark}[theorem]{Remark}

\newcommand{\ADM}{\mathrm{ADM}}
\newcommand{\tr}{\mathrm{tr}}
\newcommand{\Div}{\mathrm{div}}
\newcommand{\Ric}{\mathrm{Ric}}
\newcommand{\Area}{\mathrm{Area}}

\title{\textbf{Rigidity and the Spacetime Penrose Inequality:\\
A Proof by Contradiction Approach}}
\author{Da Xu}
\date{December 2025}

\begin{document}
\maketitle

\begin{abstract}
We investigate whether a rigidity argument can prove the spacetime Penrose
inequality. The strategy is: assume $M_{\ADM} < \sqrt{A/(16\pi)}$ where $A$
is the area of a trapped surface $\Sigma$, and derive a contradiction to
the DEC or to the existence of $\Sigma$.
\end{abstract}

%===========================================================================
\section{The Rigidity Strategy}
%===========================================================================

\subsection{Statement}

\begin{theorem}[Penrose Inequality - Contrapositive]
Let $(M^3, g, k)$ be asymptotically flat initial data satisfying DEC with
ADM mass $M$. Suppose $M < \sqrt{A/(16\pi)}$ for some $A > 0$.

Then there exists no closed trapped surface $\Sigma$ in $(M, g, k)$ with
$\Area(\Sigma) = A$.
\end{theorem}

This is logically equivalent to the Penrose inequality:
\begin{equation}
    \text{Trapped } \Sigma \text{ exists} \implies M_{\ADM} \geq \sqrt{\frac{\Area(\Sigma)}{16\pi}}
\end{equation}

\subsection{Why Rigidity Might Work}

In the Riemannian case, the Penrose inequality has a rigidity statement:
\begin{theorem}[Riemannian Rigidity - Bray, Huisken-Ilmanen]
If $(M, g)$ has $R \geq 0$, is asymptotically flat, and contains a minimal
surface $\Sigma$ (outermost) with:
\begin{equation}
    M_{\ADM} = \sqrt{\frac{\Area(\Sigma)}{16\pi}}
\end{equation}
then $(M, g)$ is isometric to the Schwarzschild exterior.
\end{theorem}

\textbf{Key observation:} The rigidity case uniquely characterizes Schwarzschild.
Perhaps the spacetime case has similar structure.

%===========================================================================
\section{The Geometric Constraints}
%===========================================================================

\subsection{What Trapping Implies}

If $\Sigma$ is trapped:
\begin{align}
    H &< 0 \quad \text{(mean curvature)} \\
    |H| &\geq |\tr_\Sigma k| \quad \text{(from $\theta^+\theta^- \geq 0$)} \\
    H^2 - (\tr_\Sigma k)^2 &\geq 0
\end{align}

\subsection{What Low Mass Implies}

If $M_{\ADM} < \sqrt{A/(16\pi)}$, then comparing to Schwarzschild:
\begin{itemize}
    \item The mass is ``too small'' for the area
    \item In Schwarzschild with mass $m$, the horizon has $A = 16\pi m^2$
    \item So $m = \sqrt{A/(16\pi)}$ is the critical mass
    \item Having $M < \sqrt{A/(16\pi)}$ means we're ``sub-Schwarzschild''
\end{itemize}

\subsection{The Geometric Meaning of ``Sub-Schwarzschild''}

In Schwarzschild with mass $m$, at radius $r$:
\begin{equation}
    g = \left(1 - \frac{2m}{r}\right)^{-1}dr^2 + r^2 d\Omega^2
\end{equation}

The horizon is at $r = 2m$ with area $A = 4\pi(2m)^2 = 16\pi m^2$.

If we want a surface of area $A = 16\pi M^2$ for some fixed $M$, the
``natural'' mass is exactly $M$.

Having actual mass $< M$ means the geometry is ``less curved'' than
Schwarzschild at the same area.

%===========================================================================
\section{The Contradiction Argument}
%===========================================================================

\subsection{Setup}

Assume:
\begin{enumerate}
    \item $(M, g, k)$ is asymptotically flat, satisfies DEC
    \item $M_{\ADM} = m < m_0 := \sqrt{A_0/(16\pi)}$ for some fixed $A_0$
    \item There exists a trapped surface $\Sigma$ with $\Area(\Sigma) = A_0$
\end{enumerate}

Goal: Derive a contradiction.

\subsection{First Attempt: Volume Comparison}

\begin{lemma}[Bishop-Gromov Type]
If $R \geq 0$, then volume grows at most like Euclidean:
\begin{equation}
    \mathrm{Vol}(B_r) \leq \frac{4\pi}{3} r^3
\end{equation}
\end{lemma}

In general, with DEC but $R$ not necessarily $\geq 0$:
\begin{equation}
    R = 2\mu + |k|^2 - (\tr k)^2
\end{equation}

So $R$ can be negative if $|k|^2 < (\tr k)^2$ (which can happen).

\textbf{Conclusion:} Volume comparison doesn't directly work in the spacetime case.

\subsection{Second Attempt: Capacity Comparison}

\begin{definition}[Capacity]
The capacity of a surface $\Sigma$ in $(M, g)$ is:
\begin{equation}
    \mathrm{Cap}(\Sigma) = \inf_{u} \int_M |\nabla u|^2 \, dV
\end{equation}
where the infimum is over $u$ with $u = 0$ on $\Sigma$ and $u \to 1$ at infinity.
\end{definition}

\begin{lemma}[Capacity and Mass]
In asymptotically flat manifolds:
\begin{equation}
    M_{\ADM} = \frac{1}{4\pi} \mathrm{Cap}(\Sigma) + O(1/r)
\end{equation}
for large spheres $\Sigma$.
\end{lemma}

\begin{proposition}[Capacity Lower Bound]
For a surface $\Sigma$ with $\Area(\Sigma) = A$ in a manifold with $R \geq 0$:
\begin{equation}
    \mathrm{Cap}(\Sigma) \geq 4\pi \sqrt{\frac{A}{16\pi}} = \sqrt{\pi A}
\end{equation}
with equality for a sphere in Schwarzschild.
\end{proposition}

\textbf{This gives:} $M_{\ADM} \geq \sqrt{A/(16\pi)}$ for minimal surfaces
(where $H = 0$).

\textbf{Problem:} The proof uses $R \geq 0$ and $H = 0$. Neither holds for
trapped surfaces in the spacetime case.

\subsection{Third Attempt: The Jang Equation Reexamined}

The Jang equation provides a way to reduce to the Riemannian case:
\begin{equation}
    \Div\left(\frac{\nabla f}{\sqrt{1 + |\nabla f|^2}}\right) = g^{ij}\left(k_{ij} - \frac{f_if_j k_{ij}}{1 + |\nabla f|^2}\right)\frac{1}{\sqrt{1 + |\nabla f|^2}}
\end{equation}

Solutions blow up at MOTS. The trick is to use the blowup to create a
minimal surface with the right properties.

\textbf{The obstruction:} For trapped surfaces that are not MOTS ($\theta^+ < 0$),
the Jang equation may not produce a useful minimal surface.

%===========================================================================
\section{A New Rigidity Criterion}
%===========================================================================

\subsection{The Scaling Argument}

Consider a family of initial data $(M, g_\lambda, k_\lambda)$ where:
\begin{equation}
    g_\lambda = \lambda^2 g, \quad k_\lambda = \lambda k
\end{equation}

Under this scaling:
\begin{align}
    M_{\ADM}(g_\lambda) &= \lambda M_{\ADM}(g) \\
    \Area(\Sigma, g_\lambda) &= \lambda^2 \Area(\Sigma, g)
\end{align}

The Penrose ratio:
\begin{equation}
    \frac{M_{\ADM}}{\sqrt{\Area/(16\pi)}} = \frac{\lambda M}{\sqrt{\lambda^2 A/(16\pi)}} = \frac{M}{\sqrt{A/(16\pi)}}
\end{equation}
is scale-invariant!

\subsection{The Rigidity Test}

\begin{proposition}[Scale-Invariant Rigidity]
If the Penrose inequality holds with equality for some $(M, g, k)$ and $\Sigma$:
\begin{equation}
    M_{\ADM}(g) = \sqrt{\frac{\Area(\Sigma)}{16\pi}}
\end{equation}
then $(M, g, k)$ is rigid in the sense that the ratio $M/\sqrt{A}$ cannot
be decreased by any deformation preserving DEC and trapping.
\end{proposition}

\textbf{Idea:} The equality case should be Schwarzschild (in some slicing).
Any deviation from Schwarzschild should increase the ratio.

%===========================================================================
\section{The Isoperimetric Approach}
%===========================================================================

\subsection{Isoperimetric Mass}

\begin{definition}[Isoperimetric Mass]
For initial data $(M, g, k)$, define:
\begin{equation}
    M_{\mathrm{iso}} := \sup_\Sigma \sqrt{\frac{\Area(\Sigma)}{16\pi}} - \frac{\mathrm{Vol}(B_\Sigma)^{2/3}}{c}
\end{equation}
where the supremum is over compact surfaces $\Sigma$ enclosing regions $B_\Sigma$,
and $c$ is the Euclidean isoperimetric constant.
\end{definition}

\begin{theorem}[Isoperimetric Mass Bound - Known for $R \geq 0$]
If $(M, g)$ has $R \geq 0$ and is asymptotically flat:
\begin{equation}
    M_{\mathrm{iso}} \leq M_{\ADM}
\end{equation}
\end{theorem}

\textbf{Problem:} This requires $R \geq 0$, which doesn't hold in general
for spacetime initial data.

\subsection{Modified Isoperimetric for DEC}

\begin{conjecture}[DEC Isoperimetric]
There exists a modified isoperimetric functional $\tilde{M}_{\mathrm{iso}}$
depending on both $g$ and $k$ such that:
\begin{enumerate}
    \item $\tilde{M}_{\mathrm{iso}} \leq M_{\ADM}$ under DEC
    \item For trapped surfaces: $\tilde{M}_{\mathrm{iso}} \geq \sqrt{A/(16\pi)}$
\end{enumerate}
\end{conjecture}

If such a functional exists, combining (1) and (2) gives the Penrose inequality.

%===========================================================================
\section{The Maximum Principle Attempt}
%===========================================================================

\subsection{The Idea}

In the Riemannian case, the positive mass theorem can be proven using the
maximum principle for the Schrödinger operator $-\Delta + R/8$.

Can we find an operator whose maximum principle gives the Penrose inequality?

\subsection{The Candidate Operator}

\begin{definition}[Penrose Operator]
For a function $u$ on $M$, define:
\begin{equation}
    L_P u := -\Delta u + \frac{R + |k|^2 - (\tr k)^2}{8} u = -\Delta u + \frac{\mu}{4} u
\end{equation}
\end{definition}

Under DEC, $\mu \geq |J| \geq 0$, so $L_P$ is a Schrödinger operator with
non-negative potential.

\begin{lemma}[Maximum Principle]
If $L_P u \leq 0$ and $u \to c > 0$ at infinity, then $u \geq c_0 > 0$ on $M$.
\end{lemma}

\subsection{Connection to Penrose}

\textbf{Attempt:} Construct a function $u$ related to the distance from $\Sigma$
such that:
\begin{itemize}
    \item $L_P u \leq 0$ outside $\Sigma$
    \item The boundary behavior at $\Sigma$ involves the area and trapping
    \item The decay at infinity involves $M_{\ADM}$
\end{itemize}

\textbf{Status:} This construction is not complete.

%===========================================================================
\section{Honest Assessment}
%===========================================================================

\begin{tcolorbox}[colback=red!5, colframe=red!75!black, title=Current Status]
\textbf{The rigidity approach has not yielded a proof.}

Key obstacles:
\begin{enumerate}
    \item \textbf{No $R \geq 0$:} Most known techniques require non-negative
    scalar curvature, which is not guaranteed under DEC with $k \neq 0$.
    
    \item \textbf{Trapping vs. minimality:} Trapped surfaces are not minimal;
    they have $H < 0$. The geometric tools for minimal surfaces don't directly
    apply.
    
    \item \textbf{The $k$ contribution:} The extrinsic curvature $k$ enters in
    complicated ways through $\tr_\Sigma k$ and $\theta^\pm$.
    
    \item \textbf{No canonical object:} In the Riemannian case, the minimal
    surface is canonical. In the spacetime case, there's no uniquely defined
    surface associated with a trapped surface.
\end{enumerate}
\end{tcolorbox}

%===========================================================================
\section{What Would Be Needed}
%===========================================================================

\begin{tcolorbox}[colback=green!5, colframe=green!75!black, title=Future Directions]
To complete a rigidity proof, one would need:

\textbf{Option A: New Comparison Geometry}
\begin{itemize}
    \item Develop comparison theorems for DEC (not just $R \geq 0$)
    \item These would involve both $\mu$ and $J$ components
    \item Unclear if such theorems exist
\end{itemize}

\textbf{Option B: Spacetime Jang Equation}
\begin{itemize}
    \item Modify the Jang equation to handle $\theta^+ < 0$ (not just MOTS)
    \item Use the blowup structure differently
    \item This is active research (see paper under consideration)
\end{itemize}

\textbf{Option C: Inverse Mean Curvature in Spacetime}
\begin{itemize}
    \item Develop a spacetime version of IMCF
    \item Handle the $\tr_\Sigma k$ terms directly
    \item Find a monotonic quantity
\end{itemize}

\textbf{Option D: Optimal Transport}
\begin{itemize}
    \item Use Wasserstein geometry to compare to Schwarzschild
    \item This is highly speculative
\end{itemize}
\end{tcolorbox}

\subsection{The Core Open Problem}

The fundamental open problem is:

\begin{quote}
\textbf{Find a geometric quantity $\mathcal{Q}(\Sigma)$ depending on a trapped
surface $\Sigma$ in DEC initial data such that:}
\begin{enumerate}
    \item $\mathcal{Q}(\Sigma) \geq \sqrt{\Area(\Sigma)/(16\pi)}$
    \item $\mathcal{Q}(\Sigma) \leq M_{\ADM}$
    \item Both inequalities can be proven without sign restrictions on $\tr_\Sigma k$
\end{enumerate}
\end{quote}

The search for such $\mathcal{Q}$ continues.

\end{document}
