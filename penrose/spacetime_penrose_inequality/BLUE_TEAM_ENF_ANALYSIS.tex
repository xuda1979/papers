%% BLUE_TEAM_ENF_ANALYSIS.tex
%%
%% CRITICAL ANALYSIS: The Expansion-Normalized Flow Approach
%% December 2025
%%
%% Blue Team Attack: Find the gaps!

\documentclass[11pt]{amsart}
\usepackage{amsmath,amssymb,amsthm}
\usepackage{xcolor}
\usepackage{tcolorbox}

\tcbuselibrary{theorems}

\newtcolorbox{gap}{
    colback=red!10!white,
    colframe=red!75!black,
    title={\textbf{GAP IDENTIFIED}}
}

\newtcolorbox{issue}{
    colback=orange!10!white,
    colframe=orange!75!black,
    title={\textbf{TECHNICAL ISSUE}}
}

\newtcolorbox{question}{
    colback=yellow!10!white,
    colframe=yellow!75!black,
    title={\textbf{QUESTION}}
}

\newtheorem{theorem}{Theorem}
\newtheorem{lemma}[theorem]{Lemma}
\newtheorem{proposition}[theorem]{Proposition}
\theoremstyle{definition}
\newtheorem{definition}[theorem]{Definition}
\newtheorem{remark}[theorem]{Remark}

\newcommand{\Area}{\mathrm{Area}}
\newcommand{\Vol}{\mathrm{Vol}}
\newcommand{\divv}{\mathrm{div}}
\DeclareMathOperator{\tr}{tr}

\title{Blue Team Critical Analysis:\\
Gaps in the ENF Approach}
\author{December 2025}

\begin{document}
\maketitle

\begin{abstract}
We critically examine the Expansion-Normalized Flow (ENF) approach to 
Area Dominance, identifying potential gaps and technical issues that 
require resolution.
\end{abstract}

%% ============================================================================
\section{Summary of the Claimed Proof}
%% ============================================================================

The ENF approach claims:

\begin{enumerate}
    \item Define the Expansion-Normalized Flow (ENF) on $(g, k)$
    \item ENF has MOTS as fixed points
    \item Under ENF, trapped surfaces have INCREASING area
    \item MOTS area is CONSTANT
    \item Flow converges to some MOTS $\Sigma'$
    \item MOTS Hierarchy: $\Area(\Sigma') \le \Area(\Sigma^*)$
    \item Therefore: $\Area(\Sigma) \le \Area(\Sigma') \le \Area(\Sigma^*)$
\end{enumerate}

%% ============================================================================
\section{Gap 1: ENF Existence and Regularity}
%% ============================================================================

\begin{gap}
\textbf{The ENF is defined but its existence is NOT proven.}

The ENF equations:
\begin{align}
    \frac{\partial g_{ij}}{\partial t} &= -2\theta^+ (h_{ij} - k_{ij})\\
    \frac{\partial k_{ij}}{\partial t} &= -\theta^+ (R_{ij} - k_i^l k_{lj} + kk_{ij})
\end{align}

involve $\theta^+$, which is evaluated on a SPECIFIC SURFACE $\Sigma$.

\textbf{Questions:}
\begin{itemize}
    \item What surface is $\theta^+$ evaluated on?
    \item As the flow evolves, how does this surface change?
    \item Is this a well-posed PDE system?
\end{itemize}
\end{gap}

\subsection{The Problem}

$\theta^+$ depends on:
\begin{itemize}
    \item The metric $g$
    \item The extrinsic curvature $k$
    \item The SURFACE $\Sigma$ itself
\end{itemize}

If we fix $\Sigma$ as a set, then $\theta^+$ changes only due to $(g, k)$ evolution.

But physically, we want to TRACK the trapped surface as it evolves!

\begin{issue}
\textbf{The flow is NOT a standard PDE.}

Standard parabolic PDEs evolve fields on a fixed domain.

The ENF couples:
\begin{itemize}
    \item Evolution of $(g, k)$ on $\mathcal{C}$
    \item Evolution of the surface $\Sigma$ inside $\mathcal{C}$
\end{itemize}

This is a FREE BOUNDARY PROBLEM - much harder to analyze!
\end{issue}

%% ============================================================================
\section{Gap 2: Area Monotonicity Calculation}
%% ============================================================================

\begin{gap}
\textbf{The area evolution formula needs careful verification.}

Claimed:
\begin{equation}
    \frac{d\Area}{dt} = -\theta^+ \int_\Sigma \theta^- dA > 0
\end{equation}

for trapped surfaces ($\theta^+ < 0$, $\theta^- < 0$).
\end{gap}

\subsection{Checking the Calculation}

Area evolution under metric change $\dot{g}$:
\begin{equation}
    \frac{d\Area}{dt} = \frac{1}{2}\int_\Sigma \gamma^{ab}\dot{\gamma}_{ab} dA
\end{equation}

where $\gamma$ is the induced metric on $\Sigma$.

With $\dot{g}_{ij} = -2\theta^+(h_{ij} - k_{ij})$:

The induced metric changes as:
\begin{equation}
    \dot{\gamma}_{ab} = \dot{g}_{ij}e^i_a e^j_b = -2\theta^+(h_{ab} - k_{ab}|_\Sigma)
\end{equation}

where $e^i_a$ are tangent vectors to $\Sigma$.

\begin{question}
\textbf{What is $h_{ab}$?}

$h_{ab}$ is the second fundamental form of $\Sigma$ in $\mathcal{C}$.

Its trace is $H = h^a_a$.

But the formula uses $h_{ij}$ as a (3,3) tensor, then restricts to $\Sigma$.

\textbf{Is this the same as $h_{ab}$?}
\end{question}

\subsection{Clarification Needed}

The ENF equations use $h_{ij}$ on all of $\mathcal{C}$, not just on $\Sigma$.

But $h_{ij}$ (second fundamental form) is only defined ON $\Sigma$!

\begin{issue}
\textbf{The ENF equations as written are INCONSISTENT.}

$h_{ij}$ is only defined on $\Sigma$, but the evolution equations are for 
$g_{ij}$ on all of $\mathcal{C}$.

The flow cannot be applied pointwise on $\mathcal{C}$!
\end{issue}

%% ============================================================================
\section{Gap 3: What Does ENF Actually Evolve?}
%% ============================================================================

\begin{gap}
\textbf{The ENF does not make sense as written.}

To fix this, we need to REDEFINE the ENF to be well-posed.
\end{gap}

\subsection{Possible Fix 1: Localize to $\Sigma$}

Only evolve $(g, k)$ on a neighborhood of $\Sigma$, using the second 
fundamental form of $\Sigma$ in that neighborhood.

\textbf{Problem:} The constraint equations involve the whole manifold.

\subsection{Possible Fix 2: Extend $h$ to $\mathcal{C}$}

Define $h_{ij}(x)$ for all $x \in \mathcal{C}$ by:
\begin{equation}
    h_{ij}(x) = (\text{second fundamental form of the surface through } x)
\end{equation}

using some foliation.

\textbf{Problem:} Which foliation? The choice affects the flow!

\subsection{Possible Fix 3: Rethink the Flow}

Perhaps the ENF should be a SURFACE flow (like MCF or IMCF) rather than 
a metric flow.

Let $(g, k)$ be fixed and flow $\Sigma$ by:
\begin{equation}
    \frac{\partial\Sigma}{\partial t} = f(\theta^+, \theta^-)\nu
\end{equation}

for some function $f$.

%% ============================================================================
\section{Gap 4: The MOTS Hierarchy}
%% ============================================================================

\begin{gap}
\textbf{The MOTS Hierarchy theorem has an unproven assumption.}

We claimed: Inner MOTS have smaller area than outer MOTS.

The proof relied on:
\begin{itemize}
    \item Stability analysis (correct)
    \item The sign of $P$ in the trapped region (NOT PROVEN)
\end{itemize}
\end{gap}

\subsection{The $P$ Sign Issue}

We claimed that for black hole formation, $P < 0$ in the trapped region.

This is PHYSICALLY EXPECTED but NOT PROVEN from DEC!

\begin{question}
\textbf{Does DEC imply $P < 0$ in trapped regions?}

The momentum constraint:
\begin{equation}
    \divv(k - (\tr k)g) = 8\pi J
\end{equation}

Under DEC: $|J| \le \mu$.

But this does NOT determine the sign of $P = \tr_\Sigma k$!
\end{question}

\subsection{Counterexample Scenario}

Consider initial data with:
\begin{itemize}
    \item A MOTS $\Sigma_1$ at small radius
    \item A MOTS $\Sigma_2$ at large radius (outermost)
    \item $P > 0$ on $\Sigma_1$ (outgoing momentum)
\end{itemize}

In this case, the area comparison $\Area(\Sigma_1) \le \Area(\Sigma_2)$ is 
NOT guaranteed by our argument.

\begin{issue}
\textbf{The MOTS Hierarchy requires additional assumptions beyond DEC.}

Either:
\begin{enumerate}
    \item Prove $P < 0$ from some physical condition
    \item Find a different proof that doesn't use $P$ sign
    \item Accept MOTS Hierarchy as an additional assumption
\end{enumerate}
\end{issue}

%% ============================================================================
\section{Gap 5: Convergence to MOTS}
%% ============================================================================

\begin{gap}
\textbf{The flow may not converge to a MOTS.}

Claimed: ENF drives $\theta^+ \to 0$, so the limit is MOTS.

But:
\begin{itemize}
    \item The flow might not exist for all time
    \item The flow might oscillate
    \item The "limit" might be singular
\end{itemize}
\end{gap}

\subsection{Singularity Formation}

Geometric flows often develop singularities:
\begin{itemize}
    \item Ricci flow: neck pinches
    \item MCF: Type I and II singularities
    \item IMCF: jumps at minimal surfaces
\end{itemize}

\begin{question}
\textbf{What are the possible singularities of ENF?}

The ENF is driven by $\theta^+$, which approaches zero at MOTS.

The flow "slows down" near MOTS - this is good for convergence.

But what happens if the surface topology changes?
\end{question}

%% ============================================================================
\section{Gap 6: Area Monotonicity Revisited}
%% ============================================================================

Let me redo the area calculation more carefully.

\begin{question}
\textbf{Assume ENF is a surface flow (not metric flow).}

$\frac{\partial\Sigma}{\partial t} = -\theta^+\nu$ (outward when $\theta^+ < 0$).

The area evolution is:
\begin{equation}
    \frac{d\Area}{dt} = \int_\Sigma H \cdot (-\theta^+) dA = -\int_\Sigma H\theta^+ dA
\end{equation}

Write $H = \theta^+ - P$:
\begin{equation}
    \frac{d\Area}{dt} = -\int (\theta^+ - P)\theta^+ dA = -\int (\theta^+)^2 dA + \int P\theta^+ dA
\end{equation}

The first term: $-\int(\theta^+)^2 dA < 0$ (DECREASES area!)

The second term: $\int P\theta^+ dA$ has unclear sign.

\textbf{Total: NOT obviously positive!}
\end{question}

\begin{gap}
\textbf{The area monotonicity for the surface flow version of ENF is NOT proven.}

$\frac{d\Area}{dt} = -\int(\theta^+)^2 dA + \int P\theta^+ dA$

For trapped ($\theta^+ < 0$):
\begin{itemize}
    \item First term: $-(\text{negative})^2 < 0$
    \item Second term: $P \cdot (\text{negative})$, sign depends on $P$
\end{itemize}

If $P < 0$: second term is positive, might dominate.

If $P > 0$: second term is negative, area definitely decreases!
\end{gap}

\textbf{This is the SAME obstruction we had before - the sign of $P$!}

%% ============================================================================
\section{Conclusion: The ENF Approach Has Gaps}
%% ============================================================================

\begin{enumerate}
    \item \textbf{ENF definition:} The flow as written is not well-posed (Gap 1, 3)
    \item \textbf{Area monotonicity:} Depends on the sign of $P$ (Gap 2, 6)
    \item \textbf{MOTS Hierarchy:} Requires $P < 0$ assumption (Gap 4)
    \item \textbf{Convergence:} Not rigorously proven (Gap 5)
\end{enumerate}

\textbf{The fundamental issue remains:}

\begin{center}
\fbox{\parbox{0.8\textwidth}{
The sign of $P = \tr_\Sigma k$ is NOT determined by DEC.

Area Dominance requires $P < 0$ (or something equivalent).

This is an ADDITIONAL PHYSICAL CONDITION beyond DEC.
}}
\end{center}

%% ============================================================================
\section{Possible Resolutions}
%% ============================================================================

\subsection{Resolution 1: Accept $P < 0$ as Physical}

For gravitational collapse scenarios, $P < 0$ is physically expected.

Define: \textbf{Collapsing initial data} has $P < 0$ on trapped surfaces.

Then Area Dominance holds for collapsing data.

\subsection{Resolution 2: Find a $P$-Independent Argument}

Perhaps there's an argument that doesn't use the sign of $P$ directly.

This would require a completely different approach.

\subsection{Resolution 3: Prove $P < 0$ from Dynamics}

Use the EVOLUTION equations (not just constraints) to show that 
dynamically-formed trapped surfaces have $P < 0$.

This connects to cosmic censorship and black hole formation theorems.

\subsection{Resolution 4: Area Dominance May Be False}

Perhaps there exist DEC-satisfying initial data where Area Dominance fails.

These would be "exotic" data not arising from gravitational collapse.

\textbf{The Penrose inequality might only hold under additional conditions.}

\end{document}
