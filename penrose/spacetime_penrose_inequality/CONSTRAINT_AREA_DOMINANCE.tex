%% CONSTRAINT_AREA_DOMINANCE.tex
%%
%% New Attack: Using Constraint Equations Directly
%% December 2025

\documentclass[11pt]{amsart}
\usepackage{amsmath,amssymb,amsthm}
\usepackage{xcolor}

\newtheorem{theorem}{Theorem}
\newtheorem{lemma}[theorem]{Lemma}
\newtheorem{proposition}[theorem]{Proposition}
\newtheorem{corollary}[theorem]{Corollary}
\theoremstyle{definition}
\newtheorem{definition}[theorem]{Definition}
\newtheorem{remark}[theorem]{Remark}

\newcommand{\Area}{\mathrm{Area}}
\newcommand{\divv}{\mathrm{div}}
\DeclareMathOperator{\tr}{tr}

\title{Constraint Equations and Area Dominance:\\
A Direct PDE Attack}
\date{December 2025}

\begin{document}
\maketitle

\section{The Strategy}

\textbf{Goal:} Prove $\Area(\Sigma) \le \Area(\Sigma^*)$ for trapped $\Sigma$ inside MOTS $\Sigma^*$.

\textbf{New idea:} Use the constraint equations as \textbf{elliptic PDEs} whose solutions encode area bounds.

\section{The Constraint Equations}

On initial data $(\mathcal{C}^3, g, k)$:

\textbf{Hamiltonian constraint:}
\begin{equation}\label{eq:ham}
    R_g - |k|^2 + (\tr k)^2 = 16\pi\mu
\end{equation}

\textbf{Momentum constraint:}
\begin{equation}\label{eq:mom}
    \divv(k - (\tr k)g) = 8\pi J
\end{equation}

\textbf{DEC:} $\mu \ge |J|$.

\section{Key Identity: Expansion vs Constraint}

For a surface $\Sigma$ with unit normal $\nu$:

\begin{equation}
    \theta^\pm = H \pm P
\end{equation}

where $H = \divv_\Sigma \nu$ (mean curvature) and $P = \tr_\Sigma k$ (trace of extrinsic curvature on $\Sigma$).

\textbf{Important:}
\begin{equation}
    \theta^+\theta^- = H^2 - P^2
\end{equation}

For trapped surfaces: $\theta^+\theta^- > 0$, so $|H| > |P|$.

For MOTS: $\theta^+ = 0$, so $H = -P$.

\section{The Schoen-Yau Identity}

Consider the identity (from the constraint equations):

\begin{equation}
    \int_\Sigma (R_g - 2\mathrm{Ric}(\nu,\nu)) dA = \int_\Sigma (|k|^2 - (\tr k)^2 + 16\pi\mu + 2\mathrm{Ric}(\nu,\nu) - R_g) dA
\end{equation}

Using Gauss equation on $\Sigma$:
\begin{equation}
    R_\Sigma = R_g - 2\mathrm{Ric}(\nu,\nu) + H^2 - |\mathrm{I\!I}|^2
\end{equation}

By Gauss-Bonnet for $\Sigma \cong S^2$:
\begin{equation}
    \int_\Sigma R_\Sigma dA = 8\pi
\end{equation}

Therefore:
\begin{equation}\label{eq:SY}
    8\pi = \int_\Sigma (R_g - 2\mathrm{Ric}(\nu,\nu) + H^2 - |\mathrm{I\!I}|^2) dA
\end{equation}

\section{Controlling Area via Curvature}

From \eqref{eq:SY}:
\begin{equation}
    8\pi = \int_\Sigma H^2 dA + \int_\Sigma (R_g - 2\mathrm{Ric}(\nu,\nu) - |\mathrm{I\!I}|^2) dA
\end{equation}

For a round sphere of radius $r$: $H = 2/r$, $\Area = 4\pi r^2$, so $\int H^2 dA = 16\pi/r^2 \cdot r^2/4 = 4\pi/r^2 \cdot 4\pi r^2 = 16\pi$.

Wait, let me redo: $H = 2/r$ constant, $\Area = 4\pi r^2$, so 
\begin{equation}
    \int H^2 dA = H^2 \cdot \Area = \frac{4}{r^2} \cdot 4\pi r^2 = 16\pi
\end{equation}

And for flat $\mathbb{R}^3$: $R_g = 0$, $\mathrm{Ric} = 0$, $|\mathrm{I\!I}|^2 = 2/r^2$, so
\begin{equation}
    \int |\mathrm{I\!I}|^2 dA = \frac{2}{r^2} \cdot 4\pi r^2 = 8\pi
\end{equation}

Check: $16\pi + 0 - 8\pi = 8\pi$. ✓

\section{The Area Bound Attempt}

\textbf{Key inequality:} By Cauchy-Schwarz:
\begin{equation}
    \left(\int_\Sigma H \, dA\right)^2 \le \Area(\Sigma) \cdot \int_\Sigma H^2 dA
\end{equation}

From the trapped condition $\theta^+ = H + P < 0$:
\begin{equation}
    H < -P
\end{equation}

Integrating:
\begin{equation}
    \int_\Sigma H \, dA < -\int_\Sigma P \, dA
\end{equation}

\textbf{Bounding $\int P \, dA$:}

Using $P = \tr_\Sigma k = k(\nu,\nu)$ on $\Sigma$... no wait, $P = \gamma^{ij}k_{ij}$ where $\gamma$ is the induced metric.

Actually, $P = k_{ij}\gamma^{ij}$ involves only tangential components of $k$.

\textbf{Key formula:}
\begin{equation}
    P = \tr k - k(\nu,\nu)
\end{equation}

(Total trace minus normal-normal component.)

\section{The Flux Through $\Sigma$}

Define the \textbf{expansion flux}:
\begin{equation}
    \Phi(\Sigma) = \int_\Sigma P \, dA = \int_\Sigma (\tr k - k(\nu,\nu)) dA
\end{equation}

Using the momentum constraint \eqref{eq:mom}:
\begin{equation}
    \divv_g(k - (\tr k)g) = 8\pi J
\end{equation}

Let $X$ be a vector field. Then:
\begin{equation}
    \int_\Sigma (k - (\tr k)g)(\nu, X) dA = \int_{\Omega} (\divv(k-(\tr k)g) \cdot X + (k-(\tr k)g)(\nabla X)) dV
\end{equation}

With $X = \nabla u$ for harmonic $u$, this becomes constrained by DEC.

\section{The Harmonic Function Method}

Let $\Omega$ be the region between $\Sigma$ (inner) and $\Sigma^*$ (outer boundary).

Solve the Dirichlet problem:
\begin{equation}
    \Delta u = 0 \quad \text{in } \Omega, \qquad u|_\Sigma = 0, \quad u|_{\Sigma^*} = 1
\end{equation}

The \textbf{capacity} of $(\Sigma, \Sigma^*)$:
\begin{equation}
    \mathrm{Cap}(\Sigma, \Sigma^*) = \int_\Omega |\nabla u|^2 dV = \int_{\Sigma^*} \frac{\partial u}{\partial\nu} dA = -\int_\Sigma \frac{\partial u}{\partial\nu} dA
\end{equation}

\textbf{Standard estimate:}
\begin{equation}
    \mathrm{Cap}(\Sigma, \Sigma^*) \ge c \cdot \frac{\Area(\Sigma) \cdot \Area(\Sigma^*)}{d^2}
\end{equation}
where $d = $ distance between $\Sigma$ and $\Sigma^*$.

But this doesn't directly give $\Area(\Sigma) \le \Area(\Sigma^*)$.

\section{The Mass Aspect Function}

On MOTS $\Sigma^*$, define the \textbf{mass aspect function}:
\begin{equation}
    \mu_{\Sigma^*} = \frac{1}{2}(R_{\Sigma^*} - |\chi|^2 - |\mathrm{I\!I}|^2 + P^2)
\end{equation}

where $\chi$ is the shear of the null normal.

The Hawking mass:
\begin{equation}
    m_H(\Sigma^*) = \sqrt{\frac{\Area(\Sigma^*)}{16\pi}} \cdot \frac{1}{8\pi}\int_{\Sigma^*} \mu_{\Sigma^*} dA
\end{equation}

For MOTS with stability operator $L$:
\begin{equation}
    L\phi = -\Delta\phi + 2\omega\cdot\nabla\phi + Q\phi
\end{equation}

where $Q \ge 0$ under DEC (stability of outermost MOTS).

\section{A New Functional: Constraint-Based}

\textbf{Definition:} For surface $\Sigma$, define:
\begin{equation}
    \mathcal{M}[\Sigma] = \frac{1}{16\pi}\int_\Sigma (R_g - |k|^2 + (\tr k)^2) dA = \int_\Sigma \mu \, dA
\end{equation}

(The integrated matter energy on $\Sigma$.)

\textbf{Key insight:} $\mathcal{M}[\Sigma]$ depends on the matter content, not the geometry of $\Sigma$ directly.

\section{The Monotonicity Principle for Hawking Mass}

Hawking mass:
\begin{equation}
    m_H(\Sigma) = \sqrt{\frac{\Area}{16\pi}}\left(1 - \frac{1}{16\pi}\int_\Sigma \theta^+\theta^- dA\right)
\end{equation}

\textbf{Monotonicity (Bray-Hayward):}

Under inverse mean curvature flow in time-symmetric case ($k = 0$):
\begin{equation}
    \frac{dm_H}{dt} \ge 0
\end{equation}

In general, Hawking mass is monotonic under flows in null directions under certain conditions.

\textbf{Application:}

If we can flow from $\Sigma$ to $\Sigma^*$ with $m_H$ non-decreasing:
\begin{equation}
    m_H(\Sigma) \le m_H(\Sigma^*)
\end{equation}

For trapped $\Sigma$: $\theta^+\theta^- > 0$, so 
\begin{equation}
    m_H(\Sigma) = \sqrt{\frac{\Area(\Sigma)}{16\pi}}\left(1 - \text{positive}\right) < \sqrt{\frac{\Area(\Sigma)}{16\pi}}
\end{equation}

For MOTS $\Sigma^*$: $\theta^+ = 0$, so
\begin{equation}
    m_H(\Sigma^*) = \sqrt{\frac{\Area(\Sigma^*)}{16\pi}}
\end{equation}

If $m_H(\Sigma) \le m_H(\Sigma^*)$:
\begin{equation}
    \sqrt{\frac{\Area(\Sigma)}{16\pi}}(1 - \epsilon) \le \sqrt{\frac{\Area(\Sigma^*)}{16\pi}}
\end{equation}

This gives:
\begin{equation}
    \Area(\Sigma) \le \frac{\Area(\Sigma^*)}{(1-\epsilon)^2}
\end{equation}

This bounds $\Area(\Sigma)$ in terms of $\Area(\Sigma^*)$, but not sharply!

\textbf{The bound is:}
\begin{equation}
    \Area(\Sigma) \le \Area(\Sigma^*) \cdot \frac{1}{(1-\epsilon)^2}
\end{equation}

where $\epsilon = \frac{1}{16\pi}\int_\Sigma \theta^+\theta^- dA > 0$.

Since $(1-\epsilon)^2 < 1$, we have $\frac{1}{(1-\epsilon)^2} > 1$, so:
\begin{equation}
    \Area(\Sigma) \le \Area(\Sigma^*) \cdot (\text{something } > 1)
\end{equation}

This doesn't give $\Area(\Sigma) \le \Area(\Sigma^*)$!

\section{Critical Observation}

\textbf{The Hawking mass monotonicity gives a WEAKER bound than needed.}

We need $\Area(\Sigma) \le \Area(\Sigma^*)$, but Hawking mass only gives:
\begin{equation}
    \Area(\Sigma) \le C \cdot \Area(\Sigma^*)
\end{equation}
for some $C > 1$ depending on how trapped $\Sigma$ is.

\section{New Direction: The Isoperimetric Profile}

\textbf{Definition:} The isoperimetric profile of $(\mathcal{C}, g)$:
\begin{equation}
    I(V) = \inf\{\Area(\partial\Omega) : \Vol(\Omega) = V\}
\end{equation}

\textbf{Key property:} In regions of positive scalar curvature, $I(V)$ is bounded above by the Euclidean profile.

\textbf{Observation:} The trapped region (inside the outermost MOTS) has constrained scalar curvature via the constraint equations.

Under DEC:
\begin{equation}
    R_g = |k|^2 - (\tr k)^2 + 16\pi\mu \ge |k|^2 - (\tr k)^2
\end{equation}

The sign of $R_g$ is not determined in general.

\section{The Breakthrough: Using Stability}

\textbf{Key fact:} The outermost MOTS $\Sigma^*$ is \textbf{strictly stable}, meaning:
\begin{equation}
    \int_{\Sigma^*} \phi L\phi \, dA \ge 0 \quad \text{for all } \phi
\end{equation}

where $L = -\Delta + 2\omega\cdot\nabla + Q$ with $Q \ge 0$ under DEC.

\textbf{Stability implies:}
\begin{equation}
    \int_{\Sigma^*} (|\nabla\phi|^2 + Q\phi^2) dA \ge 0
\end{equation}

\textbf{For $\phi = 1$:}
\begin{equation}
    \int_{\Sigma^*} Q \, dA \ge 0
\end{equation}

The stability operator $Q$ encodes:
\begin{equation}
    Q = -\frac{1}{2}(R_{\Sigma^*} - |\chi|^2 + 8\pi T_{\mu\nu}\ell^+{}^\mu \ell^+{}^\nu + \frac{1}{2}(\tr_{\Sigma^*} k)^2)
\end{equation}

Under DEC, the null energy term is non-negative.

\section{Area Comparison via Stability}

\textbf{Variation formula:}

Consider deforming $\Sigma^*$ inward. The area variation:
\begin{equation}
    \delta\Area = \int_{\Sigma^*} H \phi \, dA
\end{equation}

For MOTS: $H = -P$, so:
\begin{equation}
    \delta\Area = -\int_{\Sigma^*} P\phi \, dA
\end{equation}

Moving inward ($\phi < 0$): $\delta\Area = -\int P\phi \, dA$.

If $P > 0$ (expanding universe), moving inward increases area!
If $P < 0$ (contracting), moving inward decreases area.

\textbf{For trapped surfaces inside:}

A surface $\Sigma$ inside $\Sigma^*$ has $\theta^+ < 0$.

The area of $\Sigma$ vs $\Sigma^*$ depends on:
\begin{enumerate}
    \item The "path" from $\Sigma$ to $\Sigma^*$
    \item The curvature along the path
    \item The constraint equations
\end{enumerate}

\section{Conclusion: The Problem is Geometric}

\textbf{Finding:}

Area Dominance is NOT a purely local property. It depends on:
\begin{itemize}
    \item Global geometry of the region between $\Sigma$ and $\Sigma^*$
    \item The constraint equations
    \item DEC (which constrains but doesn't determine $H$)
\end{itemize}

\textbf{What would prove it:}

A proof must use:
\begin{enumerate}
    \item The trapped condition $\theta^+ < 0$ (gives $H < -P$)
    \item The constraint equations (global constraints)
    \item DEC (energy conditions)
    \item Stability of outermost MOTS
\end{enumerate}

\textbf{The gap remains:} Converting these constraints into $\Area(\Sigma) \le \Area(\Sigma^*)$.

\section{Honest Assessment}

Area Dominance is:
\begin{itemize}
    \item TRUE in Schwarzschild
    \item TRUE in spherically symmetric spacetimes
    \item EXPECTED to be true generally (physical intuition)
    \item NOT PROVEN in full generality
\end{itemize}

\textbf{This is a genuine open problem in mathematical GR.}

\end{document}
