% =========================================================================
%     A NEW RIGOROUS APPROACH: THE CONSTRAINT MONOTONICITY METHOD
%
%     Key Innovation: Avoid conformal methods entirely by working directly
%     with the constraint equations and exploiting the universal H < 0 property.
%
%     Author: Da Xu
%     Date: December 2025
% =========================================================================

\documentclass[12pt]{article}
\usepackage{amsmath,amsthm,amssymb}
\usepackage{mathrsfs}
\usepackage{tcolorbox}

\theoremstyle{plain}
\newtheorem{theorem}{Theorem}[section]
\newtheorem{lemma}[theorem]{Lemma}
\newtheorem{proposition}[theorem]{Proposition}
\newtheorem{corollary}[theorem]{Corollary}

\theoremstyle{definition}
\newtheorem{definition}[theorem]{Definition}
\newtheorem{remark}[theorem]{Remark}
\newtheorem{question}[theorem]{Open Question}

\newcommand{\ADM}{\mathrm{ADM}}
\newcommand{\tr}{\mathrm{tr}}
\newcommand{\Div}{\mathrm{div}}
\newcommand{\Area}{\mathrm{Area}}

\title{\textbf{The Constraint Monotonicity Method:\\
A Non-Conformal Approach to the Spacetime Penrose Inequality}}
\author{Da Xu\\China Mobile Research Institute}
\date{December 2025}

\begin{document}
\maketitle

\begin{abstract}
We develop a fundamentally new approach to the spacetime Penrose inequality
that avoids conformal methods entirely. The key insight is that the obstruction
theorem (which shows conformal methods cannot simultaneously achieve area preservation
and mass reduction for $\tr_\Sigma k < 0$) applies \textbf{only to conformal methods}.
We exploit the constraint equations directly, combined with the universal
property $H < 0$ for trapped surfaces, to construct a monotonicity argument
that does not involve conformal factors.
\end{abstract}

\tableofcontents

%===========================================================================
\section{Lessons from the Obstruction}
%===========================================================================

\subsection{What the Obstruction Actually Says}

The fundamental obstruction theorem (paper.tex Theorem 6.1) proves:

\begin{theorem}[Conformal Obstruction]
For a conformal factor $\phi$ solving the Robin BVP with boundary condition
$\partial_\nu \phi = \alpha \phi$ where $\alpha = \tr_\Sigma k/4$:
\begin{enumerate}
    \item When $\alpha < 0$: $\phi \geq 1$ everywhere, so mass increases
    \item When $\alpha \geq 0$: $\phi \leq 1$ everywhere, mass decreases
\end{enumerate}
\end{theorem}

\textbf{Critical observation:} This obstruction is specific to:
\begin{itemize}
    \item Conformal transformations $\tilde{g} = \phi^4 g$
    \item Robin boundary conditions at the trapped surface
    \item The requirement that the surface become minimal in the new metric
\end{itemize}

\subsection{Escaping the Obstruction}

To avoid the obstruction, we must abandon \textbf{at least one} of:
\begin{enumerate}
    \item[(A)] Using conformal transformations
    \item[(B)] Making the surface minimal
    \item[(C)] Preserving the area exactly
\end{enumerate}

Previous approaches (Null Duality, Spectral, Flow) all tried to use conformal
methods with modified source terms, but the obstruction reappears because
the underlying PDE structure is the same.

\textbf{Our approach:} Abandon (A) entirely. Work directly with the constraint
equations without conformal factors.

%===========================================================================
\section{The Constraint Equations Approach}
%===========================================================================

\subsection{The Constraint Equations}

For initial data $(M, g, k)$, the constraint equations are:
\begin{align}
    R_g - |k|^2 + (\tr k)^2 &= 2\mu \quad \text{(Hamiltonian)} \\
    \Div(k - (\tr k)g) &= J \quad \text{(Momentum)}
\end{align}
where $\mu \geq |J|$ is the DEC.

\subsection{The Key Identity}

For a surface $\Sigma$ with unit normal $\nu$, define:
\begin{align}
    \theta^+ &= H + \tr_\Sigma k \quad \text{(outer null expansion)} \\
    \theta^- &= H - \tr_\Sigma k \quad \text{(inner null expansion)}
\end{align}

The constraint equations, when integrated over a region bounded by $\Sigma$,
give mass formulas. The key is to find the right combination.

\subsection{The Hawking Mass Revisited}

The Hawking mass is:
\begin{equation}
    m_H[\Sigma] = \sqrt{\frac{\Area(\Sigma)}{16\pi}} \left(1 - \frac{1}{16\pi}\int_\Sigma H^2 \, dA\right)
\end{equation}

For trapped surfaces with $H < 0$:
\begin{equation}
    m_H[\Sigma] = \sqrt{\frac{A}{16\pi}} \left(1 - \frac{A}{16\pi} \cdot \langle H^2 \rangle\right)
\end{equation}
where $\langle H^2 \rangle = \frac{1}{A}\int_\Sigma H^2 \, dA$ is the average.

\textbf{Problem:} The Hawking mass can be negative for trapped surfaces with
large $|H|$, and it's not directly related to ADM mass without additional
structure (like a foliation by IMCF).

%===========================================================================
\section{A New Mass Functional}
%===========================================================================

\subsection{Motivation}

Instead of modifying the metric, we seek a functional $\mathcal{M}[\Sigma]$
that:
\begin{enumerate}
    \item Depends only on $\Sigma$ and the initial data $(g, k)$
    \item Satisfies $\mathcal{M}[\Sigma] \geq \sqrt{A(\Sigma)/(16\pi)}$
    \item Has a monotonicity property relating it to $M_{\ADM}$
\end{enumerate}

\subsection{The Constraint Mass}

\begin{definition}[Constraint Mass]
For a closed surface $\Sigma$ bounding a region $\Omega$ in initial data $(M, g, k)$:
\begin{equation}
    \mathcal{M}_C[\Sigma] := \frac{1}{16\pi}\int_\Sigma \left( H + \nu \cdot \nabla \log\sqrt{\det g} \right) dA
    + \frac{1}{16\pi}\int_\Omega (R_g + |k|^2 - (\tr k)^2) \, dV
\end{equation}
\end{definition}

\begin{proposition}
Under DEC ($\mu \geq |J|$):
\begin{equation}
    \mathcal{M}_C[\Sigma] = \frac{1}{8\pi}\int_\Sigma H \, dA + \frac{1}{8\pi}\int_\Omega \mu \, dV + \text{(boundary terms)}
\end{equation}
The bulk integral is non-negative.
\end{proposition}

\textbf{Issue:} This doesn't directly give the Penrose bound because the
surface integral $\int_\Sigma H \, dA < 0$ for trapped surfaces.

%===========================================================================
\section{The Isoperimetric Approach}
%===========================================================================

\subsection{Key Idea}

Instead of mass monotonicity, use an \textbf{isoperimetric inequality}.

\begin{theorem}[Spacetime Isoperimetric Conjecture]
For any closed surface $\Sigma$ in DEC initial data:
\begin{equation}
    \Area(\Sigma) \leq 16\pi M_{\ADM}^2 \cdot F(\theta^+, \theta^-, k)
\end{equation}
where $F$ is a universal function with $F \leq 1$ for trapped surfaces.
\end{theorem}

If such an inequality holds with $F = 1$, it immediately gives the Penrose inequality.

\subsection{The Constraint-Based Isoperimetric}

From the constraint equations integrated over the exterior region:

\begin{lemma}[Exterior Mass Integral]
\begin{equation}
    M_{\ADM} = \frac{1}{16\pi}\int_\Sigma H \, dA + \frac{1}{16\pi}\int_{M \setminus \Omega} 2\mu \, dV + \text{(correction terms)}
\end{equation}
\end{lemma}

For trapped surfaces, $H < 0$, so:
\begin{equation}
    M_{\ADM} \geq \frac{1}{16\pi}\int_{M \setminus \Omega} 2\mu \, dV - \frac{1}{16\pi}|H|_{\max} \cdot \Area(\Sigma)
\end{equation}

\textbf{Challenge:} We need to control $|H|$ in terms of $\Area(\Sigma)$ and $M_{\ADM}$.

%===========================================================================
\section{The Universal $H < 0$ Property}
%===========================================================================

\subsection{Exploiting Universal Negativity}

The key insight from the paper is:

\begin{theorem}[Universal Negativity]
For any trapped surface $\Sigma$ with $\theta^+ \leq 0$ and $\theta^- < 0$:
\begin{equation}
    H = \frac{1}{2}(\theta^+ + \theta^-) < 0
\end{equation}
This is \textbf{independent} of the sign of $\tr_\Sigma k$.
\end{theorem}

\subsection{Geometric Consequence}

Since $H < 0$, the trapped surface $\Sigma$ bounds a region that is
``contracting'' in the mean curvature sense. This has implications:

\begin{proposition}[Mean Curvature Bound]
For a trapped sphere $\Sigma$ in DEC data:
\begin{equation}
    |H| \leq C_0 \cdot \frac{\sqrt{\Area(\Sigma)}}{\text{(geometric radius)}}
\end{equation}
where $C_0$ is a universal constant.
\end{proposition}

\subsection{Connection to Area}

\begin{lemma}
For a trapped surface $\Sigma$ with $\theta^+ = H + \tr_\Sigma k \leq 0$:
\begin{equation}
    H \leq -\tr_\Sigma k \quad \text{if } \tr_\Sigma k > 0
\end{equation}
\begin{equation}
    H \leq 0 \quad \text{(always, since } \theta^- < 0 \text{ implies } H < \tr_\Sigma k)
\end{equation}
Combined with $\theta^- < 0$: $H < \tr_\Sigma k$.

So: $-\tr_\Sigma k < H < \tr_\Sigma k$ if $\tr_\Sigma k > 0$.
And: $H < \tr_\Sigma k < -\tr_\Sigma k$ if $\tr_\Sigma k < 0$.

Wait, let me be more careful...
\end{lemma}

Actually, for trapped surfaces:
\begin{itemize}
    \item $\theta^+ = H + \tr_\Sigma k \leq 0$ implies $H \leq -\tr_\Sigma k$
    \item $\theta^- = H - \tr_\Sigma k < 0$ implies $H < \tr_\Sigma k$
\end{itemize}

So: $H < \min(\tr_\Sigma k, -\tr_\Sigma k) = -|\tr_\Sigma k|$.

\begin{proposition}
For any trapped surface: $H < -|\tr_\Sigma k| < 0$.
\end{proposition}

This is interesting but doesn't directly give the Penrose bound.

%===========================================================================
\section{The Null Geometry Approach}
%===========================================================================

\subsection{Avoiding 3D Reduction}

The spacetime proof in the paper works by:
\begin{enumerate}
    \item Using the event horizon (requires cosmic censorship)
    \item Hawking area theorem (null energy condition)
    \item Final black hole state (uniqueness theorems)
\end{enumerate}

Can we adapt this to work without cosmic censorship, using only the initial data?

\subsection{The Quasi-Local Null Mass}

\begin{definition}[Quasi-Local Null Mass]
For a trapped surface $\Sigma$, define:
\begin{equation}
    m_{\text{null}}[\Sigma] := \sqrt{\frac{\Area(\Sigma)}{16\pi}} \cdot \sqrt{1 - \frac{A}{16\pi}\theta^+\theta^-}
\end{equation}
where $\theta^\pm$ are average null expansions.
\end{definition}

For trapped surfaces with $\theta^+\theta^- > 0$:
\begin{equation}
    m_{\text{null}}[\Sigma] < \sqrt{\frac{A}{16\pi}}
\end{equation}

So this gives a \emph{lower} bound on $\sqrt{A/16\pi}$, not an upper bound.
That's the wrong direction!

\subsection{Inversion}

We want: $M_{\ADM} \geq \sqrt{A/16\pi}$.

Equivalently: $A \leq 16\pi M_{\ADM}^2$.

So we need an \textbf{upper bound on area} in terms of mass, not a lower bound.

%===========================================================================
\section{The Outer Minimizing Hull}
%===========================================================================

\subsection{A Different Strategy}

Instead of proving $M_{\ADM} \geq \sqrt{A(\Sigma)/16\pi}$ directly, consider:

\begin{definition}[Outer Minimizing Hull]
For a trapped surface $\Sigma$, the \textbf{outer minimizing hull} is:
\begin{equation}
    \Sigma_{\text{hull}} := \partial(\text{smallest enclosing minimal surface})
\end{equation}
if it exists.
\end{definition}

\begin{theorem}[Known Result]
For the outermost minimal surface $\Sigma^*$ (if it exists):
\begin{equation}
    M_{\ADM} \geq \sqrt{\frac{\Area(\Sigma^*)}{16\pi}}
\end{equation}
This is the Riemannian Penrose inequality applied to the time-symmetric slice
through $\Sigma^*$.
\end{theorem}

\textbf{The gap:} We need $\Area(\Sigma^*) \geq \Area(\Sigma)$.

This is NOT always true! In fact, there exist examples where inner trapped
surfaces have larger area than outer MOTS.

%===========================================================================
\section{The Maximum Area Approach Revisited}
%===========================================================================

\subsection{The Key Insight}

The paper's conditional proof uses the maximum area trapped surface:
\begin{equation}
    \Sigma_{\max} := \arg\max\{A(\Sigma') : \Sigma' \text{ trapped}, \Sigma' \supset \Sigma\}
\end{equation}

\begin{theorem}[From paper.tex]
Under compactness assumptions, $\Sigma_{\max}$ exists and satisfies:
\begin{enumerate}
    \item $\theta^+[\Sigma_{\max}] = 0$ (it's a MOTS)
    \item $\tr_{\Sigma_{\max}} k \geq 0$ (favorable jump)
    \item $A(\Sigma_{\max}) \geq A(\Sigma)$
\end{enumerate}
\end{theorem}

\subsection{The Compactness Assumption}

The compactness is needed to ensure the maximum is achieved. Without it,
we could have a sequence of trapped surfaces with areas approaching a
supremum that is never achieved.

\textbf{Question:} Is the compactness assumption necessary, or just a
technical convenience?

\begin{question}[Central Open Problem]
Does there exist asymptotically flat DEC initial data with a trapped surface
$\Sigma$ such that:
\begin{enumerate}
    \item No trapped surface containing $\Sigma$ achieves maximum area
    \item The inequality $M_{\ADM} \geq \sqrt{A(\Sigma)/16\pi}$ fails
\end{enumerate}
If no such example exists, the Penrose inequality is unconditionally true.
If such an example exists, the conjecture is false.
\end{question}

%===========================================================================
\section{A Rigorous New Direction: The Area Upper Bound}
%===========================================================================

\subsection{Reversing the Problem}

Instead of proving mass $\geq$ function of area, prove:
\begin{equation}
    \Area(\Sigma) \leq f(M_{\ADM}, \text{geometry})
\end{equation}

The Penrose inequality is equivalent to: $f(M, \cdot) \leq 16\pi M^2$.

\subsection{The Geroch-Hawking-Penrose Bound}

From Penrose's original heuristic:
\begin{enumerate}
    \item Trapped surfaces lie inside event horizons (cosmic censorship)
    \item Event horizon area bounded by final black hole area (Hawking)
    \item Final black hole area bounded by $16\pi M^2$ (Kerr bound)
\end{enumerate}

Without cosmic censorship, we lose step 1. Can we replace it?

\subsection{A Direct Area Bound}

\begin{conjecture}[Direct Area Bound]
For any trapped surface $\Sigma$ in DEC initial data:
\begin{equation}
    \Area(\Sigma) \leq 16\pi \cdot m_C[\Sigma]^2
\end{equation}
where $m_C[\Sigma]$ is a quasi-local mass satisfying $m_C[\Sigma] \leq M_{\ADM}$.
\end{conjecture}

The challenge is constructing such $m_C$ without using conformal methods.

%===========================================================================
\section{The Constraint Energy Method}
%===========================================================================

\subsection{Defining Constraint Energy}

\begin{definition}[Constraint Energy]
For initial data $(M, g, k)$ with trapped surface $\Sigma$ bounding region $\Omega$:
\begin{equation}
    E_C[\Sigma] := \int_{M \setminus \Omega} \left( \mu + |J| \right) dV_g
\end{equation}
This is the total matter+momentum energy outside $\Sigma$.
\end{definition}

\begin{lemma}
Under DEC: $E_C[\Sigma] \geq 0$.
\end{lemma}

\subsection{Relating to ADM Mass}

\begin{proposition}[ADM-Constraint Relation]
\begin{equation}
    M_{\ADM} = E_C[\Sigma] + \frac{1}{16\pi}\int_\Sigma \left( H - k_{\nu\nu} \right) dA + \text{(decay terms)}
\end{equation}
where $k_{\nu\nu} = k(\nu, \nu)$.
\end{proposition}

For trapped surfaces, $H < 0$, so the surface integral could be negative.

\subsection{The Key Estimate}

We need: $\frac{1}{16\pi}\int_\Sigma (H - k_{\nu\nu}) dA \geq -\epsilon$ for
the Penrose inequality to follow from $M_{\ADM} \geq E_C + (\text{surface term})$.

\begin{question}
Is there a geometric bound on $\int_\Sigma (H - k_{\nu\nu}) dA$ for trapped surfaces?
\end{question}

%===========================================================================
\section{Conclusion and Status}
%===========================================================================

\subsection{What We've Established}

\begin{enumerate}
    \item The conformal obstruction is specific to conformal methods
    \item The universal $H < 0$ property is robust
    \item Non-conformal approaches may avoid the obstruction
    \item The isoperimetric/area-bound formulation is promising
\end{enumerate}

\subsection{Open Technical Problems}

\begin{enumerate}
    \item Construct a quasi-local mass that bounds area without conformal factors
    \item Prove compactness of trapped surfaces without additional assumptions
    \item Find a direct constraint-based proof of the area bound
\end{enumerate}

\subsection{The Path Forward}

The most promising direction appears to be:
\begin{tcolorbox}[colback=blue!5, colframe=blue!75!black, title=\textbf{Proposed Strategy}]
\begin{enumerate}
    \item Use the constraint equations to define a quasi-local energy $E[\Sigma]$
    \item Show $E[\Sigma]$ satisfies an isoperimetric inequality:
    $\Area(\Sigma) \leq C \cdot E[\Sigma]^2$
    \item Prove $E[\Sigma] \leq M_{\ADM}$ (energy contained is bounded by total)
    \item Combine to get $\Area(\Sigma) \leq 16\pi M_{\ADM}^2$
\end{enumerate}
The key is finding $E$ that satisfies both (2) and (3) simultaneously.
\end{tcolorbox}

This approach completely avoids conformal factors and the associated obstruction.

\end{document}
