\documentclass[11pt]{article}
\usepackage{amsmath,amssymb,amsthm}
\usepackage[margin=1in]{geometry}
\usepackage{tcolorbox}
\usepackage{xcolor}

\newtheorem{theorem}{Theorem}[section]
\newtheorem{lemma}[theorem]{Lemma}
\newtheorem{proposition}[theorem]{Proposition}
\newtheorem{corollary}[theorem]{Corollary}
\newtheorem{definition}[theorem]{Definition}
\newtheorem{remark}[theorem]{Remark}
\newtheorem{openproblem}{Open Problem}

\newtcolorbox{gap}{colback=red!5!white, colframe=red!75!black, title=\textbf{GAP}}
\newtcolorbox{success}{colback=green!5!white, colframe=green!75!black, title=\textbf{RIGOROUS}}
\newtcolorbox{key}{colback=blue!5!white, colframe=blue!75!black, title=\textbf{KEY INSIGHT}}

\title{\textbf{Proof Attempt: Trapping Product Method}\\
\large Exploiting $\theta^+\theta^- > 0$ for Trapped Surfaces}
\author{Research Notes}
\date{December 2025}

\begin{document}
\maketitle

\begin{abstract}
We attempt to prove the spacetime Penrose inequality using the observation that $\theta^+\theta^- > 0$ for trapped surfaces. This product is \textbf{symmetric} in the null expansions and avoids the sign ambiguity of $\mathrm{tr}_\Sigma k$.
\end{abstract}

%==============================================================================
\section{The Key Observation}
%==============================================================================

\begin{key}
For a trapped surface $\Sigma$ with $\theta^+ \leq 0$ and $\theta^- < 0$:
\begin{equation}
\theta^+ \cdot \theta^- \geq 0
\end{equation}
with equality iff $\theta^+ = 0$ (i.e., $\Sigma$ is a MOTS).

This is \textbf{independent} of the sign of $\mathrm{tr}_\Sigma k = \frac{1}{2}(\theta^+ - \theta^-)$.
\end{key}

\begin{lemma}[Algebraic Relations]
\begin{align}
\theta^+ + \theta^- &= 2H < 0 \quad \text{(trapped)} \\
\theta^+ - \theta^- &= 2\,\mathrm{tr}_\Sigma k \quad \text{(sign undetermined)} \\
\theta^+ \cdot \theta^- &= H^2 - (\mathrm{tr}_\Sigma k)^2 \geq 0 \quad \text{(trapped)}
\end{align}
The third relation gives:
\begin{equation}
|H| \geq |\mathrm{tr}_\Sigma k|
\end{equation}
for trapped surfaces!
\end{lemma}

\begin{success}
This is a rigorous algebraic fact.
\end{success}

\begin{corollary}[Bound on Unfavorable Part]
For any trapped surface:
\begin{equation}
|\mathrm{tr}_\Sigma k| \leq |H| = -H
\end{equation}
In particular, if $\mathrm{tr}_\Sigma k < 0$:
\begin{equation}
-\mathrm{tr}_\Sigma k \leq -H
\end{equation}
\end{corollary}

%==============================================================================
\section{The Trapping Mass Functional}
%==============================================================================

\begin{definition}[Trapping Mass]
For a trapped surface $\Sigma$, define:
\begin{equation}
m_T(\Sigma) := \sqrt{\frac{A(\Sigma)}{16\pi}} \left(1 - \frac{1}{16\pi}\int_\Sigma \theta^+\theta^- \, dA\right)
\end{equation}
\end{definition}

\begin{lemma}[Properties of Trapping Mass]
\begin{enumerate}
\item For trapped surfaces: $\theta^+\theta^- \geq 0$, so $m_T(\Sigma) \leq \sqrt{A/(16\pi)}$
\item For MOTS ($\theta^+ = 0$): $m_T(\Sigma) = \sqrt{A/(16\pi)}$
\item Connection to Hawking mass: $m_H(\Sigma) = \sqrt{\frac{A}{16\pi}}\left(1 - \frac{1}{16\pi}\int_\Sigma H^2\,dA + \frac{1}{16\pi}\int_\Sigma (\mathrm{tr}_\Sigma k)^2\,dA\right)$
\end{enumerate}
\end{lemma}

\begin{proof}
Property 3: Using $\theta^+\theta^- = H^2 - (\mathrm{tr}_\Sigma k)^2$:
\begin{align}
m_T(\Sigma) &= \sqrt{\frac{A}{16\pi}}\left(1 - \frac{1}{16\pi}\int_\Sigma (H^2 - (\mathrm{tr}_\Sigma k)^2)\,dA\right) \\
&= \sqrt{\frac{A}{16\pi}}\left(1 - \frac{1}{16\pi}\int_\Sigma H^2\,dA + \frac{1}{16\pi}\int_\Sigma (\mathrm{tr}_\Sigma k)^2\,dA\right)
\end{align}
Compare to Hawking mass $m_H = \sqrt{\frac{A}{16\pi}}\left(1 - \frac{1}{16\pi}\int H^2\right)$ for $k=0$.
\end{proof}

%==============================================================================
\section{Monotonicity Attempt}
%==============================================================================

\begin{openproblem}[Trapping Mass Monotonicity]
Find a flow $\Sigma_t$ from a trapped surface $\Sigma_0$ to the MOTS $\Sigma^*$ such that:
\begin{equation}
\frac{d}{dt} m_T(\Sigma_t) \geq 0
\end{equation}
\end{openproblem}

\begin{proposition}[Evolution of Trapping Mass]
Under a flow $\partial_t X = \phi \nu$ with $\phi > 0$:
\begin{align}
\frac{dA}{dt} &= \int_\Sigma H\phi \, dA \\
\frac{d}{dt}\int_\Sigma \theta^+\theta^- \, dA &= \int_\Sigma \left[\frac{\partial(\theta^+\theta^-)}{\partial t} + \theta^+\theta^- H\phi\right] dA
\end{align}
The evolution of $\theta^\pm$ involves the stability operators and Raychaudhuri-type equations.
\end{proposition}

\begin{gap}
\textbf{The evolution of $\theta^+\theta^-$ is complicated.} Using:
\begin{equation}
\frac{\partial \theta^+}{\partial t} = -L^+[\phi] - \theta^+ H\phi
\end{equation}
where $L^+$ is the MOTS stability operator, we get:
\begin{align}
\frac{\partial(\theta^+\theta^-)}{\partial t} &= \theta^- \frac{\partial\theta^+}{\partial t} + \theta^+ \frac{\partial\theta^-}{\partial t} \\
&= -\theta^- L^+[\phi] - \theta^+ L^-[\phi] - (\theta^+\theta^-)H\phi - (\theta^+\theta^-)H\phi
\end{align}
This involves both stability operators $L^\pm$, which have different properties.
\end{gap}

%==============================================================================
\section{The $\theta^+\theta^-$-IMCF}
%==============================================================================

\begin{definition}[Product IMCF]
Define the flow:
\begin{equation}
\frac{\partial X}{\partial t} = \frac{1}{\sqrt{\theta^+\theta^-}} \nu
\end{equation}
This is the inverse mean curvature flow with $H$ replaced by $\sqrt{\theta^+\theta^-}$.
\end{definition}

\begin{lemma}[Area Evolution]
\begin{equation}
\frac{dA}{dt} = \int_\Sigma \frac{H}{\sqrt{\theta^+\theta^-}} \, dA
\end{equation}
For trapped surfaces with $H < 0$ and $\theta^+\theta^- > 0$:
\begin{equation}
\frac{dA}{dt} < 0
\end{equation}
Area is \textbf{decreasing}!
\end{lemma}

\begin{gap}
\textbf{Wrong direction!} For the Penrose inequality, we want area to increase (or at least have a monotone mass-like quantity). The Product IMCF decreases area because $H < 0$ for trapped surfaces.

Moreover, the flow is singular at MOTS where $\theta^+ = 0$, making $\theta^+\theta^- = 0$.
\end{gap}

%==============================================================================
\section{Alternative: Outward $\theta^+$-Flow}
%==============================================================================

\begin{definition}[$\theta^+$-Flow]
\begin{equation}
\frac{\partial X}{\partial t} = -\theta^+ \nu
\end{equation}
Since $\theta^+ \leq 0$ for trapped surfaces, $-\theta^+ \geq 0$, so this flows \textbf{outward}.
\end{definition}

\begin{lemma}[Area Evolution under $\theta^+$-Flow]
\begin{equation}
\frac{dA}{dt} = \int_\Sigma H \cdot (-\theta^+) \, dA = -\int_\Sigma H\theta^+ \, dA
\end{equation}
For trapped surfaces: $H < 0$ and $\theta^+ \leq 0$, so $H\theta^+ \geq 0$, giving:
\begin{equation}
\frac{dA}{dt} \leq 0
\end{equation}
Still decreasing!
\end{lemma}

\begin{remark}[Why Area Always Decreases]
For \textbf{any} outward flow ($\phi > 0$) on trapped surfaces:
\begin{equation}
\frac{dA}{dt} = \int_\Sigma H\phi \, dA < 0
\end{equation}
because $H < 0$ for trapped surfaces.

\textbf{This is the fundamental geometric fact:} Trapped surfaces are mean-convex toward the interior. Moving outward decreases area.
\end{remark}

%==============================================================================
\section{The Weak Solution Approach}
%==============================================================================

\begin{key}
The Huisken-Ilmanen weak IMCF allows area to \textbf{jump upward} at singularities. Perhaps a similar weak formulation can work for the $\theta^+$-flow.
\end{key}

\begin{definition}[Weak $\theta^+$-Flow (Tentative)]
A weak solution to the $\theta^+$-flow is a family of sets $\{E_t\}_{t \geq 0}$ with $\partial E_t = \Sigma_t$ such that:
\begin{enumerate}
\item $E_0 \subset E_t$ for $t > 0$ (monotone in inclusion)
\item $\Sigma_t$ satisfies the $\theta^+$-flow classically away from singularities
\item At singularities, $\Sigma_t$ ``jumps'' to the next outer surface
\end{enumerate}
\end{definition}

\begin{gap}
\textbf{Existence of weak solutions is not established.}

The Huisken-Ilmanen theory uses the level set method with the equation:
\begin{equation}
|\nabla u| \cdot \mathrm{div}\left(\frac{\nabla u}{|\nabla u|}\right) = 1
\end{equation}
For the $\theta^+$-flow, we would need:
\begin{equation}
|\nabla u| \cdot \theta^+\left(\{u = t\}\right) = -1
\end{equation}
But $\theta^+$ depends on $k$, making this a more complicated PDE.

The existence and uniqueness theory for this equation is \textbf{OPEN}.
\end{gap}

%==============================================================================
\section{Assessment}
%==============================================================================

\begin{tcolorbox}[colback=yellow!10!white, colframe=orange!75!black, title=\textbf{STATUS: INCOMPLETE}]
The Trapping Product approach reveals important structure:

\textbf{Positive result:} $\theta^+\theta^- > 0$ for trapped surfaces, giving $|H| \geq |\mathrm{tr}_\Sigma k|$

\textbf{Issue 1:} Any smooth outward flow on trapped surfaces \textbf{decreases area} (because $H < 0$)

\textbf{Issue 2:} Weak solutions (allowing jumps) might help, but existence theory is \textbf{OPEN}

\textbf{Issue 3:} The trapping mass $m_T$ has no obvious monotonicity under natural flows

\textbf{Key insight:} The problem is not the sign of $\mathrm{tr}_\Sigma k$ per se, but the sign of $H$. For trapped surfaces, $H < 0$, which causes area to decrease under outward flows. This is a \textbf{geometric obstruction}, not just an algebraic sign issue.

\textbf{Conclusion:} The $\theta^+\theta^- > 0$ observation is valuable but does \textbf{not} directly lead to a proof. Need weak solutions or a completely different approach.
\end{tcolorbox}

\end{document}
