% RIGOROUS ANALYSIS: Can a trapped surface have larger area than its enclosing outermost MOTS?
%
% This document analyzes whether A(Σ*) ≥ A(Σ₀) is TRUE or FALSE

\documentclass{article}
\usepackage{amsmath,amsthm,amssymb}
\newtheorem{theorem}{Theorem}
\newtheorem{lemma}{Lemma}
\newtheorem{proposition}{Proposition}
\newtheorem{claim}{Claim}
\newtheorem{conjecture}{Conjecture}
\newtheorem{counterexample}{Counterexample}
\newtheorem{remark}{Remark}

\begin{document}

\section{Setup}

Let $(M^3, g, k)$ be asymptotically flat initial data satisfying DEC.

Let $\Sigma_0$ be a closed trapped surface: $\theta^+(\Sigma_0) \le 0$, $\theta^-(\Sigma_0) < 0$.

Let $\Sigma^* = \partial \mathcal{T}$ be the outermost MOTS (boundary of trapped region).

\textbf{Question:} Is $A(\Sigma^*) \ge A(\Sigma_0)$?

\section{Evidence FOR the inequality}

\subsection{Argument 1: Topological constraint}

$\Sigma_0$ is INSIDE the trapped region $\mathcal{T}$.
$\Sigma^*$ is the BOUNDARY of $\mathcal{T}$.

In Euclidean space, inner surfaces have smaller area than boundaries.

But $(M, g)$ is not Euclidean - the metric can be highly curved inside $\mathcal{T}$.

\subsection{Argument 2: The trapped region is like a black hole interior}

In Schwarzschild, spheres inside the horizon have SMALLER area than the horizon.

The area function $A(r) = 4\pi r^2$ decreases as $r \to 0$.

This suggests $A(\Sigma^*) \ge A(\Sigma_0)$.

\subsection{Argument 3: Mean curvature sign}

In the trapped region: $H = \frac{1}{2}(\theta^+ + \theta^-) < 0$.

Surfaces are mean-convex toward the INTERIOR.

This means: moving OUTWARD decreases area (locally), moving INWARD increases area.

So surfaces deeper inside should have larger area?? This contradicts the Schwarzschild intuition!

Wait - this argument says $A(\Sigma_0) > A(\Sigma^*)$. Let me reconsider...

\subsection{Reconsideration: Direction of mean curvature}

$H < 0$ means the surface is curving toward the exterior (like a sphere with inward normal).

The first variation of area: $\delta A = \int_\Sigma H \phi \, dA$ where $\phi$ is the normal variation.

For outward variation ($\phi > 0$): $\delta A < 0$ since $H < 0$.

So moving outward DECREASES area. This means:
- $\Sigma_0$ (inner) should have LARGER area than $\Sigma^*$ (outer)!

This is the OPPOSITE of what we want!

\section{Evidence AGAINST the inequality}

\subsection{The mean curvature argument (above)}

If all intermediate surfaces have $H < 0$, then any foliation from $\Sigma_0$ to $\Sigma^*$ has decreasing area. Thus $A(\Sigma_0) > A(\Sigma^*$.

\textbf{This suggests the inequality is FALSE in general!}

\subsection{Potential counterexample construction}

Consider initial data with:
\begin{itemize}
    \item A small trapped surface $\Sigma_0$ deep inside with large area (highly curved region)
    \item An outermost MOTS $\Sigma^*$ with smaller area
\end{itemize}

This violates no known constraints.

\section{Resolution: The foliation does NOT exist in general}

\textbf{Key insight:} The argument assumes a smooth foliation from $\Sigma_0$ to $\Sigma^*$.

But such a foliation may NOT EXIST! The trapped region may have complicated topology.

The Andersson-Metzger theorem guarantees $\partial \mathcal{T} = \Sigma^*$ exists and is smooth.

But it does NOT guarantee that $\Sigma_0$ can be connected to $\Sigma^*$ by a foliation.

\section{The Hawking Area Theorem (spacetime version)}

In SPACETIME, the Hawking area theorem says:

Along a null hypersurface generated by geodesics with $\theta \le 0$, the cross-sectional area is non-increasing in the future direction.

This is a SPACETIME result, not an initial data result.

\textbf{Key question:} Does the initial data $(M, g, k)$ embed in a spacetime where $\Sigma_0$ and $\Sigma^*$ lie on a common null hypersurface?

If yes, then $A(\Sigma^*) \ge A(\Sigma_0)$ by Hawking.

\section{New Theorem: Null Foliation Area Monotonicity}

\begin{theorem}[Conditional Area Monotonicity]
Let $(M, g, k)$ satisfy DEC. Suppose $\Sigma_0$ and $\Sigma^*$ can be connected by a null foliation $\{\Sigma_t\}$ in any spacetime development of $(M, g, k)$, with $\Sigma_t$ being a cross-section of an outgoing null hypersurface. Then:
\[
A(\Sigma^*) \ge A(\Sigma_0).
\]
\end{theorem}

\begin{proof}
By the Raychaudhuri equation and NEC (implied by DEC):
\[
\frac{d\theta^+}{ds} \le -\frac{1}{2}(\theta^+)^2 \le 0.
\]
Since $\theta^+(\Sigma_0) \le 0$, we have $\theta^+ \le 0$ along the entire null foliation.

The area evolution is:
\[
\frac{dA}{ds} = \int_{\Sigma_s} \theta^+ \, dA \le 0.
\]
Since we're moving from $\Sigma_0$ (inner) to $\Sigma^*$ (outer) along outgoing null directions, and $\theta^+$ goes from $\theta^+(\Sigma_0) \le 0$ to $\theta^+(\Sigma^*) = 0$, the area is non-decreasing:

Wait, which direction is $s$ increasing? If $s$ increases outward, then $dA/ds \le 0$ means area DECREASES outward.

Let me reconsider...
\end{proof}

\textbf{Confusion:} The sign of $dA/ds$ depends on the direction of the null flow.

Along OUTGOING null rays ($\ell^+$): $\theta^+ < 0$ means rays are CONVERGING, so area DECREASES.

But we want to go from $\Sigma_0$ (inside) to $\Sigma^*$ (outside), which is the OUTWARD direction.

If area decreases outward, then $A(\Sigma_0) > A(\Sigma^*)$.

\section{FINAL ANALYSIS}

The null expansion $\theta^+ = H + \tr_\Sigma k$ measures the rate of change of area along OUTGOING null rays.

$\theta^+ < 0$ means outgoing light rays are CONVERGING - the area of wavefronts DECREASES as they propagate outward.

This is the definition of being "trapped" - even outgoing light is falling inward.

\textbf{Conclusion:} In a trapped region, area DECREASES as you move outward along null rays.

Therefore: $A(\Sigma_0) \ge A(\Sigma^*)$ - the OPPOSITE of what we wanted!

\section{What this means for the Penrose Inequality}

If $A(\Sigma_0) \ge A(\Sigma^*)$, then:
\[
M_{\mathrm{ADM}} \ge \sqrt{\frac{A(\Sigma^*)}{16\pi}} \le \sqrt{\frac{A(\Sigma_0)}{16\pi}}
\]

The inequality for $\Sigma^*$ does NOT imply the inequality for $\Sigma_0$!

\textbf{The favorable jump condition is ESSENTIAL.}

The paper's claim to prove an "unconditional" Penrose inequality is FALSE without new mathematics.

\section{The Correct Statement of the Theorem}

\begin{theorem}[Spacetime Penrose Inequality - Correct Statement]
Let $(M, g, k)$ be asymptotically flat with DEC. Let $\Sigma_0$ be a closed trapped surface.

\textbf{Case 1:} If $\tr_{\Sigma_0} k \ge 0$ (favorable jump), then:
\[
M_{\mathrm{ADM}} \ge \sqrt{\frac{A(\Sigma_0)}{16\pi}}.
\]

\textbf{Case 2:} If $\Sigma_0$ is a stable MOTS, then the favorable jump is automatic and the inequality holds.

\textbf{Case 3:} For general trapped surfaces with $\tr_{\Sigma_0} k < 0$, the inequality:
\[
M_{\mathrm{ADM}} \ge \sqrt{\frac{A(\Sigma_0)}{16\pi}}
\]
remains \textbf{OPEN}.
\end{theorem}

\end{document}
