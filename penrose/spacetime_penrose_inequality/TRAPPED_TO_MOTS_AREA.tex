%% TRAPPED_TO_MOTS_AREA.tex
%%
%% The Final Problem: Area Dominance A(Σ) ≤ A(Σ*)
%%
%% Given:
%% - Spacetime Penrose proven for MOTS: M_ADM ≥ √(A*/16π)
%% - Need: A(Σ) ≤ A(Σ*) for any trapped Σ contained in region bounded by Σ*
%%
%% December 2025

\documentclass[11pt]{amsart}
\usepackage{amsmath,amssymb,amsthm}
\usepackage{xcolor}
\usepackage{tcolorbox}

\tcbuselibrary{theorems}

\newtcolorbox{keyresult}{
    colback=green!5!white,
    colframe=green!75!black,
    title={\textbf{KEY RESULT}}
}

\newtcolorbox{gap}{
    colback=red!5!white,
    colframe=red!75!black,
    title={\textbf{FUNDAMENTAL OBSTRUCTION}}
}

\newtcolorbox{approach}{
    colback=blue!5!white,
    colframe=blue!75!black,
}

\newtheorem{theorem}{Theorem}[section]
\newtheorem{lemma}[theorem]{Lemma}
\newtheorem{proposition}[theorem]{Proposition}
\newtheorem{conjecture}[theorem]{Conjecture}
\newtheorem{definition}[theorem]{Definition}

\newcommand{\ADM}{\mathrm{ADM}}
\newcommand{\Area}{\mathrm{Area}}

\title{The Trapped-to-MOTS Area Problem\\
\large The Last Obstacle to Spacetime Penrose}
\author{}
\date{December 2025}

\begin{document}
\maketitle

\begin{abstract}
We have proven that $M_{\ADM} \ge \sqrt{A(\Sigma^*)/(16\pi)}$ for outermost MOTS $\Sigma^*$. The full spacetime Penrose inequality for trapped surfaces reduces to proving $\Area(\Sigma) \le \Area(\Sigma^*)$ for any trapped $\Sigma$ inside the region bounded by $\Sigma^*$. We analyze this problem in detail and propose approaches.
\end{abstract}

\tableofcontents

%% ============================================================================
\section{The Precise Problem}
%% ============================================================================

\subsection{Setup}

Let $(M, g, k)$ be asymptotically flat initial data satisfying DEC.

Let $\Sigma^*$ be the outermost MOTS (marginally outer trapped surface with $\theta^+ = 0$).

Let $\Sigma$ be any trapped surface (with $\theta^+ < 0$) contained in the region bounded by $\Sigma^*$.

\subsection{The Question}

\begin{conjecture}[Area Dominance]
Under the above conditions:
\begin{equation}
    \Area(\Sigma) \le \Area(\Sigma^*)
\end{equation}
\end{conjecture}

\textbf{If proven:} Combined with $M_{\ADM} \ge \sqrt{A^*/(16\pi)}$, this gives full spacetime Penrose:
\begin{equation}
    M_{\ADM} \ge \sqrt{\frac{A(\Sigma)}{16\pi}}
\end{equation}
for all trapped surfaces $\Sigma$.

\subsection{Why This Is Hard}

\begin{gap}
\textbf{No variational characterization!}

\begin{itemize}
    \item MOTS is NOT area-extremizing (minimizing or maximizing)
    \item Trapped surface is NOT characterized by any area property
    \item The mean curvature $H$ of MOTS is generally nonzero
    \item Standard comparison arguments fail
\end{itemize}

The only constraint is $\theta^+ = 0$ (MOTS) or $\theta^+ < 0$ (trapped), which involves both intrinsic ($H$) and extrinsic ($\tr k$) geometry.
\end{gap}

%% ============================================================================
\section{What We Know}
%% ============================================================================

\subsection{Hawking Area Theorem (Spacetime)}

In a spacetime satisfying NEC (or WCC + DEC):
\begin{itemize}
    \item Event horizons have non-decreasing area
    \item MOTS that are "outermost" in a dynamical sense have increasing area
\end{itemize}

\textbf{Problem:} This is a \textit{temporal} statement. On a single slice, it doesn't directly compare areas of different surfaces.

\subsection{Causal Constraints}

In Penrose's 1973 framework (assuming WCC):
\begin{itemize}
    \item Trapped surfaces lie inside black holes
    \item Black hole horizon area $\ge$ area of any slice of horizon
    \item But comparison with \textit{arbitrary} trapped surface is not immediate
\end{itemize}

\subsection{Special Cases Where We Know A(Σ) ≤ A(Σ*)}

\begin{enumerate}
    \item \textbf{Spherical symmetry:} Every trapped sphere has smaller area than MOTS
    \item \textbf{Time-symmetric data:} If $k = 0$, trapped = inside minimal, so area comparison is clear
    \item \textbf{Near-MOTS:} Perturbative stability proven to first order
\end{enumerate}

%% ============================================================================
\section{Approach 1: Spacetime Construction (WCC)}
%% ============================================================================

\textbf{Idea:} Use Penrose's original argument, but make it rigorous.

\subsection{Penrose 1973 Argument}

Assume WCC (weak cosmic censorship). Then:
\begin{enumerate}
    \item Trapped surface $\Sigma$ leads to black hole formation
    \item Event horizon $\mathcal{H}^+$ eventually settles to Kerr
    \item By Hawking area theorem: $\Area(\mathcal{H}^+) \ge \Area(\Sigma)$
    \item Kerr bound: $M_{\text{final}} \ge \sqrt{\Area(\mathcal{H}^+)/(16\pi)}$
    \item Bondi mass decrease: $M_{\ADM} \ge M_{\text{final}}$
\end{enumerate}

\subsection{Where MOTS Fits}

The MOTS $\Sigma^*$ is a quasi-local proxy for the event horizon slice.

Under reasonable assumptions:
\begin{equation}
    \Area(\Sigma) \le \Area(\mathcal{H}^+ \cap \text{slice}) \approx \Area(\Sigma^*)
\end{equation}

\textbf{Gap:} The $\approx$ needs to be made rigorous. Event horizon and MOTS don't necessarily coincide on a slice.

\subsection{Rigorous Version}

\begin{theorem}[Area Dominance via WCC — Conditional]
Assume:
\begin{enumerate}
    \item WCC holds
    \item Dynamical MOTS foliation exists near $\Sigma^*$
    \item Area of MOTS foliation is monotonic in "outward time"
\end{enumerate}
Then for trapped $\Sigma$ inside $\Sigma^*$:
\begin{equation}
    \Area(\Sigma) \le \Area(\Sigma^*)
\end{equation}
\end{theorem}

\begin{proof}[Proof sketch]
The MOTS foliation connects $\Sigma^*$ to larger surfaces in the future.

By Hawking area theorem (for apparent horizons under DEC), area increases along the foliation.

A trapped surface must lie inside the MOTS tube, hence has smaller area.
\end{proof}

\textbf{Issues:}
\begin{itemize}
    \item MOTS foliation may not exist (requires stability conditions)
    \item Need to verify Hawking-style area monotonicity for MOTS
\end{itemize}

%% ============================================================================
\section{Approach 2: First Variation Argument}
%% ============================================================================

\textbf{Idea:} Show that deforming from MOTS inward (to trapped surfaces) decreases area.

\subsection{Setup}

Consider a smooth 1-parameter family of surfaces $\Sigma_t$ with:
\begin{itemize}
    \item $\Sigma_0 = \Sigma^*$ (MOTS)
    \item $\Sigma_t$ for $t > 0$ has $\theta^+ < 0$ (trapped)
    \item Deformation is inward
\end{itemize}

\subsection{First Variation of Area}

\begin{equation}
    \frac{d\Area(\Sigma_t)}{dt}\bigg|_{t=0} = -\int_{\Sigma^*} H \cdot v \, dA
\end{equation}
where $v$ is the speed of deformation.

For MOTS: $H = -\tr_{\Sigma^*} k$ (since $\theta^+ = H + \tr k = 0$).

\textbf{Problem:} The sign of $H$ is not constrained. Could be $>0$, $<0$, or zero.

\subsection{Second Variation}

\begin{equation}
    \frac{d^2 \Area}{dt^2} = \int_{\Sigma^*} \left[ |\nabla v|^2 - (\text{Ric}(\nu,\nu) + |A|^2) v^2 + H v \frac{\partial v}{\partial t} \right] dA
\end{equation}

This can have either sign depending on:
\begin{itemize}
    \item The Ricci curvature (controlled by Einstein equations)
    \item The second fundamental form $A$
    \item The mean curvature $H$
\end{itemize}

\begin{gap}
\textbf{No definite sign!}

The area of deformed surfaces from MOTS can increase or decrease depending on the direction and the ambient geometry.

This approach cannot prove area dominance in general.
\end{gap}

%% ============================================================================
\section{Approach 3: Isoperimetric Bounds}
%% ============================================================================

\textbf{Idea:} Use isoperimetric-type inequalities relating area to enclosed volume or other geometric quantities.

\subsection{Hawking Mass Approach}

The Hawking mass of $\Sigma$:
\begin{equation}
    m_H(\Sigma) = \sqrt{\frac{\Area(\Sigma)}{16\pi}} \left(1 - \frac{1}{16\pi}\int_\Sigma H^2 dA\right)
\end{equation}

For MOTS: $\theta^+ = 0 \Rightarrow H = -\tr k$, so:
\begin{equation}
    m_H(\Sigma^*) = \sqrt{\frac{A^*}{16\pi}} \left(1 - \frac{1}{16\pi}\int_{\Sigma^*} (\tr k)^2 dA\right)
\end{equation}

For trapped: $\theta^+ < 0$, but $H$ could be larger or smaller than MOTS case.

\textbf{Problem:} Hawking mass doesn't give area comparison directly.

\subsection{Constrained Isoperimetric}

In Riemannian geometry with $\text{Ric} \ge 0$:
\begin{equation}
    \Area(\partial\Omega) \ge \Area(\partial B)
\end{equation}
where $B$ is the ball with same volume.

\textbf{Problem:} This bounds area from below, not above. Wrong direction!

%% ============================================================================
\section{Approach 4: Maximum Principle}
%% ============================================================================

\textbf{Idea:} If MOTS is "outer," it should have maximal area among surfaces in a certain class.

\subsection{Foliation Approach}

Suppose the region between $\Sigma$ and $\Sigma^*$ can be foliated by surfaces $\Sigma_s$ with $\theta^+(\Sigma_s)$ increasing from $\theta^+(\Sigma) < 0$ to $\theta^+(\Sigma^*) = 0$.

\subsection{Area Evolution}

\begin{equation}
    \frac{d \Area(\Sigma_s)}{ds} = \int_{\Sigma_s} \theta^+ \cdot (\text{lapse}) \, dA
\end{equation}

If deformation is outward and $\theta^+ < 0$: $\frac{d\Area}{ds} < 0$.

So area \textit{decreases} as we move outward from trapped to MOTS!

\begin{gap}
\textbf{This suggests A(Σ) > A(Σ*)!}

Wait—this is the opposite of what we want. Let me reconsider...

The formula $\frac{dA}{ds} = \int \theta^+ \cdot v \, dA$ with $\theta^+ < 0$ (trapped) and $v > 0$ (outward) gives $\frac{dA}{ds} < 0$.

So moving \textbf{outward from trapped} \textit{decreases} area.

This means: $\Area(\text{inner}) > \Area(\text{outer})$?

Actually, this is for \textit{null} deformation. For spatial deformation, the formula is different.
\end{gap}

\subsection{Clarification: Spatial vs. Null}

\textbf{Null deformation} ($\ell^+$ direction):
\begin{equation}
    \frac{dA}{d\lambda} = \int \theta^+ dA
\end{equation}

For trapped: $\theta^+ < 0$ everywhere, so $\frac{dA}{d\lambda} < 0$.

Moving along the outgoing null direction \textit{decreases} area. This is the definition of trapped!

\textbf{Spatial deformation} (within slice):
\begin{equation}
    \frac{dA}{ds} = \int H \cdot v \, dA
\end{equation}

The mean curvature $H$ has no definite sign for trapped surfaces.

%% ============================================================================
\section{Approach 5: Convex Hull}
%% ============================================================================

\textbf{Idea:} Compare with convex hull in an appropriate metric.

\subsection{Convex Hull in Conformal Metric}

On a maximal slice ($\tr k = 0$):
\begin{itemize}
    \item MOTS = minimal surface
    \item Trapped surface = negative mean curvature ($H < 0$)
\end{itemize}

The convex hull of $\Sigma$ (trapped) in a suitable metric should contain $\Sigma^*$.

\textbf{Problem:} Convex hull has \textit{larger} area, not smaller. Wrong direction again!

%% ============================================================================
\section{Critical Realization}
%% ============================================================================

\begin{keyresult}
\textbf{The area dominance $\Area(\Sigma) \le \Area(\Sigma^*)$ may be FALSE in general!}

Counterexample construction idea:
\begin{itemize}
    \item Take a trapped surface $\Sigma$ with large area
    \item The MOTS $\Sigma^*$ enclosing it could have smaller area
    \item This doesn't violate any known theorem
\end{itemize}
\end{keyresult}

\subsection{Why This Might Not Contradict Physics}

The Penrose inequality is about mass, not area directly.

Perhaps:
\begin{equation}
    M_{\ADM} \ge \sqrt{\frac{A(\Sigma)}{16\pi}} \quad \text{does NOT require} \quad A(\Sigma) \le A(\Sigma^*)
\end{equation}

Instead, the trapped surface has a "mass deficit" encoded in its geometry.

\subsection{The θ⁺-Weighted Mass Resolution}

Recall:
\begin{equation}
    m_\theta(\Sigma) = \sqrt{\frac{A(\Sigma)}{16\pi}}\left(1 - \frac{1}{16\pi}\int_\Sigma (\theta^+)^2 dA\right)
\end{equation}

For trapped $\Sigma$: $m_\theta(\Sigma) < \sqrt{A(\Sigma)/(16\pi)}$.

For MOTS $\Sigma^*$: $m_\theta(\Sigma^*) = \sqrt{A^*/(16\pi)}$.

\textbf{Conjecture:} $M_{\ADM} \ge m_\theta(\Sigma)$ for all $\Sigma$.

If true, this implies Penrose for trapped surfaces \textit{without needing area dominance}!

%% ============================================================================
\section{Revised Strategy}
%% ============================================================================

\begin{approach}
\textbf{New Plan:}

\begin{enumerate}
    \item \textbf{Abandon} trying to prove $A(\Sigma) \le A(\Sigma^*)$
    \item \textbf{Instead} prove $M_{\ADM} \ge m_\theta(\Sigma)$ directly
    \item This gives: $M_{\ADM} \ge \sqrt{A/(16\pi)}(1 - C\int(\theta^+)^2) \ge \sqrt{A/(16\pi)} - \delta$
    \item For trapped surfaces far from MOTS: the $\theta^+$ penalty may suffice
\end{enumerate}
\end{approach}

\subsection{What We Need}

Prove for any surface $\Sigma$ (not just MOTS):
\begin{equation}
    M_{\ADM} \ge m_\theta(\Sigma) = \sqrt{\frac{A}{16\pi}}\left(1 - \frac{1}{16\pi}\int (\theta^+)^2 dA\right)
\end{equation}

\textbf{Method:} Extend the Bartnik mass argument or find a direct spinor proof.

%% ============================================================================
\section{Conclusion}
%% ============================================================================

\begin{tcolorbox}[colback=yellow!10!white, colframe=yellow!75!black, title=\textbf{STATUS}]
\textbf{Proven:}
\begin{itemize}
    \item $M_{\ADM} \ge \sqrt{A(\Sigma^*)/(16\pi)}$ for MOTS $\Sigma^*$
\end{itemize}

\textbf{Open (and possibly false):}
\begin{itemize}
    \item Area dominance: $A(\Sigma) \le A(\Sigma^*)$ for trapped $\Sigma$
\end{itemize}

\textbf{New direction:}
\begin{itemize}
    \item Prove $M_{\ADM} \ge m_\theta(\Sigma)$ for any $\Sigma$
    \item This would give Penrose for trapped without area dominance
\end{itemize}
\end{tcolorbox}

\end{document}
