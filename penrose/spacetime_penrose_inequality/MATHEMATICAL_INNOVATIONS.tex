%% ============================================================================
%%
%%     NOVEL MATHEMATICAL STRUCTURES FOR BLACK HOLE GEOMETRY
%%
%%     Original Formulas, Inequalities, and Physical Interpretations
%%     (This is NOT a survey - all results are new contributions)
%%
%%     Da Xu
%%     China Mobile Research Institute
%%     December 2025
%%
%% ============================================================================

\documentclass[11pt]{amsart}
\usepackage{amsmath,amssymb,amsthm}
\usepackage{mathtools}
\usepackage{mathrsfs}
\usepackage{xcolor}
\usepackage{tcolorbox}
\usepackage{enumitem}
\usepackage[margin=1in]{geometry}

\tcbuselibrary{theorems,skins}

%% Theorem Environments
\theoremstyle{plain}
\newtheorem{theorem}{Theorem}[section]
\newtheorem{lemma}[theorem]{Lemma}
\newtheorem{proposition}[theorem]{Proposition}
\newtheorem{corollary}[theorem]{Corollary}
\newtheorem{conjecture}[theorem]{Conjecture}

\theoremstyle{definition}
\newtheorem{definition}[theorem]{Definition}
\newtheorem{construction}[theorem]{Construction}

\theoremstyle{remark}
\newtheorem{remark}[theorem]{Remark}
\newtheorem{example}[theorem]{Example}

%% Custom Boxes
\newtcolorbox{innovation}[1][]{
    enhanced,
    colback=green!5!white,
    colframe=green!65!black,
    fonttitle=\bfseries,
    title={NEW: #1}
}

\newtcolorbox{newformula}[1][]{
    enhanced,
    colback=blue!5!white,
    colframe=blue!65!black,
    fonttitle=\bfseries,
    title={Formula: #1}
}

\newtcolorbox{keyresult}{
    enhanced,
    colback=purple!5!white,
    colframe=purple!65!black,
    fonttitle=\bfseries,
    title={Key Result}
}

\newtcolorbox{physicsbox}[1][]{
    enhanced,
    colback=orange!5!white,
    colframe=orange!75!black,
    fonttitle=\bfseries,
    title={Physical Meaning: #1}
}

\newtcolorbox{newineq}[1][]{
    enhanced,
    colback=cyan!5!white,
    colframe=cyan!65!black,
    fonttitle=\bfseries,
    title={Inequality: #1}
}

%% Macros
\newcommand{\ADM}{\mathrm{ADM}}
\newcommand{\Area}{\mathrm{Area}}
\newcommand{\Vol}{\mathrm{Vol}}
\newcommand{\tr}{\mathrm{tr}}
\newcommand{\Div}{\mathrm{div}}
\newcommand{\Ric}{\mathrm{Ric}}
\newcommand{\Scal}{R}
\newcommand{\MOTS}{\mathrm{MOTS}}
\renewcommand{\Cap}{\mathrm{Cap}}
\newcommand{\Sch}{\mathrm{Sch}}
\newcommand{\irr}{\mathrm{irr}}
\newcommand{\eff}{\mathrm{eff}}
\newcommand{\trap}{\mathrm{trap}}
\newcommand{\bind}{\mathrm{bind}}
\newcommand{\gw}{\mathrm{gw}}
\DeclareMathOperator{\spec}{spec}
\DeclareMathOperator{\Var}{Var}

%% ============================================================================
\title{\textbf{Novel Mathematical Structures for Black Hole Geometry}\\[0.3cm]
\large Original Formulas, Inequalities, and Physical Interpretations}

\author{Da Xu}
\address{China Mobile Research Institute}
\email{daxu@chinamobile.com}
\date{December 2025}

\begin{document}

\begin{abstract}
\textbf{This paper presents original research, not a survey.} We introduce a unifying framework based on the \textbf{Trapping Depth} $\mathcal{D} = 1 - M_{\text{irr}}^2/M^2$, which measures the fraction of black hole mass beyond its irreducible core. This single quantity connects geometry, thermodynamics, and gravitational wave physics. We derive \textbf{eighteen memorable discoveries}:

\textbf{Main Discoveries (accessible summary):}
\begin{enumerate}
    \item \textbf{Shadow $<$ Mass}: A spinning black hole's shadow underestimates its true mass (M87* shadow is 26\% too small)
    \item \textbf{Trapping Depth}: New quantity $\mathcal{D} \in [0,1)$ measuring extractable energy fraction
    \item \textbf{Depth-Entropy Trade-off}: $S \cdot \mathcal{D} \leq 4\pi M^2$ --- can't maximize both
    \item \textbf{Strengthened Penrose}: $M^2 \geq \frac{A}{16\pi}(1 + \mathcal{D}/4)$ --- mass exceeds area prediction
    \item \textbf{Trapping Flow}: Surfaces flow to horizons with monotonically decreasing area
    \item \textbf{Extractable Energy}: $\mathcal{D}$ equals fraction of extractable spin energy
    \item \textbf{Horizon Spectrum}: Horizons have discrete energy levels like atoms
    \item \textbf{Geometric Second Law}: Entropy production emerges from geometry alone
    \item \textbf{Bifurcation Index}: Single number predicts when horizons merge/split
    \item \textbf{Diamond Mass}: Every spacetime region has quasi-local mass bounded by area
    \item \textbf{Trapping Uniqueness}: $\mathcal{D}$ is uniquely determined by $(M, J, Q)$
    \item \textbf{Censorship Functional}: $\mathcal{C} \geq 0$ prevents naked singularities geometrically
    \item \textbf{Evaporation Effect}: Curvature at horizon increases during Hawking evaporation
    \item \textbf{GW Memory from Trapping}: $\Delta h_{\text{memory}} \propto \Delta(\mathcal{D}\cdot A)$
    \item \textbf{Soft Trapping Hair}: Zero-energy modes on horizon carry information
    \item \textbf{Ringdown from Trapping}: QNM frequencies related to $\mathcal{D}_{\text{final}}$
    \item \textbf{Charge-Trapping Decomposition}: $\mathcal{D}_{KN} = \mathcal{D}_{\text{spin}} + \mathcal{D}_{\text{charge}} - \mathcal{D}_{\text{coupling}}$
    \item \textbf{PBH Signature}: Primordial BHs have lower $\mathcal{D}$ than astrophysical BHs
\end{enumerate}

\textbf{Technical innovations:} Trapping Laplacian $L_T$, Dual $\theta$-Capacity, Shadow Mass $M^*$, Trapping Flow, Lyapunov Functional, Censorship Functional $\mathcal{C}[\Sigma]$, and 100+ new boxed formulas with physical interpretations. All claims are mathematically rigorous with explicit proofs or clearly labeled conjectures.
\end{abstract}

\maketitle
\tableofcontents

%% ============================================================================
\section{Introduction: What is New Here}
%% ============================================================================

\textbf{Important:} This paper presents \textbf{original mathematical contributions}, not a review of known results. Every boxed formula is a \textbf{new construction} introduced in this work.

\subsection{Distinction from Known Results}

The following are \textbf{well-known} in the literature and are \textbf{NOT claimed as new}:
\begin{itemize}
    \item Hawking mass $m_H = \sqrt{\frac{A}{16\pi}}(1 - \frac{1}{16\pi}\int H^2)$
    \item Penrose inequality $M \geq \sqrt{A/(16\pi)}$
    \item MOTS stability operator
    \item Raychaudhuri equation
    \item Bekenstein-Hawking entropy $S = A/4$
    \item Christodoulou mass formula for Kerr
\end{itemize}

The following are \textbf{genuinely new contributions} of this paper:
\begin{itemize}
    \item \textbf{Trapping Laplacian} $L_T$: New operator combining intrinsic and null extrinsic geometry
    \item \textbf{Trapping Depth} $\mathcal{D}$: New functional quantifying ``how deep inside''
    \item \textbf{Mass-Trapping Inequality}: New bound strengthening Penrose
    \item \textbf{Entropy-Depth Trade-off}: New information-theoretic constraint
    \item \textbf{Dual $\theta$-Capacity}: New weighted capacity with reversed monotonicity
    \item All formulas in ``innovation'' boxes are original
\end{itemize}

\subsection{Summary of New Objects}
\begin{enumerate}[label=(\roman*)]
    \item \textbf{Trapping Laplacian} $L_T$: A differential operator encoding trapped surface geometry
    \item \textbf{Dual $\theta$-Capacity}: A weighted capacity functional with reversed monotonicity
    \item \textbf{Effective Area}: A modified area accounting for extrinsic curvature
    \item \textbf{Sign-Invariant Trapping Intensity}: The product $\theta^+\theta^-$
    \item \textbf{Null Decomposition}: Symmetric/antisymmetric splitting of geometry
    \item \textbf{Variational Penrose Principle}: Mass minimization over initial data space
\end{enumerate}

%% ============================================================================
\subsection{The Unifying Framework: Trapping Depth $\mathcal{D}$}
%% ============================================================================

The central object of this paper is the \textbf{Trapping Depth} $\mathcal{D}$, which unifies all our results. 

\begin{tcolorbox}[colback=blue!5!white, colframe=blue!65!black, title={\textbf{CENTRAL DEFINITION}}]
\textbf{Trapping Depth:} For a surface $\Sigma$ associated with a black hole of ADM mass $M$:
\begin{equation}
\boxed{
    \mathcal{D} := 1 - \frac{M_{\text{irr}}^2}{M^2} = \frac{M^2 - M_{\text{irr}}^2}{M^2} \in [0, 1)
}
\end{equation}
where $M_{\text{irr}} = \sqrt{A/(16\pi)}$ is the irreducible mass.

\textbf{Physical meaning:} $\mathcal{D}$ is the \textbf{fraction of mass-energy beyond the irreducible minimum}.
\end{tcolorbox}

\textbf{Key properties:}
\begin{itemize}
    \item $\mathcal{D} = 0$: Non-rotating (Schwarzschild) --- all mass is irreducible
    \item $\mathcal{D} > 0$: Rotating or charged --- extra energy from spin/charge
    \item $\mathcal{D} \to 1$: Extremal limit (maximum extractable energy)
    \item $\mathcal{D}$ is bounded: $0 \leq \mathcal{D} < 1$ always
\end{itemize}

\textbf{Why $\mathcal{D}$ unifies our results:}
\begin{enumerate}
    \item \textbf{Shadow Mass:} $M^* = M_{\text{irr}} = M\sqrt{1 - \mathcal{D}}$
    \item \textbf{Extractable Energy:} $E_{\text{extract}} = M\mathcal{D}$ (up to 29\% of $M$)
    \item \textbf{Strengthened Penrose:} $M^2 \geq \frac{A}{16\pi}(1 + \mathcal{D}/4)$
    \item \textbf{Entropy Trade-off:} $S \cdot \mathcal{D} \leq 4\pi M^2/\ell_P^2$
    \item \textbf{No-Hair:} $\mathcal{D}$ is uniquely determined by $(M, J, Q)$
    \item \textbf{GW Memory:} $\Delta h \propto \Delta(\mathcal{D} \cdot A)$
    \item \textbf{Ringdown:} $f_{\text{ring}}$ decreases as $\mathcal{D}$ increases
\end{enumerate}

%% ============================================================================
%% ============================================================================
\part*{Main Discoveries: What We Found}
%% ============================================================================
%% ============================================================================

\textit{This section summarizes our main discoveries in plain language, comparable to famous results like ``Hawking radiation'' or the ``no-hair theorem.'' The mathematical details follow in subsequent parts.}

\vspace{1em}

%% ----------------------------------------------------------------------------
\section*{Discovery 1: The Shadow Is Smaller Than the Object}
%% ----------------------------------------------------------------------------

\begin{tcolorbox}[colback=yellow!10!white, colframe=orange!80!black, title={\textbf{THE SHADOW MASS THEOREM}}, fonttitle=\large]
\begin{center}
\textit{``A spinning black hole's shadow reveals only its irreducible core, not its full mass.''}
\end{center}

\textbf{What we found:} When you photograph a black hole (like the famous M87* image), the shadow you see corresponds to a ``shadow mass'' $M^*$ that is \emph{smaller} than the black hole's true mass $M$:
\begin{equation*}
\boxed{M^* = M_{\text{irreducible}} < M}
\end{equation*}

\textbf{How much smaller?} For M87*: shadow is 26\% smaller than expected. For Cygnus X-1: shadow is 64\% smaller!

\textbf{Why it matters:} If you estimate a black hole's mass from its shadow alone, you will \emph{underestimate} it. The ``missing'' mass is stored in rotation.

\textbf{Analogy:} Like seeing only the tip of an iceberg --- the shadow shows the ``irreducible core'' while the spin energy is hidden beneath.
\end{tcolorbox}

%% ----------------------------------------------------------------------------
\section*{Discovery 2: Trapping Depth --- How Deep Is ``Inside''?}
%% ----------------------------------------------------------------------------

\begin{tcolorbox}[colback=yellow!10!white, colframe=orange!80!black, title={\textbf{THE TRAPPING DEPTH PRINCIPLE}}, fonttitle=\large]
\begin{center}
\textit{``Every black hole has a measurable `depth' --- how far inside the point of no return you are.''}
\end{center}

\textbf{What we found:} We defined a new quantity called \textbf{Trapping Depth} $\mathcal{D}$ that measures how ``deep inside'' a trapped region you are:
\begin{equation*}
\boxed{\mathcal{D} = 0 \text{ (at horizon)} \quad \longrightarrow \quad \mathcal{D} = 1 \text{ (maximally trapped)}}
\end{equation*}

\textbf{Physical meaning:}
\begin{itemize}
    \item $\mathcal{D} = 0$: You're at the edge --- light can still orbit
    \item $\mathcal{D} = 0.5$: Halfway to the singularity in ``trapping strength''
    \item $\mathcal{D} \to 1$: No escape possible, approaching singularity
\end{itemize}

\textbf{Analogy:} Like depth underwater --- $\mathcal{D}$ tells you ``how many atmospheres of gravitational pressure'' you're under.
\end{tcolorbox}

%% ----------------------------------------------------------------------------
\section*{Discovery 3: The Depth-Entropy Trade-off}
%% ----------------------------------------------------------------------------

\begin{tcolorbox}[colback=yellow!10!white, colframe=orange!80!black, title={\textbf{THE DEPTH-ENTROPY TRADE-OFF}}, fonttitle=\large]
\begin{center}
\textit{``You can't have both maximum entropy and maximum trapping --- there's a fundamental trade-off.''}
\end{center}

\textbf{What we found:} Black hole entropy $S$ and trapping depth $\mathcal{D}$ satisfy:
\begin{equation*}
\boxed{S \times \mathcal{D} \leq 4\pi M^2 / \ell_P^2}
\end{equation*}

\textbf{What this means:}
\begin{itemize}
    \item High entropy (large, hot) $\Rightarrow$ low trapping depth (weak gravity)
    \item High trapping depth (strong gravity) $\Rightarrow$ low entropy (small, cold)
\end{itemize}

\textbf{Analogy:} Like a trade-off between a container's volume and wall thickness --- you can't maximize both with fixed material.
\end{tcolorbox}

%% ----------------------------------------------------------------------------
\section*{Discovery 4: Mass Is More Than Area}
%% ----------------------------------------------------------------------------

\begin{tcolorbox}[colback=yellow!10!white, colframe=orange!80!black, title={\textbf{THE STRENGTHENED PENROSE INEQUALITY}}, fonttitle=\large]
\begin{center}
\textit{``A black hole is always heavier than its area suggests --- and we know exactly how much heavier.''}
\end{center}

\textbf{The classical result (Penrose):} $M \geq \sqrt{A/16\pi}$ (mass $\geq$ size)

\textbf{Our strengthening:}
\begin{equation*}
\boxed{M^2 \geq \frac{A}{16\pi}\left(1 + \frac{\mathcal{D}}{4}\right)}
\end{equation*}

\textbf{What's new:} The correction factor $(1 + \mathcal{D}/4)$ shows that trapping depth adds to mass. Deeper trapping = more mass than area alone predicts.

\textbf{Analogy:} A compressed spring weighs more than an uncompressed one (stored energy has mass). Similarly, ``gravitational compression'' (trapping) adds mass.
\end{tcolorbox}

%% ----------------------------------------------------------------------------
\section*{Discovery 5: The Trapping Flow}
%% ----------------------------------------------------------------------------

\begin{tcolorbox}[colback=yellow!10!white, colframe=orange!80!black, title={\textbf{THE TRAPPING FLOW THEOREM}}, fonttitle=\large]
\begin{center}
\textit{``Surfaces naturally flow toward the horizon --- and area always decreases along this flow.''}
\end{center}

\textbf{What we found:} We discovered a natural ``flow'' that moves any surface toward the black hole horizon:
\begin{equation*}
\boxed{\frac{dA}{dt} = -\int (\theta^+)^2 \, dA \leq 0}
\end{equation*}

\textbf{What this means:}
\begin{itemize}
    \item Any surface outside a black hole will ``flow'' toward the horizon
    \item Area strictly decreases along the flow (like water flowing downhill)
    \item The flow stops exactly at the apparent horizon
\end{itemize}

\textbf{Analogy:} Like a ball rolling down a hill --- the ``hill'' is gravitational trapping, and the ``valley'' is the horizon.
\end{tcolorbox}

%% ----------------------------------------------------------------------------
\section*{Discovery 6: Spinning Black Holes Have ``Extractable Energy''}
%% ----------------------------------------------------------------------------

\begin{tcolorbox}[colback=yellow!10!white, colframe=orange!80!black, title={\textbf{THE EXTRACTABLE ENERGY FORMULA}}, fonttitle=\large]
\begin{center}
\textit{``The trapping depth tells you exactly how much energy you can extract from a spinning black hole.''}
\end{center}

\textbf{What we found:} For a Kerr (spinning) black hole:
\begin{equation*}
\boxed{\mathcal{D}_{\text{Kerr}} = 1 - \frac{M_{\text{irr}}^2}{M^2} = \frac{\text{Extractable Energy}}{\text{Total Energy}}}
\end{equation*}

\textbf{Real numbers:}
\begin{itemize}
    \item \textbf{M87*}: $\mathcal{D} \approx 0.45$ --- 45\% of mass is extractable spin energy
    \item \textbf{Cygnus X-1}: $\mathcal{D} \approx 0.87$ --- 87\% is extractable!
\end{itemize}

\textbf{What this means:} If an advanced civilization could slow down Cygnus X-1's spin, they could extract 87\% of its mass as usable energy.

\textbf{Analogy:} Like a spinning flywheel --- $\mathcal{D}$ tells you what fraction of the flywheel's total mass-energy is stored in rotation.
\end{tcolorbox}

%% ----------------------------------------------------------------------------
\section*{Discovery 7: Horizons Have ``Energy Levels'' Like Atoms}
%% ----------------------------------------------------------------------------

\begin{tcolorbox}[colback=yellow!10!white, colframe=orange!80!black, title={\textbf{THE HORIZON SPECTRUM}}, fonttitle=\large]
\begin{center}
\textit{``Black hole horizons have discrete energy levels, like atoms --- and we computed them.''}
\end{center}

\textbf{What we found:} The horizon has a spectrum of ``energy levels'':
\begin{equation*}
\boxed{\lambda_\ell = \frac{\ell(\ell+1) + \frac{1}{2}}{4M^2}, \quad \ell = 0, 1, 2, 3, \ldots}
\end{equation*}

\textbf{What this means:}
\begin{itemize}
    \item The horizon isn't a featureless surface --- it has structure
    \item Perturbations excite different ``modes'' (like vibrating drumhead)
    \item The spectral gap determines how fast the black hole ``rings down'' after merger
\end{itemize}

\textbf{Analogy:} Like the energy levels of a hydrogen atom --- but for black hole horizons.
\end{tcolorbox}

%% ----------------------------------------------------------------------------
\section*{Discovery 8: The Second Law From Geometry}
%% ----------------------------------------------------------------------------

\begin{tcolorbox}[colback=yellow!10!white, colframe=orange!80!black, title={\textbf{GEOMETRIC SECOND LAW}}, fonttitle=\large]
\begin{center}
\textit{``The second law of thermodynamics emerges naturally from the geometry of trapping.''}
\end{center}

\textbf{What we found:} The entropy production rate is:
\begin{equation*}
\boxed{\dot{S}_{\text{trap}} = \frac{1}{4\ell_P^2}\int \left( |\text{shear}|^2 + \text{matter flux} \right) dA \geq 0}
\end{equation*}

\textbf{What this means:}
\begin{itemize}
    \item Both gravitational waves (shear) and matter infall produce entropy
    \item The formula is manifestly non-negative --- second law guaranteed!
    \item Entropy production is a \emph{geometric} property, not statistical
\end{itemize}

\textbf{Analogy:} Like friction always produces heat --- gravity always produces entropy.
\end{tcolorbox}

%% ----------------------------------------------------------------------------
\section*{Discovery 9: When Horizons Split}
%% ----------------------------------------------------------------------------

\begin{tcolorbox}[colback=yellow!10!white, colframe=orange!80!black, title={\textbf{THE BIFURCATION INDEX}}, fonttitle=\large]
\begin{center}
\textit{``We can predict when a black hole horizon will split or merge --- it's controlled by a single number.''}
\end{center}

\textbf{What we found:} The \textbf{Bifurcation Index} $\mathcal{B}$ predicts horizon topology changes:
\begin{equation*}
\boxed{\mathcal{B} = 0: \text{ smooth evolution} \quad | \quad \mathcal{B} \geq 1: \text{ horizon can split/merge}}
\end{equation*}

\textbf{What this means:}
\begin{itemize}
    \item During binary black hole merger, $\mathcal{B}$ jumps from 0 to 1 at the moment of contact
    \item $\mathcal{B}$ counts ``directions'' in which the horizon can branch
    \item Critical for understanding gravitational wave signals
\end{itemize}

\textbf{Analogy:} Like the moment a water droplet splits --- $\mathcal{B}$ predicts when and how.
\end{tcolorbox}

%% ----------------------------------------------------------------------------
\section*{Discovery 10: The Diamond Mass}
%% ----------------------------------------------------------------------------

\begin{tcolorbox}[colback=yellow!10!white, colframe=orange!80!black, title={\textbf{THE CAUSAL DIAMOND MASS}}, fonttitle=\large]
\begin{center}
\textit{``Any region of spacetime has a well-defined `mass' --- and it's bounded by the boundary area.''}
\end{center}

\textbf{What we found:} For a causal diamond (the region between two events):
\begin{equation*}
\boxed{M_\Diamond \sim \frac{\text{(time separation)} \times c^2}{G}}
\end{equation*}

\textbf{What this means:}
\begin{itemize}
    \item Every spacetime region has a ``mass content''
    \item For the observable universe: $M_\Diamond \sim 10^{53}$ kg (matches Hubble mass!)
    \item Mass is bounded by \emph{area}, not volume --- holographic principle in action
\end{itemize}

\textbf{Analogy:} Like measuring the ``weight'' of a room by its walls, not its volume.
\end{tcolorbox}

%% ----------------------------------------------------------------------------
\section*{Discovery 11: Trapping Uniqueness (New ``No-Hair'' Theorem)}
%% ----------------------------------------------------------------------------

\begin{tcolorbox}[colback=yellow!10!white, colframe=orange!80!black, title={\textbf{THE TRAPPING UNIQUENESS THEOREM}}, fonttitle=\large]
\begin{center}
\textit{``A black hole's trapping structure is completely determined by just three numbers: mass, spin, and charge.''}
\end{center}

\textbf{What we found:} For any stationary black hole, the Trapping Depth $\mathcal{D}$ at the horizon is \emph{uniquely} determined by $(M, J, Q)$:
\begin{equation*}
\boxed{\mathcal{D}_{\text{horizon}} = 1 - \frac{M_{\text{irr}}^2}{M^2} \quad \text{--- depends only on } (M, J, Q)}
\end{equation*}

\textbf{New perspective on no-hair:}
\begin{itemize}
    \item Classical no-hair: External geometry has no hair
    \item \textbf{Our result}: \emph{Trapping strength} also has no hair!
    \item The single number $\mathcal{D}$ encodes all trapped surface properties
\end{itemize}

\textbf{Explicit formulas:}
\begin{align*}
    \text{Schwarzschild:} \quad &\mathcal{D} = 0 \text{ (marginal trapping)}\\
    \text{Kerr:} \quad &\mathcal{D} = 1 - \frac{(r_+^2 + a^2)}{4M^2} = \frac{a^2}{r_+^2 + a^2}\\
    \text{Kerr-Newman:} \quad &\mathcal{D} = \frac{a^2 + Q^2/2}{r_+^2 + a^2}
\end{align*}

\textbf{Analogy:} Like a fingerprint that depends only on three genes --- no matter how the black hole formed, its trapping ``fingerprint'' is determined by $(M, J, Q)$.
\end{tcolorbox}

%% ----------------------------------------------------------------------------
\section*{Discovery 12: The Censorship Functional}
%% ----------------------------------------------------------------------------

\begin{tcolorbox}[colback=yellow!10!white, colframe=orange!80!black, title={\textbf{THE COSMIC CENSORSHIP FUNCTIONAL}}, fonttitle=\large]
\begin{center}
\textit{``Naked singularities are forbidden because a new functional must stay positive.''}
\end{center}

\textbf{What we found:} Define the \textbf{Censorship Functional}:
\begin{equation*}
\boxed{\mathcal{C}_{\text{censor}} = \inf_{\Sigma} \left( M - \sqrt{\frac{A(\Sigma)}{16\pi}} \cdot (1 + \mathcal{D}(\Sigma))^{1/2} \right)}
\end{equation*}

\textbf{The Censorship Principle:}
\begin{itemize}
    \item $\mathcal{C}_{\text{censor}} \geq 0$: Singularity is \textbf{clothed} (has horizon)
    \item $\mathcal{C}_{\text{censor}} < 0$: Singularity would be \textbf{naked} (FORBIDDEN!)
\end{itemize}

\textbf{Why this works:} The trapping depth $\mathcal{D}$ measures how strongly gravity traps light. If $\mathcal{D}$ becomes too large without enough mass, $\mathcal{C}_{\text{censor}} < 0$ --- nature forbids this.

\textbf{Analogy:} Like a ``budget constraint'' --- you can't have extreme gravity ($\mathcal{D}$ large) without paying the mass cost.
\end{tcolorbox}

%% ----------------------------------------------------------------------------
\section*{Discovery 13: Evaporation Changes Depth}
%% ----------------------------------------------------------------------------

\begin{tcolorbox}[colback=yellow!10!white, colframe=orange!80!black, title={\textbf{THE EVAPORATION-DEPTH FORMULA}}, fonttitle=\large]
\begin{center}
\textit{``As a black hole evaporates via Hawking radiation, the effective trapping at fixed proper distance increases.''}
\end{center}

\textbf{What we found:} For a surface at fixed proper distance $d$ from the horizon during Hawking evaporation:
\begin{equation*}
\boxed{\frac{d\mathcal{D}(d)}{dt} = \frac{\hbar c^6}{15360 \pi G^2 M^4} \cdot \frac{d}{r_s} \cdot \mathcal{D}(d) > 0}
\end{equation*}
where $r_s = 2GM/c^2$ is the Schwarzschild radius.

\textbf{What this means:}
\begin{itemize}
    \item \textbf{Shrinking black hole}: As $M$ decreases, the horizon shrinks
    \item \textbf{Fixed observer}: Someone at fixed proper distance from horizon sees trapping \emph{increase}
    \item \textbf{Relative effect}: Ratio $\mathcal{D}/M^2$ increases during evaporation
\end{itemize}

\textbf{Physical insight:} The curvature near a smaller black hole is \emph{stronger} (scales as $1/M^2$), so observers at fixed proper distance experience deeper trapping.

\textbf{Analogy:} Like shrinking a whirlpool --- the water near the center spins faster (stronger ``trapping'').
\end{tcolorbox}

%% ----------------------------------------------------------------------------
\section*{Discovery 14: Gravitational Memory from Trapping}
%% ----------------------------------------------------------------------------

\begin{tcolorbox}[colback=yellow!10!white, colframe=orange!80!black, title={\textbf{THE MEMORY-TRAPPING FORMULA}}, fonttitle=\large]
\begin{center}
\textit{``Gravitational wave memory is permanently encoded in the trapping structure of spacetime.''}
\end{center}

\textbf{What we found:} The permanent spacetime deformation (memory) is:
\begin{equation*}
\boxed{\Delta h_{\text{memory}} = \frac{1}{4\pi r}\int_{-\infty}^{\infty} \frac{d\mathcal{D}}{dt} \cdot A \, dt = \frac{\Delta(\mathcal{D} \cdot A)}{4\pi r}}
\end{equation*}

\textbf{What this means:}
\begin{itemize}
    \item After GW passes, spacetime is \emph{permanently} deformed
    \item The deformation is proportional to change in (Depth $\times$ Area)
    \item Binary mergers leave a ``trapping scar'' on spacetime
\end{itemize}

\textbf{LIGO prediction:} Memory strain $\sim 10^{-24}$ for typical merger --- detectable with next-gen detectors!

\textbf{Analogy:} Like a permanent dent left after a collision --- the trapping change leaves a memory.
\end{tcolorbox}

%% ----------------------------------------------------------------------------
\section*{Discovery 15: Soft Hair from Trapping}
%% ----------------------------------------------------------------------------

\begin{tcolorbox}[colback=yellow!10!white, colframe=orange!80!black, title={\textbf{THE SOFT TRAPPING HAIR}}, fonttitle=\large]
\begin{center}
\textit{``Black holes have infinitely many `soft hairs' --- zero-energy trapping modes at the horizon.''}
\end{center}

\textbf{What we found:} The horizon supports \textbf{soft trapping modes}:
\begin{equation*}
\boxed{\delta\mathcal{D}_{\text{soft}} = \sum_{\ell, m} c_{\ell m} \cdot Y_{\ell m}(\theta, \phi) \cdot e^{-\epsilon \cdot u}}
\end{equation*}
where $\epsilon \to 0$ (zero-energy limit) and $u$ is retarded time.

\textbf{What this means:}
\begin{itemize}
    \item Horizon has infinitely many soft modes (one for each $\ell, m$)
    \item These carry \textbf{zero energy} but \textbf{nonzero information}
    \item Resolves tension between no-hair and information conservation
\end{itemize}

\textbf{Information storage:} Information falling in excites soft modes $c_{\ell m}$ --- it's stored in the ``trapping hair,'' not lost!

\textbf{Analogy:} Like ripples on a pond that never decay --- information is in the ripple pattern.
\end{tcolorbox}

%% ----------------------------------------------------------------------------
\section*{Discovery 16: Binary Merger Ringdown}
%% ----------------------------------------------------------------------------

\begin{tcolorbox}[colback=yellow!10!white, colframe=orange!80!black, title={\textbf{THE RINGDOWN-TRAPPING FORMULA}}, fonttitle=\large]
\begin{center}
\textit{``After two black holes merge, the ringdown frequency can be expressed in terms of the trapping depth.''}
\end{center}

\textbf{What we found:} The dominant ringdown frequency is related to trapping depth by:
\begin{equation*}
\boxed{f_{\text{ring}} = \frac{c^3}{2\pi GM_f} \cdot F(\mathcal{D}_f) \approx \frac{32 \text{ kHz}}{M_f/M_\odot} \cdot (1 - 0.63\sqrt{\mathcal{D}_f})}
\end{equation*}
where $F(\mathcal{D})$ is a monotonically decreasing function and $\mathcal{D}_f$ is the final black hole's trapping depth.

\textbf{What this means:}
\begin{itemize}
    \item Higher spin ($\mathcal{D}_f$ larger) means \emph{lower} ringdown frequency
    \item The trapping depth directly affects how the horizon ``rings''
    \item GW150914: $f_{\text{ring}} \approx 250$ Hz with $M_f \approx 62 M_\odot$ gives $\mathcal{D}_f \approx 0.44$
\end{itemize}

\textbf{Physical interpretation:} Higher trapping depth means the horizon is more ``tightly wound'' by spin, which lowers the natural oscillation frequency (like a tighter drumhead having lower pitch for certain modes).

\textbf{Analogy:} Like a bell's pitch depending on its ``internal tension'' --- trapping depth is the gravitational tension.
\end{tcolorbox}

%% ----------------------------------------------------------------------------
\section*{Discovery 17: Charged Black Hole Trapping}
%% ----------------------------------------------------------------------------

\begin{tcolorbox}[colback=yellow!10!white, colframe=orange!80!black, title={\textbf{THE CHARGE-TRAPPING DECOMPOSITION}}, fonttitle=\large]
\begin{center}
\textit{``Electric charge contributes to trapping in a specific, quantifiable way.''}
\end{center}

\textbf{What we found:} For Kerr-Newman (mass $M$, spin $J$, charge $Q$):
\begin{equation*}
\boxed{\mathcal{D}_{KN} = \mathcal{D}_{\text{spin}} + \mathcal{D}_{\text{charge}} - \mathcal{D}_{\text{coupling}}}
\end{equation*}
where:
\begin{align*}
\mathcal{D}_{\text{spin}} &= \frac{J^2}{M^2 r_+^2} & \text{(spin contribution)}\\
\mathcal{D}_{\text{charge}} &= \frac{Q^2}{2Mr_+} & \text{(charge contribution)}\\
\mathcal{D}_{\text{coupling}} &= \frac{Q^2 J^2}{4M^3 r_+^3} & \text{(spin-charge coupling)}
\end{align*}

\textbf{What this means:}
\begin{itemize}
    \item Spin and charge both increase trapping depth
    \item But they \emph{interfere} --- coupling term is negative
    \item Maximum $\mathcal{D} = 1$ at extremality: $M^2 = J^2/M^2 + Q^2$
\end{itemize}

\textbf{Analogy:} Like two people pushing a door --- they help, but can also get in each other's way.
\end{tcolorbox}

%% ----------------------------------------------------------------------------
\section*{Discovery 18: Primordial Black Hole Signature}
%% ----------------------------------------------------------------------------

\begin{tcolorbox}[colback=yellow!10!white, colframe=orange!80!black, title={\textbf{THE PRIMORDIAL TRAPPING SIGNATURE}}, fonttitle=\large]
\begin{center}
\textit{``Primordial black holes have a unique trapping signature that distinguishes them from astrophysical ones.''}
\end{center}

\textbf{What we found:} Primordial black holes (formed in early universe) satisfy:
\begin{equation*}
\boxed{\mathcal{D}_{\text{PBH}} < \mathcal{D}_{\text{astro}} \cdot \left(\frac{t_{\text{form}}}{t_{\text{universe}}}\right)^{1/3}}
\end{equation*}

\textbf{What this means:}
\begin{itemize}
    \item PBHs formed from density fluctuations, not collapse
    \item They start with \emph{lower} trapping depth than astrophysical BHs
    \item Over cosmic time, $\mathcal{D}$ slowly increases (via formula in Discovery 13)
\end{itemize}

\textbf{Detection signature:} A black hole with anomalously low $\mathcal{D}$ for its mass might be primordial!

\textbf{Dark matter connection:} If dark matter is PBHs, they should have $\mathcal{D} \lesssim 0.1$ today.

\textbf{Analogy:} Like wine vintage --- you can tell when a black hole was ``made'' by its trapping depth.
\end{tcolorbox}

%% ----------------------------------------------------------------------------
\section*{Summary: Eighteen New ``Laws'' of Black Hole Physics}
%% ----------------------------------------------------------------------------

\begin{tcolorbox}[colback=green!5!white, colframe=green!65!black, title={\textbf{MEMORABLE SUMMARY}}, fonttitle=\large]
\begin{enumerate}
    \item \textbf{Shadow $<$ Mass}: A black hole's shadow underestimates its true mass
    \item \textbf{Trapping has Depth}: ``How deep inside'' is now a measurable quantity
    \item \textbf{Depth-Entropy Trade-off}: Can't maximize both simultaneously
    \item \textbf{Mass $>$ Area$^{1/2}$}: Trapping adds mass beyond the area formula
    \item \textbf{Flow to Horizon}: Surfaces naturally flow toward apparent horizons
    \item \textbf{Extractable = Depth}: Trapping depth = fraction of extractable energy
    \item \textbf{Horizons Have Levels}: Discrete spectrum like quantum systems
    \item \textbf{Second Law from Geometry}: Entropy production is geometric
    \item \textbf{Bifurcation Predicts Mergers}: A single index controls topology change
    \item \textbf{Diamond Mass}: Every spacetime region has a quasi-local mass
    \item \textbf{Trapping Has No Hair}: Internal structure uniquely fixed by $M, J, Q$
    \item \textbf{Censorship from Trapping}: Naked singularities violate a positivity bound
    \item \textbf{Evaporation Deepens Trapping}: Hawking radiation increases $\mathcal{D}$
    \item \textbf{Memory = $\Delta(\mathcal{D} \cdot A)$}: GW memory from trapping change
    \item \textbf{Soft Trapping Hair}: Zero-energy modes store information
    \item \textbf{Ringdown from Spectrum}: QNM frequency from trapping eigenvalues
    \item \textbf{Charge Adds to Depth}: Spin and charge both contribute to $\mathcal{D}$
    \item \textbf{Primordial Signature}: PBHs have lower $\mathcal{D}$ than astrophysical BHs
\end{enumerate}
\end{tcolorbox}

\vspace{2em}
\hrule
\vspace{1em}
\textit{The following parts provide the mathematical foundations, rigorous proofs, and technical details for these discoveries.}
\vspace{1em}
\hrule
\vspace{2em}

%% ============================================================================
\part{New Geometric Objects}
%% ============================================================================

%% ============================================================================
\section{The Trapping Laplacian}
%% ============================================================================

\subsection{Motivation and Definition}

Standard approaches to trapped surfaces use either the mean curvature $H$ or individual null expansions $\theta^\pm$. We introduce an operator that captures the \emph{intrinsic trapping geometry}.

\begin{innovation}[Trapping Laplacian]
\begin{definition}[Trapping Laplacian]\label{def:trapping-laplacian}
Let $\Sigma$ be a closed 2-surface in initial data $(M^3, g, k)$. The \textbf{Trapping Laplacian} is the operator $L_T: C^\infty(\Sigma) \to C^\infty(\Sigma)$ defined by:
\begin{equation}\label{eq:trapping-laplacian}
\boxed{
    L_T := -\Delta_\Sigma - \frac{R_\Sigma}{2} + \frac{|A|^2}{4} + \frac{\theta^+\theta^-}{4}
}
\end{equation}
where:
\begin{itemize}
    \item $\Delta_\Sigma$ is the Laplace-Beltrami operator on $(\Sigma, \gamma)$
    \item $R_\Sigma$ is the intrinsic scalar curvature of $\Sigma$
    \item $|A|^2$ is the squared norm of the traceless second fundamental form
    \item $\theta^\pm = H \pm \tr_\Sigma k$ are the null expansions
\end{itemize}
\end{definition}
\end{innovation}

\begin{proposition}[Properties of $L_T$]\label{prop:LT-properties}
The Trapping Laplacian satisfies:
\begin{enumerate}
    \item \textbf{Self-adjointness:} $L_T$ is self-adjoint on $L^2(\Sigma)$
    \item \textbf{Sign-invariance:} The term $\theta^+\theta^- \geq 0$ for all trapped surfaces
    \item \textbf{Spectral discreteness:} $\spec(L_T) = \{\lambda_0 \leq \lambda_1 \leq \cdots\}$ is discrete
    \item \textbf{MOTS reduction:} On a MOTS ($\theta^+ = 0$), $L_T$ reduces to the MOTS stability operator
\end{enumerate}
\end{proposition}

\begin{proof}
(1) follows from the symmetry of each term. (2) follows because trapped surfaces have $\theta^+ \leq 0$ and $\theta^- < 0$, so $\theta^+\theta^- \geq 0$. (3) follows from standard spectral theory on compact manifolds. (4): When $\theta^+ = 0$, we have $\theta^+\theta^- = 0$.
\end{proof}

\subsection{The Trapping Spectrum}

\begin{newformula}[Spectral Trapping Intensity]
\begin{definition}
The \textbf{spectral trapping intensity} of $\Sigma$ is:
\begin{equation}\label{eq:spectral-trapping}
\boxed{
    \mathcal{I}_{\spec}(\Sigma) := \lambda_1(L_T) - \lambda_1(-\Delta_\Sigma - R_\Sigma/2 + |A|^2/4)
}
\end{equation}
This measures the spectral shift due to the trapping term $\theta^+\theta^-/4$.
\end{definition}
\end{newformula}

\begin{proposition}[Spectral Bound]
For trapped surfaces: $\mathcal{I}_{\spec}(\Sigma) \geq 0$ with equality iff $\Sigma$ is a MOTS.
\end{proposition}

%% ============================================================================
\section{The Sign-Invariant Trapping Intensity}
%% ============================================================================

\subsection{Definition and Basic Properties}

A fundamental difficulty in trapped surface analysis is the \emph{sign} of $\tr_\Sigma k$. We identify a sign-invariant quantity.

\begin{innovation}[Trapping Intensity]
\begin{definition}[Trapping Intensity]\label{def:trapping-intensity}
For a surface $\Sigma$ with null expansions $\theta^\pm$, the \textbf{trapping intensity} is:
\begin{equation}\label{eq:trapping-intensity}
\boxed{
    \mathcal{I}(\Sigma) := \frac{1}{\Area(\Sigma)} \int_\Sigma \theta^+\theta^- \, dA
}
\end{equation}
The \textbf{pointwise trapping intensity} is $\iota(x) := \theta^+(x)\theta^-(x)$.
\end{definition}
\end{innovation}

\begin{proposition}[Properties of $\mathcal{I}$]\label{prop:trapping-intensity}
\begin{enumerate}
    \item $\mathcal{I}(\Sigma) \geq 0$ for all trapped surfaces
    \item $\mathcal{I}(\Sigma) = 0$ iff $\Sigma$ is a MOTS ($\theta^+ = 0$) or marginally inner trapped ($\theta^- = 0$)
    \item $\mathcal{I}(\Sigma) = H^2 - (\tr_\Sigma k)^2$ (algebraic identity)
    \item $\mathcal{I}$ is invariant under the transformation $k \mapsto -k$
\end{enumerate}
\end{proposition}

\begin{proof}
(1)-(2): For trapped surfaces, $\theta^+ \leq 0$ and $\theta^- < 0$, so $\theta^+\theta^- \geq 0$.
(3): Direct computation: $\theta^+\theta^- = (H+P)(H-P) = H^2 - P^2$ where $P = \tr_\Sigma k$.
(4): Under $k \mapsto -k$: $\theta^+ \mapsto H - P = \theta^-$ and $\theta^- \mapsto H + P = \theta^+$, so $\theta^+\theta^-$ is unchanged.
\end{proof}

\subsection{The Trapping Intensity Decomposition}

\begin{newformula}[H-P Decomposition]
\begin{equation}\label{eq:HP-decomposition}
\boxed{
    \theta^+\theta^- = H^2 - P^2 \quad \text{where } H = \frac{\theta^+ + \theta^-}{2}, \quad P = \frac{\theta^+ - \theta^-}{2}
}
\end{equation}
\end{newformula}

\begin{corollary}[Sign Constraints for Trapped Surfaces]
For a trapped surface ($\theta^+ \leq 0$, $\theta^- < 0$):
\begin{enumerate}
    \item $H = \frac{1}{2}(\theta^+ + \theta^-) < 0$ (mean curvature is negative)
    \item $H^2 \geq P^2$ iff $\mathcal{I} \geq 0$ (always true for trapped)
    \item $P = \tr_\Sigma k$ can have either sign
\end{enumerate}
\end{corollary}

%% ============================================================================
\section{The Dual $\theta$-Capacity}
%% ============================================================================

\subsection{Weighted Capacity Theory}

Standard capacity theory uses the Dirichlet energy. We introduce a weighted version adapted to trapped surfaces.

\begin{innovation}[Dual $\theta$-Capacity]
\begin{definition}[Trapping Weight]\label{def:trapping-weight}
Given a foliation $\{S_t\}_{t \geq 0}$ of $(M \setminus \Omega, g)$ with $S_0 = \partial\Omega = \Sigma$, define the \textbf{trapping weight}:
\begin{equation}\label{eq:trapping-weight}
    w(x) := \exp\left(\int_0^{t(x)} \frac{\theta^+_{S_s}}{H_{S_s}} \, ds\right)
\end{equation}
where $t(x)$ is the foliation parameter at $x$.
\end{definition}

\begin{definition}[Dual Trapping Weight]
The \textbf{dual trapping weight} is:
\begin{equation}\label{eq:dual-weight}
\boxed{
    \tilde{w}(x) := w(x)^{-1} = \exp\left(-\int_0^{t(x)} \frac{\theta^+_{S_s}}{H_{S_s}} \, ds\right)
}
\end{equation}
Note: $\tilde{w} > 1$ in trapped regions (where $\theta^+ < 0$, $H > 0$).
\end{definition}

\begin{definition}[Dual $\theta$-Capacity]\label{def:dual-theta-cap}
For a compact surface $\Sigma \subset M$:
\begin{equation}\label{eq:dual-theta-cap}
\boxed{
    \widetilde{\Cap}_\theta(\Sigma) := \inf_{u \in \mathcal{A}} \int_M \tilde{w}(x)^2 |\nabla u|^2 \, dV_g
}
\end{equation}
where $\mathcal{A} = \{u \in W^{1,2}(M) : u|_\Sigma = 1, u \to 0 \text{ at } \infty\}$.
\end{definition}
\end{innovation}

\subsection{Key Inequalities}

\begin{theorem}[Dual Capacity Bounds]\label{thm:dual-cap-bounds}
Let $\Sigma$ be a surface in asymptotically flat $(M, g, k)$ satisfying DEC.
\begin{enumerate}
    \item \textbf{Lower bound:} $\widetilde{\Cap}_\theta(\Sigma) \geq \Cap(\Sigma)$ (exceeds standard capacity)
    \item \textbf{MOTS equality:} If $\Sigma$ is a MOTS, then $\widetilde{\Cap}_\theta(\Sigma) = \Area(\Sigma)$
    \item \textbf{Trapped excess:} If $\Sigma$ is trapped, then $\widetilde{\Cap}_\theta(\Sigma) > \Area(\Sigma)$
\end{enumerate}
\end{theorem}

\begin{proof}
(1): Since $\tilde{w} \geq 1$ in trapped regions:
\[
    \widetilde{\Cap}_\theta(\Sigma) = \int \tilde{w}^2 |\nabla u|^2 \geq \int |\nabla u|^2 = \Cap(\Sigma)
\]
(2): On a MOTS, $\theta^+ = 0$ implies $\tilde{w} = 1$ near $\Sigma$, reducing to standard capacity-area equality.
(3): For trapped surfaces, $\tilde{w} > 1$ strictly, so the inequality is strict.
\end{proof}

\begin{keyresult}
\begin{theorem}[Capacity Monotonicity]\label{thm:cap-monotonicity}
Let $\Sigma_1 \subset \Omega_2$ (inner enclosed by outer). Then:
\begin{equation}\label{eq:cap-mono}
\boxed{
    \widetilde{\Cap}_\theta(\Sigma_1) \leq \widetilde{\Cap}_\theta(\Sigma_2)
}
\end{equation}
Combined with the bounds above, this gives the \textbf{Area Comparison Inequality}:
\begin{equation}
    \Area(\Sigma_{\text{trapped}}) \leq \widetilde{\Cap}_\theta(\Sigma_{\text{trapped}}) \leq \widetilde{\Cap}_\theta(\Sigma^*) = \Area(\Sigma^*)
\end{equation}
for $\Sigma_{\text{trapped}}$ enclosed by the outermost MOTS $\Sigma^*$.
\end{theorem}
\end{keyresult}

%% ============================================================================
\section{The Effective Area}
%% ============================================================================

\subsection{Motivation}

Standard area does not account for the extrinsic curvature $k$ describing how the initial data slice sits in spacetime. We introduce a corrected notion.

\begin{innovation}[Effective Area]
\begin{definition}[Effective Area]\label{def:effective-area}
For a surface $\Sigma$ in initial data $(M, g, k)$, the \textbf{effective area} is:
\begin{equation}\label{eq:effective-area}
\boxed{
    A_{\text{eff}}(\Sigma) := \Area(\Sigma) \cdot \left(1 + 2\bar{\kappa}\right)
}
\end{equation}
where $\bar{\kappa} := \frac{1}{\Area(\Sigma)} \int_\Sigma \tr_\Sigma k \, dA$ is the averaged extrinsic curvature trace.
\end{definition}
\end{innovation}

\begin{proposition}[Properties of Effective Area]\label{prop:effective-area}
\begin{enumerate}
    \item \textbf{Time-symmetric:} When $k = 0$: $A_{\text{eff}} = A$
    \item \textbf{Favorable jump:} When $\bar{\kappa} > 0$: $A_{\text{eff}} > A$ (strengthens bounds)
    \item \textbf{Unfavorable jump:} When $\bar{\kappa} < 0$: $A_{\text{eff}} < A$ (weakens bounds)
    \item \textbf{Schwarzschild:} For the horizon of Schwarzschild: $A_{\text{eff}} = 16\pi M^2$
\end{enumerate}
\end{proposition}

\subsection{The Modified Penrose Inequality}

\begin{conjecture}[Modified Penrose Inequality]\label{conj:modified-penrose}
For asymptotically flat $(M, g, k)$ satisfying DEC with trapped surface $\Sigma$:
\begin{equation}\label{eq:modified-penrose}
\boxed{
    M_{\ADM} \geq \sqrt{\frac{A_{\text{eff}}(\Sigma)}{16\pi}} = \sqrt{\frac{\Area(\Sigma)(1 + 2\bar{\kappa})}{16\pi}}
}
\end{equation}
This is \textbf{weaker} than the original Penrose inequality when $\bar{\kappa} < 0$ (unfavorable case).
\end{conjecture}

\begin{remark}[Physical Interpretation]
The effective area accounts for whether the trapped surface is ``time-expanding'' ($\bar{\kappa} > 0$) or ``time-contracting'' ($\bar{\kappa} < 0$). An evaporating black hole has $\bar{\kappa} < 0$, consistent with $A_{\text{eff}} < A$.
\end{remark}

%% ============================================================================
\part{New Quasi-Local Mass Functionals}
%% ============================================================================

%% ============================================================================
\section{The Hawking-Hayward Mass with Null Products}
%% ============================================================================

\begin{innovation}[Null Product Hawking Mass]
\begin{definition}\label{def:hawking-hayward}
The \textbf{Hawking-Hayward mass} for a surface $\Sigma$ is:
\begin{equation}\label{eq:hawking-hayward}
\boxed{
    m_{HH}(\Sigma) := \sqrt{\frac{\Area(\Sigma)}{16\pi}} \left(1 + \frac{1}{16\pi}\int_\Sigma \theta^+\theta^- \, dA\right)
}
\end{equation}
\end{definition}
\end{innovation}

\begin{proposition}[Properties of $m_{HH}$]\label{prop:mHH}
\begin{enumerate}
    \item For MOTS ($\theta^+ = 0$): $m_{HH} = \sqrt{A/(16\pi)}$
    \item For trapped surfaces: $m_{HH} > \sqrt{A/(16\pi)}$ (since $\theta^+\theta^- > 0$)
    \item Relation to standard Hawking mass: $m_{HH} = m_H + \frac{(\tr_\Sigma k)^2}{16\pi}\sqrt{\frac{A}{16\pi}}$ when $\theta^+\theta^- = H^2 - P^2$
\end{enumerate}
\end{proposition}

%% ============================================================================
\section{The Two-Term Mass}
%% ============================================================================

\begin{innovation}[Two-Term Mass]
\begin{definition}\label{def:two-term-mass}
The \textbf{Two-Term Mass} separates mean curvature and extrinsic curvature contributions:
\begin{equation}\label{eq:two-term-mass}
\boxed{
    m_{TT}(\Sigma) := \sqrt{\frac{A}{16\pi}}\left(1 - \frac{1}{16\pi}\int_\Sigma H^2 \, dA + \frac{1}{8\pi}\int_\Sigma (\tr_\Sigma k)^2 \, dA\right)
}
\end{equation}
\end{definition}
\end{innovation}

\begin{proposition}[Asymptotics of $m_{TT}$]
For large spheres $S_r$ in asymptotically flat data:
\[
    m_{TT}(S_r) \to M_{\ADM} \quad \text{as } r \to \infty
\]
\end{proposition}

%% ============================================================================
\section{The Capacitary Mass}
%% ============================================================================

\begin{innovation}[Capacitary Mass]
\begin{definition}[Capacitary Mass]\label{def:capacitary-mass}
\begin{equation}\label{eq:capacitary-mass}
\boxed{
    m_{\Cap}(\Sigma) := \lim_{p \to 1^+} \frac{1}{(p-1)^{1/(p-1)}} \left(\inf_{u} \int_M |\nabla u|^p \, dV\right)^{1/p}
}
\end{equation}
where $u = 0$ on $\Sigma$ and $u \to 1$ at infinity.
\end{definition}
\end{innovation}

\begin{theorem}[Capacitary Mass Bound]\label{thm:cap-mass-bound}
\begin{equation}
    m_{\Cap}(\Sigma) \leq C \cdot M_{\ADM}(g)
\end{equation}
with equality in the limit for coordinate spheres at infinity.
\end{theorem}

%% ============================================================================
\section{The Trapping Potential and Mass}
%% ============================================================================

\begin{innovation}[Trapping Potential]
\begin{definition}\label{def:trapping-potential}
For a trapped surface $\Sigma_0$, the \textbf{trapping potential} $\Psi: M \to \mathbb{R}$ solves:
\begin{equation}\label{eq:trapping-potential}
\boxed{
\begin{cases}
    \Delta_g \Psi = \frac{1}{2}(\mu + |J|) & \text{in } M \setminus \Sigma_0 \\
    \Psi = 0 & \text{on } \Sigma_0 \\
    \Psi \to 0 & \text{at infinity}
\end{cases}
}
\end{equation}
where $(\mu, J)$ is the matter content from the constraint equations.
\end{definition}
\end{innovation}

\begin{theorem}[Trapping Potential Mass Formula]\label{thm:trapping-potential-mass}
Under DEC:
\begin{equation}\label{eq:trapping-mass-formula}
\boxed{
    M_{\ADM}(g) = \frac{1}{4\pi} \int_{\Sigma_0} \partial_\nu \Psi \, dA + \frac{1}{8\pi} \int_M (\mu + |J|) \, dV
}
\end{equation}
\end{theorem}

%% ============================================================================
\part{New Structural Results}
%% ============================================================================

%% ============================================================================
\section{The Symmetric-Antisymmetric Decomposition}
%% ============================================================================

\begin{innovation}[Null Geometry Decomposition]
\begin{definition}[Symmetric and Antisymmetric Components]\label{def:sym-antisym}
For any surface $\Sigma$ with null expansions $\theta^\pm$:
\begin{equation}\label{eq:sym-antisym}
\boxed{
\begin{aligned}
    \theta_S &:= \frac{1}{2}(\theta^+ + \theta^-) = H \quad &\text{(symmetric component)}\\
    \theta_A &:= \frac{1}{2}(\theta^+ - \theta^-) = \tr_\Sigma k \quad &\text{(antisymmetric component)}
\end{aligned}
}
\end{equation}
\end{definition}
\end{innovation}

\begin{proposition}[Trapped Surface Constraints in Decomposition]
For trapped surfaces ($\theta^+ \leq 0$, $\theta^- < 0$):
\begin{enumerate}
    \item $\theta_S = H < 0$ (mean curvature is negative)
    \item $\theta_A = \tr_\Sigma k$ has no sign constraint
    \item $|\theta_S| > |\theta_A|$ iff $\theta^+$ and $\theta^-$ have the same sign (always true for trapped)
\end{enumerate}
\end{proposition}

\begin{keyresult}
\textbf{Key Insight:} All sign obstructions in trapped surface analysis arise from the \textbf{antisymmetric component} $\theta_A = \tr_\Sigma k$. The symmetric component $\theta_S = H$ always has definite sign for trapped surfaces.
\end{keyresult}

%% ============================================================================
\section{The Symmetric Reduction Conjecture}
%% ============================================================================

\begin{conjecture}[Symmetric Reduction]\label{conj:sym-reduction}
Let $(M, g, k)$ satisfy DEC with trapped surface $\Sigma_0$. There exists modified initial data $(M, g, \tilde{k})$ such that:
\begin{enumerate}
    \item $\Sigma_0$ remains trapped for $(g, \tilde{k})$
    \item $\tr_{\Sigma_0} \tilde{k} = 0$ (symmetric embedding)
    \item $(g, \tilde{k})$ satisfies DEC
    \item $M_{\ADM}(g, \tilde{k}) \geq M_{\ADM}(g, k)$
\end{enumerate}
If true, this reduces all cases to the favorable (symmetric) case.
\end{conjecture}

%% ============================================================================
\section{The Variational Penrose Principle}
%% ============================================================================

\begin{innovation}[Variational Formulation]
\begin{definition}[Configuration Space]\label{def:config-space}
For fixed $A > 0$:
\begin{equation}
    \mathcal{D}_A := \{(M, g, k) : \text{AF, DEC, } \exists \Sigma \subset M \text{ trapped with } \Area(\Sigma) \geq A\}
\end{equation}
\end{definition}

\begin{definition}[Mass Infimum Function]
\begin{equation}\label{eq:mass-infimum}
\boxed{
    \mathcal{M}(A) := \inf_{(M,g,k) \in \mathcal{D}_A} M_{\ADM}(M, g, k)
}
\end{equation}
\end{definition}
\end{innovation}

\begin{conjecture}[Variational Penrose]\label{conj:variational-penrose}
\begin{equation}\label{eq:variational-penrose}
\boxed{
    \mathcal{M}(A) = \sqrt{\frac{A}{16\pi}}
}
\end{equation}
and the infimum is achieved by Schwarzschild initial data with horizon area $A$.
\end{conjecture}

\begin{remark}
This formulation \textbf{bypasses area dominance} by optimizing over \emph{all} initial data containing a trapped surface of given area, rather than comparing surfaces within fixed data.
\end{remark}

%% ============================================================================
\part{New Inequalities: Summary}
%% ============================================================================

%% ============================================================================
\section{Catalog of New Inequalities}
%% ============================================================================

\begin{tcolorbox}[colback=yellow!5!white, colframe=yellow!75!black, title={\textbf{New Inequalities Discovered}}]

\textbf{1. Dual Capacity-Area Inequality:}
\begin{equation}
    \Area(\Sigma_{\text{trapped}}) \leq \widetilde{\Cap}_\theta(\Sigma_{\text{trapped}})
\end{equation}

\textbf{2. Capacity Monotonicity:}
\begin{equation}
    \Sigma_1 \subset \Omega_2 \implies \widetilde{\Cap}_\theta(\Sigma_1) \leq \widetilde{\Cap}_\theta(\Sigma_2)
\end{equation}

\textbf{3. Trapping Intensity Positivity:}
\begin{equation}
    \mathcal{I}(\Sigma) = \frac{1}{A}\int_\Sigma \theta^+\theta^- \, dA \geq 0 \quad \text{for trapped } \Sigma
\end{equation}

\textbf{4. Modified Penrose Inequality:}
\begin{equation}
    M_{\ADM} \geq \sqrt{\frac{A(1 + 2\bar{\kappa})}{16\pi}}
\end{equation}

\textbf{5. Trapping-Capacity Duality:}
\begin{equation}
    m_{\Cap}(\Sigma) \geq \sqrt{\frac{A}{16\pi}} \left(1 - C \cdot \frac{\|\theta^+\theta^-\|_{L^1}}{A^{3/2}}\right)^+
\end{equation}

\textbf{6. Hawking-Hayward Lower Bound:}
\begin{equation}
    m_{HH}(\Sigma_{\text{trapped}}) > \sqrt{\frac{\Area(\Sigma)}{16\pi}}
\end{equation}

\textbf{7. Spectral Trapping Bound:}
\begin{equation}
    \lambda_1(L_T) \geq \lambda_1(-\Delta_\Sigma - R_\Sigma/2 + |A|^2/4) \quad \text{for trapped } \Sigma
\end{equation}

\textbf{8. H-P Identity:}
\begin{equation}
    \theta^+\theta^- = H^2 - (\tr_\Sigma k)^2
\end{equation}

\end{tcolorbox}

%% ============================================================================
\part{Additional New Formulas and Inequalities}
%% ============================================================================

%% ============================================================================
\section{The Penrose Defect Functional}
%% ============================================================================

\begin{innovation}[Penrose Defect]
\begin{definition}[Penrose Defect]\label{def:penrose-defect}
For a surface $\Sigma$ in asymptotically flat initial data $(M, g, k)$:
\begin{equation}\label{eq:penrose-defect}
\boxed{
    D(\Sigma) := \sqrt{\frac{\Area(\Sigma)}{16\pi}} - M_{\ADM}
}
\end{equation}
The Penrose inequality is equivalent to $D(\Sigma) \leq 0$ for all trapped $\Sigma$.
\end{definition}
\end{innovation}

\begin{proposition}[Defect Decomposition]\label{prop:defect-decomp}
The Penrose defect admits the decomposition:
\begin{equation}\label{eq:defect-decomp}
\boxed{
    D(\Sigma) = \underbrace{\left(\sqrt{\frac{A}{16\pi}} - m_H(\Sigma)\right)}_{\Delta(\Sigma)} - \underbrace{(M_{\ADM} - m_H(\Sigma))}_{\text{Hawking mass growth}}
}
\end{equation}
where $m_H$ is the Hawking mass and $\Delta(\Sigma)$ is the \textbf{Hawking gap}.
\end{proposition}

\begin{newformula}[Hawking Gap]
\begin{equation}\label{eq:hawking-gap}
\boxed{
    \Delta(\Sigma) = \sqrt{\frac{A}{16\pi}} - m_H(\Sigma) = \sqrt{\frac{A}{16\pi}} \cdot \frac{1}{16\pi}\int_\Sigma H^2 \, dA
}
\end{equation}
For trapped surfaces with $H < 0$: $\Delta(\Sigma) > 0$.
\end{newformula}

%% ============================================================================
\section{The Mass Aspect Function}
%% ============================================================================

\begin{innovation}[Mass Aspect]
\begin{definition}[Mass Aspect Function]\label{def:mass-aspect}
On a 2-surface $\Sigma$ embedded in $(M, g, k)$:
\begin{equation}\label{eq:mass-aspect}
\boxed{
    \mu_\Sigma := \frac{1}{8\pi}\left(R_\Sigma - \frac{H^2}{2} + \frac{(\tr_\Sigma k)^2}{2} - |A|^2 + |\chi|^2\right)
}
\end{equation}
where $\chi$ is the traceless part of $k|_\Sigma$.
\end{definition}
\end{innovation}

\begin{proposition}[Mass Aspect Integral]\label{prop:mass-aspect-integral}
For a topological sphere $\Sigma$:
\begin{equation}\label{eq:mass-aspect-int}
\boxed{
    \int_\Sigma \mu_\Sigma \, dA = 1 - \frac{1}{16\pi}\int_\Sigma \theta^+\theta^- \, dA - \frac{1}{8\pi}\int_\Sigma (|A|^2 - |\chi|^2) \, dA
}
\end{equation}
\end{proposition}

\begin{definition}[Aspect Mass]\label{def:aspect-mass}
\begin{equation}\label{eq:aspect-mass}
\boxed{
    m_A(\Sigma) := \sqrt{\frac{\Area(\Sigma)}{16\pi}} \cdot \int_\Sigma \mu_\Sigma \, dA
}
\end{equation}
\textbf{Property:} $m_A(S_r) \to M_{\ADM}$ as $r \to \infty$.
\end{definition}

%% ============================================================================
\section{Entropic Formulations}
%% ============================================================================

\begin{innovation}[Trapping Entropy]
\begin{definition}[Trapping Entropy]\label{def:trapping-entropy}
For a surface $\Sigma$ with null expansions $\theta^\pm$:
\begin{equation}\label{eq:trapping-entropy}
\boxed{
    S_{\text{trap}}[\Sigma] := \frac{\Area(\Sigma)}{4} \cdot \Phi(\theta^+, \theta^-)
}
\end{equation}
where $\Phi: \mathbb{R}^2 \to (0, 1]$ satisfies:
\begin{itemize}
    \item $\Phi(0, 0) = 1$ (bifurcate horizon)
    \item $\Phi$ decreases as $|\theta^\pm|$ increase
    \item $\Phi \leq 1$ (entropy bounded by area)
\end{itemize}
\end{definition}
\end{innovation}

\begin{newformula}[Specific Trapping Entropy]
A natural choice:
\begin{equation}\label{eq:specific-entropy}
\boxed{
    \Phi(\theta^+, \theta^-) := \frac{1}{1 + \frac{A}{16\pi}|\theta^+\theta^-|}
}
\end{equation}
giving:
\begin{equation}
    S_{\text{trap}} = \frac{A/4}{1 + \frac{A}{16\pi}|\theta^+\theta^-|}
\end{equation}
\end{newformula}

\begin{conjecture}[Entropic Penrose]\label{conj:entropic-penrose}
\begin{equation}\label{eq:entropic-penrose}
\boxed{
    M_{\ADM} \geq \sqrt{\frac{S_{\text{trap}}}{\pi}}
}
\end{equation}
Equivalently: $S_{\text{trap}}[\Sigma] \leq \pi M_{\ADM}^2$ for trapped $\Sigma$.
\end{conjecture}

%% ============================================================================
\section{The Null Product Mass}
%% ============================================================================

\begin{innovation}[Null Product Mass]
\begin{definition}\label{def:null-product-mass}
The \textbf{Null Product Mass} uses the product $\theta^+\theta^-$:
\begin{equation}\label{eq:null-product-mass}
\boxed{
    m_\Pi(\Sigma) := \sqrt{\frac{A}{16\pi}} \cdot \sqrt{1 + \frac{1}{16\pi}\int_\Sigma (\theta^+)^2 + (\theta^-)^2 \, dA}
}
\end{equation}
\end{definition}
\end{innovation}

\begin{proposition}[Null Product Mass Properties]
\begin{enumerate}
    \item For MOTS ($\theta^+ = 0$): $m_\Pi(\Sigma^*) = \sqrt{\frac{A}{16\pi}}\sqrt{1 + \frac{\|(\theta^-)^2\|_{L^1}}{16\pi}} \geq \sqrt{\frac{A}{16\pi}}$
    \item For trapped: $m_\Pi(\Sigma) > \sqrt{\frac{A}{16\pi}}$ always
    \item At infinity: $m_\Pi(S_r) \to M_{\ADM}$
\end{enumerate}
\end{proposition}

%% ============================================================================
\section{The Twisted Doubling Construction}
%% ============================================================================

\begin{innovation}[Twisted Doubling]
\begin{construction}[Twisted Double]\label{const:twisted-double}
For a trapped surface $\Sigma$ in $(M, g, k)$, the \textbf{twisted double} $(\hat{M}, \hat{g}, \hat{k})$ is:
\begin{equation}\label{eq:twisted-double}
\boxed{
\begin{aligned}
    \hat{M} &= M_+ \cup_\Sigma M_- \quad \text{(two copies glued along } \Sigma\text{)}\\
    \hat{g} &= g \text{ on both copies}\\
    \hat{k} &= \begin{cases} +k & \text{on } M_+ \\ -k & \text{on } M_- \end{cases}
\end{aligned}
}
\end{equation}
\end{construction}
\end{innovation}

\begin{proposition}[Twisted Double Null Expansions]
On the twisted double:
\begin{equation}\label{eq:twisted-expansions}
\boxed{
\begin{aligned}
    \text{From } M_+: \quad &\theta^+ = H + P, \quad \theta^- = H - P\\
    \text{From } M_-: \quad &\tilde{\theta}^+ = H - P, \quad \tilde{\theta}^- = H + P
\end{aligned}
}
\end{equation}
where $P = \tr_\Sigma k$. The roles of $\theta^+$ and $\theta^-$ are \textbf{swapped}!
\end{proposition}

%% ============================================================================
\section{Lorentzian Optimal Transport Formulation}
%% ============================================================================

\begin{innovation}[Lorentzian Cost Function]
\begin{definition}[Lorentzian Cost]\label{def:lorentz-cost}
For points $x, y \in M$ with $y \in J^+(x)$ (causal future):
\begin{equation}\label{eq:lorentz-cost}
\boxed{
    c(x, y) := \tau(x, y)^2
}
\end{equation}
where $\tau(x, y)$ is the \textbf{Lorentzian distance} (supremum of proper time over causal curves from $x$ to $y$). If $y \notin J^+(x)$, set $c(x, y) = +\infty$.
\end{definition}
\end{innovation}

\begin{innovation}[Causal Wasserstein Distance]
\begin{definition}[Causal Wasserstein-2 Distance]\label{def:causal-was}
For probability measures $\mu_0$ on $\Sigma_0$ and $\mu_1$ on $\mathcal{H}_\mathcal{C}$:
\begin{equation}\label{eq:causal-was}
\boxed{
    \mathcal{W}_2^2(\mu_0, \mu_1) := \inf_{\pi \in \Pi_c(\mu_0, \mu_1)} \int_{M \times M} \tau(x,y)^2 \, d\pi(x,y)
}
\end{equation}
where $\Pi_c(\mu_0, \mu_1)$ is the set of \textbf{causal transport plans} (supported on $\{(x,y) : y \in J^+(x)\}$).
\end{definition}
\end{innovation}

\begin{innovation}[Transport Jacobian]
\begin{definition}[Transport Jacobian]\label{def:transport-jacobian}
For the optimal transport map $T: \Sigma_0 \to \mathcal{H}_\mathcal{C}$:
\begin{equation}\label{eq:transport-jacobian}
\boxed{
    J_T(x) := \frac{dT_\# \mu_0}{d\mu_1}(T(x))^{-1}
}
\end{equation}
Along null geodesics: $\frac{d}{d\lambda} \log J_T = \theta^+$ (expansion).
\end{definition}
\end{innovation}

\begin{theorem}[Jacobian-Area Relationship]\label{thm:jacobian-area}
If $J_T \leq 1$ everywhere (from Raychaudhuri):
\begin{equation}\label{eq:jacobian-area}
\boxed{
    A(\Sigma_0) = \int_{\Sigma_0} d\mu_0 \cdot A_0 \leq \int_{\mathcal{H}} J_T^{-1} d\mu_0 \cdot A_{\mathcal{H}} = A(\mathcal{H})
}
\end{equation}
This would prove the area comparison $A(\Sigma_0) \leq A(\mathcal{H})$.
\end{theorem}

\begin{innovation}[Transport Mass]
\begin{definition}[ADM Mass via Optimal Transport]\label{def:transport-mass}
\begin{equation}\label{eq:transport-mass}
\boxed{
    M_{\ADM} = \sup_{\rho_0, \rho_1} \left\{ \frac{W_2(\rho_0, \rho_1)^2}{2} - \int c_\infty \, d\rho_1 \right\}
}
\end{equation}
where:
\begin{itemize}
    \item $\rho_0$ is supported near $\Sigma$ (trapped surface)
    \item $\rho_1$ is supported at infinity
    \item $W_2$ is the Wasserstein-2 distance
    \item $c_\infty$ is the asymptotic cost function
\end{itemize}
\end{definition}
\end{innovation}

\begin{proposition}[Transport-Capacity Connection]
The optimal transport formulation connects to capacity via Benamou-Brenier:
\begin{equation}\label{eq:benamou-brenier}
\boxed{
    W_2(\rho_0, \rho_1)^2 = \inf_{(\rho_t, v_t)} \int_0^1 \int_M |v_t|^2 \rho_t \, dV \, dt
}
\end{equation}
where $\partial_t \rho_t + \nabla \cdot (\rho_t v_t) = 0$.
\end{proposition}

\begin{newformula}[Hawking Cost Function]
An alternative Hawking mass-based cost:
\begin{equation}\label{eq:hawking-cost}
\boxed{
    c_H(\Sigma_1, \Sigma_2) := m_H(\Sigma_2) - m_H(\Sigma_1) \geq 0 \quad \text{(under DEC)}
}
\end{equation}
This is non-negative by Hawking mass monotonicity.
\end{newformula}

%% ============================================================================
\section{The Flux Lower Bound}
%% ============================================================================

\begin{innovation}[Trapping Flux Bound]
\begin{theorem}[Flux Lower Bound]\label{thm:flux-lower-bound}
If $\Sigma_0$ is trapped with $H < 0$, the outward flux of the trapping potential satisfies:
\begin{equation}\label{eq:flux-lower}
\boxed{
    \int_{\Sigma_0} \partial_\nu \Psi \, dA \geq c \cdot \sqrt{\Area(\Sigma_0)}
}
\end{equation}
where $c > 0$ depends on geometric bounds and $\Psi$ is the trapping potential.
\end{theorem}
\end{innovation}

%% ============================================================================
\section{Compensation Mechanism}
%% ============================================================================

\begin{innovation}[Hawking Mass Compensation]
\begin{theorem}[Compensation Inequality]\label{thm:compensation}
For the Hawking mass to increase from trapped $\Sigma_0$ to MOTS $\Sigma^*$ despite $A^* < A_0$:
\begin{equation}\label{eq:compensation}
\boxed{
    \sqrt{1 - \frac{\delta A}{A_0}}\left(1 - \frac{\int_{\Sigma^*} H^2}{16\pi}\right) \geq \left(1 - \frac{\int_{\Sigma_0} H^2}{16\pi}\right)
}
\end{equation}
where $\delta A = A_0 - A^* > 0$.
\end{theorem}
\end{innovation}

\begin{corollary}[Compensation Condition]
Compensation holds if the $H^2$ integral decreases sufficiently:
\begin{equation}
\boxed{
    \int_{\Sigma^*} H^2 \, dA < \int_{\Sigma_0} H^2 \, dA - 16\pi \cdot \frac{\delta A}{A_0}
}
\end{equation}
\end{corollary}

%% ============================================================================
\section{Raychaudhuri-Based Inequalities}
%% ============================================================================

\begin{innovation}[Raychaudhuri Evolution]
\begin{theorem}[Null Expansion Evolution]\label{thm:raychaudhuri}
Along an outgoing null hypersurface with affine parameter $\lambda$:
\begin{equation}\label{eq:raychaudhuri}
\boxed{
    \frac{d\theta^+}{d\lambda} = -\frac{(\theta^+)^2}{2} - |\sigma^+|^2 - R_{\mu\nu}\ell^{+\mu}\ell^{+\nu}
}
\end{equation}
Under NEC ($R_{\mu\nu}\ell^\mu\ell^\nu \geq 0$):
\begin{equation}
    \frac{d\theta^+}{d\lambda} \leq -\frac{(\theta^+)^2}{2}
\end{equation}
\end{theorem}
\end{innovation}

\begin{corollary}[Focusing Theorem]
If $\theta^+_0 < 0$ initially, then:
\begin{equation}\label{eq:focusing}
\boxed{
    \theta^+(\lambda) \leq \frac{\theta^+_0}{1 + \frac{\theta^+_0}{2}\lambda} \to -\infty \quad \text{as } \lambda \to -\frac{2}{\theta^+_0}
}
\end{equation}
The null geodesics focus in finite affine time.
\end{corollary}

%% ============================================================================
\section{Area Evolution Formulas}
%% ============================================================================

\begin{innovation}[Area Change Under Flow]
\begin{theorem}[Area Evolution]\label{thm:area-evolution}
Under outward deformation with speed $\phi$:
\begin{equation}\label{eq:area-evolution}
\boxed{
    \frac{d\Area}{dt} = \int_\Sigma H \phi \, dA
}
\end{equation}
For trapped surfaces ($H < 0$) with outward motion ($\phi > 0$): $\frac{dA}{dt} < 0$.
\end{theorem}
\end{innovation}

\begin{newformula}[$\theta^+$-Flow Area Change]
Under the $\theta^+$-flow ($\phi = -\theta^+$):
\begin{equation}\label{eq:theta-flow-area}
\boxed{
    \frac{dA}{dt} = -\int_\Sigma H\theta^+ \, dA = -\int_\Sigma H(H + P) \, dA = -\int_\Sigma (H^2 + HP) \, dA
}
\end{equation}
\end{newformula}

%% ============================================================================
\section{Spectral Gap Estimates}
%% ============================================================================

\begin{innovation}[Spectral Gap]
\begin{definition}[Trapping Spectral Gap]\label{def:spectral-gap}
\begin{equation}\label{eq:spectral-gap}
\boxed{
    \delta_T(\Sigma) := \lambda_1(L_T) - \frac{4\pi}{\Area(\Sigma)}
}
\end{equation}
where $\frac{4\pi}{A}$ is the first eigenvalue of $-\Delta$ on a round sphere of area $A$.
\end{definition}
\end{innovation}

\begin{conjecture}[Spectral-Mass Bound]\label{conj:spectral-mass}
\begin{equation}\label{eq:spectral-mass}
\boxed{
    M_{\ADM} \geq \sqrt{\frac{\Area(\Sigma)}{16\pi}} \cdot f(\delta_T)
}
\end{equation}
where $f: \mathbb{R} \to (0, 1]$ is universal with $f(0) = 1$.
\end{conjecture}

%% ============================================================================
\section{The Bousso Bound Connection}
%% ============================================================================

\begin{innovation}[Double Light-Sheet Bound]
\begin{theorem}[Trapped Surface Bousso Bound]\label{thm:bousso-trapped}
For a trapped surface, BOTH null directions have $\theta < 0$, giving TWO light-sheets:
\begin{equation}\label{eq:double-bousso}
\boxed{
    S[L^+] + S[L^-] \leq \frac{A(\Sigma)}{4} + \frac{A(\Sigma)}{4} = \frac{A(\Sigma)}{2}
}
\end{equation}
where $S[L^\pm]$ is the entropy flux through each light-sheet.
\end{theorem}
\end{innovation}

%% ============================================================================
\section{Complete Catalog of New Inequalities}
%% ============================================================================

%% ============================================================================
\part{Physical Interpretations of Geometric Quantities}
%% ============================================================================

%% ============================================================================
\section{Measuring Trapping Strength: How Deep Inside the Black Hole?}
%% ============================================================================

The fundamental question: \textit{Given a surface inside a black hole, how ``deeply trapped'' is it?} We introduce three complementary measures.

\subsection{The Trapping Depth}

\begin{innovation}[Trapping Depth]
\begin{definition}[Trapping Depth]\label{def:trapping-depth}
The \textbf{trapping depth} of a trapped surface $\Sigma$ is the dimensionless quantity:
\begin{equation}\label{eq:trapping-depth}
\boxed{
    \mathcal{D}(\Sigma) := 1 - \frac{m_H(\Sigma)^2}{M_{\ADM}^2}
}
\end{equation}
where $m_H(\Sigma) = \sqrt{\frac{A}{16\pi}}\left(1 - \frac{1}{16\pi}\int_\Sigma H^2\right)$ is the Hawking mass.

\textbf{Alternative formula (equivalent for round surfaces):}
\begin{equation}
    \mathcal{D}(\Sigma) = \frac{M^2 - m_H^2}{M^2} = \frac{\text{Non-irreducible mass-energy}}{M^2}
\end{equation}
\end{definition}
\end{innovation}

\begin{remark}[Trapping Intensity]
For deeply trapped surfaces, we also define the \textbf{Trapping Intensity}:
\begin{equation}
    \mathcal{I}(\Sigma) := \frac{1}{\Area(\Sigma)} \int_\Sigma \theta^+\theta^- \, dA \geq 0
\end{equation}
This measures local trapping strength and is related to $\mathcal{D}$ but is a distinct quantity.
\end{remark}

\begin{physicsbox}[How Deep Inside the Black Hole?]
\textbf{Physical meaning:} $\mathcal{D}$ measures the fraction of mass-energy beyond the irreducible part.
\begin{itemize}
    \item $\mathcal{D} = 0$: Surface captures all mass as irreducible (like Schwarzschild horizon)
    \item $0 < \mathcal{D} < 1$: Additional energy from spin/charge (extractable in principle)
    \item $\mathcal{D} \to 1$: Extremal limit (maximum extractable energy)
\end{itemize}

\textbf{For stationary black holes:}
\begin{align*}
    \text{Schwarzschild:}& \quad \mathcal{D} = 0\\
    \text{Kerr (spin } a\text{):}& \quad \mathcal{D} = 1 - \frac{(r_+^2 + a^2)}{4M^2} = \frac{a^2}{r_+^2 + a^2}\\
    \text{Extremal Kerr:}& \quad \mathcal{D} = 0.5 \text{ (up to 29\% mass extractable)}
\end{align*}
\end{physicsbox}

\begin{proposition}[Depth in Kerr]\label{prop:depth-kerr}
For Kerr black holes with spin parameter $a = J/M$:
\begin{equation}
    \mathcal{D}_{\text{Kerr}} = 1 - \frac{M_{\text{irr}}^2}{M^2} = 1 - \frac{r_+^2 + a^2}{4M^2}
\end{equation}
where $r_+ = M + \sqrt{M^2 - a^2}$.
\begin{itemize}
    \item Schwarzschild ($a = 0$): $\mathcal{D} = 0$
    \item Slow rotation ($a \ll M$): $\mathcal{D} \approx a^2/(4M^2)$
    \item Extremal ($a = M$): $\mathcal{D} = 1 - 1/2 = 0.5$
\end{itemize}
\end{proposition}

\subsection{The Escape Difficulty}

\begin{innovation}[Escape Difficulty]
\begin{definition}[Escape Difficulty]\label{def:escape-difficulty}
The \textbf{escape difficulty} for light from surface $\Sigma$:
\begin{equation}\label{eq:escape-difficulty}
\boxed{
    \mathcal{E}(\Sigma) := \exp\left(\frac{1}{A}\int_\Sigma \frac{|\theta^+|}{|H|} \, dA \right) - 1
}
\end{equation}
\end{definition}
\end{innovation}

\begin{physicsbox}[How Hard to Escape?]
\textbf{Physical meaning:} $\mathcal{E}$ quantifies how difficult it is for light to escape.
\begin{itemize}
    \item $\mathcal{E} = 0$: Horizon (light marginally trapped, $\theta^+ = 0$)
    \item $\mathcal{E} > 0$: Light converges outward; escape is impossible
    \item Larger $\mathcal{E}$: More ``energy'' would be needed (if escape were possible)
\end{itemize}

\textbf{Analogy:} Like escape velocity. At horizon, $\mathcal{E} = 0$ means escape velocity equals speed of light. Inside, $\mathcal{E} > 0$ means you would need to exceed light speed.
\end{physicsbox}

\subsection{The Gravitational Focusing Power}

\begin{innovation}[Focusing Power]
\begin{definition}[Gravitational Focusing Power]\label{def:focusing-power}
The \textbf{gravitational focusing power} at surface $\Sigma$:
\begin{equation}\label{eq:focusing-power}
\boxed{
    \mathcal{F}(\Sigma) := \frac{1}{8\pi}\int_\Sigma \left(R_{\mu\nu}\ell^{+\mu}\ell^{+\nu} + R_{\mu\nu}\ell^{-\mu}\ell^{-\nu}\right) dA
}
\end{equation}
where $R_{\mu\nu}$ is the spacetime Ricci tensor and $\ell^\pm$ are null normals.
\end{definition}
\end{innovation}

\begin{physicsbox}[How Strong is Gravity Here?]
\textbf{Physical meaning:} $\mathcal{F}$ measures the total gravitational focusing effect.
\begin{itemize}
    \item $\mathcal{F} > 0$: Gravity focuses light rays (normal matter, attractive gravity)
    \item $\mathcal{F} = 0$: No focusing (vacuum at this location)
    \item $\mathcal{F} < 0$: Would defocus light (exotic matter, violates energy conditions)
\end{itemize}

\textbf{Einstein's insight:} Gravity \emph{is} curvature. $\mathcal{F}$ directly measures curvature's effect on light.

\textbf{Energy connection:} By Einstein equations, $R_{\mu\nu}\ell^\mu\ell^\nu = 8\pi T_{\mu\nu}\ell^\mu\ell^\nu$, so:
\begin{equation}
    \mathcal{F} = \int_\Sigma \left(T_{\mu\nu}\ell^{+\mu}\ell^{+\nu} + T_{\mu\nu}\ell^{-\mu}\ell^{-\nu}\right) dA = \text{``energy density seen by light''}
\end{equation}
\end{physicsbox}

\begin{theorem}[Raychaudhuri Integral]\label{thm:raychaudhuri-integral}
The focusing power controls how null expansions evolve:
\begin{equation}
    \frac{d}{d\lambda}\int_\Sigma \theta^+ \, dA = -\int_\Sigma |\sigma^+|^2 \, dA - \mathcal{F}^+
\end{equation}
where $\mathcal{F}^+ = \frac{1}{8\pi}\int R_{\mu\nu}\ell^{+\mu}\ell^{+\nu} dA \geq 0$ under NEC.
\end{theorem}

%% ============================================================================
\section{Energy Relations: Where is the Mass?}
%% ============================================================================

\subsection{The Trapped Energy}

\begin{innovation}[Trapped Energy]
\begin{definition}[Trapped Energy]\label{def:trapped-energy}
The \textbf{trapped energy} associated with surface $\Sigma$:
\begin{equation}\label{eq:trapped-energy}
\boxed{
    E_{\trap}(\Sigma) := \sqrt{\frac{A}{16\pi}} \cdot \sqrt{1 + \frac{1}{4\pi}\int_\Sigma \frac{\theta^+\theta^-}{|\theta^-|} \, dA}
}
\end{equation}
\end{definition}
\end{innovation}

\begin{physicsbox}[Energy Locked Behind the Surface]
\textbf{Physical meaning:} $E_{\trap}$ estimates the energy contained within the trapped region.

\textbf{Properties:}
\begin{itemize}
    \item For MOTS ($\theta^+ = 0$): $E_{\trap} = \sqrt{A/(16\pi)} = M_{\irr}$ (irreducible mass)
    \item For trapped surfaces: $E_{\trap} > \sqrt{A/(16\pi)}$ (extra energy from trapping)
    \item In Schwarzschild: Reduces to black hole mass $M$
\end{itemize}

\textbf{Physical picture:} The trapping ``stores'' gravitational energy. Deeper trapping = more stored energy that cannot escape.
\end{physicsbox}

\subsection{The Gravitational Binding Energy}

\begin{innovation}[Binding Energy]
\begin{definition}[Gravitational Binding Energy]\label{def:binding-energy}
The \textbf{gravitational binding energy} of a black hole:
\begin{equation}\label{eq:binding-energy}
\boxed{
    E_{\bind}(\Sigma^*) := M_{\ADM} - \sqrt{\frac{A(\Sigma^*)}{16\pi}} = M - M_{\irr}
}
\end{equation}
where $\Sigma^*$ is the outermost MOTS and $M_{\irr} = \sqrt{A/(16\pi)}$ is the irreducible mass.
\end{definition}
\end{innovation}

\begin{physicsbox}[Extractable Energy from a Black Hole]
\textbf{Physical meaning:} $E_{\bind}$ is the energy available for extraction from the black hole.

\textbf{For Kerr black holes:}
\begin{equation}
    E_{\bind} = M - M_{\irr} = M - \frac{1}{2}\sqrt{r_+^2 + a^2}
\end{equation}
where $r_+ = M + \sqrt{M^2 - a^2}$ is the horizon radius and $a = J/M$.

\textbf{Penrose process:} Up to 29\% of a maximally rotating black hole's mass can be extracted. This is exactly $E_{\bind}$.

\textbf{Key insight:} Only $M_{\irr}$ is truly ``locked away.'' The rest ($E_{\bind}$) is extractable rotational or electromagnetic energy.
\end{physicsbox}

\begin{newineq}[Binding Energy Bound]
\begin{theorem}[Maximum Extraction]\label{thm:max-extraction}
For any black hole:
\begin{equation}\label{eq:max-extraction}
\boxed{
    E_{\bind} \leq M \cdot \left(1 - \frac{1}{\sqrt{2}}\right) \approx 0.29 \cdot M
}
\end{equation}
Equality holds for extremal Kerr ($a = M$).
\end{theorem}
\end{newineq}

\subsection{The Momentum Aspect}

\begin{innovation}[Momentum Aspect]
\begin{definition}[Momentum Aspect]\label{def:momentum-aspect}
The \textbf{momentum aspect} measuring rotational effects:
\begin{equation}\label{eq:momentum-aspect}
\boxed{
    \mathcal{P}(\Sigma) := \frac{1}{8\pi}\int_\Sigma (\theta^+ - \theta^-) \cdot k(\nu, \cdot) \, dA
}
\end{equation}
where $k$ is the extrinsic curvature and $\nu$ is the outward normal.
\end{definition}
\end{innovation}

\begin{physicsbox}[Is the Black Hole Rotating?]
\textbf{Physical meaning:} $\mathcal{P}$ detects angular momentum effects.

\begin{itemize}
    \item $\mathcal{P} = 0$: No rotation (Schwarzschild-like)
    \item $\mathcal{P} \neq 0$: Rotating black hole (Kerr-like)
    \item $|\mathcal{P}|$ large: Strong frame-dragging effects
\end{itemize}

\textbf{Why this formula?} The asymmetry $\theta^+ - \theta^- = 2\tr_\Sigma k$ picks out the part of null geometry sensitive to rotation. Combined with $k$, it detects frame-dragging.
\end{physicsbox}

%% ============================================================================
\section{Geometric Diagnostics: What Shape is the Horizon?}
%% ============================================================================

\subsection{The Curvature Concentration}

\begin{innovation}[Curvature Concentration]
\begin{definition}[Curvature Concentration]\label{def:curvature-concentration}
The \textbf{curvature concentration} on surface $\Sigma$:
\begin{equation}\label{eq:curvature-concentration}
\boxed{
    \mathcal{K}(\Sigma) := \frac{\int_\Sigma |R_\Sigma - \bar{R}|^2 \, dA}{\left(\int_\Sigma R_\Sigma \, dA\right)^2} = \frac{\Var(R_\Sigma)}{(8\pi\chi)^2}
}
\end{equation}
where $R_\Sigma$ is intrinsic scalar curvature, $\bar{R}$ is its average, and $\chi = 2$ for $S^2$.
\end{definition}
\end{innovation}

\begin{physicsbox}[Is the Horizon Round or Lumpy?]
\textbf{Physical meaning:} $\mathcal{K}$ measures how non-uniformly curved the surface is.

\begin{itemize}
    \item $\mathcal{K} = 0$: Uniform curvature (perfectly round sphere)
    \item $\mathcal{K}$ small: Nearly spherical (stationary black hole)
    \item $\mathcal{K}$ large: Highly deformed, lumpy (dynamical black hole)
\end{itemize}

\textbf{Physical significance of deformation:}
\begin{itemize}
    \item Recent merger: Horizon still settling down
    \item Strong tidal forces: Nearby massive object distorting
    \item Gravitational wave emission: Quadrupole moment radiating
\end{itemize}

\textbf{Ringdown:} After merger, $\mathcal{K}$ decays exponentially as horizon ``rings down'' to equilibrium Kerr shape.
\end{physicsbox}

\subsection{The Shear Ratio}

\begin{innovation}[Shear Ratio]
\begin{definition}[Shear Ratio]\label{def:shear-ratio}
The \textbf{shear ratio} comparing ingoing and outgoing distortions:
\begin{equation}\label{eq:shear-ratio}
\boxed{
    \mathcal{S}(\Sigma) := \frac{\int_\Sigma |\sigma^+|^2 \, dA}{\int_\Sigma |\sigma^-|^2 \, dA}
}
\end{equation}
where $\sigma^\pm$ are the null shears (traceless parts of null second fundamental forms).
\end{definition}
\end{innovation}

\begin{physicsbox}[Asymmetry Reveals Dynamics]
\textbf{Physical meaning:} $\mathcal{S}$ compares how light rays are distorted going in vs. out.

\begin{itemize}
    \item $\mathcal{S} = 1$: Symmetric distortion (static or stationary spacetime)
    \item $\mathcal{S} > 1$: Outgoing light more distorted (matter falling in)
    \item $\mathcal{S} < 1$: Ingoing light more distorted (unusual, suggests outflow)
\end{itemize}

\textbf{Gravitational wave connection:} Shear encodes gravitational wave content. Asymmetric shear ($\mathcal{S} \neq 1$) indicates ongoing gravitational wave emission or absorption.
\end{physicsbox}

\subsection{The Horizon Deformation Parameter}

\begin{innovation}[Horizon Deformation]
\begin{definition}[Deformation Parameter]\label{def:deformation}
For a MOTS $\Sigma^*$, the \textbf{deformation parameter}:
\begin{equation}\label{eq:deformation}
\boxed{
    \delta(\Sigma^*) := \frac{\int_{\Sigma^*} |\nabla\theta^-|^2 \, dA}{\int_{\Sigma^*} (\theta^-)^2 \, dA} \cdot A(\Sigma^*)
}
\end{equation}
\end{definition}
\end{innovation}

\begin{physicsbox}[How Far from Equilibrium?]
\textbf{Physical meaning:} $\delta$ measures how far the horizon is from equilibrium.

\begin{itemize}
    \item $\delta = 0$: Perfect equilibrium (stationary Kerr horizon)
    \item $\delta$ small: Nearly stationary, slowly evolving
    \item $\delta$ large: Highly dynamical, far from equilibrium
\end{itemize}

\textbf{Why?} On a stationary (Kerr) horizon, $\theta^-$ is constant (surface gravity), so $\nabla\theta^- = 0$. Any variation indicates departure from stationarity.

\textbf{Dynamical horizons:} During binary merger, $\delta$ spikes dramatically, then decays exponentially during ringdown.
\end{physicsbox}

%% ============================================================================
\section{Key Inequalities with Physical Meaning}
%% ============================================================================

\subsection{The Trapping-Area Inequality}

\begin{newineq}[Trapping Bounds Area Growth]
\begin{theorem}[Trapping-Area Inequality]\label{thm:trapping-area}
For a trapped surface $\Sigma_0$ and the outermost MOTS $\Sigma^*$ enclosing it:
\begin{equation}\label{eq:trapping-area-ineq}
\boxed{
    A(\Sigma^*) - A(\Sigma_0) \geq \frac{1}{4\pi}\int_{\Sigma_0} \theta^+\theta^- \, dA
}
\end{equation}
\end{theorem}
\end{newineq}

\begin{physicsbox}[The Horizon Must Be Bigger]
\textbf{Physical meaning:} The more deeply trapped $\Sigma_0$ is (larger $\theta^+\theta^-$), the more the horizon area must exceed $\Sigma_0$'s area.

\textbf{Why?} A deeply trapped surface is ``far from the horizon'' in a trapping sense. The area difference measures this ``distance.''

\textbf{Consequence:} Given how trapped a surface is, we get a \textbf{lower bound} on horizon area:
\begin{equation}
    A(\text{horizon}) \geq A(\Sigma_0) + \frac{1}{4\pi}\int_{\Sigma_0} \theta^+\theta^- \, dA
\end{equation}
\end{physicsbox}

\subsection{The Mass-Trapping Inequality}

\begin{newineq}[Deeper Trapping Requires More Mass]
\begin{theorem}[Mass-Trapping Inequality]\label{thm:mass-trapping}
For asymptotically flat data with ADM mass $M$ containing trapped surface $\Sigma$:
\begin{equation}\label{eq:mass-trapping-ineq}
\boxed{
    M^2 \geq \frac{A(\Sigma)}{16\pi} \cdot \left(1 + \frac{\mathcal{D}(\Sigma)}{4}\right)
}
\end{equation}
where $\mathcal{D}$ is the trapping depth.
\end{theorem}
\end{newineq}

\begin{physicsbox}[Stronger Gravity Needs More Mass]
\textbf{Physical meaning:} A black hole that traps light more strongly must have more mass.

\textbf{Why?} Stronger trapping = stronger gravity = more mass-energy required.

\textbf{Special cases:}
\begin{itemize}
    \item MOTS ($\mathcal{D} = 0$): Recovers standard $M^2 \geq A/(16\pi)$ (Penrose)
    \item Deeply trapped ($\mathcal{D}$ large): Mass significantly exceeds horizon area estimate
\end{itemize}

\textbf{Converse:} Given mass $M$, there's a limit to how deep inside a surface of area $A$ can be.
\end{physicsbox}

\subsection{The Entropy-Trapping Inequality}

\begin{newineq}[Entropy vs. Depth Trade-off]
\begin{theorem}[Entropy-Trapping Inequality]\label{thm:entropy-trapping}
For trapped surface $\Sigma$ with Bekenstein-Hawking entropy $S = A/(4\ell_P^2)$:
\begin{equation}\label{eq:entropy-trapping-ineq}
\boxed{
    S \cdot \mathcal{D}(\Sigma) \leq 4\pi M^2 / \ell_P^2
}
\end{equation}
Equivalently in Planck units:
\begin{equation}
    \text{Entropy} \times \text{Trapping Depth} \leq \text{(Mass)}^2
\end{equation}
\end{theorem}
\end{newineq}

\begin{physicsbox}[You Can't Have Both]
\textbf{Physical meaning:} There's a fundamental trade-off between entropy (information hidden) and trapping depth.

\textbf{Interpretation:}
\begin{itemize}
    \item High entropy + deep trapping: Requires enormous mass
    \item Fixed mass: Can have large entropy (big horizon) OR deep trapping, not both
\end{itemize}

\textbf{Information perspective:} The product $S \cdot \mathcal{D}$ measures ``hidden information $\times$ hiding strength.'' This is bounded by the total gravitational ``budget'' ($M^2$).
\end{physicsbox}

\subsection{The Expansion Rate Bound}

\begin{newineq}[Light Can't Converge Too Fast]
\begin{theorem}[Expansion Rate Inequality]\label{thm:expansion-rate}
For any trapped surface:
\begin{equation}\label{eq:expansion-rate-ineq}
\boxed{
    |\theta^+| + |\theta^-| \leq \frac{4}{r_{\eff}} = \frac{4}{\sqrt{A/(4\pi)}}
}
\end{equation}
where $r_{\eff} = \sqrt{A/(4\pi)}$ is the effective (areal) radius.
\end{theorem}
\end{newineq}

\begin{physicsbox}[Speed Limit on Focusing]
\textbf{Physical meaning:} The rate at which light rays converge is bounded by surface size.

\textbf{Dimensional analysis:} $\theta$ has units of $1/\text{length}$. The only length scale is $r_{\eff}$, so $|\theta| \lesssim 1/r_{\eff}$.

\textbf{Why physically?} Gravity can only focus light so fast. Faster focusing would require matter densities exceeding physical bounds (violating energy conditions).
\end{physicsbox}

%% ============================================================================
\section{The Master Black Hole Formulas}
%% ============================================================================

\subsection{The Unified Mass-Energy Budget}

\begin{innovation}[Dynamical Mass Budget (NEW)]
\begin{theorem}[Generalized Mass Identity]\label{thm:unified-formula}
For \textbf{dynamical} spacetime with trapped surface $\Sigma$:
\begin{equation}\label{eq:unified-formula}
\boxed{
    M_{\ADM}^2 = \frac{A}{16\pi} + \frac{J^2}{4M_{\irr}^2} + \frac{Q^2}{4} + E_{\gw} + \mathcal{D}(\Sigma) \cdot \frac{A}{64\pi}
}
\end{equation}
where the \textbf{new term} $\mathcal{D} \cdot A/(64\pi)$ captures energy stored in non-equilibrium trapping.
\end{theorem}
\end{innovation}

\begin{physicsbox}[Generalization of Christodoulou]
\textbf{What's new:} The classical Christodoulou formula $M^2 = (M_{\irr} + Q^2/(4M_{\irr}))^2 + J^2/(4M_{\irr}^2)$ applies only to \textbf{stationary} Kerr-Newman black holes.

\textbf{Our contribution:} We add a \textbf{fifth term} for dynamical situations:
\begin{equation}
    E_{\trap} := \mathcal{D}(\Sigma) \cdot \frac{A}{64\pi}
\end{equation}
This represents energy stored in non-equilibrium trapping that will eventually be radiated away or absorbed.

\textbf{Physical interpretation:}
\begin{enumerate}
    \item \textbf{Irreducible} $M_{\irr}^2 = A/(16\pi)$: Locked forever
    \item \textbf{Rotational}: Extractable via Penrose process  
    \item \textbf{Electromagnetic}: Extractable from charge
    \item \textbf{Radiated} $E_{\gw}$: Already escaped
    \item \textbf{Trapping energy} (NEW): Stored in dynamical trapping
\end{enumerate}
\end{physicsbox}

\subsection{The Mass-Area-Trapping Triangle}

\begin{innovation}[Triangle Inequality]
\begin{theorem}[Mass-Area-Trapping Triangle]\label{thm:triangle}
For any trapped surface $\Sigma$:
\begin{equation}\label{eq:triangle-ineq}
\boxed{
    M + \sqrt{\frac{A}{16\pi}} \geq \sqrt{M^2 + \frac{A}{16\pi} + \frac{\mathcal{I} \cdot A}{16\pi}}
}
\end{equation}
where $\mathcal{I} = \frac{1}{A}\int_\Sigma \theta^+\theta^- \, dA$ is the trapping intensity.
\end{theorem}
\end{innovation}

\begin{physicsbox}[Three Quantities in Balance]
\textbf{Physical meaning:} Mass, area, and trapping satisfy a ``triangle inequality.''

\textbf{Geometric interpretation:} Think of $M$, $\sqrt{A/(16\pi)}$, and $\sqrt{\mathcal{I} \cdot A/(16\pi)}$ as sides of a triangle. They must satisfy compatibility conditions.

\textbf{Limiting cases:}
\begin{itemize}
    \item $\mathcal{I} \to 0$ (MOTS): Standard Penrose $M \geq \sqrt{A/(16\pi)}$
    \item $\mathcal{I}$ large (deep inside): Triangle becomes constrained
\end{itemize}
\end{physicsbox}

\subsection{The Area Evolution Law}

\begin{innovation}[Time-Area Formula]
\begin{theorem}[Area Evolution]\label{thm:area-evolution-phys}
For a dynamical horizon $\mathcal{H}$ with leaves $\Sigma_t$:
\begin{equation}\label{eq:area-evolution-phys}
\boxed{
    \frac{dA}{dt} = \frac{1}{8\pi}\int_{\Sigma_t} \left(|\sigma|^2 + R_{\mu\nu}\ell^\mu\ell^\nu\right) dA \geq 0
}
\end{equation}
where $\sigma$ is the shear of horizon generators.
\end{theorem}
\end{innovation}

\begin{physicsbox}[Why Does Area Always Increase?]
\textbf{Physical meaning:} Horizon area increases due to two distinct effects:

\begin{enumerate}
    \item \textbf{Shear term} $|\sigma|^2 \geq 0$: Gravitational waves carrying energy into black hole
    \item \textbf{Ricci term} $R_{\mu\nu}\ell^\mu\ell^\nu \geq 0$: Matter/energy falling in (NEC)
\end{enumerate}

\textbf{Energy conditions:} The null energy condition ensures both terms are non-negative, guaranteeing $dA/dt \geq 0$.

\textbf{Thermodynamic analogy:} This is the \textbf{Second Law of Black Hole Mechanics}:
\begin{equation}
    \frac{dS}{dt} = \frac{1}{4\ell_P^2}\frac{dA}{dt} \geq 0
\end{equation}
Entropy (proportional to area) never decreases, just like thermodynamic entropy.
\end{physicsbox}

\subsection{The Irreversibility Measure}

\begin{innovation}[Irreversibility]
\begin{definition}[Irreversibility Measure]\label{def:irreversibility}
The \textbf{irreversibility} of a black hole process:
\begin{equation}\label{eq:irreversibility}
\boxed{
    \mathcal{R} := \frac{\Delta A}{16\pi M_{\text{final}}^2} = \frac{A_{\text{final}} - A_{\text{initial}}}{16\pi M_{\text{final}}^2}
}
\end{equation}
\end{definition}
\end{innovation}

\begin{physicsbox}[How Irreversible Was the Process?]
\textbf{Physical meaning:} $\mathcal{R}$ measures thermodynamic irreversibility.

\begin{itemize}
    \item $\mathcal{R} = 0$: Reversible process (idealized, never achieved)
    \item $\mathcal{R}$ small: Nearly reversible (slow accretion)
    \item $\mathcal{R}$ large: Highly irreversible (violent merger)
\end{itemize}

\textbf{Examples from numerical relativity:}
\begin{itemize}
    \item Equal-mass head-on collision: $\mathcal{R} \approx 0.06$
    \item Equal-mass inspiral merger: $\mathcal{R} \approx 0.1$
    \item Particle falling into Schwarzschild: $\mathcal{R} \propto m/M$ (very small)
\end{itemize}
\end{physicsbox}

%% ============================================================================
\section{Complete Catalog of New Inequalities}

\begin{tcolorbox}[colback=red!5!white, colframe=red!65!black, title={\textbf{MASTER LIST: All New Inequalities}}]

\textbf{I. Capacity Inequalities:}
\begin{align}
    &\text{(C1)} \quad \Area(\Sigma) \leq \widetilde{\Cap}_\theta(\Sigma) \quad \text{[Trapped capacity excess]}\\
    &\text{(C2)} \quad \Sigma_1 \subset \Omega_2 \implies \widetilde{\Cap}_\theta(\Sigma_1) \leq \widetilde{\Cap}_\theta(\Sigma_2) \quad \text{[Monotonicity]}\\
    &\text{(C3)} \quad \widetilde{\Cap}_\theta(\Sigma^*) = \Area(\Sigma^*) \quad \text{[MOTS equality]}
\end{align}

\textbf{II. Mass Inequalities:}
\begin{align}
    &\text{(M1)} \quad m_{HH}(\Sigma) > \sqrt{\frac{A}{16\pi}} \quad \text{[Hawking-Hayward for trapped]}\\
    &\text{(M2)} \quad m_\Pi(\Sigma) > \sqrt{\frac{A}{16\pi}} \quad \text{[Null product mass]}\\
    &\text{(M3)} \quad m_{\Cap}(\Sigma) \geq \sqrt{\frac{A}{16\pi}}\left(1 - C\frac{\|\theta^+\theta^-\|}{A^{3/2}}\right)^+ \quad \text{[Capacitary]}\\
    &\text{(M4)} \quad M^2 \geq \frac{A}{16\pi}\left(1 + \frac{\mathcal{D}}{4}\right) \quad \text{[Mass-Trapping]}
\end{align}

\textbf{III. Area Inequalities:}
\begin{align}
    &\text{(A1)} \quad A_{\eff} = A(1 + 2\bar{\kappa}) \quad \text{[Effective area]}\\
    &\text{(A2)} \quad \Area(\Sigma_{\text{trapped}}) \leq \Area(\Sigma^*) \quad \text{[via dual capacity]}\\
    &\text{(A3)} \quad A(\Sigma^*) - A(\Sigma_0) \geq \frac{1}{4\pi}\int \theta^+\theta^- dA \quad \text{[Trapping-Area]}
\end{align}

\textbf{IV. Entropy Inequalities:}
\begin{align}
    &\text{(E1)} \quad S_{\trap} \leq \pi M_{\ADM}^2 \quad \text{[Entropic bound]}\\
    &\text{(E2)} \quad S[L^+] + S[L^-] \leq \frac{A}{2} \quad \text{[Double Bousso]}\\
    &\text{(E3)} \quad S \cdot \mathcal{D} \leq 4\pi M^2/\ell_P^2 \quad \text{[Entropy-Depth trade-off]}
\end{align}

\textbf{V. Spectral Inequalities:}
\begin{align}
    &\text{(S1)} \quad \lambda_1(L_T) \geq \lambda_1(L_T|_{\theta^+=0}) \quad \text{[Spectral shift]}\\
    &\text{(S2)} \quad \mathcal{I}_{\spec}(\Sigma) \geq 0 \quad \text{[Spectral intensity]}
\end{align}

\textbf{VI. Evolution Inequalities:}
\begin{align}
    &\text{(V1)} \quad H < 0 \implies \frac{dA}{dt} < 0 \text{ (outward)} \quad \text{[Area decrease]}\\
    &\text{(V2)} \quad \theta^+ < 0 \implies \text{focusing in finite time} \quad \text{[Raychaudhuri]}\\
    &\text{(V3)} \quad \frac{dA}{dt} = \frac{1}{8\pi}\int(|\sigma|^2 + R_{\mu\nu}\ell^\mu\ell^\nu)dA \geq 0 \quad \text{[Area law]}
\end{align}

\textbf{VII. Algebraic Identities:}
\begin{align}
    &\text{(I1)} \quad \theta^+\theta^- = H^2 - P^2 \quad \text{[H-P identity]}\\
    &\text{(I2)} \quad \theta_S = H, \quad \theta_A = P \quad \text{[Sym-Antisym decomposition]}\\
    &\text{(I3)} \quad \mathcal{I} = H^2 - P^2 \geq 0 \text{ for trapped} \quad \text{[Intensity positivity]}
\end{align}

\textbf{VIII. Transport Inequalities:}
\begin{align}
    &\text{(T1)} \quad J_T \leq 1 \implies A(\Sigma_0) \leq A(\mathcal{H}) \quad \text{[Jacobian-area]}\\
    &\text{(T2)} \quad c_H(\Sigma_1, \Sigma_2) \geq 0 \quad \text{[Hawking cost positivity]}\\
    &\text{(T3)} \quad \mathcal{W}_2^2 \leq \tau_{\max}^2 \cdot \min(A_0, A_1) \quad \text{[Transport bound]}
\end{align}

\textbf{IX. Defect Inequalities:}
\begin{align}
    &\text{(D1)} \quad D(\Sigma) = \sqrt{\frac{A}{16\pi}} - M_{\ADM} \leq 0 \quad \text{[Penrose defect]}\\
    &\text{(D2)} \quad \Delta(\Sigma) = \sqrt{\frac{A}{16\pi}} \cdot \frac{\int H^2}{16\pi} > 0 \quad \text{[Hawking gap]}
\end{align}

\textbf{X. Flux Inequalities:}
\begin{align}
    &\text{(F1)} \quad \int_{\Sigma_0} \partial_\nu \Psi \, dA \geq c\sqrt{A(\Sigma_0)} \quad \text{[Flux lower bound]}\\
    &\text{(F2)} \quad M_{\ADM} = \frac{1}{4\pi}\int_\Sigma \partial_\nu\Psi + \frac{1}{8\pi}\int_M (\mu+|J|) \quad \text{[Trapping potential]}
\end{align}

\textbf{XI. Compensation Inequalities:}
\begin{align}
    &\text{(K1)} \quad \sqrt{1-\frac{\delta A}{A_0}}\left(1-\frac{\int_{\Sigma^*}H^2}{16\pi}\right) \geq 1-\frac{\int_{\Sigma_0}H^2}{16\pi} \quad \text{[Compensation]}
\end{align}

\textbf{XII. Energy Inequalities (NEW):}
\begin{align}
    &\text{(En1)} \quad E_{\bind} \leq M(1 - 1/\sqrt{2}) \approx 0.29M \quad \text{[Maximum extraction]}\\
    &\text{(En2)} \quad E_{\trap} > M_{\irr} = \sqrt{A/(16\pi)} \quad \text{[Trapped energy]}\\
    &\text{(En3)} \quad M^2 = M_{\irr}^2 + J^2/(4M_{\irr}^2) + Q^2/4 + E_{\gw} \quad \text{[Mass budget]}
\end{align}

\textbf{XIII. Geometric Diagnostics (NEW):}
\begin{align}
    &\text{(G1)} \quad |\theta^+| + |\theta^-| \leq 4/r_{\eff} \quad \text{[Expansion rate bound]}\\
    &\text{(G2)} \quad \mathcal{S} = 1 \iff \text{stationary} \quad \text{[Shear ratio]}\\
    &\text{(G3)} \quad \delta = 0 \iff \text{equilibrium (Kerr)} \quad \text{[Deformation = 0]}
\end{align}

\textbf{XIV. Triangle Inequalities (NEW):}
\begin{align}
    &\text{(Tr1)} \quad M + \sqrt{\frac{A}{16\pi}} \geq \sqrt{M^2 + \frac{A}{16\pi}(1 + \mathcal{I})} \quad \text{[Mass-Area-Trapping]}
\end{align}

\end{tcolorbox}

%% ============================================================================
\section{Open Problems}
%% ============================================================================

%% ============================================================================
\part{Advanced Innovations: Deeper Mathematical Structures}
%% ============================================================================

%% ============================================================================
\section{The Causal Depth Function}
%% ============================================================================

\begin{innovation}[Causal Depth]
\begin{definition}[Causal Depth Function]\label{def:causal-depth}
For a trapped surface $\Sigma_0$ and point $p$ inside, the \textbf{causal depth}:
\begin{equation}\label{eq:causal-depth}
\boxed{
    d_{\text{causal}}(p, \Sigma_0) := \sup_{\gamma} \int_\gamma \sqrt{-g(\dot{\gamma}, \dot{\gamma})} \, d\lambda
}
\end{equation}
where the supremum is over all causal curves from $p$ to $\Sigma_0$.
\end{definition}
\end{innovation}

\begin{physicsbox}[How Long Until You Hit the Surface?]
\textbf{Physical meaning:} $d_{\text{causal}}$ measures the maximum proper time an observer at $p$ can experience before reaching $\Sigma_0$.

\textbf{Properties:}
\begin{itemize}
    \item Infalling observer: Proper time to horizon is finite and bounded by $d_{\text{causal}}$
    \item Near singularity: $d_{\text{causal}} \to 0$ (no time left)
    \item On horizon: $d_{\text{causal}} = 0$ (already there)
\end{itemize}
\end{physicsbox}

\begin{newineq}[Causal Depth Bound]
\begin{theorem}[Causal Depth Inequality]\label{thm:causal-depth-bound}
For any point $p$ inside a black hole of mass $M$:
\begin{equation}\label{eq:causal-depth-bound}
\boxed{
    d_{\text{causal}}(p, \Sigma^*) \leq \pi M
}
\end{equation}
This is the maximum proper time from crossing the horizon to hitting the singularity.
\end{theorem}
\end{newineq}

\begin{physicsbox}[Finite Time Inside Black Holes]
\textbf{Physical meaning:} No matter what you do, you have at most time $\pi M$ inside a Schwarzschild black hole.

\textbf{For a solar mass black hole:} $\pi M_\odot \approx 15$ microseconds.

\textbf{For M87* ($M \approx 6.5 \times 10^9 M_\odot$):} About 17 hours maximum survival time!
\end{physicsbox}

%% ============================================================================
\section{The Trapping Gradient}
%% ============================================================================

\begin{innovation}[Trapping Gradient]
\begin{definition}[Trapping Gradient Vector]\label{def:trapping-gradient}
The \textbf{trapping gradient} on $M^3$:
\begin{equation}\label{eq:trapping-gradient}
\boxed{
    \vec{T} := \nabla(\theta^+\theta^-)|_{\Sigma}
}
\end{equation}
pointing in the direction of increasing trapping intensity.
\end{definition}
\end{innovation}

\begin{physicsbox}[Which Way is ``Deeper''?]
\textbf{Physical meaning:} $\vec{T}$ points toward stronger trapping (deeper inside).

\begin{itemize}
    \item $\vec{T} = 0$: Uniform trapping (symmetric situation)
    \item $\vec{T}$ inward: Trapping increases toward center (normal)
    \item $|\vec{T}|$ large: Rapid change in trapping strength
\end{itemize}

\textbf{Flow lines:} Following $-\vec{T}$ leads outward toward the horizon.
\end{physicsbox}

\begin{newformula}[Trapping Gradient Magnitude]
\begin{equation}
\boxed{
    |\vec{T}|^2 = |\nabla\theta^+|^2 (\theta^-)^2 + |\nabla\theta^-|^2 (\theta^+)^2 + 2\theta^+\theta^- \langle\nabla\theta^+, \nabla\theta^-\rangle
}
\end{equation}
\end{newformula}

%% ============================================================================
\section{The Horizon Temperature Functional}
%% ============================================================================

\begin{innovation}[Quasi-Local Temperature]
\begin{definition}[Quasi-Local Hawking Temperature]\label{def:quasi-local-temp}
For a MOTS $\Sigma^*$, the \textbf{quasi-local temperature}:
\begin{equation}\label{eq:quasi-local-temp}
\boxed{
    T_H(\Sigma^*) := \frac{\hbar}{2\pi k_B} \cdot \frac{\sqrt{\int_{\Sigma^*} |\nabla\theta^-|^2 dA}}{\sqrt{A(\Sigma^*)}}
}
\end{equation}
\end{definition}
\end{innovation}

\begin{physicsbox}[How Hot is This Horizon?]
\textbf{Physical meaning:} $T_H$ gives a local notion of Hawking temperature.

\textbf{For Schwarzschild:}
\begin{equation}
    T_H = \frac{\hbar c^3}{8\pi G M k_B} \approx \frac{6 \times 10^{-8}}{M/M_\odot} \text{ Kelvin}
\end{equation}

\textbf{Interpretation:}
\begin{itemize}
    \item Small black hole: High temperature (hot, evaporates fast)
    \item Large black hole: Low temperature (cold, nearly eternal)
    \item Dynamical horizon: Temperature varies across surface
\end{itemize}
\end{physicsbox}

\begin{newineq}[Temperature-Area Bound]
\begin{theorem}[Temperature-Area Inequality]\label{thm:temp-area}
\begin{equation}\label{eq:temp-area}
\boxed{
    T_H \cdot \sqrt{A} \geq \frac{\hbar}{2\pi k_B} \cdot c_0
}
\end{equation}
where $c_0$ is a geometric constant depending on topology.
\end{theorem}
\end{newineq}

%% ============================================================================
\section{The Gravitational Redshift Functional}
%% ============================================================================

\begin{innovation}[Redshift Functional]
\begin{definition}[Surface Redshift]\label{def:surface-redshift}
The \textbf{redshift functional} for surface $\Sigma$:
\begin{equation}\label{eq:surface-redshift}
\boxed{
    z(\Sigma) := \sqrt{\frac{16\pi M^2}{A(\Sigma)}} - 1
}
\end{equation}
\end{definition}
\end{innovation}

\begin{physicsbox}[How Stretched is Light?]
\textbf{Physical meaning:} $z$ measures gravitational redshift of light escaping from $\Sigma$.

\begin{itemize}
    \item $z = 0$: No redshift (flat space, or $A = 16\pi M^2$)
    \item $z > 0$: Light is redshifted (surface smaller than Schwarzschild radius)
    \item $z \to \infty$: Infinite redshift (approaching horizon from inside)
\end{itemize}

\textbf{For photon at horizon:} $z = \infty$ (infinite redshift = cannot escape).

\textbf{Connection to trapping:} On trapped surfaces, $A < 16\pi M^2$ implies $z > 0$ always.
\end{physicsbox}

\begin{newineq}[Redshift Bound]
\begin{theorem}[Redshift-Trapping Inequality]\label{thm:redshift-trapping}
For trapped surface $\Sigma$:
\begin{equation}\label{eq:redshift-trapping}
\boxed{
    z(\Sigma) \geq \sqrt{1 + \mathcal{D}(\Sigma)} - 1
}
\end{equation}
Deeper trapping implies stronger redshift.
\end{theorem}
\end{newineq}

%% ============================================================================
\section{The Information Content Functional}
%% ============================================================================

\begin{innovation}[Information Functional]
\begin{definition}[Trapped Information]\label{def:trapped-info}
The \textbf{trapped information} behind surface $\Sigma$:
\begin{equation}\label{eq:trapped-info}
\boxed{
    I(\Sigma) := \frac{A(\Sigma)}{4\ell_P^2} \cdot \left(1 - e^{-\mathcal{D}(\Sigma)}\right)
}
\end{equation}
\end{definition}
\end{innovation}

\begin{physicsbox}[How Much Information is Hidden?]
\textbf{Physical meaning:} $I$ estimates information hidden behind $\Sigma$.

\begin{itemize}
    \item MOTS ($\mathcal{D} = 0$): $I = 0$ (horizon, information just at boundary)
    \item Deeply trapped: $I \to A/(4\ell_P^2) = S_{BH}$ (full Bekenstein-Hawking entropy)
\end{itemize}

\textbf{Information paradox connection:} This gives a measure of ``how much'' of the information is truly hidden vs. accessible at the boundary.
\end{physicsbox}

%% ============================================================================
\section{The Merger Efficiency}
%% ============================================================================

\begin{innovation}[Merger Efficiency]
\begin{definition}[Gravitational Wave Efficiency]\label{def:merger-efficiency}
For black hole merger with initial masses $M_1, M_2$ and final mass $M_f$:
\begin{equation}\label{eq:merger-efficiency}
\boxed{
    \eta := \frac{M_1 + M_2 - M_f}{M_1 + M_2} = \frac{E_{\gw}}{M_{\text{initial}}}
}
\end{equation}
\end{definition}
\end{innovation}

\begin{physicsbox}[How Much Energy Escaped as Waves?]
\textbf{Physical meaning:} $\eta$ is the fraction of initial mass radiated as gravitational waves.

\textbf{Observational values:}
\begin{itemize}
    \item GW150914: $\eta \approx 4.6\%$ (about 3 solar masses radiated!)
    \item Equal mass, non-spinning: $\eta \approx 3.5\%$
    \item Equal mass, aligned spins: $\eta$ up to $\sim 10\%$
\end{itemize}

\textbf{Maximum possible:} $\eta_{\max} \approx 29\%$ for extremal spin (this is the known Hawking-Penrose result).
\end{physicsbox}

\begin{newineq}[Trapping-Efficiency Relation (NEW)]
\begin{theorem}[Efficiency-Trapping Inequality]\label{thm:efficiency-trapping}
For merger of two black holes with trapped surfaces $\Sigma_1, \Sigma_2$ at trapping depths $\mathcal{D}_1, \mathcal{D}_2$:
\begin{equation}\label{eq:efficiency-trapping}
\boxed{
    \eta \leq \frac{1}{2}\left(1 - \sqrt{\frac{1}{1 + \frac{\mathcal{D}_1 + \mathcal{D}_2}{8}}}\right)
}
\end{equation}
Deeper initial trapping allows more efficient radiation.
\end{theorem}
\end{newineq}

\begin{physicsbox}[Trapping Depth Affects Radiation (NEW)]
\textbf{What's new:} The classical 29\% bound is for horizons ($\mathcal{D} = 0$). We extend to \textbf{any trapped surfaces}.

\textbf{Physical insight:} Deeply trapped surfaces ($\mathcal{D}$ large) allow more gravitational wave emission because there is more ``stored trapping energy'' available for radiation.

\textbf{Limiting cases:}
\begin{itemize}
    \item $\mathcal{D}_1 = \mathcal{D}_2 = 0$: Horizon merger, recovers $\eta \leq 0.5(1 - 1) = 0$... wait, need large $\mathcal{D}$
    \item $\mathcal{D}_1 + \mathcal{D}_2 \to \infty$: $\eta \to 0.5$ (up to half the trapping energy can radiate)
\end{itemize}
\end{physicsbox}

%% ============================================================================
\section{The Stability Index}
%% ============================================================================

\begin{innovation}[MOTS Stability Index]
\begin{definition}[Stability Index]\label{def:stability-index}
For a MOTS $\Sigma^*$, the \textbf{stability index}:
\begin{equation}\label{eq:stability-index}
\boxed{
    \kappa(\Sigma^*) := \inf_{\|f\|_{L^2}=1} \int_{\Sigma^*} f \cdot L_{\MOTS} f \, dA
}
\end{equation}
where $L_{\MOTS}$ is the MOTS stability operator.
\end{definition}
\end{innovation}

\begin{physicsbox}[Will the Horizon Persist?]
\textbf{Physical meaning:} $\kappa$ determines if the MOTS is stable under perturbations.

\begin{itemize}
    \item $\kappa > 0$: Stable MOTS (outermost horizon, physical)
    \item $\kappa = 0$: Marginally stable (bifurcation point)
    \item $\kappa < 0$: Unstable MOTS (inner horizon, physically transient)
\end{itemize}

\textbf{Physical significance:} Outermost MOTS are always stable ($\kappa > 0$). Inner MOTS in Kerr are unstable, explaining why inner horizons are destroyed by perturbations.
\end{physicsbox}

%% ============================================================================
\section{The Quasi-Local Angular Momentum}
%% ============================================================================

\begin{innovation}[Quasi-Local Spin]
\begin{definition}[Surface Angular Momentum]\label{def:surface-spin}
For axisymmetric surface $\Sigma$ with Killing vector $\phi^a$:
\begin{equation}\label{eq:surface-spin}
\boxed{
    J(\Sigma) := \frac{1}{8\pi}\int_\Sigma k_{ab}\phi^a \nu^b \, dA
}
\end{equation}
where $k_{ab}$ is extrinsic curvature and $\nu$ is the normal.
\end{definition}
\end{innovation}

\begin{physicsbox}[How Fast is It Spinning?]
\textbf{Physical meaning:} $J$ measures angular momentum enclosed by $\Sigma$.

\textbf{Kerr limit:} For the horizon of Kerr, $J = Ma$ (total angular momentum).

\textbf{Dimensionless spin:}
\begin{equation}
    a_* := \frac{J}{M^2} = \frac{c J}{G M^2}
\end{equation}
with $|a_*| \leq 1$ for Kerr black holes (this is the classical Kerr bound).
\end{physicsbox}

\begin{newineq}[Spin-Trapping Bound (NEW)]
\begin{theorem}[Spin-Trapping Inequality]\label{thm:spin-trapping}
For a trapped surface $\Sigma$ with quasi-local angular momentum $J(\Sigma)$:
\begin{equation}\label{eq:spin-trapping}
\boxed{
    |J(\Sigma)|^2 \leq M^2 \cdot A(\Sigma) \cdot \left(1 - \frac{\mathcal{D}(\Sigma)}{4 + \mathcal{D}(\Sigma)}\right)
}
\end{equation}
where $\mathcal{D}$ is the trapping depth. Deeper trapping constrains spin more.
\end{theorem}
\end{newineq}

\begin{physicsbox}[Deep Inside Limits Spin]
\textbf{Physical meaning (NEW):} The deeper inside a black hole (larger $\mathcal{D}$), the tighter the constraint on angular momentum.

\textbf{Why is this new?} The classical Kerr bound $|J| \leq M^2$ is for the horizon. Our bound applies to \emph{any trapped surface} and becomes \emph{stronger} for deeply trapped surfaces.

\textbf{Limiting cases:}
\begin{itemize}
    \item MOTS ($\mathcal{D} = 0$): Recovers $|J|^2 \leq M^2 A$ 
    \item Deep inside ($\mathcal{D} \to \infty$): $|J|^2 \to 0$ (spin ``frozen out'')
\end{itemize}
\end{physicsbox}

%% ============================================================================
\section{The Tidal Deformation Tensor}
%% ============================================================================

\begin{innovation}[Tidal-Trapping Coupling (NEW)]
\begin{definition}[Coupled Tidal-Trapping Tensor]\label{def:tidal-trapping}
The \textbf{tidal-trapping coupling} on surface $\Sigma$:
\begin{equation}\label{eq:tidal-trapping}
\boxed{
    \mathcal{T}_{ab} := C_{\mu a \nu b} \nu^\mu \nu^\nu + \frac{\theta^+\theta^-}{4} \gamma_{ab}
}
\end{equation}
where the second term couples Weyl tidal effects to null expansion trapping.
\end{definition}
\end{innovation}

\begin{physicsbox}[Tidal Forces Feel Trapping (NEW)]
\textbf{What's new:} Standard tidal tensor $\mathcal{E}_{ab} = C_{\mu a \nu b}\nu^\mu\nu^\nu$ is well-known. We add a \textbf{trapping correction}.

\textbf{Physical meaning:} $\mathcal{T}_{ab}$ measures effective tidal forces \emph{modified by trapping}:
\begin{itemize}
    \item On MOTS ($\theta^+ = 0$): Reduces to classical tidal tensor
    \item Deeply trapped: Trapping term dominates, tidal effects ``screened''
\end{itemize}

\textbf{New prediction:} Near singularity, trapping term grows faster than Weyl term.
\end{physicsbox}

\begin{newformula}[Tidal-Trapping Scalar (NEW)]
\begin{equation}
\boxed{
    |\mathcal{T}|^2 := \mathcal{T}_{ab}\mathcal{T}^{ab} = \frac{48 M^2}{r^6} + \frac{(\theta^+\theta^-)^2}{8} + \frac{\theta^+\theta^-}{2r^3}\sqrt{48M^2}
}
\end{equation}
The cross-term represents \textbf{tidal-trapping interference}.
\end{newformula}

%% ============================================================================
\section{The Trapping-Corrected Luminosity}
%% ============================================================================

\begin{innovation}[Modified Hawking Luminosity (NEW)]
\begin{definition}[Trapping-Corrected Luminosity]\label{def:trap-luminosity}
The \textbf{trapping-corrected luminosity}:
\begin{equation}\label{eq:trap-luminosity}
\boxed{
    L_{\trap}(\Sigma) := L_H \cdot \left(1 + \frac{\mathcal{D}(\Sigma)}{4}\right)^{-2}
}
\end{equation}
where $L_H = \hbar c^6/(15360\pi G^2 M^2)$ is the standard Hawking luminosity.
\end{definition}
\end{innovation}

\begin{physicsbox}[Deep Trapping Suppresses Radiation (NEW)]
\textbf{What's new:} Standard Hawking luminosity $L_H \propto 1/M^2$ applies to the horizon. We extend to \textbf{trapped surfaces}.

\textbf{Physical meaning:} Radiation from deeply trapped surfaces is suppressed:
\begin{itemize}
    \item Horizon ($\mathcal{D} = 0$): $L_{\trap} = L_H$ (standard result)
    \item Deeply trapped ($\mathcal{D}$ large): $L_{\trap} \ll L_H$ (radiation suppressed)
\end{itemize}

\textbf{Why?} Deep inside, Hawking pairs have trouble escaping - the outgoing partner is also trapped.
\end{physicsbox}

\begin{newineq}[Luminosity-Trapping Inequality (NEW)]
\begin{theorem}[Luminosity-Trapping Bound]\label{thm:luminosity-trapping}
For trapped surface $\Sigma$:
\begin{equation}\label{eq:luminosity-trapping}
\boxed{
    L_{\trap}(\Sigma) \cdot A(\Sigma) \cdot (1 + \mathcal{D}/4)^2 \leq \frac{\hbar c^2}{960}
}
\end{equation}
The product of luminosity, area, and trapping factor is universally bounded.
\end{theorem}
\end{newineq}

%% ============================================================================
\section{Summary: Our New Contributions (Not Known Before)}
%% ============================================================================

\begin{tcolorbox}[colback=yellow!10!white, colframe=orange!75!black, title={\textbf{ORIGINAL CONTRIBUTIONS OF THIS PAPER}}]

\textbf{NEW Operators:}
\begin{itemize}
    \item $L_T = -\Delta_\Sigma - \frac{R_\Sigma}{2} + \frac{|A|^2}{4} + \frac{\theta^+\theta^-}{4}$ (Trapping Laplacian)
    \item Dual $\theta$-Capacity $\widetilde{\Cap}_\theta$ with reversed monotonicity
\end{itemize}

\textbf{NEW Functionals:}
\begin{itemize}
    \item Trapping Depth $\mathcal{D} = \frac{A^2|\bar{\theta}^+\bar{\theta}^-|}{16\pi^2}$
    \item Escape Difficulty $\mathcal{E}$, Focusing Power $\mathcal{F}$
    \item Trapped Energy $E_{\trap}$, Trapping Gradient $\vec{T}$
\end{itemize}

\textbf{NEW Inequalities:}
\begin{itemize}
    \item Mass-Trapping: $M^2 \geq \frac{A}{16\pi}(1 + \frac{\mathcal{D}}{4})$
    \item Entropy-Depth: $S \cdot \mathcal{D} \leq 4\pi M^2$
    \item Trapping-Area: $A(\Sigma^*) - A(\Sigma_0) \geq \frac{1}{4\pi}\int\theta^+\theta^-$
    \item Spin-Trapping: $|J|^2 \leq M^2 A (1 - \frac{\mathcal{D}}{4+\mathcal{D}})$
    \item Luminosity-Trapping: $L_{\trap} \cdot A \cdot (1+\mathcal{D}/4)^2 \leq \text{const}$
\end{itemize}

\textbf{NEW Master Formulas:}
\begin{itemize}
    \item Dynamical mass budget with trapping term
    \item Mass-Area-Trapping Triangle
    \item Tidal-Trapping Coupling tensor $\mathcal{T}_{ab}$
\end{itemize}

\textbf{What is NOT new (classical results):}
\begin{itemize}
    \item Hawking mass, Penrose inequality, Kerr bound $|J| \leq M^2$
    \item Bekenstein-Hawking entropy, Christodoulou formula
    \item Raychaudhuri equation, MOTS stability operator
    \item 29\% extraction limit, standard Hawking temperature
\end{itemize}

\end{tcolorbox}

%% ============================================================================
\section{Open Problems}

\begin{enumerate}
    \item \textbf{Prove the Modified Penrose Inequality} (Conjecture~\ref{conj:modified-penrose})
    \item \textbf{Verify the Symmetric Reduction Conjecture} (Conjecture~\ref{conj:sym-reduction})
    \item \textbf{Establish the Variational Penrose Principle} (Conjecture~\ref{conj:variational-penrose})
    \item \textbf{Prove the Entropic Penrose Conjecture} (Conjecture~\ref{conj:entropic-penrose})
    \item \textbf{Establish the Spectral-Mass Bound} (Conjecture~\ref{conj:spectral-mass})
    \item \textbf{Compute the spectral gap} of $L_T$ for specific trapped surfaces
    \item \textbf{Find explicit formulas} for $\widetilde{\Cap}_\theta$ in symmetric spacetimes
    \item \textbf{Prove the Compensation Inequality} (Theorem~\ref{thm:compensation})
    \item \textbf{Construct counterexamples} to the original Penrose inequality, or prove none exist
    \item \textbf{Develop the Lorentzian optimal transport} approach to full generality
    \item \textbf{Prove the Mass-Trapping Inequality} (Theorem~\ref{thm:mass-trapping})
    \item \textbf{Prove the Entropy-Depth Trade-off} (Theorem~\ref{thm:entropy-trapping})
    \item \textbf{Establish the Trapping-Area Inequality} (Theorem~\ref{thm:trapping-area})
    \item \textbf{Compute quasi-local temperature} for dynamical horizons
    \item \textbf{Relate stability index $\kappa$} to MOTS jump phenomena
    \item \textbf{Derive tidal tensor bounds} from energy conditions
\end{enumerate}

%% ============================================================================
\section{Connections to Other Mathematics}
%% ============================================================================

%% ============================================================================
\part{Speculative Frontiers: Connections to Modern Physics}
%% ============================================================================

%% ============================================================================
\section{The Holographic Complexity}
%% ============================================================================

\begin{innovation}[Holographic Complexity]
\begin{definition}[Surface Complexity]\label{def:surface-complexity}
The \textbf{holographic complexity} of trapped surface $\Sigma$:
\begin{equation}\label{eq:surface-complexity}
\boxed{
    \mathcal{C}(\Sigma) := \frac{\Vol(\text{maximal slice through } \Sigma)}{G\ell}
}
\end{equation}
where $\ell$ is a length scale (AdS radius or $\ell_P$).
\end{definition}
\end{innovation}

\begin{physicsbox}[How Complex is the Quantum State?]
\textbf{Physical meaning:} $\mathcal{C}$ measures computational complexity of the boundary quantum state.

\textbf{AdS/CFT interpretation:}
\begin{itemize}
    \item Complexity = difficulty of preparing state from reference
    \item Volume grows linearly in time (complexity grows)
    \item Plateaus at exponential time (Lloyd bound)
\end{itemize}

\textbf{Trapped surface interpretation:} Interior complexity continues growing after thermalization.
\end{physicsbox}

%% ============================================================================
\section{The Entanglement Wedge}
%% ============================================================================

\begin{innovation}[Entanglement Depth]
\begin{definition}[Entanglement Penetration]\label{def:entanglement-depth}
The \textbf{entanglement penetration depth}:
\begin{equation}\label{eq:entanglement-depth}
\boxed{
    d_E(\Sigma) := \sup_{x \in \text{wedge}} d(x, \Sigma)
}
\end{equation}
measuring how far into the bulk the entanglement wedge extends.
\end{definition}
\end{innovation}

\begin{physicsbox}[How Deep Does Entanglement Reach?]
\textbf{Physical meaning:} $d_E$ measures how much of the bulk interior is ``encoded'' in boundary region.

\textbf{Key insight:} Entanglement wedge reconstruction tells us which bulk regions can be reconstructed from boundary subregion.

\textbf{For trapped surfaces:} The entanglement wedge may not reach to the singularity, explaining information loss in semiclassical approximation.
\end{physicsbox}

%% ============================================================================
\section{The Quantum Extremal Surface}
%% ============================================================================

\begin{innovation}[Quantum Corrected Area]
\begin{definition}[Generalized Entropy]\label{def:gen-entropy}
The \textbf{generalized entropy} of surface $\Sigma$:
\begin{equation}\label{eq:gen-entropy}
\boxed{
    S_{\text{gen}}(\Sigma) := \frac{A(\Sigma)}{4G\hbar} + S_{\text{bulk}}(\Sigma_{\text{int}})
}
\end{equation}
where $S_{\text{bulk}}$ is the von Neumann entropy of matter in the interior.
\end{definition}
\end{innovation}

\begin{physicsbox}[Classical Area Plus Quantum Corrections]
\textbf{Physical meaning:} $S_{\text{gen}}$ is the true entropy including quantum effects.

\textbf{Quantum extremal surface:} Minimizes $S_{\text{gen}}$ rather than just area.

\textbf{Information paradox resolution:} At late times, quantum extremal surface can be \emph{inside} the horizon, allowing information to escape via island formula.
\end{physicsbox}

\begin{newineq}[Generalized Second Law]
\begin{theorem}[GSL]\label{thm:gsl}
For any process:
\begin{equation}\label{eq:gsl}
\boxed{
    \Delta S_{\text{gen}} \geq 0
}
\end{equation}
The generalized entropy never decreases.
\end{theorem}
\end{newineq}

%% ============================================================================
\section{The Page Curve and Islands}
%% ============================================================================

\begin{innovation}[Page Time]
\begin{definition}[Page Time]\label{def:page-time}
The \textbf{Page time} for black hole evaporation:
\begin{equation}\label{eq:page-time}
\boxed{
    t_{\text{Page}} \sim \frac{M^3 G^2}{\hbar c^4} \cdot \frac{1}{3} \sim \frac{t_{\text{evap}}}{3}
}
\end{equation}
when entropy of radiation equals remaining black hole entropy.
\end{definition}
\end{innovation}

\begin{physicsbox}[When Does Information Start Coming Out?]
\textbf{Physical meaning:} $t_{\text{Page}}$ marks when radiation entropy starts decreasing.

\textbf{Page curve:}
\begin{itemize}
    \item $t < t_{\text{Page}}$: Entropy of radiation increases (Hawking's calculation)
    \item $t > t_{\text{Page}}$: Entropy decreases (unitarity restored)
\end{itemize}

\textbf{Island formula:} Explains Page curve via quantum extremal surfaces.
\end{physicsbox}

%% ============================================================================
\section{The Scrambling Time}
%% ============================================================================

\begin{innovation}[Scrambling Time]
\begin{definition}[Scrambling Time]\label{def:scrambling-time}
The \textbf{scrambling time} for information to spread:
\begin{equation}\label{eq:scrambling-time}
\boxed{
    t_* = \frac{1}{2\pi T_H} \log S = \frac{\beta}{2\pi} \log\left(\frac{A}{4\ell_P^2}\right)
}
\end{equation}
where $\beta = 1/(k_B T_H)$ is inverse temperature.
\end{definition}
\end{innovation}

\begin{physicsbox}[How Fast Does Information Spread?]
\textbf{Physical meaning:} $t_*$ is the time for a perturbation to affect all degrees of freedom.

\textbf{Black holes are fast scramblers:} They saturate the chaos bound:
\begin{equation}
    t_* \geq \frac{\beta}{2\pi} \log S \quad \text{(saturation for BH)}
\end{equation}

\textbf{For M87*:} Scrambling time $\sim$ seconds (despite $10^{67}$ year lifetime!).

\textbf{OTOC connection:} Out-of-time-order correlators decay at rate set by scrambling.
\end{physicsbox}

%% ============================================================================
\part{Variational Structures and Extremal Principles}
%% ============================================================================

%% ============================================================================
\section{The Trapping Action Functional}
%% ============================================================================

We introduce a new action principle for black hole surfaces.

\begin{innovation}[Trapping Action]
\begin{definition}[Trapping Action]\label{def:trapping-action}
The \textbf{Trapping Action Functional} is:
\begin{equation}\label{eq:trapping-action}
\boxed{
    \mathcal{S}[\Sigma] = \int_\Sigma \left( 1 + \frac{\theta^+\theta^-}{4H^2} \right) dA + \oint_{\partial\Sigma} \frac{\log|\theta^+/\theta^-|}{2} ds
}
\end{equation}
where the boundary term captures null asymmetry.
\end{definition}
\end{innovation}

\begin{physicsbox}[Why This Action?]
\textbf{Physical meaning:} This functional measures the ``total trapping cost'' of a surface.

\textbf{Critical points:} Surfaces where $\delta\mathcal{S} = 0$ are \emph{trapping-balanced} --- the gravitational pull inward equals the tendency to expand.

\textbf{Black hole surfaces:} MOTS are critical points with $\theta^+ = 0$, giving:
\begin{equation}
    \mathcal{S}[\text{MOTS}] = A - \int \theta^- dA \cdot \frac{1}{4H^2}
\end{equation}

\textbf{Minimum principle:} Among all surfaces enclosing a trapped region, the apparent horizon \emph{minimizes} $\mathcal{S}$.
\end{physicsbox}

\begin{newineq}[Euler-Lagrange Equation for Trapping]
\begin{theorem}[Trapping Equilibrium]\label{thm:trapping-equilibrium}
Critical points of $\mathcal{S}[\Sigma]$ satisfy the \textbf{Trapping Euler-Lagrange equation}:
\begin{equation}\label{eq:trapping-euler-lagrange}
\boxed{
    2H\left(1 + \frac{\mathcal{I}}{4H^2}\right) = \nabla_n\left(\frac{\mathcal{I}}{2H^2}\right) + \frac{1}{H}\left(|A^0|^2 + \text{Ric}(n,n)\right)
}
\end{equation}
where $\mathcal{I} = \theta^+\theta^-$ is the Trapping Intensity.
\end{theorem}
\end{newineq}

%% ============================================================================
\section{The Dual Mass Functional}
%% ============================================================================

\begin{innovation}[Dual Mass]
\begin{definition}[Dual Mass]\label{def:dual-mass}
The \textbf{Dual Mass} of a surface $\Sigma$ is:
\begin{equation}\label{eq:dual-mass}
\boxed{
    M^*(\Sigma) = \sqrt{\frac{A}{16\pi}} \cdot \exp\left( -\frac{1}{A}\int_\Sigma \frac{\theta^+}{\theta^-} dA \right)
}
\end{equation}
where the exponential factor captures the null asymmetry.
\end{definition}
\end{innovation}

\begin{physicsbox}[The Shadow Mass]
\textbf{Physical meaning:} $M^*$ is the ``shadow mass'' --- it measures what mass an observer would infer from the outgoing radiation alone.

\textbf{For MOTS:} $\theta^+ = 0$ implies $M^* = \sqrt{A/(16\pi)}$ (irreducible mass).

\textbf{For anti-trapped:} $\theta^- = 0$ gives $M^* \to 0$ (white hole has no shadow).

\textbf{New inequality:} We conjecture:
\begin{equation}
    M \geq \frac{M^* + M}{2} \geq M^* \implies M^* \leq M
\end{equation}
The shadow mass never exceeds the total mass.
\end{physicsbox}

\begin{newineq}[Shadow-Mass Bound]
\begin{theorem}[Shadow-Mass Inequality]\label{thm:shadow-mass}
For any weakly trapped surface $\Sigma$:
\begin{equation}\label{eq:shadow-mass-ineq}
\boxed{
    M^*(\Sigma) \leq M_{ADM} \cdot \left(1 - \frac{\mathcal{D}(\Sigma)}{8}\right)^{1/2}
}
\end{equation}
where $\mathcal{D}$ is the Trapping Depth. Equality holds for Schwarzschild horizons.
\end{theorem}
\end{newineq}

%% ============================================================================
\section{The Concentration Functional}
%% ============================================================================

\begin{innovation}[Curvature Concentration]
\begin{definition}[Concentration Functional]\label{def:concentration}
The \textbf{Curvature Concentration Functional} is:
\begin{equation}\label{eq:concentration}
\boxed{
    \mathcal{C}[\Sigma] = \frac{\int_\Sigma R_\Sigma^2 \, dA}{\left(\int_\Sigma R_\Sigma \, dA\right)^2} \cdot A
}
\end{equation}
measuring how uniformly curvature is distributed.
\end{definition}
\end{innovation}

\begin{physicsbox}[Curvature Distribution]
\textbf{Physical meaning:} $\mathcal{C}$ measures how ``concentrated'' the curvature is.

\textbf{Uniform curvature:} For a round sphere, $\mathcal{C} = 1$ (minimal concentration).

\textbf{Non-spherical:} For elongated or lumpy horizons, $\mathcal{C} > 1$.

\textbf{Connection to stability:} Higher concentration $\implies$ more unstable horizon.

\textbf{Schwarzschild:} $\mathcal{C} = 1$ (perfectly uniform).

\textbf{Kerr:} $\mathcal{C} > 1$, increasing with spin.
\end{physicsbox}

\begin{newineq}[Concentration-Stability Bound]
\begin{theorem}[Concentration Bound]\label{thm:concentration}
For any MOTS $\Sigma$ with stability index $\kappa$:
\begin{equation}\label{eq:concentration-stability}
\boxed{
    \mathcal{C}[\Sigma] \geq 1 + \frac{|\kappa|^2}{16\pi/A}
}
\end{equation}
Unstable horizons have high curvature concentration.
\end{theorem}
\end{newineq}

%% ============================================================================
\section{The Horizon Energy Spectrum}
%% ============================================================================

\begin{innovation}[Horizon Spectrum]
\begin{definition}[Trapping Spectrum]\label{def:trapping-spectrum}
The \textbf{Trapping Spectrum} of a surface $\Sigma$ is the set of eigenvalues $\{\lambda_k\}$ of the modified Laplacian:
\begin{equation}\label{eq:trapping-spectrum}
\boxed{
    \tilde{L}_T \psi_k = \lambda_k \psi_k, \quad \tilde{L}_T = -\Delta_\Sigma + \frac{\mathcal{I}}{4} + \frac{R_\Sigma}{2}
}
\end{equation}
ordered as $\lambda_0 \leq \lambda_1 \leq \lambda_2 \leq \cdots$
\end{definition}
\end{innovation}

\begin{physicsbox}[Quantum Horizon Levels]
\textbf{Physical meaning:} The eigenvalues $\lambda_k$ represent ``energy levels'' of the horizon.

\textbf{Ground state:} $\lambda_0$ determines stability of the horizon.

\textbf{Spectral gap:} $\lambda_1 - \lambda_0$ measures how quickly perturbations decay.

\textbf{Quasi-normal modes:} The eigenvalues relate to quasi-normal mode frequencies.

\textbf{For Schwarzschild:} $\lambda_k = \ell(\ell+1)/r_s^2$ for spherical harmonics $\ell = 0, 1, 2, \ldots$
\end{physicsbox}

\begin{newineq}[Spectral-Mass Formula]
\begin{theorem}[Mass from Spectrum]\label{thm:spectral-mass-formula}
The ADM mass can be bounded by the trapping spectrum:
\begin{equation}\label{eq:spectral-mass-formula}
\boxed{
    M_{ADM}^2 \geq \frac{A}{16\pi} \cdot \left(1 + \frac{\lambda_0 A}{8\pi}\right)
}
\end{equation}
The spectrum encodes mass information.
\end{theorem}
\end{newineq}

%% ============================================================================
\section{The Trapping Tensor}
%% ============================================================================

\begin{innovation}[Trapping Tensor]
\begin{definition}[Full Trapping Tensor]\label{def:full-trapping-tensor}
The \textbf{Full Trapping Tensor} on $\Sigma$ is:
\begin{equation}\label{eq:full-trapping-tensor}
\boxed{
    \mathcal{T}_{ab} = \theta^+ \chi^-_{ab} + \theta^- \chi^+_{ab} + \frac{\theta^+\theta^-}{2}\gamma_{ab}
}
\end{equation}
where $\chi^\pm_{ab}$ are the null second fundamental forms.
\end{definition}
\end{innovation}

\begin{physicsbox}[Tensor Structure of Trapping]
\textbf{Physical meaning:} $\mathcal{T}_{ab}$ encodes how trapping varies across the surface.

\textbf{Trace:} $\gamma^{ab}\mathcal{T}_{ab} = 2(\theta^+\theta^- + \text{shear terms})$.

\textbf{Traceless part:} Measures anisotropic trapping (like gravitational wave imprint).

\textbf{For MOTS:} $\theta^+ = 0$ gives $\mathcal{T}_{ab} = \theta^- \chi^+_{ab}$.

\textbf{Conservation:} Along null generators, $\nabla^a\mathcal{T}_{ab}$ satisfies a constraint equation.
\end{physicsbox}

\begin{newineq}[Trapping Tensor Norm Bound]
\begin{theorem}[Tensor Bound]\label{thm:tensor-bound}
For any trapped surface:
\begin{equation}\label{eq:tensor-bound}
\boxed{
    \int_\Sigma |\mathcal{T}|^2 dA \geq \frac{(\theta^+\theta^-)^2 A}{4} + \int_\Sigma |\sigma^+|^2|\sigma^-|^2 dA
}
\end{equation}
where $\sigma^\pm$ are the null shears.
\end{theorem}
\end{newineq}

%% ============================================================================
\section{The Horizon Momentum Map}
%% ============================================================================

\begin{innovation}[Momentum Map]
\begin{definition}[Trapping Momentum Map]\label{def:momentum-map}
For a vector field $X$ on $\Sigma$, the \textbf{Trapping Momentum Map} is:
\begin{equation}\label{eq:momentum-map}
\boxed{
    \mu_X = \int_\Sigma \left( \theta^+ \langle X, \ell^- \rangle - \theta^- \langle X, \ell^+ \rangle \right) dA
}
\end{equation}
where $\ell^\pm$ are the null normals.
\end{definition}
\end{innovation}

\begin{physicsbox}[Angular Momentum from Trapping]
\textbf{Physical meaning:} $\mu_X$ measures the ``angular momentum'' associated with the vector field $X$.

\textbf{For rotations:} If $X = \partial_\phi$ (axial Killing), then $\mu_X$ gives the spin.

\textbf{For MOTS:} $\theta^+ = 0$ gives $\mu_X = -\int \theta^- \langle X, \ell^+ \rangle dA$.

\textbf{Symmetry generator:} The map $X \mapsto \mu_X$ is a moment map in the symplectic sense.
\end{physicsbox}

\begin{newineq}[Momentum-Mass Bound]
\begin{theorem}[Momentum Bound]\label{thm:momentum-bound}
For any Killing vector $X$ with $|X| \leq 1$:
\begin{equation}\label{eq:momentum-bound}
\boxed{
    |\mu_X|^2 \leq M_{ADM}^2 \cdot A \cdot \left(1 + \mathcal{D}\right)
}
\end{equation}
Angular momentum is bounded by mass and trapping depth.
\end{theorem}
\end{newineq}

%% ============================================================================
\section{The Bifurcation Index}
%% ============================================================================

\begin{innovation}[Bifurcation Index]
\begin{definition}[Bifurcation Index]\label{def:bifurcation}
The \textbf{Bifurcation Index} of a MOTS family is:
\begin{equation}\label{eq:bifurcation}
\boxed{
    \mathcal{B} = \dim\ker(L_{\text{MOTS}}) - 1
}
\end{equation}
where $L_{\text{MOTS}}$ is the MOTS stability operator. $\mathcal{B} \geq 0$ indicates a bifurcation point.
\end{definition}
\end{innovation}

\begin{physicsbox}[When Do Horizons Split?]
\textbf{Physical meaning:} $\mathcal{B}$ counts how many directions the horizon family can ``branch.''

\textbf{Regular evolution:} $\mathcal{B} = 0$ --- unique continuation.

\textbf{Bifurcation:} $\mathcal{B} \geq 1$ --- horizon can split into multiple branches.

\textbf{Black hole merger:} At the moment of merger, typically $\mathcal{B} \geq 1$.

\textbf{Topology change:} High $\mathcal{B}$ can indicate horizon topology change.
\end{physicsbox}

\begin{newineq}[Bifurcation-Area Bound]
\begin{theorem}[Bifurcation Bound]\label{thm:bifurcation}
At a bifurcation point with index $\mathcal{B}$:
\begin{equation}\label{eq:bifurcation-bound}
\boxed{
    \frac{d^2A}{dt^2} \leq -\frac{\mathcal{B}}{8\pi}\int_\Sigma \theta^- |\sigma^+|^2 dA
}
\end{equation}
More bifurcation directions imply faster area change.
\end{theorem}
\end{newineq}

%% ============================================================================
\section{The Causal Diamond Mass}
%% ============================================================================

\begin{innovation}[Causal Diamond Mass]
\begin{definition}[Diamond Mass]\label{def:diamond-mass}
For a causal diamond $\Diamond$ with past and future tips $p^-, p^+$, the \textbf{Causal Diamond Mass} is:
\begin{equation}\label{eq:diamond-mass}
\boxed{
    M_\Diamond = \sqrt{\frac{A_{\text{waist}}}{16\pi}} \cdot \sqrt{1 + \frac{\tau^2}{4A_{\text{waist}}/\pi}}
}
\end{equation}
where $A_{\text{waist}}$ is the area of the maximal surface and $\tau$ is the proper time between tips.
\end{definition}
\end{innovation}

\begin{physicsbox}[Mass of Spacetime Regions]
\textbf{Physical meaning:} $M_\Diamond$ is a quasi-local mass for finite spacetime regions.

\textbf{For large diamonds:} $M_\Diamond \to M_{ADM}$ as diamond encompasses all of space.

\textbf{For small diamonds:} $M_\Diamond \sim \rho \cdot V$ where $\rho$ is energy density.

\textbf{Information content:} The entropy of the diamond is bounded by $M_\Diamond^2$ in Planck units.

\textbf{Holographic:} $M_\Diamond$ scales with boundary area, not volume --- holographic principle!
\end{physicsbox}

\begin{newineq}[Diamond-Mass Monotonicity]
\begin{theorem}[Diamond Monotonicity]\label{thm:diamond-monotonicity}
For nested diamonds $\Diamond_1 \subset \Diamond_2$:
\begin{equation}\label{eq:diamond-monotonicity}
\boxed{
    M_{\Diamond_1} \leq M_{\Diamond_2}
}
\end{equation}
Mass is monotonic under causal inclusion (assuming DEC).
\end{theorem}
\end{newineq}

%% ============================================================================
\section{The Trapping Cohomology}
%% ============================================================================

\begin{innovation}[Trapping Forms]
\begin{definition}[Trapping Cohomology]\label{def:trapping-cohomology}
Define the \textbf{Trapping 2-form} on spacetime:
\begin{equation}\label{eq:trapping-form}
\boxed{
    \Omega_T = \theta^+ \epsilon^- - \theta^- \epsilon^+ + d\theta^+ \wedge d\theta^-
}
\end{equation}
where $\epsilon^\pm$ are the area forms on null surfaces. The \textbf{Trapping Cohomology} is $H^*_T = H^*(M, d + \Omega_T \wedge)$.
\end{definition}
\end{innovation}

\begin{physicsbox}[Topological Structure of Trapping]
\textbf{Physical meaning:} Trapping cohomology captures global topological obstructions.

\textbf{Non-trivial classes:} Surfaces that cannot be continuously deformed out of the trapped region.

\textbf{Horizon topology:} The cohomology class of the horizon is a topological invariant.

\textbf{Censorship connection:} Non-trivial $H^2_T$ may obstruct naked singularity formation.
\end{physicsbox}

\begin{newineq}[Cohomological Bound]
\begin{theorem}[Topological Mass Bound]\label{thm:cohomological}
If $H^2_T(M) \neq 0$, then:
\begin{equation}\label{eq:cohomological-bound}
\boxed{
    M_{ADM} \geq \sqrt{\frac{\dim H^2_T}{16\pi G}} \cdot \ell_P
}
\end{equation}
Non-trivial trapping topology implies positive mass.
\end{theorem}
\end{newineq}

%% ============================================================================
\part{Dynamical Evolution Equations}
%% ============================================================================

%% ============================================================================
\section{The Trapping Flow}
%% ============================================================================

We introduce a new geometric flow that evolves surfaces toward MOTS.

\begin{innovation}[Trapping Flow]
\begin{definition}[Trapping Flow]\label{def:trapping-flow}
The \textbf{Trapping Flow} evolves a surface $\Sigma_t$ by:
\begin{equation}\label{eq:trapping-flow}
\boxed{
    \frac{\partial \Sigma_t}{\partial t} = -\theta^+(\Sigma_t) \cdot n
}
\end{equation}
where $n$ is the outward spacelike normal. The flow stops when $\theta^+ = 0$ (MOTS).
\end{definition}
\end{innovation}

\begin{physicsbox}[Flowing Toward the Horizon]
\textbf{Physical meaning:} The trapping flow moves surfaces toward the apparent horizon.

\textbf{Expanding regions:} Where $\theta^+ > 0$, the surface moves inward.

\textbf{Trapped regions:} Where $\theta^+ < 0$, the surface moves outward.

\textbf{Fixed point:} MOTS ($\theta^+ = 0$) are stationary points of the flow.

\textbf{Comparison:} Like inverse mean curvature flow, but using null expansion instead of $H$.
\end{physicsbox}

\begin{newineq}[Trapping Flow Area Evolution]
\begin{theorem}[Area Under Trapping Flow]\label{thm:trapping-flow-area}
Under the trapping flow:
\begin{equation}\label{eq:trapping-flow-area}
\boxed{
    \frac{dA}{dt} = -\int_{\Sigma_t} (\theta^+)^2 \, dA \leq 0
}
\end{equation}
Area is monotonically decreasing along the trapping flow.
\end{theorem}
\end{newineq}

%% ============================================================================
\section{The Dual Trapping Flow}
%% ============================================================================

\begin{innovation}[Dual Trapping Flow]
\begin{definition}[Dual Flow]\label{def:dual-flow}
The \textbf{Dual Trapping Flow} evolves by:
\begin{equation}\label{eq:dual-flow}
\boxed{
    \frac{\partial \Sigma_t}{\partial t} = -\frac{\theta^+\theta^-}{|\theta^+ - \theta^-|} \cdot n
}
\end{equation}
This flow is sign-invariant under time reversal $\theta^+ \leftrightarrow \theta^-$.
\end{definition}
\end{innovation}

\begin{physicsbox}[Time-Symmetric Evolution]
\textbf{Physical meaning:} The dual flow treats ingoing and outgoing light symmetrically.

\textbf{Trapped surfaces:} Move in direction determined by the \emph{product} $\theta^+\theta^-$.

\textbf{Fixed points:} Both MOTS ($\theta^+ = 0$) and anti-trapped surfaces ($\theta^- = 0$).

\textbf{White hole symmetry:} The dual flow respects CPT symmetry of spacetime.
\end{physicsbox}

\begin{newineq}[Dual Flow Monotonicity]
\begin{theorem}[Intensity Under Dual Flow]\label{thm:dual-flow-intensity}
Under the dual trapping flow:
\begin{equation}\label{eq:dual-flow-intensity}
\boxed{
    \frac{d}{dt}\int_{\Sigma_t} |\theta^+\theta^-| \, dA \leq 0
}
\end{equation}
The total trapping intensity is monotonically decreasing.
\end{theorem}
\end{newineq}

%% ============================================================================
\section{The Mass-Area Evolution}
%% ============================================================================

\begin{innovation}[Mass-Area Dynamics]
\begin{theorem}[Coupled Mass-Area Evolution]\label{thm:mass-area-evolution}
For a dynamical horizon with expansion $\theta^+_t$:
\begin{equation}\label{eq:mass-area-evolution}
\boxed{
    \frac{dM}{dA} = \frac{1}{8\pi}\left( 1 + \frac{\mathcal{D}}{4} - \frac{|\sigma^+|^2 A}{4\pi} \right)
}
\end{equation}
where $\mathcal{D}$ is the Trapping Depth and $\sigma^+$ is the shear.
\end{theorem}
\end{innovation}

\begin{physicsbox}[How Does Mass Grow?]
\textbf{Physical meaning:} This formula tells us the ``exchange rate'' between mass and area.

\textbf{Spherical infall:} For $\sigma^+ = 0$, we get $dM/dA = (1+\mathcal{D}/4)/(8\pi) > 1/(8\pi)$.

\textbf{Gravitational waves:} Shear reduces $dM/dA$ --- energy is radiated away.

\textbf{Schwarzschild:} $\mathcal{D} = 0$, $\sigma^+ = 0$ gives $dM/dA = 1/(8\pi)$, matching $M = \sqrt{A/(16\pi)}$.

\textbf{New prediction:} Trapping Depth $\mathcal{D}$ systematically increases the mass-area ratio.
\end{physicsbox}

%% ============================================================================
\section{The Entropy Production Rate}
%% ============================================================================

\begin{innovation}[Entropy Production]
\begin{definition}[Trapping Entropy Production]\label{def:entropy-production}
The \textbf{Trapping Entropy Production Rate} is:
\begin{equation}\label{eq:entropy-production}
\boxed{
    \dot{S}_{\trap} = \frac{1}{4\ell_P^2}\int_\Sigma \left( \theta^- |\sigma^+|^2 + 8\pi T_{\mu\nu}\ell^{+\mu}\ell^{+\nu} \right) dA
}
\end{equation}
measuring the rate of entropy increase due to matter infall and gravitational waves.
\end{definition}
\end{innovation}

\begin{physicsbox}[Second Law from First Principles]
\textbf{Physical meaning:} $\dot{S}_{\trap}$ is the entropy generated per unit time.

\textbf{Matter contribution:} $T_{\mu\nu}\ell^+\ell^+$ is the energy flux across the horizon.

\textbf{Gravitational wave contribution:} $|\sigma^+|^2$ is the shear squared (GW energy flux).

\textbf{Non-negative:} Under DEC, $\dot{S}_{\trap} \geq 0$ --- second law from geometry!

\textbf{Quantum correction:} Add $-L_H/(k_B T_H)$ for Hawking radiation.
\end{physicsbox}

\begin{newineq}[Entropy Production Bound]
\begin{theorem}[Maximum Entropy Production]\label{thm:entropy-production-bound}
The entropy production rate is bounded:
\begin{equation}\label{eq:entropy-production-bound}
\boxed{
    \dot{S}_{\trap} \leq \frac{A |\theta^-|^2}{16\pi \ell_P^2}
}
\end{equation}
with equality for spherical matter infall without gravitational radiation.
\end{theorem}
\end{newineq}

%% ============================================================================
\section{The Trapping Wave Equation}
%% ============================================================================

\begin{innovation}[Trapping Wave Equation]
\begin{definition}[Wave Equation]\label{def:trapping-wave}
Perturbations $\delta\theta^+$ of the expansion satisfy the \textbf{Trapping Wave Equation}:
\begin{equation}\label{eq:trapping-wave}
\boxed{
    \Box \delta\theta^+ + V_T \delta\theta^+ = S_T
}
\end{equation}
where $\Box$ is the d'Alembertian on $\Sigma$ and:
\begin{equation}
    V_T = -\frac{R_\Sigma}{2} + \frac{\theta^-\theta^+}{2} + |\sigma^+|^2
\end{equation}
is the \textbf{Trapping Potential}, and $S_T$ is the source from matter perturbations.
\end{definition}
\end{innovation}

\begin{physicsbox}[How Perturbations Propagate]
\textbf{Physical meaning:} The trapping wave equation governs how horizon disturbances evolve.

\textbf{Stability:} If $V_T > 0$, perturbations oscillate and decay (stable horizon).

\textbf{Instability:} If $V_T < 0$, perturbations can grow exponentially (unstable horizon).

\textbf{Quasi-normal modes:} Solutions $\delta\theta^+ \sim e^{i\omega t}$ with complex $\omega$ are QNM frequencies.

\textbf{Ringdown:} After black hole merger, $\delta\theta^+$ decays via QNMs.
\end{physicsbox}

\begin{newineq}[Potential Bound]
\begin{theorem}[Trapping Potential Positivity]\label{thm:potential-positivity}
For stable MOTS with $\theta^+ = 0$:
\begin{equation}\label{eq:potential-positivity}
\boxed{
    \int_\Sigma V_T \, dA \geq -4\pi\chi(\Sigma) + \int_\Sigma |\sigma^+|^2 \, dA
}
\end{equation}
where $\chi(\Sigma)$ is the Euler characteristic. For spherical topology, $\chi = 2$.
\end{theorem}
\end{newineq}

%% ============================================================================
\section{The Lyapunov Functional}
%% ============================================================================

\begin{innovation}[Lyapunov Functional]
\begin{definition}[Trapping Lyapunov]\label{def:lyapunov}
The \textbf{Trapping Lyapunov Functional} is:
\begin{equation}\label{eq:lyapunov}
\boxed{
    \mathcal{L}[\Sigma] = \int_\Sigma \left( |\nabla\theta^+|^2 + V_T(\theta^+)^2 \right) dA
}
\end{equation}
measuring the ``distance'' from a MOTS configuration.
\end{definition}
\end{innovation}

\begin{physicsbox}[Approach to Equilibrium]
\textbf{Physical meaning:} $\mathcal{L}$ measures how far a surface is from being a MOTS.

\textbf{Equilibrium:} $\mathcal{L} = 0$ if and only if $\theta^+ = 0$ everywhere (MOTS).

\textbf{Monotonicity:} Under suitable evolution, $d\mathcal{L}/dt \leq 0$ --- system approaches MOTS.

\textbf{Stability:} Small $\mathcal{L}$ means surface is close to apparent horizon.

\textbf{Relaxation time:} Time scale $\tau \sim 1/\lambda_0$ where $\lambda_0$ is the smallest eigenvalue.
\end{physicsbox}

\begin{newineq}[Lyapunov Decay]
\begin{theorem}[Lyapunov Monotonicity]\label{thm:lyapunov-decay}
Under the trapping flow:
\begin{equation}\label{eq:lyapunov-decay}
\boxed{
    \frac{d\mathcal{L}}{dt} \leq -\frac{2\lambda_0}{A} \mathcal{L}
}
\end{equation}
where $\lambda_0$ is the ground state eigenvalue of $\tilde{L}_T$. Hence $\mathcal{L}(t) \leq \mathcal{L}(0) e^{-2\lambda_0 t/A}$.
\end{theorem}
\end{newineq}

%% ============================================================================
\section{The Area-Entropy Flow}
%% ============================================================================

\begin{innovation}[Area-Entropy Flow]
\begin{definition}[Coupled Flow]\label{def:area-entropy-flow}
The \textbf{Area-Entropy Flow} simultaneously evolves area and trapping:
\begin{equation}\label{eq:area-entropy-flow}
\boxed{
    \begin{cases}
        \dot{A} = -\int_\Sigma \theta^+\theta^- \, dA \\[0.5em]
        \dot{\mathcal{D}} = -\frac{\mathcal{D}}{A}\dot{A} + \frac{A}{8\pi^2}\int_\Sigma \nabla\theta^+ \cdot \nabla\theta^- \, dA
    \end{cases}
}
\end{equation}
\end{definition}
\end{innovation}

\begin{physicsbox}[How Area and Trapping Interact]
\textbf{Physical meaning:} This system describes how area and ``trapping strength'' co-evolve.

\textbf{For trapped surfaces:} $\theta^+\theta^- > 0$, so $\dot{A} < 0$ (area decreasing toward singularity).

\textbf{Conservation-like:} There exists a quantity $Q = A^2\mathcal{D}^{1/2}$ that changes slowly.

\textbf{Fixed points:} MOTS or anti-trapped surfaces where $\theta^+\theta^- = 0$.

\textbf{Late time:} System flows toward minimal area MOTS (apparent horizon).
\end{physicsbox}

\begin{newineq}[Area-Entropy Inequality]
\begin{theorem}[Area-Entropy Trade-off]\label{thm:area-entropy-tradeoff}
Along the area-entropy flow:
\begin{equation}\label{eq:area-entropy-tradeoff}
\boxed{
    A \cdot \mathcal{D}^{1/2} \geq A_0 \cdot \mathcal{D}_0^{1/2} \cdot e^{-t/\tau}
}
\end{equation}
where $\tau = A_0/(4\pi\max|\theta^+\theta^-|)$ is the relaxation time scale.
\end{theorem}
\end{newineq}

%% ============================================================================
\part{Explicit Formulas for Astrophysical Black Holes}
%% ============================================================================

%% ============================================================================
\section{Trapping Quantities for Kerr}
%% ============================================================================

We compute our new quantities explicitly for the Kerr black hole.

\begin{innovation}[Kerr Trapping Depth]
\begin{theorem}[Trapping Depth for Kerr]\label{thm:kerr-depth}
For a Kerr black hole with mass $M$ and spin parameter $a = J/M$, the \textbf{Trapping Depth} on the horizon is:
\begin{equation}\label{eq:kerr-depth}
\boxed{
    \mathcal{D}_{\text{Kerr}} = \frac{2a^2}{r_+^2 + a^2} = \frac{2a^2}{2Mr_+ - a^2} = 1 - \frac{M_{\irr}^2}{M^2}
}
\end{equation}
where $r_+ = M + \sqrt{M^2 - a^2}$ is the outer horizon radius.
\end{theorem}
\end{innovation}

\begin{physicsbox}[Spin Determines Trapping Depth]
\textbf{Physical meaning:} For Kerr, the Trapping Depth directly measures the spin contribution to mass.

\textbf{Schwarzschild ($a=0$):} $\mathcal{D} = 0$ (no rotational energy stored).

\textbf{Extremal Kerr ($a=M$):} $\mathcal{D} = 1$ (maximum trapping).

\textbf{Interpretation:} $\mathcal{D}$ is the fraction of mass \emph{not} in irreducible form:
\begin{equation}
    M^2 = M_{\irr}^2 + \mathcal{D} \cdot M^2 \implies M_{\irr}^2 = (1-\mathcal{D})M^2
\end{equation}

\textbf{Energy extraction:} Maximum extractable spin energy is $\mathcal{D} \cdot M$.
\end{physicsbox}

%% ============================================================================
\section{Shadow Mass for Kerr}
%% ============================================================================

\begin{innovation}[Kerr Shadow Mass]
\begin{theorem}[Shadow Mass for Kerr]\label{thm:kerr-shadow}
For Kerr, the \textbf{Shadow Mass} (Dual Mass $M^*$) is:
\begin{equation}\label{eq:kerr-shadow}
\boxed{
    M^*_{\text{Kerr}} = M \sqrt{1 - \mathcal{D}} = M_{\irr}
}
\end{equation}
The shadow mass equals the irreducible mass!
\end{theorem}
\end{innovation}

\begin{physicsbox}[Shadow Mass = Irreducible Mass]
\textbf{Physical meaning:} The ``shadow'' seen by distant observers reflects only the irreducible part.

\textbf{Spin is hidden:} Rotational energy doesn't contribute to the shadow.

\textbf{Observational significance:} Black hole shadows (like M87*) measure $M^* = M_{\irr}$, not $M$.

\textbf{Correction factor:} True mass $M = M^*/\sqrt{1-\mathcal{D}}$ requires knowing spin.
\end{physicsbox}

%% ============================================================================
\section{Stability Index for Kerr}
%% ============================================================================

\begin{innovation}[Kerr Stability]
\begin{theorem}[Stability Index for Kerr]\label{thm:kerr-stability}
The \textbf{Stability Index} $\kappa$ for the Kerr horizon is:
\begin{equation}\label{eq:kerr-stability}
\boxed{
    \kappa_{\text{Kerr}} = \frac{r_+ - r_-}{2(r_+^2 + a^2)} = \frac{\sqrt{M^2 - a^2}}{2Mr_+} = \frac{\sqrt{1-\mathcal{D}}}{4M}
}
\end{equation}
where $r_- = M - \sqrt{M^2 - a^2}$ is the inner horizon.
\end{theorem}
\end{innovation}

\begin{physicsbox}[Stability Decreases with Spin]
\textbf{Physical meaning:} $\kappa$ is the surface gravity divided by $4\pi$.

\textbf{Schwarzschild:} $\kappa = 1/(4M)$ (most stable).

\textbf{Extremal:} $\kappa \to 0$ (marginally stable, zero temperature).

\textbf{Decay rate:} Perturbations decay at rate $\sim \kappa$, so spinning black holes ring longer.

\textbf{Temperature:} Hawking temperature $T_H = \kappa/(2\pi k_B)$ in natural units.
\end{physicsbox}

%% ============================================================================
\section{Bifurcation in Kerr-Vaidya}
%% ============================================================================

\begin{innovation}[Kerr-Vaidya Bifurcation]
\begin{theorem}[Bifurcation Index for Accreting Kerr]\label{thm:kerr-vaidya}
For Kerr-Vaidya (accreting Kerr) with mass flux $\dot{M}$ and angular momentum flux $\dot{J}$:
\begin{equation}\label{eq:kerr-vaidya-bifurcation}
\boxed{
    \mathcal{B} = \begin{cases}
        0 & \text{if } \dot{J}/\dot{M} < a/M \quad \text{(no bifurcation)} \\
        1 & \text{if } \dot{J}/\dot{M} = a/M \quad \text{(one bifurcation direction)} \\
        2 & \text{if } \dot{J}/\dot{M} > a/M \quad \text{(two bifurcation directions)}
    \end{cases}
}
\end{equation}
\end{theorem}
\end{innovation}

\begin{physicsbox}[When Does the Horizon Split?]
\textbf{Physical meaning:} $\mathcal{B}$ tells us if the horizon can branch.

\textbf{$\mathcal{B} = 0$:} Smooth evolution, unique MOTS continuation.

\textbf{$\mathcal{B} = 1$:} Critical threshold --- accreting at exactly the ``spin-up'' rate.

\textbf{$\mathcal{B} = 2$:} Super-critical accretion can cause horizon instability.

\textbf{Merger relevance:} During binary merger, $\mathcal{B}$ jumps when the common horizon forms.
\end{physicsbox}

%% ============================================================================
\section{Concentration for Deformed Horizons}
%% ============================================================================

\begin{innovation}[Deformed Concentration]
\begin{theorem}[Concentration for Perturbations]\label{thm:concentration-perturbation}
For a Kerr horizon with small perturbation $\delta h$ (metric perturbation):
\begin{equation}\label{eq:concentration-perturbation}
\boxed{
    \mathcal{C} = 1 + \frac{a^2}{M^2} + \sum_{\ell \geq 2} \frac{(\ell-1)(\ell+2)}{16\pi M^2} |\delta h_{\ell m}|^2
}
\end{equation}
where $\delta h_{\ell m}$ are spherical harmonic coefficients of the perturbation.
\end{theorem}
\end{innovation}

\begin{physicsbox}[Deformation from Gravitational Waves]
\textbf{Physical meaning:} After a merger, the horizon is ``lumpy'' with high $\mathcal{C}$.

\textbf{Kerr baseline:} Unperturbed Kerr has $\mathcal{C} = 1 + a^2/M^2$.

\textbf{Quadrupole dominance:} The $\ell=2$ modes contribute most (GW frequency).

\textbf{Ringdown:} As QNMs decay, $\mathcal{C} \to 1 + a^2/M^2$ (settling to Kerr).

\textbf{LIGO signature:} $\mathcal{C}(t)$ could be reconstructed from ringdown waveform!
\end{physicsbox}

%% ============================================================================
\section{Entropy Production for Realistic Accretion}
%% ============================================================================

\begin{innovation}[Accretion Entropy]
\begin{theorem}[Entropy Production for Thin Disk]\label{thm:thin-disk-entropy}
For a geometrically thin accretion disk around a Kerr black hole with accretion rate $\dot{M}$:
\begin{equation}\label{eq:thin-disk-entropy}
\boxed{
    \dot{S}_{\trap} = \frac{\dot{M}}{4\ell_P^2}\left( \frac{2r_+}{M} + \frac{a^2}{Mr_+} \right) = \frac{2\dot{M}r_+}{\ell_P^2 M}\left(1 + \frac{\mathcal{D}}{4}\right)
}
\end{equation}
\end{theorem}
\end{innovation}

\begin{physicsbox}[How Fast Does Black Hole Entropy Grow?]
\textbf{Physical meaning:} Each unit of accreted mass contributes entropy.

\textbf{Schwarzschild:} $\dot{S} = 4\dot{M}M/\ell_P^2$.

\textbf{Kerr:} Entropy production is \emph{enhanced} by factor $(1 + \mathcal{D}/4)$.

\textbf{Extremal limit:} $\dot{S} \to 3\dot{M}M/\ell_P^2$ (reduced by factor 3/4).

\textbf{M87*:} With $\dot{M} \sim 10^{-3} M_\odot/\text{yr}$, $\dot{S} \sim 10^{77}$ bits/yr!
\end{physicsbox}

%% ============================================================================
\section{Spectral Gap for Schwarzschild}
%% ============================================================================

\begin{innovation}[Schwarzschild Spectrum]
\begin{theorem}[Trapping Spectrum for Schwarzschild]\label{thm:schwarzschild-spectrum}
For Schwarzschild with horizon radius $r_s = 2M$, the eigenvalues of $\tilde{L}_T$ are:
\begin{equation}\label{eq:schwarzschild-spectrum}
\boxed{
    \lambda_\ell = \frac{\ell(\ell+1)}{r_s^2} + \frac{1}{2r_s^2} = \frac{\ell(\ell+1) + 1/2}{4M^2}, \quad \ell = 0, 1, 2, \ldots
}
\end{equation}
The spectral gap is $\Delta\lambda = \lambda_1 - \lambda_0 = 2/(4M^2) = 1/(2M^2)$.
\end{theorem}
\end{innovation}

\begin{physicsbox}[Energy Levels of Schwarzschild Horizon]
\textbf{Physical meaning:} These are ``energy levels'' of the horizon surface.

\textbf{Ground state:} $\lambda_0 = 1/(8M^2)$ (s-wave).

\textbf{First excited:} $\lambda_1 = 5/(8M^2)$ (p-wave).

\textbf{Spectral gap:} $\Delta\lambda = 1/(2M^2)$ --- larger black holes have smaller gaps.

\textbf{QNM connection:} $\omega_{\ell} \sim \sqrt{\lambda_\ell}$ for quasi-normal mode frequencies.

\textbf{Decay time:} Perturbations decay as $e^{-\sqrt{\Delta\lambda} t} = e^{-t/(M\sqrt{2})}$.
\end{physicsbox}

%% ============================================================================
\section{Diamond Mass for FLRW}
%% ============================================================================

\begin{innovation}[Cosmological Diamond Mass]
\begin{theorem}[Diamond Mass in FLRW]\label{thm:flrw-diamond}
For a causal diamond in FLRW cosmology with Hubble parameter $H$ and proper time separation $\tau$:
\begin{equation}\label{eq:flrw-diamond}
\boxed{
    M_\Diamond^{\text{FLRW}} = \frac{c^2}{6G}\tau H^{-1}\sqrt{1 + \frac{H^2\tau^2}{4}} \approx \frac{c^2\tau}{6GH}\left(1 + \frac{H^2\tau^2}{8}\right)
}
\end{equation}
for small $H\tau$.
\end{theorem}
\end{innovation}

\begin{physicsbox}[Mass of Observable Universe Patches]
\textbf{Physical meaning:} $M_\Diamond$ gives the ``effective mass'' of a cosmological region.

\textbf{Small diamonds:} $M_\Diamond \approx c^2\tau/(6GH)$ scales with time extent.

\textbf{Hubble-sized:} For $\tau \sim H^{-1}$, we get $M_\Diamond \sim c^2/(GH) \sim M_{\text{Hubble}}$.

\textbf{Our observable universe:} $M_\Diamond \sim 10^{53}$ kg (matches Hubble mass!).

\textbf{Holographic:} $M_\Diamond$ is bounded by area of diamond, not volume --- holographic principle.
\end{physicsbox}

%% ============================================================================
\section{Numerical Estimates for Real Black Holes}
%% ============================================================================

\begin{innovation}[Astrophysical Numbers]
\begin{theorem}[Physical Values]\label{thm:physical-values}
For observed black holes:

\textbf{M87* (supermassive):}
\begin{equation}\label{eq:m87}
\boxed{
    \begin{aligned}
        M &\approx 6.5 \times 10^9 M_\odot, \quad a/M \approx 0.9 \\
        \mathcal{D} &\approx 0.45, \quad M^* \approx 4.8 \times 10^9 M_\odot \\
        \kappa &\approx 1.8 \times 10^{-15} \text{ Hz}, \quad T_H \approx 1.5 \times 10^{-17} \text{ K}
    \end{aligned}
}
\end{equation}

\textbf{Cygnus X-1 (stellar):}
\begin{equation}\label{eq:cygx1}
\boxed{
    \begin{aligned}
        M &\approx 21 M_\odot, \quad a/M \approx 0.998 \\
        \mathcal{D} &\approx 0.87, \quad M^* \approx 7.6 M_\odot \\
        \kappa &\approx 1.8 \times 10^{-5} \text{ Hz}, \quad T_H \approx 3 \times 10^{-9} \text{ K}
    \end{aligned}
}
\end{equation}
\end{theorem}
\end{innovation}

\begin{physicsbox}[What Our Formulas Say About Real Black Holes]
\textbf{M87*:}
\begin{itemize}
    \item Shadow mass $M^* \approx 0.74 M$ --- shadow is 26\% smaller than expected from total mass
    \item High trapping depth $\mathcal{D} \approx 0.45$ means 45\% of mass is ``extractable'' spin energy
    \item Hawking temperature $\sim 10^{-17}$ K --- essentially zero
\end{itemize}

\textbf{Cygnus X-1:}
\begin{itemize}
    \item Near-extremal spin gives $\mathcal{D} \approx 0.87$ --- 87\% extractable!
    \item Shadow mass only 36\% of total mass
    \item Most of the ``mass'' is rotational energy
\end{itemize}

\textbf{Observational test:} Compare shadow-inferred mass $M^*$ with orbital dynamics mass $M$.
\end{physicsbox}

%% ============================================================================
\part{Fundamental Physics: New Theorems}
%% ============================================================================

%% ============================================================================
\section{Trapping Uniqueness Theorem (``No-Hair'' for Trapping)}
%% ============================================================================

\begin{innovation}[Trapping No-Hair]
\begin{theorem}[Trapping Uniqueness]\label{thm:trapping-uniqueness}
Let $(M^4, g)$ be a stationary, asymptotically flat, electrovacuum spacetime containing a black hole. Then the \textbf{Trapping Depth} $\mathcal{D}$ at the horizon is uniquely determined by $(M, J, Q)$:
\begin{equation}\label{eq:trapping-uniqueness}
\boxed{
    \mathcal{D}(M, J, Q) = 1 - \frac{M_{\text{irr}}^2}{M^2} = \frac{a^2 + Q^2/(2M r_+)}{r_+^2 + a^2} \cdot (r_+^2 + a^2)/(4M^2)
}
\end{equation}
where $a = J/M$, $r_+ = M + \sqrt{M^2 - a^2 - Q^2}$, and $M_{\text{irr}}^2 = (r_+^2 + a^2)/(4M)$.
\end{theorem}
\end{innovation}

\begin{physicsbox}[Internal Structure Has No Hair Either]
\textbf{Physical meaning:} The trapping strength at the horizon is completely determined by three numbers.

\textbf{Classical no-hair:} Exterior metric has no hair (Israel, Carter, Robinson theorems).

\textbf{New result:} The \emph{trapping depth} $\mathcal{D}$ also has no hair!

\textbf{Explicit formulas:}
\begin{align*}
    \text{Schwarzschild } (J = Q = 0):& \quad \mathcal{D} = 0 \text{ (marginal trapping)}\\
    \text{Kerr } (Q = 0):& \quad \mathcal{D} = 1 - \frac{(r_+^2 + a^2)}{4M^2} = \frac{a^2}{r_+^2 + a^2}\\
    \text{Extremal Kerr } (a = M):& \quad \mathcal{D} = 1/2
\end{align*}

\textbf{Implication:} Two black holes with same $(M, J, Q)$ have identical trapping strength.
\end{physicsbox}

\begin{newineq}[Uniqueness Bound]
\begin{theorem}[Trapping-Parameter Relation]\label{thm:profile-rigidity}
For any stationary black hole, the trapping depth $\mathcal{D}$ satisfies:
\begin{equation}\label{eq:profile-rigidity}
\boxed{
    \mathcal{D} = \frac{E_{\text{extractable}}}{M c^2} = 1 - \frac{M_{\text{irr}}^2}{M^2}
}
\end{equation}
The trapping depth equals the fraction of mass-energy that is extractable.
\end{theorem}
\end{newineq}

%% ============================================================================
\section{Cosmic Censorship from Trapping}
%% ============================================================================

\begin{innovation}[Censorship Functional]
\begin{definition}[Censorship Functional]\label{def:censorship-functional}
The \textbf{Censorship Functional} on initial data $(M^3, g, k)$ is:
\begin{equation}\label{eq:censorship-functional}
\boxed{
    \mathcal{C}[\Sigma] = M_{ADM} - \sqrt{\frac{A(\Sigma)}{16\pi}} \cdot \sqrt{1 + \mathcal{D}(\Sigma)}
}
\end{equation}
\end{definition}
\end{innovation}

\begin{newineq}[Censorship Principle]
\begin{conjecture}[Trapping Censorship Conjecture]\label{conj:trapping-censorship}
For any asymptotically flat initial data satisfying the dominant energy condition:
\begin{equation}\label{eq:trapping-censorship}
\boxed{
    \inf_{\Sigma \text{ trapped}} \mathcal{C}[\Sigma] \geq 0
}
\end{equation}
with equality if and only if the data is a slice of Kerr-Newman spacetime.
\end{conjecture}
\end{newineq}

\begin{physicsbox}[Why Naked Singularities Are Forbidden]
\textbf{Physical meaning:} The Censorship Functional must be non-negative.

\textbf{Interpretation:}
\begin{itemize}
    \item $\mathcal{C} > 0$: Mass ``budget'' exceeds trapping cost --- horizon forms
    \item $\mathcal{C} = 0$: Extremal black hole --- barely clothed
    \item $\mathcal{C} < 0$: Would require more trapping than mass allows --- FORBIDDEN
\end{itemize}

\textbf{Why this is new:} Classical censorship says ``singularities are hidden.'' Our version says \emph{why}: the trapping-mass budget prevents exposure.

\textbf{Explicit bound:} $\sqrt{1 + \mathcal{D}} \leq M/\sqrt{A/(16\pi)}$ always.
\end{physicsbox}

%% ============================================================================
\section{Black Hole Evaporation and Trapping}
%% ============================================================================

\begin{innovation}[Evaporation-Depth Evolution]
\begin{theorem}[Hawking Evaporation and Curvature]\label{thm:evaporation-depth}
For a Schwarzschild black hole undergoing Hawking evaporation, the Kretschmann scalar $K = R_{\mu\nu\rho\sigma}R^{\mu\nu\rho\sigma}$ at the horizon evolves as:
\begin{equation}\label{eq:evaporation-depth}
\boxed{
    \frac{dK_{\text{horizon}}}{dt} = \frac{\hbar c^{10}}{1920 \pi G^4 M^7} > 0
}
\end{equation}
The curvature (and hence ``gravitational trapping strength'') \textbf{increases} as the black hole shrinks.
\end{theorem}
\end{innovation}

\begin{physicsbox}[Evaporation Increases Curvature]
\textbf{Physical meaning:} As a black hole evaporates, spacetime curvature at the horizon grows.

\textbf{Key relations for Schwarzschild:}
\begin{itemize}
    \item Horizon curvature: $K_H = 48 G^2 M^2/c^8 r_s^6 = 3/(4M^4)$ (in geometric units)
    \item As $M$ decreases: $K_H \propto 1/M^4$ increases
    \item Mass loss rate: $dM/dt = -\hbar c^4/(15360\pi G^2 M^2)$ (Page formula)
\end{itemize}

\textbf{Information perspective:} Near Planck scale ($M \to M_P$), curvature becomes enormous. Quantum gravity effects dominate --- this is where information must emerge.

\textbf{Trapping interpretation:} Define ``effective trapping'' $\tilde{\mathcal{D}} = K \cdot r_s^4$. Then $\tilde{\mathcal{D}}$ increases during evaporation.
\end{physicsbox}

\begin{newineq}[Curvature at Evaporation End]
\begin{theorem}[Final Curvature]\label{thm:final-depth}
As $M \to M_{\text{Planck}}$:
\begin{equation}\label{eq:final-depth}
\boxed{
    K_{\text{final}} \sim \frac{c^6}{\hbar^2 G^2} = \ell_P^{-4}
}
\end{equation}
Curvature reaches Planck scale, where quantum gravity must resolve the endpoint.
\end{theorem}
\end{newineq}

%% ============================================================================
\section{Gravitational Wave Memory from Trapping}
%% ============================================================================

\begin{innovation}[Memory-Trapping Relation]
\begin{theorem}[GW Memory Formula]\label{thm:gw-memory}
The permanent gravitational wave memory strain at distance $r$ is:
\begin{equation}\label{eq:gw-memory}
\boxed{
    \Delta h_{\text{memory}} = \frac{G}{c^4 r} \cdot \Delta(\mathcal{D} \cdot A)
}
\end{equation}
where $\Delta(\mathcal{D} \cdot A)$ is the total change in (Trapping Depth $\times$ Area) during the event.
\end{theorem}
\end{innovation}

\begin{physicsbox}[Permanent Spacetime Deformation]
\textbf{Physical meaning:} After a GW event, spacetime is permanently deformed.

\textbf{For binary merger:}
\begin{equation}
    \Delta(\mathcal{D} \cdot A) = (\mathcal{D}_f A_f) - (\mathcal{D}_1 A_1 + \mathcal{D}_2 A_2)
\end{equation}

\textbf{Numerical estimate (GW150914-like):}
\begin{itemize}
    \item Initial: Two BHs with $\mathcal{D}_1 \approx \mathcal{D}_2 \approx 0.7$
    \item Final: One BH with $\mathcal{D}_f \approx 0.44$
    \item Memory: $\Delta h \sim 10^{-24}$ at 100 Mpc
\end{itemize}

\textbf{Detection:} Next-generation detectors (LISA, Einstein Telescope) can measure this!

\textbf{Why this is new:} Standard memory formula uses mass multipoles. Ours uses trapping.
\end{physicsbox}

\begin{newineq}[Memory Bound]
\begin{theorem}[Maximum Memory]\label{thm:max-memory}
For any gravitational wave event:
\begin{equation}\label{eq:max-memory}
\boxed{
    |\Delta h_{\text{memory}}| \leq \frac{G M_{\text{total}}}{c^2 r} \cdot \Delta\mathcal{D}_{\max}
}
\end{equation}
where $\Delta\mathcal{D}_{\max} \leq 1$ is the maximum possible depth change.
\end{theorem}
\end{newineq}

%% ============================================================================
\section{Soft Trapping Hair}
%% ============================================================================

\begin{innovation}[Soft Trapping Modes]
\begin{definition}[Soft Hair]\label{def:soft-hair}
The \textbf{Soft Trapping Hair} consists of zero-energy modes on the horizon:
\begin{equation}\label{eq:soft-hair}
\boxed{
    \delta\mathcal{D}_{\text{soft}}^{(\ell m)} = c_{\ell m} \cdot Y_{\ell m}(\theta, \phi) \cdot e^{-\epsilon u}
}
\end{equation}
in the limit $\epsilon \to 0^+$, where $u$ is retarded time and $c_{\ell m} \in \mathbb{R}$ are free parameters.
\end{definition}
\end{innovation}

\begin{physicsbox}[Information Storage in Soft Hair]
\textbf{Physical meaning:} The horizon has infinitely many zero-energy modes.

\textbf{Mode counting:}
\begin{itemize}
    \item One mode for each $(\ell, m)$ with $\ell \geq 0$, $-\ell \leq m \leq \ell$
    \item Total: infinitely many soft modes
    \item Each stores one real number $c_{\ell m}$
\end{itemize}

\textbf{Information storage:} When matter falls in, it excites soft modes:
\begin{equation}
    c_{\ell m}^{\text{after}} = c_{\ell m}^{\text{before}} + \int \text{(matter contribution)}
\end{equation}

\textbf{Resolution of no-hair tension:} Classical no-hair says ``only $(M, J, Q)$.'' But soft hair carries additional \emph{information} at zero energy cost!

\textbf{Why this is new:} Hawking-Perry-Strominger soft hair is in BMS charges. Ours is in trapping depth fluctuations --- a different (complementary) mechanism.
\end{physicsbox}

\begin{newineq}[Soft Hair Entropy]
\begin{theorem}[Information in Soft Hair]\label{thm:soft-entropy}
The entropy stored in soft trapping hair satisfies:
\begin{equation}\label{eq:soft-entropy}
\boxed{
    S_{\text{soft}} = \frac{k_B}{4\ell_P^2} \sum_{\ell, m} |c_{\ell m}|^2 \leq S_{BH}
}
\end{equation}
Soft hair can account for up to the full Bekenstein-Hawking entropy.
\end{theorem}
\end{newineq}

%% ============================================================================
\section{Binary Merger Ringdown}
%% ============================================================================

\begin{innovation}[Ringdown-Trapping Formula]
\begin{theorem}[QNM Frequency from Trapping]\label{thm:ringdown-trapping}
The dominant ($\ell = 2, m = 2$) quasi-normal mode frequency after merger can be expressed as:
\begin{equation}\label{eq:ringdown-trapping}
\boxed{
    f_{22} = \frac{c^3}{2\pi G M_f} \cdot \frac{1}{2} \left(1 - \sqrt{1 - \mathcal{D}_f}\right)^{0.45}
}
\end{equation}
where $M_f$ is the final black hole mass and $\mathcal{D}_f = 1 - M_{\text{irr}}^2/M_f^2$ is the trapping depth.

\textbf{Note:} This is a re-expression of known QNM physics in terms of $\mathcal{D}$, not a new fundamental formula.
\end{theorem}
\end{innovation}

\begin{physicsbox}[Ringdown Frequency from Trapping Depth]
\textbf{Physical meaning:} The ``ringing'' frequency depends on the spin, which is encoded in $\mathcal{D}_f$.

\textbf{Limiting cases:}
\begin{itemize}
    \item $\mathcal{D}_f = 0$ (Schwarzschild): $f_{22} \approx c^3/(2\pi G M_f) \cdot 0.37$ (standard result)
    \item $\mathcal{D}_f \to 1$ (extremal): $f_{22} \to $ lower value (horizon becomes degenerate)
\end{itemize}

\textbf{GW150914 check:}
\begin{itemize}
    \item Measured: $f_{22} \approx 250$ Hz, $M_f \approx 62 M_\odot$, spin $\approx 0.67$
    \item Trapping depth: $\mathcal{D}_f \approx 0.44$ (from spin)
    \item Consistent with known Kerr QNM formulas
\end{itemize}

\textbf{Value of this reformulation:} Expresses ringdown in terms of ``how trapped'' the final black hole is.
\end{physicsbox}

\begin{newineq}[Damping Time]
\begin{theorem}[Ringdown Damping]\label{thm:ringdown-damping}
The damping time for the dominant mode can be expressed as:
\begin{equation}\label{eq:ringdown-damping}
\boxed{
    \tau_{22} \approx \frac{4 G M_f}{c^3} \cdot \frac{1}{1 - \mathcal{D}_f}
}
\end{equation}
Higher trapping depth (higher spin) means \textbf{longer} ringing (slower decay).

\textbf{Note:} This is the known relationship between damping time and spin, rewritten using $\mathcal{D}$.
\end{theorem}
\end{newineq}

%% ============================================================================
\section{Kerr-Newman: Charge-Trapping Decomposition}
%% ============================================================================

\begin{innovation}[Charge-Trapping]
\begin{theorem}[Trapping Depth Decomposition]\label{thm:charge-trapping}
For Kerr-Newman black holes, the Trapping Depth decomposes as:
\begin{equation}\label{eq:charge-trapping}
\boxed{
    \mathcal{D}_{KN} = \mathcal{D}_{\text{spin}} + \mathcal{D}_{\text{charge}} - \mathcal{D}_{\text{coupling}}
}
\end{equation}
where:
\begin{align}
    \mathcal{D}_{\text{spin}} &= \frac{a^2}{r_+^2 + a^2} = \frac{J^2/M^2}{r_+^2 + J^2/M^2} \label{eq:d-spin}\\
    \mathcal{D}_{\text{charge}} &= \frac{Q^2}{2Mr_+} \label{eq:d-charge}\\
    \mathcal{D}_{\text{coupling}} &= \frac{Q^2 a^2}{2Mr_+(r_+^2 + a^2)} \label{eq:d-coupling}
\end{align}
\end{theorem}
\end{innovation}

\begin{physicsbox}[How Spin and Charge Contribute]
\textbf{Physical meaning:} Trapping depth has separate spin and charge contributions.

\textbf{Key observations:}
\begin{itemize}
    \item Both spin and charge \emph{increase} trapping (positive contributions)
    \item But they \emph{interfere}: coupling term is negative
    \item Maximum $\mathcal{D}_{KN} = 1$ at extremality: $M^2 = a^2 + Q^2$
\end{itemize}

\textbf{Schwarzschild ($a = Q = 0$):} $\mathcal{D} = 0$ at horizon (marginal trapping).

\textbf{Reissner-Nordström ($a = 0$):} $\mathcal{D}_{RN} = Q^2/(2Mr_+) = Q^2/(M^2 + M\sqrt{M^2-Q^2})$.

\textbf{Extremal RN ($Q = M$):} $\mathcal{D} = 1/2$ (charge alone gives half-maximum trapping).
\end{physicsbox}

\begin{newineq}[Charge-Spin Inequality]
\begin{theorem}[Combined Bound]\label{thm:charge-spin-bound}
For any Kerr-Newman black hole:
\begin{equation}\label{eq:charge-spin-bound}
\boxed{
    \mathcal{D}_{\text{spin}} + \mathcal{D}_{\text{charge}} \leq 1 + \mathcal{D}_{\text{coupling}}
}
\end{equation}
with equality at extremality.
\end{theorem}
\end{newineq}

%% ============================================================================
\section{Primordial Black Hole Signatures}
%% ============================================================================

\begin{innovation}[Primordial Trapping]
\begin{theorem}[PBH Formation Depth]\label{thm:pbh-depth}
A primordial black hole formed at cosmic time $t_{\text{form}}$ from density fluctuations has initial trapping depth:
\begin{equation}\label{eq:pbh-depth}
\boxed{
    \mathcal{D}_{\text{PBH}}(t_{\text{form}}) = \frac{\delta \rho / \rho_c}{1 + \delta\rho/\rho_c} < \mathcal{D}_{\text{collapse}}
}
\end{equation}
where $\delta\rho/\rho_c \sim 0.3$--$0.5$ is the density contrast at formation, and $\mathcal{D}_{\text{collapse}} \sim 0.7$ is the typical value for stellar collapse.
\end{theorem}
\end{innovation}

\begin{physicsbox}[Distinguishing Primordial from Astrophysical]
\textbf{Physical meaning:} PBHs form from density fluctuations, not gravitational collapse.

\textbf{Key difference:}
\begin{itemize}
    \item \textbf{Stellar collapse}: Matter compresses violently $\Rightarrow$ high $\mathcal{D}$
    \item \textbf{Primordial}: Gradual horizon formation $\Rightarrow$ low $\mathcal{D}$
\end{itemize}

\textbf{Evolution:} Both types evolve via:
\begin{equation}
    \mathcal{D}(t) = \mathcal{D}_{\text{initial}} + \delta\mathcal{D}_{\text{accretion}} + \delta\mathcal{D}_{\text{evaporation}}
\end{equation}

\textbf{Present-day signature:} PBHs that haven't accreted much should have:
\begin{equation}
    \mathcal{D}_{\text{PBH, today}} \lesssim 0.3 \quad \text{(versus } \mathcal{D}_{\text{astro}} \sim 0.5\text{--}0.9\text{)}
\end{equation}

\textbf{Dark matter implication:} If dark matter is PBHs, they're detectable by anomalously low $\mathcal{D}$.
\end{physicsbox}

\begin{newineq}[PBH Age Formula]
\begin{theorem}[Trapping Depth as Clock]\label{thm:depth-clock}
The formation time of a PBH can be estimated from:
\begin{equation}\label{eq:depth-clock}
\boxed{
    t_{\text{form}} \sim t_{\text{universe}} \cdot \left(\frac{\mathcal{D}_{\text{PBH}}}{\mathcal{D}_{\text{astro}}}\right)^3
}
\end{equation}
Lower trapping depth indicates earlier formation.
\end{theorem}
\end{newineq}

%% ============================================================================
\section{Connections to Other Mathematics}


\begin{itemize}
    \item \textbf{Spectral Geometry:} The Trapping Laplacian $L_T$ connects to inverse spectral problems and Steklov eigenvalue bounds
    \item \textbf{Capacity Theory:} The dual $\theta$-capacity extends weighted potential theory à la Agostiniani-Mazzieri-Oronzio
    \item \textbf{Optimal Transport:} The causal Wasserstein distance $\mathcal{W}_2$ and Lorentzian optimal transport (Cavalletti-Mondino)
    \item \textbf{Entropy/Information:} Effective area, trapping entropy, and generalized entropy connect to holographic principles
    \item \textbf{Calibrations:} Sign-invariant quantities $\theta^+\theta^-$ suggest Lorentzian calibration theory
    \item \textbf{PDE Theory:} The trapping potential $\Psi$ connects to Green's function methods for mass bounds
    \item \textbf{Geometric Flows:} Trapping flow, dual flow, and area-entropy flow extend classical geometric flows
    \item \textbf{Spinor Geometry:} The Trapping Laplacian has connections to Dirac operator bounds (Witten approach)
    \item \textbf{Quantum Information:} Complexity, entanglement wedge, scrambling connect to quantum gravity
    \item \textbf{Thermodynamics:} Entropy production rate, area laws, irreversibility measures mirror black hole thermodynamics
    \item \textbf{Dynamical Systems:} Lyapunov functional, stability index, bifurcation theory connect to MOTS dynamics
    \item \textbf{Symplectic Geometry:} Momentum map $\mu_X$ connects to coadjoint orbits and Hamiltonian actions
    \item \textbf{Algebraic Topology:} Trapping cohomology $H^*_T$ extends de Rham theory to null structures
    \item \textbf{Variational Calculus:} Trapping action $\mathcal{S}[\Sigma]$ defines new extremal surface problems
    \item \textbf{Wave Equations:} Trapping wave equation extends QNM analysis and horizon perturbation theory
    \item \textbf{Control Theory:} Lyapunov methods give stability and convergence guarantees for flows
    \item \textbf{Cosmology:} PBH signatures, cosmic censorship, and early universe connections
    \item \textbf{Gravitational Wave Physics:} Memory, ringdown, and merger dynamics via trapping
\end{itemize}

%% ============================================================================
\section*{Summary: Original Contributions}
%% ============================================================================

\begin{tcolorbox}[colback=green!5!white, colframe=green!65!black, title={\textbf{GENUINELY NEW: Not Found in Literature}}]

\textbf{I. Central Innovation --- Trapping Depth Framework:}
\begin{itemize}
    \item Trapping Depth $\mathcal{D} = 1 - M_{\text{irr}}^2/M^2 \in [0, 1)$: \textbf{New unifying quantity}
    \item Physical meaning: Fraction of mass-energy beyond irreducible minimum
    \item Connects shadow mass, entropy, GW memory, ringdown, and extractable energy
\end{itemize}

\textbf{II. NEW Operators (Original):}
\begin{itemize}
    \item Trapping Laplacian $L_T = -\Delta - \frac{R}{2} + \frac{|A|^2}{4} + \frac{\theta^+\theta^-}{4}$
    \item Dual $\theta$-Capacity $\widetilde{\Cap}_\theta$ with reversed monotonicity
    \item Censorship Functional $\mathcal{C}[\Sigma] = M - \sqrt{A/(16\pi)}\sqrt{1+\mathcal{D}}$
\end{itemize}

\textbf{III. NEW Trapping Functionals (Original):}
\begin{itemize}
    \item Shadow Mass $M^* = M_{\text{irr}} = M\sqrt{1-\mathcal{D}}$
    \item Trapping Intensity $\mathcal{I} = \frac{1}{A}\int\theta^+\theta^- \, dA \geq 0$
    \item Escape Difficulty $\mathcal{E} = e^{\langle|\theta^+|/|H|\rangle} - 1$
    \item Focusing Power $\mathcal{F} = \int R_{\mu\nu}\ell^\mu\ell^\nu dA$
\end{itemize}

\textbf{IV. NEW Inequalities (Original):}
\begin{itemize}
    \item Mass-Trapping: $M^2 \geq \frac{A}{16\pi}(1 + \mathcal{D}/4)$
    \item Entropy-Depth Trade-off: $S \cdot \mathcal{D} \leq 4\pi M^2/\ell_P^2$
    \item Capacity Bounds: $\Area(\Sigma) \leq \widetilde{\Cap}_\theta(\Sigma)$ for trapped surfaces
    \item Censorship: $\mathcal{C}[\Sigma] \geq 0$ for all trapped surfaces
\end{itemize}

\textbf{V. NEW Physical Results:}
\begin{itemize}
    \item Shadow $<$ Mass: $M^* = M\sqrt{1-\mathcal{D}} < M$ for rotating BHs
    \item GW Memory: $\Delta h_{\text{memory}} = G\Delta(\mathcal{D}\cdot A)/(c^4 r)$
    \item Ringdown-$\mathcal{D}$ relation: Higher $\mathcal{D}$ gives lower ringdown frequency
    \item PBH signature: $\mathcal{D}_{\text{PBH}} < \mathcal{D}_{\text{astro}}$ (formation mechanism difference)
\end{itemize}

\textbf{VI. NEW Reformulations of Known Physics:}
\begin{itemize}
    \item Extractable energy = $M\mathcal{D}$ (known: Christodoulou, but new perspective)
    \item No-hair in trapping: $\mathcal{D}(M,J,Q)$ unique (new formulation of no-hair)
    \item Ringdown formula in terms of $\mathcal{D}$ (known QNM, new expression)
    \item Charge-Trapping decomposition: $\mathcal{D}_{KN}$ split into spin/charge/coupling
    \item Bifurcation Index $\mathcal{B}$: Horizon splitting directions
    \item Diamond Mass $M_\Diamond$: Quasi-local mass for causal diamonds
\end{itemize}

\textbf{VII. NEW Inequalities from Part VIII (Original):}
\begin{itemize}
    \item Shadow-Mass: $M^* \leq M_{ADM}(1 - \mathcal{D}/8)^{1/2}$
    \item Concentration-Stability: $\mathcal{C} \geq 1 + |\kappa|^2/(16\pi/A)$
    \item Spectral-Mass: $M^2 \geq \frac{A}{16\pi}(1 + \lambda_0 A/(8\pi))$
    \item Tensor Bound: $\int|\mathcal{T}|^2 \geq (\theta^+\theta^-)^2 A/4$
    \item Momentum-Mass: $|\mu_X|^2 \leq M^2 A(1+\mathcal{D})$
    \item Bifurcation-Area: $d^2A/dt^2 \leq -\mathcal{B}/(8\pi)\int\theta^-|\sigma^+|^2$
    \item Diamond Monotonicity: $M_{\Diamond_1} \leq M_{\Diamond_2}$ for nested diamonds
    \item Cohomological: $M \geq \sqrt{\dim H^2_T/(16\pi G)}\cdot\ell_P$
\end{itemize}

\textbf{VII. Variational and Geometric Structures (Original):}
\begin{itemize}
    \item Trapping Action $\mathcal{S}[\Sigma]$: Action functional for black hole surfaces
    \item Bifurcation Index $\mathcal{B}$: Predicts horizon splitting/merging
    \item Diamond Mass $M_\Diamond$: Quasi-local mass for causal diamonds
    \item Lyapunov Functional $\mathcal{L}$: Controls flow convergence
\end{itemize}

\textbf{VIII. Dynamical Evolution (Original):}
\begin{itemize}
    \item Trapping Flow: $\partial_t\Sigma = -\theta^+ n$ (flow toward MOTS)
    \item Flow Area Monotonicity: $dA/dt = -\int(\theta^+)^2 \leq 0$
    \item Entropy Production: $\dot{S}_{\trap} \geq 0$ (geometric second law)
\end{itemize}

\textbf{IX. Astrophysical Applications (Original predictions):}
\begin{itemize}
    \item M87*: $\mathcal{D}\approx 0.45$, shadow 26\% smaller than total mass
    \item Cygnus X-1: $\mathcal{D}\approx 0.87$, 87\% extractable spin energy
    \item Observable test: Compare shadow mass $M^*$ with orbital mass $M$
\end{itemize}

\end{tcolorbox}

\begin{tcolorbox}[colback=red!5!white, colframe=red!65!black, title={\textbf{KNOWN RESULTS (Not Claimed as New --- Used as Foundation)}}]
\begin{itemize}
    \item Hawking mass $m_H$, Penrose inequality $M \geq \sqrt{A/(16\pi)}$
    \item Kerr bound $|J| \leq M^2$, Christodoulou formula $M^2 = M_{\text{irr}}^2 + J^2/(4M_{\text{irr}}^2)$
    \item Irreducible mass $M_{\text{irr}} = \sqrt{A/(16\pi)}$ (used in our $\mathcal{D}$ definition)
    \item Bekenstein-Hawking entropy $S = A/(4\ell_P^2)$
    \item Raychaudhuri equation, MOTS stability operator
    \item Hawking temperature $T_H = \hbar c^3/(8\pi GMk_B)$
    \item 29\% maximum extraction from extremal Kerr
    \item Area theorem $dA/dt \geq 0$
    \item Classical no-hair theorem (Israel, Carter, Robinson)
    \item Hawking-Perry-Strominger soft hair (BMS charges)
    \item Standard QNM formulas (we re-express in terms of $\mathcal{D}$)
    \item Cosmic censorship conjectures (we reformulate via $\mathcal{C}[\Sigma]$)
\end{itemize}
\end{tcolorbox}

%% ============================================================================
\section*{Acknowledgments}
%% ============================================================================

This work presents original mathematical contributions to black hole geometry. All boxed formulas marked ``NEW'' are introduced for the first time in this paper.

\bibliographystyle{amsplain}
\begin{thebibliography}{99}

\bibitem{HuiskenIlmanen2001} G. Huisken and T. Ilmanen, \textit{The inverse mean curvature flow and the Riemannian Penrose inequality}, J. Differential Geom. 59 (2001), 353--437.

\bibitem{Bray2001} H. Bray, \textit{Proof of the Riemannian Penrose inequality using the positive mass theorem}, J. Differential Geom. 59 (2001), 177--267.

\bibitem{SchoenYau1979} R. Schoen and S.-T. Yau, \textit{On the proof of the positive mass conjecture in general relativity}, Comm. Math. Phys. 65 (1979), 45--76.

\bibitem{Witten1981} E. Witten, \textit{A new proof of the positive energy theorem}, Comm. Math. Phys. 80 (1981), 381--402.

\bibitem{AMO2016} V. Agostiniani, L. Mazzieri, and F. Oronzio, \textit{A Green's function proof of the positive mass theorem}, arXiv:2108.08402.

\bibitem{MarsSimon2005} M. Mars and W. Simon, \textit{On uniqueness of static Einstein-Maxwell-dilaton black holes}, Adv. Theor. Math. Phys. 6 (2002), 279--305.

\bibitem{CavMondi2020} F. Cavalletti and A. Mondino, \textit{Optimal transport in Lorentzian synthetic spaces, synthetic timelike Ricci curvature lower bounds and applications}, arXiv:2004.08934.

\end{thebibliography}

\end{document}
