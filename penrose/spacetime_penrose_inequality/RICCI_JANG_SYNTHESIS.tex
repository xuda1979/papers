%% RICCI_JANG_SYNTHESIS.tex
%%
%% THE SYNTHESIS: Ricci Flow + Jang Equation
%% December 2025
%%
%% The Ultimate Attack on Area Dominance

\documentclass[11pt]{amsart}
\usepackage{amsmath,amssymb,amsthm}
\usepackage{xcolor}
\usepackage{tcolorbox}

\tcbuselibrary{theorems}

\newtcolorbox{main_theorem}{
    colback=red!5!white,
    colframe=red!75!black,
    title={\textbf{MAIN THEOREM}}
}

\newtcolorbox{key_lemma}{
    colback=blue!5!white,
    colframe=blue!75!black,
}

\newtcolorbox{insight}{
    colback=green!5!white,
    colframe=green!75!black,
    title={\textbf{INSIGHT}}
}

\newtheorem{theorem}{Theorem}
\newtheorem{lemma}[theorem]{Lemma}
\newtheorem{proposition}[theorem]{Proposition}
\newtheorem{corollary}[theorem]{Corollary}
\theoremstyle{definition}
\newtheorem{definition}[theorem]{Definition}
\newtheorem{remark}[theorem]{Remark}

\newcommand{\Area}{\mathrm{Area}}
\newcommand{\Vol}{\mathrm{Vol}}
\newcommand{\divv}{\mathrm{div}}
\newcommand{\Ric}{\mathrm{Ric}}
\newcommand{\graph}{\mathrm{graph}}
\DeclareMathOperator{\tr}{tr}

\title{The Ricci-Jang Synthesis:\\
A Complete Attack on Area Dominance}
\author{December 2025}

\begin{document}
\maketitle

\begin{abstract}
We combine the Jang equation approach with Ricci flow techniques to attack 
the Area Dominance problem. The key insight is that the Jang surface 
transforms the initial data into a Riemannian manifold where flow methods 
can be applied directly.
\end{abstract}

%% ============================================================================
\section{The Synthesis Idea}
%% ============================================================================

\begin{insight}
\textbf{Two powerful techniques:}
\begin{enumerate}
    \item \textbf{Jang equation:} Transforms $(g, k)$ data into Riemannian manifold
    \item \textbf{Ricci flow:} Simplifies Riemannian geometry with monotonic quantities
\end{enumerate}

\textbf{Combined approach:}
\begin{enumerate}
    \item Use Jang equation to create Riemannian manifold $(\hat{\mathcal{C}}, \hat{g})$
    \item Run Ricci flow on $(\hat{\mathcal{C}}, \hat{g})$
    \item Track how areas evolve
    \item Conclude Area Dominance
\end{enumerate}
\end{insight}

%% ============================================================================
\section{The Jang Equation}
%% ============================================================================

\subsection{Definition}

Given initial data $(\mathcal{C}^3, g, k)$, the Jang equation seeks $f: \mathcal{C} \to \mathbb{R}$ such that:
\begin{equation}
    H_{\graph(f)} = P_{\graph(f)}
\end{equation}

where the graph is in $\mathcal{C} \times \mathbb{R}$ with metric $g + df^2$.

Equivalently:
\begin{equation}
    \divv\left(\frac{\nabla f}{\sqrt{1 + |\nabla f|^2}}\right) = \left(g^{ij} - \frac{f^i f^j}{1 + |\nabla f|^2}\right)k_{ij}
\end{equation}

\subsection{Key Properties}

\begin{enumerate}
    \item At MOTS: $\theta^+ = H + P = 0$ on the slice means $H = -P$.
    
    The Jang equation $H_{\graph} = P_{\graph}$ relates to this.
    
    \item At trapped surfaces: $\theta^+ < 0$ means $H < -P$, which affects the Jang solution.
    
    \item The Jang surface "blows up" near MOTS - $f \to \pm\infty$ near $\Sigma^*$.
\end{enumerate}

\subsection{The Jang Metric}

Define on the graph:
\begin{equation}
    \hat{g} = g + df \otimes df
\end{equation}

The scalar curvature:
\begin{equation}
    \hat{R} = R_g - 2\Ric_g(\nabla f, \nabla f)/(1+|\nabla f|^2) + |h - k|^2 - (\tr h - \tr k)^2 + \ldots
\end{equation}

where $h$ is the second fundamental form of the graph.

Under DEC: $\hat{R} \ge 0$ with controlled error terms.

%% ============================================================================
\section{The Blow-Up Analysis}
%% ============================================================================

\subsection{Behavior Near MOTS}

Near the MOTS $\Sigma^*$ where $\theta^+ = 0$:
\begin{itemize}
    \item The Jang solution $f$ blows up: $f \to +\infty$
    \item The graph becomes asymptotically cylindrical: $\Sigma^* \times \mathbb{R}$
    \item The Jang surface "caps off" at infinity over $\Sigma^*$
\end{itemize}

\subsection{The Regularization}

To handle the blow-up, we can:
\begin{enumerate}
    \item Cut off the cylinder at height $T$
    \item Cap off with a standard end
    \item Or work with the cylindrical structure directly
\end{enumerate}

%% ============================================================================
\section{Area in the Jang Surface}
%% ============================================================================

\begin{lemma}[Area Comparison]
Let $\Sigma$ be a surface in $\mathcal{C}$ and $\hat{\Sigma} = \graph(f)|_\Sigma$ be its lift to the Jang surface.

Then:
\begin{equation}
    \Area_{\hat{g}}(\hat{\Sigma}) = \int_\Sigma \sqrt{1 + |\nabla f|^2} \, dA_g \ge \Area_g(\Sigma)
\end{equation}

with equality if and only if $f$ is constant on $\Sigma$.
\end{lemma}

\textbf{Implication:} The area in the Jang surface is LARGER than in the original metric.

%% ============================================================================
\section{The MOTS in the Jang Surface}
%% ============================================================================

\begin{insight}
Near the MOTS $\Sigma^*$, the Jang surface is a cylinder $\Sigma^* \times \mathbb{R}$.

Cross-sections of this cylinder all have the same area: $\Area(\Sigma^*)$!

The "horizon" in the Jang picture is this cylinder.
\end{insight}

\subsection{The Riemannian Penrose Inequality on Jang Surface}

The Bray-Khuri proof uses:
\begin{equation}
    M_{\text{ADM}}(\hat{g}) \ge \sqrt{\frac{\Area(\text{horizon in } \hat{g})}{16\pi}}
\end{equation}

The "horizon" is the MOTS $\Sigma^*$, giving:
\begin{equation}
    M_{\text{ADM}} \ge \sqrt{\frac{\Area(\Sigma^*)}{16\pi}}
\end{equation}

This is the MOTS Penrose inequality!

%% ============================================================================
\section{The Trapped Surface in the Jang Surface}
%% ============================================================================

\subsection{Lifting the Trapped Surface}

The trapped surface $\Sigma$ (in $\mathcal{C}$) lifts to $\hat{\Sigma}$ in the Jang surface.

$\hat{\Sigma}$ is NOT on the cylinder - it's in the interior of the Jang surface.

\subsection{Area Relationship}

\begin{equation}
    \Area_{\hat{g}}(\hat{\Sigma}) \ge \Area_g(\Sigma)
\end{equation}

And the cylinder (MOTS) has area $\Area_g(\Sigma^*)$.

\textbf{Question:} Is $\Area_{\hat{g}}(\hat{\Sigma}) \le \Area_g(\Sigma^*)$?

If yes, then:
\begin{equation}
    \Area_g(\Sigma) \le \Area_{\hat{g}}(\hat{\Sigma}) \le \Area_g(\Sigma^*)
\end{equation}

which is Area Dominance!

%% ============================================================================
\section{Ricci Flow on the Jang Surface}
%% ============================================================================

\begin{insight}
Run Ricci flow on the Jang surface $(\hat{\mathcal{C}}, \hat{g})$.

The cylindrical end (over $\Sigma^*$) has special behavior under Ricci flow.
\end{insight}

\subsection{Ricci Flow with Cylindrical Ends}

On the cylinder $\Sigma^* \times \mathbb{R}$:
\begin{itemize}
    \item The cylinder is Ricci-flat if $\Sigma^*$ is flat
    \item For $\Sigma^* = S^2$: positive Ricci, cylinder shrinks
    \item The flow preserves the cylindrical structure (approximately)
\end{itemize}

\subsection{Evolution of $\hat{\Sigma}$}

Under Ricci flow on $\hat{g}$:
\begin{equation}
    \frac{\partial}{\partial t}\Area_{\hat{g}(t)}(\hat{\Sigma}) = -\int_{\hat{\Sigma}} (\Ric_{11} + \Ric_{22}) dA
\end{equation}

The Ricci curvature in the interior vs. on the cylinder determines which shrinks faster.

%% ============================================================================
\section{The Key Comparison}
%% ============================================================================

\begin{lemma}[Comparison Under Ricci Flow]
Let $\hat{g}(t)$ be the Ricci flow on the Jang surface.

Let $A_\Sigma(t) = \Area_{\hat{g}(t)}(\hat{\Sigma})$ and $A_*(t) = \Area_{\hat{g}(t)}(\Sigma^*)$.

Then:
\begin{equation}
    \frac{d}{dt}\left(\frac{A_\Sigma(t)}{A_*(t)}\right) = \frac{A_\Sigma}{A_*}\left(\bar{K}_{\Sigma^*} - \bar{K}_{\hat{\Sigma}}\right)
\end{equation}

where $\bar{K}$ denotes averaged tangential Ricci curvature.
\end{lemma}

\textbf{For the ratio to decrease:} $\bar{K}_{\hat{\Sigma}} > \bar{K}_{\Sigma^*}$.

i.e., $\hat{\Sigma}$ has higher curvature than the cylinder $\Sigma^*$.

\subsection{Curvature Comparison}

On the cylinder $\Sigma^* \times \mathbb{R}$:
\begin{itemize}
    \item Intrinsic curvature of $\Sigma^*$ (if $\Sigma^* = S^2$, then $K = 1/r^2$)
    \item The $\mathbb{R}$ direction is flat
    \item Ricci curvature is concentrated on the $\Sigma^*$ directions
\end{itemize}

In the interior (near $\hat{\Sigma}$):
\begin{itemize}
    \item Curvature from the original $(\mathcal{C}, g)$ geometry
    \item Additional curvature from the Jang deformation
    \item The scalar curvature $\hat{R} \ge 0$ (by Jang construction)
\end{itemize}

%% ============================================================================
\section{The Entropy Approach}
%% ============================================================================

\begin{insight}
Use Perelman's entropy to control the evolution.
\end{insight}

\subsection{Perelman's $\mathcal{F}$-Functional}

\begin{equation}
    \mathcal{F}(\hat{g}, f) = \int_{\hat{\mathcal{C}}} (\hat{R} + |\nabla f|^2) e^{-f} d\hat{V}
\end{equation}

Under Ricci flow coupled with $\frac{\partial f}{\partial t} = -\Delta f$:
\begin{equation}
    \frac{d\mathcal{F}}{dt} = 2\int |\Ric + \nabla^2 f|^2 e^{-f} d\hat{V} \ge 0
\end{equation}

\subsection{Localizing to Surfaces}

For $f$ concentrated near $\hat{\Sigma}$ vs. near $\Sigma^*$:

The entropy "localizes" to these regions and the monotonicity gives control.

%% ============================================================================
\section{The Main Theorem}
%% ============================================================================

\begin{main_theorem}
\textbf{Theorem (Area Dominance via Ricci-Jang Method):}

Let $(\mathcal{C}, g, k)$ be asymptotically flat initial data satisfying DEC.

Let $\Sigma$ be a trapped surface and $\Sigma^*$ the outermost MOTS with 
$\Sigma$ inside $\Sigma^*$.

Assume:
\begin{enumerate}
    \item The Jang equation has a solution with blow-up at $\Sigma^*$
    \item Ricci flow on the Jang surface exists and has controlled behavior
    \item The curvature comparison $\bar{K}_{\hat{\Sigma}} \ge \bar{K}_{\Sigma^*}$ holds
\end{enumerate}

Then $\Area(\Sigma) \le \Area(\Sigma^*)$.
\end{main_theorem}

\begin{proof}
\textbf{Step 1:} Solve the Jang equation to get $(\hat{\mathcal{C}}, \hat{g})$.

\textbf{Step 2:} Note that $\Area_{\hat{g}}(\hat{\Sigma}) \ge \Area_g(\Sigma)$ and 
the cylinder over $\Sigma^*$ has cross-sectional area $\Area_g(\Sigma^*)$.

\textbf{Step 3:} Run Ricci flow on $\hat{g}$.

\textbf{Step 4:} By the curvature comparison, the ratio 
$\Area_{\hat{g}(t)}(\hat{\Sigma})/\Area_{\hat{g}(t)}(\Sigma^*)$ is non-increasing.

\textbf{Step 5:} In the long-time limit, the Jang surface simplifies 
(e.g., becomes rotationally symmetric or has controlled geometry).

\textbf{Step 6:} In the simplified geometry, Area Dominance holds 
(e.g., by explicit calculation or isoperimetric arguments).

\textbf{Step 7:} By monotonicity, Area Dominance holds in the original geometry.
\end{proof}

%% ============================================================================
\section{The Technical Gaps}
%% ============================================================================

\subsection{Gap 1: Curvature Comparison}

The assumption $\bar{K}_{\hat{\Sigma}} \ge \bar{K}_{\Sigma^*}$ needs proof.

This requires understanding how the Jang deformation affects curvature in 
the interior vs. near the MOTS.

\subsection{Gap 2: Ricci Flow on Cylindrical Ends}

Ricci flow on manifolds with cylindrical ends is delicate:
\begin{itemize}
    \item The cylinder may shrink, expand, or stay fixed
    \item Singularities may form
    \item The flow may not converge to a simple limit
\end{itemize}

\subsection{Gap 3: Long-Time Behavior}

The long-time behavior of Ricci flow on the Jang surface needs analysis:
\begin{itemize}
    \item Does it converge?
    \item What is the limiting geometry?
    \item Is Area Dominance obvious in the limit?
\end{itemize}

%% ============================================================================
\section{A Simpler Version}
%% ============================================================================

\begin{insight}
Instead of running Ricci flow, use the Jang surface directly with the 
ISOPERIMETRIC INEQUALITY.
\end{insight}

\subsection{Isoperimetric on Jang Surface}

On $(\hat{\mathcal{C}}, \hat{g})$ with $\hat{R} \ge 0$:

By comparison with Euclidean space (or using Huisken-Ilmanen):
\begin{equation}
    \Area(\partial\Omega) \ge c \cdot \Vol(\Omega)^{2/3}
\end{equation}

\subsection{Applying to $\hat{\Sigma}$}

The surface $\hat{\Sigma}$ bounds a region $\Omega$ in the Jang surface.

The cylinder $\Sigma^*$ bounds the "horizon region."

If $\Omega \subset $ (region bounded by cylinder):
\begin{equation}
    \Area_{\hat{g}}(\hat{\Sigma}) \ge c \cdot \Vol(\Omega)^{2/3}
\end{equation}

But this gives a LOWER bound, not an upper bound!

\subsection{The Issue}

The isoperimetric inequality gives lower bounds on area.

We need an UPPER bound: $\Area(\hat{\Sigma}) \le \Area(\Sigma^*)$.

This requires a different approach.

%% ============================================================================
\section{The Maximum Principle Approach}
%% ============================================================================

\begin{insight}
Use the MAXIMUM PRINCIPLE on the Jang surface.

The trapped surface $\hat{\Sigma}$ lies INSIDE the cylinder.

Show that no surface inside the cylinder can have area larger than the 
cylinder itself.
\end{insight}

\subsection{The Geometric Setting}

The Jang surface $\hat{\mathcal{C}}$ has:
\begin{itemize}
    \item An asymptotically flat end
    \item A cylindrical end over $\Sigma^*$
    \item Non-negative scalar curvature $\hat{R} \ge 0$
\end{itemize}

\subsection{Mean Curvature on the Jang Surface}

On the lifted surface $\hat{\Sigma}$:

The mean curvature $\hat{H}$ relates to the original $H$ and the Jang function $f$.

\textbf{Key:} If $\hat{H} > 0$ on $\hat{\Sigma}$, then $\hat{\Sigma}$ is mean-convex.

Flowing outward increases area until reaching the cylinder.

\subsection{The Crux}

We need to show $\hat{H} > 0$ on $\hat{\Sigma}$.

The Jang equation was designed so that $\hat{H} = \hat{P}$ on the graph.

For trapped surfaces: the relationship between $\hat{H}$ and the trapped 
condition needs analysis.

%% ============================================================================
\section{Conclusion}
%% ============================================================================

The Ricci-Jang synthesis provides a promising framework:

\textbf{Key Insights:}
\begin{enumerate}
    \item Jang surface transforms the problem to Riemannian geometry
    \item The MOTS becomes a cylindrical end with fixed cross-sectional area
    \item Ricci flow or maximum principle methods can be applied
    \item The area comparison becomes a comparison within the Jang surface
\end{enumerate}

\textbf{Remaining Challenges:}
\begin{enumerate}
    \item Proving the curvature comparison
    \item Understanding Ricci flow on the Jang surface
    \item Showing mean-convexity of $\hat{\Sigma}$
    \item Completing the maximum principle argument
\end{enumerate}

\textbf{This synthesis represents the most promising path to Area Dominance.}

\end{document}
