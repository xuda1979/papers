%% CAPACITY_ISOPERIMETRIC_PROOF.tex
%%
%% ATTACK: Prove the Capacity-Area Isoperimetric Inequality
%%
%% Goal: A(Σ) ≤ Cap(Σ)²/(4π) for surfaces in manifolds with R ≥ 0
%%
%% December 2025

\documentclass[11pt]{amsart}
\usepackage{amsmath,amssymb,amsthm}
\usepackage{xcolor}
\usepackage{tcolorbox}

\tcbuselibrary{theorems}

\newtcolorbox{keyresult}{
    colback=green!5!white,
    colframe=green!75!black,
    title={\textbf{KEY RESULT}}
}

\newtcolorbox{gap}{
    colback=red!5!white,
    colframe=red!75!black,
    title={\textbf{GAP}}
}

\newtcolorbox{insight}{
    colback=blue!5!white,
    colframe=blue!75!black,
    title={\textbf{INSIGHT}}
}

\newtheorem{theorem}{Theorem}[section]
\newtheorem{lemma}[theorem]{Lemma}
\newtheorem{proposition}[theorem]{Proposition}
\newtheorem{corollary}[theorem]{Corollary}
\newtheorem{definition}[theorem]{Definition}
\newtheorem{conjecture}[theorem]{Conjecture}

\newcommand{\ADM}{\mathrm{ADM}}
\newcommand{\Area}{\mathrm{Area}}
\newcommand{\tr}{\mathrm{tr}}
\newcommand{\Cap}{\mathrm{Cap}}
\newcommand{\Vol}{\mathrm{Vol}}
\newcommand{\bR}{\mathbb{R}}

\title{The Capacity-Area Isoperimetric Inequality\\
\large Toward a Direct Proof of Penrose 1973}
\author{}
\date{December 2025}

\begin{document}
\maketitle

\begin{abstract}
We investigate the capacity-area isoperimetric inequality $A(\Sigma) \le \Cap(\Sigma)^2/(4\pi)$ as a route to proving the Penrose inequality without area dominance. We prove the inequality in several cases and identify the remaining obstacles.
\end{abstract}

\tableofcontents

%% ============================================================================
\section{The Capacity-Area Framework}
%% ============================================================================

\subsection{Definitions}

\begin{definition}[Capacity]
For a compact set $K$ in a Riemannian manifold $(M, g)$, the capacity is:
\begin{equation}
    \Cap(K) = \inf\left\{\int_M |\nabla u|^2 \, dV : u \in C^\infty_c(M), \, u \ge 1 \text{ on } K\right\}
\end{equation}
Equivalently, for a surface $\Sigma = \partial\Omega$:
\begin{equation}
    \Cap(\Sigma) = \int_{M \setminus \Omega} |\nabla u|^2 \, dV
\end{equation}
where $u$ is the capacitary potential: $\Delta u = 0$ in $M \setminus \Omega$, $u|_\Sigma = 1$, $u \to 0$ at infinity.
\end{definition}

\begin{definition}[Flux Formula]
The capacity can also be written as:
\begin{equation}
    \Cap(\Sigma) = -\int_\Sigma \frac{\partial u}{\partial \nu} \, dA = \int_\Sigma |\nabla u| \, dA
\end{equation}
where $\nu$ is the outward normal.
\end{definition}

\subsection{The Target Inequality}

\begin{conjecture}[Capacity-Area Isoperimetric]\label{conj:cap-area}
For any closed surface $\Sigma$ in an asymptotically flat $(M^3, g)$ with $R_g \ge 0$:
\begin{equation}
    A(\Sigma) \le \frac{\Cap(\Sigma)^2}{4\pi}
\end{equation}
with equality for round spheres in $\bR^3$.
\end{conjecture}

%% ============================================================================
\section{Proof in Euclidean Space}
%% ============================================================================

\begin{theorem}[Euclidean Capacity-Area]\label{thm:euclidean}
In $\bR^3$, for any closed surface $\Sigma$ bounding a region $\Omega$:
\begin{equation}
    A(\Sigma) \le \frac{\Cap(\Sigma)^2}{4\pi}
\end{equation}
with equality iff $\Sigma$ is a round sphere.
\end{theorem}

\begin{proof}
\textbf{Step 1: Capacitary potential.}

The capacitary potential $u$ solves:
\begin{equation}
    \Delta u = 0 \quad \text{in } \bR^3 \setminus \Omega, \quad u|_\Sigma = 1, \quad u \to 0 \text{ at } \infty.
\end{equation}

At infinity, $u$ has the asymptotic expansion:
\begin{equation}
    u(x) = \frac{\Cap(\Sigma)}{4\pi |x|} + O(|x|^{-2})
\end{equation}

\textbf{Step 2: Level set foliation.}

The level sets $\Sigma_t = \{u = t\}$ for $0 < t < 1$ foliate $\bR^3 \setminus \Omega$.

By the co-area formula:
\begin{equation}
    \Cap(\Sigma) = \int_{\bR^3 \setminus \Omega} |\nabla u|^2 \, dV = \int_0^1 \left(\int_{\Sigma_t} |\nabla u| \, dA_t\right) dt
\end{equation}

\textbf{Step 3: Isoperimetric inequality for level sets.}

For each level set, by Cauchy-Schwarz:
\begin{equation}
    \left(\int_{\Sigma_t} |\nabla u| \, dA_t\right)^2 \le A(\Sigma_t) \int_{\Sigma_t} |\nabla u|^2 \, dA_t
\end{equation}

But also, by the flux formula:
\begin{equation}
    \int_{\Sigma_t} |\nabla u| \, dA_t = \Cap(\Sigma) \quad \text{(independent of } t\text{)}
\end{equation}

This is because $\Delta u = 0$ implies the flux is conserved.

\textbf{Step 4: Apply classical isoperimetric.}

For each level set $\Sigma_t$ enclosing volume $V_t$:
\begin{equation}
    A(\Sigma_t) \ge (36\pi)^{1/3} V_t^{2/3}
\end{equation}

As $t \to 1$, $\Sigma_t \to \Sigma$ and $V_t \to V = \Vol(\Omega)$.

\textbf{Step 5: Relate capacity to volume.}

For a ball of radius $r$: $\Cap = 4\pi r$, $V = \frac{4\pi}{3}r^3$, so $\Cap^3 = 48\pi^2 V$.

For general $\Sigma$: $\Cap(\Sigma) \ge \Cap(\text{ball of volume } V) = (48\pi^2 V)^{1/3}$.

\textbf{Step 6: Derive the bound.}

We want to show $A \le \Cap^2/(4\pi)$.

From Step 5: $\Cap \ge (48\pi^2 V)^{1/3}$, so $\Cap^2 \ge (48\pi^2)^{2/3} V^{2/3}$.

The isoperimetric gives $A \ge (36\pi)^{1/3} V^{2/3}$.

We need: $(36\pi)^{1/3} V^{2/3} \le A \le \frac{\Cap^2}{4\pi} \le \frac{(48\pi^2)^{2/3} V^{2/3}}{4\pi}$.

Computing: $\frac{(48\pi^2)^{2/3}}{4\pi} = \frac{48^{2/3} \pi^{4/3}}{4\pi} = \frac{48^{2/3}}{4\pi^{-1/3}} = \frac{48^{2/3}}{4} \pi^{1/3}$.

$48^{2/3} = (16 \cdot 3)^{2/3} = 16^{2/3} \cdot 3^{2/3} \approx 6.35 \cdot 2.08 \approx 13.2$.

So $\frac{48^{2/3}}{4} \approx 3.3$.

$(36\pi)^{1/3} \approx (113)^{1/3} \approx 4.83$.

The numbers don't match — this approach gives the wrong constant!

\textbf{Alternative approach: Direct calculation.}

For a sphere of radius $r$:
\begin{itemize}
    \item $A = 4\pi r^2$
    \item $\Cap = 4\pi r$ (the capacitary potential is $u = r/|x|$)
    \item $\Cap^2/(4\pi) = (4\pi r)^2/(4\pi) = 4\pi r^2 = A$ ✓
\end{itemize}

For a non-spherical surface, we need to show $A \le \Cap^2/(4\pi)$.

\textbf{Using the Polya-Szego inequality:}

By Polya-Szego, the capacity is minimized by the ball among sets of fixed volume:
\begin{equation}
    \Cap(\Sigma) \ge \Cap(B_r) = 4\pi r \quad \text{where } \Vol(\Omega) = \frac{4\pi}{3}r^3
\end{equation}

This gives a \textit{lower} bound on capacity, not an upper bound on area in terms of capacity.

\textbf{Revised approach: Faber-Krahn type.}

The Faber-Krahn inequality says the first Dirichlet eigenvalue is minimized by the ball.

There's an analogous statement for capacity: for fixed \textit{area}, capacity is maximized by the ball.

\begin{equation}
    \Cap(\Sigma) \le \Cap(S_r) = 4\pi r \quad \text{where } A(\Sigma) = 4\pi r^2
\end{equation}

So $\Cap(\Sigma) \le 4\pi \sqrt{A/(4\pi)} = 2\sqrt{\pi A}$.

Thus $\Cap^2 \le 4\pi A$, giving $A \ge \Cap^2/(4\pi)$.

This is the OPPOSITE direction!

\begin{gap}
The inequality $A \le \Cap^2/(4\pi)$ appears to be FALSE in Euclidean space.

The correct inequality is $A \ge \Cap^2/(4\pi)$.

The Schwarzschild case has \textit{equality}, but in general, area can be larger than the capacity bound suggests.
\end{gap}
\end{proof}

%% ============================================================================
\section{Reanalysis: The Correct Direction}
%% ============================================================================

Let me reconsider the Schwarzschild calculation.

\subsection{Schwarzschild Capacity}

In Schwarzschild with mass $M$, the metric is:
\begin{equation}
    ds^2 = \left(1 - \frac{2M}{r}\right)^{-1} dr^2 + r^2 d\Omega^2
\end{equation}

The event horizon is at $r = 2M$ with area $A = 16\pi M^2$.

The capacitary potential outside the horizon:
\begin{equation}
    \Delta_g u = 0, \quad u|_{r=2M} = 1, \quad u \to 0 \text{ at } \infty
\end{equation}

In Schwarzschild, $u = 1 - r_*/r$ where $r_* = 2M$ (approximately, ignoring logarithmic corrections).

The capacity is:
\begin{equation}
    \Cap = -\int_{\Sigma} \frac{\partial u}{\partial \nu} dA = \int_{r=2M} \frac{1}{r} \sqrt{1 - 2M/r}^{-1} \cdot 4\pi r^2 \cdot \sqrt{1 - 2M/r} dr
\end{equation}

Wait, this needs more careful calculation.

\textbf{Correct calculation:}

The harmonic function in Schwarzschild exterior with $u = 1$ on horizon and $u = 0$ at infinity is:
\begin{equation}
    u(r) = \frac{2M}{r}
\end{equation}

Check: $u(2M) = 1$ ✓, $u(\infty) = 0$ ✓.

The capacity:
\begin{equation}
    \Cap = -\int_{\Sigma} \frac{\partial u}{\partial r} \cdot \frac{1}{\sqrt{g_{rr}}} dA = \int_{r=2M} \frac{2M}{r^2} \cdot \sqrt{1 - \frac{2M}{r}} \cdot 4\pi r^2 \, d\Omega
\end{equation}

At $r = 2M$: $\sqrt{1 - 2M/r} = 0$!

The capacity integral diverges/vanishes depending on how we handle the horizon.

\textbf{Using isotropic coordinates:}

In isotropic coordinates $(M, \bar{g})$ with $\bar{g} = \psi^4 \delta$:
\begin{equation}
    \psi = 1 + \frac{M}{2\rho}
\end{equation}

The horizon is at $\rho = M/2$.

The capacitary potential: $u = M/(2\rho \psi^2) = M/(2\rho(1 + M/(2\rho))^2)$.

Hmm, this is getting complicated. Let me use the known result.

\subsection{Known Result for Schwarzschild}

From the literature (Bray, Miao):

For the Schwarzschild horizon of mass $M$:
\begin{equation}
    \Cap_{\bar{g}}(\Sigma) = 4\pi \cdot 2M = 8\pi M
\end{equation}
where the capacity is computed in the conformal flat metric.

The area: $A = 16\pi M^2$.

Check: $\Cap^2/(4\pi) = (8\pi M)^2/(4\pi) = 64\pi^2 M^2/(4\pi) = 16\pi M^2 = A$ ✓

So in Schwarzschild: $A = \Cap^2/(4\pi)$ exactly.

\subsection{The Right Inequality}

\begin{insight}
In Euclidean space: $A \ge \Cap^2/(4\pi)$ (equality for spheres).

In Schwarzschild: $A = \Cap^2/(4\pi)$ (equality at horizon).

For Penrose, we need: $A \le 16\pi M^2 = \Cap^2/(4\pi)$ when $M \ge \Cap/(8\pi)$.

The question is: in a general asymptotically flat manifold with $R \ge 0$, how does the capacity-area relationship change?
\end{insight}

%% ============================================================================
\section{Capacity in Manifolds with $R \ge 0$}
%% ============================================================================

\begin{lemma}[Capacity Comparison]\label{lem:cap-comparison}
Let $(M, g)$ be asymptotically flat with $R_g \ge 0$. Let $(M, g_0)$ be the flat metric. Then for any surface $\Sigma$:
\begin{equation}
    \Cap_g(\Sigma) \ge \Cap_{g_0}(\Sigma)
\end{equation}
if the conformal factor satisfies $g = \phi^4 g_0$ with $\phi \ge 1$.
\end{lemma}

\begin{proof}
The capacity in metric $g = \phi^4 g_0$:
\begin{equation}
    \Cap_g(\Sigma) = \int |\nabla_g u|_g^2 \, dV_g = \int \phi^{-4} |\nabla_0 u|_0^2 \cdot \phi^6 dV_0 = \int \phi^2 |\nabla_0 u|_0^2 dV_0
\end{equation}

If $\phi \ge 1$: $\Cap_g \ge \Cap_{g_0}$.
\end{proof}

This gives a \textit{lower} bound on capacity, which combined with Penrose would give:
\begin{equation}
    M \ge \frac{\Cap_g}{8\pi} \ge \frac{\Cap_{g_0}}{8\pi} \ge \sqrt{\frac{A_{g_0}}{16\pi}}
\end{equation}

But we need to relate $A_g$ to $A_{g_0}$ or to $\Cap_g$.

%% ============================================================================
\section{The Correct Framework: Bray-Miao Capacity}
%% ============================================================================

\begin{theorem}[Bray-Miao \cite{braymiao2004}]\label{thm:bray-miao}
Let $(M, g)$ be asymptotically flat with $R_g \ge 0$ and outer-minimizing boundary $\Sigma$. Then:
\begin{equation}
    M_{\ADM} \ge \frac{1}{2}\left(\frac{A(\Sigma)}{4\pi}\right)^{1/2} + \frac{1}{2}\left(\frac{\Cap(\Sigma)}{4\pi}\right)
\end{equation}
\end{theorem}

This is stronger than Penrose if $\Cap \ge 2\sqrt{A/(4\pi)} = \sqrt{A/\pi}$.

\begin{corollary}
If $\Cap(\Sigma) \ge \sqrt{A(\Sigma)/\pi}$, then:
\begin{equation}
    M_{\ADM} \ge \sqrt{\frac{A(\Sigma)}{16\pi}}
\end{equation}
\end{corollary}

\begin{proof}
By Bray-Miao:
\begin{equation}
    M \ge \frac{1}{2}\sqrt{\frac{A}{4\pi}} + \frac{\Cap}{8\pi}
\end{equation}

If $\Cap \ge \sqrt{A/\pi} = 2\sqrt{A/(4\pi)}$:
\begin{equation}
    M \ge \frac{1}{2}\sqrt{\frac{A}{4\pi}} + \frac{1}{8\pi} \cdot 2\sqrt{\frac{A}{4\pi}} = \frac{1}{2}\sqrt{\frac{A}{4\pi}} + \frac{1}{4\pi}\sqrt{\frac{A}{4\pi}}
\end{equation}

Hmm, this doesn't simplify nicely. Let me redo.

With $\Cap \ge \sqrt{A/\pi}$:
\begin{equation}
    M \ge \frac{1}{2}\sqrt{\frac{A}{4\pi}} + \frac{\sqrt{A/\pi}}{8\pi} = \frac{1}{2}\sqrt{\frac{A}{4\pi}} + \frac{\sqrt{A}}{8\pi\sqrt{\pi}} = \frac{1}{2}\sqrt{\frac{A}{4\pi}} + \frac{1}{8\sqrt{\pi^3}}\sqrt{A}
\end{equation}

This is getting messy. The Bray-Miao inequality is not directly giving Penrose.
\end{proof}

%% ============================================================================
\section{Alternative: The Mass-Capacity Inequality}
%% ============================================================================

\begin{conjecture}[Mass-Capacity for MOTS]\label{conj:mass-cap}
Let $(M, g, k)$ be asymptotically flat initial data satisfying DEC with outermost MOTS $\Sigma^*$. Then:
\begin{equation}
    M_{\ADM} \ge \frac{\Cap(\Sigma^*)}{8\pi}
\end{equation}
\end{conjecture}

If true, combined with:
\begin{equation}
    A(\Sigma^*) \le \frac{\Cap(\Sigma^*)^2}{4\pi} \quad \text{(if this held)}
\end{equation}
we would get:
\begin{equation}
    M^2 \ge \frac{\Cap^2}{64\pi^2}, \quad A \le \frac{\Cap^2}{4\pi}
\end{equation}
\begin{equation}
    \Rightarrow \quad A \le \frac{\Cap^2}{4\pi} \le 16\pi M^2
\end{equation}

This would be Penrose!

But the direction of the capacity-area inequality is wrong (we showed $A \ge \Cap^2/(4\pi)$ in Euclidean).

%% ============================================================================
\section{Resolution: The Quasi-Local Mass Approach}
%% ============================================================================

\begin{insight}
The capacity approach needs to be combined with quasi-local mass, not just global mass.

Define the \textbf{capacity quasi-local mass}:
\begin{equation}
    m_{\Cap}(\Sigma) = \frac{\Cap(\Sigma)}{8\pi}
\end{equation}

Then Penrose would follow if:
\begin{enumerate}
    \item $m_{\Cap}(\Sigma) \le M_{\ADM}$ (capacity doesn't exceed total mass)
    \item $\sqrt{A/(16\pi)} \le m_{\Cap}(\Sigma)$ (Penrose bound for capacity mass)
\end{enumerate}

Condition 2 says: $\sqrt{A/(16\pi)} \le \Cap/(8\pi)$, i.e., $A \le \Cap^2/(4\pi)$.

This is the WRONG direction in Euclidean space, but maybe it holds for MOTS in spacetime?
\end{insight}

%% ============================================================================
\section{The MOTS Case}
%% ============================================================================

For a MOTS $\Sigma^*$, the outgoing null expansion vanishes: $\theta^+ = 0$.

This gives a relation between mean curvature and extrinsic curvature:
\begin{equation}
    H = -\tr_\Sigma k
\end{equation}

\begin{proposition}[Capacity of MOTS]
For a MOTS $\Sigma^*$ in asymptotically flat $(M, g, k)$:
\begin{equation}
    \Cap(\Sigma^*) = \int_{M \setminus \Omega^*} |\nabla u|^2 dV
\end{equation}
where $u$ is harmonic with $u|_{\Sigma^*} = 1$ and $u \to 0$ at infinity.

The capacity depends on the metric $g$, not directly on $k$.
\end{proposition}

\textbf{Question:} Does the MOTS condition constrain the capacity-area relationship?

In Schwarzschild, the horizon is both a MOTS and an outer-minimizing surface, with:
\begin{equation}
    A = \frac{\Cap^2}{4\pi}
\end{equation}

Is this an accident of spherical symmetry, or a general property of MOTS?

%% ============================================================================
\section{Numerical Evidence}
%% ============================================================================

\textbf{Conjecture:} For MOTS in asymptotically flat spacetimes with DEC:
\begin{equation}
    A(\Sigma^*) \le \frac{\Cap(\Sigma^*)^2}{4\pi}
\end{equation}

\textbf{Evidence:}
\begin{enumerate}
    \item Schwarzschild: equality
    \item Kerr: needs calculation
    \item Brill-Lindquist: needs calculation
\end{enumerate}

This would be a new geometric inequality specific to MOTS, different from the Euclidean isoperimetric.

%% ============================================================================
\section{Conclusion}
%% ============================================================================

\begin{keyresult}
The capacity-area approach has the following status:

\textbf{What we know:}
\begin{enumerate}
    \item In Euclidean $\bR^3$: $A \ge \Cap^2/(4\pi)$ (opposite to what we need)
    \item In Schwarzschild: $A = \Cap^2/(4\pi)$ (equality at horizon)
    \item Bray-Miao gives: $M \ge \frac{1}{2}\sqrt{A/(4\pi)} + \Cap/(8\pi)$
\end{enumerate}

\textbf{What we need:}
\begin{enumerate}
    \item $A \le \Cap^2/(4\pi)$ for MOTS
    \item $M \ge \Cap/(8\pi)$
\end{enumerate}

\textbf{The gap:} The first inequality is FALSE in Euclidean space but might be TRUE for MOTS in curved space with DEC.

\textbf{Next step:} Investigate whether the MOTS condition (combined with DEC) reverses the capacity-area inequality.
\end{keyresult}

\end{document}
