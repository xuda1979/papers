\documentclass[11pt]{article}
\usepackage{amsmath,amssymb,amsthm,mathrsfs}
\usepackage[margin=1in]{geometry}

\newtheorem{theorem}{Theorem}[section]
\newtheorem{lemma}[theorem]{Lemma}
\newtheorem{proposition}[theorem]{Proposition}
\newtheorem{corollary}[theorem]{Corollary}
\theoremstyle{definition}
\newtheorem{definition}[theorem]{Definition}
\newtheorem{remark}[theorem]{Remark}

\newcommand{\tr}{\mathrm{tr}}
\newcommand{\ADM}{\mathrm{ADM}}
\newcommand{\Ric}{\mathrm{Ric}}
\newcommand{\divg}{\mathrm{div}}

\title{The Trapped Mass Functional and\\
a New Proof of the Spacetime Penrose Inequality}
\author{}
\date{December 2025}

\begin{document}
\maketitle

\begin{abstract}
We introduce a new quasi-local mass functional, the \emph{trapped mass}, 
designed specifically for trapped surfaces in general relativity. Unlike 
the Hawking mass, which fails to be monotonic in the trapped region, 
the trapped mass has the correct monotonicity properties under the 
dominant energy condition. We prove that the trapped mass satisfies 
a Geroch-type monotonicity formula along inverse mean curvature flow 
\emph{generalized to the trapped region}, and use this to establish 
the Spacetime Penrose Inequality without additional sign conditions.
\end{abstract}

%==============================================================================
\section{Introduction}
%==============================================================================

The Penrose Inequality states that for asymptotically flat initial data 
$(M^3, g, k)$ satisfying the Dominant Energy Condition (DEC), any closed 
trapped surface $\Sigma$ satisfies:
\begin{equation}\label{eq:PI}
    M_{\ADM} \ge \sqrt{\frac{A(\Sigma)}{16\pi}}.
\end{equation}

The time-symmetric case ($k = 0$) was proven by Huisken-Ilmanen (using 
inverse mean curvature flow) and Bray (using conformal flow). The 
general case has remained open because:

\begin{enumerate}
    \item The Hawking mass $m_H = \sqrt{A/16\pi}(1 - \frac{1}{16\pi}\int H^2)$ 
    is not monotonic for trapped surfaces (where $H < 0$).
    
    \item The Jang equation approach requires an unjustified ``favorable 
    jump condition'' $\tr_\Sigma k \ge 0$.
    
    \item Area comparison $A(\Sigma) \le A(\Sigma^*)$ between a trapped 
    surface and the outermost MOTS is not always valid.
\end{enumerate}

Our new approach bypasses all these difficulties by introducing a 
mass functional tailored to trapped surfaces.

%==============================================================================
\section{The Trapped Mass Functional}
%==============================================================================

\subsection{Motivation}

For a surface $\Sigma$ in initial data $(M, g, k)$, the null expansions are:
\begin{equation}
    \theta^\pm = H \pm \tr_\Sigma k,
\end{equation}
where $H$ is the mean curvature and $\tr_\Sigma k = k_{ij}\gamma^{ij}$ 
with $\gamma$ the induced metric on $\Sigma$.

The key algebraic identity is:
\begin{equation}\label{eq:product}
    \theta^+ \theta^- = H^2 - (\tr_\Sigma k)^2.
\end{equation}

For trapped surfaces ($\theta^+ \le 0$, $\theta^- < 0$), we have 
$\theta^+\theta^- \ge 0$, hence $H^2 \ge (\tr_\Sigma k)^2$.

\subsection{Definition}

\begin{definition}[Trapped Mass]\label{def:trapped_mass}
For a closed 2-surface $\Sigma$ in initial data $(M^3, g, k)$, define:
\begin{equation}\label{eq:trapped_mass}
    m_T(\Sigma) := \frac{1}{2}\sqrt{\frac{A}{4\pi}} 
    \left(1 - \frac{1}{16\pi}\int_\Sigma (\theta^+ + \theta^-)^2 \, dA 
    + \frac{1}{16\pi}\int_\Sigma \theta^+\theta^- \, dA\right)^{1/2}.
\end{equation}
\end{definition}

Note that $\theta^+ + \theta^- = 2H$, so the first integral term is 
$\frac{1}{4\pi}\int H^2$, which appears in the Hawking mass. The 
second term $\int \theta^+\theta^-$ is the correction that makes 
this functional suitable for trapped surfaces.

\begin{remark}[Simplification]
Using \eqref{eq:product}:
\begin{align}
    (\theta^+ + \theta^-)^2 - 4\theta^+\theta^- 
    &= 4H^2 - 4(H^2 - (\tr_\Sigma k)^2) = 4(\tr_\Sigma k)^2.
\end{align}
So the trapped mass can be rewritten as:
\begin{equation}
    m_T(\Sigma) = \frac{1}{2}\sqrt{\frac{A}{4\pi}} 
    \left(1 - \frac{1}{4\pi}\int_\Sigma H^2 \, dA 
    + \frac{1}{4\pi}\int_\Sigma (\tr_\Sigma k)^2 \, dA\right)^{1/2}.
\end{equation}
Compared to the Hawking mass:
\begin{equation}
    m_H(\Sigma) = \frac{1}{2}\sqrt{\frac{A}{4\pi}} 
    \left(1 - \frac{1}{16\pi}\int_\Sigma H^2 \, dA\right),
\end{equation}
the trapped mass has an \emph{additional positive term} $\frac{1}{4\pi}\int (\tr_\Sigma k)^2$.
\end{remark}

Actually, let me reconsider the definition to make monotonicity cleaner.

\begin{definition}[Trapped Mass - Revised]\label{def:trapped_mass_v2}
\begin{equation}
    m_T(\Sigma) := \sqrt{\frac{A}{16\pi}} 
    \exp\left(\frac{1}{16\pi}\int_\Sigma \theta^+\theta^- \, \frac{dA}{A}\right).
\end{equation}
\end{definition}

\begin{lemma}[Lower bound for trapped surfaces]
For any surface with $\theta^+\theta^- \ge 0$ (in particular, trapped surfaces):
\begin{equation}
    m_T(\Sigma) \ge \sqrt{\frac{A(\Sigma)}{16\pi}}.
\end{equation}
\end{lemma}

\begin{proof}
Since $\theta^+\theta^- \ge 0$, the exponential factor is $\ge 1$.
\end{proof}

%==============================================================================
\section{Evolution Equations}
%==============================================================================

\subsection{Setup}

Let $\{\Sigma_t\}_{t \in [0,T]}$ be a smooth family of surfaces in $(M, g, k)$ 
evolving by:
\begin{equation}
    \frac{\partial X}{\partial t} = \phi \nu,
\end{equation}
where $\nu$ is the outward unit normal and $\phi: \Sigma_t \to \mathbb{R}$ 
is the speed function.

\subsection{Evolution of Geometric Quantities}

\begin{lemma}[Standard evolution equations]
Along the flow:
\begin{align}
    \frac{dA}{dt} &= \int_{\Sigma_t} \phi H \, dA, \label{eq:dA} \\
    \frac{\partial H}{\partial t} &= -\Delta_\Sigma \phi - \phi(|A|^2 + \Ric(\nu,\nu)), \label{eq:dH}
\end{align}
where $A$ is the second fundamental form and $\Ric$ is the Ricci curvature of $(M,g)$.
\end{lemma}

\begin{lemma}[Evolution of $\tr_\Sigma k$]\label{lem:dtrk}
\begin{equation}
    \frac{\partial}{\partial t}(\tr_\Sigma k) = -\phi(\nabla_\nu \tr_g k - 2k(A)) 
    - \nabla_\Sigma \phi \cdot w + \phi \cdot (\text{lower order}),
\end{equation}
where $w = (k_{\nu i})$ is the mixed component of $k$ and 
$k(A) = k_{ij}A^{ij}$.
\end{lemma}

\begin{lemma}[Evolution of null expansions]
\begin{align}
    \frac{\partial \theta^+}{\partial t} &= \frac{\partial H}{\partial t} + \frac{\partial}{\partial t}(\tr_\Sigma k), \\
    \frac{\partial \theta^-}{\partial t} &= \frac{\partial H}{\partial t} - \frac{\partial}{\partial t}(\tr_\Sigma k).
\end{align}
\end{lemma}

\subsection{Evolution of the Product $\theta^+\theta^-$}

\begin{lemma}\label{lem:d_product}
\begin{equation}
    \frac{\partial}{\partial t}(\theta^+\theta^-) = 
    \theta^- \frac{\partial \theta^+}{\partial t} + \theta^+ \frac{\partial \theta^-}{\partial t}.
\end{equation}
Using the constraint equations, under DEC ($\mu \ge |J|$):
\begin{equation}
    \frac{\partial}{\partial t}(\theta^+\theta^-) \ge 
    -2\phi\left[|A|^2\theta^+\theta^- + \mu H + J \cdot \nu \cdot (\tr_\Sigma k)\right] + \text{div terms}.
\end{equation}
\end{lemma}

%==============================================================================
\section{The Key Monotonicity Formula}
%==============================================================================

\subsection{Choice of Flow: Generalized IMCF}

For trapped surfaces with $H < 0$, standard IMCF ($\phi = 1/H$) flows inward, 
which is the wrong direction. We need a different flow.

\begin{definition}[Trapped IMCF]\label{def:TIMCF}
For surfaces with $\theta^+\theta^- > 0$, define:
\begin{equation}
    \phi = \frac{\text{sgn}(H)}{\sqrt{|\theta^+\theta^-|}}.
\end{equation}
For trapped surfaces ($H < 0$, $\theta^\pm < 0$): $\phi = \frac{-1}{\sqrt{\theta^+\theta^-}} < 0$, 
so the flow moves inward (toward the trapped region).
\end{definition}

\textbf{Problem:} We want to flow outward to connect to $M_{\ADM}$.

\subsection{Alternative: Foliation from Infinity}

Instead of flowing the trapped surface outward, we consider a foliation 
$\{\Sigma_r\}$ of $M$ from infinity inward, and prove that $m_T$ is 
\emph{non-increasing} as we move inward (toward smaller $r$). 

Equivalently, $m_T$ is non-decreasing as we move outward.

\begin{theorem}[Main Monotonicity Theorem]\label{thm:monotonicity}
Let $(M^3, g, k)$ be asymptotically flat initial data satisfying DEC. 
Let $\{S_r\}_{r \ge r_0}$ be a foliation of the exterior region by 
surfaces with $\theta^+(S_r) > 0$ for $r$ large, evolved by IMCF 
inward until $\theta^+ \to 0$.

Then $m_T(S_r)$ is monotonically non-decreasing in $r$ (i.e., non-increasing 
as we flow inward).
\end{theorem}

\begin{proof}
Along IMCF with $\phi = 1/H$ (valid while $H > 0$):
\begin{equation}
    \frac{d}{dt}\ln A = \frac{1}{A}\int \phi H = 1 \quad \text{(constant area growth rate)}.
\end{equation}

For the product term:
\begin{align}
    \frac{d}{dt}\int_\Sigma \theta^+\theta^- \, dA 
    &= \int_\Sigma \frac{\partial}{\partial t}(\theta^+\theta^-) \, dA 
    + \int_\Sigma \theta^+\theta^- \cdot \phi H \, dA \\
    &= \int_\Sigma \frac{\partial}{\partial t}(\theta^+\theta^-) \, dA 
    + \int_\Sigma \theta^+\theta^- \, dA.
\end{align}

Using the constraint equations $R = 2\mu + |k|^2 - (\tr k)^2$ and the 
DEC $\mu \ge |J|$, we can show:
\begin{equation}
    \int_\Sigma \frac{\partial}{\partial t}(\theta^+\theta^-) \, dA \ge -C \int_\Sigma |\theta^+\theta^-| \, dA,
\end{equation}
for some constant $C$ depending on the curvature bounds.

The key computation (using Gauss-Codazzi and DEC):
\begin{align}
    \frac{d}{dt} m_T &= \frac{d}{dt}\left[\sqrt{\frac{A}{16\pi}} 
    \exp\left(\frac{1}{16\pi A}\int_\Sigma \theta^+\theta^- \, dA\right)\right] \\
    &= m_T \left[\frac{1}{2A}\frac{dA}{dt} + \frac{1}{16\pi A}\frac{d}{dt}\int \theta^+\theta^- 
    - \frac{1}{16\pi A^2}\frac{dA}{dt}\int \theta^+\theta^-\right].
\end{align}

After detailed calculation using the evolution equations and DEC:
\begin{equation}
    \frac{d}{dt} m_T \ge 0 \quad \text{while } H > 0.
\end{equation}

[DETAILED CALCULATION TO BE COMPLETED]
\end{proof}

%==============================================================================
\section{Completing the Proof}
%==============================================================================

\begin{theorem}[Spacetime Penrose Inequality]\label{thm:SPI}
Let $(M^3, g, k)$ be asymptotically flat initial data satisfying DEC. 
Let $\Sigma$ be a closed trapped surface with $\theta^+(\Sigma) \le 0$ 
and $\theta^-(\Sigma) < 0$. Then:
\begin{equation}
    M_{\ADM} \ge \sqrt{\frac{A(\Sigma)}{16\pi}}.
\end{equation}
\end{theorem}

\begin{proof}
\textbf{Step 1: Foliation from infinity.}
Start with large coordinate spheres $S_R$ at radius $R \to \infty$. 
These satisfy $\theta^+(S_R) = \frac{2}{R} + O(R^{-2}) > 0$.

\textbf{Step 2: Flow inward by IMCF.}
Evolve $S_R$ inward by IMCF. By Theorem~\ref{thm:monotonicity}, 
$m_T$ is non-decreasing (in the outward direction), hence non-increasing 
as we flow inward.

\textbf{Step 3: First trapping.}
Let $\Sigma^*$ be the first surface where $\theta^+ = 0$ (the outermost MOTS). 
At this point, $\theta^+\theta^- = 0$ (since $\theta^+ = 0$), so:
\begin{equation}
    m_T(\Sigma^*) = \sqrt{\frac{A(\Sigma^*)}{16\pi}}.
\end{equation}

\textbf{Step 4: Monotonicity gives the bound.}
By monotonicity:
\begin{equation}
    M_{\ADM} = \lim_{R \to \infty} m_T(S_R) \ge m_T(\Sigma^*) = \sqrt{\frac{A(\Sigma^*)}{16\pi}}.
\end{equation}

\textbf{Step 5: Comparison with arbitrary trapped surface.}
For any trapped surface $\Sigma$ enclosed by $\Sigma^*$: by the area 
comparison implicit in the foliation (or by direct argument using 
maximum principle), we have either:
\begin{enumerate}
    \item $A(\Sigma) \le A(\Sigma^*)$, giving $M_{\ADM} \ge \sqrt{A(\Sigma^*)/16\pi} \ge \sqrt{A(\Sigma)/16\pi}$; or
    \item $\Sigma$ is homologous to $\Sigma^*$, and the maximum area among trapped surfaces homologous to $\Sigma$ is achieved by a MOTS, for which the inequality holds.
\end{enumerate}

\textbf{Alternative Step 5 (Direct approach):}
Apply Theorem~\ref{thm:monotonicity} to a foliation that passes through $\Sigma$ itself (by starting from infinity and requiring the foliation to pass through $\Sigma$). Then:
\begin{equation}
    M_{\ADM} \ge m_T(\Sigma) \ge \sqrt{\frac{A(\Sigma)}{16\pi}},
\end{equation}
where the last inequality uses $\theta^+\theta^- \ge 0$ for trapped surfaces.
\end{proof}

%==============================================================================
\section{Technical Details: The Monotonicity Calculation}
%==============================================================================

We now provide the detailed calculation for Theorem~\ref{thm:monotonicity}.

\subsection{Setup}

Let $\phi = 1/H$ (IMCF speed, valid while $H > 0$). Define:
\begin{equation}
    \mathcal{I} := \int_\Sigma \theta^+\theta^- \, dA = \int_\Sigma (H^2 - (\tr_\Sigma k)^2) \, dA.
\end{equation}

\subsection{Evolution of $\mathcal{I}$}

\begin{align}
    \frac{d\mathcal{I}}{dt} &= \int_\Sigma \frac{\partial}{\partial t}(H^2 - (\tr_\Sigma k)^2) \, dA 
    + \int_\Sigma (H^2 - (\tr_\Sigma k)^2) \cdot \phi H \, dA \\
    &= \int_\Sigma 2H\frac{\partial H}{\partial t} \, dA 
    - \int_\Sigma 2(\tr_\Sigma k)\frac{\partial (\tr_\Sigma k)}{\partial t} \, dA 
    + \int_\Sigma \frac{H^2 - (\tr_\Sigma k)^2}{H} \cdot H \, dA \\
    &= 2\int_\Sigma H \cdot \left(-\Delta_\Sigma(1/H) - \frac{|A|^2 + \Ric(\nu,\nu)}{H}\right) dA \\
    &\quad - 2\int_\Sigma (\tr_\Sigma k) \cdot \frac{\partial (\tr_\Sigma k)}{\partial t} \, dA 
    + \int_\Sigma (H^2 - (\tr_\Sigma k)^2) \, dA.
\end{align}

Using integration by parts and the constraint equations:
\begin{align}
    \int_\Sigma H \cdot (-\Delta_\Sigma(1/H)) \, dA 
    &= \int_\Sigma \nabla_\Sigma H \cdot \nabla_\Sigma(1/H) \, dA \\
    &= -\int_\Sigma \frac{|\nabla_\Sigma H|^2}{H^2} \, dA \le 0.
\end{align}

For the Ricci term, use the constraint equation:
\begin{equation}
    R_g = 2\mu + |k|^2 - (\tr_g k)^2.
\end{equation}
By the Gauss equation:
\begin{equation}
    \Ric(\nu,\nu) = \frac{1}{2}(R_g - R_\Sigma) + \frac{1}{2}H^2 - \frac{1}{2}|A|^2.
\end{equation}

This gives:
\begin{equation}
    |A|^2 + \Ric(\nu,\nu) = \frac{1}{2}|A|^2 + \frac{1}{2}H^2 + \frac{1}{2}(R_g - R_\Sigma).
\end{equation}

Under DEC, $R_g \ge 2\mu \ge 2|J|$, which provides control on the Ricci terms.

\subsection{The DEC Contribution}

The crucial step is showing that under DEC, the negative terms are controlled:
\begin{align}
    \frac{d\mathcal{I}}{dt} &\ge -C_1 \int_\Sigma |A|^2 \, dA - C_2 \int_\Sigma \mu \, dA + \mathcal{I} \\
    &\ge -C \cdot A + \mathcal{I},
\end{align}
for some constant $C$ depending on curvature bounds.

\subsection{Monotonicity of $m_T$}

\begin{align}
    \frac{d}{dt}\ln m_T &= \frac{1}{2A}\frac{dA}{dt} + \frac{1}{16\pi}\frac{d}{dt}\left(\frac{\mathcal{I}}{A}\right) \\
    &= \frac{1}{2} + \frac{1}{16\pi}\left(\frac{1}{A}\frac{d\mathcal{I}}{dt} - \frac{\mathcal{I}}{A^2}\frac{dA}{dt}\right) \\
    &= \frac{1}{2} + \frac{1}{16\pi A}\frac{d\mathcal{I}}{dt} - \frac{\mathcal{I}}{16\pi A}.
\end{align}

Substituting the bound on $\frac{d\mathcal{I}}{dt}$:
\begin{align}
    \frac{d}{dt}\ln m_T &\ge \frac{1}{2} + \frac{1}{16\pi A}(-CA + \mathcal{I}) - \frac{\mathcal{I}}{16\pi A} \\
    &= \frac{1}{2} - \frac{C}{16\pi} \\
    &\ge 0 \quad \text{if } C \le 8\pi.
\end{align}

\textbf{Key claim:} Under DEC with suitable curvature bounds (satisfied in 
the asymptotic region), the constant $C$ can be made small enough that 
$\frac{d}{dt}\ln m_T \ge 0$.

%==============================================================================
\section{Conclusion and Remaining Work}
%==============================================================================

We have outlined a new approach to the Spacetime Penrose Inequality 
using the \emph{trapped mass} functional $m_T$. The key features are:

\begin{enumerate}
    \item $m_T(\Sigma) \ge \sqrt{A(\Sigma)/16\pi}$ for trapped surfaces 
    (automatic from $\theta^+\theta^- \ge 0$).
    
    \item $m_T$ should be monotonically non-decreasing along IMCF from 
    infinity, under DEC.
    
    \item At infinity, $m_T \to M_{\ADM}$.
    
    \item Combining these gives the Penrose inequality.
\end{enumerate}

\textbf{Remaining work:}
\begin{itemize}
    \item Complete the detailed computation showing $C \le 8\pi$ under DEC.
    \item Handle the transition when IMCF reaches the MOTS ($H \to 0$).
    \item Prove regularity of the foliation (weak solutions if needed).
    \item Handle the case when $\Sigma$ is not enclosed by the outermost MOTS.
\end{itemize}

\end{document}
