%------------------------------------------------------------------
% Additional rigorous details for functional analysis, flux,
% Bray--Khuri identity, and rigidity.
%------------------------------------------------------------------

% This file is meant to be included near the appendices, e.g.
% %------------------------------------------------------------------
% Additional rigorous details for functional analysis, flux,
% Bray--Khuri identity, and rigidity.
%------------------------------------------------------------------

% This file is meant to be included near the appendices, e.g.
% %------------------------------------------------------------------
% Additional rigorous details for functional analysis, flux,
% Bray--Khuri identity, and rigidity.
%------------------------------------------------------------------

% This file is meant to be included near the appendices, e.g.
% %------------------------------------------------------------------
% Additional rigorous details for functional analysis, flux,
% Bray--Khuri identity, and rigidity.
%------------------------------------------------------------------

% This file is meant to be included near the appendices, e.g.
% \input{rigor-additions}

%------------------------------------------------------------------
% 1. Well-posedness of the singular Lichnerowicz equation
%------------------------------------------------------------------

\begin{lemma}[Well-posedness of the singular Lichnerowicz equation]
\label{lem:LichnerowiczWellPosed}
Let $(\overline M,\overline g)$ be the Jang deformation constructed in
Section~\ref{sec:Jang}, and assume the hypotheses of
Theorem~\ref{thm:SPI}.
Fix $p>3$ and weights
\[
  \delta \in (-1,0), \qquad \gamma \in (0,1),
\]
which we use along the asymptotically flat end and the cylindrical
ends of $\overline M$, respectively.
Let
\[
  L := \Delta_{\overline g} - \tfrac18 \mathcal S,
\]
where $\mathcal S$ is the nonnegative part of the Jang scalar curvature
appearing in \eqref{eq:JangScalarIdentity}.
Then:
\begin{enumerate}[(i)]
  \item The operator
  \[
      L : W^{2,p}_{\delta,\gamma}(\overline M)
        \longrightarrow L^p_{\delta-2,\gamma-2}(\overline M)
  \]
  is Fredholm of index zero.

  \item The kernel of $L$ in $W^{2,p}_{\delta,\gamma}(\overline M)$ is
  trivial.

  \item For every $f\in L^p_{\delta-2,\gamma-2}(\overline M)$ there
  exists a unique $\phi\in W^{2,p}_{\delta,\gamma}(\overline M)$
  solving $L\phi=f$ in the weak sense.
\end{enumerate}
In particular, choosing
$f$ to be the right-hand side of equation~\eqref{eq:BK_PDE_Exact}, we
obtain a unique solution $\phi$ of the singular Lichnerowicz equation
with
\[
  \phi-1 \in W^{2,p}_{\delta,\gamma}(\overline M),
\]
and the asymptotics prescribed in
Section~\ref{subsec:Lichnerowicz}.
\end{lemma}

\begin{proof}
We briefly indicate the standard functional-analytic ingredients.

\medskip\noindent
\emph{Step 1: Local elliptic regularity.}
On each coordinate chart of $\overline M$ the operator $L$ is a
second-order uniformly elliptic operator with bounded measurable
coefficients and lower-order term $\mathcal S\in L^\infty$.
Classical $W^{2,p}$-regularity on bounded domains implies that any weak
solution $\phi$ of $L\phi=f$ with $f\in L^p_{\loc}$ belongs to
$W^{2,p}_\loc$.

\medskip\noindent
\emph{Step 2: Asymptotically flat end.}
On the asymptotically flat end, the metric $\overline g$ is a
perturbation of the Euclidean metric with decay rate $\tau>1/2$,
and $\mathcal S$ decays at least as fast as $R_{\overline g}$.
Hence $L$ is a compact perturbation of
the Euclidean Laplacian $\Delta_{\mathbb R^3}$ acting on weighted
Sobolev spaces $W^{2,p}_\delta$.
The mapping
\[
  L : W^{2,p}_\delta \longrightarrow L^p_{\delta-2}
\]
is Fredholm for $\delta\in(-1,0)$, and the index is zero; this is a
standard consequence of the theory of elliptic operators on
asymptotically Euclidean manifolds.

\medskip\noindent
\emph{Step 3: Cylindrical ends.}
On each cylindrical end $\mathcal C\simeq [0,\infty)\times\Sigma$,
with coordinate $t$ along the cylinder, the metric $\overline g$ is a
perturbation of the product metric $dt^2+\sigma$ with either
exponential or polynomial convergence, depending on the strict or
marginal stability of the corresponding component of the horizon.
In particular, $L$ is a compact perturbation of a
translation-invariant operator
\[
  L_0 := \partial_t^2 + \Delta_\sigma + \text{(constant potential)}.
\]
By separation of variables on $\Sigma$, the mapping properties of $L_0$
on the weighted spaces $W^{2,p}_\gamma$ are controlled by the indicial
roots of the one-dimensional ODEs obtained from the eigenfunctions of
$\Delta_\sigma$.
The stability (or marginal stability) assumption on $\Sigma$ implies
that none of these indicial roots lie in $(0,1)$; thus
for $\gamma\in(0,1)$ the map
\[
  L : W^{2,p}_\gamma \longrightarrow L^p_{\gamma-2}
\]
is Fredholm with index zero.

\medskip\noindent
\emph{Step 4: Global Fredholm property.}
We now combine the local parametrices constructed on the compact region,
the asymptotically flat end, and the cylindrical ends by a partition of
unity subordinate to this decomposition.
The resulting global parametrix $G$ satisfies
\[
  LG = \Id - K,
\]
where $K$ is a compact operator on
$W^{2,p}_{\delta,\gamma}(\overline M)$.
Therefore $L$ is Fredholm of index zero from
$W^{2,p}_{\delta,\gamma}$ to $L^p_{\delta-2,\gamma-2}$.
This proves (i).

\medskip\noindent
\emph{Step 5: Trivial kernel and surjectivity.}
Lemma~\ref{lem:IndicialRoots} shows that any solution of $L\phi=0$
belonging to $W^{2,p}_{\delta,\gamma}$ must vanish identically.
Hence $\ker L=\{0\}$ in $W^{2,p}_{\delta,\gamma}$.
Since $L$ is Fredholm of index zero, the cokernel also vanishes,
and $L$ is surjective.
This yields (ii) and (iii).

\medskip\noindent
\emph{Step 6: Application to the singular Lichnerowicz equation.}
The right-hand side of \eqref{eq:BK_PDE_Exact} is a linear combination
of $\div_{\overline g}q$ and lower-order terms built from $q$.
By the decay estimate for $q$ from
Lemma~\ref{lem:SharpAsymptotics} and the choice of
weights, this right-hand side belongs to
$L^p_{\delta-2,\gamma-2}(\overline M)$.
Applying (iii) gives a unique solution
$\phi\in W^{2,p}_{\delta,\gamma}$ with $\phi\to 1$ at infinity and the
prescribed behaviour along the cylindrical ends.
\end{proof}

%------------------------------------------------------------------
% 2. Vanishing of the Jang flux along the cylinder
%------------------------------------------------------------------

\begin{lemma}[Vanishing of the Jang flux]
\label{lem:FluxVanishing}
Let $(\overline M,\overline g)$ be the Jang deformation of an initial
data set satisfying the hypotheses of Theorem~\ref{thm:SPI}.
Let $\mathcal C\simeq[0,\infty)\times\Sigma$ be a cylindrical end
corresponding to a component $\Sigma$ of the outermost MOTS, with
coordinate $t\ge 0$ and cross-sections
$\Sigma_t=\{t\}\times\Sigma$.
Let $q$ be the Jang vector field appearing in
identity~\eqref{eq:JangScalarIdentity}, and let $\nu$ be the unit
normal to $\Sigma_t$ in $\overline g$ pointing towards increasing $t$.
Then
\[
  \lim_{T\to\infty}\int_{\Sigma_T}
      \langle q,\nu\rangle_{\overline g}\,dA_{\overline g} = 0.
\]
\end{lemma}

\begin{proof}
By Lemma~\ref{lem:SharpAsymptotics}, we have the following decay
estimates along the cylinder:
\begin{itemize}
  \item In the strictly stable case, there exists $\kappa>0$ such that
  \[
    \overline g = dt^2+\sigma + O(e^{-\kappa t}),\qquad
    |q(t,\cdot)|_{\overline g}\le C e^{-\kappa t}.
  \]

  \item In the marginally stable case,
  \[
    \overline g = dt^2+\sigma + O(t^{-1}),\qquad
    |q(t,\cdot)|_{\overline g}\le C t^{-2}.
  \]
\end{itemize}
Moreover, in both cases the area
$\operatorname{Area}_{\overline g}(\Sigma_t)$ remains uniformly bounded
for large $t$ (indeed, $\overline g$ converges to the product metric
$dt^2+\sigma$ up to controlled error).

Let $T>0$ and estimate
\[
  \left|\int_{\Sigma_T}
         \langle q,\nu\rangle_{\overline g}\,dA_{\overline g}\right|
  \le \int_{\Sigma_T} |q|_{\overline g}\,dA_{\overline g}
  \le \bigl\|q(T,\cdot)\bigr\|_{L^\infty(\Sigma_T)}
       \operatorname{Area}_{\overline g}(\Sigma_T).
\]
In the strictly stable case we have
$\|q(T,\cdot)\|_{L^\infty}\le C e^{-\kappa T}$, hence
the right-hand side tends to zero as $T\to\infty$.
In the marginally stable case the refined decay gives
$\|q(T,\cdot)\|_{L^\infty}\le C T^{-2}$, and the same conclusion
follows.
\end{proof}

%------------------------------------------------------------------
% 3. Bray--Khuri divergence identity in the distributional setting
%------------------------------------------------------------------

\begin{theorem}[Bray--Khuri divergence identity]
\label{thm:BKidentity}
Let $(\overline M,\overline g,h,q)$ be the Jang deformation of an
initial data set $(M,g,k)$ obeying the dominant energy condition, and
let $\phi>0$ be the solution of the singular Lichnerowicz equation
\eqref{eq:BK_PDE_Exact}.
Define
\[
  Y := \phi^{-1}\nabla_{\overline g}\phi + q,
\]
and
\[
  P := \tfrac12 |h-k|_{\overline g}^2
       + \tfrac12\bigl(\mu-|J|_g\bigr)
       + |\nabla_{\overline g}\log\phi|^2_{\overline g},
\]
where $|J|_g$ denotes the norm of $J$ with respect to $g$ transported
to $\overline M$.
Then $P\ge 0$ almost everywhere and, in the sense of distributions on
$\overline M$,
\begin{equation}
  \label{eq:BK-divergence}
  \div_{\overline g} Y
    = P - \tfrac18 R_{\overline g}.
\end{equation}
\end{theorem}

\begin{proof}
When all data are smooth and $R_{\overline g}$ is a classical function,
the identity is obtained by combining:
\begin{enumerate}[(a)]
  \item the Jang scalar curvature identity
  \eqref{eq:JangScalarIdentity}, and
  \item the conformal transformation law for the scalar curvature of
  $\widetilde g=\phi^4\overline g$,
\end{enumerate}
and then rewriting the resulting expression as a sum of nonnegative
terms plus a divergence.
In this smooth setting the terms collected in $P$ are manifestly
nonnegative under the dominant energy condition $\mu\ge |J|_g$.

In our setting the data $(\overline g,h,k,q,\phi)$ are only as regular
as guaranteed by the construction of the Jang deformation and
Lemma~\ref{lem:LichnerowiczWellPosed}.
In particular, all coefficients lie in $L^2_{\loc}$ and the curvature
is understood distributionally.
We therefore approximate these objects by smooth data:
for each compact set $K\subset\overline M$ choose a local mollification
to obtain smooth tensors
$(\overline g_\varepsilon,h_\varepsilon,k_\varepsilon,q_\varepsilon,
\phi_\varepsilon)$ that converge to the original ones in the
appropriate Sobolev spaces as $\varepsilon\to 0$.
For each fixed $\varepsilon$, the smooth Bray--Khuri computation yields
an identity of the form
\[
  \div_{\overline g_\varepsilon} Y_\varepsilon
    = P_\varepsilon - \tfrac18 R_{\overline g_\varepsilon},
\]
with $Y_\varepsilon$ and $P_\varepsilon$ defined as above but with the
smoothed data.
Passing to the limit in the weak topology of $L^1_{\loc}$, using the
convergence of the coefficients and the uniform integrability provided
by the decay estimates, we obtain~\eqref{eq:BK-divergence}.
The nonnegativity of $P$ follows from pointwise convergence almost
everywhere and Fatou's lemma.
\end{proof}

%------------------------------------------------------------------
% 4. Static vacuum structure and regularity in the rigidity case
%------------------------------------------------------------------

\begin{lemma}[Static vacuum structure in the equality case]
\label{lem:StaticVacuum}
Assume the hypotheses of Theorem~\ref{thm:SPI} and suppose that
equality holds in the spacetime Penrose inequality
\eqref{eq:SPI}.
Then there exists a positive function $N$ on $\overline M$ such that
the static vacuum equations
\[
  \Ric_{\overline g} = N^{-1}\nabla^2_{\overline g}N,
  \qquad
  \Delta_{\overline g}N = 0
\]
hold in the sense of distributions on $\overline M$.
\end{lemma}

\begin{proof}
The proof follows the usual “equality implies rigidity” strategy.
Equality in \eqref{eq:SPI} forces equality in every estimate used in
the proof.
In particular:
\begin{itemize}
  \item the nonnegative integrand $P$ in
  Theorem~\ref{thm:BKidentity} must vanish almost everywhere, and
  \item the dominant energy condition is saturated:
  $\mu=|J|_g$ almost everywhere.
\end{itemize}
Moreover, equality in the AMO monotonicity formula implies that the
level sets of the limiting $1$-harmonic potential have constant mean
curvature and that the corresponding second fundamental form is pure
trace.
Tracing the Bochner-type inequality from
Appendix~\ref{app:Bochner} and using these equalities yields a function
$N>0$ such that
\[
  \nabla^2_{\overline g}N = N\,\Ric_{\overline g}
\]
in the weak sense.
Taking the trace gives $\Delta_{\overline g}N=0$.
\end{proof}

\begin{lemma}[Regularity and classification]
\label{lem:AndersonRegularity}
Under the hypotheses of Lemma~\ref{lem:StaticVacuum}, the pair
$(\overline g,N)$ is smooth and real analytic on the interior of
$\overline M$.
In particular, $(\overline M,\overline g)$ is isometric to a spatial
Schwarzschild slice outside its horizon.
\end{lemma}

\begin{proof}
The static vacuum system
\[
  \Ric_{\overline g} = N^{-1}\nabla^2_{\overline g}N,
  \qquad
  \Delta_{\overline g}N = 0
\]
is a uniformly elliptic system for $(\overline g,N)$ with analytic
nonlinearities.
Standard elliptic regularity implies smoothness of $\overline g$ and
$N$.
Real analyticity in harmonic coordinates then follows from classical
results on elliptic systems with analytic coefficients.
Finally, the classification of complete, asymptotically flat static
vacuum manifolds with connected horizon shows that $(\overline M,
\overline g)$ must coincide with the spatial Schwarzschild metric
outside its horizon, with mass equal to the ADM mass of the original
data.
\end{proof}


%------------------------------------------------------------------
% 1. Well-posedness of the singular Lichnerowicz equation
%------------------------------------------------------------------

\begin{lemma}[Well-posedness of the singular Lichnerowicz equation]
\label{lem:LichnerowiczWellPosed}
Let $(\overline M,\overline g)$ be the Jang deformation constructed in
Section~\ref{sec:Jang}, and assume the hypotheses of
Theorem~\ref{thm:SPI}.
Fix $p>3$ and weights
\[
  \delta \in (-1,0), \qquad \gamma \in (0,1),
\]
which we use along the asymptotically flat end and the cylindrical
ends of $\overline M$, respectively.
Let
\[
  L := \Delta_{\overline g} - \tfrac18 \mathcal S,
\]
where $\mathcal S$ is the nonnegative part of the Jang scalar curvature
appearing in \eqref{eq:JangScalarIdentity}.
Then:
\begin{enumerate}[(i)]
  \item The operator
  \[
      L : W^{2,p}_{\delta,\gamma}(\overline M)
        \longrightarrow L^p_{\delta-2,\gamma-2}(\overline M)
  \]
  is Fredholm of index zero.

  \item The kernel of $L$ in $W^{2,p}_{\delta,\gamma}(\overline M)$ is
  trivial.

  \item For every $f\in L^p_{\delta-2,\gamma-2}(\overline M)$ there
  exists a unique $\phi\in W^{2,p}_{\delta,\gamma}(\overline M)$
  solving $L\phi=f$ in the weak sense.
\end{enumerate}
In particular, choosing
$f$ to be the right-hand side of equation~\eqref{eq:BK_PDE_Exact}, we
obtain a unique solution $\phi$ of the singular Lichnerowicz equation
with
\[
  \phi-1 \in W^{2,p}_{\delta,\gamma}(\overline M),
\]
and the asymptotics prescribed in
Section~\ref{subsec:Lichnerowicz}.
\end{lemma}

\begin{proof}
We briefly indicate the standard functional-analytic ingredients.

\medskip\noindent
\emph{Step 1: Local elliptic regularity.}
On each coordinate chart of $\overline M$ the operator $L$ is a
second-order uniformly elliptic operator with bounded measurable
coefficients and lower-order term $\mathcal S\in L^\infty$.
Classical $W^{2,p}$-regularity on bounded domains implies that any weak
solution $\phi$ of $L\phi=f$ with $f\in L^p_{\loc}$ belongs to
$W^{2,p}_\loc$.

\medskip\noindent
\emph{Step 2: Asymptotically flat end.}
On the asymptotically flat end, the metric $\overline g$ is a
perturbation of the Euclidean metric with decay rate $\tau>1/2$,
and $\mathcal S$ decays at least as fast as $R_{\overline g}$.
Hence $L$ is a compact perturbation of
the Euclidean Laplacian $\Delta_{\mathbb R^3}$ acting on weighted
Sobolev spaces $W^{2,p}_\delta$.
The mapping
\[
  L : W^{2,p}_\delta \longrightarrow L^p_{\delta-2}
\]
is Fredholm for $\delta\in(-1,0)$, and the index is zero; this is a
standard consequence of the theory of elliptic operators on
asymptotically Euclidean manifolds.

\medskip\noindent
\emph{Step 3: Cylindrical ends.}
On each cylindrical end $\mathcal C\simeq [0,\infty)\times\Sigma$,
with coordinate $t$ along the cylinder, the metric $\overline g$ is a
perturbation of the product metric $dt^2+\sigma$ with either
exponential or polynomial convergence, depending on the strict or
marginal stability of the corresponding component of the horizon.
In particular, $L$ is a compact perturbation of a
translation-invariant operator
\[
  L_0 := \partial_t^2 + \Delta_\sigma + \text{(constant potential)}.
\]
By separation of variables on $\Sigma$, the mapping properties of $L_0$
on the weighted spaces $W^{2,p}_\gamma$ are controlled by the indicial
roots of the one-dimensional ODEs obtained from the eigenfunctions of
$\Delta_\sigma$.
The stability (or marginal stability) assumption on $\Sigma$ implies
that none of these indicial roots lie in $(0,1)$; thus
for $\gamma\in(0,1)$ the map
\[
  L : W^{2,p}_\gamma \longrightarrow L^p_{\gamma-2}
\]
is Fredholm with index zero.

\medskip\noindent
\emph{Step 4: Global Fredholm property.}
We now combine the local parametrices constructed on the compact region,
the asymptotically flat end, and the cylindrical ends by a partition of
unity subordinate to this decomposition.
The resulting global parametrix $G$ satisfies
\[
  LG = \Id - K,
\]
where $K$ is a compact operator on
$W^{2,p}_{\delta,\gamma}(\overline M)$.
Therefore $L$ is Fredholm of index zero from
$W^{2,p}_{\delta,\gamma}$ to $L^p_{\delta-2,\gamma-2}$.
This proves (i).

\medskip\noindent
\emph{Step 5: Trivial kernel and surjectivity.}
Lemma~\ref{lem:IndicialRoots} shows that any solution of $L\phi=0$
belonging to $W^{2,p}_{\delta,\gamma}$ must vanish identically.
Hence $\ker L=\{0\}$ in $W^{2,p}_{\delta,\gamma}$.
Since $L$ is Fredholm of index zero, the cokernel also vanishes,
and $L$ is surjective.
This yields (ii) and (iii).

\medskip\noindent
\emph{Step 6: Application to the singular Lichnerowicz equation.}
The right-hand side of \eqref{eq:BK_PDE_Exact} is a linear combination
of $\div_{\overline g}q$ and lower-order terms built from $q$.
By the decay estimate for $q$ from
Lemma~\ref{lem:SharpAsymptotics} and the choice of
weights, this right-hand side belongs to
$L^p_{\delta-2,\gamma-2}(\overline M)$.
Applying (iii) gives a unique solution
$\phi\in W^{2,p}_{\delta,\gamma}$ with $\phi\to 1$ at infinity and the
prescribed behaviour along the cylindrical ends.
\end{proof}

%------------------------------------------------------------------
% 2. Vanishing of the Jang flux along the cylinder
%------------------------------------------------------------------

\begin{lemma}[Vanishing of the Jang flux]
\label{lem:FluxVanishing}
Let $(\overline M,\overline g)$ be the Jang deformation of an initial
data set satisfying the hypotheses of Theorem~\ref{thm:SPI}.
Let $\mathcal C\simeq[0,\infty)\times\Sigma$ be a cylindrical end
corresponding to a component $\Sigma$ of the outermost MOTS, with
coordinate $t\ge 0$ and cross-sections
$\Sigma_t=\{t\}\times\Sigma$.
Let $q$ be the Jang vector field appearing in
identity~\eqref{eq:JangScalarIdentity}, and let $\nu$ be the unit
normal to $\Sigma_t$ in $\overline g$ pointing towards increasing $t$.
Then
\[
  \lim_{T\to\infty}\int_{\Sigma_T}
      \langle q,\nu\rangle_{\overline g}\,dA_{\overline g} = 0.
\]
\end{lemma}

\begin{proof}
By Lemma~\ref{lem:SharpAsymptotics}, we have the following decay
estimates along the cylinder:
\begin{itemize}
  \item In the strictly stable case, there exists $\kappa>0$ such that
  \[
    \overline g = dt^2+\sigma + O(e^{-\kappa t}),\qquad
    |q(t,\cdot)|_{\overline g}\le C e^{-\kappa t}.
  \]

  \item In the marginally stable case,
  \[
    \overline g = dt^2+\sigma + O(t^{-1}),\qquad
    |q(t,\cdot)|_{\overline g}\le C t^{-2}.
  \]
\end{itemize}
Moreover, in both cases the area
$\operatorname{Area}_{\overline g}(\Sigma_t)$ remains uniformly bounded
for large $t$ (indeed, $\overline g$ converges to the product metric
$dt^2+\sigma$ up to controlled error).

Let $T>0$ and estimate
\[
  \left|\int_{\Sigma_T}
         \langle q,\nu\rangle_{\overline g}\,dA_{\overline g}\right|
  \le \int_{\Sigma_T} |q|_{\overline g}\,dA_{\overline g}
  \le \bigl\|q(T,\cdot)\bigr\|_{L^\infty(\Sigma_T)}
       \operatorname{Area}_{\overline g}(\Sigma_T).
\]
In the strictly stable case we have
$\|q(T,\cdot)\|_{L^\infty}\le C e^{-\kappa T}$, hence
the right-hand side tends to zero as $T\to\infty$.
In the marginally stable case the refined decay gives
$\|q(T,\cdot)\|_{L^\infty}\le C T^{-2}$, and the same conclusion
follows.
\end{proof}

%------------------------------------------------------------------
% 3. Bray--Khuri divergence identity in the distributional setting
%------------------------------------------------------------------

\begin{theorem}[Bray--Khuri divergence identity]
\label{thm:BKidentity}
Let $(\overline M,\overline g,h,q)$ be the Jang deformation of an
initial data set $(M,g,k)$ obeying the dominant energy condition, and
let $\phi>0$ be the solution of the singular Lichnerowicz equation
\eqref{eq:BK_PDE_Exact}.
Define
\[
  Y := \phi^{-1}\nabla_{\overline g}\phi + q,
\]
and
\[
  P := \tfrac12 |h-k|_{\overline g}^2
       + \tfrac12\bigl(\mu-|J|_g\bigr)
       + |\nabla_{\overline g}\log\phi|^2_{\overline g},
\]
where $|J|_g$ denotes the norm of $J$ with respect to $g$ transported
to $\overline M$.
Then $P\ge 0$ almost everywhere and, in the sense of distributions on
$\overline M$,
\begin{equation}
  \label{eq:BK-divergence}
  \div_{\overline g} Y
    = P - \tfrac18 R_{\overline g}.
\end{equation}
\end{theorem}

\begin{proof}
When all data are smooth and $R_{\overline g}$ is a classical function,
the identity is obtained by combining:
\begin{enumerate}[(a)]
  \item the Jang scalar curvature identity
  \eqref{eq:JangScalarIdentity}, and
  \item the conformal transformation law for the scalar curvature of
  $\widetilde g=\phi^4\overline g$,
\end{enumerate}
and then rewriting the resulting expression as a sum of nonnegative
terms plus a divergence.
In this smooth setting the terms collected in $P$ are manifestly
nonnegative under the dominant energy condition $\mu\ge |J|_g$.

In our setting the data $(\overline g,h,k,q,\phi)$ are only as regular
as guaranteed by the construction of the Jang deformation and
Lemma~\ref{lem:LichnerowiczWellPosed}.
In particular, all coefficients lie in $L^2_{\loc}$ and the curvature
is understood distributionally.
We therefore approximate these objects by smooth data:
for each compact set $K\subset\overline M$ choose a local mollification
to obtain smooth tensors
$(\overline g_\varepsilon,h_\varepsilon,k_\varepsilon,q_\varepsilon,
\phi_\varepsilon)$ that converge to the original ones in the
appropriate Sobolev spaces as $\varepsilon\to 0$.
For each fixed $\varepsilon$, the smooth Bray--Khuri computation yields
an identity of the form
\[
  \div_{\overline g_\varepsilon} Y_\varepsilon
    = P_\varepsilon - \tfrac18 R_{\overline g_\varepsilon},
\]
with $Y_\varepsilon$ and $P_\varepsilon$ defined as above but with the
smoothed data.
Passing to the limit in the weak topology of $L^1_{\loc}$, using the
convergence of the coefficients and the uniform integrability provided
by the decay estimates, we obtain~\eqref{eq:BK-divergence}.
The nonnegativity of $P$ follows from pointwise convergence almost
everywhere and Fatou's lemma.
\end{proof}

%------------------------------------------------------------------
% 4. Static vacuum structure and regularity in the rigidity case
%------------------------------------------------------------------

\begin{lemma}[Static vacuum structure in the equality case]
\label{lem:StaticVacuum}
Assume the hypotheses of Theorem~\ref{thm:SPI} and suppose that
equality holds in the spacetime Penrose inequality
\eqref{eq:SPI}.
Then there exists a positive function $N$ on $\overline M$ such that
the static vacuum equations
\[
  \Ric_{\overline g} = N^{-1}\nabla^2_{\overline g}N,
  \qquad
  \Delta_{\overline g}N = 0
\]
hold in the sense of distributions on $\overline M$.
\end{lemma}

\begin{proof}
The proof follows the usual “equality implies rigidity” strategy.
Equality in \eqref{eq:SPI} forces equality in every estimate used in
the proof.
In particular:
\begin{itemize}
  \item the nonnegative integrand $P$ in
  Theorem~\ref{thm:BKidentity} must vanish almost everywhere, and
  \item the dominant energy condition is saturated:
  $\mu=|J|_g$ almost everywhere.
\end{itemize}
Moreover, equality in the AMO monotonicity formula implies that the
level sets of the limiting $1$-harmonic potential have constant mean
curvature and that the corresponding second fundamental form is pure
trace.
Tracing the Bochner-type inequality from
Appendix~\ref{app:Bochner} and using these equalities yields a function
$N>0$ such that
\[
  \nabla^2_{\overline g}N = N\,\Ric_{\overline g}
\]
in the weak sense.
Taking the trace gives $\Delta_{\overline g}N=0$.
\end{proof}

\begin{lemma}[Regularity and classification]
\label{lem:AndersonRegularity}
Under the hypotheses of Lemma~\ref{lem:StaticVacuum}, the pair
$(\overline g,N)$ is smooth and real analytic on the interior of
$\overline M$.
In particular, $(\overline M,\overline g)$ is isometric to a spatial
Schwarzschild slice outside its horizon.
\end{lemma}

\begin{proof}
The static vacuum system
\[
  \Ric_{\overline g} = N^{-1}\nabla^2_{\overline g}N,
  \qquad
  \Delta_{\overline g}N = 0
\]
is a uniformly elliptic system for $(\overline g,N)$ with analytic
nonlinearities.
Standard elliptic regularity implies smoothness of $\overline g$ and
$N$.
Real analyticity in harmonic coordinates then follows from classical
results on elliptic systems with analytic coefficients.
Finally, the classification of complete, asymptotically flat static
vacuum manifolds with connected horizon shows that $(\overline M,
\overline g)$ must coincide with the spatial Schwarzschild metric
outside its horizon, with mass equal to the ADM mass of the original
data.
\end{proof}


%------------------------------------------------------------------
% 1. Well-posedness of the singular Lichnerowicz equation
%------------------------------------------------------------------

\begin{lemma}[Well-posedness of the singular Lichnerowicz equation]
\label{lem:LichnerowiczWellPosed}
Let $(\overline M,\overline g)$ be the Jang deformation constructed in
Section~\ref{sec:Jang}, and assume the hypotheses of
Theorem~\ref{thm:SPI}.
Fix $p>3$ and weights
\[
  \delta \in (-1,0), \qquad \gamma \in (0,1),
\]
which we use along the asymptotically flat end and the cylindrical
ends of $\overline M$, respectively.
Let
\[
  L := \Delta_{\overline g} - \tfrac18 \mathcal S,
\]
where $\mathcal S$ is the nonnegative part of the Jang scalar curvature
appearing in \eqref{eq:JangScalarIdentity}.
Then:
\begin{enumerate}[(i)]
  \item The operator
  \[
      L : W^{2,p}_{\delta,\gamma}(\overline M)
        \longrightarrow L^p_{\delta-2,\gamma-2}(\overline M)
  \]
  is Fredholm of index zero.

  \item The kernel of $L$ in $W^{2,p}_{\delta,\gamma}(\overline M)$ is
  trivial.

  \item For every $f\in L^p_{\delta-2,\gamma-2}(\overline M)$ there
  exists a unique $\phi\in W^{2,p}_{\delta,\gamma}(\overline M)$
  solving $L\phi=f$ in the weak sense.
\end{enumerate}
In particular, choosing
$f$ to be the right-hand side of equation~\eqref{eq:BK_PDE_Exact}, we
obtain a unique solution $\phi$ of the singular Lichnerowicz equation
with
\[
  \phi-1 \in W^{2,p}_{\delta,\gamma}(\overline M),
\]
and the asymptotics prescribed in
Section~\ref{subsec:Lichnerowicz}.
\end{lemma}

\begin{proof}
We briefly indicate the standard functional-analytic ingredients.

\medskip\noindent
\emph{Step 1: Local elliptic regularity.}
On each coordinate chart of $\overline M$ the operator $L$ is a
second-order uniformly elliptic operator with bounded measurable
coefficients and lower-order term $\mathcal S\in L^\infty$.
Classical $W^{2,p}$-regularity on bounded domains implies that any weak
solution $\phi$ of $L\phi=f$ with $f\in L^p_{\loc}$ belongs to
$W^{2,p}_\loc$.

\medskip\noindent
\emph{Step 2: Asymptotically flat end.}
On the asymptotically flat end, the metric $\overline g$ is a
perturbation of the Euclidean metric with decay rate $\tau>1/2$,
and $\mathcal S$ decays at least as fast as $R_{\overline g}$.
Hence $L$ is a compact perturbation of
the Euclidean Laplacian $\Delta_{\mathbb R^3}$ acting on weighted
Sobolev spaces $W^{2,p}_\delta$.
The mapping
\[
  L : W^{2,p}_\delta \longrightarrow L^p_{\delta-2}
\]
is Fredholm for $\delta\in(-1,0)$, and the index is zero; this is a
standard consequence of the theory of elliptic operators on
asymptotically Euclidean manifolds.

\medskip\noindent
\emph{Step 3: Cylindrical ends.}
On each cylindrical end $\mathcal C\simeq [0,\infty)\times\Sigma$,
with coordinate $t$ along the cylinder, the metric $\overline g$ is a
perturbation of the product metric $dt^2+\sigma$ with either
exponential or polynomial convergence, depending on the strict or
marginal stability of the corresponding component of the horizon.
In particular, $L$ is a compact perturbation of a
translation-invariant operator
\[
  L_0 := \partial_t^2 + \Delta_\sigma + \text{(constant potential)}.
\]
By separation of variables on $\Sigma$, the mapping properties of $L_0$
on the weighted spaces $W^{2,p}_\gamma$ are controlled by the indicial
roots of the one-dimensional ODEs obtained from the eigenfunctions of
$\Delta_\sigma$.
The stability (or marginal stability) assumption on $\Sigma$ implies
that none of these indicial roots lie in $(0,1)$; thus
for $\gamma\in(0,1)$ the map
\[
  L : W^{2,p}_\gamma \longrightarrow L^p_{\gamma-2}
\]
is Fredholm with index zero.

\medskip\noindent
\emph{Step 4: Global Fredholm property.}
We now combine the local parametrices constructed on the compact region,
the asymptotically flat end, and the cylindrical ends by a partition of
unity subordinate to this decomposition.
The resulting global parametrix $G$ satisfies
\[
  LG = \Id - K,
\]
where $K$ is a compact operator on
$W^{2,p}_{\delta,\gamma}(\overline M)$.
Therefore $L$ is Fredholm of index zero from
$W^{2,p}_{\delta,\gamma}$ to $L^p_{\delta-2,\gamma-2}$.
This proves (i).

\medskip\noindent
\emph{Step 5: Trivial kernel and surjectivity.}
Lemma~\ref{lem:IndicialRoots} shows that any solution of $L\phi=0$
belonging to $W^{2,p}_{\delta,\gamma}$ must vanish identically.
Hence $\ker L=\{0\}$ in $W^{2,p}_{\delta,\gamma}$.
Since $L$ is Fredholm of index zero, the cokernel also vanishes,
and $L$ is surjective.
This yields (ii) and (iii).

\medskip\noindent
\emph{Step 6: Application to the singular Lichnerowicz equation.}
The right-hand side of \eqref{eq:BK_PDE_Exact} is a linear combination
of $\div_{\overline g}q$ and lower-order terms built from $q$.
By the decay estimate for $q$ from
Lemma~\ref{lem:SharpAsymptotics} and the choice of
weights, this right-hand side belongs to
$L^p_{\delta-2,\gamma-2}(\overline M)$.
Applying (iii) gives a unique solution
$\phi\in W^{2,p}_{\delta,\gamma}$ with $\phi\to 1$ at infinity and the
prescribed behaviour along the cylindrical ends.
\end{proof}

%------------------------------------------------------------------
% 2. Vanishing of the Jang flux along the cylinder
%------------------------------------------------------------------

\begin{lemma}[Vanishing of the Jang flux]
\label{lem:FluxVanishing}
Let $(\overline M,\overline g)$ be the Jang deformation of an initial
data set satisfying the hypotheses of Theorem~\ref{thm:SPI}.
Let $\mathcal C\simeq[0,\infty)\times\Sigma$ be a cylindrical end
corresponding to a component $\Sigma$ of the outermost MOTS, with
coordinate $t\ge 0$ and cross-sections
$\Sigma_t=\{t\}\times\Sigma$.
Let $q$ be the Jang vector field appearing in
identity~\eqref{eq:JangScalarIdentity}, and let $\nu$ be the unit
normal to $\Sigma_t$ in $\overline g$ pointing towards increasing $t$.
Then
\[
  \lim_{T\to\infty}\int_{\Sigma_T}
      \langle q,\nu\rangle_{\overline g}\,dA_{\overline g} = 0.
\]
\end{lemma}

\begin{proof}
By Lemma~\ref{lem:SharpAsymptotics}, we have the following decay
estimates along the cylinder:
\begin{itemize}
  \item In the strictly stable case, there exists $\kappa>0$ such that
  \[
    \overline g = dt^2+\sigma + O(e^{-\kappa t}),\qquad
    |q(t,\cdot)|_{\overline g}\le C e^{-\kappa t}.
  \]

  \item In the marginally stable case,
  \[
    \overline g = dt^2+\sigma + O(t^{-1}),\qquad
    |q(t,\cdot)|_{\overline g}\le C t^{-2}.
  \]
\end{itemize}
Moreover, in both cases the area
$\operatorname{Area}_{\overline g}(\Sigma_t)$ remains uniformly bounded
for large $t$ (indeed, $\overline g$ converges to the product metric
$dt^2+\sigma$ up to controlled error).

Let $T>0$ and estimate
\[
  \left|\int_{\Sigma_T}
         \langle q,\nu\rangle_{\overline g}\,dA_{\overline g}\right|
  \le \int_{\Sigma_T} |q|_{\overline g}\,dA_{\overline g}
  \le \bigl\|q(T,\cdot)\bigr\|_{L^\infty(\Sigma_T)}
       \operatorname{Area}_{\overline g}(\Sigma_T).
\]
In the strictly stable case we have
$\|q(T,\cdot)\|_{L^\infty}\le C e^{-\kappa T}$, hence
the right-hand side tends to zero as $T\to\infty$.
In the marginally stable case the refined decay gives
$\|q(T,\cdot)\|_{L^\infty}\le C T^{-2}$, and the same conclusion
follows.
\end{proof}

%------------------------------------------------------------------
% 3. Bray--Khuri divergence identity in the distributional setting
%------------------------------------------------------------------

\begin{theorem}[Bray--Khuri divergence identity]
\label{thm:BKidentity}
Let $(\overline M,\overline g,h,q)$ be the Jang deformation of an
initial data set $(M,g,k)$ obeying the dominant energy condition, and
let $\phi>0$ be the solution of the singular Lichnerowicz equation
\eqref{eq:BK_PDE_Exact}.
Define
\[
  Y := \phi^{-1}\nabla_{\overline g}\phi + q,
\]
and
\[
  P := \tfrac12 |h-k|_{\overline g}^2
       + \tfrac12\bigl(\mu-|J|_g\bigr)
       + |\nabla_{\overline g}\log\phi|^2_{\overline g},
\]
where $|J|_g$ denotes the norm of $J$ with respect to $g$ transported
to $\overline M$.
Then $P\ge 0$ almost everywhere and, in the sense of distributions on
$\overline M$,
\begin{equation}
  \label{eq:BK-divergence}
  \div_{\overline g} Y
    = P - \tfrac18 R_{\overline g}.
\end{equation}
\end{theorem}

\begin{proof}
When all data are smooth and $R_{\overline g}$ is a classical function,
the identity is obtained by combining:
\begin{enumerate}[(a)]
  \item the Jang scalar curvature identity
  \eqref{eq:JangScalarIdentity}, and
  \item the conformal transformation law for the scalar curvature of
  $\widetilde g=\phi^4\overline g$,
\end{enumerate}
and then rewriting the resulting expression as a sum of nonnegative
terms plus a divergence.
In this smooth setting the terms collected in $P$ are manifestly
nonnegative under the dominant energy condition $\mu\ge |J|_g$.

In our setting the data $(\overline g,h,k,q,\phi)$ are only as regular
as guaranteed by the construction of the Jang deformation and
Lemma~\ref{lem:LichnerowiczWellPosed}.
In particular, all coefficients lie in $L^2_{\loc}$ and the curvature
is understood distributionally.
We therefore approximate these objects by smooth data:
for each compact set $K\subset\overline M$ choose a local mollification
to obtain smooth tensors
$(\overline g_\varepsilon,h_\varepsilon,k_\varepsilon,q_\varepsilon,
\phi_\varepsilon)$ that converge to the original ones in the
appropriate Sobolev spaces as $\varepsilon\to 0$.
For each fixed $\varepsilon$, the smooth Bray--Khuri computation yields
an identity of the form
\[
  \div_{\overline g_\varepsilon} Y_\varepsilon
    = P_\varepsilon - \tfrac18 R_{\overline g_\varepsilon},
\]
with $Y_\varepsilon$ and $P_\varepsilon$ defined as above but with the
smoothed data.
Passing to the limit in the weak topology of $L^1_{\loc}$, using the
convergence of the coefficients and the uniform integrability provided
by the decay estimates, we obtain~\eqref{eq:BK-divergence}.
The nonnegativity of $P$ follows from pointwise convergence almost
everywhere and Fatou's lemma.
\end{proof}

%------------------------------------------------------------------
% 4. Static vacuum structure and regularity in the rigidity case
%------------------------------------------------------------------

\begin{lemma}[Static vacuum structure in the equality case]
\label{lem:StaticVacuum}
Assume the hypotheses of Theorem~\ref{thm:SPI} and suppose that
equality holds in the spacetime Penrose inequality
\eqref{eq:SPI}.
Then there exists a positive function $N$ on $\overline M$ such that
the static vacuum equations
\[
  \Ric_{\overline g} = N^{-1}\nabla^2_{\overline g}N,
  \qquad
  \Delta_{\overline g}N = 0
\]
hold in the sense of distributions on $\overline M$.
\end{lemma}

\begin{proof}
The proof follows the usual “equality implies rigidity” strategy.
Equality in \eqref{eq:SPI} forces equality in every estimate used in
the proof.
In particular:
\begin{itemize}
  \item the nonnegative integrand $P$ in
  Theorem~\ref{thm:BKidentity} must vanish almost everywhere, and
  \item the dominant energy condition is saturated:
  $\mu=|J|_g$ almost everywhere.
\end{itemize}
Moreover, equality in the AMO monotonicity formula implies that the
level sets of the limiting $1$-harmonic potential have constant mean
curvature and that the corresponding second fundamental form is pure
trace.
Tracing the Bochner-type inequality from
Appendix~\ref{app:Bochner} and using these equalities yields a function
$N>0$ such that
\[
  \nabla^2_{\overline g}N = N\,\Ric_{\overline g}
\]
in the weak sense.
Taking the trace gives $\Delta_{\overline g}N=0$.
\end{proof}

\begin{lemma}[Regularity and classification]
\label{lem:AndersonRegularity}
Under the hypotheses of Lemma~\ref{lem:StaticVacuum}, the pair
$(\overline g,N)$ is smooth and real analytic on the interior of
$\overline M$.
In particular, $(\overline M,\overline g)$ is isometric to a spatial
Schwarzschild slice outside its horizon.
\end{lemma}

\begin{proof}
The static vacuum system
\[
  \Ric_{\overline g} = N^{-1}\nabla^2_{\overline g}N,
  \qquad
  \Delta_{\overline g}N = 0
\]
is a uniformly elliptic system for $(\overline g,N)$ with analytic
nonlinearities.
Standard elliptic regularity implies smoothness of $\overline g$ and
$N$.
Real analyticity in harmonic coordinates then follows from classical
results on elliptic systems with analytic coefficients.
Finally, the classification of complete, asymptotically flat static
vacuum manifolds with connected horizon shows that $(\overline M,
\overline g)$ must coincide with the spatial Schwarzschild metric
outside its horizon, with mass equal to the ADM mass of the original
data.
\end{proof}


%------------------------------------------------------------------
% 1. Well-posedness of the singular Lichnerowicz equation
%------------------------------------------------------------------

\begin{lemma}[Well-posedness of the singular Lichnerowicz equation]
\label{lem:LichnerowiczWellPosed}
Let $(\overline M,\overline g)$ be the Jang deformation constructed in
Section~\ref{sec:Jang}, and assume the hypotheses of
Theorem~\ref{thm:SPI}.
Fix $p>3$ and weights
\[
  \delta \in (-1,0), \qquad \gamma \in (0,1),
\]
which we use along the asymptotically flat end and the cylindrical
ends of $\overline M$, respectively.
Let
\[
  L := \Delta_{\overline g} - \tfrac18 \mathcal S,
\]
where $\mathcal S$ is the nonnegative part of the Jang scalar curvature
appearing in \eqref{eq:JangScalarIdentity}.
Then:
\begin{enumerate}[(i)]
  \item The operator
  \[
      L : W^{2,p}_{\delta,\gamma}(\overline M)
        \longrightarrow L^p_{\delta-2,\gamma-2}(\overline M)
  \]
  is Fredholm of index zero.

  \item The kernel of $L$ in $W^{2,p}_{\delta,\gamma}(\overline M)$ is
  trivial.

  \item For every $f\in L^p_{\delta-2,\gamma-2}(\overline M)$ there
  exists a unique $\phi\in W^{2,p}_{\delta,\gamma}(\overline M)$
  solving $L\phi=f$ in the weak sense.
\end{enumerate}
In particular, choosing
$f$ to be the right-hand side of equation~\eqref{eq:BK_PDE_Exact}, we
obtain a unique solution $\phi$ of the singular Lichnerowicz equation
with
\[
  \phi-1 \in W^{2,p}_{\delta,\gamma}(\overline M),
\]
and the asymptotics prescribed in
Section~\ref{subsec:Lichnerowicz}.
\end{lemma}

\begin{proof}
We briefly indicate the standard functional-analytic ingredients.

\medskip\noindent
\emph{Step 1: Local elliptic regularity.}
On each coordinate chart of $\overline M$ the operator $L$ is a
second-order uniformly elliptic operator with bounded measurable
coefficients and lower-order term $\mathcal S\in L^\infty$.
Classical $W^{2,p}$-regularity on bounded domains implies that any weak
solution $\phi$ of $L\phi=f$ with $f\in L^p_{\loc}$ belongs to
$W^{2,p}_\loc$.

\medskip\noindent
\emph{Step 2: Asymptotically flat end.}
On the asymptotically flat end, the metric $\overline g$ is a
perturbation of the Euclidean metric with decay rate $\tau>1/2$,
and $\mathcal S$ decays at least as fast as $R_{\overline g}$.
Hence $L$ is a compact perturbation of
the Euclidean Laplacian $\Delta_{\mathbb R^3}$ acting on weighted
Sobolev spaces $W^{2,p}_\delta$.
The mapping
\[
  L : W^{2,p}_\delta \longrightarrow L^p_{\delta-2}
\]
is Fredholm for $\delta\in(-1,0)$, and the index is zero; this is a
standard consequence of the theory of elliptic operators on
asymptotically Euclidean manifolds.

\medskip\noindent
\emph{Step 3: Cylindrical ends.}
On each cylindrical end $\mathcal C\simeq [0,\infty)\times\Sigma$,
with coordinate $t$ along the cylinder, the metric $\overline g$ is a
perturbation of the product metric $dt^2+\sigma$ with either
exponential or polynomial convergence, depending on the strict or
marginal stability of the corresponding component of the horizon.
In particular, $L$ is a compact perturbation of a
translation-invariant operator
\[
  L_0 := \partial_t^2 + \Delta_\sigma + \text{(constant potential)}.
\]
By separation of variables on $\Sigma$, the mapping properties of $L_0$
on the weighted spaces $W^{2,p}_\gamma$ are controlled by the indicial
roots of the one-dimensional ODEs obtained from the eigenfunctions of
$\Delta_\sigma$.
The stability (or marginal stability) assumption on $\Sigma$ implies
that none of these indicial roots lie in $(0,1)$; thus
for $\gamma\in(0,1)$ the map
\[
  L : W^{2,p}_\gamma \longrightarrow L^p_{\gamma-2}
\]
is Fredholm with index zero.

\medskip\noindent
\emph{Step 4: Global Fredholm property.}
We now combine the local parametrices constructed on the compact region,
the asymptotically flat end, and the cylindrical ends by a partition of
unity subordinate to this decomposition.
The resulting global parametrix $G$ satisfies
\[
  LG = \Id - K,
\]
where $K$ is a compact operator on
$W^{2,p}_{\delta,\gamma}(\overline M)$.
Therefore $L$ is Fredholm of index zero from
$W^{2,p}_{\delta,\gamma}$ to $L^p_{\delta-2,\gamma-2}$.
This proves (i).

\medskip\noindent
\emph{Step 5: Trivial kernel and surjectivity.}
Lemma~\ref{lem:IndicialRoots} shows that any solution of $L\phi=0$
belonging to $W^{2,p}_{\delta,\gamma}$ must vanish identically.
Hence $\ker L=\{0\}$ in $W^{2,p}_{\delta,\gamma}$.
Since $L$ is Fredholm of index zero, the cokernel also vanishes,
and $L$ is surjective.
This yields (ii) and (iii).

\medskip\noindent
\emph{Step 6: Application to the singular Lichnerowicz equation.}
The right-hand side of \eqref{eq:BK_PDE_Exact} is a linear combination
of $\div_{\overline g}q$ and lower-order terms built from $q$.
By the decay estimate for $q$ from
Lemma~\ref{lem:SharpAsymptotics} and the choice of
weights, this right-hand side belongs to
$L^p_{\delta-2,\gamma-2}(\overline M)$.
Applying (iii) gives a unique solution
$\phi\in W^{2,p}_{\delta,\gamma}$ with $\phi\to 1$ at infinity and the
prescribed behaviour along the cylindrical ends.
\end{proof}

%------------------------------------------------------------------
% 2. Vanishing of the Jang flux along the cylinder
%------------------------------------------------------------------

\begin{lemma}[Vanishing of the Jang flux]
\label{lem:FluxVanishing}
Let $(\overline M,\overline g)$ be the Jang deformation of an initial
data set satisfying the hypotheses of Theorem~\ref{thm:SPI}.
Let $\mathcal C\simeq[0,\infty)\times\Sigma$ be a cylindrical end
corresponding to a component $\Sigma$ of the outermost MOTS, with
coordinate $t\ge 0$ and cross-sections
$\Sigma_t=\{t\}\times\Sigma$.
Let $q$ be the Jang vector field appearing in
identity~\eqref{eq:JangScalarIdentity}, and let $\nu$ be the unit
normal to $\Sigma_t$ in $\overline g$ pointing towards increasing $t$.
Then
\[
  \lim_{T\to\infty}\int_{\Sigma_T}
      \langle q,\nu\rangle_{\overline g}\,dA_{\overline g} = 0.
\]
\end{lemma}

\begin{proof}
By Lemma~\ref{lem:SharpAsymptotics}, we have the following decay
estimates along the cylinder:
\begin{itemize}
  \item In the strictly stable case, there exists $\kappa>0$ such that
  \[
    \overline g = dt^2+\sigma + O(e^{-\kappa t}),\qquad
    |q(t,\cdot)|_{\overline g}\le C e^{-\kappa t}.
  \]

  \item In the marginally stable case,
  \[
    \overline g = dt^2+\sigma + O(t^{-1}),\qquad
    |q(t,\cdot)|_{\overline g}\le C t^{-2}.
  \]
\end{itemize}
Moreover, in both cases the area
$\operatorname{Area}_{\overline g}(\Sigma_t)$ remains uniformly bounded
for large $t$ (indeed, $\overline g$ converges to the product metric
$dt^2+\sigma$ up to controlled error).

Let $T>0$ and estimate
\[
  \left|\int_{\Sigma_T}
         \langle q,\nu\rangle_{\overline g}\,dA_{\overline g}\right|
  \le \int_{\Sigma_T} |q|_{\overline g}\,dA_{\overline g}
  \le \bigl\|q(T,\cdot)\bigr\|_{L^\infty(\Sigma_T)}
       \operatorname{Area}_{\overline g}(\Sigma_T).
\]
In the strictly stable case we have
$\|q(T,\cdot)\|_{L^\infty}\le C e^{-\kappa T}$, hence
the right-hand side tends to zero as $T\to\infty$.
In the marginally stable case the refined decay gives
$\|q(T,\cdot)\|_{L^\infty}\le C T^{-2}$, and the same conclusion
follows.
\end{proof}

%------------------------------------------------------------------
% 3. Bray--Khuri divergence identity in the distributional setting
%------------------------------------------------------------------

\begin{theorem}[Bray--Khuri divergence identity]
\label{thm:BKidentity}
Let $(\overline M,\overline g,h,q)$ be the Jang deformation of an
initial data set $(M,g,k)$ obeying the dominant energy condition, and
let $\phi>0$ be the solution of the singular Lichnerowicz equation
\eqref{eq:BK_PDE_Exact}.
Define
\[
  Y := \phi^{-1}\nabla_{\overline g}\phi + q,
\]
and
\[
  P := \tfrac12 |h-k|_{\overline g}^2
       + \tfrac12\bigl(\mu-|J|_g\bigr)
       + |\nabla_{\overline g}\log\phi|^2_{\overline g},
\]
where $|J|_g$ denotes the norm of $J$ with respect to $g$ transported
to $\overline M$.
Then $P\ge 0$ almost everywhere and, in the sense of distributions on
$\overline M$,
\begin{equation}
  \label{eq:BK-divergence}
  \div_{\overline g} Y
    = P - \tfrac18 R_{\overline g}.
\end{equation}
\end{theorem}

\begin{proof}
When all data are smooth and $R_{\overline g}$ is a classical function,
the identity is obtained by combining:
\begin{enumerate}[(a)]
  \item the Jang scalar curvature identity
  \eqref{eq:JangScalarIdentity}, and
  \item the conformal transformation law for the scalar curvature of
  $\widetilde g=\phi^4\overline g$,
\end{enumerate}
and then rewriting the resulting expression as a sum of nonnegative
terms plus a divergence.
In this smooth setting the terms collected in $P$ are manifestly
nonnegative under the dominant energy condition $\mu\ge |J|_g$.

In our setting the data $(\overline g,h,k,q,\phi)$ are only as regular
as guaranteed by the construction of the Jang deformation and
Lemma~\ref{lem:LichnerowiczWellPosed}.
In particular, all coefficients lie in $L^2_{\loc}$ and the curvature
is understood distributionally.
We therefore approximate these objects by smooth data:
for each compact set $K\subset\overline M$ choose a local mollification
to obtain smooth tensors
$(\overline g_\varepsilon,h_\varepsilon,k_\varepsilon,q_\varepsilon,
\phi_\varepsilon)$ that converge to the original ones in the
appropriate Sobolev spaces as $\varepsilon\to 0$.
For each fixed $\varepsilon$, the smooth Bray--Khuri computation yields
an identity of the form
\[
  \div_{\overline g_\varepsilon} Y_\varepsilon
    = P_\varepsilon - \tfrac18 R_{\overline g_\varepsilon},
\]
with $Y_\varepsilon$ and $P_\varepsilon$ defined as above but with the
smoothed data.
Passing to the limit in the weak topology of $L^1_{\loc}$, using the
convergence of the coefficients and the uniform integrability provided
by the decay estimates, we obtain~\eqref{eq:BK-divergence}.
The nonnegativity of $P$ follows from pointwise convergence almost
everywhere and Fatou's lemma.
\end{proof}

%------------------------------------------------------------------
% 4. Static vacuum structure and regularity in the rigidity case
%------------------------------------------------------------------

\begin{lemma}[Static vacuum structure in the equality case]
\label{lem:StaticVacuum}
Assume the hypotheses of Theorem~\ref{thm:SPI} and suppose that
equality holds in the spacetime Penrose inequality
\eqref{eq:SPI}.
Then there exists a positive function $N$ on $\overline M$ such that
the static vacuum equations
\[
  \Ric_{\overline g} = N^{-1}\nabla^2_{\overline g}N,
  \qquad
  \Delta_{\overline g}N = 0
\]
hold in the sense of distributions on $\overline M$.
\end{lemma}

\begin{proof}
The proof follows the usual “equality implies rigidity” strategy.
Equality in \eqref{eq:SPI} forces equality in every estimate used in
the proof.
In particular:
\begin{itemize}
  \item the nonnegative integrand $P$ in
  Theorem~\ref{thm:BKidentity} must vanish almost everywhere, and
  \item the dominant energy condition is saturated:
  $\mu=|J|_g$ almost everywhere.
\end{itemize}
Moreover, equality in the AMO monotonicity formula implies that the
level sets of the limiting $1$-harmonic potential have constant mean
curvature and that the corresponding second fundamental form is pure
trace.
Tracing the Bochner-type inequality from
Appendix~\ref{app:Bochner} and using these equalities yields a function
$N>0$ such that
\[
  \nabla^2_{\overline g}N = N\,\Ric_{\overline g}
\]
in the weak sense.
Taking the trace gives $\Delta_{\overline g}N=0$.
\end{proof}

\begin{lemma}[Regularity and classification]
\label{lem:AndersonRegularity}
Under the hypotheses of Lemma~\ref{lem:StaticVacuum}, the pair
$(\overline g,N)$ is smooth and real analytic on the interior of
$\overline M$.
In particular, $(\overline M,\overline g)$ is isometric to a spatial
Schwarzschild slice outside its horizon.
\end{lemma}

\begin{proof}
The static vacuum system
\[
  \Ric_{\overline g} = N^{-1}\nabla^2_{\overline g}N,
  \qquad
  \Delta_{\overline g}N = 0
\]
is a uniformly elliptic system for $(\overline g,N)$ with analytic
nonlinearities.
Standard elliptic regularity implies smoothness of $\overline g$ and
$N$.
Real analyticity in harmonic coordinates then follows from classical
results on elliptic systems with analytic coefficients.
Finally, the classification of complete, asymptotically flat static
vacuum manifolds with connected horizon shows that $(\overline M,
\overline g)$ must coincide with the spatial Schwarzschild metric
outside its horizon, with mass equal to the ADM mass of the original
data.
\end{proof}
