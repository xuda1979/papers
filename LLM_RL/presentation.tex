\documentclass[aspectratio=169,11pt]{beamer}
\usetheme{Madrid}
\usecolortheme{whale}

% 中文支持
\usepackage{ctex}
\usepackage{amsmath,amssymb,amsfonts}
\usepackage{graphicx}
\usepackage{booktabs}
\usepackage{tikz}
\usetikzlibrary{shapes,arrows,positioning,calc}
% algorithm包需要特殊处理在beamer中
\usepackage{algorithm}
\usepackage{algorithmic}
\makeatletter
\newcommand{\removelatexerror}{\let\@latex@error\@gobble}
\makeatother
\usepackage{listings}
\usepackage{hyperref}
\usepackage{multicol}

% 代码样式
\lstset{
    basicstyle=\ttfamily\footnotesize,
    breaklines=true,
    frame=single
}

% 颜色定义
\definecolor{deepblue}{RGB}{0,51,102}
\definecolor{deepgreen}{RGB}{0,102,51}
\definecolor{deepred}{RGB}{153,0,0}

\setbeamertemplate{navigation symbols}{}
\setbeamertemplate{footline}[frame number]

\title[LLM强化学习训练]{大语言模型的强化学习训练\\提升推理、数学与代码能力}
\subtitle{无需人类标注的自我学习方法}
\author{许达}
\date{\today}
\institute{未来院三室}

\begin{document}

%=========================================
\begin{frame}
\titlepage
\end{frame}

%=========================================
\begin{frame}{目录}
\tableofcontents
\end{frame}

%=========================================
\section{背景与动机}
%=========================================

\begin{frame}{为什么需要强化学习训练LLM?}
\begin{columns}
\begin{column}{0.5\textwidth}
\textbf{传统方法的局限}
\begin{itemize}
    \item 监督微调(SFT)依赖大量人工标注数据
    \item 人类标注成本高、难以扩展
    \item 复杂推理任务的标注质量难以保证
    \item 模型只能学习到人类已知的解法
\end{itemize}
\end{column}
\begin{column}{0.5\textwidth}
\textbf{强化学习的优势}
\begin{itemize}
    \item \textcolor{deepgreen}{自动探索}:模型自主发现解题策略
    \item \textcolor{deepgreen}{可扩展}:无需人工标注
    \item \textcolor{deepgreen}{突破上限}:可能发现人类未知的方法
    \item \textcolor{deepgreen}{自我改进}:持续迭代优化
\end{itemize}
\end{column}
\end{columns}

\vspace{0.5cm}
\begin{block}{核心思想}
通过\textbf{奖励信号}(如答案正确性、代码执行结果)引导模型自主学习推理能力
\end{block}
\end{frame}

%=========================================
\begin{frame}{里程碑式进展}
\begin{center}
\begin{tikzpicture}[scale=0.9]
    % 时间线
    \draw[thick,->] (0,0) -- (14,0);
    
    % 节点
    \node[fill=blue!20,rounded corners,text width=2cm,align=center] at (1,1.5) {\small 2022\\RLHF\\InstructGPT};
    \draw[thick] (1,0) -- (1,0.8);
    
    \node[fill=green!20,rounded corners,text width=2cm,align=center] at (3.5,1.5) {\small 2022\\Constitutional AI\\RLAIF};
    \draw[thick] (3.5,0) -- (3.5,0.8);
    
    \node[fill=orange!20,rounded corners,text width=2cm,align=center] at (6,-1.5) {\small 2023\\Let's Verify\\Step by Step};
    \draw[thick] (6,0) -- (6,-0.8);
    
    \node[fill=purple!20,rounded corners,text width=2cm,align=center] at (8.5,1.5) {\small 2024\\DeepSeekMath\\GRPO};
    \draw[thick] (8.5,0) -- (8.5,0.8);
    
    \node[fill=red!20,rounded corners,text width=2.2cm,align=center] at (11,-1.5) {\small 2025.1\\DeepSeek-R1\\纯RL推理};
    \draw[thick] (11,0) -- (11,-0.8);
    
    \node[fill=yellow!30,rounded corners,text width=2cm,align=center] at (13,1.5) {\small 2024.9\\OpenAI o1\\推理模型};
    \draw[thick] (13,0) -- (13,0.8);
\end{tikzpicture}
\end{center}
\end{frame}

%=========================================
\begin{frame}{主要研究方向}
\begin{columns}
\begin{column}{0.33\textwidth}
\begin{block}{数学推理}
\begin{itemize}
    \item GSM8K, MATH
    \item 竞赛级问题
    \item 证明生成
\end{itemize}
\end{block}
\end{column}
\begin{column}{0.33\textwidth}
\begin{block}{代码生成}
\begin{itemize}
    \item HumanEval, MBPP
    \item 竞赛编程
    \item 代码调试
\end{itemize}
\end{block}
\end{column}
\begin{column}{0.33\textwidth}
\begin{block}{逻辑推理}
\begin{itemize}
    \item 多步推理
    \item 常识推理
    \item 科学问答
\end{itemize}
\end{block}
\end{column}
\end{columns}

\vspace{0.5cm}
\textbf{关键论文}:
\begin{itemize}
    \item Constitutional AI (Anthropic, 2022)
    \item Let's Verify Step by Step (OpenAI, 2023)
    \item DeepSeekMath (DeepSeek, 2024)
    \item DeepSeek-R1 (DeepSeek, 2025)
\end{itemize}
\end{frame}

%=========================================
\section{核心算法详解}
%=========================================

\begin{frame}{PPO算法回顾}
\textbf{Proximal Policy Optimization (PPO)} 是RL训练LLM的主流算法

\vspace{0.3cm}
\textbf{目标函数}:
$$L^{CLIP}(\theta) = \hat{\mathbb{E}}_t \left[ \min\left( r_t(\theta) \hat{A}_t, \text{clip}(r_t(\theta), 1-\epsilon, 1+\epsilon) \hat{A}_t \right) \right]$$

其中:
\begin{itemize}
    \item $r_t(\theta) = \frac{\pi_\theta(a_t|s_t)}{\pi_{\theta_{old}}(a_t|s_t)}$ 是策略比率
    \item $\hat{A}_t$ 是优势函数估计
    \item $\epsilon$ 是裁剪参数(通常0.1-0.2)
\end{itemize}

\vspace{0.3cm}
\textbf{关键组件}:
\begin{enumerate}
    \item \textbf{Actor}:策略网络 $\pi_\theta$(生成回答的LLM)
    \item \textbf{Critic}:价值网络 $V_\phi$(评估状态价值)
    \item \textbf{Reward Model}:奖励模型 $R_\psi$(评分)
\end{enumerate}
\end{frame}

%=========================================
\begin{frame}[fragile]{GRPO算法 - DeepSeek的创新}
\textbf{Group Relative Policy Optimization}:去除Critic网络,简化训练

\vspace{0.3cm}
\removelatexerror
\begin{algorithmic}[1]
\FOR{每个问题 $q$}
    \STATE 采样一组回答 $\{o_1, o_2, ..., o_G\}$ 从 $\pi_{\theta_{old}}$
    \STATE 计算奖励 $\{r_1, r_2, ..., r_G\}$
    \STATE 归一化优势: $\hat{A}_i = \frac{r_i - \text{mean}(\mathbf{r})}{\text{std}(\mathbf{r})}$
    \STATE 更新策略最大化:
\ENDFOR
\end{algorithmic}

$$\mathcal{J}_{GRPO}(\theta) = \mathbb{E}_{q, \{o_i\}} \left[ \frac{1}{G}\sum_{i=1}^{G} \min\left( \frac{\pi_\theta(o_i|q)}{\pi_{\theta_{old}}(o_i|q)} \hat{A}_i, \text{clip}(\cdot) \hat{A}_i \right) \right]$$

\vspace{0.2cm}
$$- \beta \cdot \mathbb{D}_{KL}[\pi_\theta || \pi_{ref}]$$
\end{frame}

%=========================================
\begin{frame}{GRPO vs PPO 对比}
\begin{center}
\begin{tabular}{lcc}
\toprule
\textbf{特性} & \textbf{PPO} & \textbf{GRPO} \\
\midrule
Critic网络 & 需要 & \textcolor{deepgreen}{不需要} \\
内存占用 & 高 & \textcolor{deepgreen}{低} \\
训练稳定性 & 依赖Critic质量 & \textcolor{deepgreen}{组内相对比较} \\
采样数量 & 单个/少量 & 多个(组采样) \\
优势估计 & GAE & \textcolor{deepgreen}{组内归一化} \\
实现复杂度 & 高 & \textcolor{deepgreen}{低} \\
\bottomrule
\end{tabular}
\end{center}

\vspace{0.5cm}
\textbf{GRPO核心创新}:
\begin{itemize}
    \item 用\textbf{组内相对排名}代替绝对价值估计
    \item 同一问题的多个回答互相比较
    \item 显著降低计算成本,保持训练效果
\end{itemize}
\end{frame}

%=========================================
\begin{frame}{奖励函数设计}
\textbf{可验证奖励}(无需人类标注):

\vspace{0.3cm}
\begin{columns}
\begin{column}{0.5\textwidth}
\textbf{数学问题}
\begin{itemize}
    \item 答案匹配:$r = \mathbf{1}[\text{answer} = \text{ground\_truth}]$
    \item 格式奖励:正确使用\texttt{\textbackslash boxed\{\}}
    \item 中间步骤验证(可选)
\end{itemize}

\vspace{0.3cm}
\textbf{代码问题}
\begin{itemize}
    \item 执行结果:$r = \frac{\#\text{passed tests}}{\#\text{total tests}}$
    \item 编译成功:额外奖励
    \item 效率奖励(可选)
\end{itemize}
\end{column}
\begin{column}{0.5\textwidth}
\textbf{复合奖励}
$$R = \alpha \cdot R_{accuracy} + \beta \cdot R_{format} + \gamma \cdot R_{length}$$

\vspace{0.3cm}
\textbf{长度惩罚}(防止冗长)
$$R_{length} = -\lambda \cdot \max(0, L - L_{threshold})$$

\vspace{0.3cm}
\textbf{格式奖励}
$$R_{format} = \begin{cases} 
r_+ & \text{格式正确} \\
r_- & \text{格式错误}
\end{cases}$$
\end{column}
\end{columns}
\end{frame}

%=========================================
\section{DeepSeek-R1:纯RL推理模型}
%=========================================

\begin{frame}{DeepSeek-R1 概述}
\textbf{2025年1月发布},首个公开的纯RL训练推理模型

\vspace{0.5cm}
\begin{columns}
\begin{column}{0.5\textwidth}
\textbf{核心发现}
\begin{itemize}
    \item 纯RL可以\textbf{涌现}推理能力
    \item 无需SFT冷启动
    \item 自发学会Chain-of-Thought
    \item 出现"Aha moment"顿悟现象
\end{itemize}
\end{column}
\begin{column}{0.5\textwidth}
\textbf{性能表现}
\begin{itemize}
    \item AIME 2024: \textbf{79.8\%}(接近o1)
    \item MATH-500: \textbf{97.3\%}
    \item Codeforces: \textbf{2029} rating
    \item 超越多数闭源模型
\end{itemize}
\end{column}
\end{columns}

\vspace{0.5cm}
\begin{alertblock}{重要突破}
证明了\textbf{大规模RL训练}可以让模型自主学习复杂推理,而非仅仅模仿人类解法
\end{alertblock}
\end{frame}

%=========================================
\begin{frame}{DeepSeek-R1 训练流程}
\textbf{四阶段训练流程}:

\vspace{0.3cm}
\begin{enumerate}
    \item \textbf{Stage 1: 冷启动SFT} - 少量长CoT数据(数千条)
    \item \textbf{Stage 2: 推理RL} - GRPO + 规则奖励
    \item \textbf{Stage 3: 拒绝采样SFT} - 用RL检查点生成数据
    \item \textbf{Stage 4: 全场景RL} - 扩展到更多任务
\end{enumerate}

\vspace{0.3cm}
$$\text{Base Model} \rightarrow \text{冷启动SFT} \rightarrow \text{推理RL} \rightarrow \text{拒绝采样} \rightarrow \text{全场景RL}$$

\vspace{0.3cm}
\textbf{关键点}:
\begin{itemize}
    \item Stage 2使用\textbf{纯规则奖励},无需奖励模型
    \item 奖励 = 准确性 + 格式正确性
    \item 模型自发产生长链推理
\end{itemize}
\end{frame}

%=========================================
\begin{frame}{DeepSeek-R1-Zero:纯RL实验}
\textbf{无任何SFT},直接从Base模型开始RL训练

\vspace{0.5cm}
\textbf{惊人发现}:
\begin{enumerate}
    \item \textbf{自我验证}:模型学会检查自己的答案
    \item \textbf{反思}:发现错误后重新思考
    \item \textbf{长链推理}:自动产生详细推导步骤
    \item \textbf{探索多种方法}:尝试不同解题路径
\end{enumerate}

\vspace{0.3cm}
\begin{exampleblock}{Aha Moment示例}
\textit{"Wait, let me reconsider this problem..."}\\
\textit{"Hmm, I made an error in step 3. Let me redo..."}\\
模型自发产生的反思性语言!
\end{exampleblock}

\vspace{0.3cm}
\textbf{局限}:可读性差、语言混杂,需要后续SFT改进
\end{frame}

%=========================================
\begin{frame}{DeepSeek-R1 奖励设计}
\textbf{极简奖励函数}(Stage 2 推理RL):

\vspace{0.3cm}
$$R = R_{accuracy} + R_{format}$$

\begin{itemize}
    \item \textbf{准确性奖励}:
    \begin{itemize}
        \item 数学:答案与标准答案匹配
        \item 代码:通过测试用例
        \item 使用规则验证,无需神经网络
    \end{itemize}
    
    \item \textbf{格式奖励}:
    \begin{itemize}
        \item 要求输出包含 \texttt{<think>...</think>} 推理过程
        \item 最终答案在 \texttt{<answer>...</answer>} 中
        \item 惩罚格式不正确的输出
    \end{itemize}
\end{itemize}

\vspace{0.3cm}
\begin{alertblock}{为什么有效?}
\begin{itemize}
    \item 准确性奖励提供\textbf{稀疏但准确}的信号
    \item 模型被迫学习正确的推理过程来提高准确率
    \item 无需人类标注的推理链
\end{itemize}
\end{alertblock}
\end{frame}

%=========================================
\begin{frame}{蒸馏到小模型}
DeepSeek-R1的推理能力可以\textbf{蒸馏}到小模型

\vspace{0.5cm}
\begin{center}
\begin{tabular}{lccc}
\toprule
\textbf{模型} & \textbf{参数量} & \textbf{AIME 2024} & \textbf{MATH-500} \\
\midrule
DeepSeek-R1 & 671B & 79.8\% & 97.3\% \\
\midrule
R1-Distill-Qwen-32B & 32B & 72.6\% & 94.3\% \\
R1-Distill-Qwen-14B & 14B & 69.7\% & 93.9\% \\
R1-Distill-Qwen-7B & 7B & 55.5\% & 92.8\% \\
R1-Distill-Qwen-1.5B & 1.5B & 28.9\% & 83.9\% \\
\midrule
OpenAI o1-mini & - & 63.6\% & 90.0\% \\
\bottomrule
\end{tabular}
\end{center}

\vspace{0.3cm}
\textbf{蒸馏方法}:用R1生成的长链推理数据对小模型进行SFT
\end{frame}

%=========================================
\section{OpenAI o1 技术分析}
%=========================================

\begin{frame}{OpenAI o1 概述}
\textbf{2024年9月发布},开启"推理模型"新范式

\vspace{0.5cm}
\begin{columns}
\begin{column}{0.5\textwidth}
\textbf{官方披露}
\begin{itemize}
    \item 使用强化学习训练
    \item 产生长链"思考"过程
    \item 思考过程对用户隐藏
    \item 性能随计算量scaling
\end{itemize}
\end{column}
\begin{column}{0.5\textwidth}
\textbf{性能亮点}
\begin{itemize}
    \item AIME 2024: \textbf{83.3\%}(13.4/15)
    \item Codeforces: \textbf{89 percentile}
    \item GPQA Diamond: \textbf{78\%}
    \item PhD级科学问答
\end{itemize}
\end{column}
\end{columns}

\vspace{0.5cm}
\begin{block}{核心理念}
\textit{"Learning to reason with reinforcement learning"}\\
通过RL学习如何进行有效的推理
\end{block}
\end{frame}

%=========================================
\begin{frame}{o1 技术推测}
\textbf{基于公开信息和研究社区分析}

\vspace{0.3cm}
\textbf{可能的训练流程}:
\begin{enumerate}
    \item \textbf{过程奖励模型(PRM)}训练
    \begin{itemize}
        \item 参考"Let's Verify Step by Step"论文
        \item 人工标注推理步骤的正确性
        \item 训练模型评估每个步骤
    \end{itemize}
    
    \item \textbf{大规模RL训练}
    \begin{itemize}
        \item 使用PRM作为奖励信号
        \item 可能结合结果奖励(ORM)
        \item 大量计算资源
    \end{itemize}
    
    \item \textbf{搜索/采样策略}
    \begin{itemize}
        \item 推理时多次采样
        \item 可能使用树搜索
        \item Best-of-N或MCTS变体
    \end{itemize}
\end{enumerate}
\end{frame}

%=========================================
\begin{frame}{过程奖励模型 (PRM)}
\textbf{Let's Verify Step by Step} (OpenAI, 2023)

\vspace{0.3cm}
\textbf{核心思想}:奖励每个推理步骤,而非只看最终答案

\vspace{0.3cm}
\begin{columns}
\begin{column}{0.5\textwidth}
\textbf{结果奖励 (ORM)}
$$R = \begin{cases} 
+1 & \text{最终答案正确} \\
0 & \text{最终答案错误}
\end{cases}$$
\begin{itemize}
    \item 信号稀疏
    \item 难以定位错误
    \item Credit assignment问题
\end{itemize}
\end{column}
\begin{column}{0.5\textwidth}
\textbf{过程奖励 (PRM)}
$$R = \sum_{t=1}^{T} r_t$$
$$r_t = P(\text{step } t \text{ is correct})$$
\begin{itemize}
    \item 密集信号
    \item 精确定位错误步骤
    \item 需要人工标注
\end{itemize}
\end{column}
\end{columns}

\vspace{0.3cm}
\textbf{实验结论}:PRM在数学推理上显著优于ORM
\end{frame}

%=========================================
\begin{frame}{o1 Scaling特性}
\textbf{推理时计算的scaling law}

\vspace{0.3cm}
\begin{columns}
\begin{column}{0.6\textwidth}
\textbf{关键发现}:
\begin{itemize}
    \item 性能随推理时间/token数量提升
    \item "思考"越多,答案越准确
    \item 存在新的scaling维度
\end{itemize}

\vspace{0.3cm}
\textbf{两种scaling}:
\begin{enumerate}
    \item \textbf{训练时scaling}:更多数据、更大模型
    \item \textbf{推理时scaling}:更多思考、更多搜索
\end{enumerate}
\end{column}
\begin{column}{0.4\textwidth}
\begin{center}
\textit{(推理时间 vs 准确率)}

\vspace{0.5cm}
更多思考时间 $\rightarrow$ 更高准确率

\vspace{0.5cm}
类似人类:\\
难题需要更多思考
\end{center}
\end{column}
\end{columns}

\vspace{0.3cm}
\begin{alertblock}{范式转变}
从"更大的模型"到"更多的思考"
\end{alertblock}
\end{frame}

%=========================================
\section{Google与Anthropic方法}
%=========================================

\begin{frame}{Constitutional AI (Anthropic)}
\textbf{RLAIF}:用AI反馈代替人类反馈

\vspace{0.5cm}
\textbf{训练流程}:
\begin{enumerate}
    \item \textbf{Self-Critique}:模型评估自己的回答
    \item \textbf{Revision}:根据"宪法"原则修改回答
    \item \textbf{RL训练}:用AI评分作为奖励
\end{enumerate}

\vspace{0.3cm}
\textbf{"宪法"示例}:
\begin{itemize}
    \item 选择最有帮助、诚实、无害的回答
    \item 选择最不具有欺骗性的回答
    \item 选择最尊重用户自主权的回答
\end{itemize}

\vspace{0.3cm}
\textbf{优势}:
\begin{itemize}
    \item 大幅减少人类标注需求
    \item 可扩展到更多场景
    \item 原则可编程、可调整
\end{itemize}
\end{frame}

%=========================================
\begin{frame}{Google的推理研究}
\textbf{主要工作}:

\vspace{0.3cm}
\begin{enumerate}
    \item \textbf{Chain-of-Thought Prompting}
    \begin{itemize}
        \item 提示模型展示推理步骤
        \item Zero-shot CoT: "Let's think step by step"
        \item 显著提升数学推理能力
    \end{itemize}
    
    \vspace{0.3cm}
    \item \textbf{Self-Consistency}
    \begin{itemize}
        \item 采样多个推理路径
        \item 投票选择最一致的答案
        \item 简单有效的提升方法
    \end{itemize}
    
    \vspace{0.3cm}
    \item \textbf{Gemini系列}
    \begin{itemize}
        \item 多模态理解
        \item 长上下文处理
        \item 推理能力持续提升
    \end{itemize}
\end{enumerate}
\end{frame}

%=========================================
\section{技术细节与实现}
%=========================================

\begin{frame}{训练数据构造}
\textbf{高质量问题来源}:

\vspace{0.3cm}
\begin{columns}
\begin{column}{0.5\textwidth}
\textbf{数学}
\begin{itemize}
    \item GSM8K (小学数学)
    \item MATH (竞赛数学)
    \item AIME/AMC历年真题
    \item 合成数学问题
    \item 网络爬取的数学问答
\end{itemize}
\end{column}
\begin{column}{0.5\textwidth}
\textbf{代码}
\begin{itemize}
    \item LeetCode题目
    \item Codeforces竞赛题
    \item HumanEval/MBPP
    \item GitHub代码库
    \item 合成编程问题
\end{itemize}
\end{column}
\end{columns}

\vspace{0.5cm}
\textbf{数据增强}:
\begin{itemize}
    \item 问题改写/变体生成
    \item 难度渐进
    \item 多种解法生成
\end{itemize}
\end{frame}

%=========================================
\begin{frame}{采样与探索策略}
\textbf{RL训练中的采样}:

\vspace{0.3cm}
\begin{enumerate}
    \item \textbf{温度采样}
    $$P(token) \propto \exp(\text{logit}/T)$$
    \begin{itemize}
        \item 训练时$T > 1$增加探索
        \item 推理时$T < 1$提高质量
    \end{itemize}
    
    \vspace{0.2cm}
    \item \textbf{Top-p (Nucleus) 采样}
    \begin{itemize}
        \item 只考虑累积概率达到$p$的tokens
        \item 平衡多样性和质量
    \end{itemize}
    
    \vspace{0.2cm}
    \item \textbf{Best-of-N}
    \begin{itemize}
        \item 生成N个候选
        \item 用奖励模型选择最佳
        \item 推理时常用策略
    \end{itemize}
\end{enumerate}
\end{frame}

%=========================================
\begin{frame}{KL散度约束}
\textbf{防止策略偏离太远}:

$$\mathcal{L} = \mathcal{L}_{RL} - \beta \cdot \mathbb{D}_{KL}[\pi_\theta || \pi_{ref}]$$

\vspace{0.3cm}
\textbf{为什么需要KL约束?}
\begin{itemize}
    \item 防止\textbf{奖励黑客}(reward hacking)
    \item 保持模型的语言能力
    \item 稳定训练过程
\end{itemize}

\vspace{0.3cm}
\textbf{KL计算}(近似):
$$\mathbb{D}_{KL}[\pi_\theta || \pi_{ref}] \approx \mathbb{E}_{\pi_\theta} \left[ \log \frac{\pi_\theta(a|s)}{\pi_{ref}(a|s)} \right]$$

\vspace{0.3cm}
\textbf{$\beta$选择}:
\begin{itemize}
    \item 太大:限制学习
    \item 太小:不稳定/奖励黑客
    \item 典型值:0.01 - 0.1
\end{itemize}
\end{frame}

%=========================================
\begin{frame}{训练稳定性技巧}
\textbf{大规模RL训练的挑战与解决}:

\vspace{0.3cm}
\begin{enumerate}
    \item \textbf{梯度裁剪}
    $$g \leftarrow \min\left(1, \frac{c}{||g||}\right) \cdot g$$
    
    \item \textbf{学习率调度}
    \begin{itemize}
        \item Warmup阶段
        \item Cosine decay
        \item 比SFT更小的学习率
    \end{itemize}
    
    \item \textbf{奖励归一化}
    $$r_{norm} = \frac{r - \mu_r}{\sigma_r}$$
    
    \item \textbf{多次PPO epochs}
    \begin{itemize}
        \item 每批数据更新多次
        \item 但需要控制更新幅度
    \end{itemize}
\end{enumerate}
\end{frame}

%=========================================
\section{失败尝试与经验教训}
%=========================================

\begin{frame}{DeepSeek报告的失败尝试}
\textbf{重要的负面结果}:

\vspace{0.3cm}
\begin{enumerate}
    \item \textbf{过程奖励模型(PRM)效果不佳}
    \begin{itemize}
        \item 训练PRM本身很困难
        \item 需要大量步骤级标注
        \item 泛化能力有限
        \item 最终放弃,使用结果奖励
    \end{itemize}
    
    \vspace{0.2cm}
    \item \textbf{Monte Carlo Tree Search (MCTS)未成功}
    \begin{itemize}
        \item 搜索空间太大
        \item 价值估计不准确
        \item 计算成本过高
    \end{itemize}
    
    \vspace{0.2cm}
    \item \textbf{直接从Base模型RL的问题}
    \begin{itemize}
        \item 可读性差
        \item 语言混杂
        \item 需要少量SFT冷启动
    \end{itemize}
\end{enumerate}
\end{frame}

%=========================================
\begin{frame}{常见问题与解决}
\begin{columns}
\begin{column}{0.5\textwidth}
\textbf{奖励黑客}
\begin{itemize}
    \item 问题:找到漏洞获取高奖励
    \item 解决:KL约束、多样化奖励
\end{itemize}

\vspace{0.3cm}
\textbf{模式崩塌}
\begin{itemize}
    \item 问题:输出变得单一
    \item 解决:熵正则化、温度控制
\end{itemize}

\vspace{0.3cm}
\textbf{训练不稳定}
\begin{itemize}
    \item 问题:损失震荡、发散
    \item 解决:小学习率、梯度裁剪
\end{itemize}
\end{column}
\begin{column}{0.5\textwidth}
\textbf{遗忘问题}
\begin{itemize}
    \item 问题:失去原有能力
    \item 解决:混合训练数据
\end{itemize}

\vspace{0.3cm}
\textbf{长度控制}
\begin{itemize}
    \item 问题:输出过长或过短
    \item 解决:长度惩罚/奖励
\end{itemize}

\vspace{0.3cm}
\textbf{格式问题}
\begin{itemize}
    \item 问题:不遵循指定格式
    \item 解决:格式奖励、SFT预训练
\end{itemize}
\end{column}
\end{columns}
\end{frame}

%=========================================
\section{未来方向}
%=========================================

\begin{frame}{研究前沿}
\textbf{值得关注的方向}:

\vspace{0.3cm}
\begin{enumerate}
    \item \textbf{更高效的RL算法}
    \begin{itemize}
        \item 减少样本复杂度
        \item 更稳定的训练
        \item 更低的计算成本
    \end{itemize}
    
    \item \textbf{更好的奖励设计}
    \begin{itemize}
        \item 自动发现奖励函数
        \item 多目标优化
        \item 可解释的奖励
    \end{itemize}
    
    \item \textbf{推理时scaling}
    \begin{itemize}
        \item 更高效的搜索算法
        \item 自适应计算量
        \item 推理成本优化
    \end{itemize}
    
    \item \textbf{多模态推理}
    \begin{itemize}
        \item 视觉推理
        \item 跨模态reasoning
    \end{itemize}
\end{enumerate}
\end{frame}

%=========================================
\begin{frame}{开放问题}
\textbf{尚未解决的挑战}:

\vspace{0.5cm}
\begin{itemize}
    \item \textbf{泛化性}:如何让推理能力迁移到新领域?
    
    \vspace{0.2cm}
    \item \textbf{可解释性}:模型真的在"推理"还是在模式匹配?
    
    \vspace{0.2cm}
    \item \textbf{长程规划}:如何进行更长horizon的推理?
    
    \vspace{0.2cm}
    \item \textbf{世界模型}:是否需要显式的世界知识?
    
    \vspace{0.2cm}
    \item \textbf{安全性}:推理模型可能更擅长欺骗?
    
    \vspace{0.2cm}
    \item \textbf{效率}:如何降低推理成本?
\end{itemize}
\end{frame}

%=========================================
\section{总结}
%=========================================

\begin{frame}{核心要点回顾}
\begin{enumerate}
    \item \textbf{RL是提升LLM推理能力的有效方法}
    \begin{itemize}
        \item 无需大量人类标注
        \item 可以超越人类示范
    \end{itemize}
    
    \vspace{0.2cm}
    \item \textbf{GRPO简化了训练流程}
    \begin{itemize}
        \item 无需Critic网络
        \item 组内相对比较
    \end{itemize}
    
    \vspace{0.2cm}
    \item \textbf{可验证奖励是关键}
    \begin{itemize}
        \item 数学:答案匹配
        \item 代码:执行验证
    \end{itemize}
    
    \vspace{0.2cm}
    \item \textbf{推理时Scaling开辟新维度}
    \begin{itemize}
        \item 更多思考=更好结果
        \item Test-time compute
    \end{itemize}
\end{enumerate}
\end{frame}

%=========================================
\begin{frame}{参考资料}
\footnotesize
\begin{itemize}
    \item DeepSeek-R1: Incentivizing Reasoning Capability in LLMs via RL (2025)
    \item DeepSeekMath: Pushing the Limits of Mathematical Reasoning (2024)
    \item Let's Verify Step by Step (OpenAI, 2023)
    \item Constitutional AI: Harmlessness from AI Feedback (Anthropic, 2022)
    \item Training Language Models to Follow Instructions with Human Feedback (OpenAI, 2022)
    \item OpenAI o1 System Card (2024)
    \item Scaling Laws for Reward Model Overoptimization (2022)
    \item Chain-of-Thought Prompting Elicits Reasoning in LLMs (Google, 2022)
\end{itemize}

\vspace{0.5cm}
\begin{center}
\Large 谢谢!欢迎提问讨论
\end{center}
\end{frame}

\end{document}
