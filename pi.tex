\documentclass[11pt]{amsart}

% ---------------------- Packages -------------------------------
\usepackage[a4paper,margin=1in]{geometry}
\usepackage{amsmath,amssymb,amsthm,amsfonts}
\usepackage{booktabs}
\usepackage{hyperref}
%\usepackage{algorithm}
%\usepackage{algpseudocode}
\usepackage{mathtools}
\usepackage{enumitem}
\usepackage{cleveref}

% ------------------- Theorem Environments ----------------------
\newtheorem{theorem}{Theorem}[section]
\newtheorem{lemma}[theorem]{Lemma}
\newtheorem{corollary}[theorem]{Corollary}
\theoremstyle{remark}
\newtheorem{remark}[theorem]{Remark}

% -------------------- Custom Commands --------------------------
\newcommand{\cG}{\mathfrak{c}}      % Heat-kernel constant
\newcommand{\diam}{\operatorname{diam}}   % Diameter operator

\title{Spectral Recovery of $\pi$ from Quadratic-Growth Graphs via Zeta Functions}
\author{Da Xu}
\address{China Mobile Research Institute, Beijing, P.R.~China}
%\email{...}

\subjclass{Primary 39A12; Secondary 31C20, 60J10, 05C81}
\keywords{spectral zeta function, graph Laplacian, heat kernel, random walk, quadratic volume growth, quasi-isometry}
\date{}

\begin{document}
\begin{abstract}
Let $G$ be an infinite graph of bounded degree and
\emph{quadratic volume growth}, i.e.\ $|B_R(x)|\asymp R^{2}$ for every
vertex $x$ and $R\ge1$.
For a finite connected subgraph $G_n\subset G$ with
$N_n:=|V(G_n)|\to\infty$ and Dirichlet Laplacian eigenvalues
$0=\lambda^{(n)}_{0}<\lambda^{(n)}_{1}\le\cdots\le\lambda^{(n)}_{N_n-1}$,
set
\[
  Z_n(1)=\sum_{k=1}^{N_n-1}\frac1{\lambda^{(n)}_{k}}.
\]
Write $\cG=\cG(G)$ for the intrinsic universal constant defined via the lazy
random-walk return probability
\begin{equation}
    \sum_{t=1}^{R}p_{t}^{\textnormal{lazy}}(x,x)=2\cG\log R+O(1) \label{eq:return-sum-abstract}
\end{equation}
(see \eqref{eq:return-sum} below for details).

We prove the sharp asymptotic
\[
  Z_n(1) \sim 2\cG\,N_n\log N_n \quad\text{as } n\to\infty,
\]
which is equivalent to \(\lim_{n\to\infty} N_n\log N_n / Z_n(1) = 1/(2\cG)\).

For graphs quasi-isometric to $\mathbb Z^{2}$ the Bass-Kumagai heat-kernel constant satisfies $\cG=1/\pi$, so the limit equals $\pi/2$. This spectrally recovers a universal constant related to $\pi$ emerging from quadratic growth geometry.
\end{abstract}

\maketitle

% MSC and Keywords
\bigskip\noindent \textbf{2020 Mathematics Subject Classification.} Primary 39A12; Secondary 31C20, 60J10, 05C81.

\noindent \textbf{Keywords.} spectral zeta function, graph Laplacian, heat kernel, random walk, quadratic volume growth, quasi-isometry.

% ===============================================================
\section{Motivation and Significance}

The classical result $\lim_{n\to\infty} N_n\log N_n/Z_n(1) = \pi$ for Euclidean domains 
and lattices connects geometric growth to spectral properties (see 
\cite{Colin85} for smooth domains, \cite{Frank10} for flat tori). This work extends this 
fundamental relationship to general graphs with quadratic volume growth, revealing 
a universal spectral invariant related to $\pi$. Our results have implications for:

\begin{itemize}
\item \textbf{Random walks on graphs:} The constant $\cG$ governs return probabilities, 
      linking spectral zeta functions to stochastic processes.
\item \textbf{Spectral geometry:} Provides discrete analogues of Weyl law asymptotics 
      for non-Euclidean spaces with quadratic growth.
\item \textbf{Network analysis:} Offers spectral methods to detect $\mathbb{Z}^2$-like 
      substructures in complex networks via convergence to $\pi/2$.
\end{itemize}

% ===============================================================
\section{Main Result}
\begin{remark}\label{rem:assumptions}
The bounded-degree condition ensures uniform parabolic Harnack inequalities \cite{Delmotte99}, which underlie the heat kernel estimates used throughout the paper. 
The assumption that the exhaustion is by metric balls is used to control the boundary layer via the isoperimetric inequality. 
It remains open whether the result holds for general exhaustions.

Quadratic growth is essential: cubic growth yields $Z_n(1) \asymp N_n$ (no log factor), 
while linear growth gives $Z_n(1) \asymp N_n^2$. Violations alter the asymptotic fundamentally.
\end{remark}
 

\begin{theorem}\label{thm:main}
Let $G$ be an infinite, bounded-degree graph with quadratic growth and
associated heat-kernel constant $\cG$.
For any exhaustion $\{G_n\}$ of $G$ by metric balls $G_n = B_{R_n}(x_0)$ with $N_n\to\infty$,
\[
  \lim_{n\to\infty}\frac{N_n\log N_n}{Z_n(1)}=\frac{1}{2\cG}.
\]
Equivalently, $Z_n(1) \sim 2\cG N_n \log N_n$ as $n\to\infty$.
\end{theorem}


\begin{remark}[Universality of $\pi$]\label{rem:pi2}
For every graph $G$ quasi-isometric to $\mathbb Z^{2}$ (e.g., square or triangular grids), 
Bass and Kumagai \cite{BassKumagai08}, together with Delmotte's parabolic Harnack theorem 
\cite{Delmotte99}, yield $\cG=1/\pi$. 
Note that $\cG = 1/\pi$ for $\mathbb{Z}^2$, by comparing the discrete return probability to the Brownian return law on $\mathbb{R}^2$. 
The emergence of $\pi$ reflects the \emph{universal geometry} of 
quadratic growth: quasi-isometry preserves heat-kernel asymptotics, forcing $\cG=1/\pi$ 
through the continuum limit of Brownian motion on $\mathbb{R}^2$. 
The constant $\cG$ generalizes the \emph{capacity per unit area} in potential theory: 
for $\mathbb{Z}^2$, $\cG = 1/\pi$ matches the reciprocal continuum capacity of the unit disk in $\mathbb{R}^2$ under Brownian motion. Specifically, this capacity equals $\pi$ for the unit disk, so $\cG = 1/\pi$ is its reciprocal.
Thus, \Cref{thm:main} 
yields the spectral limit $N_n\log N_n/Z_n(1)\to\pi/2$. This result provides a spectral recovery of a universal constant directly related to $\pi$, differing by a factor of 2 from the classical continuous case, a discrepancy arising from the specific asymptotic form of the discrete heat kernel sum.
\end{remark}

 

% ===============================================================
\section{Preliminaries}

Let $G=(V,E)$ be infinite and connected with maximum degree
$\Delta<\infty$. Denote graph distance by $d_G(\cdot,\cdot)$ and let
$B_R(x)=\{y\in V:d_G(x,y)\le R\}$.
Quadratic growth means
\begin{equation}\label{eq:quad}
  c_1R^{2}\le |B_R(x)|\le c_2R^{2},\qquad R\ge1,\;x\in V,
\end{equation}
for some $c_1,c_2>0$.

\subsection{Random walk and heat kernel}
Let $(X_t)_{t\ge0}$ be the \emph{lazy} simple random walk
(stay put with probability $1/2$) and set
$p_t(x,y)=\mathbb{P}_x[X_t=y]$.
Bass and Kumagai \cite{BassKumagai08} and Delmotte \cite{Delmotte99}
show that on bounded-degree graphs satisfying \eqref{eq:quad}
there exists a universal constant $\cG>0$ such that
\begin{equation}\label{eq:return-sum}
  \sum_{t=1}^{R} p_t(x,x)=2\cG\log R+O(1),\qquad R\ge2,
\end{equation}
uniformly in $x$. For $\mathbb Z^{2}$ one has $\cG=1/\pi$.

\subsection{Green function and $Z_n(1)$}
For a finite $H\subset G$ with Dirichlet boundary let
$G_H(u,v)=\sum_{t\ge0}p_t^{H}(u,v)$ be its Green function.
Then
\(
  Z_H(1)=\sum_{v\in V(H)}G_H(v,v)=\operatorname{tr}(L_H^{+}),
\)
where $L_H^{+}$ is the Moore--Penrose pseudoinverse of the
combinatorial Laplacian $L_H$.

% ===============================================================
\section{Interior--Boundary Decomposition}

Fix an exhaustion $\mathbf{G_n = B_{R_n}(x_0)}$ by metric balls rooted at $x_0$.  For a parameter $\eta\in(0,\tfrac14)$ we decompose the vertex set into an 
\emph{interior} and a \emph{boundary layer}.  Set 
\[
  I_n := \{x\in G_n : d_G(x,\partial G_n) > R_n^{1-\eta}\},
  \qquad
  E_n := G_n \setminus I_n.
\]
That is, $I_n$ consists of vertices at distance strictly greater than $R_n^{1-\eta}$ from the boundary of $G_n$, and $E_n$ collects the remaining (boundary) vertices.  Since $G_n$ is a metric ball, $|G_n|=N_n$ and $R_n\asymp N_n^{1/2}$.

By the isoperimetric inequality for graphs of quadratic growth (see \cite[Lemma 2.3]{Delmotte99}), 
$|\partial B_R| \le C R$ for some constant $C>0$. Thus
\begin{align*}
|E_n| &\le |\partial B_{R_n}| \cdot R_n^{1-\eta} \\
        &\le C R_n \cdot R_n^{1-\eta} \\
        &= C R_n^{2-\eta} \\
        &= O(N_n^{1-\eta/2})\,,
\end{align*}
since $R_n \asymp N_n^{1/2}$. The exponent $\eta/2$ directly controls the convergence rate in \Cref{thm:main},
where $\eta<1/4$ is required for Harnack inequalities near the boundary. 
 
The relative error in the asymptotic is $O(N_n^{-\eta/2})$, minimized at $O(N_n^{-1/8})$ 
when $\eta \uparrow 1/4$. This rate controls the convergence speed in \Cref{thm:main}, 
with $\eta < 1/4$ ensuring boundary Harnack inequalities hold uniformly.

The relative error $\varepsilon_n \sim |E_n|/N_n = N_n^{-\eta/2}$ 
is at best $O(N_n^{-1/8})$ 
when $\eta$ approaches $1/4$. This rate is optimal under our interior-boundary decomposition; 
finer analysis (e.g., boundary Harnack principles) might improve it but lies beyond our scope.

\begin{remark}\label{rem:boundary-harnack}
The restriction $\eta < 1/4$ ensures the boundary layer $E_n$ is sufficiently thin for 
uniform parabolic Harnack inequalities \cite{Delmotte99} to hold in $I_n$. 
This is critical for the heat-kernel estimates in \Cref{lem:lower}.
\end{remark}

The rate $O(N_n^{-1/8})$ is optimal under our interior--boundary decomposition; finer analysis (for instance via more delicate boundary Harnack principles) might improve it but lies beyond our scope.
% ===============================================================
\section{Lower Bound on $Z_n(1)$}

\begin{lemma}\label{lem:lower}
For every $v \in I_n$ there exists a constant $C_1>0$ such that
\[
  G_{G_n}(v,v) \ge 4\,\cG\,(1-\eta) \log R_n - C_1.
\]
That is, the Green function at interior vertices grows at least like $4\cG(1-\eta)\log R_n$.
\end{lemma}

\begin{proof}
 

Set $T = \lfloor R_n^{2(1-\eta)} \rfloor$ and let $\tau_{\partial}$ be the exit time from $G_n$. Domain monotonicity and the strong Markov property yield the decomposition
\[
  G_{G_n}(v,v) = \sum_{t\ge 0} p_t^{G_n}(v,v) \ge \sum_{t=0}^{T} p_t(v,v) - \sum_{t=0}^{T} \mathbb{P}_v(\tau_{\partial} \le t),
\]
where the first sum counts return probabilities on the infinite graph up to time $T$, and the second subtracts trajectories that exit $G_n$ before time $T$.

By the maximal inequality for random walks on graphs satisfying the parabolic Harnack inequality \cite[Corollary 3.3]{Delmotte99}, there exist constants $C,c>0$ such that
\[
  \mathbb{P}_v\!\bigl( \max_{0 \le s \le t} d_G(v,X_s) \ge d \bigr) \le C\,\exp\Bigl(-c\,\frac{d^2}{t}\Bigr).
\]
Since $d_G(v, \partial G_n) \ge R_n^{1-\eta}$, the event $\{\tau_{\partial} \le t\}$ forces $\max_{0 \le s \le t} d_G(v,X_s) \ge R_n^{1-\eta}$. Consequently,
\[
  \mathbb{P}_v(\tau_{\partial} \le t) \le C\,\exp\Bigl(-c\,\frac{R_n^{2(1-\eta)}}{t+1}\Bigr).
\]

Summing the exit probability:
\begin{align*}
  \sum_{t=0}^T \mathbb{P}_v(\tau_{\partial} \le t) 
    &\le \sum_{t=0}^T C \exp\!\Big(-c \frac{R_n^{2(1-\eta)}}{t+1}\Big) \\
    &\le \int_0^\infty C \exp\!\Big(-c \frac{R_n^{2(1-\eta)}}{u+1}\Big) du \\
    &= \frac{C}{c} R_n^{2(1-\eta)} \int_0^\infty \exp\!\Big(-c \frac{R_n^{2(1-\eta)}}{u+1}\Big) \frac{c}{R_n^{2(1-\eta)}} du \\
    &= \frac{C}{c} \int_0^\infty e^{-w} \frac{dw}{(w + 1)^2}  \quad \text{(via } w = c\tfrac{R_n^{2(1-\eta)}}{u+1}\text{)} \\
    &\leq \frac{C}{c} \int_0^\infty e^{-w} dw = O(1)\,.
\end{align*}
Meanwhile, by \eqref{eq:return-sum} and the choice of $T$ we have
\[
  \sum_{t=1}^{T} p_t(v,v)
  = 2\cG\,\log T + O(1) = 2\cG\,[2(1-\eta)\log R_n] + O(1) = 4\cG\,(1-\eta)\log R_n + O(1).
\]
Summing the exit-probability bound over $t=0,\dots,T$ yields a contribution that is $O(1)$.  Combining the two estimates proves the desired inequality with an appropriate constant $C_1>0$.
\end{proof}

% ===============================================================
\section{Upper Bound on $Z_n(1)$}

\begin{lemma}\label{lem:upper}
For $v\in V(G_n)$,
\[
  G_{G_n}(v,v) \leq 4\cG \log(\diam(G_n)) + C_3.
\]
\end{lemma}

\begin{proof}
Set $D = \diam(G_n)$.  Domain monotonicity implies
\[
  G_{G_n}(v,v) \le G_{B_D(v)}(v,v) = \sum_{t\ge 0} p_t^{B_D(v)}(v,v).
\]
Split the last sum at $T=\lfloor D^2\rfloor$ into an initial part and a tail:
\[
  \sum_{t\ge 0} p_t^{B_D(v)}(v,v) = \sum_{t=0}^{T} p_t^{B_D(v)}(v,v) + \sum_{t>T} p_t^{B_D(v)}(v,v).
\]
The second sum decays exponentially in $t$ (a standard spectral estimate), and hence it is $O(1)$.  For the first sum we compare to the walk on the infinite graph.  Since $p_t^{B_D(v)}(v,v) \le p_t(v,v)$ for all $t$, we obtain
\[
  \sum_{t=0}^{T} p_t^{B_D(v)}(v,v) \le \sum_{t=0}^{T} p_t(v,v).
\]
By \eqref{eq:return-sum}, $\sum_{t=1}^{T} p_t(v,v) = 2\cG \log T + O(1)$.  Taking $T=D^2$ gives $2\cG \log(D^2)+O(1)=4\cG\log D+O(1)$.  Combining these estimates yields the claimed bound.
\end{proof}

\begin{corollary}\label{cor:upper}
There exists $C_4>0$ such that
\[
  Z_n(1) \le 2\cG N_n \log N_n + C_4 N_n\,.
\]
\end{corollary}

\begin{proof}
By quadratic growth one has $D = \diam(G_n) \le c_2 N_n^{1/2}$ for some constant $c_2>0$.  Substituting this into the bound of \Cref{lem:upper} gives
\[
  G_{G_n}(v,v) \le 4\cG\Bigl(\tfrac{1}{2} \log N_n + O(1)\Bigr) + O(1)
  = 2\cG \log N_n + O(1).
\]
Summing over $v \in G_n$ gives
\[
  Z_n(1) \le \sum_{v \in G_n} (2\cG \log N_n + C_3) = 2\cG N_n \log N_n + C_3 N_n\,.\qedhere
\]
\end{proof}
 
\begin{corollary}\label{cor:lower}
There exists $C_2>0$ such that
\[
  Z_n(1) \ge 2\cG(1-\eta) N_n\log N_n - C_2 N_n\,.
\]
\end{corollary}

\begin{proof}
Summing the bound of \Cref{lem:lower} over $v\in I_n$ and recalling that $G_{G_n}(v,v)\ge0$ for $v\in E_n$ gives
\[
  Z_n(1) \ge |I_n|\Bigl[4\cG(1-\eta)\log R_n - C_1\Bigr].
\]
Since $R_n\asymp N_n^{1/2}$, we have $\log R_n = \tfrac12 \log N_n + O(1)$.  Moreover $|I_n| = N_n - O(N_n^{1-\eta/2})$ by the interior–boundary decomposition.  Substituting these and simplifying yields
\[
  Z_n(1) \ge \bigl[N_n - O(N_n^{1-\eta/2})\bigr]\Bigl[2\cG(1-\eta)\log N_n - O(1)\Bigr]
  = 2\cG(1-\eta) N_n \log N_n - O(N_n),
\]
which proves the claimed lower bound.
\end{proof}
 
 

\section{Proof of \cref{thm:main}}
By \cref{cor:lower} and \cref{cor:upper}, for any $\eta > 0$ and sufficiently large $n$:
\[
  (1-\eta)2\cG N_n\log N_n - C_2 N_n 
  \leq Z_n(1) \leq 
  2\cG N_n\log N_n + C_4 N_n.
\]
Dividing by $2\cG N_n \log N_n$:
\[
  (1-\eta) - \frac{C_2}{2\cG \log N_n} 
  \leq \frac{Z_n(1)}{2\cG N_n \log N_n} 
  \leq 1 + \frac{C_4}{2\cG \log N_n}.
\]
As $n \to \infty$, $\log N_n \to \infty$, so:
\[
  \liminf_{n\to\infty} \frac{Z_n(1)}{2\cG N_n \log N_n} \geq 1-\eta
  \quad\text{and}\quad
  \limsup_{n\to\infty} \frac{Z_n(1)}{2\cG N_n \log N_n} \leq 1.
\]
Since $\eta > 0$ is arbitrary, we conclude:
\[
  \lim_{n\to\infty} \frac{Z_n(1)}{2\cG N_n \log N_n} = 1,
\]
which is equivalent to $Z_n(1) \sim 2\cG N_n \log N_n$ and proves the theorem.


\appendix

\section{Other growth regimes}\label{app:growth}
The quadratic volume growth assumption is crucial in \cref{thm:main}. We briefly contrast it with the one-dimensional and cubic-growth cases:
\begin{itemize}
\item \textbf{Linear growth (dimension $1$):} For a path graph of length $N_n$ (volume $\asymp N_n$), the return probability decays as $p_t(x,x) \sim C\,t^{-1/2}$, so $\sum_{t=1}^{R}p_t(x,x) \asymp 2\sqrt{R}$. Taking $R \sim N_n$, one finds $G_{G_n}(x,x) \asymp 2\sqrt{N_n}$ for interior vertices. Summing over $N_n$ vertices gives $Z_n(1) \asymp N_n^{3/2}$. In fact, the exact calculation for the path graph yields $\lim_{n\to\infty} Z_n(1)/N_n^2 = 1/6$.
\item \textbf{Cubic growth (dimension $3$):} In $\mathbb{Z}^3$, the random walk is transient, meaning $\sum_{t=1}^{\infty}p_t(x,x) < \infty$. Therefore, for large finite subgraphs $G_n$ (volume $\asymp N_n$), the Green's function $G_{G_n}(x,x)$ remains bounded (approaching the infinite-lattice Green's value). Thus $Z_n(1) \asymp C\,N_n$ for some constant $C$ determined by the Green's function on $\mathbb{Z}^3$.  
\end{itemize}
These cases illustrate how the $N_n \log N_n$ divergence in $Z_n(1)$ is unique to two-dimensional (quadratic) growth. Higher dimensions give only linear divergence, while one dimension yields a more severe $N_n^2$ divergence.

\section{Lazy walk return probability}\label{app:return}
The simple random walk on $\mathbb{Z}^2$ satisfies \cite{LyonsPeres16}:
\begin{equation}\label{eq:simple-asymp}
  p_t^{\textnormal{simple}}(x,x) = \frac{2\cG}{t} + O(t^{-1-\delta})\,,\quad \text{for some } \delta>0.
\end{equation}


The maximal inequality \cite[Corollary 3.3]{Delmotte99} implies that for the lazy random walk:
\[
\mathbb{P}_v\!\left( \max_{0 \le s \le t} d_G(v, X_s) \ge d \right) \le C \exp\left( -c \frac{d^2}{t} \right),
\]
which suffices for the exit probability bound in Lemma 2. The return probability asymptotics \eqref{eq:return-sum} follow identically for lazy walks.
 
The constant $\cG$ generalizes the \emph{capacity per unit area} in potential theory: 
for $\mathbb{Z}^2$, $\cG = 1/\pi$ matches the reciprocal continuum capacity of the unit disk in $\mathbb{R}^2$.

The lazy random walk stays put with probability $1/2$ at each step, otherwise moving to a neighbor. Let $N_t$ be the number of non-lazy steps in $t$ steps, so $N_t \sim \operatorname{Bin}(t,1/2)$. Then by conditioning on $N_t$:
\[
  p_t^{\textnormal{lazy}}(x,x) = \mathbb{P}(N_t=0) + \sum_{k=1}^t \mathbb{P}(N_t=k) p_k^{\textnormal{simple}}(x,x).
\]
For $t \ge 1$, $\mathbb{P}(N_t = 0) = (1/2)^t = O(t^{-1-\delta})$. Using \eqref{eq:simple-asymp} and the binomial distribution:
\begin{align*}
  p_t^{\textnormal{lazy}}(x,x) 
  &= (1/2)^t + \sum_{k=1}^t \binom{t}{k} \left(\frac{1}{2}\right)^t \left( \frac{2\cG_{\text{simple}}}{k} + O(k^{-1-\delta}) \right) \\
  &= \frac{2\cG_{\text{simple}}}{2^t} \sum_{k=1}^t \binom{t}{k} \frac{1}{k} + O(t^{-1-\delta}) \\
  &= \frac{2\cG_{\text{simple}}}{t} \mathbb{E}\!\left[ \frac{t}{N_t} \mathbf{1}_{\{N_t \geq 1\}} \right] + O(t^{-1-\delta})
\end{align*}
where $\cG_{\text{simple}} = \cG$ since both the simple and lazy random walks on $\mathbb{Z}^2$ share the same constant in their local limit theorem (i.e., $p_t^{\textnormal{lazy}}(x,x) \sim 2\cG/t$).

By the Law of Large Numbers, $N_t/t \to 1/2$ as $t\to\infty$, and fluctuations are $O(t^{-1/2})$ by the Central Limit Theorem. Thus the expectation above is $1+O(t^{-1/2})$, whence
\[
  p_t^{\textnormal{lazy}}(x,x) = \frac{2\cG}{t} + O(t^{-3/2}).
\]
Summing this asymptotic from $t=1$ to $R$ yields
\[
  \sum_{t=1}^R p_t^{\textnormal{lazy}}(x,x) = 2\cG \log R + O(1),
\]
which proves \eqref{eq:return-sum}.

\bibliographystyle{abbrv}
\begin{thebibliography}{99}

\bibitem{BassKumagai08}
R.~F. Bass and T.~Kumagai.
\newblock {Intrinsic metrics for non-local symmetric Dirichlet forms}.
\newblock {\em J. Math. Soc. Japan}, 60(3):789--818, 2008.

\bibitem{Colin85}
Y.~Colin de Verdi\`ere.
\newblock {Sur le spectre des op\'erateurs elliptiques \`a bicaract\'eristiques toutes p\'eriodiques}.
\newblock {\em Comment. Math. Helv.}, 60(2):275--288, 1985.

\bibitem{Delmotte99}
T.~Delmotte.
\newblock {Parabolic Harnack inequality and estimates of Markov chains on
  graphs}.
\newblock {\em Rev. Mat. Iberoam.}, 15(1):181--232, 1999.

\bibitem{Frank10}
R.~L. Frank and A.~M. Hansson.
\newblock {The zeta function for the Laplacian on tori}.
\newblock {\em J. Spectr. Theory}, 1(1):1--20, 2010.


\bibitem{LyonsPeres16}
R.~Lyons and Y.~Peres.
\newblock {\em Probability on Trees and Networks}.
\newblock Cambridge University Press, 2016.
\newblock Available at \url{https://rdlyons.pages.iu.edu/}.

 
 

\end{thebibliography}

\end{document}