\documentclass{article}
\usepackage[utf8]{inputenc}
\title{Proof-Carrying Answers and Property-Based Self-Fuzzing}
\author{Research Team}
\date{\today}

\begin{document}
\maketitle

\begin{abstract}
This paper introduces the concept of Proof-Carrying Answers (PCA) combined with
property-based self-fuzzing as a means to improve reliability and verifiability
of large language model outputs. PCA requires that each generated answer
includes a machine-checkable proof or verification script. Property-based
self-fuzzing uses counterexample generators to test claims by systematically
searching for violations.
\end{abstract}

\section{Introduction}
Modern LLMs suffer from hallucination and unverifiable statements. PCA
addresses this by packaging answers with proof artifacts, while self-fuzzing
generates and tests counterexamples.

\section{Method}
We describe the technique: for each claim, produce code that checks the claim
and attempts to find counterexamples. The code is executed in a sandbox
environment to ensure safety.

\section{Implementation}
The implementation is provided in \texttt{code.py}, which imports the
\texttt{sciresearch\_ai} module to leverage existing utilities and demonstrates
a simple example of generating a proof-carrying answer.

\end{document}
